%\iffalse
% schule.dtx generated using makedtx version 0.94b (c) Nicola Talbot
% Command line args:
%   -src "(.*)\.(.*)=>\1.\2"
%   -author "Johannes Pieper (johannes_pieper@yahoo.de),Johannes Kuhaupt (kujohann@seminar.ham.nw.schule.de),Daniel Spittank (kontakt@daniel.spittank.net),André Hilbig (mail@andrehilbig.de),Adrian Salamon (adriansalamon@gmail.com)"
%   -preamble "__PREAMBEL__"
%   -askforoverwrite "1"
%   -dir "source"
%   -doc "output.doku.tex"
%   -codetitle "Implementation"
%   schule
% Created on 2015/5/7 18:38
%\fi
%\iffalse
%<*package>
%% \CharacterTable
%%  {Upper-case    \A\B\C\D\E\F\G\H\I\J\K\L\M\N\O\P\Q\R\S\T\U\V\W\X\Y\Z
%%   Lower-case    \a\b\c\d\e\f\g\h\i\j\k\l\m\n\o\p\q\r\s\t\u\v\w\x\y\z
%%   Digits        \0\1\2\3\4\5\6\7\8\9
%%   Exclamation   \!     Double quote  \"     Hash (number) \#
%%   Dollar        \$     Percent       \%     Ampersand     \&
%%   Acute accent  \'     Left paren    \(     Right paren   \)
%%   Asterisk      \*     Plus          \+     Comma         \,
%%   Minus         \-     Point         \.     Solidus       \/
%%   Colon         \:     Semicolon     \;     Less than     \<
%%   Equals        \=     Greater than  \>     Question mark \?
%%   Commercial at \@     Left bracket  \[     Backslash     \\
%%   Right bracket \]     Circumflex    \^     Underscore    \_
%%   Grave accent  \`     Left brace    \{     Vertical bar  \|
%%   Right brace   \}     Tilde         \~}
%</package>
%\fi
% \iffalse
% Doc-Source file to use with LaTeX2e
% Copyright (C) 2015 Johannes Pieper (johannes_pieper@yahoo.de),Johannes Kuhaupt (kujohann@seminar.ham.nw.schule.de),Daniel Spittank (kontakt@daniel.spittank.net),André Hilbig (mail@andrehilbig.de),Adrian Salamon (adriansalamon@gmail.com), all rights reserved.
% \fi
% \iffalse
%<*driver>
\documentclass[a4paper]{ltxdoc}
\usepackage[utf8]{inputenc}
\usepackage[T1]{fontenc}
\usepackage[ngerman]{babel}
\usepackage[usenames,dvipsnames,svgnames,table]{xcolor}
\usepackage{schule,schulinf,syntaxdi,schulphy}
\definecolor{blau}{rgb}{0,0,0.75}         
\definecolor{orange}{rgb}{0.8,0.3,0}  
\usepackage{hyperref}
\hypersetup{
 pdftitle = {\LaTeX-Klassen und Pakete für den Einsatz im Bereich der
						 Schule},
 pdfsubject = {},
 pdfauthor = {Johannes Pieper, Johannes Kuhaupt, Ludger Humbert,
 							Andr\'e Hilbig, Adrian Salamon, Daniel Spittank},
 colorlinks = true,
 hypertexnames = true,
 linkcolor=blau, %
 filecolor=orange, %
 citecolor=blau,
 menucolor=orange, %
 urlcolor=orange,
 breaklinks=true
}
\usepackage{caption,xparse,xargs}
\usepackage{placeins,float,prettyref}
\usepackage{newfloat,changepage}
\usepackage{pdfpages}
\newrefformat{sec}{Abschnitt\,\ref{#1}, S.\,\pageref{#1}}
\newrefformat{paket}{Paket~\ref{#1}, S.\,\pageref{#1}}
\newrefformat{klasse}{Klasse~\ref{#1}, S.\,\pageref{#1}}
\newrefformat{fig}{Abb.\,\ref{#1}}
\newrefformat{tab}{Tab.\,\ref{#1}}
\newrefformat{ex}{Bsp.\,\ref{#1}, S.\,\pageref{#1}}
\DeclareFloatingEnvironment[name={Bsp.},
	listname={Beispielverzeichnis}, within=section]{example}
\floatstyle{ruled}
\restylefloat{example}
\NewDocumentEnvironment{beispiel}{o o m}{
 \begin{example}[ht!]
 \centering
 \vspace{0.2cm}
 \IfNoValueTF{#2}{
 	\caption{#3}
 }{
 	\caption[#2]{#3}
 }
 }{
	 \IfNoValueTF{#1}{
	 	\label{ex:\theexample}
	 }{
	 	\label{ex:#1}
	 }
 \vspace{0.2cm}
 \end{example}
 }
\EnableCrossrefs
\CodelineIndex
\RecordChanges
\makeatletter
\makeatother
\lstset{  %
  language=[LaTeX]TeX,                 
  basicstyle=\small,            
  numbers=left,                    
  numberstyle=\footnotesize,           
  stepnumber=2,                    
  numbersep=5pt,                   
  backgroundcolor=\color{white},       
  showspaces=false,                
  showstringspaces=false,          
  showtabs=false,                  
  frame=false,                    
  tabsize=2,                       
  resetmargins=true,
  captionpos=b,                    
  title={},                    
  breaklines=true,
  breakautoindent=true,
  prebreak=\mbox{ $\curvearrowright$},
  postbreak=\mbox{$\rightsquigarrow$ },
  linewidth=\columnwidth,
  breakatwhitespace=true,         
  numberstyle=\tiny\color{gray},         
  keywordstyle=\color{blue},           
  commentstyle=\color{OliveGreen},        
  stringstyle=\color{mauve},          
  morekeywords={
		zeitpunkt, punkteitem, punkteitemloesung, scaleSequenzdiagramm,
		newthread, newthreadtwo, newinst, node, chainin, draw, to,
		dokName, jahrgang, minisec, subsection, glqq, grqq, euro,
		AufgabeLoesung, AufgabenLoesung, AufgabeHinweis, AufgabenHinweis,
		attribute, anchormark
  }
}

\xspaceaddexceptions{\guillemotright,\guillemotleft}

\newcommand{\materialsammlung}{
	\url{http://ddi.uni-wuppertal.de/material/materialsammlung/index.html}
}

\begin{document}
\DocInput{schule.dtx}
\end{document}
%</driver>
%\fi
%		\CheckSum{0}
%		\title{\LaTeX-Klassen und Pakete für den Einsatz im Bereich der
%						Schule}
%    \author{Johannes Pieper, Johannes Kuhaupt, Ludger Humbert,\\
%						Daniel Spittank, Andr\'e Hilbig, Adrian Salamon}
%    \date{2015-05-07}
%    \maketitle
%    \begin{abstract}
%			Diese Zusammenstellung wird entwickelt, um Pakete und Befehle
%			bereit zu stellen, die für den Textsatz von Dokumenten zur
%			Unterichtsvorbereitung für den (Informatik)Unterricht nützlich
%			sind. Zur Zeit liegt der Schwerpunkt auf dem
%			Informatikunterricht, eine Ergänzung für den Physikunterricht
%			wird nach und nach eingearbeitet. Weitere Ergänzungen für andere
%			Fächer werden gerne entgegen genommen.
% 
%			Diese Sammlung umfasst Pakete und Klassen zum Setzen von
%			speziellen Dokumentformen, wie Klausuren, Lernzielkontrollen,
%			Unterrichtsbesuchen, Arbeits-, Informations- und
%			Lösungsblättern, sowie speziellen Elementen, wie Struktogramme, 
%			Syntax-, Sequenz-, Objekt- und Klassendiagrammen.
%
%			Ein besonderer Dank geht an Martin Weise für seine Hilfe bei der
%			Übersetzung der \enquote{Readme-Dateien} und Zusammenfassung auf
%			\emph{CTAN} ins Englische.
%		\end{abstract}
% \tableofcontents
% \listofexample
%	\clearpage
%    \section{Wichtiger Hinweis zur weiteren Entwicklung}
%		Das Schule-Paket wird derzeit überarbeitet. Die nächste Version wird grundlegende,
%		strukturelle Veränderungen beinhalten. So wird unter anderem die Vielzahl an
%		Dokumentenklassen stark reduziert und eine Umstellung der Konfiguration auf
%		Paketoptionen vorgenommen. Dies führt zu großen Veränderungen der Schnittstelle. 
%
%		Diese Änderungen werden einige bestehende Probleme (u.a. Quelltexte in Aufgaben und
%		Lösungen) beheben und die Nutzung des Pakets vereinheitlichen. Außerdem wird die
%		Nutzung in anderen Dokumentenklassen ermöglicht, sodass etwa die Aufgabenumgebungen
%		auch in Beamer-Präsentationen übernommen werden können. Klausuren werden zudem die
%		automatische Erzeugung von Erwartungshorizonten unterstützen, die einzelnen
%		Erwartungen werden dabei innerhalb der Aufgabenumgebung mit angegeben.
%		Durch die Nutzung des exsheets-Pakets bieten sich darüber hinaus neue Möglichkeiten,
%		etwa die Generierung von -- nach eigenen Kriterien -- differenzierten Arbeitsblättern
%		und Klausuren aus Aufgabensammlungen.
%
%		Eine weitere große Veränderung ist die Ausgliederung der ausbildungsrelevanten
%		Teile (Unterrichtsbesuche, Stundenverläufe etc.) des Pakets. In der
%		Vergangenheit hat sich gezeigt, dass die Anforderungen der verschiedenen, an der
%		Lehrerausbildung beteiligten, Stellen sich stark voneinander unterscheiden.
%		Daher werden die entsprechenden Funktionen des Pakets ausgegliedert, sodass sie
%		einfach in eigenen Dokumenten genutzt werden können. 
%
%    \section{Installation}
%			Um die Pakete und Klassen nutzen zu können, gibt es drei Varianten.
%			In der folgenden Beschreibung dieser Möglichkeiten wird von einer 
%			standardisierten \LaTeX-Installation ausgegangen -- weitere Hinweise 
%			können der Dokumentation der jeweiligen \TeX-Distribution entnommen 
%			werden:
% \begin{description}
%	\item[Global] Für die globale/systemweite Installation der Pakete und Klassen
%					müssen diese in das globale \LaTeX-Verzeichnis der
%					\TeX-Installation kopiert werden: in der Regel
%					\texttt{/usr/share/texmf/tex/latex/}. In diesem kann ein
%					weiteres Verzeichnis wie z.\,B. \texttt{schule} angelegt
%					werden, in das alle \texttt{.sty} und \texttt{.cls} Dateien
%					kopiert werden. 
% 	
%					Damit die Quellen anschließend dem System bekannt sind, muss
%					der Cache von \LaTeX{} neu aufgebaut werden. Bei den meisten
%					Linux-Installationen geschieht dieses durch den Aufruf von
%					\texttt{texhash}.
% 
%	\item[Benutzer] Damit ein Nutzer auf die Quellen zugreifen kann,
%					müssen diese im Benutzerverzeichnis (Home directory)
%					abgelegt werden. Dies geschieht durch das Kopieren der
%					Pakete und Klassen in das Verzeichnis
%					\texttt{texmf/tex/latex/} im Benutzerverzeichnis, das ggf.
%					erst angelegt werden muss. Auch hier kann -- wie bei der
%					globalen Installation -- ein eigenes Unterverzeichnis
%					angelegt werden.
% 
%	\item[Lokal] Um die Klassen und Pakete ohne weitere Installation
%					nutzen zu können, ist es darüber hinaus möglich, die
%					benötigten Dateien in das Verzeichnis zu kopieren, in dem
%					die Datei liegt, die übersetzt werden soll.
% \end{description}
%
% \subsection*{Voraussetzungen}
%	 Ein Grund für die Nutzung der speziellen Klassen und Pakete liegt
%	 darin, viele der häufig  benötigten Pakete zusammen zu fassen.
%	 Daher müssen diese für die Benutzung vorhanden sein. Die meisten
%	 sind Standardpakete, die mit jeder normalen Installation
%	 mitgeliefert sind.
%	 Es folgt eine Aufstellung der Voraussetzungen
%	 für das Paket \texttt{schule}, das in jedem der anderen Pakete und
%	 jeder Klasse verwendet wird:
%\begin{multicols}{3}
% \begin{smallitemize}
% 	\item ngerman
% 	\item ifthen
% 	\item xifthen
% 	\item xspace
% 	\item tabularx
% 	\item ragged2e
% 	\item amssymb
% 	\item amsmath
% 	\item graphicx
% 	\item TikZ
% 	\item paralist
% 	\item textcomp
% 	\item xmpincl
% 	\item wrapfig
% 	\item eurosym
% 	\item multirow
%	\item ccicons
%	\item svn-multi
%	\item csquotes
% \end{smallitemize}
%\end{multicols}
% 
% Folgende Pakete werden zusätzlich für \texttt{schulinf} benötigt: 
% \begin{multicols}{2}
%  \begin{smallitemize}
%  	\item pgf-umlcd
%		\item pgf-umlsd
%  	\item syntaxdi (im Bundle enthalten)
%  	\item relaycircuit (im Bundle enthalten)
%  	\item listings
%  	\item struktex
%  \end{smallitemize}
% \end{multicols}
% 
% Folgende TikZ-Bibliotheken werden für \texttt{syntaxdi} benötigt:
% \begin{smallitemize}
% 	\item arrows, chains, scopes, shadows und shapes.misc
% \end{smallitemize}
% 
% Folgende Pakete werden zusätzlich für \texttt{schulphy} benötigt:
% \begin{multicols}{2}
%  \begin{smallitemize}
%  	\item units
%  	\item mhchem
%  \end{smallitemize}
% \end{multicols}
%  
%	\clearpage
% \section{Nutzung der einzelnen Pakete}
%   In diesem Abschnitt werden alle Pakete und ihre Benutzung
%   beschrieben.
%   \subsection{Das Paket \texttt{schule}} \label{paket:schule}
%		 Beim Paket \texttt{schule} handelt es sich um eine Sammlung
%		 häufig benötigter Befehle und Umgebungen.
%
% \subsubsection{Anführungszeichen}
%	\DescribeMacro{\enquote} \DescribeMacro{\diastring}
%	Durch den Befehl \cs{enquote}\marg{Text} können Passagen in
%	Anführungszeichen gesetzt werden. Standardmäßig werden hier die
%	deutschen \enquote{Möwchen} geladen. Über die Option \verb|quotes|
%	können doppelte \glqq Hochkommata\grqq\ geladen werden:
%	\begin{center}
%   	\verb|\usepackage[quotes]{schule}|
%	\end{center}
%	Um Zeichenketten (strings) in Diagrammen, usw. kenntlich zu machen, steht 
%	der	Befehl \cs{diastring}\marg{Zeichenkette} zur Verfügung:
%	\diastring{Zeichenkette}.
%
%	\textbf{Hinweis:} Teilweise kann es zu Fehlern kommen, wenn das
%	Paket \verb|csquotes| mit eigenen Optionen geladen wird.
% 
% \subsubsection{Einfache Befehle}
% \DescribeMacro{\SuS} \DescribeMacro{\SuSn}
% Durch die Befehle \cmd{\SuS} und \cmd{\SuSn} wird eine einfache
% Kurzschreibweise für die amtlich geforderte Schreibweise von
% \enquote{\SuS} bzw. \enquote{\SuSn} bereit gestellt.
%
% \DescribeMacro{\loesung}
% Mit dem Befehl \cs{loesung}\marg{Text} ist es möglich, einen
% Textbereich in Abhängigkeit von einem Parameter ein- oder
% auszublenden. Beim Laden des Paketes kann durch Angabe des
% Parameterwerts \verb|loesung| der Textbereich mit der Lösung genau an
% der Stelle angezeigt werden, an der er im Text steht. Mit dem
% Parameterwert \verb|loesungsseite| wird dem Dokument eine eigene
% Seite hinzugefügt, die alle Lösungen aufzählt. Der Parameter kann
% auch direkt der Dokumentenklasse übergeben werden.
%
% \begin{beispiel}{Lösungen können im Fließtext (\texttt{loesung}) oder
%				 auf eine seperate Seite (\texttt{loesungsseite}) gesetzt
%         werden}
% \begin{lstlisting}[caption={},gobble=4]
%   \documentclass[loesung]{schuleab}
%	\end{lstlisting}
% oder
% \begin{lstlisting}[caption={},gobble=4]
%   \documentclass[loesungsseite]{schuleab}
%	\end{lstlisting}
% \end{beispiel} 
%
% \DescribeMacro{\luecke}
% Der Befehl \cs{luecke}\oarg{Lösung}\marg{Länge} bietet die Möglichkeit, eine
% unterstrichene Lücke im Text, wie sie in einem Lückentext benötigt
% wird, zu erzeugen. So erscheint mit \verb|\luecke{3cm}| dieses
% \luecke{3cm} im Text. Optional kann ein Lösungstext angegeben werden,
% der, sofern die Option \verb|loesung| gesetzt wurde, in die Lücke
% geschrieben wird. Somit kann durch \verb|\luecke[Lösung]{3cm}| die
% \setboolean{@loesunganzeigen}{true} \luecke[Lösung]{3cm}
% \setboolean{@loesunganzeigen}{false} ebenfalls gesetzt werden.
%
% \DescribeMacro{\chb}
% Eine Box zum Ankreuzen \chb lässt sich mit Hilfe des Befehls
% \cs{chb}\oarg{r} realisieren. Durch das optionale Argument lässt
% sich die Box als \textit{richtig} markieren. \verb|\chb[r]|
% produziert damit die Box \chb[r] und wird, sofern die Option
% \verb|loesung| gesetzt wurde, mit einem Kreuz markiert:%
% \setboolean{@loesunganzeigen}{true}\chb[r].
% \setboolean{@loesunganzeigen}{false}
%
% \DescribeMacro{\name}
% Der Name einer Person wird mit dem Befehl \cs{name}\marg{Name}
% hervorgehoben. So wird der Name von \name{Einstein} durch
% \verb|\name{Einstein}| erzeugt.
%
% \DescribeMacro{\so}
% Der Befehl \cs{so}\marg{Wert} ermöglicht es in Wertetabellen bzw.
% Schreibtischtests einzelne Werte durchzustreichen.
%
% \subsubsection{Kopf- und Fußzeilen}
% \DescribeMacro{\keineSeitenzahlen}
% Sämtliche abgeleiteten Klassen (vgl. \prettyref{sec:klassen}) setzen
% Kopf- und Fußzeilen mit dem Seitenstil \texttt{scrheadings}. Damit
% stehen die Befehle \cs{ihead}\marg{Text}, \cs{chead}\marg{Text} und
% \cs{ohead}\marg{Text} bzw. \cs{ifoot}\marg{Text},
% \cs{cfoot}\marg{Text} und \cs{ofoot}\marg{Text} zur Formatierung der
% inneren, mittigen und äußeren Kopf- bzw. Fußzeile zur Verfügung. So
% kann die normale Belegung innerhalb einer Klasse überschrieben
% werden.
%
% Der Befehl \cs{keineSeitenzahlen} schaltet die Seitenzahlen in der
% Fußleiste ab. Der Befehl sollte nach dem Laden von Paketen aus der
% \texttt{schule}-Zusammenstellung erfolgen -- aber vor
% \verb|\begin{document}|. Auch etwaige manuelle Änderungen an Kopf-
% oder Fußzeile sollten an dieser Stelle im Dokument aufgeführt werden.
%
% 
% \subsubsection{Umgebungen}
% 
% \DescribeEnv{stundenverlauf}
% Die Umgebung \texttt{stundenverlauf} stellt eine modifizierte
% \texttt{tabularx}-Tabelle bereit, deren Breiten usw. bereits passend
% an die Vorgaben für den schriftlichen Verlaufsplan bei
% Unterrichtsbesuchen (Stand: 2012) in NRW für die Schriftgröße 12px
% und DIV 14 der Klasse \texttt{scrartcl} festgelegt wurden. Außerdem
% werden die entsprechenden Überschriften erzeugt. 
% 
% \DescribeMacro{\zeitpunkt} 
% Um mögliche Zeitangaben in den Verlauf der Stunde zu integrieren,
% lässt sich innerhalb der Umgebung \texttt{stundenverlauf} der Befehl
% \cs{zeitpunkt}\marg{Zeit} nutzen. Die Angabe der Zeit kann über das
% Setzen von \cmd{\zeitanzeigen} mit den Werten  0 oder 1 je nach
% Bedarf an- bzw. abgeschaltet werden
% (vgl.~\prettyref{ex:stundenverlauf}).
% 
% \DescribeMacro{stundenverlaufquer}
%	Über die Option \texttt{stundenverlaufquer} kann die Tabelle im
%	Querformat gesetzt werden. Dazu wird eine entsprechend den obigen
%	Bedingungen modifizierte \texttt{tabular}-Tabelle auf der
%	nächstmöglichen Seite einzeln gesetzt. Normalerweise wird die
%	umgebrochene Seite mit dem folgenden Text aufgefüllt. Sofern der
%	nachfolgende Text direkt erst nach der Tabelle erscheinen soll, muss
%	an das Ende der Umgebung \texttt{stundenverlauf} der Befehl
%	\verb|\FloatBarrier| ergänzt werden:
%
%   	\verb|\usepackage[stundenverlaufquer]{schule}|
%   	
%   	\ldots
%   	
%   	\verb|\begin{stundenverlauf}|
%   	
%   	\ldots
%   	
%   	\verb|\end{stundenverlauf}\FloatBarrier|
%	
% \DescribeMacro{stundenverlaufdidkom}
% \DescribeMacro{\didkom}
% In manchen Fällen kann es notwendig sein, die Tabelle des
% Stundenverlaufs um eine weitere Spalte \enquote{didaktischer
% Kommentar} zu erweitern. Über die Option
% \texttt{stundenverlaufdidkom} wird sowohl der Befehl zum Setzen eines
% Zeitpunktes als auch die Tabelle im Hoch- und Querformat entsprechend
% angepasst. Der Befehl \cs{didkom}\marg{Text} setzt in Abhängigkeit
% von der gewählten Option den Text in die passende Spalte.
%
% \DescribeMacro{stundenverlaufASF}
% \DescribeMacro{\setASFfuss}
% Um in der Spalte \enquote{Aktions- und Sozialformen} Platz zu sparen,
% kann es notwendig sein, dass die Begriffe dort abgekürzt werden.
% Durch das Setzen der Option \verb|stundenverlaufASF| wird die Tabelle
% automatisch in der Breite angepasst und entsprechende Abkürzungen
% werden gesetzt und mit einem Hinweistext beschrieben. Folgende
% Kürzungsbefehle stehen vordefiniert zur Verfügung:
% \begin{multicols}{2}
%  \begin{description}
% 		\item[\cs{EA}] Einzelarbeit,
% 		\item[\cs{PA}] Partnerarbeit,
%		\item[\cs{GA}] Gruppenarbeit,
%		\item[\cs{LV}] Lehrervortrag,
%		\item[\cs{SV}] Schülervortrag,
%		\item[\cs{UG}] Unterrichts\-gespräch,
%		\item[\cs{AB}] Arbeitsblatt,
%		\item[\cs{TPS}] Think-Pair-Share,
%		\item[\cs{RSP}] Rollenspiel.
%  \end{description}
% \end{multicols}
% Sofern die aufgeführten Befehle benutzt werden, wird in Abhängigkeit
% der Option \verb|stundenverlaufASF| die jeweilige Sozialform
% ausgeschrieben oder abgekürzt gesetzt. Falls zusätzlich eigene
% Abkürzungen benutzt werden, so muss der Hinweistext mit dem Befehl
% \cs{setASFfuss}\marg{Text} selbst definiert werden.
%
% \DescribeMacro{setP}
% \DescribeMacro{setO}
% \DescribeMacro{setA}
% \DescribeMacro{setM}
% \DescribeMacro{setD}
% Über die Befehle \cs{setX}\marg{Spaltenparameter} kann ein beliebiger
% Prameter zur Formatierung einer Spalte selber festgelegt werden. Es
% gilt die folgende Zuordnung zu den Spalten:
% \begin{description}
%	\item[P] Unterrichtsphase,
%	\item[O] Operationen/Sachaspekte,
%	\item[A] Aktions- und Sozialform,
%	\item[M] Medien,
%	\item[D] Didaktischer Kommentar.
% \end{description}
% So kann beispielsweise die Spalte \enquote{Didaktischer Kommentar}
% mit \verb|\setD{X}| auf eine variable Breite gesetzt werden. Der
% Befehl \verb|\setP{p{3cm}}| formatiert den Text innerhalb der Spalte
% \enquote{Unterrichtsphase} in Blocksatz bei einer festen Breite von
% $\unit[3]{cm}$.
%
% \textbf{Hinweis:} Für jede Spalte kann nur einmal eine Formatierung
% festgelegt werden. Der Befehl muss vor der Umgebung
% \verb|stundenverlauf| aufgerufen werden.
%
% \DescribeMacro{setPtext}
% \DescribeMacro{setOtext}
% \DescribeMacro{setAtext}
% \DescribeMacro{setMtext}
% \DescribeMacro{setDtext}
% Entsprechend der zuvor beschrieben Zuordnung zu den einzelnen
% Spalten, kann über den Befehl \cs{setXtext}\marg{Text} eine beliebige
% Spaltenüberschrift gewählt werden. Ausnahme ist der Befehl
% \cs{setAtext}\oarg{Abkürzung}\marg{Text} mit dem es möglich ist, über
% den optionalen Parameter den Text der Abkürzung für die Spalte
% \enquote{Aktions- und Sozialform} festzulegen.
%
% \begin{beispiel}[stundenverlauf]{Beispielhafter Quelltext für einen
%				 Stundenverlaufplan}
% \begin{lstlisting}[caption={},gobble=2]
% \begin{stundenverlauf}
%   \zeitpunkt{10:30 Uhr} 
%   Einstieg & Vortrag & LV & Tafel 
%     \didkom{Zeit beachten} \\ \hline
%   \zeitpunkt{10:38 Uhr} 
%   \ldots{} & weiter im Verlauf der Stunde & EA & 
%     \didkom{AB austeilen}\\ \hline
% \end{stundenverlauf}
% \end{lstlisting}
% 
% Mit \verb|\zeitanzeigen=0| sieht der Stundenverlauf so aus:
%	\setOtext{Ope\-ra\-tion\-en/Sach\-aspekte}
% \begin{stundenverlauf}
% 	\zeitpunkt{10:30 Uhr} 
%		Einstieg & Vortrag & LV & Tafel \didkom{Zeit beachten} \\ \hline
% 	\zeitpunkt{10:38 Uhr} 
%	  \ldots{} & weiter im Verlauf der Stunde & EA & \didkom{AB
%	    austeilen}\\ \hline
% \end{stundenverlauf}
% 
% \vspace{0.5cm}
%
% Dagegen erzeugt \verb|\zeitanzeigen=1| folgendes: \zeitanzeigen=1 \\
% \begin{stundenverlauf}
% 	\zeitpunkt{10:30 Uhr} 
%		Einstieg & Vortrag & LV & Tafel \didkom{Zeit beachten} \\ \hline
% 	\zeitpunkt{10:38 Uhr} 
%		\ldots{} & weiter im Verlauf der Stunde & EA & \didkom{AB
%		austeilen} \\ \hline
% \end{stundenverlauf}
% \zeitanzeigen=0
% 
% \vspace{0.5cm}
%
% Dagegen wird mit \verb|stundenverlaufdidkom| folgendes erzeugt:
% \setboolean{@stundenverlaufdidkom}{true}
% \setD{p{3cm}}
%				\renewcommand{\stundenverlaufkopf}{%
%					\hline
%					\multicolumn{1}{|>{\raggedright\hspace{0pt}}P|}{%
%						\textbf{\Ptext}
%					} &
%					\multicolumn{1}{>{\raggedright\hspace{0pt}}O|}{
%						\textbf{\Otext}
%					} &
%					\multicolumn{1}{>{\raggedright\hspace{0pt}}A|}{%
%						\textbf{\Atext}
%					} &
%					\multicolumn{1}{>{\raggedright\hspace{0pt}}M|}{
%						\textbf{\Mtext}
%					}
%					\didkom{
%						\multicolumn{1}{>{\raggedright\hspace{0pt}}D|}{
%							\textbf{\Dtext}
%						}
%					}
%				}
% \begin{stundenverlauf}
% 	\zeitpunkt{10:30 Uhr} 
%		Einstieg & Vortrag & LV & Tafel \didkom{Zeit beachten} \\ \hline
% 	\zeitpunkt{10:38 Uhr} 
%		\ldots{} & weiter im Verlauf der Stunde & EA & \didkom{AB
%		austeilen} \\ \hline
% \end{stundenverlauf}
% \setboolean{@stundenverlaufdidkom}{false}
%  \end{beispiel}
%
% 
% \FloatBarrier
% 
% \DescribeEnv{aufgaben} 
% \DescribeMacro{\punkteitem}
% \DescribeMacro{\punkteitemloesung}
% Mit der \texttt{aufgaben}-Umgebung steht eine Umgebung zur Verfügung,
% in der alle Elemente mit \textbf{$<$Nr$>$.~Aufgabe} beginnen, wie im
% Beispiel zu sehen. Dabei kann der Befehl
% \cs{punkteitem{}}\marg{Punkte}, wie in jeder anderen Listenumgebung,
% genutzt werden. Über ihn ist es möglich anzugeben, wie viele Punkte
% es für die einzelnen Teile gibt. Hierbei wird automatisch bei nur
% einem Punkt die Schreibweise angepasst. Durch
% \cs{punkteitemloesung}\oarg{Lösung}\marg{Punkte}\marg{Aufgabentext}
% kann die Lösung der Aufgabe mit angegeben werden. Sofern die Lösungen
% auf einer extra Seite ausgegeben werden, wird dort die
% Aufgabennummerierung berücksichtigt. Wird das optionale Argument der
% Lösung nicht benötigt, verhalten sich beide Befehle identisch
% (vgl.~\prettyref{ex:aufgabenpunkte}).
% \begin{beispiel}[aufgabenpunkte]{Aufgabenumgebung -- u.\,a.
%				 automatische Zuordnung der Punkte}
% \begin{lstlisting}[caption={},gobble=2]
% \begin{aufgaben}
%    \item Erstellen Sie aus dem obigen Text mit Hilfe der
%       Methode nach Abbott ein Objektdiagramm. Berücksichtigen 
%       Sie dabei auch die Bezugsobjekte. Verwenden Sie  
%       nur Bezeichner gemäß der Vorgaben aus dem Unterricht
%    \punkteitemloesung[Ein Informatiksystem ist eine Einheit 
%       von Hard-, Software und Netzen einschließlich aller 
%       durch sie intendierten oder verursachten Gestaltungs- 
%       und Qualifizierungsprozesse bezüglich der Arbeit und 
%       Organisation.]{2}{Geben Sie eine allgemeingültige und 
%       fachlich korrekte Definition eines Informatiksystems	
%       an.}
%    \punkteitem{10} Nennen Sie die Fachgebiete der 
%       Fachwissenschaft Informatik und geben Sie pro Fachgebiet 
%       ein Anwendungsbeispiel an.
% \end{aufgaben}
% \end{lstlisting}
% \setboolean{@loesunganzeigen}{true}
% \begin{aufgaben}
%		\item Erstellen Sie aus dem obigen Text mit Hilfe der Methode nach
%						Abbott ein Objektdiagramm. Berücksichtigen Sie dabei auch
%						die Bezugsobjekte. Verwenden Sie  nur Bezeichner gemäß der
%						Vorgaben aus dem Unterricht.
%		\punkteitemloesung[Ein Informatiksystem ist eine Einheit von
%		        Hard-, Software und Netzen einschließlich aller durch sie
%        		intendierten oder verursachten Gestaltungs- und
%        		Qualifizierungsprozesse bezüglich der Arbeit und
%         		Organisation.]{2}{Geben Sie eine allgemeingültige und 
%						fachlich korrekte Definition eines Informatiksystems	an.}
%    \punkteitem{10} Nennen Sie die Fachgebiete der Fachwissenschaft 
%       Informatik und geben Sie pro Fachgebiet ein Anwendungsbeispiel 
%       an.
% \end{aufgaben}
% \end{beispiel}
% 
% \FloatBarrier
%
% \DescribeEnv{alphaEnum}
% In der \texttt{alphaEnum}-Umgebung, die nur für die erste Ebene
% möglich ist, werden alle Punkte, wie im \prettyref{ex:telefonnummer}
% zu sehen, mit fettgedruckten Buchstaben, auf die eine geschlossene
% Klammer folgt, durchnummeriert.
% \begin{beispiel}[telefonnummer]{Aufzählung mit Buchstaben}
% \begin{lstlisting}[caption={},gobble=2]
% \begin{alphaEnum}
%   \item Wählen Sie eine Datenstruktur, die geeignet ist, die
%         Telefonnummern zu speichern. Notieren Sie auf
%         einem Zettel die Gründe für die Wahl.
%   \item Ordnen Sie alle obigen Telefonnummern im Schema der
%         Datenstruktur auf einem Zettel an.
% \end{alphaEnum}
% \end{lstlisting}
% \begin{alphaEnum}
%   \item Wählen Sie eine Datenstruktur, die geeignet ist, die
%         Telefonnummern zu speichern. Notieren Sie auf
%         einem Zettel die Gründe für die Wahl.
%   \item Ordnen Sie alle obigen Telefonnummern im Schema der
%         Datenstruktur auf einem Zettel an.
% \end{alphaEnum}
% \end{beispiel}
%
%
% \DescribeEnv{smallitemize}
% \DescribeEnv{smallenumerate}
%
% \DescribeEnv{smalldescription}
% Die drei Listenumgebungen \texttt{smallitemize},
% \texttt{smallenumerate} sowie \texttt{small"-de"-scription} sind
% identisch zu den \LaTeX-Standardumgebungen, bis auf die Tatsache,
% dass zwischen den einzelnen Punkten der Abstand verkleinert wurde.
% Dieses kann man am besten an der Gegenüberstellung in
% \prettyref{ex:aufzaehlung} sehen.
% \begin{beispiel}[aufzaehlung]{Aufzählungsumgebungen mit und ohne
%				 Abstandshalter}
% \begin{minipage}[t]{.4\textwidth}
% 	\texttt{itemize}-Umgebung:
% 	\begin{itemize}
% 		\item Punkt
% 		\item Punkt
% 		\item Punkt
% 	\end{itemize}
% \end{minipage}
% \begin{minipage}[t]{.4\textwidth}
% 	\texttt{smallitemize}-Umgebung:
% 	\begin{smallitemize}
% 		\item Punkt
% 		\item Punkt
% 		\item Punkt
% 	\end{smallitemize}
% \end{minipage}
% \end{beispiel}
% \FloatBarrier
% \subsection{Das Paket \texttt{schulinf} -- Informatik}
% \label{paket:schulinf}
% Das Paket \texttt{schulinf} bindet neben dem Paket \texttt{schule}
% auch Pakete ein, damit Syntaxdiagramme (\texttt{syntaxdi}, siehe
% \prettyref{paket:syntaxdi}), Struktogramme (\texttt{struktex}) und
% Sequenzdiagramme (\texttt{pgf-umlsd}, \prettyref{paket:sequenz})
% genutzt werden können. Die entsprechenden Dokumentationen sind bei
% den jeweiligen Paketen zu finden. 
% 
% Bei der Verwendung der Klassen \texttt{schuleab}, \texttt{schulein},
% \texttt{schuleue} oder \texttt{schullsg} wird mit dem Einbinden
% dieses Pakets automatisch das Fach auf Informatik gesetzt. Außerdem
% wird ein Zusatz eingefügt, mit dem das zum Anzeigen von Quelltext
% nützliche Paket \texttt{listings} die Sonderzeichen mit dem UTF-8
% Zeichensatz richtig interpretiert.
% 
% \subsection{Zusätzliche Befehle für das Sequenzdiagramm}
% \label{paket:sequenz}
% \DescribeMacro{\scaleSequenzdiagramm}
% Da es vorkommen kann, dass Sequenzdiagramme zu breit für eine Seite
% sind, kann mit dem Befehl \cs{scaleSequenzdiagramm}\marg{Faktor} die
% Größe des Sequenzdiagramms angepasst werden, wenn er innerhalb der
% \texttt{sequencediagram}-Umgebung ausgeführt wird (vgl.
% \prettyref{ex:seqskal}).
%
%	\begin{beispiel}[seqskal][Sequenzdiagramm mit einer
%					Skalierung]{Sequenzdiagramm mit einer Skalierung
%					\\\footnotesize{(entnommen aus	\materialsammlung)}}
%		\iffalse
% % Der folgende kenntlich gemachte Abschnitt ist in der Zusammenarbeit
% % von Informatikreferendaren und ehemaligen Informatikreferendaren
% % der Studienseminare (heute ZfsL) Arnsberg, Hamm und Solingen
% % entstanden.
% %
% % Der Abschnitt steht unter der Lizenz: Creative Commons by-nc-sa
% % Version 4.0
% % http://creativecommons.org/licenses/by-nc-sa/4.0/deed.de
% %
% % Nach dieser Lizenz darf der Abschnitt beliebig kopiert und
% % bearbeitet werden, sofern das Folgeprodukt wiederum unter
% % gleichen Lizenzbedingungen vertrieben und auf die ursprünglichen
% % Urheber verwiesen wird.  Eine kommerzielle Nutzung ist
% % ausdrücklich ausgeschlossen.
% %
% % Die Namensnennung durch einen Verweis und die Lizenzangabe der
% % ursprünglichen Urheber auf den Materialien für Schülerinnen und
% % Schüler ist erforderlich.
% %
% % Die vollständige Sammlung der Dokumente steht unter
% % http://ddi.uni-wuppertal.de/material/materialsammlung/ zur
% % Verfügung.
% %
% % Das LaTeX-Paket zum Setzen der Dokumente der Sammlung steht unter
% % http://www.ctan.org/pkg/ zur Verfügung.
% %
% % ----- BEGIN ------------------------------------------------------
%		\fi
%	 \begin{lstlisting}[caption={},multicols=2,gobble=7]
%			\begin{sequencediagram}
%			  \scaleSequenzdiagramm{0.6}
%			  \newthread{fritz}{fritz}
%			  \newinst[2]{wecker}{wecker}
%			  \newinst[2]{lampe}{lampe}
%			  
%			  \begin{callself}[2]{fritz}{
%			    schlafeSechsStunden()}{}
%			  \end{callself}
%			  \begin{call}{fritz}{
%			    gibUhrzeit()}{wecker}{
%				  \diastring{5:30}
%				  }
%			  \end{call}
%			  \begin{callself}[2]{fritz}{
%			    schlafeEineStunde()}{}
%			  \end{callself}
%			  \begin{call}{fritz}{
%			    gibUhrzeit()}{wecker}{
%				  \diastring{6:30}
%				  }
%			  \end{call}
%			  \begin{call}{fritz}{
%			    schalteAn()}{lampe}{}
%			  \end{call}
%			\end{sequencediagram}
%	 \end{lstlisting}
%			\begin{sequencediagram}
%			  \scaleSequenzdiagramm{0.6}
%				\newthread{fritz}{fritz}
%				\newinst[2]{wecker}{wecker}
%				\newinst[2]{lampe}{lampe}
%				
%				\begin{callself}[2]{fritz}{schlafeSechsStunden()}{}
%				\end{callself}
%				\begin{call}{fritz}{gibUhrzeit()}{wecker}{\diastring{5:30}}
%				\end{call}
%				\begin{callself}[2]{fritz}{schlafeEineStunde()}{}
%				\end{callself}
%				\begin{call}{fritz}{gibUhrzeit()}{wecker}{\diastring{6:30}}
%				\end{call}
%				\begin{call}{fritz}{schalteAn()}{lampe}{}
%				\end{call}
%			\end{sequencediagram}
%		\iffalse
% %
% % ----- END --------------------------------------------------------
%		\fi
%  \end{beispiel}
%	\FloatBarrier
%
% \DescribeMacro{\newthreadtwo}
% Threads haben im Gegensatz zu Instanzen im Paket \texttt{pgf-umlsd}
% immer einen festen Abstand zu den Nachbarn. Durch den neuen Befehl
%
% {\centering
%  \cs{newthreadtwo}\oarg{Farbe}
%	\marg{Bezeichnung}\marg{Name}\marg{Abstand}\\
% }
%
%	\noindent ist es über den dritten Parameter möglich, diesen Abstand
%	zu verändern. Dabei verhält sich der neue Parameter für den Abstand
%	genauso wie der zugehörige optionale Parameter bei Instanzen (vgl.
%	\prettyref{ex:zusabstand}).
%
% \begin{beispiel}[zusabstand]{Zusätzlicher Abstand bei einem Thread}
%	 \begin{lstlisting}[caption={},gobble=7,multicols=2]
%			\begin{sequencediagram}
%			  \scaleSequenzdiagramm{0.6}
%			  \newthread{fritz}{fritz}
%			  \newthreadtwo{mutter}
%				  {mutter}{5cm}
%			  \newinst[2]{wecker}{wecker}
%			  \newinst[2]{lampe}{lampe}
%			  
%			  \begin{callself}[2]{fritz}{
%			    schlafe()}{}
%			  \end{callself}
%			  \begin{call}{fritz}{
%			    gibUhrzeit()}{wecker}
%			      {\diastring{5:30}}
%			  \end{call}
%			  \begin{callself}[2]{fritz}{
%			    schlafe()}{}
%			      \begin{call}{mutter}{
%			        gibUhrzeit()}{wecker}
%			          {\diastring{6:30}}
%			      \end{call}
%			      \begin{call}{mutter}{
%			        schalteAn()}{lampe}{}
%			      \end{call}
%			      \begin{call}{mutter}{
%			        weckeAuf()}{fritz}{}
%			      \end{call}
%			  \end{callself}
%			\end{sequencediagram}
%	 \end{lstlisting}
%			\begin{sequencediagram}
%			  \scaleSequenzdiagramm{0.6}
%			  \newthread{fritz}{fritz}
%			  \newthreadtwo{mutter}{mutter}{5cm}
%				\newinst[2]{wecker}{wecker}
%				\newinst[2]{lampe}{lampe}
%				
%				\begin{callself}[2]{fritz}{schlafe()}{}
%				\end{callself}
%				\begin{call}{fritz}{gibUhrzeit()}{wecker}{\diastring{5:30}}
%				\end{call}
%				\begin{callself}[2]{fritz}{schlafe()}{}
%				\begin{call}{mutter}{gibUhrzeit()}{wecker}{\diastring{6:30}}
%				\end{call}
%				\begin{call}{mutter}{schalteAn()}{lampe}{}
%				\end{call}
%				\begin{call}{mutter}{weckeAuf()}{fritz}{}
%				\end{call}
%				\end{callself}
%			\end{sequencediagram}
% \end{beispiel}
%
% \DescribeMacro{\nextlevel}
% Im Paket für Sequenzdiagramme ist vorgesehen, dass man mit
% \cmd{\prevlevel} wieder einen Schritt nach oben gehen kann.
% Zusätzlich wird ein Befehl \cmd{\nextlevel} bereitgestellt, mit dem
% man auch einen zusätzlichen Schritt nach unten gehen kann, um ggf.
% etwas mehr Platz und Abstand zu schaffen.
%
% \DescribeMacro{\anchormark}
% 	Durch den Befehl 
%	\begin{center}
%		\cs{anchormark}\oarg{Horizontale
%		Verschiebung}\marg{Nodename}\oarg{Skalierung}
%	\end{center}
%	können Objektdiagramme mit Beziehungsattributen ausgestattet werden,
%	die an der korrekten Stelle hinter dem Attributbezeichner beginnen
%	(vgl. \prettyref{ex:objattribut}).
%
%	\begin{beispiel}[objattribut][Objektdiagramm mit
%					Beziehungsattributen]{Objektdiagramme mit
%									Beziehungsattributen ausstatten
%					\\\footnotesize{(entnommen aus	\materialsammlung)}}
%		\iffalse
% % Der folgende kenntlich gemachte Abschnitt ist in der Zusammenarbeit
% % von Informatikreferendaren und ehemaligen Informatikreferendaren
% % der Studienseminare (heute ZfsL) Arnsberg, Hamm und Solingen
% % entstanden.
% %
% % Der Abschnitt steht unter der Lizenz: Creative Commons by-nc-sa
% % Version 4.0
% % http://creativecommons.org/licenses/by-nc-sa/4.0/deed.de
% %
% % Nach dieser Lizenz darf der Abschnitt beliebig kopiert und
% % bearbeitet werden, sofern das Folgeprodukt wiederum unter
% % gleichen Lizenzbedingungen vertrieben und auf die ursprünglichen
% % Urheber verwiesen wird.  Eine kommerzielle Nutzung ist
% % ausdrücklich ausgeschlossen.
% %
% % Die Namensnennung durch einen Verweis und die Lizenzangabe der
% % ursprünglichen Urheber auf den Materialien für Schülerinnen und
% % Schüler ist erforderlich.
% %
% % Die vollständige Sammlung der Dokumente steht unter
% % http://ddi.uni-wuppertal.de/material/materialsammlung/ zur
% % Verfügung.
% %
% % Das LaTeX-Paket zum Setzen der Dokumente der Sammlung steht unter
% % http://www.ctan.org/pkg/ zur Verfügung.
% %
% % ----- BEGIN ------------------------------------------------------
%		\fi
%	 \begin{lstlisting}[caption={},multicols=2,gobble=9,
%	   basicstyle=\footnotesize]
%				\begin{tikzpicture}
%				 [remember picture]
%				 \begin{object}
%				  [text width=5.5cm]
%					{gustavsRadiowecker}{-7,0}
%				  \attribute{standort= 
%					  \diastring{Gustavs 
%		          Zimmer} }
%				  \attribute{weckzeit= 
%					  \diastring{6:30} }
%				  \attribute{weckmodusAktiv= 
%					  \diastring{Wahr} }
%				  \attribute{hatAntenne=
%					  \anchormark{
%		          hatAntenne}[0.025] }
%				  \attribute{hatLautsprecher=  
%					  \anchormark{
%		          hatLautsprecher}[0.025]
%			      }
%				  \operation{einschalten()}
%				  \operation{ausschalten()}
%				  \operation{alarmAusloesen()}
%				 \end{object}
%				 \begin{object}
%				  [text width=4.5cm]
%					{gustav}{-27,-1}
%				  \attribute{name=
%					  \diastring{Gustav 
%		          Grabert} }
%				  \attribute{geburtstag=
%					  \diastring{3.10.1998} }
%				  \attribute{besitzt=
%					  \anchormark{besitzt}[0.025] 
%				  } 
%				  \attribute{kennt= 
%					  \anchormark{gKennt}[0.025]
%				  } 
%				 \end{object}
%				 \begin{object}
%					 [text width=4.5cm]
%					 {fridolin}{-27,-13}
%				  \attribute{name=
%					  \diastring{Fridolin Wagner}
%		      }
%				  \attribute{geburtstag= 
%					  \diastring{1.4.1999} }
%				  \attribute{kennt= 
%					  \anchormark{fKennt}[0.025]
%				  }
%				 \end{object}
%				 \begin{object}
%				  [text width=3.5cm]
%					{antenne}{10,-1}
%				  \attribute{laengeInMetern=2}
%				 \end{object}
%				 \begin{object}
%				  [text width=5.2cm]{
%					lautsprecher}{-7,-17}
%				  \attribute{
%					  untereFrequenzInHertz= 100}
%				  \attribute{
%					  obereFrequenzInHertz=18000}
%				 \end{object} 
%				\end{tikzpicture} 
%				\begin{tikzpicture}
%				 [remember picture,overlay]  
%				 \draw (hatAntenne.east)
%				   -- (antenne.south);
%				 \draw (hatLautsprecher)
%				   -- (lautsprecher.north);
%				 \draw (gKennt.south east)
%				   -- (fridolin.north);
%				 \draw (besitzt.east)
%				   -- (gustavsRadiowecker.west);
%				 \draw (fKennt.east)
%				   -- ($(fKennt.east)+(3.5,0)$) 
%					 -| ($(gustav.south)+(3,0.2)$) 
%					 -- ($(gustav.south east)
%					      +(-0.01,0.2)$);
%				\end{tikzpicture} 
%		 \end{lstlisting}
%		 \resizebox{\textwidth}{!}{
%				\begin{tikzpicture}[scale=0.33,remember picture]
%				 %
%					\begin{object}[text width=5.5cm]{gustavsRadiowecker}{-7,0}
%						\attribute{standort= \diastring{Gustavs Zimmer}}
%						\attribute{weckzeit= \diastring{6:30}}
%						\attribute{weckmodusAktiv= \diastring{Wahr}}
%						\attribute{hatAntenne= \anchormark{hatAntenne}[0.025]}
%						\attribute{hatLautsprecher=  \anchormark{hatLautsprecher}[0.025]}
%					 %
%						\operation{einschalten()}
%						\operation{ausschalten()}
%						\operation{alarmAusloesen()}
%					\end{object}
%				 %
%					\begin{object}[text width=4.5cm]{gustav}{-27,-1}
%						\attribute{name= \diastring{Gustav Grabert}}
%						\attribute{geburtstag= \diastring{3.10.1998}}
%						\attribute{besitzt= \anchormark{besitzt}[0.025]} 
%						\attribute{kennt= \anchormark{gKennt}[0.025]} 
%					\end{object}
%				 %
%					\begin{object}[text width=4.5cm]{fridolin}{-27,-13}
%						\attribute{name= \diastring{Fridolin Wagner}}
%						\attribute{geburtstag= \diastring{1.4.1999}}
%						\attribute{kennt= \anchormark{fKennt}[0.025]}
%					\end{object}
%				 %
%					\begin{object}[text width=3.5cm]{antenne}{10,-1}
%						\attribute{laengeInMetern= 2}
%					\end{object}
%				 % 
%					\begin{object}[text width=5.2cm]{lautsprecher}{-7,-17}
%						\attribute{untereFrequenzInHertz= 100}
%						\attribute{obereFrequenzInHertz= 18000}
%					\end{object} 
%				 \end{tikzpicture} 
%				 % 
%				 \begin{tikzpicture}[remember picture,overlay]  
%					\draw (hatAntenne.east) -- (antenne.south);
%					\draw (hatLautsprecher) -- (lautsprecher.north);
%					\draw (gKennt.south east) -- (fridolin.north);
%					\draw (besitzt.east) -- (gustavsRadiowecker.west);
%					\draw (fKennt.east) -- ($(fKennt.east)+(3.5,0)$) -| ($(gustav.south)+(3,0.2)$) -- ($(gustav.south east)+(-0.01,0.2)$);
%				 \end{tikzpicture} 
%				}
%	 \iffalse
% %
% % ----- END --------------------------------------------------------
%		\fi
%  \end{beispiel}
%	\FloatBarrier 
%
% \subsection{Das Paket \texttt{syntaxdi} -- Syntaxdiagramme}
% \label{paket:syntaxdi}
% Mit dem Paket \texttt{syntaxdi} und TikZ ist es möglich, einfache
% Syntaxdiagramme zu erstellen. Dazu sind folgende Elemente definiert
% worden, die automatisch durch Pfeile miteinander verbunden werden:
% \begin{smalldescription}
% 	\item[nonterminal] Definiert ein Non-Terminal
% 	\item[terminal] Definiert ein Terminal
%	\item[fnonterminal] Definiert ein Non-Terminal ohne automatische
%					Verzweigung
%	\item[fterminal] Definiert ein Terminal ohne automatische
%					Verzweigung
%	\item[point] Definiert einen Punkt, der ohne ankommenden Pfeil
%					gezeichnet wird
%	\item[endpoint] Definiert einen Punkt, der mit ankommenden Pfeil
%					gezeichnet wird
% \end{smalldescription}
%
% Damit kann z.\,B. das Syntaxdiagramm in \prettyref{ex:syntaxdiagramm}
% gezeichnet werden.
% \begin{beispiel}[syntaxdiagramm][Darstellung von
%				 Syntaxdiagrammen]{Darstellung von
%				 Syntaxdiagrammen\\\footnotesize{(Hier ist die Syntax von
%         \texttt{if-then-else} in Python dargestellt)}}
% \begin{lstlisting}[caption={},gobble=1,basicstyle=\footnotesize,multicols=2]
%	\node [] {};
%	\node [terminal] {if};
%	\node [nonterminal] 
%	      {Bedingung};
%	\node [terminal] {:};
%	\node [nonterminal]
%	      {Anweisungsblock};
%	\node (ersteReiheEnde)
%	      [point] {};
%	\node (ersteReiheEndeUnten)
%	      [point, below=of 
%	      ersteReiheEnde] {};
%	\node (zweiteReiheStartOben)
%	      [point, left=of 
%	      ersteReiheEndeUnten, 
%	      xshift=-75mm] {};
%	\node (zweiteReiheStart) 
%	      [point, below=of 
%	      zweiteReiheStartOben] {};
%	{
%		[start chain=elif 
%		      going right]
%			\chainin 
%			  (zweiteReiheStart);
%			\node [terminal] {elif};
%			\node [nonterminal] 
%			      {Bedingung};
%			\node [terminal] {:};
%			\node [nonterminal] 
%			      {Anweisungsblock};
%			\node (elifEnde) 
%			  [point] {};
%			\node (elifEndeOben) 
%			  [point, above=of 
%			  elifEnde] {};
%			\draw[->,left] 
%			  (elifEndeOben) 
%			  -- (ersteReiheEndeUnten);
%	}
%	\node (dritteReiheStart) 
%	  [point, below=of 
%	  zweiteReiheStart, 
%	  yshift=-5mm] {};
%	\node (vierteReiheStart) 
%	  [point, below=of 
%	  dritteReiheStart, 
%	  yshift=-5mm] {};
%	\node (vierteReiheEnde) 
%	  [point, xshift=84mm] {};
%	{
%		[start chain=else 
%		      going right]
%			\chainin 
%			  (dritteReiheStart);
%			\node [terminal] {else};
%			\node [terminal] {:};
%			\node (elseEnde) 
%			      [nonterminal] 
%			      {Anweisungsblock};
%			\draw[->] (elseEnde) 
%			  -| (vierteReiheEnde);
%	}
%	\node (ende) [endpoint] {};
% \end{lstlisting}
%\begin{tikzpicture}[syntaxdiagramm]
%	\node [] {};
%	\node [terminal] {if};
%	\node [nonterminal] {Bedingung};
%	\node [terminal] {:};
%	\node [nonterminal] {Anweisungsblock};
%	\node (ersteReiheEnde) [point] {};
%	\node (ersteReiheEndeUnten) [point, below=of ersteReiheEnde] {};
%	\node (zweiteReiheStartOben) [point, left=of ersteReiheEndeUnten,
%	      xshift=-75mm] {};
%	\node (zweiteReiheStart) [point, below=of zweiteReiheStartOben] {};
%	{
%		[start chain=elif going right]
%			\chainin (zweiteReiheStart);
%			\node [terminal] {elif};
%			\node [nonterminal] {Bedingung};
%			\node [terminal] {:};
%			\node [nonterminal] {Anweisungsblock};
%			\node (elifEnde) [point] {};
%			\node (elifEndeOben) [point, above=of elifEnde] {};
%			\draw[->,left] (elifEndeOben) -- (ersteReiheEndeUnten);
%	}
%	\node (dritteReiheStart) [point, below=of zweiteReiheStart,
%	      yshift=-5mm] {}; 
%	\node (vierteReiheStart) [point, below=of	dritteReiheStart,
%	      yshift=-5mm] {};
%	\node (vierteReiheEnde) [point, xshift=84mm] {};
%	{
%		[start chain=else going right]
%			\chainin (dritteReiheStart);
%			\node [terminal] {else};
%			\node [terminal] {:};
%			\node (elseEnde) [nonterminal] {Anweisungsblock};
%			\draw[->] (elseEnde) -| (vierteReiheEnde);
%	}
%	\node (ende) [endpoint] {};
%\end{tikzpicture}
% \end{beispiel} 
%	\FloatBarrier
%
% \subsection{Das Paket \texttt{relaycircuit} -- Schaltungen mit
%   					 Relais}
% \label{paket:relaycircuit}
%	\DescribeMacro{relais}
%		Durch das Paket \texttt{relaycircuit} ist es möglich Schaltungen
%		mit Relais zu zeichnen. Dazu wird die neue Knotenform
%		\textit{relais} deklariert, die sich in \textit{arbeits relais}
%		(Bezeichnung: AK) und \textit{ruhe relais} (Bezeichnung: RK)
%		aufteilen. \prettyref{ex:nand-relais} kann der Schaltplan eines
%		logischen NAND mittels Relais entnommen werden.
%
% \begin{beispiel}[nand-relais][Schaltpläne mit dem Paket
%				 \texttt{relaycircuit} erstellen]{Schaltpläne mit dem Paket
%				 \texttt{relaycircuit} erstellen\\	\footnotesize{Hier am
%         Beispiel einer NAND-Schaltung} }
%		\begin{lstlisting}[caption={},gobble=7,multicols=2,
%		                   basicstyle=\footnotesize]
%			\begin{tikzpicture}
%				\draw (0,6.8) node [left]
%				  {\(+\)} -- (9,6.8);
%				\draw (0,0) node [left] 
%				  {\(-\)} -- (9,0);
%				\draw (4.5,0) to[short, *-]
%				  (4.5,0) node [ground] {};
%		
%				\draw (7.4,2.5) to[short,*-] 
%				  (7.5,2.5) to[lamp] (9,2.5) 
%				  node[ground] {};
%
%				\draw (2.5,5.8) node[arbeits 
%				  relais] (a1) {};
%				\draw (2.5,4) node[arbeits 
%				  relais] (a2) {};
%				\draw (2.4,6.8) to[short,*-]
%				  (a1.anschluss);
%				\draw (a1.ausgabe) -- 
%				  (a2.anschluss);
%
%				\draw (2.5,1) node[ruhe 
%				  relais] (r1) {};
%				\draw (a2.ausgabe) -- 
%				  (r1.anschluss);
%				\draw (r1.ausgabe) 
%				  to[short,-*] (2.4,0);
%				\draw (5,1) node[ruhe relais] 
%				  (r2) {};
%				\draw (r2.ausgabe) to[short,-*] 
%				  (4.9,0);
%
%				\draw (7.5,1) node[arbeits 
%				  relais] (a3) {};
%				\draw (7.5,4) node[ruhe relais] 
%				  (r3) {};
%				\draw (a3.anschluss) -- 
%				  (r3.ausgabe);
%				\draw (a3.ausgabe) to[short,-*] 
%				  (7.4,0);
%				\draw (r3.anschluss) 
%				  to[short,-*] (7.4,6.8);
%
%				\draw (2.4,2.5) to[short,*-*] 
%				  (4.9,2.5) -| (a3.eingabe);
%				\draw (r2.anschluss) |- 
%				  (r3.eingabe);
%
%				\draw (0,4.7) node [left] {A} 
%				  to[short,-*] (0.2,4.7) -- 
%				  (a2.eingabe);
%				\draw (0.2,4.7) |- 
%				  (r1.eingabe);
%
%				\draw (0,2.1) node [left] {B} 
%				  to[short,-*] (0.4,2.1) -| 
%				  (r2.eingabe);
%				\draw (0.4,2.1) |- 
%				  (a1.eingabe);
%			\end{tikzpicture}
%		\end{lstlisting}
%		\scalebox{0.7}{
%		\begin{tikzpicture}
%			\draw (0,6.8) node [left] {\(+\)} -- (9,6.8);
%			\draw (0,0) node [left] {\(-\)} -- (9,0);
%			\draw (4.5,0) to[short, *-] (4.5,0) node [ground] {};
%		
%			\draw (7.4,2.5) to[short,*-] (7.5,2.5) to[lamp] (9,2.5) 
%			      node[ground] {};
%
%			\draw (2.5,5.8) node[arbeits relais] (a1) {};
%			\draw (2.5,4) node[arbeits relais] (a2) {};
%			\draw (2.4,6.8) to[short,*-] (a1.anschluss);
%			\draw (a1.ausgabe) -- (a2.anschluss);
%
%			\draw (2.5,1) node[ruhe relais] (r1) {};
%			\draw (a2.ausgabe) -- (r1.anschluss);
%			\draw (r1.ausgabe) to[short,-*] (2.4,0);
%			\draw (5,1) node[ruhe relais] (r2) {};
%			\draw (r2.ausgabe) to[short,-*] (4.9,0);
%
%			\draw (7.5,1) node[arbeits relais] (a3) {};
%			\draw (7.5,4) node[ruhe relais] (r3) {};
%			\draw (a3.anschluss) -- (r3.ausgabe);
%			\draw (a3.ausgabe) to[short,-*] (7.4,0);
%			\draw (r3.anschluss) to[short,-*] (7.4,6.8);
%
%			\draw (2.4,2.5) to[short,*-*] (4.9,2.5) -| (a3.eingabe);
%			\draw (r2.anschluss) |- (r3.eingabe);
%
%			\draw (0,4.7) node [left] {A} to[short,-*] 
%			      (0.2,4.7) -- (a2.eingabe);
%			\draw (0.2,4.7) |- (r1.eingabe);
%
%			\draw (0,2.1) node [left] {B} to[short,-*] 
%			      (0.4,2.1) -| (r2.eingabe);
%			\draw (0.4,2.1) |- (a1.eingabe);
%		\end{tikzpicture}
%		}
%	\end{beispiel}
% \FloatBarrier
% 
% \subsection{Das Paket \texttt{schulphy} -- Physik}
% \label{paket:schulphy}
% Zur Zeit ist das Paket Physik noch leer, bis auf das Setzen des
% Namens für Informationsblätter und Einbinden der Pakete
% \texttt{units}\footnote{
%   \url{http://mirror.ctan.org/macros/latex/contrib/units/units.pdf}
% },
% \texttt{circuitikz}\footnote{
%   \url{
%			http://mirror.ctan.org/graphics/pgf/contrib/circuitikz/circuitikzmanual.pdf
%	 }
% },
% \texttt{mhchem}\footnote{
%   \url{
%			http://mirror.ctan.org/tex-archive/macros/latex/contrib/mhchem/mhchem.pdf
%	 }
% }.
% Ein kurzes Beispiel zur Benutzung des Paketes \texttt{relaycircuit}
% soll an dieser Stelle genügen. Ausführlichere Hinweise können den
% entsprechenden Dokumentationen entnommen werden.
%
% \begin{beispiel}{Schaltpläne mit dem Paket \texttt{circuitikz}
%				 erstellen}
%	\begin{lstlisting}[caption={},gobble=5,basicstyle=\footnotesize]
%		\begin{circuitikz}
%		 \draw
%			(0,0)--(1,0) to[european resistor,l=$47$\,k$\Omega$] (3,0)--(5,0)
%			to[C, l=$470$\,$\mu$F] (7,0) -- (8,0)
%			(4.5,0) to[short, -*] (4.5,0) -- (4.5, -2)
%			(4.5,-2) -- (5,-2) to[voltmeter, l=$U_C$] (7,-2) -- (7.5,-2)
%			(7.5, -2) to[short, -*] (7.5,0)
%			(8,1) node[spdt, rotate=90] (Ums) {}
%			(Ums) node[right=0.4cm] {$WS$}
%			(Ums.out 1) node[left] {1}
%			(Ums.out 2) node[right] {2}
%			(0,0) |- (2,4) to[closing switch, l=$S$] (3,4) to[battery1,
%			  l=$U$] (5,4) -| (Ums.out 2)
%			(Ums.in) -- (8,0)
%			(Ums.out 1) |- (0,2) to[short, -*] (0,2)
%		 ;
%		\end{circuitikz}
%	\end{lstlisting}
%
%	\begin{circuitikz}[scale=0.6] \draw
%		(0,0) -- (1,0) to[european resistor, l=$47$\,k$\Omega$] 
%		(3,0) -- (5,0)
%		to[C, l=$470$\,$\mu$F] (7,0) -- (8,0)
%		(4.5,0) to[short, -*] (4.5,0) -- (4.5, -2)
%		(4.5,-2) -- (5,-2) to[voltmeter, l=$U_C$] (7,-2) -- (7.5,-2)
%		(7.5, -2) to[short, -*] (7.5,0)
%		(8,1) node[spdt, rotate=90] (Ums) {}
%		(Ums) node[right=0.4cm] {$WS$}
%		(Ums.out 1) node[left] {1}
%		(Ums.out 2) node[right] {2}
%		(0,0) |- (2,4) to[closing switch, l=$S$] (3,4) 
%		to[battery1, l=$U$] (5,4) -| (Ums.out 2)
%		(Ums.in) -- (8,0)
%		(Ums.out 1) |- (0,2) to[short, -*] (0,2)
%	;
%	\end{circuitikz}
%\end{beispiel}
% 
% \clearpage
% \section{Nutzung der einzelnen Klassen}
% \label{sec:klassen}
% \subsection{Die Klasse \texttt{schullzk} -- Lernzielkontrolle}
% \label{klasse:schullzk}
% Mit der Klasse Lernzielkontrolle wird eine Möglichkeit geschaffen,
% neben einem einheitlichen Kopf auch sofort die möglichen Punkte von
% Teilaufgaben zusammen zu rechnen. Diese werden dann bei den einzelnen
% Aufgaben, die einer Sektion (\cmd{\section}) entsprechen, angegeben.
% Falls einzelne Aufgaben mit Punkten innerhalb einer anderen Klasse
% gesetzt werden sollen, so kann das Paket \texttt{schullzk}
% eingebunden werden. Dann stehen die Befehle \texttt{punktesec},
% \texttt{aufgabensec}, \texttt{punkteitem} und \texttt{setzePunkte}
% wie gewohnt zur Verfügung. In \prettyref{ex:beispiellzk} wird eine
% etwas umfangreichere Lernzielkontrolle umgesetzt.
% 
% \DescribeMacro{\inhalt}
% Mit \cs{inhalt}\marg{Text} wird der Inhalt der Lernzielkontrolle
% angegeben. Dieser wird dann im Seitenkopf links neben dem Feld für
% den Namen aufgeführt.
% 
% \DescribeMacro{\punktesec}
% Der Befehl \cs{punktesec}\oarg{Zahl}\marg{Text} erstellt eine neue
% Sektion, hinter der die Gesamtzahl aller Punkte in dieser Sektion
% angeben sind. Neben der Angabe der Punkte durch wiederholte
% Verwendung von \cmd{\punkteitem} besteht die Möglichkeit mit dem
% optionalen Parameter eine zusätzliche Anzahl von Punkten für diesen
% Abschnitt zu vergeben.
% 
% \DescribeMacro{\aufgabensec}
% Als Erweiterung von \cmd{\punktesec} ist
% \cs{aufgabensec}\oarg{Zahl}\marg{Text} zu sehen. Er setzt zusätzlich
% noch \textbf{Aufgabe X.} vor den Titel der Sektion.
% 
% \DescribeMacro{\punkteitem}
% Durch die Neudefinition wird der Befehl
% \cs{punkteitem{}}\marg{Punkte} in dieser Klasse so abgeändert, dass
% die angegeben Punkte auch zu den Gesamtpunkten der Sektion mit
% hinzugezählt werden.
% 
% \DescribeMacro{\setzePunkte}
% Sollte in der Lernzielkontrolle eine Sektion benutzt werden, die
% nicht mit Hilfe von \cmd{\punktesec} oder \cmd{\aufgabensec}
% definiert wird, so muss vor der Definition der Sektion der Befehl
% \cmd{\setzePunkte} geschrieben werden. Nur so kann die Anzeige der
% Punkte für die anderen Sektionen richtig erfolgen.
%
%
% \begin{beispiel}[beispiellzk][Beispiel für eine Lernzielkontrolle in
%				 Informatik zum Thema »Was ist Informatik?«]{Beispiel für eine
%				 Lernzielkontrolle in Informatik zum Thema »Was ist
% Informatik?« \\\footnotesize{(entnommen aus \materialsammlung)}}
%		\iffalse
%% Der folgende kenntlich gemachte Abschnitt ist in der Zusammenarbeit
%% von Informatikreferendaren und ehemaligen Informatikreferendaren
%% der Studienseminare (heute ZfsL) Arnsberg, Hamm und Solingen
%% entstanden.
%%
%% Der Abschnitt steht unter der Lizenz: Creative Commons by-nc-sa
%% Version 4.0
%% http://creativecommons.org/licenses/by-nc-sa/4.0/deed.de
%%
%% Nach dieser Lizenz darf der Abschnitt beliebig kopiert und
%% bearbeitet werden, sofern das Folgeprodukt wiederum unter
%% gleichen Lizenzbedingungen vertrieben und auf die ursprünglichen
%% Urheber verwiesen wird.  Eine kommerzielle Nutzung ist
%% ausdrücklich ausgeschlossen.
%%
%% Die Namensnennung durch einen Verweis und die Lizenzangabe der
%% ursprünglichen Urheber auf den Materialien für Schülerinnen und
%% Schüler ist erforderlich.
%%
%% Die vollständige Sammlung der Dokumente steht unter
%% http://ddi.uni-wuppertal.de/material/materialsammlung/ zur
%% Verfügung.
%%
%% Das LaTeX-Paket zum Setzen der Dokumente der Sammlung steht unter
%% http://www.ctan.org/pkg/ zur Verfügung.
%%
%% ----- BEGIN ------------------------------------------------------
%		\fi
% 	\begin{lstlisting}[caption={},gobble=1,multicols=2,basicstyle=\tiny]
%\documentclass{schullzk}
%\usepackage[utf8]{inputenc}
%\inhalt{Definition Informatik}
%\begin{document}
% \punktesec{Aufgabe 1}
% \begin{aufgaben}
%   \punkteitem{8} \textbf{
%      Informatik -- zum
%      Begriff}
%   \begin{enumerate}
%    \item Geben Sie \textbf{Ihre}
%      Definition für Informatik an.
%    \item Ordnen Sie die folgenden
%      beiden Aussagen einer der
%      Ebenen \textbf{Pragmatik,
%      Syntax} oder \textbf{
%      Semantik} zu:
%      \begin{itemize}
%        \item »Eine Studentin 
%          sucht Literatur zu 
%          einem bestimmten
%          Thema.«
%        \item »Bildarchive werden 
%          häufig von
%          Journalistinnen
%          in Anspruch genommen, 
%          um einen Artikel zu
%          illustrieren; dabei 
%          ist meist das Thema
%          vorgegeben, aber 
%          nicht der Bildinhalt.«
%      \end{itemize}
%    \item Benennen Sie die 
%       Fachgebiete, in die
%       Informatik 
%       üblicherweise
%       aufgeteilt wird.
%    \item Ordnen Sie die folgenden
%       Begriffe den von Ihnen 
%       in 1\,c) genannten 
%       Fachgebieten zu:
%	
%       Fahrtroutenoptimierung, 
%       Software,
%       Programmiersprache,
%       Datenschutz, Linux, 
%       MP3-Player
%    \item Grenzen Sie die Begriffe
%       \textbf{Information, Daten}
%       und \textbf{Wissen}
%       voneinander ab.
%   \end{enumerate}
%  \punkteitem{8} \textbf{Informatik 
%    -- zum Begriff}
%   \begin{enumerate}
%    \item Grenzen Sie die Begriffe 
%      \textbf{Semantik, Pragmatik,
%      Syntax} voneinander ab.
%    \item Nennen Sie die
%      Fachgebiete der Informatik
%      und ordnen Sie die folgenden
%      Begriffe zu: 
%      Programmiersprache~Python,
%      Datenbank, 
%      Persönlichkeitsschutz,
%      Informatische Bildung,
%      Hardware, Betriebssystem
%    \item Ordnen Sie die folgende
%      Aussage einer der Ebenen
%      \textbf{Daten}, \textbf{
%      Wissen}, \textbf{Information}
%      zu: »Ein Dokument wird als 
%      Folge von Zeichen/Symbolen 
%      aufgefasst. Auf dieser Ebene
%      kann beispielsweise mit 
%      Methoden agiert werden, die
%      Zeichenketten in Texten oder
%      die nach Merkmalen wie Farbe,
%      Textur und Kontur suchen.«
%    \item Geben Sie \textbf{Ihre}
%      Definition für Informatik an.
%   end{enumerate}
% \end{aufgaben}
%
%		\end{lstlisting}
%		\iffalse
%%
%% ----- END ---------------------------------------------------------
%		\fi
% \end{beispiel}
%
% \FloatBarrier
%
% \subsection{Die Klasse \texttt{schulekl} -- Klausur}
% \label{klasse:schulekl}
% \DescribeMacro{\klausurname}
% Die Klasse \texttt{schulekl} ist eine Erweiterung der Klasse
% \texttt{schullzk} in Bezug auf die Kopfleiste. So wird automatisch
% der Titel zu »Klausur« geändert. Mit der Option \texttt{arbeit} bzw.
% \texttt{kursarbeit} kann dieser Titel auch auf »Klassenarbeit« bzw.
% »Kursarbeit« geändert werden. Außerdem kann die Zielgruppe durch
% \cs{klausurname}\marg{Text} spezifiziert werden.
% 
% \DescribeMacro{\datum}
% Mit Hilfe von \cs{datum}\marg{Text} kann das Datum in der Kopfzeile
% gesetzt werden. Wird es nicht angegeben, so wird das aktuelle
% Tagesdatum (\texttt{\today}) verwendet.
% 
% \DescribeMacro{\klausurergebnisangabe} 
% Der Befehl \cs{klausurergebnisangabe}\marg{sehr
% gut}\marg{gut}\marg{befriedigend}\\\marg{ausreichend}\marg{mangelhaft}\marg{ungenügend}
% ermöglicht es die Ergebnisvertei"-lung einer Klausur setzen zu
% lassen. Die Argumente des Befehls stellen die jeweilige Anzahl an
% Bewertungen mit der entsprechenden Note dar. Automatisch werden der
% gewichtete Notendurchschnitt und die Gesamtzahl berechnet und am Ende
% der Tabelle gedruckt. Sofern die Option \texttt{KMKpunkte} gesetzt
% die Klasse übergeben wird, so kann analog die Ergebnisverteilung für
% die Punkte 15 bis 1 durch
% \cs{klausurergebnisangabe}\marg{fünfzehn}\ldots\marg{null} angegeben
% werden.
%
%
% \subsection{Die Klasse \texttt{schuleub} -- Unterrichtsbesuch}
% \label{klasse:schuleub}
% Mit der Klasse \texttt{schuleub} wird die Grundlage für den Entwurf
% eines Unterrichtsbesuchs gelegt. Dabei wird automatisch eine
% entsprechende Titelseite erzeugt. Für das Examen können weitere
% wichtige Angaben hinzugefügt werden. Siehe dazu auch
% \ref{klasse:schuleub:examen}.
%
% \DescribeMacro{neuePO}
% \textbf{Hinweis:} Die aktuellen Vorgaben zum
% Unterrichtsprüfungsentwurf im Vorbereitungsdienst in
% Nordrhein-Westfalen erfordern neben einer schriftlichen Planung der
% Unterrichtsstunde als Entwurf eines Unterrichtsbesuchs auch die
% Darstellung der längerfristigen Zusammenhänge jeweils auf exakt fünf
% Seiten. Mit der Option \texttt{neuePO} werden automatisch die Option
% \texttt{examen} geladen und entsprechende Befehle und Umgebungen
% bereitgestellt, um die aktuellen Anforderungen setzen zu können (vgl.
% \ref{klasse:schuleub:neueop}).
%
% \DescribeMacro{bibBibtex}
% \DescribeMacro{bibBiblatexBibtex}
% In der Version 0.4 wurde das schule-Paket auf das Paket
% \texttt{biblatex}\footnote{
%   \url{http://mirrors.ctan.org/macros/latex/contrib/biblatex/doc/biblatex.pdf}
%	}
%	und \texttt{biber}\footnote{
%	 \url{http://mirror.ctan.org/biblio/biber/documentation/biber.pdf}
%	}
%  als Backend zur Erstellung und Verwaltung von
%	Literaturverzeichnissen umgestellt. Sofern die Verwendung der
%	vorherigen Pakete (\verb|natbib|) und Einstellungen erzwungen
%	werden sollen, sollte der Klasse die Option \verb|bibBibtex|
%	übergeben werden. Um in UTF8 kodierte Bibliotheken zu unterstützen,
%	wird \verb|biber| als Backend anstelle von \verb|bibtex| verwendet.
%	Falls dennoch \verb|bibtex| genutzt werden soll, so kann dies über
%	die Klassenoption \verb|bibBiblatexBibtex| erzwungen werden. Die
%	wichtigsten Befehle werden im Folgenden in Kürze aufgeführt.
%	
%	\DescribeMacro{\ExecuteBibliographyOptions}
%	\DescribeMacro{\bibliography}
%	\DescribeMacro{\printbibliography}
%	Mit dem Befehl
%	\cs{ExecuteBibliographyOptions}\oarg{entry-type}\marg{key=value}
%	können beliebige Optionen für das Paket \verb|biblatex| (mit
%	Ausnahme des Backends) gesetzt werden. So kann etwa das Aussehen,
%	der Zitierstil oder ein ebenda-Tracker eingestellt werden. Um eine
%	Bibliotheksdatei anzugeben, wird \cs{bibliography}\marg{Datei}
%	verwendet. Das Literaturverzeichnis wird durch den Befehl
%	\cs{printbibliography} gesetzt. Weitere Informationen seien der
%	zuvor erwähnten Dokumentation von \verb|biblatex| und \verb|biber|
%	zu entnehmen. 
% 
% \subsubsection{Daten}
% 
% \DescribeMacro{\thema}
% \DescribeMacro{\Thema}
% Mit Hilfe des Befehls \cs{thema}\marg{Text} kann das Thema des
% Unterrichtsbesuch festgelegt werden. Soll später das Thema im
% Verlaufe des Entwurfs genutzt werden, so kann dieses mit Hilfe von
% \cmd{\Thema} geschehen.
%  % 
% \DescribeMacro{\reihe}
% \DescribeMacro{\Reihe}
% Mit Hilfe des Befehls \cs{reihe}\marg{Text} kann die Reihe, innerhalb
% der die durchgeführte Stunde verortet ist, festgelegt werden. Soll
% später die Reihe im Verlauf des Entwurfs genutzt werden, so kann
% dieses durch \cmd{\Reihe} geschehen. Für die neue PO wird die Reihe
% auch auf dem Deckblatt ausgewiesen.
% 
% \DescribeMacro{\seminaradresse}
% \DescribeMacro{\seminarinfo}
% \DescribeMacro{\ort}
% \DescribeMacro{\besuchtitel}
% Im oberen Bereich der Titelseite werden die verschiedenen Angaben
% angezeigt. Diese können gesetzt werden mit Hilfe der Befehle
% \cs{seminaradresse}\marg{Text} für die mehrzeilige Anschrift des
% Seminars, \cs{ort}\marg{Text} für den Ort vor dem Datum und
% \cs{besuchtitel}\marg{Text} für die Angabe des Grundes des Besuchs,
% wie z.\,B. \enquote{2. Unterrichtsbesuch im Fach Informatik}.
% Zusätzlich kann mit \cs{seminarinfo}\marg{Text} der Seminartitel,
% etwa \enquote{Seminar für das Lehramt an Gymnasien und
% Gesamtschulen}, für das Deckblatt nach neuer PO gesetzt werden.
% 
% \DescribeMacro{\lerngruppe}
% \DescribeMacro{\datum}
% \DescribeMacro{\zeit}
% \DescribeMacro{\stunde}
% \DescribeMacro{\schule}
% \DescribeMacro{\raum}
% Um die Übersicht mit den Daten für den Unterrichtsbesuch zu füllen,
% dienen die folgenden Befehle:
% \begin{smallitemize}
%	\item \cs{lerngruppe}\oarg{Kurzform}\marg{Text} Bezeichnung der
%					Lerngruppe
% 	\item \cs{datum}\marg{Text} Datum des Unterrichtsbesuchs
%	\item \cs{zeit}\marg{Startzeit}\marg{Endzeit} Start und Endzeit. Der
%					Zusatz \enquote{Uhr} wird automatisch ergänzt.
% 	\item \cs{stunde}\marg{Zahl} Angabe der Stunde
% 	\item \cs{schule}\marg{Text} Name der Schule
% 	\item \cs{raum}\marg{Text} Name bzw. Nummer des Raums
% \end{smallitemize}
% Die Lerngruppe wird auch in der Mitte im Seitenkopf mit angegeben.
% Sollte der Name der Lerngruppe zu groß werden, ist es möglich hierfür
% eine Kurzform anzugeben, wie z.\,B.\\ 
% \verb|\lerngruppe[Diff Informatik]{Differenzierungskurs Informatik}|.
% 
% \DescribeMacro{\weiblich}
% \DescribeMacro{\maennlich}
% Um die Anzahl der \SuS zu bestimmen wird die Anzahl der Schülerinnen
% mit \cs{weiblich}\marg{Zahl} und die Anzahl der Schüler mit
% \cs{maennlich}\marg{Zahl} angegeben. Dadurch wird automatisch auch
% die Gesamtzahl der Lernenden berechnet und mit auf der Titelseite
% angegeben.
%
% \DescribeMacro{\foerderbedarf}
% In der neuen PO können auf dem Deckblatt mit
% \cs{foerderbedarf}\marg{Zahl} Schüler und Schülerinnen mit
% Förderbedarf explizit angegeben werden, deren Anzahl sich nicht auf
% die Gesamtzahl der Lernenden auswirkt. Sofern der Förderbedarf nicht
% angegeben oder auf \enquote{-1} gesetzt wird, wird (außer im Examen)
% der Förderbedarf ausgeblendet. Falls in jedem Fall die Anzeige des
% Förderbedarfs auf der Titelseite erzwungen werden soll, kann der
% Förderbedarf mit dem Wert \enquote{0} erzwungen werden.
% 
% \subsubsection{Beteiligte Personen}
% 
% \DescribeMacro{\referendar}
% Sowohl in der Kopfzeile als auch auf der Titelseite wird der
% Referendar aufgenommen. Er wird mit \cs{referendar}\marg{Name}
% angegeben. Sollte es sich um eine weibliche Referendarin handeln, so
% kann zusätzlich hinter dem Befehl in \oarg{Endung} die passende
% Endung, also \enquote{in}, für die Auflistung angegeben werden.
% Dieses sieht dann wie folgt aus: \verb|\referendar{Lisa Maus}[in]|.
%
% \DescribeMacro{\ausbildungsl}
% \DescribeMacro{\ako}
% \DescribeMacro{\schulleiter}
% \DescribeMacro{\hauptseminar}
% Im zweiten Abschnitt der Daten werden die für die Ausbildung
% beteiligten Personen angegeben. Auch hier kann bei einer weiblichen
% Person die Endung angegeben werden, analog wie bei \cmd{\referendar}.
% Die dazu gehörenden Befehle sind:
% \begin{smallitemize}
% 	\item \cs{ausbildungsl}\marg{Name} für den Ausbildungslehrer
%	\item \cs{ako}\marg{Name} für den Ausbildungskoordinator (alte PO)
% 	\item \cs{schulleiter}\marg{Name} für den Schulleiter
% 	\item \cs{hauptseminar}\marg{Name} für den Hauptseminarleiter
% \end{smallitemize}
% 
% \DescribeMacro{\fachEins}
% \DescribeMacro{\fachZwei}
% Dazu kommen die beiden Fachseminarleiter, bei denen zusätzlich das
% jeweilige Fach mit anzugeben ist. Die Befehle sehen dann wie folgt
% aus:
% \begin{smallitemize}
% 	\item  \cs{fachEins}\marg{Fach}\marg{Name} für den einen Fachleiter
%	\item  \cs{fachZwei}\marg{Fach}\marg{Name} für den anderen
%					Fachleiter
% \end{smallitemize}
% Auch bei ihnen gilt die Möglichkeit der Endung analog zum Referendar.
%
% \DescribeMacro{\foerderbedarfl}
% In der neuen PO kann die zusätzliche Lehrkraft zum inklusiven,
% gemeinsamen Unterricht mit
%	\begin{center}
%	  \cs{foerderbedarfl}
%		\marg{Vorname}\marg{Nachname}\marg{Förderschwerpunkt}
%	\end{center}
% angegeben und so auf dem Deckblatt ausgewiesen werden.
% 
% \subsubsection{Examen} 
% \label{klasse:schuleub:examen}
% Bei den Entwürfen der unterrichtspraktischen Prüfungen sind
% zusätzliche Angaben zu machen. So ist unter anderem am Ende des
% Entwurfs jeweils eine Erklärung bzw. eine Versicherung abzulegen,
% dass der Entwurf eigenständig angefertigt wurde. Um diese
% einzublenden ist der Klasse als weitere Option \texttt{examen} zu
% übergeben.
% 
% \DescribeMacro{\vorsitz}
% \DescribeMacro{\schulvertreter}
% \DescribeMacro{\fremderseminar}
% \DescribeMacro{\bekannterseminar}
% In diesem Fall sind auch die vier an der Prüfung beteiligen Personen
% mit anzugeben. Dafür stehen entsprechende Befehle bereit, bei denen
% auch wieder analog zum Referendar die Endung mit angegeben werden
% kann:
% \begin{smallitemize}
% 	\item \cs{vorsitz}\marg{Name} für den Prüfungsvorsitzenden
%	\item \cs{schulvertreter}\marg{Name} für den Schulvertreter (alte
%					PO)
%	\item \cs{fremderseminar}\marg{Name} für den fremden
%					Seminarausbilder
%	\item \cs{bekannterseminar}\marg{Name} für den  bekannten
%					Seminarausbilder
% \end{smallitemize}
%
% Bei Verwendung der Option \texttt{neuePO} besteht die
% Prüfungskommision (Stand: April 2014) aus dem Vorsitzenden (mit
% Option \texttt{[r]}) bzw. der Vorsitzenden (ohne Option weiblich),
% dem Seminarausbilder bzw. der Seminarausbilderin (mit Option
% \texttt{[in]}) und dem fremden Seminarausbilder bzw. der
% Seminarausbilderin (mit Option \texttt{[in]}) -- vgl.
% \prettyref{ex:upruef}.
% 
% \subsubsection{Spezielle Anforderungen durch die aktuelle PO}
% \label{klasse:schuleub:neueop}
% \DescribeMacro{teila}
% \DescribeMacro{teilb}
%	Die aktuellen Vorgaben zum Unterrichtsprüfungsentwurf im
%	Vorbereitungsdienst in Nordrhein-Westfalen verlangen in einem Teil A
%	eine schriftliche Planung der Unterrichtsstunde als Entwurf eines
%	Unterrichtsbesuchs. Der zu verfassende Text sollte in die Umgebung
%	\texttt{teila} eingeschlossen werden. Entsprechend wird auch die
%	Darstellung der längerfristigen Zusammenhänge als Teil B in der
%	Umgebung \texttt{teilb} gesetzt. Danach kann ein
%	Literaturverzeichnis ausgegeben werden. Weitere Überschriften
%	innerhalb der Teile sollten ab \texttt{subsection}-Niveau erstellt
%	werden, da innerhalb der Umgebungen die Nummerierung angepasst und
%	automatisch der Titel des jeweiligen Teils auf
%	\texttt{section}-Ebene ausgegeben wird. Sofern nicht die Option
%	\texttt{examen} benutzt wird, können auch benutzerdefinierte
%	Strukturierungen erstellt werden. Durch Benutzung der Umgebung
%	\texttt{teila} usw. wird die Strukturierung der Überschriften
%	entsprechend den Vorgaben für das Examen angepasst.
%
%  Ein Musterbeispiel für einen Unterrichtsprüfungsentwurf nach der
%  neuen Prüfungsordnung kann in \prettyref{ex:upruef} gefunden werden
%  (für den Stundenverlauf vgl. \prettyref{ex:stundenverlauf}).
%
%	\DescribeMacro{ziele}
%	Die Umgebung \verb|ziele| ermöglicht es die Lernziele bzw. den
%	Beitrag an der Kompetenzentwicklung der \SuS für die jeweilige
%	Unterrichtsstunde anzugeben. Es ist möglich über ein optionales
%	Argument ein \enquote{Haupt(lern)ziel} anzugeben.  Über das erste
%	Argument wird die Überschrift angegeben. Das zweite Argument
%	definiert den einleitenden Satz (vgl.  \prettyref{ex:lernziele}).
%
%	\newenvironment{ziele}[3][]{%
%		\ifthenelse{\not\isempty{#1}}{%
%			\textbf{Hauptlernziel:} #1
%
%		}{}
%		\textbf{#2:}
%	
%		#3
%		\begin{smallitemize}
%	}{
%		\end{smallitemize}
%	}
%	\begin{beispiel}[lernziele]{Angabe von Lernzielen}
%		\begin{lstlisting}[caption={},gobble=5]
%		 \begin{ziele}[Die \SuS entwickeln Ideen zur Abgrenzung und
%						       Definition des Fachs Informatik.]
%									 {Ziele/Kompetenzen}
%									 {Die \SuS können\dots}
%       \item \dots 
%       \item \dots 
%       \item \dots 
%      \end{ziele}
%		\end{lstlisting}
%		\begin{flushleft}
%		 \begin{ziele}[Die \SuS entwickeln Ideen zur Abgrenzung und
%						 Definition des Fachs Informatik.]{Ziele/Kompetenzen}{Die
%						 \SuS können\dots}
%       \item \dots 
%       \item \dots 
%       \item \dots 
%      \end{ziele}
%		\end{flushleft}
%	\end{beispiel}		
%
%	\DescribeMacro{kurzentwurf}
%	\DescribeMacro{zieleMulti}
%	Manchmal ist es notwendig, einen kurzen Vorentwurf einer
%	Unterrichtsstunde abzugeben. Oft wird dann die Vorgabe gestellt,
%	dass der Stundenverlaufplan und die konkreten Lernziele der
%	Unterrichtsstunde auf einer einzelnen Seite Platz finden sollen.
%	Hier ist es meist sinnvoll, eine Seite im Querformat zu benutzen.
%	Durch Angabe der Option \verb|kurzentwurf| wird das Format der Seite
%	und des Stundenverlaufplans automatisch angepasst (vgl.
%	\prettyref{ex:kurzentwurf}).
%
%	Es ist sinnvoll die zuvor beschriebene Umgebung \verb|lernziele| zu
%	benutzen, um die Lernziele der Stunde anzugeben. Damit der Platz im
%	Querformat besser ausgenutzt wird, werden durch Angabe der Option
%	\verb|zieleMulti| die Ziele in zwei Spalten gesetzt.
%
%	\begin{beispiel}[kurzentwurf]{Musterhafter Kurzentwurf einer Unterrichtsstunde}
%		\begin{lstlisting}[gobble=1,caption={},multicols=2]
%\documentclass[
%	a4paper,11pt,
%	kurzentwurf,
%	zieleMulti,
%	oneside,neuePO]
%{schuleub}
%\usepackage[
%	stundenverlaufquer,
%	stundenverlaufdidkom,
%	stundenverlaufASF]
%{schule}
%\thema{Was ist Informatik?}
%\referendar{Willi Wuster}
%\lerngruppe[EF]
%  {Einführungsphase}
%\datum{20.04.2024}
%
%\begin{document}
% \begin{ziele}
%	[Die \SuS entwickeln Ideen 
%	 zur Abgrenzung und 
%	 Definition des Fachs 
%	 Informatik.]
%	{Ziele/Kompetenzen}
%	{Die \SuS können\dots}
%	 \item \dots 
%	 \item \dots 
%	 \item \dots 
% \end{ziele}
% \begin{stundenverlauf}
%  Einstieg& Impuls & \UG & 
%   \didkom{}\\\hline
%  Erarbeitung & Arbeitsblätter 
%   & \GA & Plakate \didkom{}
%   \\\hline
%  Auswertung & Die Plakate 
%   werden vorgestellt & \SV & 
%   Plakate \didkom{}\\\hline
% \end{stundenverlauf}
%
%		\end{lstlisting}
%	\end{beispiel}
%
% \subsubsection{Anhängen externer Dokumente}
% \DescribeMacro{externesDokumentEinseitig}
% \DescribeMacro{externesDokumentMehrseitig}
% Es ist oft notwendig, dass Dokumente, die den \SuSn gegeben werden,
% in genau dieser Form an ein Dokument zur Unterrichtsplanung
% (Unterrichtsentwurf) gehängt werden. Da mit dem \texttt{schule}-Paket
% erstellte Dokumente im PDF-Format vorliegen, binden die Befehle
% \begin{center}\cs{externesDokumentEinseitig}
% 		\marg{Dateiname.pdf}
% \end{center}
% und
% \begin{center}
%   \cs{externesDokumentMehrseitig}\oarg{Optionen}
%	 \marg{Dateiname.pdf}
% \end{center}
% eine entsprechende Datei ein und passen die Skalierung automatisch an
% die Seitenränder an. Das optionale Argument des Befehls für ein
% Dokument mit mehreren Seiten wird als optionales Argument (etwa
% \texttt{nup=1x2}, \texttt{landscape}) an \cs{includepdf}
% weitergegeben -- allerdings nur ab Seite 2, da die erste Seite immer
% standardmäßig gesetzt. 
%
% Die bereitgestellten Befehle sollten \textbf{nur} benutzt werden, um
% den Umbruch zu verhindern, der entsteht, wenn eine einzelne bzw. die
% erste Seite einer PDF-Datei direkt unterhalb einer Überschrift
% (z.\,B. Anhang) platziert werden soll. Ansonsten sind die
% entsprechenden Aufrufe für \cs{includegraphics} bzw.
% \cs{includepdf}\footnote{
% 	\url{
%		http://mirror.ctan.org/tex-archive/macros/latex/contrib/pdfpages/pdfpages.pdf}
%	}
% direkt zu benutzen.
%
% \begin{beispiel}[upruef]{Musterhafter Aufbau des
%				 Unterrichtsprüfungsentwurf}
%	\begin{lstlisting}[caption={},gobble=1,multicols=2,basicstyle=\normalsize]
%\documentclass[
%		a4paper,11pt,
%		oneside,neuePO]
%	{schuleub}
%\usepackage[utf8]{inputenc}
%
%\thema{Mein Besuchsthema}
%\reihe{Reihe zum Test}
%\seminarinfo{Seminar für 
%           das Lehramt an
%           Gymnasien und 
%           Gesamtschulen}
%\seminaradresse{Teststr.~24, 
%        58035 Wursthausen}
%\ort{Wursthausen}
%\besuchtitel{Lustiger 
%               Besuch}
%\lerngruppe[EF]{
%		Einführungsphase}
%\datum{20.04.2024}
%\zeit{08:00}{10:00}
%\stunde{1}
%\schule{Traumgymnasium}
%\raum{B 224}
%
%\weiblich{20}
%\maennlich{10}
%\foerderbedarf{2}
%
%\referendar{Willi Wuster}
%
%\ausbildungsl{Müller}
%\foerderbedarfl{Sarbina}{
%		Simons}{Sehen}
%\ako{Meier}[in]
%\schulleiter{Humboldt}
%\hauptseminar{Lagrange}
%
%\fachEins{Informatik}{
%		Torvalds}
%\fachZwei{Chinesisch}{
%		Xianxu}
%
%\vorsitz{Newton}[r]
%		 % Frau Newton: 
%		 % \vorsitz{Newton}
%\schulvertreter{Einstein}
%\fremderseminar{Knuth}[in]
%\bekannterseminar{Turing}
%
%\begin{document}
% \begin{teila}
%  \subsection{Erster 
%               Punkt}
%   \begin{stundenverlauf}
%    \zeitpunkt{10:30 Uhr} 
%     Einstieg & Vortrag 
%      & LV & Tafel 
%      \\ \hline
%    \zeitpunkt{10:38 Uhr} 
%    \ldots{} & weiter im 
%     Verlauf der Stunde & 
%     EA & \\ \hline
%   \end{stundenverlauf}
% \end{teila}
%
% \begin{teilb}
%  \subsection{Erster Punkt}
% \end{teilb}
%		
% \addsec{
%  Literaturverzeichnis}
%		
% \begin{anhang}
%  \externesDokumentMehrseitig
%   {ab.pdf}
% \end{anhang}
%
%	 \end{lstlisting}
% \end{beispiel}
% \FloatBarrier
% \subsection{Die Klasse \texttt{schuleab} -- Arbeitsblatt}
% \label{klasse:schuleab}
% Die Klasse \texttt{schuleab} liefert die Grundlage für ein
% Arbeitsblatt. Durch ihre Nutzung wird das Paket \texttt{schule}
% eingebunden und kümmert sich um den Kopf des Dokuments. So ist oben
% links das Fach ggf. mit dem Jahrgang angegeben, in der Mitte der Name
% des Dokuments bzw. der Aufgabe und rechts oben die Angabe, dass es
% sich um ein Arbeitsblatt handelt. Diese Angabe im oberen rechten Teil
% kann ggf. um ein entsprechende Nummer ergänzt werden.
%
% \DescribeMacro{onesitepages}
%	Standardmäßig ist die Klasse so konfiguriert, dass ein Arbeitsblatt
%	mit nur einer Seite keine Seitenzahlen erhält. Mit der Option
%	\verb|onesitepages| können die Seitenzahlen wie gewohnt
%	eingeschaltet werden.
%
%	\DescribeMacro{showlastpage}
%	Durch Angabe der Option \verb|showlastpage| wird neben der aktuellen
%	Seitenzahl auch die Gesamtzahl aller Seiten des Dokuments
%	angezeigt, z.\,B. \enquote{Seite 5 von 10}. Damit die letzte Seite
%	beim Setzen ermittelt werden kann, werden automatisch entsprechende
%	Verknüpfungen gesetzt. Sofern sich die Seitenanzahl ändert, muss der
%	Aufruf von pdf\LaTeX\ wiederholt werden, um die richtige Zahl 
%	ausgeben zu können.
%
%	\DescribeMacro{kopfSuSName}
%	\DescribeMacro{\kopfSuSNameLaenge}
%  Sofern die \SuS auf dem Arbeitsblatt ihren Namen notieren sollen,
%	kann durch die Option \verb|kopfSuSName| ein entsprechendes Feld
%	gesetzt werden. Dazu wird in der Kopfzeile eine zweite Zeile
%	erzeugt. Die Länge des Eingabefelds kann bei Bedarf über den Befehl
%	\cs{kopfSuSName}\marg{Länge} verändert werden.
%
%	\DescribeMacro{kopfDatum}
%	\DescribeMacro{kopfDatumAtkuell}
%	\DescribeMacro{\kopfDatum}
%	\DescribeMacro{\kopfDatumLaenge}
%	Mit der Option \verb|kopfDatum| kann in der rechten Kopfzeile eine
%	weitere Zeile zur Angabe eines Datums gesetzt werden. Mit
%	\cs{kopfDatumLaenge}\marg{Länge} kann die Länge des Eingabefelds
%	verändert werden. Bei Bedarf kann auch die Option
%	\verb|kopfDatumAktuell| gewählt werden. Nun kann mit
%	\cs{kopfDatum}\marg{Datum} ein beliebiges Datum, z.\,B. \cs{today},
%	gesetzt werden.
%	
%	\DescribeMacro{\dokName}
% Mit dem Befehl \cs{dokName}\marg{Text} wird der Namen des Dokuments
% angeben, der wie oben beschrieben, in der Mitte des Kopfes
% dargestellt wird. Sollte dieser Teil fehlen, wird eine Warnung
% ausgegeben.
% 
% \DescribeMacro{\fach}
% Durch \cs{fach}\marg{Text} kann man angeben, für welches Fach das
% Arbeitsblatt ist. Auch hier führt die Nichtangabe zu einer Warnung.
% 
% \DescribeMacro{\jahrgang}
% Das Fach im Kopf kann mit \cs{jahrgang}\marg{Zahl} um einen Jahrgang
% ergänzt werden, wenn das Fach z.\,B. durch Einbindung eines
% fachspezifischen Paketes wie \texttt{schulinf} bereits definiert
% wurde. 
% 
% \DescribeMacro{\dokNummer}
% Mit \cs{dokNummer}\marg{Zahl} kann in der oberen rechten Ecke dem
% Text \enquote{Arbeitsblatt} eine zusätzliche Nummerierung
% \enquote{Nr. \textit{Zahl}} hinzugefügt werden.
%	\begin{beispiel}[abbottab][Arbeitsblatt zur Identifikation von
%					Objekten mit der \enquote{Methode nach Abbott}]{Arbeitsblatt
%					zur Identifikation von Objekten mit der \enquote{Methode
%	nach Abbott}\\ \footnotesize{entnommen aus: \materialsammlung} }
%		\iffalse
%% Der folgende kenntlich gemachte Abschnitt ist in der Zusammenarbeit
%% von Informatikreferendaren und ehemaligen Informatikreferendaren
%% der Studienseminare (heute ZfsL) Arnsberg, Hamm und Solingen
%% entstanden.
%%
%% Der Abschnitt steht unter der Lizenz: Creative Commons by-nc-sa
%% Version 4.0
%% http://creativecommons.org/licenses/by-nc-sa/4.0/deed.de
%%
%% Nach dieser Lizenz darf der Abschnitt beliebig kopiert und
%% bearbeitet werden, sofern das Folgeprodukt wiederum unter
%% gleichen Lizenzbedingungen vertrieben und auf die ursprünglichen
%% Urheber verwiesen wird.  Eine kommerzielle Nutzung ist
%% ausdrücklich ausgeschlossen.
%%
%% Die Namensnennung durch einen Verweis und die Lizenzangabe der
%% ursprünglichen Urheber auf den Materialien für Schülerinnen und
%% Schüler ist erforderlich.
%%
%% Die vollständige Sammlung der Dokumente steht unter
%% http://ddi.uni-wuppertal.de/material/materialsammlung/ zur
%% Verfügung.
%%
%% Das LaTeX-Paket zum Setzen der Dokumente der Sammlung steht unter
%% http://www.ctan.org/pkg/ zur Verfügung.
%%
%% ----- BEGIN ------------------------------------------------------
%		\fi
%		\begin{lstlisting}[caption={},gobble=7]
%			\usepackage[utf8]{inputenc}
%			\usepackage{schulinf}
%			\dokName{Fahrkartenauskunft}
%			\jahrgang{EF}
%
%			\begin{document}
%				\section*{Problembeschreibung Fahrkartenauskunft}
%					\subsection*{Ausgangssituation} 
%						Das örtliche Nahverkehrsunternehmen »NahUnt« will
%						an den Bushaltestellen Fahrscheinautomaten 
%						installieren. An dem Automaten kann der Kunde eine
%						Entfernungszone per Knopfdruck wählen. Es gibt drei
%						Entfernungszonen mit unterschiedlichen Preisen: 
%						1.Zone: 1,10~\euro, 2.Zone: 1,90~\euro, 3.Zone: 
%						4,20~\euro. In einem Display steht als erstes der 
%						Text »Bitte wählen Sie eine Entfernungszone aus«. 
%						Nach der Betätigung einer Entfernungszonentaste soll 
%						die ausgewählte Zone und der Preis angezeigt werden. 
%
%				\minisec{Aufgabe}
%					\begin{enumerate}
%						\item Ermitteln Sie die vorkommenden Objekte und 
%						      die zugehörigen Attribute und Attributwerte 
%						      und notieren Sie diese mit Objektkarten.
%						\item Erstellen Sie das Objektdiagramm.
%						\item Fassen Sie die Objekte geeignet zu Klassen 
%						      zusammen und dokumentieren diese mit 
%						      Klassenkarten.
%						\item Erstellen Sie das Klassendiagramm.
%					\end{enumerate}
%			
%		\end{lstlisting}
%		\iffalse
% %
% % ----- END --------------------------------------------------------
%		\fi
%	\end{beispiel}
%
% \subsection{Die Klasse \texttt{schulein} -- Informationsblatt}
% \label{klasse:schulein}
% Die Klasse \texttt{schulein} entspricht komplett der Klasse
% \texttt{schuleab} mit dem Unterschied, dass in der oberen rechten
% Ecke \enquote{Informationsblatt} anstatt \enquote{Arbeitsblatt}
% steht. Gesteuert wird der Kopf sonst genauso wie beim Arbeitsblatt.
% 
% \subsection{Die Klasse \texttt{schuleue} -- Übersichtsblatt}
% \label{klasse:schuleue}
% Auch die Klasse \texttt{schuleue} entspricht komplett der Klasse
% \texttt{schuleab}. Hier ist nur \enquote{Arbeitsblatt} im Kopf gegen
% \enquote{Übersicht} ausgetausht. Die Steuerung läuft analog.
% 
% \subsection{Die Klasse \texttt{schullsg} -- Lösung}
% \label{klasse:schullsg}
% Genauso wie beim Informationsblatt und bei der Übersicht stammt die
% Klasse \texttt{schullsg} vond der Klasse \texttt{schuleab} mit dem
% Zusatz \enquote{Lösung} in der rechten oberen Ecke ab.
% 
% \subsection{Die Klasse \texttt{schuleit} -- Leitprogramm}
% \label{klasse:schuleit}
% Die Klasse \texttt{schuleit} dient als Grundlage für Leitprogramme,
% mit denen sich Lernende selbstständig zusammenhängende Gegenstände
% erarbeiten können. Neben einem einheitlichen Layout bietet es die
% Möglichkeit, in den Kapiteln Aufgaben unterzubringen und diese mit
% Hinweisen und Lösungen zu verknüpfen, die jeweils in einem späteren
% Kapitel komplett dargestellt werden. Dabei werden auch entsprechende
% Links gesetzt, um zwischen den Hinweisen oder Lösungen und dem
% Aufgabentext springen zu können.
%
% Um ein Leitprogramm zu erstellen muss nur die Klasse des Dokuments
% auf \verb|\documentclass{schuleit}| gesetzt werden. Dadurch wird eine
% angepasste \verb|scrreprt|-Klasse geladen. Die oberste
% Strukturierungsebene für das Dokument ist somit \verb|chapter|.
%
% \subsubsection{Umgebungen für Aufgaben im Leitprogramm}
% \DescribeMacro{Aufgabe}
% \DescribeMacro{Aufgaben}
% Zur Darstellung von Aufgaben gibt es zwei Umgebungen. Die Umgebung
% \verb|Aufgabe| ermöglicht es in einem grau unterlegten Kasten einen
% Aufgabentext zu setzen. Die Aufgaben werden durchgehend in der Form
% \enquote{\enquote{aktuelle Kapitelnummer}.\enquote{Aufgabennummer}}
% nummeriert.
% \begin{beispiel}{Setzen einer einzelnen Aufgabe}
%  \begin{lstlisting}[gobble=5,caption={}]
%		\begin{Aufgabe}
%      Begründen Sie Ihre vermutete Zuordnung der folgenden 
%			Elemente aus der Informatik zu den sechs Fachgebieten 
%			der Informatik. 
%      \begin{enumerate}
%       \item Ausspionieren von Informatiksystemen  
%       \item Warten auf die Antwort einer Suchmaschine
%       \item Ein Dokument wird ausgedruckt
%       \item Jeder Mensch soll programmieren können
%       \item Die Geschwindigkeit eines Prozessors hat zugenommen
%       \item Soziale Netzwerke
%      \end{enumerate}			
%		\end{Aufgabe}
%	\end{lstlisting}
% \end{beispiel}
%
% Um eine Aufgabe mit Teilaufgaben setzen zu können, steht dagegen die
% Umgebung \verb|Aufgaben| zur Verfügung. Dann wird eine angepasste
% Aufzählung geladen. Der erste optionale Parameter der Umgebung kann
% benutzt werden, um die Zählweise der Teilaufgaben festzulegen. Hierzu
% können entsprechend dem \verb|paralist|"~Paket\footnote{\url{
% mirror.ctan.org/tex-archive/macros/latex/contrib/paralist/paralist.pdf
% }} z.\,B. Werte wie \emph{a)} oder \emph{i)} gewählt werden. Sofern
% nichts angegeben wird, werden Buchstaben in der Form \enquote{a),
% b), \ldots} zur Nummerierung der Teilaufgaben verwendet. Der zweite
% optionale Parameter kann einen einleitenden Erklärungstext enthalten.
% Die gesamte Aufgabe erhält nach dem selben Prinzip wie \verb|Aufgabe|
% eine Nummer, die in der Überschrift angezeigt wird. Beide Umgebungen
% können innerhalb eines Dokuments kombiniert werden.
%
% \begin{beispiel}{Setzen einer Aufgabe mit Teilaufgaben}
%  \begin{lstlisting}[gobble=5,caption={}]
%	  \begin{Aufgaben}[i)][Begründen Sie Ihre vermutete Zuordnung 
%			 der folgenden Elemente aus der Informatik zu den sechs 
%			 Fachgebieten der Informatik.] 
%     \item Ausspionieren von Informatiksystemen  
%     \item Warten auf die Antwort einer Suchmaschine
%     \item Ein Dokument wird ausgedruckt
%     \item Jeder Mensch soll programmieren können
%     \item Die Geschwindigkeit eines Prozessors hat zugenommen
%     \item Soziale Netzwerke
%		\end{aufgabe}
%	\end{lstlisting}
% \end{beispiel}
%
% \subsubsection{Angabe von Hinweisen zu Aufgaben}
% \DescribeMacro{\AufgabeHinweis}
% \DescribeMacro{\AufgabenHinweis}
% Mit \cs{AufgabeHinweis}\marg{Hinweistext} bzw.
% \cs{AufgabenHinweis}\marg{Hinweistext} können Aufgaben mit Hinweisen
% verknüpft werden. Dazu wird ein Link mit einem blauen \enquote{H}
% neben die entsprechende Aufgabe gesetzt. Die beiden Makros
% unterscheiden sich in der Bedienung nicht. Falls einer Teilaufgabe
% ein Hinweis hinzugefügt werden soll, muss der Befehl
% \cs{AufgabenHinweis} benutzt werden, sofern die Referenzierung mit
% der speziellen Teilaufgabe notwendig ist
% (vgl.\,S.~\pageref{schuleit-aufgaben}).
%
% \begin{beispiel}{Setzen von Hinweisen innerhalb einer Aufgabe}
%  \begin{lstlisting}[gobble=5,caption={}]
%	  \begin{Aufgaben}[i)][Begründen Sie Ihre vermutete Zuordnung 
%			 der folgenden Elemente aus der Informatik zu den sechs 
%			 Fachgebieten der Informatik. \AufgabeHinweis{Eine Übersicht 
%			 zu den Fachgebieten findet sich in der Abbildung~2.}]
%     \item Ausspionieren von Informatiksystemen  
%     \item Warten auf die Antwort einer Suchmaschine
%     \item Ein Dokument wird ausgedruckt
%     \item Jeder Mensch soll programmieren können
%     \item Die Geschwindigkeit eines Prozessors hat zugenommen
%		 \item Soziale Netzwerke \AufgabenHinweis{In welchen sozialen
%						 Netzwerken sind Sie angemeldet?}
%		\end{aufgabe}
%	\end{lstlisting}
% \end{beispiel}
%
% \DescribeMacro{\hinweisanzeigen}
% Die definierten Hinweis(texte) können zentral an einer Stelle im Dokument
% ausgegeben werden. Dazu wird einfach der Befehl \cs{hinweisanzeigen}
% aufgerufen.
%
% \subsubsection{Angabe von Lösungen zu Aufgaben}
% \DescribeMacro{\AufgabeLoesung}
% \DescribeMacro{\AufgabenLoesung}
% Mit \cs{AufgabeLoesung}\marg{Lösungstext} bzw.
% \cs{AufgabenLoesung}\marg{Lösungstext} können Lösungen mit
% Aufgaben(teilen) verknüpft werden. Sofern eine Lösung explizit mit
% einer Teilaufgabe verknüpft werden soll, so muss \cs{AufgabenLoesung}
% verwendet werden (vgl.\,S.~\pageref{schuleit-aufgaben}).
%
% \begin{beispiel}{Setzen von Lösungen innerhalb einer Aufgabe}
%  \begin{lstlisting}[gobble=5,caption={}]
%	  \begin{Aufgaben}[i)][Begründen Sie Ihre vermutete Zuordnung 
%			 der folgenden Elemente aus der Informatik zu den sechs 
%			 Fachgebieten der Informatik. \AufgabeLoesung{Sie finden 
%			 die Lösung in Abbildung~2.}]
%		 \item Ausspionieren von Informatiksystemen
%						 \AufgabenLoesung{Informatik und Gesellschaft --
%						 Verantwortung} 
%     \item Warten auf die Antwort einer Suchmaschine
%     \item Ein Dokument wird ausgedruckt
%     \item Jeder Mensch soll programmieren können
%     \item Die Geschwindigkeit eines Prozessors hat zugenommen
%		 \item Soziale Netzwerke 
%		\end{aufgabe}
%	\end{lstlisting}
% \end{beispiel}
%
% \DescribeMacro{\loesungzeigen}
% Die Lösungen können an einer beliebigen Stelle im Dokument durch den
% Befehl \cs{loesungzeigen} gesetzt werden.
%
% \clearpage
%
%	 \includepdf[nup=1x2,pages=-,scale=0.7,landscape,frame=true,
%     pagecommand={\label{schuleit-aufgaben}\thispagestyle{plain}}]
% 	  {doc/schule-schuleit-aufgaben-crop.pdf}
%  
%		\StopEventually{
%    }
%		\clearpage
%		\section{ToDo}
%			Im Laufe der Jahre wurde das Paket immer wieder erweitert. Nicht
%			nur die Anpassung an veränderte Anforderungen, etwa bei den
%			Unterrichtsbesuchen, sondern auch neue Funktionalitäten fließen
%			in das Paket ein. Die folgende Liste soll die nächsten geplanten
%			Funktionen bzw. Entwicklungsschritte angeben.
%			\subsection{Erledigt -- Changelog}
%			\iffalse
%			% Der String Version: Version: 0.6 respektive 2015-05-07 wird beim Aufruf
%			% des build-scripts durch die passende Nummer bzw. das passende
%			% Datum ersetzt.
%			\fi
%				\begin{itemize}
%				 \item \textbf{Version: 0.6} -- 2015-05-07
%						\begin{itemize}
%							\item TikZ-Library circuits.logic.IEC in \texttt{schuleinf} eingebunden.
%							\item Paket \texttt{varwidth} eingebunden.
%							\item Hinweis auf neue Version in der Dokumentation ergänzt.
%						\end{itemize}
%				 \item \textbf{Version: 0.5} -- 2015-02-25
%						\begin{itemize}
%							\item Angabe für Datum in Klasse \texttt{schullzk} 
%											verschoben
%							\item \cs{keineSeitenzahlen} schaltet die Seitenzahlen 
%											aus
%							\item Erklärungen in der Dokumentation zur Verwendung 
%											von Kopf- und Fußzeilen	hinzugefügt
%							\item Kopfzeile zur Orientierung im Kapitel und
%											Abschnitt in Leitprogrammen ergänzt
%							\item Probleme beim Einbinden von PDFs und der 
%											automatischen Skalierung von \texttt{adjustbox}
%											behoben.
%							\item Nummerierung in Klasse \texttt{schuleub} bereinigt.
%							\item Beispiele für Unterrichtsbesuche korrigiert.
%							\item Durchstreichen von Werten in
%											Wertetabellen/Schreibtischtests mittels
%											\cs{so}\marg{Wert}.
%							\item Beziehungsattribute in Objektdiagrammen können mit
%											\begin{center}
%												\cs{anchormark}\oarg{Horizontale Verschiebung}
%												\marg{Nodename}\oarg{Skalierung}
%											\end{center}
%											gesetzt werden.
%							\item Ergebnisangabe in Klausuren in der Dokumentation
%											korrigiert und zusätzlich die Möglichkeit der in
%											der Oberstufe üblichen Bewertung nach
%											KMK-Punkten implementiert -- Dank für die
%											Hinweise an Adrian Devries
%						\end{itemize}
%				 \item \textbf{Version: 0.4} -- 2014-09-07
%				  \begin{itemize}
%						\item Fix für \verb|\chb[r]| in Lösungen im
%										Leitprogramm.
%						\item Dokumentation des Leitprogramms begonnen.
%						\item Bereinigung der Nummerierung von Aufgaben für
%										Hinweise und Lösungen im Leitprogramm
%						\item Möglichkeit zum Anpassen der Breiten und Bezeichner
%										im Stundenverlauf
%						\item Anpassungen in der Klasse schuleub, um einen
%										Kurzentwurf zu erstellen
%						\item Falsche Option \texttt{stundenverlauf-plan-quer}
%										berichtigt...
%						\item Bei Verwendung des Pakets \texttt{beamerarticle} gab
%										es bisher eine Überschneidung mit dem Befehl
%										\cs{Loesung}. Daher wurde dieser umbenannt in
%										\cs{AufgabenLoesung}.
%						\item Umstellung des Literaturverzeichnis in
%										\texttt{schuleub} auf \texttt{biblatex} und
%										\texttt{bibtex}.
%						\item Lösungen in Lücken angeben können.
%						\item Checkboxen als Lösung markieren.
%						\item Optional ein Feld zur Notierung eines Namens
%										und des Datums in Arbeitsblättern hinzugefügt.
%						\item Optional die gesamte Seitenzahl in Arbeitsblättern 
%										einblenden.
%						\item Versionierung in READMEs, Dokumentation, Paketen und
%										Klassen vereinheitlicht.
%						\item In der Klasse schuleub wurden weitere Anpassungen
%										vorgenommen. Es ist nun auch in der aktuellen
%										Prüfungsordnung möglich die strikten Vorgaben an
%										das Examen losgelöst von einem \textit{normalen}
%										Entwurf zu setzen.
%						\item Fehler im Satz der Doku bereinigt.
%						\item Dem \texttt{xspace}-Paket wurden \textit{exceptions}
%										zur Erkennung von \texttt{enquote} mitgegeben.
%				  \end{itemize}
%				 \item \textbf{Version 0.3} -- 2014-07-30
%					\begin{itemize}
%						\item Durch Angabe der Option \texttt{stundenverlaufquer}
%										ist es nun möglich den Stundenverlaufsplan im
%										Querformat zu setzen. 
%						\item Außerdem wurden die Option
%										\texttt{stundenverlaufdidkom} und der Befehl
%										\cs{didkom}\marg{Text} deklariert, um eine
%										zusätzliche Spalte \enquote{didaktischer
%										Kommentar} in den Verlaufsplan zu integrieren.
%						\item Bereitstellung des Befehls
%						\verb|\punkteitemloesung|, um Lösungen einer
%						Aufgabenumgebungen auf einer extra Seite entsprechend der
%						Nummerierung anzeigen zu können.
%						\item Fehlendes \enquote{und} im Befehl \verb|\SuSn|
%										ergänzt.
%						\item Verweise auf \url{http://ctan.org} wurden in der
%										Dokumentation angepasst.
%					\end{itemize}
%				 \item \textbf{Version 0.2} -- 2014-07-21
%					\begin{itemize}
%						\item Festlegung der Standardkodierung auf \texttt{utf8}
%										in \texttt{inputenc}.
%						\item Festlegung der Zeichenkodierung auf \texttt{T1} in
%										\texttt{fontenc}.
%						\item Integrierung einer Möglichkeit den Stil von
%										Anführungszeichen zu standardisieren.
%						\item Seitenzahlen für einseitige Arbeitsblätter
%										deaktiviert.
%					\end{itemize}
%				\end{itemize}
%			\subsection{Must-have}
%				\begin{itemize}
%					\item Optimierung der Anzeige von Lösungen
%						\begin{itemize}
%							\item Es sollte eine orthogonale und optimierte
%											Anzeige in Kombination mit dem Leitprogramm
%											entstehen.
%							\item Ausgabe auf einem extra Lösungsblatt: 
%								\begin{itemize}
%									\item Aussehen der Überschriften optimieren
%									\item Zählung der Punkte in Klausuren anpassen, wenn
%													die Punkte auch in der Lösung angegeben
%													wurden
%								\end{itemize}
%							\item \texttt{verbatim}, \texttt{lstlisting},
%											\texttt{lstinline} in Lösungen nutzbar
%											machen
%							\item Lösungen innerhalb der Aufgabenstellung anzeigen:
%								\begin{itemize}
%									\item Sofern die Lösung nicht mit gesetzt werden
%													soll, trotzdem eine Möglichkeit zu haben,
%													den dafür benötigten Platz zu reservieren,
%													so dass Lösungs- und Arbeitsblatt bis auf
%													den Lösungstext gleich gesetzt werden
%								\end{itemize}
%						\end{itemize}
%					\item Die Dokumentation des Leitprogrammes muss noch
%									erfolgen; die Funktionen sind noch nicht
%									vollständig. (angefangen, siehe Version 0.4)
%				\end{itemize}
%			\subsection{Nice-to-have}
%				\begin{itemize}
%					\item Weitere für die Schule nützliche Dokumenttypen
%									integrieren, z.\,B. Lerntagebücher.
%				\end{itemize}
%
%
%\section{Implementation}
%\iffalse
%    \begin{macrocode}
%<*relaycircuit.sty>
%    \end{macrocode}
%\fi
% \subsection{Das Paket \texttt{relaycircuit}}
%	 Die ausführliche Beschreibung des Pakets ist in der
%	 Paketbeschreibung (\ref{paket:relaycircuit}) zu finden.
%
%  Beginn der Definition, Voraussetzung der \LaTeXe{} Version und die
%  eigene Identifizierung
%    \begin{macrocode}
\NeedsTeXFormat{LaTeX2e}[1995/12/01]
\ProvidesPackage{relaycircuit}[2015/05/07 v0.6 %
                               Relais-Schaltungen mit TikZ]
%    \end{macrocode}
% Einbinden der geforderten Pakete
%    \begin{macrocode}
\RequirePackage{tikz}
\RequirePackage[siunitx,european]{circuitikz}
\usetikzlibrary{arrows,shadows,shapes.misc,scopes}
%    \end{macrocode}
%
% \subsubsection{TikZ-Definitionen}
%
%    \begin{macrocode}
\pgfdeclareshape{ruhe relais}{
	  \savedanchor\northwest{
			\pgf@y = 0.7cm
			\pgf@x = -1cm
	  }
	  \savedanchor\left{%
	  	\pgf@y=0pt
	  }
	  \savedanchor\inEingabe{%
			\pgf@y= 0.7cm
			\pgf@x= -0.7cm
	  }
		\anchor{eingabe}{
			\inEingabe
		}
		\anchor{anschluss}{
			\inEingabe
			\pgf@x= -\pgf@x
		}
		\anchor{ausgabe} {
			\inEingabe
			\pgf@x= -\pgf@x
			\pgf@y= -\pgf@y
		}
  	\anchor{center}{
			\northwest
			\pgf@y=0pt
			\pgf@x=-.8\pgf@x  
	  }
	  \anchor{east}{
	  	\left
	  	\pgf@x=-.8\pgf@x  
	  }
	  \anchor{west}{
	  	\left
	  }
	  \anchor{south}{
		\northwest
		\pgf@x=0pt
		\pgf@y=-\pgf@y
	  }
	  \anchor{north}{
		\northwest
		\pgf@x=0pt
	  }
	  \anchor{south west}{
		\northwest
		\pgf@y=-\pgf@y
	  }
	  \anchor{north east}{
		\northwest
		\pgf@x=-\pgf@x
	  }
	  \anchor{north west}{
		\northwest
	  }
	  \anchor{south east}{
		\northwest
		\pgf@x=-\pgf@x
		\pgf@y=-\pgf@y
	  }

	  \backgroundpath{

			%Eingabe und Erdung
 			\pgfpathmoveto{\pgfpoint{-0.7cm}{0.7cm}}
			\pgfpathlineto{\pgfpoint{-0.7cm}{-0.6cm}}
			\pgfpathmoveto{\pgfpoint{-0.8cm}{-0.6cm}}
			\pgfpathlineto{\pgfpoint{-0.6cm}{-0.6cm}}

			%Anschluss
 			\pgfpathmoveto{\pgfpoint{0.7cm}{0.7cm}}
			\pgfpathlineto{\pgfpoint{0.7cm}{0.3cm}}
			\pgfpathlineto{\pgfpoint{0.6cm}{-0.35cm}}
			%Ausgabe
 			\pgfpathmoveto{\pgfpoint{0.7cm}{-0.7cm}}
			\pgfpathlineto{\pgfpoint{0.7cm}{-0.3cm}}
			\pgfpathlineto{\pgfpoint{0.6cm}{-0.3cm}}

			\pgfusepath{stroke}

			\pgfsetdash{{2pt}{2pt}}{1pt} 
 			\pgfpathmoveto{\pgfpoint{-0.7cm}{0cm}}
			\pgfpathlineto{\pgfpoint{0.65cm}{0cm}}

			\pgfusepath{stroke}
			\pgfsetdash{}{0pt} 

			\pgfusepath{draw}

			%Kasten
			\pgfsetlinewidth{2\pgflinewidth}
			\pgfpathrectanglecorners{\pgfpoint{1cm}{0.5cm}%
                                    }{\pgfpoint{-1cm}{-0.5cm}}
			
			\pgfusepath{draw}

			\pgftext[base,y=-0.4cm]{RK}
	  }
}

\pgfdeclareshape{arbeits relais}{
		\inheritsavedanchors[from=ruhe relais] 
		\inheritanchor[from=ruhe relais]{center}
		\inheritanchor[from=ruhe relais]{north}
		\inheritanchor[from=ruhe relais]{south}
		\inheritanchor[from=ruhe relais]{west}
		\inheritanchor[from=ruhe relais]{east}
		\inheritanchor[from=ruhe relais]{north west}
		\inheritanchor[from=ruhe relais]{south west}
		\inheritanchor[from=ruhe relais]{north east}
		\inheritanchor[from=ruhe relais]{south east}
		\inheritanchor[from=ruhe relais]{eingabe}
		\inheritanchor[from=ruhe relais]{ausgabe}
		\inheritanchor[from=ruhe relais]{anschluss}

	  \backgroundpath{

			%Eingabe und Erdung
 			\pgfpathmoveto{\pgfpoint{-0.7cm}{0.7cm}}
			\pgfpathlineto{\pgfpoint{-0.7cm}{-0.6cm}}
			\pgfpathmoveto{\pgfpoint{-0.8cm}{-0.6cm}}
			\pgfpathlineto{\pgfpoint{-0.6cm}{-0.6cm}}

			%Anschluss
 			\pgfpathmoveto{\pgfpoint{0.7cm}{0.7cm}}
			\pgfpathlineto{\pgfpoint{0.7cm}{0.3cm}}
			\pgfpathlineto{\pgfpoint{0.95cm}{-0.35cm}}
			%Ausgabe
 			\pgfpathmoveto{\pgfpoint{0.7cm}{-0.7cm}}
			\pgfpathlineto{\pgfpoint{0.7cm}{-0.3cm}}
			\pgfpathlineto{\pgfpoint{0.8cm}{-0.3cm}}

			\pgfusepath{stroke}

			\pgfsetdash{{2pt}{2pt}}{1pt} 
 			\pgfpathmoveto{\pgfpoint{-0.7cm}{0cm}}
			\pgfpathlineto{\pgfpoint{0.825cm}{0cm}}

			\pgfusepath{stroke}
			\pgfsetdash{}{0pt} 

			\pgfusepath{draw}

			%Kasten
			\pgfsetlinewidth{2\pgflinewidth}
			\pgfpathrectanglecorners{\pgfpoint{1cm}{0.5cm}}%
                                    {\pgfpoint{-1cm}{-0.5cm}}
			
			\pgfusepath{draw}

			\pgftext[base,y=-0.4cm]{AK}
	  }
}

%    \end{macrocode}
%
% Ende des Pakets \texttt{relaycircuit}
%\iffalse
%    \begin{macrocode}
%</relaycircuit.sty>
%    \end{macrocode}
%\fi
%\iffalse
%    \begin{macrocode}
%<*schule.sty>
%    \end{macrocode}
%\fi
% \subsection{Das Paket \texttt{schule}}
%	 Die ausführliche Beschreibung des Pakets ist in der
%	 Paketbeschreibung (\ref{paket:schule}) zu finden.
%
%  Beginn der Definition, Voraussetzung der \LaTeXe{} Version und die
%  eigene Identifizierung
%    \begin{macrocode}
\NeedsTeXFormat{LaTeX2e}[1995/12/01]
\ProvidesPackage{schule}[2015/05/07 v0.6 %
                         Kommandos fuer die Schule]
%    \end{macrocode}
% Einbinden der benötigten Pakete
%    \begin{macrocode}
\RequirePackage{etex} 
\RequirePackage[ngerman]{babel}
\RequirePackage{ifthen}
\RequirePackage{xifthen}
\RequirePackage{xspace}
\xspaceaddexceptions{\guillemotright,\guillemotleft}
\RequirePackage{tabularx}
\RequirePackage{rotating}
\RequirePackage{ragged2e}
\RequirePackage{amssymb}
\RequirePackage{amsmath}
\RequirePackage{graphicx}
\RequirePackage{tikz}
\RequirePackage{paralist}
\RequirePackage{textcomp}
\RequirePackage{xmpincl}
\RequirePackage{wrapfig}
\RequirePackage{eurosym}
\RequirePackage{multirow}
\RequirePackage{multicol}
\RequirePackage{ccicons}
\RequirePackage[autoload]{svn-multi}
\RequirePackage{cancel}
\usepackage{varwidth}  
%    \end{macrocode}
% 		Festlegung des Stils für Anführungszeichen
%			\begin{macrocode}
\RequirePackage[german=guillemets]{csquotes}
\DeclareOption{quotes}{\ExecuteQuoteOptions{german=quotes}}
%			\end{macrocode}
% 
%    Setzen der Klassenoptionen für das Anzeigen der Lösung
%    \begin{macrocode}
\newboolean{@loesunganzeigen}
\setboolean{@loesunganzeigen}{false}
\DeclareOption{loesung}{\setboolean{@loesunganzeigen}{true}}
%    \end{macrocode}
% 
%    Setzen der Klassenoptionen für das Anzeigen der Lösung auf einem
%    gesonderten Blatt und Durchnummerieren der Lösungen. Einmal mit
%    nur mit einem s, um Kompatibilität zu garantieren.
%    \begin{macrocode}
\newboolean{@loesunganzeigen@Seite}
\setboolean{@loesunganzeigen@Seite}{false}
\DeclareOption{loesungseite}%
{\setboolean{@loesunganzeigen@Seite}{true}\newcounter{@loesung@nr}}
\DeclareOption{loesungsseite}%
{\setboolean{@loesunganzeigen@Seite}{true}\newcounter{@loesung@nr}}
%    \end{macrocode}
% 
%    Setzen der Klassenoptionen für das Drehen des Stundenverlaufs und
%    das Einschalten der zusätzlichen Spalte 'didaktischer Kommentar`.
%    \begin{macrocode}
\newboolean{@stundenverlaufquer}
\setboolean{@stundenverlaufquer}{false}
\DeclareOption{stundenverlaufquer}%
{\setboolean{@stundenverlaufquer}{true}}
\newboolean{@stundenverlaufdidkom}
\setboolean{@stundenverlaufdidkom}{false}
\DeclareOption{stundenverlaufdidkom}%
{\setboolean{@stundenverlaufdidkom}{true}}
\newboolean{@stundenverlaufASF}
\setboolean{@stundenverlaufASF}{false}
\DeclareOption{stundenverlaufASF}%
{\setboolean{@stundenverlaufASF}{true}}

\ProcessOptions
\newcommand{\diastring}[1]{\textquotedbl#1\textquotedbl}
%    \end{macrocode}
%
% \subsubsection{Kurzbefehle}
% \begin{macro}{\SuS}
% 	\begin{macro}{\SuSn}
% 		Definition für die Kurzform von \SuS und \SuSn
%    \begin{macrocode}
\newcommand{\SuS}{Sch\-\"uler\-in\-nen und Sch\"u\-ler\xspace}
\newcommand{\SuSn}{Sch\-\"uler\-in\-nen und Sch\"u\-lern\xspace}
%    \end{macrocode}
% 	\end{macro}
% \end{macro}

% \begin{macro}{\cclizenz}
% 		Anzeigen der CC-Lizenz in der Fußzeile. Optional kann z.\,B. die
% 		Versionnummer angegeben werden.
%    \begin{macrocode}
\newcommand{\set@lizenz}[2]{
	\csname cc#2\endcsname \xspace #1
}
\newcommand{\cclizenz}[2][]{
	\ofoot[\set@lizenz{#1}{#2}]{\set@lizenz{#1}{#2}}
}
%    \end{macrocode}
% \end{macro}

% \begin{macro}{\footerQRLink}
% 		Zeigt links neben der Fußzeile den übergebenen QR-Code an und --
% 		falls vorhanden -- die Revisionsnummer mit Datum
%    \begin{macrocode}
\newcommand{\footer@QRLink}[1]{\ifthenelse{\equal{#1}{}}{}{%
	\begin{tikzpicture}[remember picture,overlay]
		\node [xshift=0.4cm,yshift=1.7cm, above right] at (current
		page.south west) 
		{\includegraphics[width=2cm]{#1}};
	\end{tikzpicture}
}%
\ifthenelse{\svnrev > 0}{
	\begin{scriptsize} Revision \svnrev 
		\ifsvnfilemodified{*}{} vom \svnday-\svnmonth-\svnyear 
	\end{scriptsize}}{}%
}
\newcommand{\footerQRLink}[1]{
\ifoot[\footer@QRLink{#1}]{\footer@QRLink{#1}}
}
%    \end{macrocode}
% \end{macro}
% 
% \begin{macro}{\loesung}
% Angabe einer Lösung, deren Ausgabe sich über die Paketoption regeln
% lässt
%    \begin{macrocode}
\newcommand{\loesung}[1]{
	\ifthenelse{\boolean{@loesunganzeigen}}	
		{\textbf{L\"osung:} #1}
		{}
	\ifthenelse{\boolean{@loesunganzeigen@Seite}}
		{\label{loesung@\the@loesung@nr}\global\expandafter\def\csname
		loesung@\the@loesung@nr\endcsname{#1}	
		\addtocounter{@loesung@nr}{1}}
		{}
}
%    \end{macrocode}
% Definition, um Zusammenarbeit mit LZK zu gewährleisten
%    \begin{macrocode}
\newcommand{\setzePunkte}{}
%    \end{macrocode}
% Im Bedarfsfall die Lösung am Ende des Dokuments ausgeben
%    \begin{macrocode}
\AtEndDocument{
	\ifthenelse{\boolean{@loesunganzeigen@Seite}}%
		{\newpage \setzePunkte%
		\let\save@thesection\thesection%
		\renewcommand{\thesection}{}%
		\section{L\"osungen:}% 
		\let\thesection\save@thesection%
		\newcounter{@zeige@nr}%
		\whiledo {\value{@zeige@nr} < \value{@loesung@nr}}%
			{\nameref{loesung@\the@zeige@nr}:\\ \expandafter\csname
			loesung@\the@zeige@nr\endcsname\par%
			\stepcounter {@zeige@nr}}}%
		{}
}
%    \end{macrocode}
% \end{macro}
%
% \begin{macro}{\so}
% Durchstreichen (\emph{strike out}) von Werten in
% Wertetabellen/Schreibtischtests
   \providecommand{\so}[1]{$\bcancel{#1}$}
%	\end{macro}
%
% \begin{macro}{\zeitpunkt}
% Mögliche Angabe einer Zeit in der \cmd{\stundenverlauf}-Umgebung,
% Abfrage ist etwas komplizierter, damit die Tabelle bei Nichtanzeige
% weiterhin richtig dargestellt wird.
%    \begin{macrocode}
\newcount\@pruefwert
\newcount\zeitanzeigen
\@pruefwert=1
\zeitanzeigen=0
\ifthenelse{\boolean{@stundenverlaufdidkom}}{
	\newcommand{\zeitpunkt}[1]{
	\ifnum \zeitanzeigen=\@pruefwert
		\multicolumn{5}{|c|}{#1} \\ \hline
	\fi}
}{
	\newcommand{\zeitpunkt}[1]{
	\ifnum \zeitanzeigen=\@pruefwert
		\multicolumn{4}{|c|}{#1} \\ \hline
	\fi}
}
%    \end{macrocode}
% \end{macro}
%
% \begin{macro}{\luecke}
% Erzeugt eine Lücke für den Lückentext
%    \begin{macrocode}
\newcommand{\luecke}[2][]{%
	\rule[0pt]{#2}{.5pt}%
	\ifthenelse{\boolean{@loesunganzeigen}\and\not\isempty{#1}}{
		\hspace{-#2}
		\hspace{-1em}
		\makebox[#2][c]{\textit{#1}}%
	}{}%
}
%    \end{macrocode}
% \end{macro}
% 
% \begin{macro}{\chb}
% Eine Box zum Ankreuzen
%    \begin{macrocode}
\newcommand{\chb}[1][]{%
	$\Box$%
	\ifthenelse{\equal{#1}{r}\and\boolean{@loesunganzeigen}}{%
  	\hspace{-0.77em}X%
	}{}%
	\xspace
}
%    \end{macrocode}
% \end{macro}
% 
% \begin{macro}{\name}
% Hervorhebung für Namen
%    \begin{macrocode}
\newcommand{\name}[1]{\textsc{#1}}
%    \end{macrocode}
% \end{macro}
%
% \begin{macro}{\keineSeitenzahlen}
% Eine Box zum Ankreuzen
%    \begin{macrocode}
\newcommand{\keineSeitenzahlen}{%
	\cfoot{}
}
%    \end{macrocode}
% \end{macro}
%
% \begin{macro}{\punkteitem}
% \begin{macro}{\punkteitemloesung}
% Erzeugt ein Item für eine Aufgabe, dem die Anzahl der möglichen
% Punkte (optional eine Lösung) übergeben wird
%    \begin{macrocode}
\newcommand{\punkteitem}[1]{%
\ifthenelse%
{\equal{#1}{1}}%
    {\item \textbf{(1 Punkt)}}%
    {\item \textbf{(#1 Punkte)}}%
}
\newcommand{\punkteitemloesung}[3][]{
\ifthenelse{\equal{#2}{1}}
	{\item \textbf{(1 Punkt)} #3}
	{\item \textbf{(#2 Punkte)} #3}
\ifthenelse{\equal{#1}{}}{}{
	\ifthenelse{\boolean{@loesunganzeigen}}
		{\\\textbf{L\"osung:} #1}{}
	\ifthenelse{\boolean{@loesunganzeigen@Seite}}
		{\label{loesung@\the@loesung@nr}\global\expandafter\def\csname
		loesung@\the@loesung@nr\endcsname{
			\textbf{\ref{loesung@\the@zeige@nr}. Aufgabe:} #1} 
		\addtocounter{@loesung@nr}{1}}{}
	}
}
%    \end{macrocode}
% \end{macro}
% \end{macro}
%
% \begin{macro}{\scaleTikz}
% 	Befehl, um TikZ-Graphiken mit Schrift zu skalieren.
%    \begin{macrocode}
\newcommand{\scaleTikz}[1]{
	\tikzstyle{every node}=[scale=#1]
	\tikzstyle{every path}=[scale=#1]
 	\tikzstyle{inststyle}+=[rounded corners= #1 * 3mm] %
 	% hier werden weitere Optionen angegeben
}
%    \end{macrocode}
% \end{macro}
%
% \subsubsection{Umgebungen}
% \begin{environment}{stundenverlauf}
% Definition des Stundenverlaufs: eine Tabelle mit entsprechendem Kopf
%    \begin{macrocode}
\newcommand{\didkom}[1]{
	\ifthenelse{\boolean{@stundenverlaufdidkom}}{& #1}{}
}
\newcommand{\Ptext}{Unterrichts\-phasen}
\newcommand{\Otext}{Operationen/Sachaspekte}
\newcommand{\Atext}{%
	\ifthenelse{\boolean{@stundenverlaufASF}}{%
		ASF
	}{%
		Aktions- und Sozialformen
	}
}
\newcommand{\Mtext}{Medien}
\newcommand{\Dtext}{Didaktischer Kommentar}

\newcommand{\setPtext}[1]{\newcommand{\Ptext}{#1}}
\newcommand{\setOtext}[1]{\renewcommand{\Otext}{#1}}
\newcommand{\setAtext}[2][ASF]{\renewcommand{\Atext}{%
	\ifthenelse{\boolean{@stundenverlaufASF}}{%
					#1
	}{%
					#2
	}
}
}
\newcommand{\setMtext}[1]{\renewcommand{\Mtext}{#1}}
\newcommand{\setDtext}[1]{\renewcommand{\Dtext}{#1}}
\newcommand{\stundenverlaufkopf}{%
	\hline
  \multicolumn{1}{|>{\raggedright\hspace{0pt}}P|}{%
		\textbf{\Ptext}
	} &
  \multicolumn{1}{>{\raggedright\hspace{0pt}}O|}{
		\textbf{\Otext}
	} &
	\multicolumn{1}{>{\raggedright\hspace{0pt}}A|}{%
		\textbf{\Atext}
	} &
	\textbf{\Mtext}
	\didkom{
		\multicolumn{1}{>{\raggedright\hspace{0pt}}D|}{
			\textbf{\Dtext}
		}
	}
}
\newcommand{\ASFfussText}{
	\footnotesize{\textbf {Abkürzungen:} Aktions- und
						Sozialformen (ASF), Einzelarbeit (EA), Partnerarbeit (PA),
						Gruppenarbeit (GA), Lehrervortrag (LV), Schülervortrag
						(SV), Unterrichtsgespräch (UG), Arbeitsblatt (AB),
		Schülerinnen und Schüler (SuS), Think-Pair-Share (T-P-S),
		Rollenspiel (RSP)}
}
\newcommand{\ASFfuss}{
	\ifthenelse{\boolean{@stundenverlaufASF}}{%
		\vspace{0.5cm}

		\ASFfussText
	}{}
}
\newcommand{\setASFfuss}[1]{
	\renewcommand{\ASFfussText}{
		\footnotesize{\textbf {Abkürzungen:} #1}
	}
}
	
\ifthenelse{\boolean{@stundenverlaufASF}}{
	\newcommand{\EA}{EA}
	\newcommand{\PA}{PA}
	\newcommand{\GA}{GA}
	\newcommand{\LV}{LV}
	\newcommand{\SV}{SV}
	\newcommand{\UG}{UG}
	\newcommand{\AB}{AB}
	\newcommand{\TPS}{T-P-S}
	\newcommand{\RSP}{RSP}
}{
	\newcommand{\EA}{Einzel\-arbeit\xspace}
	\newcommand{\PA}{Partner\-arbeit\xspace}
	\newcommand{\GA}{Gruppen\-arbeit\xspace}
	\newcommand{\LV}{Lehrer\-vortrag\xspace}
	\newcommand{\SV}{Schüler\-vortrag\xspace}
	\newcommand{\UG}{Unter\-richts\-gespräch\xspace}
	\newcommand{\AB}{Arbeits\-blatt\xspace}
	\newcommand{\TPS}{Think-Pair-Share\xspace}
	\newcommand{\RSP}{Rollen\-spiel\xspace}
}
\newboolean{@setP}
\newboolean{@setO}
\newboolean{@setA}
\newboolean{@setM}
\newboolean{@setD}
\setboolean{@setP}{false}
\setboolean{@setO}{false}
\setboolean{@setA}{false}
\setboolean{@setM}{false}
\setboolean{@setD}{false}
\newcommand{\setP}[1]{
  \ifthenelse{\boolean{@setP}}{}{
	  \newcolumntype{P}{#1}
	  \setboolean{@setP}{true}
  }
}
\newcommand{\setO}[1]{
  \ifthenelse{\boolean{@setO}}{}{
    \newcolumntype{O}{#1}
    \setboolean{@setO}{true}
  }
}
\newcommand{\setA}[1]{
  \ifthenelse{\boolean{@setA}}{}{
    \newcolumntype{A}{#1}
    \setboolean{@setA}{true}
  }
}
\newcommand{\setM}[1]{
  \ifthenelse{\boolean{@setM}}{}{
    \newcolumntype{M}{#1}
    \setboolean{@setM}{true}
  }
}
\newcommand{\setD}[1]{
  \ifthenelse{\boolean{@setD}}{}{
    \newcolumntype{D}{#1}
    \setboolean{@setD}{true}
  }
}
\newenvironment{stundenverlauf}
{
  \ifthenelse{\boolean{@stundenverlaufquer}}{%
		\ifthenelse{\boolean{@stundenverlaufdidkom}}{%
			\ifthenelse{\boolean{@stundenverlaufASF}}{%
			% mit didaktischer Kommentar, im Querformat, mit ASF
				\setP{p{2.35cm}}%
				\setO{p{8.5cm}}%
				\setA{p{1cm}}%
				\setM{p{1.5cm}}%
				\setD{p{7cm}}%
			}{ % mit didkatischer Kommentar, im Querformat, ohne ASF
				\setP{p{2.35cm}}%
				\setO{p{7.5cm}}%
				\setA{p{3cm}}%
				\setM{p{1.5cm}}%
				\setD{p{6cm}}%
			}
		}{ % ohne didaktischer Kommentar, im Querformat, mit ASF
			\ifthenelse{\boolean{@stundenverlaufASF}}{%
				\setP{p{3cm}}%
				\setO{p{13cm}}%
				\setA{p{1cm}}%
				\setM{p{3.35cm}}%
			}{ % ohne didaktischer Kommentar, im Querformat, ohne ASF
				\setP{p{2.35cm}}%
				\setO{p{11cm}}%
				\setA{p{3.5cm}}%
				\setM{p{3.35cm}}%
			}
		}%
	}{%
		\ifthenelse{\boolean{@stundenverlaufdidkom}}{%
			\ifthenelse{\boolean{@stundenverlaufASF}}{%
			% mit didaktischer Kommentar, ohne Querformat, mit ASF
				\setP{p{2.35cm}}%
				\setO{X}%
				\setA{p{1cm}}%
				\setM{p{1.5cm}}%
				\setD{p{4.5cm}}%
			}{ % mit didkatischer Kommentar, ohne Querformat, ohne ASF
				\setP{p{2.35cm}}%
				\setO{X}%
				\setA{p{1.7cm}}%
				\setM{p{1.6cm}}%
				\setD{p{3cm}}%
			}
		}{
			\ifthenelse{\boolean{@stundenverlaufASF}}{%
			% ohne didaktischer Kommentar, ohne Querformat, mit ASF
				\setP{p{2.35cm}}%
				\setO{X}%
				\setA{p{1cm}}%
				\setM{p{1.75cm}}%
			}{ % ohne didkatischer Kommentar, ohne Querformat, ohne ASF
				\setP{p{2.35cm}}%
				\setO{X}%
				\setA{p{1.7cm}}%
				\setM{p{1.45cm}}%
			}
		}
	}%
\ifthenelse{\boolean{@stundenverlaufquer}}{%
\begin{sidewaystable}
\ifthenelse{\boolean{@stundenverlaufdidkom}}{%
\begin{tabular}{|P|O|A|M|D|}
\stundenverlaufkopf \\ \hline
}{
\begin{tabular}{|P|O|A|M|}
\stundenverlaufkopf \\ \hline
}
}{
\ifthenelse{\boolean{@stundenverlaufdidkom}}{%
\RaggedLeft
\tabularx{\textwidth}{|P|O|A|M|D|}
\stundenverlaufkopf \\ \hline
}{
\tabularx{\textwidth}{|P|O|A|M|}
\stundenverlaufkopf \\ \hline
}
}
}
{%
\ifthenelse{\boolean{@stundenverlaufquer}}{%
\\\hline
\end{tabular}%
\ASFfuss
\end{sidewaystable}%
}{%
\ASFfuss
\endtabularx \justifying
}
}
%    \end{macrocode}
% \end{environment}
%
% \begin{environment}{aufgaben}
% Aufzählungsumgebung, Durchnummerierung mit folgendem Punkt und
% Aufgabe
%    \begin{macrocode}
\newenvironment{aufgaben}
	{\begin{enumerate}
	\renewcommand{\labelenumi}{\textbf{\arabic{enumi}. Aufgabe}}}
	{\end{enumerate}}
%    \end{macrocode}
% \end{environment}
%
% \begin{environment}{alphaEnum}
% Umgebung der ersten Ebene, die mit Buchstaben ausgezeichnet wird
%    \begin{macrocode}
\newenvironment{alphaEnum}
	{\begin{enumerate}
	\renewcommand{\labelenumi}{\textbf{\alph{enumi})}}}
	{\end{enumerate}}
%    \end{macrocode}
% \end{environment}
%
% \begin{environment}{smallitemize}
%  \begin{environment}{smallenumerate}
%   \begin{environment}{smalldescription}
%  Umgebungen mit verkleinertem vertikalen Abstand zwischen den
%  einzelnen Punkten 
%    \begin{macrocode}
\newenvironment{smallitemize}
	{\begin{itemize}\itemsep -2pt}{\end{itemize}}
\newenvironment{smallenumerate}
	{\begin{enumerate}\itemsep -2pt}{\end{enumerate}}
\newenvironment{smalldescription}
	{\begin{description}\itemsep -2pt}{\end{description}}
%    \end{macrocode}
%   \end{environment}
%  \end{environment}
% \end{environment}
%
% Ende des Pakets \texttt{schule}
%\iffalse
%    \begin{macrocode}
%</schule.sty>
%    \end{macrocode}
%\fi
%\iffalse
%    \begin{macrocode}
%<*schuleab.cls>
%    \end{macrocode}
%\fi
% \subsection{Die Klasse \texttt{schuleab}}
%	Die ausführliche Beschreibung der Klasse ist in der
%	Klassenbeschreibung (\ref{klasse:schuleab}) zu finden.
%
%  Beginn der Definition, Voraussetzung der \LaTeXe{} Version und die
%  eigene Identifizierung
%    \begin{macrocode}
\NeedsTeXFormat{LaTeX2e}[1995/12/01]
\ProvidesClass{schuleab}[2015/05/07 v0.6 %
                         Vorlage für ein Arbeitsblatt]
%    \end{macrocode}
% 
% Alle Optionen werden an die Klasse scrartcl weitergegeben.
%    \begin{macrocode}
\RequirePackage{ifthen}
\DeclareOption{loesung}{\PassOptionsToPackage{loesung}{schule}}
\DeclareOption{loesungseite}{
	\PassOptionsToPackage{loesungseite}{schule}
}
\DeclareOption{loesungsseite}{
	\PassOptionsToPackage{loesungsseite}{schule}
}
\newboolean{@kopfSuSName}
\setboolean{@kopfSuSName}{false}
\DeclareOption{kopfSuSName}{\setboolean{@kopfSuSName}{true}}
\newcommand{\kopfSuSName}{Name: \luecke{\@kopfSuSNameLaenge}}
\newboolean{@kopfDatum}
\setboolean{@kopfDatum}{false}
\DeclareOption{kopfDatum}{\setboolean{@kopfDatum}{true}}
\newcommand{\KopfDatum}{Datum: \luecke{\@kopfDatumLaenge}}
\DeclareOption{kopfDatumAktuell}{
				\setboolean{@kopfDatum}{true}
				\renewcommand{\KopfDatum}{Datum: \@kopfDatum}
}
\newboolean{@onesitepages}
\setboolean{@onesitepages}{false}
\DeclareOption{onesitepages}{\setboolean{@onesitepages}{true}}
\newboolean{@showlastpage}
\setboolean{@showlastpage}{false}
\DeclareOption{showlastpage}{\setboolean{@showlastpage}{true}}
\DeclareOption*{\PassOptionsToClass{\CurrentOption}{scrartcl}}
\ProcessOptions\relax
%    \end{macrocode}
%
%    Laden der Klasse und der nötigen Pakete und Setzen des
%    Seitenstils.
%    \begin{macrocode}
\LoadClass[parskip=half,DIV12]{scrartcl}
\RequirePackage[utf8]{inputenc}
\RequirePackage[T1]{fontenc}
\RequirePackage{schule}
\RequirePackage[headsepline]{scrpage2}
\pagestyle{scrheadings}
%    \end{macrocode}

% \subsubsection{Kopfbereich}
% \begin{macro}{\dokName}
% Definition zum Setzen des Namens des Dokuments
%    \begin{macrocode}
\def\dokName#1{\gdef\@dokName{#1}}
%    \end{macrocode}
% \end{macro}
% 
% \begin{macro}{\dokNummer}
% Definition zum Setzen der Nummer des Dokuments
%    \begin{macrocode}
\def\dokNummer#1{\gdef\@dokNummer{#1}}
%    \end{macrocode}
% \end{macro}
% 
% \begin{macro}{\jahrgang}
% Definition zum Setzen des Jahrgangs
%    \begin{macrocode}
\def\jahrgang#1{\gdef\@jahrgang{#1}}
%    \end{macrocode}
% \end{macro}
% 
% \begin{macro}{\fach}
% Definition zum Setzen des Fachs
%    \begin{macrocode}
\def\fach#1{\gdef\@fach{#1}}
%    \end{macrocode}
% \end{macro}
% Definitionen zum Formatieren der Kopfzeile
%    \begin{macrocode}
\def\kopfDatum#1{\gdef\@kopfDatum{#1}}
\def\kopfDatumLaenge#1{\gdef\@kopfDatumLaenge{#1}}
\kopfDatumLaenge{3cm}
\def\kopfSuSNameLaenge#1{\gdef\@kopfSuSNameLaenge{#1}}
\kopfSuSNameLaenge{5cm}
%    \end{macrocode}
% 
% 
% Setzen der Kopfzeile des Dokuments.
%    \begin{macrocode}
\ihead{%
	\ifthenelse{\boolean{@kopfSuSName}}{\kopfSuSName\\}{%
    \ifthenelse{\boolean{@kopfDatum}}{\\}{}%
  }%
  \ifthenelse{\isundefined{\@fach}}%
    {?? \@latex@warning@no@line{Das Fach ist nicht angegeben}}%
    {\@fach}%
  \ifthenelse{\isundefined{\@jahrgang}}%
    {}%
    { \@jahrgang}%
}
\chead{%
  \ifthenelse{\boolean{@kopfSuSName}\or\boolean{@kopfDatum}}{\\}{}%
  \ifthenelse{\isundefined{\@dokName}}%
  {?? \@latex@warning@no@line{Der Name des Dokuments ist nicht%
    angegeben}}%
  {\@dokName}%
}
\ohead{%
  \ifthenelse{\boolean{@kopfDatum}}{\KopfDatum\\}{%
  	\ifthenelse{\boolean{@kopfSuSName}}{\\}{}%
  }%
  Arbeitsblatt%
  \ifthenelse{\isundefined{\@dokNummer}}
  {}
  { Nr. \@dokNummer}
}
%    \end{macrocode}
% 
% 
% Setzen der Fußzeile des Dokuments.
%    \begin{macrocode}
\footerQRLink{}
%    \end{macrocode}
%
% Entfernen der Seitenzahl, sofern das Arbeitsblatt nur eine Seite
% hat. Optionale Einblendung der gesamten Seitenanzahl.
%    \begin{macrocode}
\cfoot{
		\ifthenelse{\boolean{@showlastpage}}{
			Seite \thepage\ von \pageref*{letzteseite}
		}{
			\thepage
		}	
}
\AtEndDocument{
	\label{letzteseite}
	\ifthenelse{\not\boolean{@onesitepages}\and\value{page}=1}{
		\cfoot{}
	}{}
}
%    \end{macrocode}
% 
% Ende der Klasse \texttt{schuleab}
%\iffalse
%    \begin{macrocode}
%</schuleab.cls>
%    \end{macrocode}
%\fi
%\iffalse
%    \begin{macrocode}
%<*schulein.cls>
%    \end{macrocode}
%\fi
% \subsection{Die Klasse \texttt{schulein}}
%	Die ausführliche Beschreibung der Klasse ist in der
%	Klassenbeschreibung (\ref{klasse:schulein}) zu finden.
%
%  Beginn der Definition, Voraussetzung der \LaTeXe{} Version und die
%  eigene Identifizierung
%    \begin{macrocode}
\NeedsTeXFormat{LaTeX2e}[1995/12/01]
\ProvidesClass{schulein}[2015/05/07 v0.6 %
                         Vorlage für ein Informationsblatt]
%    \end{macrocode}
% 
% Alle Optionen werden an die Klasse scrartcl weitergegeben.
%    \begin{macrocode}
\DeclareOption{loesung}{\PassOptionsToClass{loesung}{schuleab}}
\DeclareOption{loesungseite}%
    {\PassOptionsToClass{loesungseite}{schuleab}}
\DeclareOption{loesungsseite}%
    {\PassOptionsToClass{loesungsseite}{schuleab}}
\DeclareOption*{\PassOptionsToClass{\CurrentOption}{scrartcl}}
\ProcessOptions\relax
%    \end{macrocode}
%
%    Laden der Klasse
%    \begin{macrocode}
\LoadClass{schuleab}
%    \end{macrocode}

% \subsubsection{Kopfbereich}
% 
% Setzen der Kopfzeile des Dokuments.
%    \begin{macrocode}
\ohead{Informationsblatt%
\ifthenelse{\isundefined{\@dokNummer}}
		{}
		{ Nr. \@dokNummer}}
%    \end{macrocode}
% 
% Ende der Klasse \texttt{schulein}
%\iffalse
%    \begin{macrocode}
%</schulein.cls>
%    \end{macrocode}
%\fi
%\iffalse
%    \begin{macrocode}
%<*schuleit.cls>
%    \end{macrocode}
%\fi
% \subsection{Die Klasse \texttt{schuleit}}
%	Die ausführliche Beschreibung der Klasse ist in der
%	Klassenbeschreibung (\ref{klasse:schuleit}) zu finden.
%
%  Beginn der Definition, Voraussetzung der \LaTeXe{} Version und die
%  eigene Identifizierung
%    \begin{macrocode}
\NeedsTeXFormat{LaTeX2e}[1995/12/01]
\ProvidesClass{schuleit}[2015/05/07 v0.6 %
                         Vorlage für ein Leitprogramm]
%    \end{macrocode}
% 
% Alle Optionen werden an die Klasse scrreprt weitergegeben.
%    \begin{macrocode}
\DeclareOption*{\PassOptionsToClass{\CurrentOption}{scrreprt}}
\ProcessOptions\relax
%    \end{macrocode}
%
%    Laden der Klasse
%    \begin{macrocode}
\LoadClass[12pt,a4paper,openany,
						chapterprefix,
						bibliography=totoc,
						numbers=noendperiod,
						parskip=half]
	{scrreprt}
%    \end{macrocode}
%
%    Laden der Pakete 
%    \begin{macrocode}
\RequirePackage[utf8]{inputenc}
\RequirePackage[T1]{fontenc}

% Automatische Skalierung zu grosser (breiter) Grafiken ==> ggf. nach
% schule ?
%\RequirePackage[Export]{adjustbox}
%\adjustboxset{max size={\textwidth}{0.9\textheight}}

\RequirePackage{schule}
\RequirePackage{mdframed}
\RequirePackage{scrpage2}
\RequirePackage{paralist}
\RequirePackage{xargs}
\RequirePackage{xparse}
%    \end{macrocode}
%
%    Setzen der Fuß- und Kopfzeilen
%    \begin{macrocode}
\pagestyle{scrheadings}
\clearscrheadfoot
\cfoot[\pagemark]{\pagemark}

\renewcommand{\chaptermark}[1]{ \markboth{#1}{} }
\renewcommand{\sectionmark}[1]{ \markright{#1}{} }
\ihead{ {\normalfont\leftmark\ --} \textit{\rightmark} }
%    \end{macrocode}
%
%    Definieren von Farben für Kapitel etc
%    \begin{macrocode}
\definecolor{chapter}{rgb}{0,0.25,0.56}
\definecolor{section}{rgb}{0.27,0.33,0.90}
\definecolor{subsection}{rgb}{0.54,0.66,0.90}
\definecolor{subsubsection}{rgb}{0.14,0.17,0.95}
\definecolor{paragraph}{cmyk}{0.5,0,.1,.39} 

\definecolor{LightGrey}{rgb}{0.9,0.9,0.9}

\definecolor{grey1}{rgb}{.1,.1,.1}
\definecolor{grey2}{rgb}{.2,.2,.2}
\definecolor{grey3}{rgb}{.3,.3,.3}
\definecolor{grey4}{rgb}{.4,.4,.4}
\definecolor{grey5}{rgb}{.5,.5,.5}
\definecolor{grey7}{rgb}{.7,.7,.7}
\definecolor{grey8}{rgb}{.8,.8,.8}
\definecolor{grey9}{rgb}{.9,.9,.9}

\newcommand{\uebungBild}{
\begin{tikzpicture}[y=0.80pt, x=0.8pt,yscale=-1, inner sep=0pt, outer
				sep=0pt] \path[draw=grey5,fill=grey7,line join=miter,line
				cap=butt,line width=0.209pt] (12.0208,0.5895) --
				(1.3666,3.8375) -- (3.3843,20.6431) -- (20.4113,14.4179) --
				cycle; \path[draw=grey1,fill=grey9,line join=round,line
				cap=butt,line width=0.209pt] (7.1932,6.1991) --
				(15.9622,12.7759) -- (16.8321,11.7667) -- (7.8544,5.3292) --
				cycle; \path[draw=grey1,fill=grey3,line join=round,line
				cap=butt,line width=0.209pt] (7.8544,5.3292) --
				(5.9231,4.7550) -- (7.1932,6.1991) -- cycle;
\end{tikzpicture}	
}

\newcommand{\hinweisBild}{
\begin{tikzpicture}[y=0.80pt, x=0.8pt,yscale=-1, inner sep=0pt, outer
				sep=0pt] \path[draw=black,fill=grey2,line join=miter,line
				cap=butt,miter limit=4.00,line width=0.160pt] (6.4521,1.4334)
				-- (7.4295,0.4560) .. controls (7.9473,0.7196) and
				(8.1724,0.8553) .. (9.3447,1.4053) .. controls (9.3447,1.4053)
				and (9.4550,2.1225) .. (9.7278,3.7868) .. controls
				(9.7547,3.9511) and (10.4938,7.4506) .. (10.4938,7.4506) ..
				controls (10.4938,7.4506) and (11.0227,10.4067) ..
				(11.5264,11.8139) .. controls (11.7222,12.3609) and
				(12.1912,12.8073) .. (12.2925,13.3794) .. controls
				(12.3294,13.5882) and (12.2747,13.8058) .. (12.2258,14.0122)
				.. controls (12.1526,14.3214) and (12.0628,14.5394) ..
				(11.8899,14.9041) .. controls (11.5477,15.6258) and
				(10.6101,16.3538) .. (10.6271,16.2438);
				\path[draw=black,fill=grey8,line join=round,line cap=butt,line
				width=0.200pt] (5.8974,0.5893) -- (7.4295,0.4519) .. controls
				(8.0049,6.9583) and (10.9185,15.1671) .. (10.8935,15.6401) ..
				controls (10.8820,15.8585) and (10.8157,16.1331) ..
				(10.6271,16.2438) .. controls (10.4156,16.3680) and
				(10.0239,16.3854) .. (9.8943,16.1772) .. controls
				(8.7782,14.3840) and (8.5425,12.9279) .. (8.0291,11.2477) ..
				controls (7.4491,9.3496) and (7.0458,7.3997) ..
				(6.6635,5.4522) .. controls (6.3474,3.8419) and
				(5.8974,0.5893) .. (5.8974,0.5893) -- cycle;
				\path[draw=black,fill=black,line join=round,line
				cap=butt,miter limit=4.00,line width=0.160pt]
				(13.0918,15.9107) -- (13.6664,17.7926) -- (13.0575,20.6334) --
				(11.7179,19.3081) -- (13.0918,15.9107);
				\path[draw=black,fill=grey4,line join=round,line
				cap=butt,miter limit=4.00,line width=0.160pt]
				(13.0918,15.9107) -- (12.1925,18.0424) -- (10.8935,18.0424) --
				(12.3258,15.7775) -- cycle; \path[draw=black,fill=grey8,line
				join=round,line cap=butt,miter limit=4.00,line width=0.160pt]
				(10.8935,18.0424) .. controls (10.8935,18.0424) and
				(11.1714,19.7148) .. (11.1600,19.6412) .. controls
				(11.0947,19.2217) and (11.1324,20.0124) .. (11.7262,20.8070)
				.. controls (11.9101,21.0529) and (12.7921,20.8070) ..
				(12.9586,20.7403) .. controls (13.1251,20.6737) and
				(12.9586,19.3414) .. (12.9586,19.3414) -- (12.1925,18.0424) --
				cycle;
\end{tikzpicture}
}

\addtokomafont{chapter}{\color{chapter}}
\addtokomafont{section}{\color{section}}
\addtokomafont{subsection}{\color{subsection}}
\addtokomafont{subsubsection}{\color{subsubsection}}
\addtokomafont{paragraph}{\color{paragraph}}

\newmdenv[backgroundcolor=LightGrey,linewidth=0pt]{grey@Frame}

\makeatletter% siehe FAQ (aber wirklich nachsehen!)
\newcommand*{\headingpar}{\par\nobreak\@afterheading}
\makeatother% siehe FAQ

\newenvironment{greyFrame}[2]%
{\begin{grey@Frame}#1
 \raisebox{+0.9ex}{
 	\begin{large}#2\end{large}
 }\vspace*{-0.2cm}\headingpar}%
{\end{grey@Frame}}

\newcounter{aufgabe}[chapter]
\newenvironment{Aufgabe}
{\stepcounter{aufgabe}\label{aufg\arabic{chapter}\arabic{aufgabe}}
\begin{greyFrame}{\uebungBild{}}{Aufgabe %
\arabic{chapter}.\arabic{aufgabe}}}%
	{\end{greyFrame}}%

%\newcounter{aufgabennr}[chapter]
\NewDocumentEnvironment{Aufgaben}{o o}%
{%
 \stepcounter{aufgabe}\label{aufg\arabic{chapter}\arabic{aufgabe}}%
 %\setcounter{aufgabennr}{1}				
 \begin{greyFrame}{\uebungBild{}}{Aufgaben %
				 \arabic{chapter}.\arabic{aufgabe}}%
 \headingpar\begin{Form}\headingpar%
 \IfNoValueTF{#2}{}{#2}%
 \ifthenelse{\equal{#1}{}}{%
 	\begin{compactenum}[a)]%
 }{%
	\IfNoValueTF{#1}{%
	 \begin{compactenum}[a)]%
	}{%
	 \begin{compactenum}[#1]%
	}%
 }%
	 %\setcounter{enumi}{\theaufgabennr}
	 %\renewcommand{\labelenumi}{\arabic{chapter}.\arabic{aufgabe}.\roman{enumi}}
}%
{%
	%\setcounter{aufgabennr}{\theenumi}
	\end{compactenum}%
	\end{Form}%
	\end{greyFrame}%
}%

\newenvironment{Hinweis}
	{\begin{greyFrame}{\hinweisBild{}}{Hinweis}}%
	{\end{greyFrame}}%


\newcommand{\TextFeld}[1]{%
				\vspace*{3pt}\newline\TextField[width=.93\textwidth,%
				height=#1,multiline=true,borderwidth=0]{}%
}%

\newcounter{loesungnr}
\newcommand{\AufgabeLoesung}[2][]{%
  \label{loesungback\theloesungnr}%
  \global\expandafter\edef\csname%
    loesungname\theloesungnr\endcsname{%
      \arabic{chapter}.\arabic{aufgabe}%
    }%
	\ifthenelse{\isempty{#1}}{%
		\global\expandafter\edef\csname%
		  loesungref\theloesungnr\endcsname{__NONE__}%
	}{%
		\global\expandafter\edef\csname%
	  	loesungref\theloesungnr\endcsname{loesungback\theloesungnr}%
	}%
  \global\expandafter\def\csname loesung\theloesungnr\endcsname{#2}%
  \hyperref[loesung\theloesungnr]{%
    \colorbox{black!30}{\color{blue!90}L}%
  }%
  \addtocounter{loesungnr}{1}%
}

\newcommand{\AufgabenLoesung}[1]{%
				\AufgabeLoesung[ref]{#1}%
}

\newcounter{zeigenr}%
\newcommand{\loesungzeigen}{%
\setcounter{zeigenr}{0}%
\ifthenelse{\value{loesungnr}>0}{%
% temporärer fix, um \chb	anzeigen zu können
\setboolean{@loesunganzeigen}{true} 
\begin{description}%
\whiledo {\value{zeigenr} < \value{loesungnr}}{%
\item[\expandafter\csname loesungname\thezeigenr\endcsname%
			\ifthenelse{%
			  \equal{\expandafter\csname loesungref\thezeigenr\endcsname}%
        {__NONE__}%
      }{}{\,\ref{\expandafter\csname loesungref\thezeigenr\endcsname}}%
		 ]%
\label{loesung\thezeigenr}%
\expandafter\csname loesung\thezeigenr\endcsname%
\xspace\hyperref[loesungback\thezeigenr]{%
\colorbox{black!30}{\color{blue!90}Zurück}%
}%
\stepcounter {zeigenr}%
}%
\end{description}%
% temporärer fix, um \chb	anzeigen zu können
\setboolean{@loesunganzeigen}{false} 
}{}%
}%

\newcounter{hinweisnr}
\newcommand{\AufgabeHinweis}[2][]{%
	\label{hinweisback\thehinweisnr}%
	\ifthenelse{\isempty{#1}}{%
		\global\expandafter\edef\csname%
		  hinweisref\thehinweisnr\endcsname{__NONE__}%
	}{%
		\global\expandafter\edef\csname%
	  	hinweisref\thehinweisnr\endcsname{hinweisback\thehinweisnr}%
	}%
	\global\expandafter\edef\csname%
	hinweisname\thehinweisnr\endcsname{\arabic{chapter}.\arabic{aufgabe}}%
	\global\expandafter\def\csname hinweis\thehinweisnr\endcsname{#2}%
	\hyperref[hinweis\thehinweisnr]{%
		\colorbox{black!30}{\color{blue!90}H}%
	}%
	\addtocounter{hinweisnr}{1}%
}%
\newcommand{\AufgabenHinweis}[1]{%
	\AufgabeHinweis[ref]{#1}%
}%

\newcommand{\hinweiszeigen}{%
\setcounter{zeigenr}{0}%
\ifthenelse{\value{hinweisnr} > 0}{%
\begin{description}%
	\whiledo {\value{zeigenr} < \value{hinweisnr}}{%
		\item[\expandafter\csname hinweisname\thezeigenr\endcsname%
			\ifthenelse{%
			  \equal{\expandafter\csname hinweisref\thezeigenr\endcsname}%
        {__NONE__}%
      }{}{\,\ref{\expandafter\csname hinweisref\thezeigenr\endcsname}}%
		 ]%
		\label{hinweis\thezeigenr}%
		\expandafter\csname%
		hinweis\thezeigenr\endcsname%
		\xspace\hyperref[hinweisback\thezeigenr]{%
			\colorbox{black!30}{\color{blue!90}Zurück}%
		}%
		\stepcounter {zeigenr}%
	}%
\end{description}%
}{}%
}%
%    \end{macrocode}
% 
% Ende der Klasse \texttt{schuleit}
%\iffalse
%    \begin{macrocode}
%</schuleit.cls>
%    \end{macrocode}
%\fi
%\iffalse
%    \begin{macrocode}
%<*schulekl.cls>
%    \end{macrocode}
%\fi
% \subsection{Die Klasse \texttt{schulekl}}
%	Die ausführliche Beschreibung der Klasse ist in der
%	Klassenbeschreibung (\ref{klasse:schulekl}) zu finden.
%
%  Beginn der Definition, Voraussetzung der \LaTeXe{} Version und die
%  eigene Identifizierung
%    \begin{macrocode}
\NeedsTeXFormat{LaTeX2e}[1995/12/01]
\ProvidesClass{schulekl}[2015/05/07 v0.6 %
               Vorlage für eine Klausur]
%    \end{macrocode}
% 
%    Laden der Klasse und der nötigen Pakete und Setzen des
%    Seitenstils.
%    \begin{macrocode}
\LoadClass{schullzk}
\RequirePackage{schullzk}
\RequirePackage[headsepline]{scrpage2}
\pagestyle{scrheadings}
%    \end{macrocode}
% 
%    Setzen der Klassenoptionen, dass es sich um eine Klassen- oder
%    Kursarbeit handelt.  \changes{v1.1}{2010/03/15}{Klassenarbeit als
%    Option hinzugefügt}
%    \begin{macrocode}
\newboolean{@klassenarbeit}
\setboolean{@klassenarbeit}{false}
\DeclareOption{arbeit}{\setboolean{@klassenarbeit}{true}}
\newboolean{@kursarbeit}
\setboolean{@kursarbeit}{false}
\DeclareOption{kursarbeit}{\setboolean{@kursarbeit}{true}}
\newboolean{@kmkpunkte}
\setboolean{@kmkpunkte}{false}
\DeclareOption{KMKpunkte}{\setboolean{@kmkpunkte}{true}}
\ProcessOptions
\RequirePackage{schulekl}
%    \end{macrocode}
% 
% Alle anderen Optionen werden an die Klasse scrartcl weitergegeben.
%    \begin{macrocode}
\DeclareOption*{\PassOptionsToClass{\CurrentOption}{scrartcl}}
\ProcessOptions\relax
%    \end{macrocode}
%
% \subsubsection{Kopfbereich}
% \begin{macro}{\klausurname}
% Definition zum Setzen des Namens der Klausur
%    \begin{macrocode}
\def\klausurname#1{\gdef\@klausurname{#1}}
%    \end{macrocode}
% \end{macro}
%
% Setzen der Kopfzeile des Dokuments.
%    \begin{macrocode}
\ihead{\ifthenelse{\boolean{@klassenarbeit}}%
		{Klassenarbeit: }%
		{\ifthenelse{\boolean{@kursarbeit}}
			{Kursarbeit: }
			{Klausur: } }%
	\ifthenelse{\isundefined{\@klausurname}}
		{?? \@latex@warning@no@line{Klausurname ist nicht angegeben}}
		{\@klausurname}
}
\ohead{Name: \hspace{5cm}}
%    \end{macrocode}
%
% 
% Ende der Klasse \texttt{schulekl}
%\iffalse
%    \begin{macrocode}
%</schulekl.cls>
%    \end{macrocode}
%\fi
%\iffalse
%    \begin{macrocode}
%<*schulekl.sty>
%    \end{macrocode}
%\fi
%	\subsection{Das Paket \texttt{schulekl}}
%		Die ausführliche Beschreibung des Pakets ist in der entsprechenden
%		Klassenbeschreibung (\ref{klasse:schulekl}) zu finden.
%
%  Beginn der Definition, Voraussetzung der \LaTeXe{} Version und die
%  eigene Identifizierung
%		 \begin{macrocode}
\NeedsTeXFormat{LaTeX2e}[1995/12/01]
\ProvidesPackage{schulekl}[2015/05/07 v0.6 %
Kommandos fuer das Setzen einer Klausur/Kursarbeit]
%		 \end{macrocode}
% Möglichkeit, um das Ergebnis der Klausur anzugeben
%    \begin{macrocode}
\ifthenelse{\boolean{@kmkpunkte}}{
\def\@klausurergebnisangabe#1#2#3#4#5#6#7#8#9{%
\newcounter{@fuenfzehn}
\ifthenelse{\equal{#1}{}}
{\setcounter{@fuenfzehn}{0}}
{\setcounter{@fuenfzehn}{#1}}
\newcounter{@vierzehn}
\ifthenelse{\equal{#2}{}}
{\setcounter{@vierzehn}{0}}
{\setcounter{@vierzehn}{#2}}
\newcounter{@dreizehn}
\ifthenelse{\equal{#3}{}}
{\setcounter{@dreizehn}{0}}
{\setcounter{@dreizehn}{#3}}
\newcounter{@zwoelf}
\ifthenelse{\equal{#4}{}}
{\setcounter{@zwoelf}{0}}
{\setcounter{@zwoelf}{#4}}
\newcounter{@elf}
\ifthenelse{\equal{#5}{}}
{\setcounter{@elf}{0}}
{\setcounter{@elf}{#5}}
\newcounter{@zehn}
\ifthenelse{\equal{#6}{}}
{\setcounter{@zehn}{0}}
{\setcounter{@zehn}{#6}}
\newcounter{@neun}
\ifthenelse{\equal{#7}{}}
{\setcounter{@neun}{0}}
{\setcounter{@neun}{#7}}
\newcounter{@acht}
\ifthenelse{\equal{#8}{}}
{\setcounter{@acht}{0}}
{\setcounter{@acht}{#8}}
\newcounter{@sieben}
\ifthenelse{\equal{#9}{}}
{\setcounter{@sieben}{0}}
{\setcounter{@sieben}{#9}}
\@klausurerweiterung
}
\def\@klausurerweiterung#1#2#3#4#5#6#7{
\newcounter{@sechs}
\ifthenelse{\equal{#1}{}}
{\setcounter{@sechs}{0}}
{\setcounter{@sechs}{#1}}
\newcounter{@fuenf}
\ifthenelse{\equal{#2}{}}
{\setcounter{@fuenf}{0}}
{\setcounter{@fuenf}{#2}}
\newcounter{@vier}
\ifthenelse{\equal{#3}{}}
{\setcounter{@vier}{0}}
{\setcounter{@vier}{#3}}
\newcounter{@drei}
\ifthenelse{\equal{#4}{}}
{\setcounter{@drei}{0}}
{\setcounter{@drei}{#4}}
\newcounter{@zwei}
\ifthenelse{\equal{#5}{}}
{\setcounter{@zwei}{0}}
{\setcounter{@zwei}{#5}}
\newcounter{@eins}
\ifthenelse{\equal{#6}{}}
{\setcounter{@eins}{0}}
{\setcounter{@eins}{#6}}
\newcounter{@null}
\ifthenelse{\equal{#7}{}}
{\setcounter{@null}{0}}
{\setcounter{@null}{#7}}

\newcounter{@gesamt}
\pgfmathsetcounter{@gesamt}{\the@fuenfzehn + \the@vierzehn + %
\the@dreizehn + \the@zwoelf + \the@elf + \the@zehn + \the@neun + %
\the@acht + \the@sieben + \the@sechs + \the@fuenf + \the@vier + %
\the@drei + \the@zwei + \the@eins + \the@null}

\newcounter{@schnitt}
\newcounter{@schnittVorne}
\pgfmathsetcounter{@schnitt}{round((\the@fuenfzehn*15 + %
\the@vierzehn*14 + \the@dreizehn*13 + \the@zwoelf*12 + %
\the@elf*11 + \the@zehn*10 + \the@neun*9 + \the@acht*8 + %
\the@sieben*7 + \the@sechs*6 + \the@fuenf*5 + \the@vier*4 + %
\the@drei*3 + \the@zwei*2 + \the@eins*1) / \the@gesamt *100)}
\pgfmathsetcounter{@schnittVorne}{\the@schnitt / 100}
\pgfmathsetcounter{@schnitt}{\the@schnitt - (\the@schnittVorne * 100)}

\minisec{Ergebnis}
\begin{minipage}{4cm}
\begin{tabular}{rr}
 \textbf{Punkte} & \textbf{Anzahl}\\
\hline
 15 & \the@fuenfzehn\\
 14 & \the@vierzehn\\
 13 & \the@dreizehn\\
 12 & \the@zwoelf\\
 11 & \the@elf\\
 10 & \the@zehn\\
  9 & \the@neun\\
  8 & \the@acht\\
  7 & \the@sieben\\
  6 & \the@sechs\\
  5 & \the@fuenf\\
  4 & \the@vier\\
  3 & \the@drei\\
  2 & \the@zwei\\
  1 & \the@eins\\
  0 & \the@null\\
\end{tabular}
\end{minipage}
\begin{minipage}{3cm}
\begin{tabular}{ll}
gesamt: & \the@gesamt \\
Schnitt: & $\the@schnittVorne , \the@schnitt$\\
\end{tabular}
\end{minipage}
}
\def\klausurergebnisangabe#1#2#3#4#5#6#7#8#9{
	\def\@klausurI{#1}
	\def\@klausurII{#2}
	\def\@klausurIII{#3}
	\def\@klausurIV{#4}
	\def\@klausurV{#5}
	\def\@klausurVI{#6}
	\def\@klausurVII{#7}
	\def\@klausurVIII{#8}
	\def\@klausurIX{#9}
	\@klausurtmp
}
\def\@klausurtmp#1#2#3#4#5#6#7{
	\gdef\@klausur@ergebnis{%
	\@klausurergebnisangabe{\@klausurI}{\@klausurII}{\@klausurIII}
		{\@klausurIV}{\@klausurV}{\@klausurVI}{\@klausurVII}
		{\@klausurVIII}{\@klausurIX}{#1}{#2}{#3}{#4}{#5}{#6}{#7}%
}
}
}{
\newcommand{\@klausurergebnisangabe}[6]{
\newcounter{@sehrgut}
\ifthenelse{\equal{#1}{}}
{\setcounter{@sehrgut}{0}}
{\setcounter{@sehrgut}{#1}}

\newcounter{@gut}
\ifthenelse{\equal{#2}{}}
{\setcounter{@gut}{0}}
{\setcounter{@gut}{#2}}

\newcounter{@befriedigend}
\ifthenelse{\equal{#3}{}}
{\setcounter{@befriedigend}{0}}
{\setcounter{@befriedigend}{#3}}

\newcounter{@ausreichend}
\ifthenelse{\equal{#4}{}}
{\setcounter{@ausreichend}{0}}
{\setcounter{@ausreichend}{#4}}

\newcounter{@mangelhaft}
\ifthenelse{\equal{#5}{}}
{\setcounter{@mangelhaft}{0}}
{\setcounter{@mangelhaft}{#5}}

\newcounter{@ungenuegend}
\ifthenelse{\equal{#6}{}}
{\setcounter{@ungenuegend}{0}}
{\setcounter{@ungenuegend}{#6}}

\newcounter{@gesamt}
\pgfmathsetcounter{@gesamt}{\the@sehrgut + \the@gut + %
    \the@befriedigend + \the@ausreichend + \the@mangelhaft + %
    \the@ungenuegend}

\newcounter{@schnitt}
\newcounter{@schnittVorne}
\pgfmathsetcounter{@schnitt}{round((\the@sehrgut + \the@gut *2 + %
    \the@befriedigend *3 + \the@ausreichend *4 + %
    \the@mangelhaft *5+ \the@ungenuegend*6) / \the@gesamt *100)}
\pgfmathsetcounter{@schnittVorne}{\the@schnitt / 100}
\pgfmathsetcounter{@schnitt}{\the@schnitt - (\the@schnittVorne * 100)}

\minisec{Ergebnis}
\begin{minipage}{4cm}
\begin{tabular}{ll}
sehr gut & \the@sehrgut \\
gut & \the@gut\\
befriedigend & \the@befriedigend\\
ausreichend & \the@ausreichend\\
mangelhaft & \the@mangelhaft\\
ungenügend & \the@ungenuegend\\
\end{tabular}
\end{minipage}
\begin{minipage}{3cm}
\begin{tabular}{ll}
gesamt: & \the@gesamt \\
Schnitt: & $\the@schnittVorne , \the@schnitt$\\
\end{tabular}
\end{minipage}
}
\def\klausurergebnisangabe#1#2#3#4#5#6{\gdef\@klausur@ergebnis{%
        \@klausurergebnisangabe{#1}{#2}{#3}{#4}{#5}{#6}}}
}
%    \end{macrocode}
% 
% Setzen des möglichen Ergebnisses am Ende der Klausur
%    \begin{macrocode}
\AtEndDocument{
 \ifthenelse{\isundefined{\@klausur@ergebnis}}{}{\@klausur@ergebnis}
}
%    \end{macrocode}
%
% Ende des Pakets \texttt{schulekl}
%\iffalse
%    \begin{macrocode}
%</schulekl.sty>
%    \end{macrocode}
%\fi
%\iffalse
%    \begin{macrocode}
%<*schuleub.cls>
%    \end{macrocode}
%\fi
% \subsection{Die Klasse \texttt{schuleub}}
%	Die ausführliche Beschreibung der Klasse ist in der
%	Klassenbeschreibung (\ref{klasse:schuleub}) zu finden.
%
%  Beginn der Definition, Voraussetzung der \LaTeXe{} Version und die
%  eigene Identifizierung
%    \begin{macrocode}
\NeedsTeXFormat{LaTeX2e}[1995/12/01]
\ProvidesClass{schuleub}[2015/05/07 v0.6 %
                         Vorlage für einen Unterrichtsbesuch]
%    \end{macrocode}
%
%    Laden der Klasse und der nötigen Pakete und Setzen des
%    Seitenstils.
%    \begin{macrocode}
\LoadClass[parskip=half,headsepline,DIV14]{scrartcl}
\RequirePackage[utf8]{inputenc}
\RequirePackage[T1]{fontenc}
\RequirePackage[headsepline]{scrpage2}
\setkomafont{pagehead}{\normalfont}
\RequirePackage{calc}
\RequirePackage{hyperref}
\RequirePackage[]{adjustbox}
\RequirePackage{pdfpages}
\pagestyle{scrheadings}
%    \end{macrocode}
%
%    Setzen der Klassen Optionen für das Examen (auslaufende PO) bzw.
%    die Schriftliche Arbeit (aktuelle PO) und die Revision
%    \begin{macrocode}
\newboolean{B@examen}
\setboolean{B@examen}{false}
\DeclareOption{examen}{\setboolean{B@examen}{true}}

\newboolean{B@neuePO}
\setboolean{B@neuePO}{false}
\DeclareOption{neuePO}{\setboolean{B@neuePO}{true}}

\newboolean{B@reversion}
\setboolean{B@reversion}{false}
\DeclareOption{reversion}{\setboolean{B@reversion}{true}}
\newboolean{B@kurzentwurf}
\setboolean{B@kurzentwurf}{false}
\DeclareOption{kurzentwurf}{\setboolean{B@kurzentwurf}{true}}

\newboolean{B@zieleMulti}
\setboolean{B@zieleMulti}{false}
\DeclareOption{zieleMulti}{\setboolean{B@zieleMulti}{true}}

\newboolean{B@bibBibtex}
\newboolean{B@bibBiblatexBibtex}
\setboolean{B@bibBibtex}{false}
\setboolean{B@bibBiblatexBibtex}{false}
\DeclareOption{bibBibtex}{\setboolean{B@bibBibtex}{true}}
\DeclareOption{bibBiblatexBibtex}{
	\setboolean{B@bibBiblatexBibtex}{true}
}
%    \end{macrocode}
% 
% Alle weiteren Optionen werden an die Klasse scrartcl weitergegeben.
%    \begin{macrocode}
\DeclareOption*{\PassOptionsToClass{\CurrentOption}{scrartcl}}
\ProcessOptions\relax
\ifthenelse{\boolean{B@kurzentwurf}}{
	\RequirePackage[bottom=1cm,top=1cm,left=1.5cm, right=2cm,
	a4paper,landscape, includehead, includefoot]{geometry}
}{}
%    \end{macrocode}
%
%    Einstellungen für die Bibliotheken, die genutzt werden
%    \begin{macrocode}
\ifthenelse{\boolean{B@bibBibtex}}{
  \RequirePackage{natbib}
  \bibpunct{[}{]}{}{a}{}{,~} 
  \bibliographystyle{dinat}
}{
	\ifthenelse{\boolean{B@bibBiblatexBibtex}}
   {\RequirePackage[backend=bibtex]{biblatex}}
   {\RequirePackage[backend=biber]{biblatex}}
}
%    \end{macrocode}
%
% \begin{macro}{\thema}
%  \begin{macro}{\Thema}
%  Definition des Themas und der Reihe (aktuelle PO) und die
%  Möglichkeit, diese auch zu nutzen
%    \begin{macrocode}
\def\thema#1{\gdef\@thema{#1}}
\newcommand{\Thema}{\@thema}
\def\reihe#1{\gdef\@reihe{#1}}
\newcommand{\Reihe}{\@reihe}
%    \end{macrocode}
%  \end{macro}
% \end{macro}
%
% \begin{macro}{\referendar}
% \begin{macro}{\seminaradresse}
% \begin{macro}{\ort}
% \begin{macro}{\besuchtitel}
% \begin{macro}{\lerngruppe}
% \begin{macro}{\datum}
% \begin{macro}{\zeit}
% \begin{macro}{\stunde}
% \begin{macro}{\schule}
% \begin{macro}{\raum}
% Setzen der Angaben zur Adresse des Seminars (ZfsL), des
% Seminarzusatzes (aktuelle PO), des Orts, des Besuchstitels, der
% Lerngruppe, des Datums, der Zeit, der Stunde, der Schule und des
% Raums.
%    \begin{macrocode}
\def\seminaradresse#1{\gdef\@seminaradresse{#1}}
\def\seminarinfo#1{\gdef\@seminarinfo{#1}}
\def\ort#1{\gdef\@ort{#1}}
\def\besuchtitel#1{\gdef\@besuchtitel{#1}}
\newcommand{\lerngruppe}[2][]{
	\gdef\@lerngruppe{#2}
	\ifthenelse{\equal{\unexpanded{#1}}{}}{\gdef\@lerngruppeKurz{#2}}
	{\gdef\@lerngruppeKurz{#1}}
}
\def\datum#1{\gdef\@datum{#1}}
\def\zeit#1#2{\gdef\@startzeit{#1} \gdef\@endzeit{#2}}
\def\stunde#1{\gdef\@stunde{#1}}
\def\schule#1{\gdef\@schule{#1}}
\def\raum#1{\gdef\@raum{#1}}
%    \end{macrocode}
% \end{macro}
% \end{macro}
% \end{macro}
% \end{macro}
% \end{macro}
% \end{macro}
% \end{macro}
% \end{macro}
% \end{macro}
% \end{macro}
%
% \begin{macro}{\teila}
% \begin{macro}{\teilb}
% \begin{macro}{\anhang}
%	Für die aktuelle PO wird eine Schriftliche Arbeit angefertigt.
%	Entsprechende Definitionen für Teil A, Teil B und Anhang 
%    \begin{macrocode}
\ifthenelse{\boolean{B@neuePO}}{
	\newenvironment{teila}{
		\renewcommand*{\thesection}{\Alph{section}}
		\renewcommand*{\thesubsection}{\Alph{section}~\arabic{subsection}} 
		\addsec{Teil A -- Schriftliche Planung der Unterrichtsstunde}
		\setcounter{section}{1}
		\setcounter{subsection}{0}
	}{\clearpage}
	\newenvironment{teilb}{
		\renewcommand*{\thesection}{\Alph{section}}
		\renewcommand*{\thesubsection}{\Alph{section}~\arabic{subsection}} 
		\addsec{Teil B -- Darstellung der längerfristigen Zusammenhänge}
		\setcounter{section}{2}
		\setcounter{subsection}{0}
	}{\clearpage}	
	\newenvironment{ziele}[3][]{%
		\ifthenelse{\not\isempty{#1}}{%
			\textbf{Hauptlernziel:} #1

		}{}
		\textbf{#2:}
	
		#3
		\ifthenelse{\boolean{B@zieleMulti}}{\begin{multicols}{2}}{}		
		\begin{smallitemize}
	}{
		\end{smallitemize}
		\ifthenelse{\boolean{B@zieleMulti}}{\end{multicols}}{}		
	}

	
	\newenvironment{anhang}{
		\addsec{Anhang}
	}{\clearpage}
}
{}
%    \end{macrocode}
% \end{macro}
% \end{macro}
% \end{macro}
%
%
% \begin{macro}{\schuladresse}
% \begin{macro}{\lehrer}
%    \begin{macrocode}
\def\schuladresse#1{\gdef\@seminaradresse{#1}}
\def\lehrer#1{\@ifnextchar[{\@referendarintern{#1}}%
    {\@referendarintern{#1}[]}}
%    \end{macrocode}
% \end{macro}
% \end{macro}
%
% \begin{macro}{\ausbildungsl}
% \begin{macro}{\ako}
% \begin{macro}{\schulleiter}
% \begin{macro}{\hauptseminar}
% \begin{macro}{\fachEins}
% \begin{macro}{\fachZwei}
% Setzen des Referendars, des Ausbildungslehrers, der Lehrkraft für GU
% (aktuelle PO), des AKOs, des Schulleiters, des
% Hauptseminarleiters und der beiden Fachsemiarleiter  ggf. mit
% Erweiterung, für die weibliche Endung
%    \begin{macrocode}
\def\referendar#1{\@ifnextchar[{\@referendarintern{#1}}%
    {\@referendarintern{#1}[]}}
\def\@referendarintern#1[#2]{\gdef\@referendarIn{#2: & #1}%
    \gdef\@referendar{#1}}
\def\ausbildungsl#1{\@ifnextchar[{\@ausbildungslintern{#1}}%
    {\@ausbildungslintern{#1}[]}}
\def\@ausbildungslintern#1[#2]{\gdef\@ausbildungsl{#2: & #1}}
\def\foerderbedarfl#1#2#3{
  \gdef \@foerderbedarflVorn{#1}
  \gdef \@foerderbedarflNachn{#2}
  \gdef \@foerderbedarflFkt{#3}
}
\def\ako#1{\@ifnextchar[{\@akointern{#1}}{\@akointern{#1}[]}}
\def\@akointern#1[#2]{\gdef\@ako{#1}\gdef\@akoart{#2}}
\def\schulleiter#1{\@ifnextchar[{\@schulleiterintern{#1}}%
    {\@schulleiterintern{#1}[]}}
\def\@schulleiterintern#1[#2]{\gdef\@schulleiter{#2: & #1}}
\def\hauptseminar#1{\@ifnextchar[{\@hauptseminarintern{#1}}%
    {\@hauptseminarintern{#1}[]}}
\def\@hauptseminarintern#1[#2]{\gdef\@hauptseminar{#2: & #1}}
\def\fachEins#1#2{\@ifnextchar[{\@fachEinsintern{#1}{#2}}%
    {\@fachEinsintern{#1}{#2}[]}}
\def\@fachEinsintern#1#2[#3]{\gdef\@fachEins{#3 #1} %
    \gdef\@fachleiterEins{#2}}
\def\fachZwei#1#2{\@ifnextchar[{\@fachZweiintern{#1}{#2}}%
    {\@fachZweiintern{#1}{#2}[]}}
\def\@fachZweiintern#1#2[#3]{\gdef\@fachZwei{#3 #1} %
    \gdef\@fachleiterZwei{#2}}
%    \end{macrocode}
%  \end{macro}
%  \end{macro}
%  \end{macro}
%  \end{macro}
%  \end{macro}
%  \end{macro}
%
% \begin{macro}{\vorsitz}
% \begin{macro}{\schulvertreter}
% \begin{macro}{\fremderseminar}
% \begin{macro}{\bekannterseminar}
% Setzen des Prüfungsvorsitzenden, des Schulvertreters, des fremden
% Seminarausbilders unde des bekannten Seminarausbilders ggf. mit
% Erweiterung, für die weibliche Endung
%    \begin{macrocode}
\def\vorsitz#1{\@ifnextchar[{\@vorsitzintern{#1}}%
    {\@vorsitzintern{#1}[]}}
\def\@vorsitzintern#1[#2]{\gdef\@vorsitz{#2: & #1}}
\def\schulvertreter#1{\@ifnextchar[{\@schulvertreterintern{#1}}%
    {\@schulvertreterintern{#1}[]}}
\def\@schulvertreterintern#1[#2]{\gdef\@schulvertreter{#2: & #1}}
\def\fremderseminar#1{\@ifnextchar[{\@fremderseminarintern{#1}}%
    {\@fremderseminarintern{#1}[]}}
\def\@fremderseminarintern#1[#2]{\gdef\@fremderseminar{#2: & #1}}
\def\bekannterseminar#1{\@ifnextchar[{\@bekannterseminarintern{#1}}%
    {\@bekannterseminarintern{#1}[]}}
\def\@bekannterseminarintern#1[#2]{\gdef\@bekannterseminar{#2: & #1}}
%    \end{macrocode}
%  \end{macro}
%  \end{macro}
%  \end{macro}
%  \end{macro}
%
% \begin{macro}{\weiblich}
% \begin{macro}{\maennlich}
%  Setzen der Anzahl der weiblichen und männlichen \SuS, sowie Angabe
%  der \SuS mit Förderbedarf (aktuelle PO)
%    \begin{macrocode}
\newcounter{@weiblich}
\def\weiblich#1{\setcounter{@weiblich}{#1}}
\newcounter{@maennlich}
\def\maennlich#1{\setcounter{@maennlich}{#1}}
\newcounter{@foerderbedarf}
\setcounter{@foerderbedarf}{-1}
\def\foerderbedarf#1{\setcounter{@foerderbedarf}{#1}}
\newcounter{@SuStotal}
%    \end{macrocode}
%  \end{macro}
%  \end{macro}
% 
% Setzen der Kopfzeile 
%    \begin{macrocode}
\ihead{\@referendar}
\chead{\@lerngruppeKurz}
\ohead{\@datum}
%    \end{macrocode}
%
% \begin{macro}{\externesDokumentEinseitig}
% \begin{macro}{\externesDokumentMehrseitig}
%		Macro, um externe Dokumente automatisch skalierbar einbinden zu
%		können.
\newcommand{\externesDokumentEinseitig}[1]{
	\centering{\fbox{
		\adjustbox{max size={\textwidth}{0.85\textheight}}{\includegraphics[page=1]{#1}}}
	}
}
\newcommand{\externesDokumentMehrseitig}[2][]{
	\centering{\fbox{
		\adjustbox{max size={\textwidth}{0.85\textheight}}{\includegraphics[page=1]{#2}}}
	}
	\ifthenelse{\equal{#1}{\empty}}{%
		\includepdf[
		  pages=2-last,
			scale=0.79,
			pagecommand={\thispagestyle{scrheadings}},
			frame=true]{#2}
	}{%
		\includepdf[pages=2-last,
		  scale=0.79,
			pagecommand={\thispagestyle{scrheadings}},
			frame=true,
			#1]{#2}
	}
}
% \end{macro}
% \end{macro}
% 
%  \begin{macro}{\makehead}
% Erstellen der Titelseite für den Besuch angepasst an die jeweilige
% PO
%    \begin{macrocode}
\newcommand\makehead{
\setcounter{@SuStotal}{\value{@weiblich} + \value{@maennlich}}
\ifthenelse{\boolean{B@neuePO}}{
\ifthenelse{\boolean{B@examen}}{
\begin{titlepage}
\begin{center}
Zentrum f\"ur schulpraktische Lehrerausbildung \@ort % 

\@seminarinfo

\vspace{6mm}

\large \textbf{Schriftliche Arbeit gem\"a{\ss} 
  \S 32 (5) OVP im Fach} \\[3mm]
\Large \textbf{\@fachEins} 
\normalsize
\end{center}

\vspace{6mm}

\begin{tabular}{ll}
  \textbf{Pr\"ufling} \small (Name, Vorname):\normalsize	
	  & \@referendar \\[3mm]
	Ausbildungsschule:
	  & \@schule \\[3mm]
	Datum der Pr\"ufung:
	  & \@datum	\\[0mm] 
	Unterrichtszeit (von -- bis):
	  & \@startzeit~Uhr -- \@endzeit~Uhr (\@stunde .~Stunde) \\[3mm]
	Lerngruppe (Klasse/Kurs/Jahrgang)*:
	  & \@lerngruppe \\[0mm]
	Lerngruppengr\"o{\ss}e (Anzahl):
	  & \the@SuStotal
\end{tabular}

\vspace{6mm}
\textbf{Thema der unterrichtspraktischen Pr\"ufung:} \\
\@thema \\[9mm] 

\textbf{Bezeichnung der zugeh\"origen Unterrichtsreihe:} \\
\@reihe\\[9mm]

\begin{tabular}{ll}
\textbf{Pr\"ufungskommission} \tabularnewline
Pr\"ufungsvorsitzende\@vorsitz \\[3mm]
Seminarausbilder\@bekannterseminar \tabularnewline
(an der Ausbildung \textbf{beteiligt}) & \\[3mm]
Seminarausbilder\@fremderseminar \tabularnewline
(an der Ausbildung \textbf{nicht beteiligt}) &
\end{tabular}

\vspace{20mm}

\small

\textbf{*) Zus\"atzliche Angaben f\"ur Gemeinsamen Unterricht~(GU):} 

\SuS mit sonderp\"adagogischem F\"orderbedarf (Anzahl):
\the@foerderbedarf

Im GU eingesetzte Lehrkraft/weitere Person (Name, Vorname; Funktion): 
\ifthenelse{\value{@foerderbedarf}>-1}{
  \@foerderbedarflNachn, \@foerderbedarflVorn; \@foerderbedarflFkt
}{--}
\end{titlepage}
}{
\begin{titlepage}
\begin{center}
Zentrum f\"ur schulpraktische Lehrerausbildung \@ort % 

\@seminarinfo

\vspace{6mm}

\large \textbf{Unterrichtsentwurf} \\[3mm]
\Large \textbf{\@fachEins} 
\normalsize
\end{center}

\vspace{6mm}

\begin{tabular}{ll}
	\textbf{Pr\"ufling} \small (Name, Vorname):\normalsize
	  & \@referendar \\[3mm]
	Ausbildungsschule:
	  & \@schule \\[3mm]
	Datum der Pr\"ufung:
	  & \@datum	\\[0mm] 
	Unterrichtszeit (von -- bis):
	  & \@startzeit~Uhr -- \@endzeit~Uhr (\@stunde .~Stunde) \\[3mm]
	Lerngruppe
	(Klasse/Kurs/Jahrgang)\ifthenelse{\value{@foerderbedarf}>-1}{*}{}:	
	  & \@lerngruppe \\[0mm]
	Lerngruppengr\"o{\ss}e (Anzahl):
	  & \the@SuStotal
\end{tabular}

\vspace{6mm}
\begin{tabular}{l}
\textbf{Thema der Unterrichtsstunde} \\
\@thema \\[9mm] 

\textbf{Bezeichnung der zugeh\"origen Unterrichtsreihe:} \\
\@reihe\\[9mm]
\end{tabular}

\begin{tabular}{ll}
Seminarausbilder\@bekannterseminar \tabularnewline
\end{tabular}

\vspace{20mm}

\small
\ifthenelse{\value{@foerderbedarf}>-1}{
\textbf{*) Zus\"atzliche Angaben f\"ur Gemeinsamen Unterricht~(GU):} 

\SuS mit sonderp\"adagogischem F\"orderbedarf (Anzahl): 
  \the@foerderbedarf

Im GU eingesetzte Lehrkraft/weitere Person (Name, Vorname; Funktion): 
  \@foerderbedarflNachn,
\@foerderbedarflVorn; \@foerderbedarflFkt}
{}
\end{titlepage}
}
}{
\begin{titlepage}
\begin{flushleft}\@seminaradresse \end{flushleft}
\begin{flushright}\@ort, \today\end{flushright} \par
\bigskip{}
\begin{center}
\textbf{\textsc{\huge Unterrichtsentwurf}} \par
(\@besuchtitel) \par
\vspace{4ex} \par
\textbf{\@thema} \par
\end{center} \par
\vfill \par
\begin{tabular}{ll}
\ifthenelse{\boolean{B@reversion}} %
    {Lehrer\@referendarIn \tabularnewline}%
        {Referendar\@referendarIn \tabularnewline}
Lerngruppe: & \@lerngruppe \tabularnewline
 & (\the@SuStotal~\SuS, \the@weiblich~weiblich %
    und \the@maennlich~m\"annlich  ) \tabularnewline
Datum: & \@datum \tabularnewline
Zeit: & \@startzeit~Uhr -- \@endzeit~Uhr (\@stunde .~Stunde) 
		\tabularnewline
\ifthenelse{\boolean{B@reversion}}{Raum: & \@raum \tabularnewline}{%
	Ausbildungsschule: & \@schule \tabularnewline %	
	Raum: & \@raum \tabularnewline
}
\ifthenelse{\boolean{B@reversion}} %
    {Schulleiter\@schulleiter \tabularnewline}{
\tabularnewline
\tabularnewline
\textbf{Ausbilder und Schulvertreter} \tabularnewline
Ausbildungslehrer\@ausbildungsl \tabularnewline
Ausbildungskoordinator\@akoart: & \@ako \tabularnewline
Schulleiter\@schulleiter \tabularnewline
Hauptseminarleiter\@hauptseminar \tabularnewline
Fachleiter\@fachEins: & \@fachleiterEins \tabularnewline
Fachleiter\@fachZwei: & \@fachleiterZwei \tabularnewline
\ifthenelse{\boolean{B@examen}} {
\tabularnewline
\tabularnewline
\textbf{Pr\"ufungskommission} \tabularnewline
Pr\"ufungsvorsitzender\@vorsitz \tabularnewline
Weiterer Schulvertreter\@schulvertreter \tabularnewline
Fremder Seminarausbilder\@fremderseminar \tabularnewline
Bekannter Seminarausbilder\@bekannterseminar \tabularnewline} {}
}
\end{tabular}
\end{titlepage}
}
}
%    \end{macrocode}
%  \end{macro}
% 
% Einfügen der Titelseite zu Beginn
%    \begin{macrocode}
\AtBeginDocument{
	\ifthenelse{\boolean{B@kurzentwurf}}{
		\setboolean{@stundenverlaufquer}{false}
		\ifthenelse{\boolean{@stundenverlaufASF}}{%
			\setP{p{2.5cm}}%
			\setO{X}
			\setA{p{1cm}}%
			\setM{p{1.5cm}}%
			\setD{p{8cm}}%
		}{ % mit didkatischer Kommentar, im Querformat, ohne ASF
			\setP{p{2.35cm}}%
			\setO{X}
			\setA{p{3cm}}%
			\setM{p{1.5cm}}%
			\setD{p{6cm}}%
		}
		{\large\textbf{\Thema}}		
	}{
		\makehead
	}
}
%    \end{macrocode}
% 
% Mögliches Setzen der Schlusserklärung im Fall des Examens bzw. für
% den Fall, dass in der schriftlichen Arbeit kein Anhang gesetzt wird
%    \begin{macrocode}
\AtEndDocument{
\ifthenelse{\boolean{B@examen}}{
	\ifthenelse{\boolean{B@neuePO}}{
		\addsec{Versicherung}
			Ich versichere, dass ich die Schriftliche Arbeit eigenst\"andig
			verfasst, keine anderen Quellen und Hilfsmittel als die
			angegebenen benutzt und die Stellen der Schriftlichen Arbeit,
			die anderen Werken dem Wortlaut oder Sinn nach entnommen sind,
			in jedem einzelnen Fall unter Angabe der Quelle als Entlehnung
			kenntlich gemacht habe. Das Gleiche gilt auch f\"ur beigegebene
			Zeichnungen, Kartenskizzen und Darstellungen. Anfang und Ende
			von w\"ortlichen Text\"ubernahmen habe ich durch An- und
			Abf\"uhrungszeichen, sinngem\"a{\ss}e \"Ubernahmen durch
			direkten Verweis auf die Verfasserin oder den Verfasser
			gekennzeichnet.

			\vspace{26mm}

			$\underset{\text{Unterschrift des Pr\"uflings}}{
				\text{\underline{\hspace{8.5cm}}}}$
	}{
		\addsec{Schlusserkl\"arung}
			Ich versichere, dass ich die schriftliche Planung eigenst\"andig
			verfasst, keine anderen Quellen und Hilfsmittel als die
			angegebenen benutzt und die Stellen der schriftlichen Planung,
			die anderen Werken dem Wortlaut oder Sinn nach entnommen sind,
			in jedem einzelnen Fall unter Angabe der Quelle als Entlehnung
			kenntlich gemacht habe. Das Gleiche gilt auch f\"ur beigegebene
			Zeichnungen, Kartenskizzen und Darstellungen. Anfang und Ende
			von w\"ortlichen Text\"ubernahmen habe ich durch An- und
			Abf\"uhrungszeichen, sinngem\"a{\ss}e \"Ubernahmen durch
			direkten Verweis auf die Verfasserin oder den Verfasser
			gekennzeichnet.
	}
}{}
}
%    \end{macrocode}
% 
% Ende der Klasse \texttt{schuleub}
%\iffalse
%    \begin{macrocode}
%</schuleub.cls>
%    \end{macrocode}
%\fi
%\iffalse
%    \begin{macrocode}
%<*schuleue.cls>
%    \end{macrocode}
%\fi
% \subsection{Die Klasse \texttt{schuleue}}
%	Die ausführliche Beschreibung der Klasse ist in der
%	Klassenbeschreibung (\ref{klasse:schuleue}) zu finden.
%
%  Beginn der Definition, Voraussetzung der \LaTeXe{} Version und die
%  eigene Identifizierung
%    \begin{macrocode}
\NeedsTeXFormat{LaTeX2e}[1995/12/01]
\ProvidesClass{schuleue}[2015/05/07 v0.6 %
                         Vorlage für eine Übersicht]
%    \end{macrocode}
% 
% Alle Optionen werden an die Klasse scrartcl weitergegeben.
%    \begin{macrocode}
\DeclareOption{loesung}%
    {\PassOptionsToClass{loesung}{schuleab}}
\DeclareOption{loesungseite}%
    {\PassOptionsToClass{loesungseite}{schuleab}}
\DeclareOption{loesungsseite}%
    {\PassOptionsToClass{loesungsseite}{schuleab}}
\DeclareOption*{\PassOptionsToClass{\CurrentOption}{scrartcl}}
\ProcessOptions\relax
%    \end{macrocode}
%
%    Laden der Klasse
%    \begin{macrocode}
\LoadClass{schuleab}
%    \end{macrocode}

% \subsubsection{Kopfbereich}
% 
% Setzen der Kopfzeile des Dokuments.
%    \begin{macrocode}
\ohead{\"Ubersicht%
\ifthenelse{\isundefined{\@dokNummer}}
		{}
		{ Nr. \@dokNummer}}
%    \end{macrocode}
% 
% Ende der Klasse \texttt{schuleue}
%\iffalse
%    \begin{macrocode}
%</schuleue.cls>
%    \end{macrocode}
%\fi
%\iffalse
%    \begin{macrocode}
%<*schulinf.sty>
%    \end{macrocode}
%\fi
% \subsection{Das Paket \texttt{schulinf}}
%	Die ausführliche Beschreibung des Pakets ist in der
%	Paketbeschreibung (\ref{paket:schulinf}) zu finden.
%
%  Beginn der Definition, Voraussetzung der \LaTeXe{} Version und die
%  eigene Identifizierung
%    \begin{macrocode}
\NeedsTeXFormat{LaTeX2e}[1995/12/01]
\ProvidesPackage{schulinf}[2015/05/07 v0.6 %
                           Kommandos fuer den Informatikunterricht]
%    \end{macrocode}
% Einbinden der benötigten Pakete
%    \begin{macrocode}
\RequirePackage{schule}
\RequirePackage{schullzk}
\RequirePackage[school]{pgf-umlcd}
\RequirePackage{listings}
\RequirePackage[underline=false,rounded corners=true]{pgf-umlsd}
\RequirePackage{syntaxdi}
\RequirePackage[pict2e]{struktex}
\RequirePackage{relaycircuit}
\usetikzlibrary{er}
\usetikzlibrary{circuits.logic.IEC}
%    \end{macrocode}
%
% Sorgt dafür, dass das Paket listings auch mit den Sonderzeichen in
% UTF-8 zurecht kommt.
%    \begin{macrocode}
\lstset{literate=%
{Ö}{{\"O}}1
{Ä}{{\"A}}1
{Ü}{{\"U}}1
{ß}{\ss}2
{ü}{{\"u}}1
{ä}{{\"a}}1
{ö}{{\"o}}1
{»}{{\frqq}}4
{«}{{\flqq}}4
}
%    \end{macrocode}
%
% \begin{environment}{klassenDokumentation}
% Darstellungsumgebung, um Klassen nach Vorlage des Zentralabiturs in
% NRW zu dokumentieren
%    \begin{macrocode}
\newenvironment{klassenDokumentation}{%
    \tabularx{\textwidth}{lX}}{\endtabularx}
%    \end{macrocode}
% \end{environment}
% \begin{macro}{\methodenDokumentation}
% 	Einzelne Zeile in der Klassendokumentation nach Vorlage des
% 	Zentralabiturs in NRW
%    \begin{macrocode}
\newcommand{\methodenDokumentation}[3]{
	{\color{gray}#1} & \textbf{#2} \\
	& #3 \\
}
%    \end{macrocode}
% \end{macro}
%
% \subsubsection{Kurzbefehle}
% \begin{macro}{\scaleSequenzdiagramm}
% 	Stellt die Kompatibilität zur vorherigen Version her
%    \begin{macrocode}
\newcommand{\scaleSequenzdiagramm}[1]{
	\scaleTikz{#1}%
}
%    \end{macrocode}
% \end{macro}
%
% \begin{macro}{\newthreadtwo}
% 	Ermöglicht im Sequenzdiagramm einen weiteren Thread, bei dem der
% 	Abstand zum nächsten gesetzt werden kann
%    \begin{macrocode}
\newcommand{\newthreadtwo}[4][gray!30]{
  \newinst[#4]{#2}{#3}
  \stepcounter{threadnum}
  \node[below of=inst\theinstnum,node distance=0.8cm]%
    (thread\thethreadnum) {};
  \tikzstyle{threadcolor\thethreadnum}=[fill=#1]
  \tikzstyle{instcolor#2}=[fill=#1]
}
%    \end{macrocode}
% \end{macro}
%
% \begin{macro}{\nextlevel}
% 	Damit kann im Sequenzdiagramm auf das nächste Level gesetzt
% 	werden,siehe auch  \cmd{\prevlevel}.
%    \begin{macrocode}
\newcommand{\nextlevel}{\addtocounter{seqlevel}{1}}
%    \end{macrocode}
% \end{macro}
%
% \begin{macro}{\anchormark}
%		Um in Objektdiagrammen Beziehungen anzugeben, wird der Befehl
%		\cmd{\anchormark} benötigt.
\ProvideDocumentCommand{\anchormark}{O{0.15 cm} m O{0.05}}{
        \tikz[overlay,remember picture,baseline=-1ex,xshift=#1] 
        \node[draw,fill=black,circle,scale=#3] (#2) {};
}
%
%	\end{macro}
% 
% Einstellung, dass als Fach Informatik angegeben wird
%    \begin{macrocode}
\def\@fach{Informatik}
%    \end{macrocode}
%
% Ende des Pakets \texttt{schulinf}
%\iffalse
%    \begin{macrocode}
%</schulinf.sty>
%    \end{macrocode}
%\fi
%\iffalse
%    \begin{macrocode}
%<*schullsg.cls>
%    \end{macrocode}
%\fi
% \subsection{Die Klasse \texttt{schullsg}}
%	Die ausführliche Beschreibung der Klasse ist in der
%	Klassenbeschreibung (\ref{klasse:schullsg}) zu finden.
%
%  Beginn der Definition, Voraussetzung der \LaTeXe{} Version und die
%  eigene Identifizierung
%    \begin{macrocode}
\NeedsTeXFormat{LaTeX2e}[1995/12/01]
\ProvidesClass{schullsg}[2015/05/07 v0.6 %
                         Vorlage für eine Lösung]
%    \end{macrocode}
% 
% Alle Optionen werden an die Klasse scrartcl weitergegeben.
%    \begin{macrocode}
\DeclareOption{loesung}%
    {\PassOptionsToClass{loesung}{schuleab}}
\DeclareOption{loesungseite}%
    {\PassOptionsToClass{loesungseite}{schuleab}}
\DeclareOption{loesungsseite}%
    {\PassOptionsToClass{loesungsseite}{schuleab}}
\DeclareOption*{\PassOptionsToClass{\CurrentOption}{scrartcl}}
\ProcessOptions\relax
%    \end{macrocode}
%
%    Laden der Klasse
%    \begin{macrocode}
\LoadClass{schuleab}
%    \end{macrocode}

% \subsubsection{Kopfbereich}
% 
% Setzen der Kopfzeile des Dokuments.
%    \begin{macrocode}
\ohead{L\"osung%
\ifthenelse{\isundefined{\@dokNummer}}
		{}
		{ Nr. \@dokNummer}}
%    \end{macrocode}
% 
% Ende der Klasse \texttt{schullsg}
%\iffalse
%    \begin{macrocode}
%</schullsg.cls>
%    \end{macrocode}
%\fi
%\iffalse
%    \begin{macrocode}
%<*schullzk.cls>
%    \end{macrocode}
%\fi
% \subsection{Die Klasse \texttt{schullzk}}
%	Die ausführliche Beschreibung der Klasse ist in der
%	Klassenbeschreibung (\ref{klasse:schullzk}) zu finden.
%
%  Beginn der Definition, Voraussetzung der \LaTeXe{} Version und die
%  eigene Identifizierung
%    \begin{macrocode}
\NeedsTeXFormat{LaTeX2e}[1995/12/01]
\ProvidesClass{schullzk}[2015/05/07 v0.6 %
      Vorlage für eine Lernzielkontrolle]
%    \end{macrocode}
% 
% Alle Optionen werden an die Klasse scrartcl weitergegeben.
%    \begin{macrocode}
\DeclareOption*{\PassOptionsToClass{\CurrentOption}{scrartcl}}
\ProcessOptions\relax
%    \end{macrocode}
%
%    Laden der Klasse und der nötigen Pakete und Setzen des
%    Seitenstils.
%    \begin{macrocode}
\LoadClass[parskip=half,DIV12]{scrartcl}
\RequirePackage[utf8]{inputenc}
\RequirePackage[T1]{fontenc}
\RequirePackage{schule}
\RequirePackage[headsepline]{scrpage2}
\RequirePackage{schullzk}
\pagestyle{scrheadings}
%    \end{macrocode}
% 
% \begin{macro}{\datum}
% Definition zum Setzen des Datums der Klausur
%    \begin{macrocode}
\def\datum#1{\gdef\@datum{#1}}
%    \end{macrocode}
% \end{macro}

% \subsubsection{Kopfbereich}
% \begin{macro}{\inhalt}
% Definition zum Setzen des Inhalts der LZK
%    \begin{macrocode}
\def\inhalt#1{\gdef\@inhalt{#1}}
%    \end{macrocode}
% \end{macro}
% Setzen der Kopfzeile des Dokuments.
%    \begin{macrocode}
\ihead{Lernzielkontrolle: 
	\ifthenelse{\isundefined{\@inhalt}}
		{?? \@latex@warning@no@line{Der Inhalt ist nicht angegeben}}
		{\@inhalt}
}
\chead{
	\ifthenelse{\isundefined{\@datum}}
		{\today}
		{\@datum}
}
\ohead{Name: \hspace{5cm}}
%    \end{macrocode}
% 
% Ende der Klasse \texttt{schullzk}
%\iffalse
%    \begin{macrocode}
%</schullzk.cls>
%    \end{macrocode}
%\fi
%\iffalse
%    \begin{macrocode}
%<*schullzk.sty>
%    \end{macrocode}
%\fi
%	\subsection{Das Paket \texttt{schullzk}}
%		Die ausführliche Beschreibung des Pakets ist in der entsprechenden
%		Klassenbeschreibung (\ref{klasse:schullzk}) zu finden.
%
%  Beginn der Definition, Voraussetzung der \LaTeXe{} Version und die
%  eigene Identifizierung
%		 \begin{macrocode}
\NeedsTeXFormat{LaTeX2e}[1995/12/01]
\ProvidesPackage{schullzk}[2015/05/07 v0.6 %
                   Kommandos fuer das Setzen einer Lernzielkontrolle]
%		 \end{macrocode}
% \subsubsection{Punktezählung}
% Zum Zählen der Gesamtpunkte in einer Sektion.
%    \begin{macrocode}
\newcounter{@gespunkte}
%    \end{macrocode}
%
% \begin{macro}{\punktesec}
% 	Definition einer Sektion mit Angabe der Gesamtpunkte
%    \begin{macrocode}
\newcommand{\punktesec}[2][0]{
\@punkte
\addtocounter{@gespunkte}{#1}
\section{#2 
(\get@punkte{\arabic{section}} Punkte)%
}}
%    \end{macrocode}
% \end{macro}
%
% \begin{macro}{\punktesec}
% 	Definition einer Sektion als Aufgabe mit Angabe der Gesamtpunkte
%    \begin{macrocode}
\newcommand{\aufgabensec}[2][0]{
	\let\save@thesection\thesection
	\renewcommand{\thesection}{\arabic{section}. Aufgabe:} 
	\punktesec[#1]{#2} 
	\let\thesection\save@thesection
}
%    \end{macrocode}
% \end{macro}
%
% \begin{macro}{\punkteitem}
% 	Neue Definition von \cmd{\punkteitem} und
% 	\cmd{\punkteitemloesung}, das die Punkte hinzuzählt.
%    \begin{macrocode}
\renewcommand{\punkteitem}[1]{%
\ifthenelse%
{\equal{#1}{1}}%
    {\item \textbf{(1 Punkt)}}%
    {\item \textbf{(#1 Punkte)}}%
\addtocounter{@gespunkte}{#1}
}
\renewcommand{\punkteitemloesung}[3][]{
\ifthenelse{\equal{#2}{1}}
	{\item \textbf{(1 Punkt)} #3}
	{\item \textbf{(#2 Punkte)} #3}
\ifthenelse{\equal{#1}{}}{}{
	\ifthenelse{\boolean{@loesunganzeigen}}
		{\\\textbf{L\"osung:} #1}{}
	\ifthenelse{\boolean{@loesunganzeigen@Seite}}
		{\phantomsection
		 \label{loesung@\the@loesung@nr}
		 \global\expandafter\def\csname
		loesung@\the@loesung@nr\endcsname{
		 \textbf{\ref{loesung@\the@zeige@nr}. Aufgabe:} #1
		} 
		\addtocounter{@loesung@nr}{1}}{}
	}
\addtocounter{@gespunkte}{#2}
}
%    \end{macrocode}
% \end{macro}
%
% \begin{macro}{\setzePunkte}
% 	Definition von \cmd{\setzePunkte}, das die Punkt in der
% 	\texttt{.aux}-Datei schreibt, wenn eine Sektion ohne Punktangabe
% 	aufgerufen wird. Muss vor dem \cmd{\section} aufgerufen werden.
%    \begin{macrocode}
\renewcommand{\setzePunkte}{\@punkte}
%    \end{macrocode}
% \end{macro}
%
% \begin{macro}{\newpunkte}
% 	Definition von \texttt{newpunkte}, mit dem in der
% 	\texttt{.aux}-Datei gearbeitet wird.
%    \begin{macrocode}
\def\newpunkte#1#2{
	\global\expandafter\def\csname punkte@#1\endcsname{#2}
}
%    \end{macrocode}
% \end{macro}
%
% \begin{macro}{\get@punkte}
% 	Definition von \cmd{\get@punkte}, das die Punkte für eine Sektion
% 	zurückliefert, durch weiteres Auseinandernehmen an
% 	\cmd{\@get@punkte}. 
%    \begin{macrocode}
\def\get@punkte#1{\expandafter\@get@punkte\csname %
                            punkte@#1\endcsname}
%    \end{macrocode}
% \end{macro}
%
% \begin{macro}{\@get@punkte}
% 	Definition von \cmd{\@get@punkte}, das die Punkte für eine Sektion
% 	zurückliefert.
%    \begin{macrocode}
\def\@get@punkte#1{%
	\ifx#1\relax
		??%
  \else
   \expandafter#1%
\fi}
%    \end{macrocode}
% \end{macro}
%
% \begin{macro}{\@punkte}
% 	Definition von \cmd{\@punkte}. Schreibt die Punkte mit
% 	\cmd{\newpunkte} in die \texttt{.aux}-Datei und setzt den Zähler
% 	wieder auf 0.
%    \begin{macrocode}
\newcommand{\@punkte}{
  \immediate\write\@auxout{%
         \string\newpunkte{\arabic{section}}{\the@gespunkte}}%
 	\expandafter\test@punkte\csname %
            punkte@\arabic{section}\endcsname{\the@gespunkte}
	\setcounter{@gespunkte}{0}
}
%    \end{macrocode}
% \end{macro}
%
% \begin{macro}{\test@punkte}
% 	Definition von \cmd{\test@punkte}  testet, ob sich eine Änderung
% 	bei den Punkten ergeben hat und dafür sorgt, dass eine Warnung
% 	ausgegeben werden kann.
%    \begin{macrocode}
\newcommand{\test@punkte}[2]{
 \ifthenelse{\equal{#1}{#2}}%
 	{}
 	{\gdef\punkte@undefined{}}
}
%    \end{macrocode}
% \end{macro}
% 
% Am Anfang des Dokuments muss der Zähler auf 0 gesetzt werden
%    \begin{macrocode}
\AtBeginDocument{\setcounter{@gespunkte}{0}}
%    \end{macrocode}
% 
% Am Ende des Dokuments werden die letzten Punkte gespeichert. Sollte
% sich dabei eine Veränderung ergeben haben, wird eine Warnung
% ausgegeben.
%    \begin{macrocode}
\AtEndDocument{
\@punkte
\ifthenelse{\isundefined{\punkte@undefined}}
	{}
	{\@latex@warning@no@line{Punktanzahl wurde geändert}}}
%    \end{macrocode}
%
% Ende des Pakets \texttt{schullzk}
%\iffalse
%    \begin{macrocode}
%</schullzk.sty>
%    \end{macrocode}
%\fi
%\iffalse
%    \begin{macrocode}
%<*schulphy.sty>
%    \end{macrocode}
%\fi
% \subsection{Das Paket \texttt{schulphy}}
%	Die ausführliche Beschreibung des Pakets ist in der
%	Paketbeschreibung (\ref{paket:schulphy}) zu finden.
%
%  Beginn der Definition, Voraussetzung der \LaTeXe{} Version und die
%  eigene Identifizierung
%    \begin{macrocode}
\NeedsTeXFormat{LaTeX2e}[1995/12/01]
\ProvidesPackage{schulphy}[2015/05/07 v0.6 %
                           Kommandos fuer den Physikunterricht]
%    \end{macrocode}
% Einbinden der geforderten Pakete
%    \begin{macrocode}
\RequirePackage{schule}
\RequirePackage{units}
\RequirePackage{circuitikz}
\RequirePackage[version=3]{mhchem}
%    \end{macrocode}
%
% \subsubsection{Kurzbefehle}
% Einstellung, dass als Fach Physik angegeben wird
%    \begin{macrocode}
\def\@fach{Physik}
%    \end{macrocode}
%
% Ende des Pakets \texttt{schulphy}
%\iffalse
%    \begin{macrocode}
%</schulphy.sty>
%    \end{macrocode}
%\fi
%\iffalse
%    \begin{macrocode}
%<*syntaxdi.sty>
%    \end{macrocode}
%\fi
% \subsection{Das Paket \texttt{syntaxdi}}
%	Die ausführliche Beschreibung des Pakets ist in der
%	Paketbeschreibung (\ref{paket:syntaxdi}) zu finden.
%
%  Beginn der Definition, Voraussetzung der \LaTeXe{} Version und die
%  eigene Identifizierung
%    \begin{macrocode}
\NeedsTeXFormat{LaTeX2e}[1995/12/01]
\ProvidesPackage{syntaxdi}[2015/05/07 v0.6 %
                           Syntaxdiagramme mit TikZ]
%    \end{macrocode}
% Einbinden der benötigten Pakete
%    \begin{macrocode}
\RequirePackage{tikz}
\usetikzlibrary{chains}
\usetikzlibrary{arrows,shadows,shapes.misc,scopes}
%    \end{macrocode}
%
% \subsubsection{TikZ-Definitionen}
%
% 	Definition für nicht terminale Symbole für Syntaxdiagramme in TikZ
%    \begin{macrocode}
\tikzset{
	fnonterminal/.style={
		rectangle,
		minimum size=6mm,
		text height=1.5ex,text depth=.25ex,
		very thick,
		draw=red!50!black!50,  % 50% red und 50% black,
		top color=white,              % oben: weisser Schatten ...
		bottom color=red!50!black!20, % unten: anderer Schatten
		font=\itshape
	}
}
\tikzset{
	nonterminal/.style={
		% Die Form:
		rectangle,
		% Die Größe:
		minimum size=6mm,
		text height=1.5ex,text depth=.25ex,
		% Der Rand:
		very thick,
		draw=red!50!black!50,  % 50% red und 50% black,
		% gemischt mit 50% white
		% Füllfarbe:
		top color=white,              % oben: weisser Schatten ...
		bottom color=red!50!black!20, % unten: anderer Schatten
		% Font
		font=\itshape
	}
}
%    \end{macrocode}
%
% 	Definitionen für terminale Symbole im Syntaxdiagramm in TikZ
%    \begin{macrocode}
\tikzset{
	fterminal/.style={
		rounded rectangle,
		minimum size=6mm,
		very thick,draw=black!50,
		text height=1.5ex,text depth=.25ex,
		top color=white,bottom color=black!20,
		font=\ttfamily
	}
}
\tikzset{
	terminal/.style={
		% Die Form:
		rounded rectangle,
		minimum size=6mm,
		% Der Rest ...
		very thick,draw=black!50,
		text height=1.5ex,text depth=.25ex,
		top color=white,bottom color=black!20,
		font=\ttfamily
	}
}
%    \end{macrocode}
%
% 	Definitionen eines Punktes für das Syntaxdiagramm in TikZ
%    \begin{macrocode}
\tikzset{
	point/.style={
		circle,
		inner sep=0pt,
		minimum size=0pt
	}
}
%    \end{macrocode}
%
% 	Definition eines Endpunktes für das Syntaxdiagramm in TikZ
%    \begin{macrocode}
\tikzset{
	endpoint/.style={
		circle,
		inner sep=0pt,
		minimum size=0pt
	}
}
%    \end{macrocode}
%
% 	Definition der Syntaxdiagramme in TikZ
%    \begin{macrocode}
\tikzset{
	syntaxdiagramm/.style={
		start chain,
		node distance=7mm and 5mm,
		every node/.style={on chain},
		nonterminal/.append style={join=by ->},
		terminal/.append style={join=by ->},
		endpoint/.append style={join=by ->},
		point/.append style={join=by -},
		skip loop/.style={to path={-- ++(0,-.5) -| (\tikztotarget)}}
	}
}
%    \end{macrocode}
%
% Ende des Pakets \texttt{syntaxdi}
%\iffalse
%    \begin{macrocode}
%</syntaxdi.sty>
%    \end{macrocode}
%\fi
%\Finale
\endinput
