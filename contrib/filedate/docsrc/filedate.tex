\ProvidesFile{filedate.tex}[2013/03/25 documenting filedate.sty]
%% 2012/06/24
\title{\pkgtitle{filedate.sty}{Access and Compare 
       \LaTeX~File~%                                %% 2012/11/06 
       Info 
       \\                                           %% 2012/11/06
       and Modification Date}} 
% \listfiles
{ \RequirePackage{makedoc} \ProcessLineMessage{}
  \MakeJobDoc{18}%% 2012/10/16
  {\SectionLevelThreeParseInput}  }           %% Three 2013/03/24
\documentclass[fleqn]{article}%% TODO paper dimensions!?
\ProvidesFile{makedoc.cfg}[2011/06/27 documentation settings] 

\author{Uwe L\"uck\thanks{\url{http://contact-ednotes.sty.de.vu}}}
% \author{Uwe L\"uck---{\tt http://contact-ednotes.sty.de.vu}}

%% hyperref:
\RequirePackage{ifpdf}
\usepackage[%
  \ifpdf
%     bookmarks=false,          %% 2010/12/22
%     bookmarksnumbered,
    bookmarksopen,              %% 2011/01/24!?
    bookmarksopenlevel=2,       %% 2011/01/23
%     pdfpagemode=UseNone,
%     pdfstartpage=10,
%     pdfstartview=FitH,
    citebordercolor={ .6 1    .6},
    filebordercolor={1    .6 1},
    linkbordercolor={1    .9  .7},
     urlbordercolor={ .7 1   1},   %% playing 2011/01/24
  \else
    draft
  \fi
]{hyperref}

\RequirePackage{niceverb}[2011/01/24] 
\RequirePackage{readprov}               %% 2010/12/08
\RequirePackage{hypertoc}               %% 2011/01/23
\RequirePackage{texlinks}               %% 2011/01/24
\makeatletter
  \@ifundefined{strong} 
               {\let\strong\textbf}     %% 2011/01/24
               {} 
  \@ifundefined{file} 
               {\let\file\texttt}       %% 2011/05/23
               {} 
\makeatother

\errorcontextlines=4
\pagestyle{headings}

\endinput

 %% shared formatting settings
\usepackage{filedate,readprov}
\MDkeywords{modification date, metadata, package documentation, 
            %% <- were these two 2012/10/25 ->
            document versions, macro programming}
\usepackage{lmodern}
\usepackage{filesdo}              %% TODO makedoc.cfg? 2013/03/25
\sloppy 
\providecommand*{\LuaTeX}{Lua\TeX}
\providecommand*{\pdfTeX}{pdf\TeX}
\providecommand*{\XeLaTeX}{X\lower.5ex\hbox{E}\kern-.125em\LaTeX}
%% <- TODO some logo package
% \newcommand*{\qtdfile}{}                          %% 2012/11/06
\newcommand*{\secref}[1]{Section~\ref{sec:#1}}      %% 2013/03/24
\begin{document}
\maketitle
\begin{MDabstract}
'filedate.sty' provides basic access to the date of a 
\LaTeX\ source file according to its `\ProvidesFile', 
`\ProvidesPackage', or `\ProvidesClass' entry---the ``info date"---, 
as well as to its modification date according to `\pdffilemoddate' 
if the latter is available. Moreover commands are provided
to compare the ``info date" with the modification date, with ``today"'s 
date, or with another date---that a script accessing modification dates 
such as \CtanPkgRef{adhocfilelist}{adhocfilelist.sh}
may insert---, and to choose the effect of comparisons 
(error vs.\ ``notice," reference date characterization). 
Thus updating the ``info date" (``\strong{date consistency}") 
of a source file may be ensured by a test 
during typesetting from it or by some (shell/\TeX) script.
v0.4 enables checking info dates automatically as soon as 
a \LaTeX\ file is loaded while typesetting or in a 
\ctanpkgref{myfilist} script.

\MDaddtoabstract{Related packages:} \ctanpkgref{filemod}, 
\ctanpkgref{getfiledate}, \ctanpkgref{zwgetfdate}, 
\ctanpkgref{fileinfo}
\end{MDabstract}
\tableofcontents

%   \newpage
\section{Features and Usage}
\subsection{Basics of Usage}                        %% 2013/03/24
\subsubsection{The Most Interesting Command}
The package allows to check whether the file \strong{info}
date <date> according to `\Provides' near the top of a \LaTeX\ 
input file <file>---i.e.,
\[`\Provides...{<file>}[<date> ...]'\]
has been updated the same day when <file> actually was \strong{modified} 
most recently. With \pdfTeX, this can be checked by 
\[|\CheckDateOfPDFmod{<file>}|\]

\subsubsection{Installing and Calling}
The file 'filedate.sty' is provided ready, installation only requires
putting it somewhere where \TeX\ finds it
(which may need updating the filename data
 base).\urlfoot{ukfaqref}{inst-wlcf}           %% corr. 2011/02/08

%% extended 2011/01/14:
Below the `\documentclass' line(s) and above `\begin{document}',
you load 'filedate.sty' (as usually) by
\[|\usepackage{filedate}|\] 
but in ``\TeX\ scripts" such as \hyperref[sec:wrong]{below}, 
\[|\RequirePackage{filedate}|\]
is better.

\subsubsection{Demonstration with a ``\TeX\ script" Example}
\label{sec:wrong}
The accompanying `wrong.tex' is an example of a ``\pkg{filedate} \TeX\ script" 
demonstrating what may go wrong.
% \begin{quotation}\tt\small
% \expandafter\def\expandafter\{\expandafter{\string{}
% \expandafter\def\expandafter\}\expandafter{\string}}
% \obeyspaces\obeylines
% \cs{ProvidesFile}\{wrong.tex\}[2012/10/15 filedate.sty demo]
% \cs{RequirePackage}\{filedate\}
% \cs{CheckDateOfPDFmod}\{wrong\}
% \cs{CheckDateOfPDFmod}\{wrong.tex\}
% \cs{CheckDateOfToday}\{wrong.tex\}
% \cs{stop}
% \end{quotation}
\MDsamplecodeinput{wrong}
\ReadFileInfos{wrong}
You may run it (by the command line \qtd{\file{latex wrong}}) and experience:
\begin{enumerate}
%  \AddQuotes
  \item `wrong.tex''s ``info date" is \qtd{\file{\theinfodateof{wrong.tex}}}, 
    but its modification date is at least one day later.
  \item 
    `\CheckDateOfPDFmod{wrong}' demonstrates that in 
    \[|\CheckDateOfPDFmod{<file>}|\] 
    <file> must be the \Wikiref{filename} \emph{including extension.}
    Otherwise the ``info date" may be (displayed as) ``unknown."
  \item 
    |\CheckDateOfPDFmod{wrong.tex}| tests against `wrong.tex''s 
    modification date according to `\pdffilemoddate'---the present 
    package documentation uses \ctanpkgref{pdftex} indeed.
  \item 
    |\CheckDateOfToday{wrong.tex}| tests against ``today"'s date, 
    which should be different from `2012/10/15'.
  \item 
    The ``script" terminates on \LaTeX's |\stop| command, 
    without typesetting anything. 
%     (or that's what I expect and what has happened when I tried it).
    \TeX\ is just used as a program, a command interpreter 
    (as with \ctanpkgref{docstrip}).
\end{enumerate}

% \subsection{The Single Commands}
% $\dots$ are described below near to their implementation.

  \pagebreak[2]                                     %% 2013/03/25

\section{Implementation and Single Commands}
\subsection{Package File Header (Legalese)}
\NeedsTeXFormat{LaTeX2e}[1994/12/01]
\ProvidesPackage{filedate}[2013/03/26 v0.41 check file dates (UL)]

%% Copyright (C) 2012 2013 Uwe Lueck,
%% http://www.contact-ednotes.sty.de.vu
%% -- author-maintained in the sense of LPPL below --
%%
%% This file can be redistributed and/or modified under
%% the terms of the LaTeX Project Public License; either
%% version 1.3c of the License, or any later version.
%% The latest version of this license is in
%%     http://www.latex-project.org/lppl.txt
%% We did our best to help you, but there is NO WARRANTY.
%%
%% Please report bugs, problems, and suggestions via
%%
%%   http://www.contact-ednotes.sty.de.vu
%%
%% % \filbreak                       %% 2012/11/10 rm. 2013/03/25
%%  === The 'readprov' Package    ===
%% \label{sec:readprov}
% \RequirePackage{readprov}
%% ---is required for \[|\ReadInfoDate{<file>}|\quad 
%% \mbox{and}\quad |\ReadCheckDateOf{<file>}{<date>}|\] 
%% %% <- <date>, \[...\] 2012/10/25
%% % \hyperref[sec:readinfo]{below} 
%% (sections~\ref{sec:readinfo} and \ref{sec:readcheck})
%% % below
%% only. 
%% Please care for providing \ctanpkgdref{readprov} %% href 2013/03/25
%% on your own if you need that.
%%
%%  === Providing Dates for Comparisons      ===    %% 2013/03/24
%% ==== Accessing ``Info Date"               ====
%% \label{sec:readinfo}
%% |\theinfodateof{<file>}| will expand to the first ``word" of the 
%% `\Provides'\code{...} entry, provided that has been read before:
\newcommand*{\theinfodateof}[1]{%
    \@ifundefined{ver@#1}{unknown}{%
         \expandafter\expandafter\expandafter
             \FD@firstword\csname ver@#1\endcsname\@gobble{} \@nil}}
\def\FD@firstword#1 #2\@nil{#1}
%% This avoids the `\relax' that `\UseDateOf' from \ctanpkgref{readprov}
%% currently adds (which doesn't harm in printing but is bad for comparing).
%%
%% |\LoadInfoDateOf{<file>}| sets |\theinfodate| to the first word of 
%% what is in the `\Provides' instruction of <file>, provided that info 
%% has been input. So far, you must care for yourself that this works.
%% % The purpose of this kind of actions is to refer to a ``type'' of 
%% % comparisons (with `\CheckDateGiven') for a multiplicity of files.
\newcommand*{\LoadInfoDateOf}[1]{%
    \edef\theinfodate{\theinfodateof{#1}}}
%% |\ReadInfoDateOf{<file>}| additionally inputs the info before:            
\newcommand*{\ReadInfoDateOf}[1]{%
    \ReadFileInfos{#1}\LoadInfoDateOf{#1}}
%% TODO provide automatically.
%%
%% ==== Accessing \cs{pdffilemoddate}        ====
%% \label{sec:pdfmod}
%% |\pdffilemoddate{<file>}| in the first instance is a \ctanpkgref{pdftex}
%% primitive. With \LuaTeX, 
%% \ctanpkgdref{pdftexcmds} provides it.            %% d 2013/03/25
%% Currently, you must care for this yourself before loading the 
%% present package. I recommend the 
%% \ctanpkgdref{filemod} documentation for          %% d 2013/03/25
%% details about \cs{pdffilemoddate}.
%%
%% %% rm. 2013/03/24:
%% % Otherwise, (with \XeLaTeX) the modification date may be obtained 
%% % by a (shell) script---the next definitions do \emph{not} deal with 
%% % the latter situation. Testing against ``today" 
%% % (|\rawtoday| in Section~\ref{sec:rawtoday})
%% % may be another alternative.
%%
%% |\thepdfmoddateof{<file>}| expands to the modification date 
%% (eight digits separated by two slashes) if `\pdffilemoddate' 
%% is available. Otherwise, we are trying to inform about 
%% unavailability:
\ifx\pdffilemoddate\@undefined
    \newcommand*{\thepdfmoddateof}{%
        \string\pdffilemoddate\space unavailable.}
\else
    \newcommand*{\thepdfmoddateof}[1]{%
        \expandafter \FD@pdftexdate \pdffilemoddate{#1}\@nil}
    \expandafter \def \expandafter
        \FD@pdftexdate\string D:#1#2#3#4#5#6#7#8#9\@nil{%
            #1#2#3#4/#5#6/#7#8}
%% ---cf.~Will \ctanpkgauref{robertson}{Robertson}'s suggestion dating from 2010 on 
%% \httpref{stackoverflow.com/questions/2118972/latex-command-for-last-modified}{%
%% \urlfmt{stackoverflow.com}} in another discussion of accessing modification dates, 
%% including use of scripts.
%% `\string D' deals with the fact that `\pdffilemoddate' returns 
%% % in                         %% rm. 2013/03/24
%% ``other" character tokens.
\fi
%% The modification date of <file> according to `\pdffilemoddate'
%% will be available as `\thepdffilemoddate' after 
%% `ReadPDFfileModDateOf{<file>}', see \secref{short-refdates}.
%%
%% ==== \cs{rawtoday}                        ====
%% \label{sec:rawtoday}
%% % Enabling `\CheckDateOf{<file>}{\rawtoday}':
%% |\rawtoday| accesses ``today"'s date as eight digits separated 
%% by two slashes (`yyyy/mm/dd'):
\newcommand*{\rawtoday}{%
    \the\year/\two@digits{\the\month}/\two@digits{\the\day}}
%%
%%  === Comparing Basically                  ===
%% \label{sec:readcheck}
%% |\CheckDateOf{<file>}{<ref-date>}| compares <file>'s \strong{info} date with 
%% the \strong{reference} date <ref-date>:
\newcommand*{\CheckDateOf}[2]{%
%% We provide a check that does not affect the order with 
%% \ctanpkgdref{myfilist}.                                  %% 2013/03/25
%     \ReadFileInfos{#1}%
%% The date according to `\Provides' will be accessible as |\theinfodate|:
    \LoadInfoDateOf{#1}%
%     \show\theinfodate
    \ReadPDFmodDateOf{#1}%
    \edef\FD@therefdate{#2}%
%     \show\FD@therefdate
    \ifx\theinfodate\FD@therefdate
        \FD@datesequal{#1}%
    \else
        \FD@datesdiff{#1}%
    \fi}
%% The \strong{reference} date may be either (i)~\emph{today} as accessed 
%% by `\rawtoday' (\secref{rawtoday}), (ii)~the \emph{modification} date as accessed 
%% by `\pdffilemoddate' (\secref{pdfmod}), (iii)~something \emph{else} relevant, 
%% e.g., a modification date determined and inserted by a shell script.
%% %% <- added 2013/03/24
%% 
%% |\ReadCheckDateOf{<file>}{<date>}| \ prepends \ |\ReadFileInfos{<file>}|
%% from the \ctanpkgref{readprov} package (cf. Section~\ref{sec:readprov}), 
%% in order to ensure that the \strong{info} date is known:
\newcommand*{\ReadCheckDateOf}[1]{%
    \ReadFileInfos{#1}\CheckDateOf{#1}}
%% TODO provide automatically.
%% 
%%  === Reporting Styles                     === %% subsections 2013/03/24
%% ==== Same Dates                           ====
%% By default, there is no report about comparisons finding \strong{equality}.
\let\FD@datesequal\@gobble
%% % We do not want to disturb `\listfiles' with \ctanpkgref{myfilist}. 
%% %% <- rm. 2012/10/25
%% |\EqualityMessages| changes this to screen and log messages: 
\newcommand*{\EqualityMessages}{\let\FD@datesequal\FD@equalmess}
\def\FD@equalmess#1{\message{ + #1 passed date check + }}
\def\FD@errdatesdiff#1{%
    \PackageError{filedate}{%
        \FD@infodate{#1}\FD@refdate}{%      %% \fd@refdate 2012/10/19
        Fix that!}}
%% |\FD@infodate{<file>}| might be used to change the current 
%% presentation of the ``info date:"
\def\FD@infodate#1{% 
    #1 has \string\Provides... date \theinfodate\space}
%% TODO here |\theinfodate| could be replaced by 
%% |\theinfodateof{#1}|, there is no essential application of 
%% `\theinfodate' currently.
%%
%% ==== Differing Dates                      ====
%% After |\DatesDiffErrors|, date \strong{differences} 
%% are reportet ``drastically" by `\PackageError':
\newcommand*{\DatesDiffErrors}{\let\FD@datesdiff\FD@errdatesdiff}
%% This is the default:
\DatesDiffErrors
%% After |\DatesDiffNotices|, date differences are reported by 
%% %% <- told -> reported  2012/10/19
%% `\typeout':
\newcommand*{\DatesDiffNotices}{\let\FD@datesdiff\FD@notedatesdiff}
\def\FD@notedatesdiff#1{\def\MessageBreak{^^J}%   %% added 2012/10/24
                        {\typeout{\FD@infodate{#1}%
                                  \FD@refdate}}}  %% added 2012/10/19
%% v0.3 adds |\DatesDiffWarnings| to get more salient reports of 
%% date differences by `\PackageWarningNoLine':
\newcommand*{\DatesDiffWarnings}{\let\FD@datesdiff\FD@warndatesdiff}
\def\FD@warndatesdiff#1{%
    \PackageWarningNoLine{filedate}%
                         {\FD@infodate{#1}\FD@refdate}}
%% ==== Kinds of Reference Dates             ====
%% When the reference date is `\pdffilemoddate', the report about 
%% a comparison may call it a ``modification date". 
%% But when the reference date is ``today", it may not be a 
%% ``modification date". Otherwise, it depends ...
%% (See \secref{readcheck} for kinds of reference dates.)
%% 
%% After |\ModDates|, reference dates are called ``modification'' dates: 
\newcommand*{\ModDates}{\let\FD@refdate\FD@moddate}
\def\FD@moddate{\MessageBreak vs. modification date \FD@therefdate}
%% After |\SomeDates|, the type of reference dates is not specified. 
%% This is more accurate when the info date is compared with |\rawtoday|.
\newcommand*{\SomeDates}{\let\FD@refdate\FD@somedate}
\def\FD@somedate{\MessageBreak vs. \FD@therefdate}
%% That's the default:
\SomeDates
%%
%%  === Shorthands                           ===
%% %% mod. again 2013/03/25
%% We may want to compare \emph{info} with \emph{reference} dates for 
%% many files. But when the \emph{reference} date is `\pdffilemoddate' 
%% for many files
%% (probably the most frequent application, see \secref{pdfmod}), 
%% we don't want to state this explicitly for each 
%% file---\secref{short-refdates}---neither 
%% for other kinds of reference dates.
%% 
%% As to enumerating \emph{file names}, the 'filesdo' package 
%% from the \ctanpkgdref{commado} bundle should help 
%% (combinations of base names and extensions, 
%%  see \secref{demo-here};                             %% 2013/03/25
%%  TODO: shorthands using 'filesdo' could be defined 
%%  in the present package).
%%
%% The files <in-file> whose dates we want to check may be just 
%% the same that some <reading-file> tries to `\input'.
%% It may me helpful if those input commands trigger the 
%% file consistency check without a need to demand this explicitly 
%% for each file---the section on ``automatic immediate checking"
%% (\secref{auto-check}) aims at this. 
%% 
%% ==== Kinds of Reference Dates             ====
%% \label{sec:short-refdates}
%% |\ReadPDFmodDateOf{<file>}| enables
%% \[|\CheckDateOf{<file>}{\thepdfmoddate}|\] 
%% by setting |\thepdfmoddate|:
\newcommand*{\ReadPDFmodDateOf}[1]{%
    \edef\thepdfmoddate{\thepdfmoddateof{#1}}}
%% % The purpose of this section is to control the reference date 
%% % without changing commands with <file> arguments. 
%% After a single |\UseReferenceDate{<date>}| all ensuing 
%% \[|\CheckDateOfGiven{<file>}|\] compare 
%% % `\theinfodateof{<file>}'
%% <file>'s ``info date"
%% with 
%% % |\thedategiven|. 
%% <date>.
%% % <date> 
%% The latter 
%% may be an explicit
%% \[<4-digits>/<2-digits>/<2-digits>\quad 
%% (`yyyy/mm/dd')\]---a script might insert it---, 
%% `\rawtoday', or `\thepdfmoddate'.
%% \ctanpkgdref{adhocfilelist} v0.7 (with option \strong{\code{-c}}) 
%% is such a script, a shell script generating a ``\TeX\ script", 
%% providing the file modification date according to Unix/Linux.
\newcommand*{\UseReferenceDate}{\def\thedategiven}
\newcommand*{\CheckDateOfGiven}[1]{\CheckDateOf{#1}{\thedategiven}}
%%
%% %%% *** Clarifying Reference Type in Reports ***
%% |\CheckDateOfPDFmod{<file>}| compares the ``info date" 
%% with the modification date according to `\pdffilemoddate', 
%% and in reporting a difference the modification 
%% date it is called a ``modification date" indeed:
\newcommand*{\CheckDateOfPDFmod}[1]{%
    \begingroup
        \ModDates
        \CheckDateOf{#1}{\thepdfmoddate}%
    \endgroup}
%% |\CheckDateOfToday{<file>}| checks if the `\Provides' date is today's, 
%% and the report of a difference somewhat emphasizes that this may not 
%% be a \emph{modification} date. 
%% (It may be a \emph{substitute} for a modification date when you 
%%  know that the file was modified ``today".)
\newcommand*{\CheckDateOfToday}[1]{%
    \begingroup
        \def\FD@refdate{%
            \MessageBreak which is not today}%
        \CheckDateOf{#1}{\rawtoday}%
    \endgroup}
%% 
%% ==== Automatic Immediate Checking         ==== 
%% \label{sec:auto-check}
%% |\FileDateAutoChecks| modifies the \qtdcode{&\Provides...}
%% commands internally so that they check their date with 
%% with the file's modification date according to 
%% |\pdffilemoddate|---this choice might be queried (TODO).
%% The modification may be undone by |\NoFileDateAutoChecks|
%% so that the \qtdcode{&\Provides...} commands get their 
%% original functionality (only) again. This will even work 
%% with \ctanpkgref{readprov} v0.4.
%% 
%% Actually, \LaTeX's internal `\@pr@videpackage' and 
%% `\@profidesfile' are used as hooks. Their original meanings 
%% are stored so they can be regained by \qtd{&\NoFile...}:
\let\FD@@provpkg\@pr@videpackage
\def\FD@provpkg[#1]{\FD@@provpkg[#1]%
                    \CheckDateOfPDFmod{\@currname.\@currext}}
\let\FD@@provfile\@providesfile
\def\FD@provfile#1[#2]{\FD@@provfile{#1}[#2]%
                       \CheckDateOfPDFmod{#1}}
%% |\FileDateAutoChecks*| in addition to `\FileDateAutoChecks'
%% checks the main file's info date, 
%% assuming the main file is `\jobname.tex' 
%% (TODO):                              %% 2013/03/25 cf. \@iinput
\newcommand*{\FileDateAutoChecks}{%
    \@ifstar{{\FD@check@pdf@job\FD@autochecks}\FD@autochecks}%
                                              \FD@autochecks} 
%% |\FD@check@pdf@job| is introduced in v0.41 for the 
%% ``restricted" automatic checks below.
\newcommand*{\FD@check@pdf@job}{\CheckDateOfPDFmod{\jobname.tex}}
\newcommand*{\FD@autochecks}{\let\@pr@videpackage\FD@provpkg
                             \let\@providesfile\FD@provfile}
\newcommand*{\NoFileDateAutoChecks}{\let\@pr@videpackage\FD@@provpkg
                                    \let\@providesfile  \FD@@provfile}
%%
%% However, see \secref{auto-level} for \strong{problems}
%% of the previous idea 
%% (unless using \ctanpkgdref{fileinfo}, i.e., 'readprov' or even 'myfilist').
%%
%% \pagebreak                                       %% 2013/03/25
%% ==== Level-Restricted Automatic Checking  ====   %% 2013/03/25
%% \label{sec:auto-level}
%% The present package is made to ensure ``date consistency" 
%% of files that the user maintains/edits. By contrast, 
%% date consistency cannot be expected for \LaTeX\ package 
%% files that were generated by \ctanpkgdref{docstrip} 
%% (many days after the `.dtx' source was released).
%% No date consistency checks should be applied to such files. 
%% (The ``criticism" of the present section does not apply to 
%%  usage with \ctanpkgref{fileinfo}, and the commands presented 
%%  here \emph{disable} 'fileinfo', so don't use them with 'fileinfo'.)
%% 
%% The problem with `\FileDateAutoChecks' (\secref{auto-check}) is, 
%% e.g., that a chapter file for a book that is `\include'd by 
%% the master file may trigger loading a font definition file 
%% (`.fd')---that is not date-consistent, though the test is 
%% applied to it. We now aim at checking only those files 
%% that are input by the file that contains the call, 
%% not to files that are input indirectly.
%% We use `\InputIfFileExists' as a hook for this. 
%% This should affect \emph{\LaTeX} user commands for file inclusion,
%% even for (the user's) packages and class files. 
%% It does \emph{not} affect the version of `\input' without braces.
%% The commands for this purpose are |\FileDateLevelChecks|, 
%% |\FileDateLevelChecks*|, and |\NoFileDateLevelChecks|
%% (replacing `Auto' by `Level' for analogy to the commands 
%%  of \secref{auto-check}).
%% 
%% The implementation is somewhat recursive, or ``counter-recursive"?
\def\FD@level@provpkg   [#1]{\FD@provpkg[#1]%
                             \NoFileDateLevelChecks}
\def\FD@level@provfile#1[#2]{\FD@provfile{#1}[#2]%
                             \NoFileDateLevelChecks}
%% |\FD@userchecks@| is introduced for \secref{auto-user}:
\newcommand*{\FD@userchecks@}{\let\@pr@videpackage\FD@level@provpkg
                              \let\@providesfile\FD@level@provfile}
\newcommand*{\FD@levelchecks}{\FD@userchecks@
                              \let\InputIfFileExists\FD@input@if@f@ex}
\let\FD@@input@if@f@ex\InputIfFileExists
\newcommand{\FD@input@if@f@ex}[3]{\FD@@input@if@f@ex{#1}{#2}{#3}%
                                  \FD@levelchecks}
\newcommand*{\FileDateLevelChecks}{%
    \@ifstar{\CheckDateOfPDFmod{\jobname.tex}\FD@levelchecks}%
                                             \FD@levelchecks} 
\newcommand*{\NoFileDateLevelChecks}{%
    \NoFileDateAutoChecks \let\InputIfFileExists\FD@@input@if@f@ex}
%% Section \ref{sec:auto-user} introduces another idea 
%% to avoid checking of \LaTeX\ package files.
%%
%% ==== User-Restricted Automatic Checking   ====   %% 2013/03/26
%% \label{sec:auto-user}
%% Another idea to avoid checking of \LaTeX\ package files 
%% while checking `\input' files automatically are 
%% |\FileDateUserChecks| and |\FileDateUserChecks*|
%% only affecting user transclusion commands %%% such as 
%% `\include{<file>}' and `\input{<file>}' while allowing 
%% nesting, i.e., e.g., even an `\input{<indir>}' in an 
%% `\include'd file is checked automatically. 
%% (See \secref{auto-level} if you do not remember what is 
%%  bad with \secref{auto-check}.)
\let\FD@@iinput\@iinput \let\FD@@include\@include
%% Using definitions from above:
\newcommand*{\FD@userchecks}{%
%% `\@iinput' is \LaTeX's internal behind `\input':
    \def\@iinput##1{%
        \FD@userchecks@\FD@@iinput{##1}\NoFileDateAutoChecks}%
%% `\@include' is \LaTeX's internal behind `\include'. 
%% Its argument is limited by a space:
    \def\@include##1 {%
        \FD@userchecks@\FD@@include##1 \NoFileDateAutoChecks}}
%% Temporary switching off `\FD@provpkg' and `\FD@provfile' 
%% is doubled to deal with files that don't have a 
%% `\Provides' entry. 
\newcommand*{\FileDateUserChecks}{%
    \@ifstar{\FD@check@pdf@job\FD@userchecks}\FD@userchecks} 
\newcommand*{\NoFileDateUserChecks}{%
    \let\@iinput\FD@@iinput \let\@include\FD@@include}
%% TODO `\usepackage', `\documentclass'?
%%
%%  === Leaving the Package File ===
\endinput
%% \pagebreak                                       %% 2013/03/25
%%  === VERSION HISTORY ===

v0.1   2012/10/15   core try, bad
v0.2   2012/10/16   code for first release 
       2012/10/17   reordering, correcting documentation
v0.21  2012/10/19   \fd@datekind\@tempb -> \fd@refdate,
                    \fd@refdate with \DatesDiffNotices! (bug)
                    \@tempb -> \fd@therefdate; doc. mod.s
v0.3   2012/10/24   \MessageBreak fix, \DatesDiffWarnings
       2012/10/25   doc.: add <date> in sec:readprov, 
                    rm. remark on \fd@datesequal, 
                    mod. text on \DatesDiffWarnings,
                    -- -> --- before `a script might'
v0.4   2012/11/10   \fd@ -> \FD@, doc. v0.7 -> v0.3, v0.21, 
                    \filbreak; code for \...AutoChecks
       2012/11/11   doc. another \filbreak, documenenting
                    \...AutoChecks, \...AutoChecks*
v0.41  2013/03/24   doc. restructured and extended
       2013/03/25   ... continued; section with \FileDateLevelChecks 
                    etc., \@ifstar without braces; 
                    doc. automatical ("archaic") -> automatic, 
                    reducing numbers of lines
       2013/03/26   \FileDateUserChecks etc., \FD@check@pdf@job


  \pagebreak                                        %% 2013/03/25
\section{Use with Present Package Documentation}
\label{sec:demo-here}
%% mod. 2012/11/06:
\noNiceVerb
 \ProvidesFile{fdatechk.tex}[2013/03/25 `filedate' checks]
\EqualityMessages
\ReadFileInfos{filedate.RLS,srcfiles}
\DoWithBasesExts{\CheckDateOfPDFmod}{filedate}{sty,tex,RLS}
\CheckDateOfPDFmod{srcfiles.tex}
\DatesDiffWarnings
\CheckDateOfToday{filedate.RLS}

\useNiceVerb

\AddQuotes

Above this paragraph,
the documentation source `filedate.tex' issues 
\[|\ProvidesFile{fdatechk.tex}[2013/03/25 `filedate' checks]
\EqualityMessages
\ReadFileInfos{filedate.RLS,srcfiles}
\DoWithBasesExts{\CheckDateOfPDFmod}{filedate}{sty,tex,RLS}
\CheckDateOfPDFmod{srcfiles.tex}
\DatesDiffWarnings
\CheckDateOfToday{filedate.RLS}
|\]
in order to run the following \TeX~script `fdatechk.tex':
\MDsamplecodeinput{fdatechk}
(That is done \emph{above} the paragraph to avoid wrong spacing 
 within the paragraph from `filedate.tex'.)
This way we check whether the ``info dates" of the package file 
`filedate.sty', of the documentation source and driver 
`filedate.tex', and of some other related files 
are the same as their modification dates 
according to |\pdffilemoddate| (using \code{pdflatex}).
When I added the (original) check on 2012-10-17, 
it indeed informed me that I had not updated 
`filedate.tex's info date 
(\code{2012/10/16}, 
 generation of first version of the file from a template, 
 draft).---%%% 2013/03/25
|\EqualityMessages| confirms that the tests were run indeed. 
%
|\DoWithBasesExts| is from the 'filesdo' package 
(\ctanpkgdref{commado} bundle).

The \TeX~script `srcfiles.tex' that in the first instance 
generates a release overview additionally inputs `fdatechk.tex'
(as of 2012-11-06) as well. This way the check is performed 
even when I rerun the documentation without updating the file list, 
as well the other way round.
%% rm. 2012/11/06:
% However, such checks rather 
% approaching package management should better be based on modification 
% \emph{times}. If this should be done by \TeX\ 
% (\CtanPkgRef{pdftex}{\pdfTeX}, \ctanpkgref{pdfcmds}), 
% it should better be based on the \ctanpkgref{filemod} package. 

\end{document}

VERSION HISTORY

2012/10/16  for v0.2    started     %% was v0.1 2012/10/19
2012/10/17              completed
2012/10/19  for v0.21   added srcfiles check, corr. history
2012/10/25  for v0.3    more than two keywords, lmodern
2012/11/06  for r0.3a   final demo with `fdatechk.tex', \AddQuotes 
                        for \qtd{\file{..., and other mod.s of 
                        doc. there; title extended
2012/11/11  for v0.4    v0.4 in abstract
2013/03/24  for v0.41   deeper sectioning level, \secref, 
                        \MDsamplecodeinput (+ `code'),
                        \subsubsections for "Basic Usage"
2013/03/25  for v0.41   \usepackage{filesdo}, \label{sec:demo-here}, 
                        mentioning `filesdo' there; 
                        folding and page breaks
