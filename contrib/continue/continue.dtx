% \iffalse meta-comment
%
% continue.dtx
%
%  Author: Peter Wilson (Herries Press) herries dot press at earthlink dot net
%  Copyright 2015 Peter R. Wilson, Donald Arseneau and Merciadri Luca
%
%  This work may be distributed and/or modified under the
%  conditions of the Latex Project Public License, either
%  version 1.3 of this license or (at your option) any
%  later version.
%  The latest version of the license is in
%    http://www.latex-project.org/lppl.txt
%  and version 1.3 or later is part of all distributions of
%  LaTeX version 2003/06/01 or later.
%
%  This work has the LPPL maintenance status "author-maintained".
%
%  This work consists of the files listed in the README file.
%
%<*driver>
\documentclass{ltxdoc}
\EnableCrossrefs
\CodelineIndex
\setcounter{StandardModuleDepth}{1}
%%%\usepackage{url}
%%%\usepackage[draft=false,
%%%            plainpages=false,
%%%            pdfpagelabels,
%%%            bookmarksnumbered,
%%%            hyperindex=false
%%%           ]{hyperref}
 \begin{document}
   \DocInput{continue.dtx}
 \end{document}
%</driver>
%
% \fi
%
% \CheckSum{12458}
%
% \DoNotIndex{\',\.,\@M,\@@input,\@addtoreset,\@arabic,\@badmath}
% \DoNotIndex{\@centercr,\@cite}
% \DoNotIndex{\@dotsep,\@empty,\@float,\@gobble,\@gobbletwo,\@ignoretrue}
% \DoNotIndex{\@input,\@ixpt,\@m}
% \DoNotIndex{\@minus,\@mkboth,\@ne,\@nil,\@nomath,\@plus,\@set@topoint}
% \DoNotIndex{\@tempboxa,\@tempcnta,\@tempdima,\@tempdimb}
% \DoNotIndex{\@tempswafalse,\@tempswatrue,\@viipt,\@viiipt,\@vipt}
% \DoNotIndex{\@vpt,\@warning,\@xiipt,\@xipt,\@xivpt,\@xpt,\@xviipt}
% \DoNotIndex{\@xxpt,\@xxvpt,\\,\ ,\addpenalty,\addtolength,\addvspace}
% \DoNotIndex{\advance,\Alph,\alph}
% \DoNotIndex{\arabic,\ast,\begin,\begingroup,\bfseries,\bgroup,\box}
% \DoNotIndex{\bullet}
% \DoNotIndex{\cdot,\cite,\CodelineIndex,\cr,\day,\DeclareOption}
% \DoNotIndex{\def,\DisableCrossrefs,\divide,\DocInput,\documentclass}
% \DoNotIndex{\DoNotIndex,\egroup,\ifdim,\else,\fi,\em,\endtrivlist}
% \DoNotIndex{\EnableCrossrefs,\end,\end@dblfloat,\end@float,\endgroup}
% \DoNotIndex{\endlist,\everycr,\everypar,\ExecuteOptions,\expandafter}
% \DoNotIndex{\fbox}
% \DoNotIndex{\filedate,\filename,\fileversion,\fontsize,\framebox,\gdef}
% \DoNotIndex{\global,\halign,\hangindent,\hbox,\hfil,\hfill,\hrule}
% \DoNotIndex{\hsize,\hskip,\hspace,\hss,\if@tempswa,\ifcase,\or,\fi,\fi}
% \DoNotIndex{\ifhmode,\ifvmode,\ifnum,\iftrue,\ifx,\fi,\fi,\fi,\fi,\fi}
% \DoNotIndex{\input}
% \DoNotIndex{\jobname,\kern,\leavevmode,\let,\leftmark}
% \DoNotIndex{\list,\llap,\long,\m@ne,\m@th,\mark,\markboth,\markright}
% \DoNotIndex{\month,\newcommand,\newcounter,\newenvironment}
% \DoNotIndex{\NeedsTeXFormat,\newdimen}
% \DoNotIndex{\newlength,\newpage,\nobreak,\noindent,\null,\number}
% \DoNotIndex{\numberline,\OldMakeindex,\OnlyDescription,\p@}
% \DoNotIndex{\pagestyle,\par,\paragraph,\paragraphmark,\parfillskip}
% \DoNotIndex{\penalty,\PrintChanges,\PrintIndex,\ProcessOptions}
% \DoNotIndex{\protect,\ProvidesClass,\raggedbottom,\raggedright}
% \DoNotIndex{\refstepcounter,\relax,\renewcommand,\reset@font}
% \DoNotIndex{\rightmargin,\rightmark,\rightskip,\rlap,\rmfamily,\roman}
% \DoNotIndex{\roman,\secdef,\selectfont,\setbox,\setcounter,\setlength}
% \DoNotIndex{\settowidth,\sfcode,\skip,\sloppy,\slshape,\space}
% \DoNotIndex{\symbol,\the,\trivlist,\typeout,\tw@,\undefined,\uppercase}
% \DoNotIndex{\usecounter,\usefont,\usepackage,\vfil,\vfill,\viiipt}
% \DoNotIndex{\viipt,\vipt,\vskip,\vspace}
% \DoNotIndex{\wd,\xiipt,\year,\z@}
% \DoNotIndex{\0,\1,\2,\3,\4,\5,\6,\7,\8,\9}
%
% \def\dtxfile{continue.dtx}
% \changes{v0.1}{2015/12/09}{First trial release}
% \def\fileversion{v0.1} \def\filedate{2015/12/09}

% ^^A Flag an option
% \makeatletter
% \newcommand{\OptFont}{\fontencoding\encodingdefault
%   \fontfamily\sfdefault
%   \fontseries\mddefault
%   \fontshape\updefault
%   \small}
% 
% \newcommand{\DescribeOption}{\leavevmode\@bsphack
%   \begingroup\MakePrivateLetters\Describe@Opt}
% \newcommand{\Describe@Opt}[1]{\endgroup
%   \marginpar{\raggedleft\PrintDescribeOption{#1}}%
%   \SpecialOptIndex{#1}\@esphack\ignorespaces}
%   \newcommand*{\PrintDescribeOption}[1]{\strut \OptFont #1\ }
%   \newcommand*{\PrintOptionName}[1]{\strut \OptFont #1\ }
% \newcommand*{\SpecialOptIndex}[1]{\@bsphack
%   \index{#1\actualchar{\protect\sffamily#1}
%     (option)\encapchar usage}%
%   \index{options:\levelchar#1\actualchar
%     {\protect\sffamily#1}\encapchar usage}\@esphack}
% 
% \makeatother
% 
% \providecommand{\phantomsection}{}
% %\OnlyDescription ^^A comment this out for the full glory
% \setcounter{StandardModuleDepth}{1}
% \makeatletter
%   \@mparswitchfalse
% \makeatother
% \renewcommand{\MakeUppercase}[1]{#1}
% \pagestyle{headings}
% \newenvironment{addtomargins}[1]{%
%   \begin{list}{}{%
%     \topsep 0pt%
%     \addtolength{\leftmargin}{#1}%
%     \addtolength{\rightmargin}{#1}%
%     \listparindent \parindent
%     \itemindent \parindent
%     \parsep \parskip}%
%   \item[]}{\end{list}}
% \MakeShortVerb{\|}
%
% \newcommand*{\Lpack}[1]{\textsf {#1}}           ^^A typeset a package
% \newcommand*{\Lopt}[1]{\textsf {#1}}            ^^A typeset an option
% \newcommand*{\file}[1]{\texttt {#1}}            ^^A typeset a file
% \newcommand*{\Lcount}[1]{\textsl {\small#1}}    ^^A typeset a counter
% \newcommand*{\pstyle}[1]{\textsl {#1}}          ^^A typeset a pagestyle
% \newcommand*{\Lenv}[1]{\texttt {#1}}            ^^A typeset an environment
% \newcommand{\BC}{\textsc{bc}}
% \newcommand{\AD}{\textsc{ad}}
%
% \title{\Lpack{Continue}: Continuation marks on recto (odd) pages\thanks{This
%        file (\texttt{\dtxfile}) has version number \fileversion, last revised
%        \filedate.}}
%
% \author{%
% Peter Wilson\thanks{\texttt{herries dot press at earthlink dot net}} with 
% Donald Arseneau and Merciadri Luca \\
% Herries Press
% }
% \date{\filedate}
% \maketitle
% \begin{abstract}
%    The \Lpack{continue} package provides for a variety of continuation indicators on
% recto (odd numbered) pages in a twosided document when the text continues on the following
% (verso) page.  
% \end{abstract}
% \tableofcontents
%
% 
%
% \section{Introduction}
%
%    In some types of documents it is customary or convenient to indicate at the
% bottom of a page that the text continues on the following page. For example, when
% a two page exam sheet is printed twosided it could be advantageous for the student
% to have an indication on the first page that there are further questions on the back 
% of the page. Another instance is when documents are printed in the expectation
% that they will be read by someone in front of an audience, so in order to minimise
% any hesitation as a page is turned over, the first word on the following page is printed
% at the bottom of the preceeding page.
%
% This manual is typeset according to the conventions of the
% \LaTeX{} \textsc{docstrip} utility which enables the automatic
% extraction of the \LaTeX{} macro source files~\cite{GOOSSENS94}.
%
%    Section~\ref{sec:usc} describes the usage of the package.
% Commented package code may be in later sections.
%
%
% \section{The \Lpack{continue} package} \label{sec:usc}
%
%    The \Lpack{continue} package is an amalgam of modified versions of
% two packages --- \Lpack{fwlw} (First Word, Last Word) by Donald 
% Arseneau~\cite{FWLW} which% among other things gets the first word
% on the next page and \Lpack{turnpageetex} by Merciadri Luca~\cite{TURNPAGEETEX}
% for placing something at the bottom of the text block. 
%
% The \Lpack{fwlw} package includes the following statement:
%
%     Copyright (C) 1993,1995 by Donald Arseneau
%     Vancouver, Canada, email asnd@triumf.ca
%     This software package may be freely used, transmitted, reproduced,
%     or modified provided that this notice is left intact.
%
% In this instance I have modified the original package as allowed above.
%
%    With respect to the \Lpack{turnpageetex} package this was released under the
% LaTeX Project Public License and I have taken the liberty of extending its functionality
% slightly, as allowed.
%
%
%
% \subsection{Options}
%
%     There are two options that can be used when calling the package denoting the
% kind and position of the continuation marker.
%
% \DescribeOption{margin}
%    With this option the continuation marker is placed in the margin aligned with
% the bottom of the text block. If the option is not used then the continuation
% marker is placed below the text text block ending at the outer margin.
%
% \DescribeOption{word}
%    With this option the package attempts to use the first `word' on the following
% verso page as the continuation marker. If the option is not used then a user
% specified marker is employed.
%
%  The relevant portions of the introduction to the \Lpack{fwlw} package are: \\
% \begin{quotation}
%  The \Lpack{fwlw} package provides a mechanism to determine ...
%  the first word on the \textit{next} page.  The `words'
%  you see may not be real words, but any unbreakable object.
%
% ...
%
%  ... labelling does not make much sense when |\chapter| generates a page
%  break, so the last page before a |\chapter| (or any |\clearpage|) gets 
%  a blank "next word" ...
%
%  Note that `words' may unfortunately be things like:
% \begin{itemize}
%    \item two~words
%    \item |[ ]|Word  ( |[ ]| represents a parindent box)
%    \item a whole displayed equation
%    \item the first column of an aligned equation
%    \item anomalously blank, if there are |\write|s or split footnotes etc.
%    \item partial words like  par-  or  -tial due to hyphenation.
% \end{itemize}
% \end{quotation}
%
%    In essence, the `word' might not be what you might expect, but for most documents
% the results are good.
%
%
%
% \subsection{Macros}
%
% When the \Lopt{word} option is not used the following macros are available:
%
% \DescribeMacro{\flagcont}
%    This command defines the continuation mark. Its default specification is: \\
%    |\newcommand*{\flagcont}{Continued}| \\
% It can be changed in the document's preamble to, perhaps: \\
%    |\renewcommand*{\flagcont}{---$>$}| \\
% to print a right pointing arrow.
%
% \DescribeMacro{\flagend}
%    This command defines the continuation mark when the last odd page is the final
% page (i.e. when there is no following recto page). Its default specification is: \\
%    |\newcommand*{\flagend}{End}| \\
% but it can be changed in the document's preamble.
%
% When the \Lopt{word} option is used the following macros are available:
%
% \DescribeMacro{\flagword}
%     This command specifies how the continuation word is formated. Its definition is: \\
%    |\newcommand*{\flagword}{\preflagword\usebox\NextWordBoxC\postflagword}| \\
% where |\NextWordBoxC| holds the first word on the next recto page (empty if there
% is no next recto page).
%
% \DescribeMacro{\preflagword}
%    This command is printed before the word continuation marker. Its default definition is: \\
%    |\newcommand*{\preflagword}{}| \\
% but can be redefined in the document preamble to put, perhaps, an opening brace
% before the word like: \\
%    |\renewcommand*{\preflagword}{[}|
%
% \DescribeMacro{\postflagword}
%    Similar to |\preflagword| except that it is printed after the marker. Its default 
%  definition is: \\
%    |\newcommand*{\postflagword}{}| \\
%
% \DescribeMacro{\contsep}
% \DescribeMacro{\contdrop}
%     When the \Lopt{margin} option is used the continuation marker is set a distance
% |\contsep| to the right of the textblock. When \Lopt{margin} is not used then the
% marker is set a distance |\contdrop| below the textblock. Their default values
% are set by: \\
% |\setlength{\contsep}{\marginparsep}| and |\setlength{\contdrop}{0.5\footskip}|
%
% \section{An example document}
%
% You can use the document |trycontinue.tex| to experiment with the options.
% It may take two (pdf)LaTeX runs for the continuation marks to settle down to their
% final values and positions.
%
%    \begin{macrocode}
%<*try>
%% trycontinue.tex  An example usage of the continue package
%%
%% The document is set in two columns with footnotes on A6 paper (I wanted
%% to get many pages from little text). It is not pretty but does shows 
%% some of the continue package's capabilities. Try modifying it by changing 
%% options, etc., and see what happens.
%% 
%% Please contact me, Peter Wilson at herries.press@earthlink.net, if there
%% are problems other than the aesthetics.
%%
\documentclass[%
  twoside,
  twocolumn,
  a6paper
  ]%
  {memoir}

\usepackage%
%%  [margin]%
%%  [word]%
  [margin,word]%
  {continue}
\usepackage{lipsum}
%%  try this without the word option
%%\renewcommand{\flagcont}{---$>$}

\renewcommand*{\preflagword}{[}

\begin{document}

\newcommand{\Footnote}[1]{}
\let\Footnote\footnote

First\Footnote{Foot 1} \lipsum[1]

Second\Footnote{Foot 2} \lipsum[2]

Third\Footnote{Foot 3} \lipsum[3]

Fourth\Footnote{Foot 4} \lipsum[4]

Fifth\Footnote{Foot 5} \lipsum[5]

Sixth\Footnote{Foot 6} \lipsum[6]

\end{document}

%</try>
%    \end{macrocode}
%
% \section{The \Lpack{continue} code} \label{sec:code}
%
%    Announce the name and version of the package, which requires
% \LaTeXe{} and the \Lpack{atbegshi}, \Lpack{picture}, \Lpack{zref-abspage} and
% \Lpack{zref-lastpage} packages and has options \Lopt{margin} and \Lopt{word}.
%    \begin{macrocode}
%<*usc>
 \NeedsTeXFormat{LaTeX2e}
 \ProvidesPackage{continue}[2015/12/04 v0.1 Continues on the following page]
 \PackageInfo{continue}{This is continue using e-TeX.}

%    \end{macrocode}
% \begin{macro}{\ifcontmargin}
% \begin{macro}{\ifcontword}
% Now for the options.
%    \begin{macrocode}
 \newif{\ifcontmargin} \contmarginfalse
 \newif{\ifcontword}   \contwordfalse
 \DeclareOption{margin}{\contmargintrue}
 \DeclareOption{word}{\contwordtrue}
 \ProcessOptions

%    \end{macrocode}
% \end{macro}
% \end{macro}
%
% And the required packages needed by the original \Lpack{turnpageetex} package.
%
%    \begin{macrocode}
 \RequirePackage{atbegshi}
 \RequirePackage{picture}
 \RequirePackage{zref-abspage}
 \RequirePackage{zref-lastpage}

%    \end{macrocode}
%
% The next chunk of code is a revised version of the \Lpack{fwlw} package. I don't
% really understand it but I have a feeling that certain parts are irrelevant
% to the purposes at hand.
%
%    \begin{macrocode}
% --------------------------------------------------------------------------
 \ifcontword
%
%    \end{macrocode}
%
% \begin{macro}{\LWC@pen}
% Declare a \textit{unique} penalty flag as a value.
%    \begin{macrocode}
 \mathchardef\LWC@pen 13452
%    \end{macrocode} 
% \end{macro}
%
% \begin{macro}{\FirstWordBoxC}
% \begin{macro}{\NextWordBoxC}
% \begin{macro}{\LastWordBoxC}
% \begin{macro}{\LWC@box}
% \begin{macro}{\LWC@saved}
% Allocate box registers
%    \begin{macrocode}
 \newbox\FirstWordBoxC     \global\setbox\FirstWordBoxC\hbox{}
 \newbox\NextWordBoxC      \global\setbox\NextWordBoxC\hbox{}
 \newbox\LastWordBoxC      \global\setbox\LastWordBoxC\hbox{}
 \newbox\LWC@box           \global\setbox\LWC@box\hbox{}
 \newbox\LWC@saved
%    \end{macrocode} 
% \end{macro}
% \end{macro}
% \end{macro}
% \end{macro}
% \end{macro}
%
% \begin{macro}{\FWLWCnormal@L@output}
%  Shell around old output routine.  Gets first word from next page by
%  letting TeX continue with |\vsize=0| to get a look at the next line.
%  Values of |\outputpenalty| for |\specialoutput| ( -10001 to -19999 ) are
%  simply run through the output routine.
%  |\supereject| and |\clearpage| give a blank "next word".
%  When called after making a stub-page the stub is returned to the
%  vertical list, the previous page is restored and shipped out normally,
%  but knowing what the next word will be.
%    \begin{macrocode}
 \edef\FWLWCnorm@L@output{\the\output}

 \output{\@tempswafalse
 \ifnum \outputpenalty>-\@MM \ifnum\outputpenalty<-\@M \@tempswatrue\fi\fi
 \if@tempswa % special (float) output
%  \message{Float handler: penalty=\the\outputpenalty}%
    \FWLWCnorm@L@output
 \else
   \ifvoid\LWC@saved % end of real page
%    \message{End of real page}%
      \global\setbox\LWC@saved\copy\@cclv % save page
%\begin{comment}
      \setbox\@tempboxa\vbox{\unvbox\@cclv \unskip\unkern\unpenalty%     
       \unskip\unkern\unpenalty \unskip\unkern\unpenalty
        \setbox\@tempboxa\lastbox
        \LWC@getlast@word\@tempboxa\LastWordBoxC
      }
%\end{comment}
      \ifnum\outputpenalty>-\@MM % not \supereject
        \xdef\LWC@vsize{\global\vsize\the\vsize 
           \global\holdinginserts\the\holdinginserts}%
        \global\vsize\z@ \global\holdinginserts\@ne 
      \else % \supereject, just output, don't look for word on next page
%      \message{caused by super-eject.}
        \global\setbox\@cclv\box\LWC@saved
        \global\setbox\NextWordBoxC\hbox{}%
        \FWLWCnorm@L@output
        \global\setbox\FirstWordBoxC\box\NextWordBoxC
      \fi
   \else % saved page => just did tiny page to get next word
%    \message{Just got next line:}{\tracingall\showboxdepth2 \showbox\@cclv}%
      \setbox\@tempboxa\vbox{\penalty\LWC@pen\unvcopy\@cclv \LWC@getall@boxes
        \ifvbox\LWC@box \penalty\LWC@pen\unvbox\LWC@box \LWC@getall@boxes\fi
        \ifvbox\LWC@box \global\setbox\NextWordBoxC\hbox{}\else
          \LWC@getfirst@word\LWC@box\NextWordBoxC
        \fi}%  Return tiny page to page list:
      \unvbox\@cclv \ifnum\outputpenalty<\@M \penalty\outputpenalty\fi
      \LWC@vsize\relax
      \global\setbox\@cclv\box\LWC@saved
      \FWLWCnorm@L@output
      \global\setbox\FirstWordBoxC\box\NextWordBoxC
 \fi\fi}

%    \end{macrocode}
% \end{macro}
%
% \begin{macro}{\LWC@getlast@word}
% Globally get last "word" from a box |#1| into another box |#2|
%
%    \begin{macrocode}
 \def\LWC@getlast@word#1#2{\setbox\@tempboxa\vbox{\hsize\maxdimen \@parboxrestore
   \hyphenpenalty\@M \exhyphenpenalty\@M 
   \rightskip\fill \looseness\@M \linepenalty\z@
   \noindent\unhbox#1\endgraf
   \unskip\unkern\unpenalty \global\setbox\LWC@box\lastbox
   }\LWC@repack{#2}}

%    \end{macrocode}
% \end{macro}
% \begin{macro}{\LWC@getfirst@word}
% Globally get first "word" from a box |#1| into another box |#2|
%
%    \begin{macrocode}
 \def\LWC@getfirst@word#1#2{\setbox\@tempboxa\vbox{\@parboxrestore
   \parshape\thr@@ \z@\z@ \z@\z@ \z@\maxdimen \parfillskip\fill
   \hyphenpenalty\@M \exhyphenpenalty\@M 
   \hbadness\@MM \overfullrule\z@ \hfuzz\maxdimen
   \ifhbox#1\noindent
    \vadjust{\penalty\LWC@pen}\penalty-\@M\unhbox#1% eliminate previous \leftskip
   \else\ifvbox#1\penalty\LWC@pen\unvbox#1\fi\fi
   \endgraf
   \@tempcnta\z@  \LWC@getall@boxes}\LWC@repack{#2}}

%    \end{macrocode}
% \end{macro}
%
% \begin{macro}{\LWC@getall@boxes}
% Go through a vertical list that starts with special penalty
%
%    \begin{macrocode}
 \def\LWC@getall@boxes{\global\setbox\LWC@box\lastbox \unskip\unkern \unskip\unkern
   \let\@tempa\relax
   \ifvoid \LWC@box \advance\@tempcnta\@ne \else \@tempcnta\z@ \fi
   \ifnum\lastpenalty=\LWC@pen \else \unpenalty\fi
   \ifnum\lastpenalty=\LWC@pen \else \unpenalty\fi
   \ifnum\lastpenalty=\LWC@pen \else \ifnum\@tempcnta<5
     \let\@tempa\LWC@getall@boxes \fi \fi \unpenalty \@tempa}

%    \end{macrocode}
% \end{macro}
%
% \begin{macro}{\LWC@repack}
% Put contents of |\LWC@box| into hbox |#1|
%
%    \begin{macrocode}
 \def\LWC@repack#1{\global\setbox#1\hbox{\ifhbox\LWC@box
    \unhbox\LWC@box\unskip\unskip\unpenalty\unskip
   \else\ifvbox\LWC@box\box\LWC@box\fi\fi}}
%    \end{macrocode} 
% \end{macro}
% Now finish with the fwlw code when it is not needed.
%    \begin{macrocode}
 \fi
%    \end{macrocode}
%
% The next chunk of code is a revised version of the original \Lpack{turnpageetex} package.
%    \begin{macrocode}
% --------------------------------------------------------------------------
%
%    \end{macrocode}
%
% \begin{macro}{\preflagword}
% \begin{macro}{\postflagword}
% \begin{macro}{\flagcont}
% \begin{macro}{\flagend}
% \begin{macro}{\flagword}
% \begin{macro}{\contsep}
% \begin{macro}{\contdrop}
% The user's macros
%
%    \begin{macrocode}

 \newcommand*{\preflagword}{}
 \newcommand*{\postflagword}{}
 \newcommand*{\flagcont}{Continued}
 \newcommand*{\flagend}{End}
 \newcommand*{\flagword}{\preflagword\usebox\NextWordBoxC\postflagword}
 \newlength{\contsep}\setlength{\contsep}{\marginparsep}
 \newlength{\contdrop}\setlength{\contdrop}{0.5\footskip}
%    \end{macrocode}
% Adjust |\flagcont| when the \Lopt{word} option is used.
%    \begin{macrocode}
 \ifcontword
  \let\flagcont\flagword
 \fi
%    \end{macrocode}
%
% \end{macro}
% \end{macro}
% \end{macro}
% \end{macro}
% \end{macro}
% \end{macro}
% \end{macro}
%
% 
%    \begin{macrocode}
 \AtBeginShipout{%
 \AtBeginShipoutUpperLeft{%
%    \end{macrocode}
%
% Specify the location of the continuation marks.
%
%    \begin{macrocode}
     \ifodd\c@page
         \dimen1=1in
         \advance\dimen1 by \textwidth
         \advance\dimen1 by \oddsidemargin
         \dimen3=1in
         \advance\dimen3 by \topmargin
         \advance\dimen3 by \headheight
         \advance\dimen3 by \headsep
         \advance\dimen3 by \textheight
         \ifcontmargin
           \advance\dimen1 by \contsep
         \else
           \advance\dimen3 by \contdrop
         \fi
%    \end{macrocode}
% Otput the page and marks.
%    \begin{macrocode}
      \ifnum\zref@extract{LastPage}{abspage}>\c@abspage
        \ifcontmargin
          \put(\dimen1,-\dimen3){{\flagcont}}%
        \else
          \put(\dimen1,-\dimen3){\llap{\flagcont}}%
        \fi
      \else
        \ifcontmargin
          \put(\dimen1,-\dimen3){{\flagend}}%
        \else
          \put(\dimen1,-\dimen3){\llap{\flagend}}%
        \fi
      \fi%
    \fi%
}}
%
%</usc>
%    \end{macrocode}
%
%
% \bibliographystyle{alpha}
% \begin{thebibliography}{GMS94}
%
% \bibitem[Ars95]{FWLW}
% Donald Arseneau.
% \newblock \emph{fwlw.sty (First Word, Last Word)}.
% \newblock 1995. (Available from CTAN in \texttt{macros/latex/contrib/fwlw}).
%
% \bibitem[GMS94]{GOOSSENS94}
% Michel Goossens, Frank Mittelbach, and Alexander Samarin.
% \newblock \emph{The LaTeX Companion}.
% \newblock Addison-Wesley Publishing Company, 1994.
%
% \bibitem[Luc15]{TURNPAGEETEX}
% Merciadri Luca.
% \newblock \emph{turnpageetex.sty}.
% \newblock 2011. (Available from CTAN in \texttt{macros/latex/contrib/turnthepage}).
%
% \end{thebibliography}
%
% \PrintIndex
%     
% \Finale
%
% \endinput



%% \CharacterTable
%%  {Upper-case    \A\B\C\D\E\F\G\H\I\J\K\L\M\N\O\P\Q\R\S\T\U\V\W\X\Y\Z
%%   Lower-case    \a\b\c\d\e\f\g\h\i\j\k\l\m\n\o\p\q\r\s\t\u\v\w\x\y\z
%%   Digits        \0\1\2\3\4\5\6\7\8\9
%%   Exclamation   \!     Double quote  \"     Hash (number) \#
%%   Dollar        \$     Percent       \%     Ampersand     \&
%%   Acute accent  \'     Left paren    \(     Right paren   \)
%%   Asterisk      \*     Plus          \+     Comma         \,
%%   Minus         \-     Point         \.     Solidus       \/
%%   Colon         \:     Semicolon     \;     Less than     \<
%%   Equals        \=     Greater than  \>     Question mark \?
%%   Commercial at \@     Left bracket  \[     Backslash     \\
%%   Right bracket \]     Circumflex    \^     Underscore    \_
%%   Grave accent  \`     Left brace    \{     Vertical bar  \|
%%   Right brace   \}     Tilde         \~}
