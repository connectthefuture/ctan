%Format: pdfLaTeX

\documentclass[12pt,a4paper,german,french]{article}

\usepackage{lettrine}
\usepackage[utf8]{inputenc}

\usepackage{Typocaps} % Voir dernière lettrine p. 2

%%% Palatino fonts...
\usepackage[T1]{fontenc}
\usepackage{palatino}
%%% ... or EC fonts ...
%\usepackage[T1]{fontenc}
%\usepackage{type1ec}
%%% ... or LM fonts
%\usepackage[T1]{fontenc}
%\usepackage{lmodern}
%%% ... or CM fonts
%\usepackage{type1cm}

\usepackage{babel}
\frenchbsetup{AutoSpacePunctuation=false}

\usepackage{graphicx,color}

\newcommand{\MF}{{\small\sffamily\scshape metafont}}
\newcommand{\MP}{{\small\sffamily\scshape metapost}}

\setlength{\textheight}{680pt}
\setlength{\textwidth}{412pt}
\setlength{\oddsidemargin}{20pt}
\setlength{\topmargin}{-35pt}

\setlength{\parindent}{0pt}
%\addtolength{\parskip}{0.5\baseineskip}
\sloppy

\begin{document}
\thispagestyle{empty}

\begin{center}
\large\bfseries Quelques exemples de lettrines
\end{center}

\vspace{\baselineskip}
\textit{Usage standard (2 lignes) :}\\
\verb+\lettrine{E}{n} plein marais...+

\lettrine{E}{n} plein marais de la Souteyranne, à quelques kilomètres
au nord d'Aigues-Mortes, se trouve la Tour Carbonnière.
Construite au XIII\ieme~siècle, elle contrôlait l'unique voie d'accès
terrestre de la ville fortifiée, celle qui menait à Psalmody,
l'une des « abbayes de sel» dont il ne reste que quelques vestiges.

\vspace{\baselineskip}
\textit{Lettrine sur une seule ligne (option {\ttfamily\upshape lines=1}) :}\\
\verb+\lettrine[lines=1]{E}{n} plein marais...+

\lettrine[lines=1]{E}{n} plein marais de la Souteyranne,
à quelques kilomètres au nord d'Aigues-Mortes, se trouve
la Tour Carbonnière.

\vspace{\baselineskip}
\textit{Lettrine sur trois lignes (option {\ttfamily\upshape lines=3}) :}\\
\verb+\lettrine[lines=3]{E}{n} plein marais...+

\lettrine[lines=3]{E}{n} plein marais de la Souteyranne,
à quelques kilomètres au nord d'Aigues-Mortes,
se trouve la Tour Carbonnière.
Construite au XIII\ieme~siècle, elle contrôlait l'unique voie d'accès
terrestre de la ville fortifiée, celle qui menait à Psalmody,
l'une des « abbayes de sel» dont il ne reste que quelques vestiges.

\vspace{\baselineskip}
\textit{Lettrine complètement dans la marge} :\\
\verb+\lettrine[lhang=1, nindent=0pt, lines=3]{J}{ustement},...+

\lettrine[lhang=1, nindent=0pt, lines=3]{J}{ustement},
à quelques kilomètres au nord d'Aigues-Mortes,
se trouve la Tour Carbonnière.
Construite au XIII\ieme~siècle, elle contrôlait l'unique voie d'accès
terrestre de la ville fortifiée, celle qui menait à Psalmody,
l'une des « abbayes de sel» dont il ne reste que quelques vestiges.
L'abbaye était ravitaillée ---~dit-on ~--- par un souterrain
qui la reliait au château de Treillan.

\vspace{\baselineskip}
\textit{Lettrine en saillie à la fois en hauteur et dans la marge} :\\
\verb+\lettrine[lines=3, lhang=0.33, loversize=0.25]{E}{n} ...+

\lettrine[lines=3, lhang=0.33, loversize=0.25]{E}{n}
plein marais de la Souteyranne,
à quelques kilomètres au nord d'Aigues-Mortes la Tour Carbonnière.
Construite au XIII\ieme~siècle, elle contrôlait l'unique voie d'accès
terrestre de la ville fortifiée, celle qui menait à Psalmody,
l'une des « abbayes de sel» \dots
% dont il ne reste que des vestiges.

\vspace{\baselineskip}
\textit{On peut ajouter un guillemet devant la lettrine} :\\
\verb+\lettrine[ante=«]{E}{n} plein marais ...+

\lettrine[ante=«]{E}{n} plein marais de la Souteyranne,
à quelques kilomètres au nord d'Aigues-Mortes, se trouve
la Tour Carbonnière.
Construite au XIII\ieme~siècle, elle contrôlait l'unique voie d'accès
terrestre de la ville fortifiée, celle qui menait à Psalmody,
l'une des « abbayes de sel» \dots
% dont il ne reste que des vestiges.

\newpage
Toutes Les lettrines suivantes  seront en gris jusqu'à nouvel ordre : \\
\verb+\renewcommand{\LettrineFontHook}{\color[gray]{0.5}}+
\renewcommand{\LettrineFontHook}{\color[gray]{0.5}}

\vspace{.5\baselineskip}
\textit{On diminue la taille de la lettrine de 10\% et on
la remonte de 10\%  à cause du « Q »}
\verb+\lettrine[lines=4, loversize=-0.1, lraise=0.1]{Q}{u'en...+

\lettrine[lines=4, loversize=-0.1, lraise=0.1]{Q}{u'en plein marais}
 de la Souteyranne, à quelques kilomètres au nord d'Aigues-Mortes,
se trouve la Tour Carbonnière, surprend les visiteurs.
Construite au XIII\ieme~siècle, elle contrôlait l'unique voie d'accès
terrestre de la ville fortifiée, celle qui menait à Psalmody,
l'une des « abbayes de sel» dont il ne reste que quelques vestiges.
L'abbaye était ravitaillée par un souterrain qui
la reliait au château de Treillan.

\vspace{.5\baselineskip}
\textit{Autre solution : ajout d’une ligne grâce à} \verb+depth+\\
\verb+\lettrine[lines=4,depth=1]{Q}{u'en plein marais} de ...+

\lettrine[lines=4,depth=1]{Q}{u'en plein marais} de la Souteyranne,
à quelques kilomètres au nord d'Aigues-Mortes,
se trouve la Tour Carbonnière, surprend les visiteurs.
Construite au XIII\ieme~siècle, elle contrôlait l'unique voie d'accès
terrestre de la ville fortifiée, celle qui menait à Psalmody,
l'une des « abbayes de sel» dont il ne reste que quelques vestiges.
L'abbaye était ravitaillée par un souterrain qui
la reliait au château de Treillan.

\vspace{.5\baselineskip}
\textit{Utilisation de l'option {\ttfamily\upshape slope}} pour que le texte
suive la pente du « À » :\\
\verb+\lettrine[lines=4, slope=0.6em, findent=-1em,+\\
\verb+          nindent=0.6em]{À}{quelques kilomètres}...+

\lettrine[lines=4, slope=0.6em, findent=-1em, nindent=0.6em]{À} {quelques
kilomètres} au nord d'Aigues-Mortes, se trouve  la Tour Carbonnière.
Construite au XIII\ieme~siècle, elle contrôlait l'unique voie d'accès
terrestre de la ville fortifiée, celle qui menait à Psalmody,
l'une des « abbayes de sel» dont il ne reste que quelques vestiges.
L'abbaye était ravitaillée ---~dit-on~--- par un souterrain qui
la reliait au château de Treillan.

\vspace{.5\baselineskip}
\textit{Utilisation de l'option {\ttfamily\upshape slope} pour que le texte
 suive la pente du {\ttfamily\upshape V}, noter que celui-ci est à
 demi-poussé dans la marge par l'option {\ttfamily\upshape lhang=0.5} :}\\
\verb+\lettrine[lines=4, slope=-0.5em, lhang=0.5, nindent=0pt]+\\
\verb+  {V}{oici} à...+

\lettrine[lines=4, slope=-0.5em, lhang=0.5, nindent=0pt]{V}{oici}
à quelques kilomètres au nord d'Aigues-Mortes la Tour Carbonnière.
Construite au XIII\ieme~siècle, elle contrôlait l'unique voie d'accès
terrestre de la ville fortifiée, celle qui menait à Psalmody,
l'une des « abbayes de sel» dont il ne reste que quelques vestiges.
L'abbaye était ravitaillée ---~dit-on~--- par un souterrain qui
la reliait au château de Treillan.

\vspace{.5\baselineskip}
\textit{Changement de police (ici Typocaps)
  et de couleur pour la lettrine :}\\
\verb+\renewcommand{\LettrineFontHook}{\Typocapsfamily\color{black}}+\\
\verb+\lettrine[findent=.3em]{E}{n} plein marais...+

{% Groupe (changement local de fonte)
\renewcommand{\LettrineFontHook}{\Typocapsfamily\color{black}}

\lettrine[findent=.3em]{E}{n} plein marais de la Souteyranne, à quelques
kilomètres au nord d'Aigues-Mortes, se trouve la Tour Carbonnière.
Construite au XIII\ieme~siècle, elle contrôlait l'unique voie d'accès
terrestre de la ville fortifiée.
\par}% Il est indispensable de terminer le paragraphe avant la fin du groupe.

\newpage
\begin{center}
\large\bfseries Utilisation d'une image comme lettrine
\end{center}

\vspace{\baselineskip}
Si la lettrine souhaitée n'est pas un caractère d'une fonte mais une
image, \verb+\lettrine+ peut encore être utilisée, il suffit
d'utiliser le booléen \texttt{image}, exemple :

\begin{otherlanguage}{german}
\fontfamily{yfrak}\selectfont\Large
\renewcommand{\LettrineTextFont}{\relax}
\lettrine[image=true,lhang=.2, loversize=.25, findent=0.1em]
{W}{er} reitet so spät durch Nacht und Wind?\\
Es ist der Vater mit seinem Kind;\\
Er hat den Knaben wohl in dem Arm,\\
Er faßt ihn sicher, er hält ihn warm.\par
\end{otherlanguage}

\vspace{\baselineskip}
Et voici le code \LaTeX{} correspondant ;
le  premier argument de \verb+\lettrine+ étant \verb+W+,
\verb+\lettrine+ fait appel au fichier \verb+W.eps+ en \LaTeX{}
ou à \verb+W.pdf+, \verb+W.jpg+, etc. en pdf\LaTeX{} (omission
possible du suffixe \verb+.eps+,  \verb+.pdf+, propriété de
\verb+graphicx.sty+) ;

\begin{verbatim}
{\selectlanguage{german}
\fontfamily{yfrak}\selectfont\Large
\renewcommand{\LettrineTextFont}{\relax}
\lettrine[image=true,lhang=.2, loversize=.25, findent=0.1em]
{W}{er} reitet so spät durch Nacht und Wind?\\
Es ist der Vater mit seinem Kind;\\
Er hat den Knaben wohl in dem Arm,\\
Er faßt ihn sicher, er hält ihn warm.\par}
\end{verbatim}

Cet exemple fait appel aux fichiers suivants :
\begin{itemize}
\item \verb+graphicx.sty+ (extension standard \LaTeXe{}),
\item \verb+blackletter1+ de Thorsten~\textsc{Bronger} (disponible sur CTAN),
\item les fontes gothiques « Fraktur » de Yannis~\textsc{Haralambous}
  (sources \MF{}~\verb+yfrak*.mf+ ou type\,1 \verb+yfrak.pfb+ également
  disponibles sur CTAN).
\end{itemize}
L'initiale gothique « W » utilisée dans
cet exemple a été créée par \MP{} (fichier \verb+W.eps+ ci-joint)
à partir du source \MF{} \verb+yinitW.mf+ de Yannis grâce aux précieuses
indications de Denis~\textsc{Roegel} que je remercie bien vivement.

Si une sortie au format PDF est souhaitée, il suffit de convertir
le fichier \verb+W.eps+ en \verb+W.pdf+ (en utilisant \verb+epstopdf+) et
de compiler le fichier avec pdf\LaTeX.

En fait, \verb+\lettrine+, utilisée avec pdf\LaTeX, permet d'inclure
des images au formats \texttt{pdf}, \texttt{png}, \texttt{jpeg} ou \MP{} en
guise de lettrines.

\vfill
\begin{flushright}
  Daniel \textsc{Flipo}\\
  \texttt{daniel.flipo@free.fr}\\
  août 2015.
\end{flushright}

\end{document}

%%% Local Variables:
%%% mode: latex
%%% coding: utf-8
%%% TeX-master: t
%%% End:
