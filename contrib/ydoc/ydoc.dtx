% \iffalse
%<*driver>
\begingroup
\def\latexfmt{LaTeX2e}
\ifx\latexfmt\fmtname
\long\def\skip#1\NeedsTeXFormat{\endgroup\NeedsTeXFormat}
\expandafter\skip
\else
\endgroup
\let\NeedsTeXFormat\end
\fi
%</driver>
% \fi
%
% \iffalse
%<*ydocstrip.tex>
% \fi
%
% \iffalse
%    \begin{macrocode}
\expandafter\ifx\csname ydocstripversion\endcsname\relax\else
    \endinput
\fi

\chardef\ydocstripversion=1

\input docstrip


\def\checkOption<#1{%
  \ifcase
    \ifx*#10\else \ifx/#11\else
    \ifx+#12\else \ifx-#13\else
    \ifx<#14\else
    \ifx=#15\else
    \ifx!#16\else
    10\fi\fi\fi\fi\fi\fi\fi\relax
  \expandafter\starOption\or
  \expandafter\slashOption\or
  \expandafter\plusOption\or
  \expandafter\minusOption\or
  \expandafter\verbOption\or
  \expandafter\varOption\or
  \expandafter\valueOption\else
  \expandafter\doOption\fi
  #1}

\def\varOption =#1#2>#3\endLine{{%
    \ifx*#1\relax
        \edef\varStop{\perCent<=/#2>}%
        \global\expandafter\let\csname ydocstrip@var@#2\endcsname\empty%
        \expandafter\def\csname ydocstrip@var@#2\endcsname##1{}%
        \loop
            \ifeof\inFile
                \errmessage{Source file ended while reading a multi-line variable content!}%
            \fi
            \read\inFile to \inLine
        \if 1\ifx\inLine\varStop 0\fi 1% if not inLine==varStop
            \expandafter\xdef\csname ydocstrip@var@#2\endcsname{\csname ydocstrip@var@#2\endcsname^^J\inLine}%
            \maybeMsg{.}%
        \repeat
    \else
    \ifx/#1\relax
        \errmessage{Error in expression: spurious '<=/#2>'}%
    \else
    \ifx+#1\relax
        \expandafter\ifx\csname ydocstrip@var@#2\endcsname\relax
            \expandafter\xdef\csname ydocstrip@var@#2\endcsname{#3}%
        \else
            \expandafter\xdef\csname ydocstrip@var@#2\endcsname{\csname ydocstrip@var@#2\endcsname^^J#3}%
        \fi
    \else
        \expandafter\gdef\csname ydocstrip@var@#1#2\endcsname{#3}%
    \fi
    \fi
    \fi
}}


\def\valueOption !#1>#2\endLine{%
    \begingroup
    \expandafter\let\expandafter\var\csname ydocstrip@var@#1\endcsname
    \ifx\var\relax
        \errmessage{Used variable '#1' was never defined!}%
    \else
    \ifx\var\empty\else
        \def\do##1##2##3{%
            \StreamPut##1{\csname ydocstrip@var@#1\endcsname}%
        }%
        \activefiles
    \fi
    \fi
    \endgroup
}
%    \end{macrocode}
% \fi
% \iffalse
%</ydocstrip.tex>
% \fi
%
% \iffalse
%<*ydoc.ins>
% \fi
% \iffalse COPYRIGHT
%%
%% Copyright (c) 2010-2011 by Martin Scharrer <martin@scharrer-online.de>
%% WWW: http://latex.scharrer-online.de/
%%
%% This work may be distributed and/or modified under the
%% conditions of the LaTeX Project Public License, either version 1.3
%% of this license or (at your option) any later version.
%% The latest version of this license is in
%%   http://www.latex-project.org/lppl.txt
%% and version 1.3 or later is part of all distributions of LaTeX
%% version 2005/12/01 or later.
%%
%% This work has the LPPL maintenance status `maintained'.
%%
%% The Current Maintainer of this work is Martin Scharrer.
%%
%% This work consists of the files ydoc.dtx and ydoc.ins
%% and the derived file ydoc.cls, ydoc.sty, ydoc-code.sty,
%% ydoc-expl.sty and ydoc-desc.sty.
%%
% \fi
%
% \iffalse
\input ydocstrip

\preamble
\endpreamble

\keepsilent
\askforoverwritefalse

\generate{%
  \file{ydocstrip.tex}{\from{ydoc.dtx}{ydocstrip.tex}}%
  \file{ydocincl.tex} {\from{ydoc.dtx}{ydocincl.tex}}%
  \file{ydoc.ins}     {\from{ydoc.dtx}{ydoc.ins}}%
  \file{ydoc.cls}     {\from{ydoc.dtx}{ydoc.cls}}%
  \file{ydoc.sty}     {\from{ydoc.dtx}{ydoc.sty}}%
  \file{ydoc.cfg}     {\from{ydoc.dtx}{ydoc.cfg}}%
  \file{ydoc-code.sty}{\from{ydoc.dtx}{ydoc-code.sty}}%
  \file{ydoc-desc.sty}{\from{ydoc.dtx}{ydoc-desc.sty}}%
  \file{ydoc-expl.sty}{\from{ydoc.dtx}{ydoc-expl.sty}}%
  \file{ydoc-doc.sty} {\from{ydoc.dtx}{ydoc-doc.sty}}%
}

\begingroup
\obeyspaces
\Msg{*************************************************************}%
\Msg{*                                                           *}%
\Msg{* To finish the installation you have to move the following *}%
\Msg{* file into a directory searched by TeX:                    *}%
\Msg{*                                                           *}%
\Msg{*     ydoc.cls                                              *}%
\Msg{*     ydoc.sty                                              *}%
\Msg{*     ydoc-code.sty                                         *}%
\Msg{*     ydoc-desc.sty                                         *}%
\Msg{*     ydoc-expl.sty                                         *}%
\Msg{*     ydoc-doc.sty                                          *}%
\Msg{*     ydocstrip.tex                                         *}%
\Msg{*     ydocincl.tex                                          *}%
\Msg{*                                                           *}%
\Msg{* To produce the documentation run the file ydoc.dtx        *}%
\Msg{* through LaTeX.                                            *}%
\Msg{*                                                           *}%
\Msg{* Happy TeXing!                                             *}%
\Msg{*                                                           *}%
\Msg{*************************************************************}%
\endgroup%

\endbatchfile
% vim: ft=tex
% \fi
% \iffalse
%</ydoc.ins>
% \fi
%
% \iffalse
%<*driver>
\NeedsTeXFormat{LaTeX2e}[1999/12/01]
\ProvidesFile{ydoc.dtx}[%
%<=*DATE>
    2012/05/03
%<=/DATE>
%<=*VERSION>
    v0.6alpha
%<=/VERSION>
    DTX file of ydoc]

\message{^^JRun this DTX file using 'tex' to extract all files!^^J}
\documentclass{ydoc}

\newdimen\tempskip

\def\tablecaption{%
  \tempskip=\abovecaptionskip
  \abovecaptionskip=\belowcaptionskip
  \belowcaptionskip=\tempskip
  \caption
}

%\OnlyDescription
\begin{document}
  \DocInput{ydoc.dtx}
  \PrintChanges
  \PrintIndex
\end{document}
%</driver>
% \fi
%
% \CheckSum{0}
%
% \CharacterTable
%  {Upper-case    \A\B\C\D\E\F\G\H\I\J\K\L\M\N\O\P\Q\R\S\T\U\V\W\X\Y\Z
%   Lower-case    \a\b\c\d\e\f\g\h\i\j\k\l\m\n\o\p\q\r\s\t\u\v\w\x\y\z
%   Digits        \0\1\2\3\4\5\6\7\8\9
%   Exclamation   \!     Double quote  \"     Hash (number) \#
%   Dollar        \$     Percent       \%     Ampersand     \&
%   Acute accent  \'     Left paren    \(     Right paren   \)
%   Asterisk      \*     Plus          \+     Comma         \,
%   Minus         \-     Point         \.     Solidus       \/
%   Colon         \:     Semicolon     \;     Less than     \<
%   Equals        \=     Greater than  \>     Question mark \?
%   Commercial at \@     Left bracket  \[     Backslash     \\
%   Right bracket \]     Circumflex    \^     Underscore    \_
%   Grave accent  \`     Left brace    \{     Vertical bar  \|
%   Right brace   \}     Tilde         \~}
%
%
% \changes{v0.1alpha}{2010/04/10}{First released alpha version}
%
% \DoNotIndex{\newcommand,\newenvironment}
%
% \GetFileInfoSVN{ydoc.dtx}
%
% \title{The \textsf{\textbf{ydoc}} Class and Packages}
% \author{Martin Scharrer}
% \repository{https://bitbucket.org/martin_scharrer/ydoc/}
% \email{martin@scharrer.de}
% \makeatletter
% \date{Version \expandafter\@gobble\fileversion\\[0.5ex]\filedate}
% \makeatother
%
% \maketitle
%
% \ifpdf
% \hypersetup{%
%   pdfauthor   = {Martin Scharrer <martin@scharrer-online.de>},
%   pdftitle    = {The ydoc package, \fileversion, \filedate},
%   pdfsubject  = {Documentation of LaTeX package ydoc.},
%   pdfkeywords = {ydoc, documentation, LaTeX}
% }%
% \fi
%
%
% \begin{abstract}
% {\color{red}\bfseries This package bundle is currently under development.
% All functionality, settings and macro as well as file names can change in later versions and may be incomplete!
% It is not ready yet to be used for other packages.
% }
%
% The \pkg{ydoc} class and packages provide macros to document the functionality and implementation of \LaTeX{} classes and packages.
% It is similar to the \pkg{ltxdoc} class with the \pkg{doc} package, but uses more modern features/packages by default
% (e.g.\ \pkg{xcolor}, \pkg{hyperref}, \pkg{listings}). However, some of the features like code indexing is not yet included.
% \end{abstract}
%
% \section{Introduction}
% The \pkg{ydoc} packages allow the documentation of \LaTeX\ packages and classes.
% The name stands for ``\emph{Y}et another \emph{Doc}umentation Package'' and is a pun on
% the fact that there are several documentation packages written by package developers
% to document their own packages. All these packages didn't suited the author and therefore he,
% take a guess, wrote his own documentation package.
% It (will) support(s) all macros and environments (but not necessary with full/identical features) provided by the \pkg{doc}
% package to allow the fast adaption of existing |.dtx| files.
%
% This documentation uses the \pkg{ydoc} packages itself and therefore also acts as a live example.
%
% \subsection{\pkg{ydoc} Files}
% The \pkg{ydoc} bundle consists (at the moment, subject to change) of the \cls{ydoc} class and the packages \pkg{ydoc}, \pkg{ydoc-code},
% \pkg{ydoc-desc}, \pkg{ydoc-expl} and \pkg{ydoc-doc}. The \cls{ydoc} class and package allow the user the freedom to use the functionality with other classes if wanted.
% The class will load the package. The \pkg{ydoc} package loads the packages \pkg{ydoc-code}, \pkg{ydoc-desc}, \pkg{ydoc-expl} and \pkg{ydoc-doc}, which
% provide the functionality to document \LaTeX\ code implementation, describe the user-level macro, include live code examples and provide replacements for the
% macros of the \pkg{doc} package, respectively.
% This packages can be loaded on their own in other kind of \LaTeX\ documents if required.
%
% \subsection{Similar Packages}
% Other documentation related classes and packages are \pkg{ltxdoc}, \pkg{doc}, \pkg{dox}, \pkg{xdoc}, \pkg{gmdoc}, \pkg{pauldoc}, \pkg{hypdoc},
% \pkg{codedoc}, \pkg{nicetext} and \pkg{tkz-doc}.
%
%
% \section{Usage}
% (section incomplete)
%
% \subsection{Code Documentation Environments}
%
% \begin{DescribeEnv}{macro}{<macro>}[<\# of args>]!\optional!{<arg~1~description>}!\ldots!{<arg~$n$~description>}
%  \MacroArgs<macro documentation>\\
%  \begin{DescribeEnv}{macrocode}
%     \MacroArgs<macro code>
%  \end{DescribeEnv}\\
%  \ldots
% \end{DescribeEnv}
% The implementation of macros can be documented using this environment. The actual \meta{macro code} must be placed in a \env{macrocode} environment.
% Longer macro definition can be split using multiple \env{macrocode} environments with interleaved documentation texts.
%
% The \pkg{ydoc} definition of the \env{macro} environment has an additional feature compare to \pkg{doc}. The arguments of the macro (|#1|, |#2|, \ldots) can be
% documented in a vertical list. The environment has an optional argument to declare the \meta{number of arguments} the macro implementation has.
% The descriptions of this macro arguments are read from the next arguments of the environment. If the \meta{number of arguments} is not given or zero (or less)
% no further arguments are read by the \env{macro} environment.
%
% \begin{DescribeEnv}{macrocode}
%     \MacroArgs<macro code>
% \end{DescribeEnv}
% This environment wraps around any \TeX\ code and types it verbatim. The environment end is read verbatim as well and must be written on a separate
% line beginning with a percent followed by exactly four spaces: `\verb*+%    \end{macrocode}+'.
%
% \begin{DescribeEnv}{environment}{<name>}[<\# of args>]!\optional!{<arg~1~description>}!\ldots!{<arg~$n$~description>}
%  \MacroArgs<environment documentation>\\
%  \begin{DescribeEnv}{macrocode}
%     \MacroArgs<macro code>
%  \end{DescribeEnv}\\
%  \ldots
% \end{DescribeEnv}
% This environment provides the same functionality as the \env{macro} environment above, but for environments instead.
%
%
% \subsection{Description Macros and Environments}
%
% \DescribeMacro\DescribeMacro<\textbackslash macro><macro arguments>
% The \cs{DescribeMacro} is used to describe macros included their arguments.
% It takes the to be described \meta{\textbackslash macro} as first argument (can also be enclosed in |{ }|).
% The macro name can include `|@|'. Any number of \meta{macro arguments} (in a broad sense, see Table~\ref{tab:macroargs}) following it
% are formatted as arguments of this macro.
% Any following non-argument token (normal text, macro, etc.)\ will make \cs{DescribeMacro} stop collecting arguments.
% For example, if a \TeX\ group should be started using |{ }| direct after \cs{DescribeMacro} a |\relax| (or a similar macro) should be inserted
% between them, otherwise the group will be taken as mandatory argument of the described macro.
%
% Multiple \cs{DescribeMacro} in a row will automatically stacked inside one framed box. If this is not wanted simply separate them with |\relax|
% or any other macro or token. See also the \env{DescribeMacros} environment below.
%
% \subsubsection*{Examples:}
% \par\noindent |\DescribeMacro\mymacro*[<optional>]{<meta text>}| will result in\\
% \mbox{\Macro\mymacro*[<optional>]{<meta text>}} (inside a framed box).
% \par\medskip\noindent The above syntax description of \cs{DescribeMacro} itself was typeset with\\
% |\DescribeMacro\DescribeMacro<\textbackslash macro><macro arguments>|.
% \par\medskip\noindent Special macros with have a partner macro as end marker can be typeset like this:\\
% |\DescribeMacro\csname<text>\AlsoMacro\endcsname|, which will result in\\
% \mbox{\Macro\csname<text>\AlsoMacro\endcsname}.
%
% \DescribeMacro\Macro<\textbackslash macro><macro arguments>
% This macro is like an in-text version of \cs{DescribeMacro}.
% The macro description stays as part of the surrounding text and is not placed inside a framed box.
% The description can be broken between lines. This can be avoided by placing it inside a |\mbox{}|.
% \cs{Macro} is equivalent to \cs{MacroArgs}\cs{AlsoMacro}.
%
% \DescribeMacro\MacroArgs<macro arguments>
% This macro formats the \meta{macro arguments} the same way as \cs{DescribeMacro} and \cs{Macro} but without a macro name.
% Like \cs{Macro} the description is placed in-text.
%
% \DescribeMacro\AlsoMacro<\textbackslash macro><further macro arguments>
% This macro can only be used inside the \meta{macro arguments} of the above macros and typesets an additional macro as part
% of the syntax of the described macro. The additional macro is normally an end- or other marker of some kind. Further macro arguments may
% follow. Macros which are not part of the syntax but normal arguments should be written as
% |<\textbackslash name>| (yielding \MacroArgs<\textbackslash name>) instead.
% The `\verb+|+' character is an abbreviation of \Macro\AlsoMacro, but only at places where this can appear.
%
% \subsubsection*{Examples:}
% |\Macro\@for<\textbackslash var> ':=' <list> \AlsoMacro\do {<code>}|\\
%  \Macro\@for<\textbackslash var> ':=' <list> \AlsoMacro\do {<code>}
%  \par\medskip\noindent
% |\Macro\pgfkeys{<key1>'='<value1>','<key2>'/.code='{<code>}}|\\
%  \Macro\pgfkeys{<key1>'='<value1>','<key2>'/.code='{<code>}}\\
%
% \DescribeMacro\MakeShortMacroArgs*{<char>}
% This macro is similar to \Macro\MakeShortVerb from the \pkg{shortvrb} package.
% It can be used to globally define one character to act like \Macro\MacroArgs till the same character is discovered again.
% Special characters must be escaped with an backslash for the definition.
% One additional benefit beside the shorter size is that the argument list is automatically terminated.
% For example |\MakeShortMacroArgs{\"}| will make `|"<arg>{<arg>}"|' act like `|\MacroArgs<arg>{<arg>}\relax|'.
% One side-effect is that should the argument list be terminated, e.g. by an unknown element or macro, then the rest of the
% text till the end-character is typeset as normal, but inside a group.
%
% The starred version will define the character equal to \Macro\Macro instead.
%
% \DescribeMacro\DeleteShortMacroArgs{<char>}
% Globally removes the special meaning from \meta{char} given to him by \Macro\MakeShortMacroArgs.
%
% Note that special characters like |`| are
% best defined |\AtBeginDocument| and deleted again |\AtEndDocument| to avoid issues if they are written to the |aux| file by some package.
%
% \begin{DescribeEnv}{DescribeMacros}
%   \Macro\Macro<\textbackslash name><arguments>\\
%   \Macro\Macro<\textbackslash name><arguments>\\
%   \ldots
% \end{DescribeEnv}
% This environment can be used to place multiple macro description into the same framed box.
% The macros are described using \cs{Macro}, which has a slightly different definition than outside of this environment, to place the
% description into a |\hbox|. The environment stacks these |\hbox|es in a |\vbox|. The macros can also be placed freely using anything
% which produces a |\hbox|, e.g.\ |\hbox{\Macro\A ~~~ \Macro\B}| or using
% a |tabular| (see also \env{DescribeMacrosTab}).
%
% \DescribeEnv[<tabular content>]{DescribeMacrosTab}{<tabular column definition>}
% This is a special version of the \env{DescribeMacros} environment which adds a tabular environment around the content.
% This is useful if a large set of small macros should be desribed at once. Placing them all below each other would result in a very bad page layout.
% The environment has one argument which is passed to \env{tabular} as the column definition. A `|@{}|' is added before and after to remove any margins.
%
%
% \begin{DescribeEnv}{DescribeEnv}{<name>}<arguments>
%   \MacroArgs<body content>\relax\space |\\|\\
%   \MacroArgs<more body content>
% \end{DescribeEnv}
% \vspace{-\beforedescskip}
% \DescribeMacro\DescribeEnv[<body content>]{<name>}<arguments>
% The \env{DescribeEnv} can be used to describe environments in the same way the \cs{DescribeMacro} macro describes macros.
% Supported \meta{arguments} are shown in Table~\ref{tab:macroargs}.
% Potential \meta{body content} can be placed between the begin and end of the environment description to explain the user what kind of material
% should be placed inside it.
% The environment also exists in macro form as \cs{DescribeEnv}, which allows to provide small \meta{body content} as an optional argument.
% Please note that for this optional argument a \cs{MacroArgs} is automatically inserted, but not for the \cs{DescribeEnv} environment content.
%
% The body content is placed into a indented |\hbox{}| stacked inside a |\vbox{}| also holding the environment begin and end line.
% The |\\| macro is redefined to create a new indented |\hbox| acting as new code line.
% Therefore this environment is similar to a one-column \env{tabular}: all macros placed into a line are only valid up to the next line end.
%
% \begin{table}
% \tablecaption{Supported `arguments' for \cs{DescribeMacro}/\cs{DescribeEnv}/\cs{MacroArgs}.}\label{tab:macroargs}
% \begin{tabular}{llll}
%  \toprule
%   Description                &  Syntax       &  Result              & Macro$^{a}$ \\
%  \midrule
%   Meta text                  &  |<text>|     & \MacroArgs<text>     & \Macro\meta{<text>} \\
%   Mandatory Argument         &  |{args}|     & \MacroArgs{args}     & \\
%   ---, with meta text        &  |{<text>}|   & \MacroArgs{<text>}   & \Macro\marg{<text>} \\
%   Optional Argument          &  |[args]|     & \MacroArgs[args]     & \\
%   ---, with meta text        &  |[<text>]|   & \MacroArgs[<text>]   & \Macro\oarg{<text>} \\
%   Picture Argument           &  |(args)|     & \MacroArgs(args)     & \\
%   ---, with meta text        &  |(<text>)|   & \MacroArgs(<text>)   & \Macro\parg{<text>} \\
%   Beamer Overlay Argument    &  |<<args>>|   & \MacroArgs<<args>>   \\
%   ---, with meta text        &  |<< <text> >>| & \MacroArgs<<<text> >>   & \Macro\aarg{<text>} \\
%   Star                       &  |*|          & \MacroArgs*          \\
%  \midrule
%   Verbatim content           &  |'$&^%_#$\'| & \MacroArgs'$&^%_#$\' \\
%   ---, produce |'| char      &  |''|         & \MacroArgs''         \\
%   Insert any \TeX\ code      &  |!\fbox{T}!| & \MacroArgs!\fbox{T}! \\
%   Unbreakable Space          &  |~|          & \\
%   Space (explicit macro)     &  |\space|     & \\
%  \midrule
%   Second macro (e.g.\ endmarker) &  |\AlsoMacro\macro| & \MacroArgs\AlsoMacro\macro \\
%   \hfill short version:        &  \verb+|\macro+     & \MacroArgs|\macro \\
%  \bottomrule
%   \multicolumn{4}{l}{{\Large\strut}$^{a}$) As alternative to be used inside normal text.} \\
%   \multicolumn{4}{l}{Note that `|args|' can itself be further macro arguments except true verbatim.}
% \end{tabular}
% \end{table}
%
% \DescribeMacro\DescribeLength<\textbackslash name>{<default value>}
% This macro can be used to describe \LaTeX\ lengths also known as dimensions. Multiple \cs{DescribeLength} macros in a row will automatically be grouped.
%
% \subsection{Format Macros}
% \begin{DescribeMacrosTab}{ll}
%     \Macro\cs{<macro name>}    & \Macro\env{<environment name>} \\
%     \Macro\pkg{<package name>} & \Macro\cls{<class name>}       \\
% \end{DescribeMacrosTab}
% This macros can be used to format names of macros, environments, packages and classes, respectively.
% At the moment they simply use |\texttt|.
%
% \DescribeMacros\hbox{\Macro\bslash~~~~\Macro\percent~~~~\Macro\braceleft~~~~\Macro\braceright}\endDescribeMacros
% This macros define expandable backslash (\texttt{\bslash}$_{12}$), percent char (\texttt{\percent}$_{12}$),
% and left (\texttt{\braceleft}$_{12}$) and right (\texttt{\braceright}$_{12}$) braces with catcode 12 (other), respectively.
% They should only be used with text-typer font when used in text, because other fonts might not have the
% correct characters. The macros must be protected when used in a moving argument.
%
% \DescribeMacrosTab{@{}ll@{}}
%     \Macro\meta{<meta text>}     &
%     \Macro\marg{<argument text>} \\
%     \Macro\oarg{<argument text>} &
%     \Macro\parg{<argument text>} \\
%     \Macro\aarg{<argument text>} &
%     \Macro\sarg                  \\
% \endDescribeMacrosTab
% This macros allow to typeset meta text and mandatory, optional, picture and beamer overlay arguments as well as a star symbol.
% They are used internally by \cs{MacroArgs} and friends. See Table~\ref{tab:macroargs} for examples.
%
% \DescribeMacrosTab{ll}
%     \Macro\metastyle &
%     \Macro\margstyle \\
%     \Macro\oargstyle &
%     \Macro\pargstyle \\
%     \Macro\aargstyle &
%     \Macro\sargstyle \\
% \endDescribeMacrosTab
% This macros are used to define the style in which the corresponding macros above are being formatted.
% They are used like \MacroArgs '{'<\textbackslash stylemacro>{<material>}'}' to allow the styles to
% use macros like |\ttfamily| or |\texttt|\marg{material}. By default the optional argument and the also optional star are
% printed in the color `|optional|' which is a 65\% gray.
%
% \subsection{Settings}
% The following macro and dimensions can be redefined by the user to adjust the layout of the package documentation.
%
% \DescribeLength\descindent    {-20pt}
% \DescribeLength\beforedescskip{~12pt plus 4pt minus 4pt}
% \DescribeLength\afterdescskip {~~6pt plus 2pt minus 2pt}
% These length define the indention and vertical distances before and after a \cs{Describe\ldots} macro or environment, respectively.
%
% \DescribeLength\descsep       {1em in \texttt{tt} font = 10.5pt}
% This macro defines the space on the left and right side between the description text and the framed box.
%
% \subsection{Macros and Environments to include LaTeX Code Examples}
% \DescribeEnv{example}
% \DescribeEnv{examplecode}
%
% (to be written)
%
% \StopEventually{}
%
% \clearpage
% \section{Implementation}
%
% \iffalse
%<*ydoc.cls>
% \fi
% \subsection{Class File}
%
%    \begin{macrocode}
\NeedsTeXFormat{LaTeX2e}[1999/12/01]
\ProvidesClass{ydoc}[%
%<!DATE>
%<!VERSION>
%<*DRIVER>
    2011/08/11 develop
%</DRIVER>
    ydoc class: document LaTeX class and packages]
%    \end{macrocode}
%
% At the moment simply load \cls{article} class with \opt{a4paper} option
% and load the \pkg{ydoc} package.
%    \begin{macrocode}
\PassOptionsToClass{a4paper}{article}
\DeclareOption*{\expandafter\PassOptionsToClass\expandafter{\CurrentOption}{article}}
\ProcessOptions\relax
\LoadClass{article}
\RequirePackage{ydoc}
%    \end{macrocode}
% \iffalse
%</ydoc.cls>
% \fi
%
% \iffalse
%<*ydoc.sty>
% \fi
% \subsection{Package File}
%
%    \begin{macrocode}
\NeedsTeXFormat{LaTeX2e}[1999/12/01]
\ProvidesPackage{ydoc}[%
%<!DATE>
%<!VERSION>
%<*DRIVER>
    2011/08/11 develop
%</DRIVER>
    ydoc package: document LaTeX class and packages]
%    \end{macrocode}
%
%    \begin{macrocode}
\RequirePackage{svn-prov}[2010/04/03]
\RequirePackage{ydoc-code}
\RequirePackage{ydoc-expl}
\RequirePackage{ydoc-desc}
\RequirePackage{ydoc-doc}

\RequirePackage{newverbs}
\MakeSpecialShortVerb{\qverb}{\"}
\AtBeginDocument{\catcode`\^^A=14\relax}

% \iffalse
%<*driver>
\begingroup
\def\latexfmt{LaTeX2e}
\ifx\latexfmt\fmtname
\long\def\skip#1\NeedsTeXFormat{\endgroup\NeedsTeXFormat}
\expandafter\skip
\else
\endgroup
\let\NeedsTeXFormat\end
\fi
%</driver>
% \fi
%
% \iffalse
%<*ydocstrip.tex>
% \fi
%
% \iffalse
%    \begin{macrocode}
\expandafter\ifx\csname ydocstripversion\endcsname\relax\else
    \endinput
\fi

\chardef\ydocstripversion=1

\input docstrip


\def\checkOption<#1{%
  \ifcase
    \ifx*#10\else \ifx/#11\else
    \ifx+#12\else \ifx-#13\else
    \ifx<#14\else
    \ifx=#15\else
    \ifx!#16\else
    10\fi\fi\fi\fi\fi\fi\fi\relax
  \expandafter\starOption\or
  \expandafter\slashOption\or
  \expandafter\plusOption\or
  \expandafter\minusOption\or
  \expandafter\verbOption\or
  \expandafter\varOption\or
  \expandafter\valueOption\else
  \expandafter\doOption\fi
  #1}

\def\varOption =#1#2>#3\endLine{{%
    \ifx*#1\relax
        \edef\varStop{\perCent<=/#2>}%
        \global\expandafter\let\csname ydocstrip@var@#2\endcsname\empty%
        \expandafter\def\csname ydocstrip@var@#2\endcsname##1{}%
        \loop
            \ifeof\inFile
                \errmessage{Source file ended while reading a multi-line variable content!}%
            \fi
            \read\inFile to \inLine
        \if 1\ifx\inLine\varStop 0\fi 1% if not inLine==varStop
            \expandafter\xdef\csname ydocstrip@var@#2\endcsname{\csname ydocstrip@var@#2\endcsname^^J\inLine}%
            \maybeMsg{.}%
        \repeat
    \else
    \ifx/#1\relax
        \errmessage{Error in expression: spurious '<=/#2>'}%
    \else
    \ifx+#1\relax
        \expandafter\ifx\csname ydocstrip@var@#2\endcsname\relax
            \expandafter\xdef\csname ydocstrip@var@#2\endcsname{#3}%
        \else
            \expandafter\xdef\csname ydocstrip@var@#2\endcsname{\csname ydocstrip@var@#2\endcsname^^J#3}%
        \fi
    \else
        \expandafter\gdef\csname ydocstrip@var@#1#2\endcsname{#3}%
    \fi
    \fi
    \fi
}}


\def\valueOption !#1>#2\endLine{%
    \begingroup
    \expandafter\let\expandafter\var\csname ydocstrip@var@#1\endcsname
    \ifx\var\relax
        \errmessage{Used variable '#1' was never defined!}%
    \else
    \ifx\var\empty\else
        \def\do##1##2##3{%
            \StreamPut##1{\csname ydocstrip@var@#1\endcsname}%
        }%
        \activefiles
    \fi
    \fi
    \endgroup
}
%    \end{macrocode}
% \fi
% \iffalse
%</ydocstrip.tex>
% \fi
%
% \iffalse
%<*ydoc.ins>
% \fi
% \iffalse COPYRIGHT
%%
%% Copyright (c) 2010-2011 by Martin Scharrer <martin@scharrer-online.de>
%% WWW: http://latex.scharrer-online.de/
%%
%% This work may be distributed and/or modified under the
%% conditions of the LaTeX Project Public License, either version 1.3
%% of this license or (at your option) any later version.
%% The latest version of this license is in
%%   http://www.latex-project.org/lppl.txt
%% and version 1.3 or later is part of all distributions of LaTeX
%% version 2005/12/01 or later.
%%
%% This work has the LPPL maintenance status `maintained'.
%%
%% The Current Maintainer of this work is Martin Scharrer.
%%
%% This work consists of the files ydoc.dtx and ydoc.ins
%% and the derived file ydoc.cls, ydoc.sty, ydoc-code.sty,
%% ydoc-expl.sty and ydoc-desc.sty.
%%
% \fi
%
% \iffalse
\input ydocstrip

\preamble
\endpreamble

\keepsilent
\askforoverwritefalse

\generate{%
  \file{ydocstrip.tex}{\from{ydoc.dtx}{ydocstrip.tex}}%
  \file{ydocincl.tex} {\from{ydoc.dtx}{ydocincl.tex}}%
  \file{ydoc.ins}     {\from{ydoc.dtx}{ydoc.ins}}%
  \file{ydoc.cls}     {\from{ydoc.dtx}{ydoc.cls}}%
  \file{ydoc.sty}     {\from{ydoc.dtx}{ydoc.sty}}%
  \file{ydoc.cfg}     {\from{ydoc.dtx}{ydoc.cfg}}%
  \file{ydoc-code.sty}{\from{ydoc.dtx}{ydoc-code.sty}}%
  \file{ydoc-desc.sty}{\from{ydoc.dtx}{ydoc-desc.sty}}%
  \file{ydoc-expl.sty}{\from{ydoc.dtx}{ydoc-expl.sty}}%
  \file{ydoc-doc.sty} {\from{ydoc.dtx}{ydoc-doc.sty}}%
}

\begingroup
\obeyspaces
\Msg{*************************************************************}%
\Msg{*                                                           *}%
\Msg{* To finish the installation you have to move the following *}%
\Msg{* file into a directory searched by TeX:                    *}%
\Msg{*                                                           *}%
\Msg{*     ydoc.cls                                              *}%
\Msg{*     ydoc.sty                                              *}%
\Msg{*     ydoc-code.sty                                         *}%
\Msg{*     ydoc-desc.sty                                         *}%
\Msg{*     ydoc-expl.sty                                         *}%
\Msg{*     ydoc-doc.sty                                          *}%
\Msg{*     ydocstrip.tex                                         *}%
\Msg{*     ydocincl.tex                                          *}%
\Msg{*                                                           *}%
\Msg{* To produce the documentation run the file ydoc.dtx        *}%
\Msg{* through LaTeX.                                            *}%
\Msg{*                                                           *}%
\Msg{* Happy TeXing!                                             *}%
\Msg{*                                                           *}%
\Msg{*************************************************************}%
\endgroup%

\endbatchfile
% vim: ft=tex
% \fi
% \iffalse
%</ydoc.ins>
% \fi
%
% \iffalse
%<*driver>
\NeedsTeXFormat{LaTeX2e}[1999/12/01]
\ProvidesFile{ydoc.dtx}[%
%<=*DATE>
    2012/05/03
%<=/DATE>
%<=*VERSION>
    v0.6alpha
%<=/VERSION>
    DTX file of ydoc]

\message{^^JRun this DTX file using 'tex' to extract all files!^^J}
\documentclass{ydoc}

\newdimen\tempskip

\def\tablecaption{%
  \tempskip=\abovecaptionskip
  \abovecaptionskip=\belowcaptionskip
  \belowcaptionskip=\tempskip
  \caption
}

%\OnlyDescription
\begin{document}
  \DocInput{ydoc.dtx}
  \PrintChanges
  \PrintIndex
\end{document}
%</driver>
% \fi
%
% \CheckSum{0}
%
% \CharacterTable
%  {Upper-case    \A\B\C\D\E\F\G\H\I\J\K\L\M\N\O\P\Q\R\S\T\U\V\W\X\Y\Z
%   Lower-case    \a\b\c\d\e\f\g\h\i\j\k\l\m\n\o\p\q\r\s\t\u\v\w\x\y\z
%   Digits        \0\1\2\3\4\5\6\7\8\9
%   Exclamation   \!     Double quote  \"     Hash (number) \#
%   Dollar        \$     Percent       \%     Ampersand     \&
%   Acute accent  \'     Left paren    \(     Right paren   \)
%   Asterisk      \*     Plus          \+     Comma         \,
%   Minus         \-     Point         \.     Solidus       \/
%   Colon         \:     Semicolon     \;     Less than     \<
%   Equals        \=     Greater than  \>     Question mark \?
%   Commercial at \@     Left bracket  \[     Backslash     \\
%   Right bracket \]     Circumflex    \^     Underscore    \_
%   Grave accent  \`     Left brace    \{     Vertical bar  \|
%   Right brace   \}     Tilde         \~}
%
%
% \changes{v0.1alpha}{2010/04/10}{First released alpha version}
%
% \DoNotIndex{\newcommand,\newenvironment}
%
% \GetFileInfoSVN{ydoc.dtx}
%
% \title{The \textsf{\textbf{ydoc}} Class and Packages}
% \author{Martin Scharrer}
% \repository{https://bitbucket.org/martin_scharrer/ydoc/}
% \email{martin@scharrer.de}
% \makeatletter
% \date{Version \expandafter\@gobble\fileversion\\[0.5ex]\filedate}
% \makeatother
%
% \maketitle
%
% \ifpdf
% \hypersetup{%
%   pdfauthor   = {Martin Scharrer <martin@scharrer-online.de>},
%   pdftitle    = {The ydoc package, \fileversion, \filedate},
%   pdfsubject  = {Documentation of LaTeX package ydoc.},
%   pdfkeywords = {ydoc, documentation, LaTeX}
% }%
% \fi
%
%
% \begin{abstract}
% {\color{red}\bfseries This package bundle is currently under development.
% All functionality, settings and macro as well as file names can change in later versions and may be incomplete!
% It is not ready yet to be used for other packages.
% }
%
% The \pkg{ydoc} class and packages provide macros to document the functionality and implementation of \LaTeX{} classes and packages.
% It is similar to the \pkg{ltxdoc} class with the \pkg{doc} package, but uses more modern features/packages by default
% (e.g.\ \pkg{xcolor}, \pkg{hyperref}, \pkg{listings}). However, some of the features like code indexing is not yet included.
% \end{abstract}
%
% \section{Introduction}
% The \pkg{ydoc} packages allow the documentation of \LaTeX\ packages and classes.
% The name stands for ``\emph{Y}et another \emph{Doc}umentation Package'' and is a pun on
% the fact that there are several documentation packages written by package developers
% to document their own packages. All these packages didn't suited the author and therefore he,
% take a guess, wrote his own documentation package.
% It (will) support(s) all macros and environments (but not necessary with full/identical features) provided by the \pkg{doc}
% package to allow the fast adaption of existing |.dtx| files.
%
% This documentation uses the \pkg{ydoc} packages itself and therefore also acts as a live example.
%
% \subsection{\pkg{ydoc} Files}
% The \pkg{ydoc} bundle consists (at the moment, subject to change) of the \cls{ydoc} class and the packages \pkg{ydoc}, \pkg{ydoc-code},
% \pkg{ydoc-desc}, \pkg{ydoc-expl} and \pkg{ydoc-doc}. The \cls{ydoc} class and package allow the user the freedom to use the functionality with other classes if wanted.
% The class will load the package. The \pkg{ydoc} package loads the packages \pkg{ydoc-code}, \pkg{ydoc-desc}, \pkg{ydoc-expl} and \pkg{ydoc-doc}, which
% provide the functionality to document \LaTeX\ code implementation, describe the user-level macro, include live code examples and provide replacements for the
% macros of the \pkg{doc} package, respectively.
% This packages can be loaded on their own in other kind of \LaTeX\ documents if required.
%
% \subsection{Similar Packages}
% Other documentation related classes and packages are \pkg{ltxdoc}, \pkg{doc}, \pkg{dox}, \pkg{xdoc}, \pkg{gmdoc}, \pkg{pauldoc}, \pkg{hypdoc},
% \pkg{codedoc}, \pkg{nicetext} and \pkg{tkz-doc}.
%
%
% \section{Usage}
% (section incomplete)
%
% \subsection{Code Documentation Environments}
%
% \begin{DescribeEnv}{macro}{<macro>}[<\# of args>]!\optional!{<arg~1~description>}!\ldots!{<arg~$n$~description>}
%  \MacroArgs<macro documentation>\\
%  \begin{DescribeEnv}{macrocode}
%     \MacroArgs<macro code>
%  \end{DescribeEnv}\\
%  \ldots
% \end{DescribeEnv}
% The implementation of macros can be documented using this environment. The actual \meta{macro code} must be placed in a \env{macrocode} environment.
% Longer macro definition can be split using multiple \env{macrocode} environments with interleaved documentation texts.
%
% The \pkg{ydoc} definition of the \env{macro} environment has an additional feature compare to \pkg{doc}. The arguments of the macro (|#1|, |#2|, \ldots) can be
% documented in a vertical list. The environment has an optional argument to declare the \meta{number of arguments} the macro implementation has.
% The descriptions of this macro arguments are read from the next arguments of the environment. If the \meta{number of arguments} is not given or zero (or less)
% no further arguments are read by the \env{macro} environment.
%
% \begin{DescribeEnv}{macrocode}
%     \MacroArgs<macro code>
% \end{DescribeEnv}
% This environment wraps around any \TeX\ code and types it verbatim. The environment end is read verbatim as well and must be written on a separate
% line beginning with a percent followed by exactly four spaces: `\verb*+%    \end{macrocode}+'.
%
% \begin{DescribeEnv}{environment}{<name>}[<\# of args>]!\optional!{<arg~1~description>}!\ldots!{<arg~$n$~description>}
%  \MacroArgs<environment documentation>\\
%  \begin{DescribeEnv}{macrocode}
%     \MacroArgs<macro code>
%  \end{DescribeEnv}\\
%  \ldots
% \end{DescribeEnv}
% This environment provides the same functionality as the \env{macro} environment above, but for environments instead.
%
%
% \subsection{Description Macros and Environments}
%
% \DescribeMacro\DescribeMacro<\textbackslash macro><macro arguments>
% The \cs{DescribeMacro} is used to describe macros included their arguments.
% It takes the to be described \meta{\textbackslash macro} as first argument (can also be enclosed in |{ }|).
% The macro name can include `|@|'. Any number of \meta{macro arguments} (in a broad sense, see Table~\ref{tab:macroargs}) following it
% are formatted as arguments of this macro.
% Any following non-argument token (normal text, macro, etc.)\ will make \cs{DescribeMacro} stop collecting arguments.
% For example, if a \TeX\ group should be started using |{ }| direct after \cs{DescribeMacro} a |\relax| (or a similar macro) should be inserted
% between them, otherwise the group will be taken as mandatory argument of the described macro.
%
% Multiple \cs{DescribeMacro} in a row will automatically stacked inside one framed box. If this is not wanted simply separate them with |\relax|
% or any other macro or token. See also the \env{DescribeMacros} environment below.
%
% \subsubsection*{Examples:}
% \par\noindent |\DescribeMacro\mymacro*[<optional>]{<meta text>}| will result in\\
% \mbox{\Macro\mymacro*[<optional>]{<meta text>}} (inside a framed box).
% \par\medskip\noindent The above syntax description of \cs{DescribeMacro} itself was typeset with\\
% |\DescribeMacro\DescribeMacro<\textbackslash macro><macro arguments>|.
% \par\medskip\noindent Special macros with have a partner macro as end marker can be typeset like this:\\
% |\DescribeMacro\csname<text>\AlsoMacro\endcsname|, which will result in\\
% \mbox{\Macro\csname<text>\AlsoMacro\endcsname}.
%
% \DescribeMacro\Macro<\textbackslash macro><macro arguments>
% This macro is like an in-text version of \cs{DescribeMacro}.
% The macro description stays as part of the surrounding text and is not placed inside a framed box.
% The description can be broken between lines. This can be avoided by placing it inside a |\mbox{}|.
% \cs{Macro} is equivalent to \cs{MacroArgs}\cs{AlsoMacro}.
%
% \DescribeMacro\MacroArgs<macro arguments>
% This macro formats the \meta{macro arguments} the same way as \cs{DescribeMacro} and \cs{Macro} but without a macro name.
% Like \cs{Macro} the description is placed in-text.
%
% \DescribeMacro\AlsoMacro<\textbackslash macro><further macro arguments>
% This macro can only be used inside the \meta{macro arguments} of the above macros and typesets an additional macro as part
% of the syntax of the described macro. The additional macro is normally an end- or other marker of some kind. Further macro arguments may
% follow. Macros which are not part of the syntax but normal arguments should be written as
% |<\textbackslash name>| (yielding \MacroArgs<\textbackslash name>) instead.
% The `\verb+|+' character is an abbreviation of \Macro\AlsoMacro, but only at places where this can appear.
%
% \subsubsection*{Examples:}
% |\Macro\@for<\textbackslash var> ':=' <list> \AlsoMacro\do {<code>}|\\
%  \Macro\@for<\textbackslash var> ':=' <list> \AlsoMacro\do {<code>}
%  \par\medskip\noindent
% |\Macro\pgfkeys{<key1>'='<value1>','<key2>'/.code='{<code>}}|\\
%  \Macro\pgfkeys{<key1>'='<value1>','<key2>'/.code='{<code>}}\\
%
% \DescribeMacro\MakeShortMacroArgs*{<char>}
% This macro is similar to \Macro\MakeShortVerb from the \pkg{shortvrb} package.
% It can be used to globally define one character to act like \Macro\MacroArgs till the same character is discovered again.
% Special characters must be escaped with an backslash for the definition.
% One additional benefit beside the shorter size is that the argument list is automatically terminated.
% For example |\MakeShortMacroArgs{\"}| will make `|"<arg>{<arg>}"|' act like `|\MacroArgs<arg>{<arg>}\relax|'.
% One side-effect is that should the argument list be terminated, e.g. by an unknown element or macro, then the rest of the
% text till the end-character is typeset as normal, but inside a group.
%
% The starred version will define the character equal to \Macro\Macro instead.
%
% \DescribeMacro\DeleteShortMacroArgs{<char>}
% Globally removes the special meaning from \meta{char} given to him by \Macro\MakeShortMacroArgs.
%
% Note that special characters like |`| are
% best defined |\AtBeginDocument| and deleted again |\AtEndDocument| to avoid issues if they are written to the |aux| file by some package.
%
% \begin{DescribeEnv}{DescribeMacros}
%   \Macro\Macro<\textbackslash name><arguments>\\
%   \Macro\Macro<\textbackslash name><arguments>\\
%   \ldots
% \end{DescribeEnv}
% This environment can be used to place multiple macro description into the same framed box.
% The macros are described using \cs{Macro}, which has a slightly different definition than outside of this environment, to place the
% description into a |\hbox|. The environment stacks these |\hbox|es in a |\vbox|. The macros can also be placed freely using anything
% which produces a |\hbox|, e.g.\ |\hbox{\Macro\A ~~~ \Macro\B}| or using
% a |tabular| (see also \env{DescribeMacrosTab}).
%
% \DescribeEnv[<tabular content>]{DescribeMacrosTab}{<tabular column definition>}
% This is a special version of the \env{DescribeMacros} environment which adds a tabular environment around the content.
% This is useful if a large set of small macros should be desribed at once. Placing them all below each other would result in a very bad page layout.
% The environment has one argument which is passed to \env{tabular} as the column definition. A `|@{}|' is added before and after to remove any margins.
%
%
% \begin{DescribeEnv}{DescribeEnv}{<name>}<arguments>
%   \MacroArgs<body content>\relax\space |\\|\\
%   \MacroArgs<more body content>
% \end{DescribeEnv}
% \vspace{-\beforedescskip}
% \DescribeMacro\DescribeEnv[<body content>]{<name>}<arguments>
% The \env{DescribeEnv} can be used to describe environments in the same way the \cs{DescribeMacro} macro describes macros.
% Supported \meta{arguments} are shown in Table~\ref{tab:macroargs}.
% Potential \meta{body content} can be placed between the begin and end of the environment description to explain the user what kind of material
% should be placed inside it.
% The environment also exists in macro form as \cs{DescribeEnv}, which allows to provide small \meta{body content} as an optional argument.
% Please note that for this optional argument a \cs{MacroArgs} is automatically inserted, but not for the \cs{DescribeEnv} environment content.
%
% The body content is placed into a indented |\hbox{}| stacked inside a |\vbox{}| also holding the environment begin and end line.
% The |\\| macro is redefined to create a new indented |\hbox| acting as new code line.
% Therefore this environment is similar to a one-column \env{tabular}: all macros placed into a line are only valid up to the next line end.
%
% \begin{table}
% \tablecaption{Supported `arguments' for \cs{DescribeMacro}/\cs{DescribeEnv}/\cs{MacroArgs}.}\label{tab:macroargs}
% \begin{tabular}{llll}
%  \toprule
%   Description                &  Syntax       &  Result              & Macro$^{a}$ \\
%  \midrule
%   Meta text                  &  |<text>|     & \MacroArgs<text>     & \Macro\meta{<text>} \\
%   Mandatory Argument         &  |{args}|     & \MacroArgs{args}     & \\
%   ---, with meta text        &  |{<text>}|   & \MacroArgs{<text>}   & \Macro\marg{<text>} \\
%   Optional Argument          &  |[args]|     & \MacroArgs[args]     & \\
%   ---, with meta text        &  |[<text>]|   & \MacroArgs[<text>]   & \Macro\oarg{<text>} \\
%   Picture Argument           &  |(args)|     & \MacroArgs(args)     & \\
%   ---, with meta text        &  |(<text>)|   & \MacroArgs(<text>)   & \Macro\parg{<text>} \\
%   Beamer Overlay Argument    &  |<<args>>|   & \MacroArgs<<args>>   \\
%   ---, with meta text        &  |<< <text> >>| & \MacroArgs<<<text> >>   & \Macro\aarg{<text>} \\
%   Star                       &  |*|          & \MacroArgs*          \\
%  \midrule
%   Verbatim content           &  |'$&^%_#$\'| & \MacroArgs'$&^%_#$\' \\
%   ---, produce |'| char      &  |''|         & \MacroArgs''         \\
%   Insert any \TeX\ code      &  |!\fbox{T}!| & \MacroArgs!\fbox{T}! \\
%   Unbreakable Space          &  |~|          & \\
%   Space (explicit macro)     &  |\space|     & \\
%  \midrule
%   Second macro (e.g.\ endmarker) &  |\AlsoMacro\macro| & \MacroArgs\AlsoMacro\macro \\
%   \hfill short version:        &  \verb+|\macro+     & \MacroArgs|\macro \\
%  \bottomrule
%   \multicolumn{4}{l}{{\Large\strut}$^{a}$) As alternative to be used inside normal text.} \\
%   \multicolumn{4}{l}{Note that `|args|' can itself be further macro arguments except true verbatim.}
% \end{tabular}
% \end{table}
%
% \DescribeMacro\DescribeLength<\textbackslash name>{<default value>}
% This macro can be used to describe \LaTeX\ lengths also known as dimensions. Multiple \cs{DescribeLength} macros in a row will automatically be grouped.
%
% \subsection{Format Macros}
% \begin{DescribeMacrosTab}{ll}
%     \Macro\cs{<macro name>}    & \Macro\env{<environment name>} \\
%     \Macro\pkg{<package name>} & \Macro\cls{<class name>}       \\
% \end{DescribeMacrosTab}
% This macros can be used to format names of macros, environments, packages and classes, respectively.
% At the moment they simply use |\texttt|.
%
% \DescribeMacros\hbox{\Macro\bslash~~~~\Macro\percent~~~~\Macro\braceleft~~~~\Macro\braceright}\endDescribeMacros
% This macros define expandable backslash (\texttt{\bslash}$_{12}$), percent char (\texttt{\percent}$_{12}$),
% and left (\texttt{\braceleft}$_{12}$) and right (\texttt{\braceright}$_{12}$) braces with catcode 12 (other), respectively.
% They should only be used with text-typer font when used in text, because other fonts might not have the
% correct characters. The macros must be protected when used in a moving argument.
%
% \DescribeMacrosTab{@{}ll@{}}
%     \Macro\meta{<meta text>}     &
%     \Macro\marg{<argument text>} \\
%     \Macro\oarg{<argument text>} &
%     \Macro\parg{<argument text>} \\
%     \Macro\aarg{<argument text>} &
%     \Macro\sarg                  \\
% \endDescribeMacrosTab
% This macros allow to typeset meta text and mandatory, optional, picture and beamer overlay arguments as well as a star symbol.
% They are used internally by \cs{MacroArgs} and friends. See Table~\ref{tab:macroargs} for examples.
%
% \DescribeMacrosTab{ll}
%     \Macro\metastyle &
%     \Macro\margstyle \\
%     \Macro\oargstyle &
%     \Macro\pargstyle \\
%     \Macro\aargstyle &
%     \Macro\sargstyle \\
% \endDescribeMacrosTab
% This macros are used to define the style in which the corresponding macros above are being formatted.
% They are used like \MacroArgs '{'<\textbackslash stylemacro>{<material>}'}' to allow the styles to
% use macros like |\ttfamily| or |\texttt|\marg{material}. By default the optional argument and the also optional star are
% printed in the color `|optional|' which is a 65\% gray.
%
% \subsection{Settings}
% The following macro and dimensions can be redefined by the user to adjust the layout of the package documentation.
%
% \DescribeLength\descindent    {-20pt}
% \DescribeLength\beforedescskip{~12pt plus 4pt minus 4pt}
% \DescribeLength\afterdescskip {~~6pt plus 2pt minus 2pt}
% These length define the indention and vertical distances before and after a \cs{Describe\ldots} macro or environment, respectively.
%
% \DescribeLength\descsep       {1em in \texttt{tt} font = 10.5pt}
% This macro defines the space on the left and right side between the description text and the framed box.
%
% \subsection{Macros and Environments to include LaTeX Code Examples}
% \DescribeEnv{example}
% \DescribeEnv{examplecode}
%
% (to be written)
%
% \StopEventually{}
%
% \clearpage
% \section{Implementation}
%
% \iffalse
%<*ydoc.cls>
% \fi
% \subsection{Class File}
%
%    \begin{macrocode}
\NeedsTeXFormat{LaTeX2e}[1999/12/01]
\ProvidesClass{ydoc}[%
%<!DATE>
%<!VERSION>
%<*DRIVER>
    2011/08/11 develop
%</DRIVER>
    ydoc class: document LaTeX class and packages]
%    \end{macrocode}
%
% At the moment simply load \cls{article} class with \opt{a4paper} option
% and load the \pkg{ydoc} package.
%    \begin{macrocode}
\PassOptionsToClass{a4paper}{article}
\DeclareOption*{\expandafter\PassOptionsToClass\expandafter{\CurrentOption}{article}}
\ProcessOptions\relax
\LoadClass{article}
\RequirePackage{ydoc}
%    \end{macrocode}
% \iffalse
%</ydoc.cls>
% \fi
%
% \iffalse
%<*ydoc.sty>
% \fi
% \subsection{Package File}
%
%    \begin{macrocode}
\NeedsTeXFormat{LaTeX2e}[1999/12/01]
\ProvidesPackage{ydoc}[%
%<!DATE>
%<!VERSION>
%<*DRIVER>
    2011/08/11 develop
%</DRIVER>
    ydoc package: document LaTeX class and packages]
%    \end{macrocode}
%
%    \begin{macrocode}
\RequirePackage{svn-prov}[2010/04/03]
\RequirePackage{ydoc-code}
\RequirePackage{ydoc-expl}
\RequirePackage{ydoc-desc}
\RequirePackage{ydoc-doc}

\RequirePackage{newverbs}
\MakeSpecialShortVerb{\qverb}{\"}
\AtBeginDocument{\catcode`\^^A=14\relax}

% \iffalse
%<*driver>
\begingroup
\def\latexfmt{LaTeX2e}
\ifx\latexfmt\fmtname
\long\def\skip#1\NeedsTeXFormat{\endgroup\NeedsTeXFormat}
\expandafter\skip
\else
\endgroup
\let\NeedsTeXFormat\end
\fi
%</driver>
% \fi
%
% \iffalse
%<*ydocstrip.tex>
% \fi
%
% \iffalse
%    \begin{macrocode}
\expandafter\ifx\csname ydocstripversion\endcsname\relax\else
    \endinput
\fi

\chardef\ydocstripversion=1

\input docstrip


\def\checkOption<#1{%
  \ifcase
    \ifx*#10\else \ifx/#11\else
    \ifx+#12\else \ifx-#13\else
    \ifx<#14\else
    \ifx=#15\else
    \ifx!#16\else
    10\fi\fi\fi\fi\fi\fi\fi\relax
  \expandafter\starOption\or
  \expandafter\slashOption\or
  \expandafter\plusOption\or
  \expandafter\minusOption\or
  \expandafter\verbOption\or
  \expandafter\varOption\or
  \expandafter\valueOption\else
  \expandafter\doOption\fi
  #1}

\def\varOption =#1#2>#3\endLine{{%
    \ifx*#1\relax
        \edef\varStop{\perCent<=/#2>}%
        \global\expandafter\let\csname ydocstrip@var@#2\endcsname\empty%
        \expandafter\def\csname ydocstrip@var@#2\endcsname##1{}%
        \loop
            \ifeof\inFile
                \errmessage{Source file ended while reading a multi-line variable content!}%
            \fi
            \read\inFile to \inLine
        \if 1\ifx\inLine\varStop 0\fi 1% if not inLine==varStop
            \expandafter\xdef\csname ydocstrip@var@#2\endcsname{\csname ydocstrip@var@#2\endcsname^^J\inLine}%
            \maybeMsg{.}%
        \repeat
    \else
    \ifx/#1\relax
        \errmessage{Error in expression: spurious '<=/#2>'}%
    \else
    \ifx+#1\relax
        \expandafter\ifx\csname ydocstrip@var@#2\endcsname\relax
            \expandafter\xdef\csname ydocstrip@var@#2\endcsname{#3}%
        \else
            \expandafter\xdef\csname ydocstrip@var@#2\endcsname{\csname ydocstrip@var@#2\endcsname^^J#3}%
        \fi
    \else
        \expandafter\gdef\csname ydocstrip@var@#1#2\endcsname{#3}%
    \fi
    \fi
    \fi
}}


\def\valueOption !#1>#2\endLine{%
    \begingroup
    \expandafter\let\expandafter\var\csname ydocstrip@var@#1\endcsname
    \ifx\var\relax
        \errmessage{Used variable '#1' was never defined!}%
    \else
    \ifx\var\empty\else
        \def\do##1##2##3{%
            \StreamPut##1{\csname ydocstrip@var@#1\endcsname}%
        }%
        \activefiles
    \fi
    \fi
    \endgroup
}
%    \end{macrocode}
% \fi
% \iffalse
%</ydocstrip.tex>
% \fi
%
% \iffalse
%<*ydoc.ins>
% \fi
% \iffalse COPYRIGHT
%%
%% Copyright (c) 2010-2011 by Martin Scharrer <martin@scharrer-online.de>
%% WWW: http://latex.scharrer-online.de/
%%
%% This work may be distributed and/or modified under the
%% conditions of the LaTeX Project Public License, either version 1.3
%% of this license or (at your option) any later version.
%% The latest version of this license is in
%%   http://www.latex-project.org/lppl.txt
%% and version 1.3 or later is part of all distributions of LaTeX
%% version 2005/12/01 or later.
%%
%% This work has the LPPL maintenance status `maintained'.
%%
%% The Current Maintainer of this work is Martin Scharrer.
%%
%% This work consists of the files ydoc.dtx and ydoc.ins
%% and the derived file ydoc.cls, ydoc.sty, ydoc-code.sty,
%% ydoc-expl.sty and ydoc-desc.sty.
%%
% \fi
%
% \iffalse
\input ydocstrip

\preamble
\endpreamble

\keepsilent
\askforoverwritefalse

\generate{%
  \file{ydocstrip.tex}{\from{ydoc.dtx}{ydocstrip.tex}}%
  \file{ydocincl.tex} {\from{ydoc.dtx}{ydocincl.tex}}%
  \file{ydoc.ins}     {\from{ydoc.dtx}{ydoc.ins}}%
  \file{ydoc.cls}     {\from{ydoc.dtx}{ydoc.cls}}%
  \file{ydoc.sty}     {\from{ydoc.dtx}{ydoc.sty}}%
  \file{ydoc.cfg}     {\from{ydoc.dtx}{ydoc.cfg}}%
  \file{ydoc-code.sty}{\from{ydoc.dtx}{ydoc-code.sty}}%
  \file{ydoc-desc.sty}{\from{ydoc.dtx}{ydoc-desc.sty}}%
  \file{ydoc-expl.sty}{\from{ydoc.dtx}{ydoc-expl.sty}}%
  \file{ydoc-doc.sty} {\from{ydoc.dtx}{ydoc-doc.sty}}%
}

\begingroup
\obeyspaces
\Msg{*************************************************************}%
\Msg{*                                                           *}%
\Msg{* To finish the installation you have to move the following *}%
\Msg{* file into a directory searched by TeX:                    *}%
\Msg{*                                                           *}%
\Msg{*     ydoc.cls                                              *}%
\Msg{*     ydoc.sty                                              *}%
\Msg{*     ydoc-code.sty                                         *}%
\Msg{*     ydoc-desc.sty                                         *}%
\Msg{*     ydoc-expl.sty                                         *}%
\Msg{*     ydoc-doc.sty                                          *}%
\Msg{*     ydocstrip.tex                                         *}%
\Msg{*     ydocincl.tex                                          *}%
\Msg{*                                                           *}%
\Msg{* To produce the documentation run the file ydoc.dtx        *}%
\Msg{* through LaTeX.                                            *}%
\Msg{*                                                           *}%
\Msg{* Happy TeXing!                                             *}%
\Msg{*                                                           *}%
\Msg{*************************************************************}%
\endgroup%

\endbatchfile
% vim: ft=tex
% \fi
% \iffalse
%</ydoc.ins>
% \fi
%
% \iffalse
%<*driver>
\NeedsTeXFormat{LaTeX2e}[1999/12/01]
\ProvidesFile{ydoc.dtx}[%
%<=*DATE>
    2012/05/03
%<=/DATE>
%<=*VERSION>
    v0.6alpha
%<=/VERSION>
    DTX file of ydoc]

\message{^^JRun this DTX file using 'tex' to extract all files!^^J}
\documentclass{ydoc}

\newdimen\tempskip

\def\tablecaption{%
  \tempskip=\abovecaptionskip
  \abovecaptionskip=\belowcaptionskip
  \belowcaptionskip=\tempskip
  \caption
}

%\OnlyDescription
\begin{document}
  \DocInput{ydoc.dtx}
  \PrintChanges
  \PrintIndex
\end{document}
%</driver>
% \fi
%
% \CheckSum{0}
%
% \CharacterTable
%  {Upper-case    \A\B\C\D\E\F\G\H\I\J\K\L\M\N\O\P\Q\R\S\T\U\V\W\X\Y\Z
%   Lower-case    \a\b\c\d\e\f\g\h\i\j\k\l\m\n\o\p\q\r\s\t\u\v\w\x\y\z
%   Digits        \0\1\2\3\4\5\6\7\8\9
%   Exclamation   \!     Double quote  \"     Hash (number) \#
%   Dollar        \$     Percent       \%     Ampersand     \&
%   Acute accent  \'     Left paren    \(     Right paren   \)
%   Asterisk      \*     Plus          \+     Comma         \,
%   Minus         \-     Point         \.     Solidus       \/
%   Colon         \:     Semicolon     \;     Less than     \<
%   Equals        \=     Greater than  \>     Question mark \?
%   Commercial at \@     Left bracket  \[     Backslash     \\
%   Right bracket \]     Circumflex    \^     Underscore    \_
%   Grave accent  \`     Left brace    \{     Vertical bar  \|
%   Right brace   \}     Tilde         \~}
%
%
% \changes{v0.1alpha}{2010/04/10}{First released alpha version}
%
% \DoNotIndex{\newcommand,\newenvironment}
%
% \GetFileInfoSVN{ydoc.dtx}
%
% \title{The \textsf{\textbf{ydoc}} Class and Packages}
% \author{Martin Scharrer}
% \repository{https://bitbucket.org/martin_scharrer/ydoc/}
% \email{martin@scharrer.de}
% \makeatletter
% \date{Version \expandafter\@gobble\fileversion\\[0.5ex]\filedate}
% \makeatother
%
% \maketitle
%
% \ifpdf
% \hypersetup{%
%   pdfauthor   = {Martin Scharrer <martin@scharrer-online.de>},
%   pdftitle    = {The ydoc package, \fileversion, \filedate},
%   pdfsubject  = {Documentation of LaTeX package ydoc.},
%   pdfkeywords = {ydoc, documentation, LaTeX}
% }%
% \fi
%
%
% \begin{abstract}
% {\color{red}\bfseries This package bundle is currently under development.
% All functionality, settings and macro as well as file names can change in later versions and may be incomplete!
% It is not ready yet to be used for other packages.
% }
%
% The \pkg{ydoc} class and packages provide macros to document the functionality and implementation of \LaTeX{} classes and packages.
% It is similar to the \pkg{ltxdoc} class with the \pkg{doc} package, but uses more modern features/packages by default
% (e.g.\ \pkg{xcolor}, \pkg{hyperref}, \pkg{listings}). However, some of the features like code indexing is not yet included.
% \end{abstract}
%
% \section{Introduction}
% The \pkg{ydoc} packages allow the documentation of \LaTeX\ packages and classes.
% The name stands for ``\emph{Y}et another \emph{Doc}umentation Package'' and is a pun on
% the fact that there are several documentation packages written by package developers
% to document their own packages. All these packages didn't suited the author and therefore he,
% take a guess, wrote his own documentation package.
% It (will) support(s) all macros and environments (but not necessary with full/identical features) provided by the \pkg{doc}
% package to allow the fast adaption of existing |.dtx| files.
%
% This documentation uses the \pkg{ydoc} packages itself and therefore also acts as a live example.
%
% \subsection{\pkg{ydoc} Files}
% The \pkg{ydoc} bundle consists (at the moment, subject to change) of the \cls{ydoc} class and the packages \pkg{ydoc}, \pkg{ydoc-code},
% \pkg{ydoc-desc}, \pkg{ydoc-expl} and \pkg{ydoc-doc}. The \cls{ydoc} class and package allow the user the freedom to use the functionality with other classes if wanted.
% The class will load the package. The \pkg{ydoc} package loads the packages \pkg{ydoc-code}, \pkg{ydoc-desc}, \pkg{ydoc-expl} and \pkg{ydoc-doc}, which
% provide the functionality to document \LaTeX\ code implementation, describe the user-level macro, include live code examples and provide replacements for the
% macros of the \pkg{doc} package, respectively.
% This packages can be loaded on their own in other kind of \LaTeX\ documents if required.
%
% \subsection{Similar Packages}
% Other documentation related classes and packages are \pkg{ltxdoc}, \pkg{doc}, \pkg{dox}, \pkg{xdoc}, \pkg{gmdoc}, \pkg{pauldoc}, \pkg{hypdoc},
% \pkg{codedoc}, \pkg{nicetext} and \pkg{tkz-doc}.
%
%
% \section{Usage}
% (section incomplete)
%
% \subsection{Code Documentation Environments}
%
% \begin{DescribeEnv}{macro}{<macro>}[<\# of args>]!\optional!{<arg~1~description>}!\ldots!{<arg~$n$~description>}
%  \MacroArgs<macro documentation>\\
%  \begin{DescribeEnv}{macrocode}
%     \MacroArgs<macro code>
%  \end{DescribeEnv}\\
%  \ldots
% \end{DescribeEnv}
% The implementation of macros can be documented using this environment. The actual \meta{macro code} must be placed in a \env{macrocode} environment.
% Longer macro definition can be split using multiple \env{macrocode} environments with interleaved documentation texts.
%
% The \pkg{ydoc} definition of the \env{macro} environment has an additional feature compare to \pkg{doc}. The arguments of the macro (|#1|, |#2|, \ldots) can be
% documented in a vertical list. The environment has an optional argument to declare the \meta{number of arguments} the macro implementation has.
% The descriptions of this macro arguments are read from the next arguments of the environment. If the \meta{number of arguments} is not given or zero (or less)
% no further arguments are read by the \env{macro} environment.
%
% \begin{DescribeEnv}{macrocode}
%     \MacroArgs<macro code>
% \end{DescribeEnv}
% This environment wraps around any \TeX\ code and types it verbatim. The environment end is read verbatim as well and must be written on a separate
% line beginning with a percent followed by exactly four spaces: `\verb*+%    \end{macrocode}+'.
%
% \begin{DescribeEnv}{environment}{<name>}[<\# of args>]!\optional!{<arg~1~description>}!\ldots!{<arg~$n$~description>}
%  \MacroArgs<environment documentation>\\
%  \begin{DescribeEnv}{macrocode}
%     \MacroArgs<macro code>
%  \end{DescribeEnv}\\
%  \ldots
% \end{DescribeEnv}
% This environment provides the same functionality as the \env{macro} environment above, but for environments instead.
%
%
% \subsection{Description Macros and Environments}
%
% \DescribeMacro\DescribeMacro<\textbackslash macro><macro arguments>
% The \cs{DescribeMacro} is used to describe macros included their arguments.
% It takes the to be described \meta{\textbackslash macro} as first argument (can also be enclosed in |{ }|).
% The macro name can include `|@|'. Any number of \meta{macro arguments} (in a broad sense, see Table~\ref{tab:macroargs}) following it
% are formatted as arguments of this macro.
% Any following non-argument token (normal text, macro, etc.)\ will make \cs{DescribeMacro} stop collecting arguments.
% For example, if a \TeX\ group should be started using |{ }| direct after \cs{DescribeMacro} a |\relax| (or a similar macro) should be inserted
% between them, otherwise the group will be taken as mandatory argument of the described macro.
%
% Multiple \cs{DescribeMacro} in a row will automatically stacked inside one framed box. If this is not wanted simply separate them with |\relax|
% or any other macro or token. See also the \env{DescribeMacros} environment below.
%
% \subsubsection*{Examples:}
% \par\noindent |\DescribeMacro\mymacro*[<optional>]{<meta text>}| will result in\\
% \mbox{\Macro\mymacro*[<optional>]{<meta text>}} (inside a framed box).
% \par\medskip\noindent The above syntax description of \cs{DescribeMacro} itself was typeset with\\
% |\DescribeMacro\DescribeMacro<\textbackslash macro><macro arguments>|.
% \par\medskip\noindent Special macros with have a partner macro as end marker can be typeset like this:\\
% |\DescribeMacro\csname<text>\AlsoMacro\endcsname|, which will result in\\
% \mbox{\Macro\csname<text>\AlsoMacro\endcsname}.
%
% \DescribeMacro\Macro<\textbackslash macro><macro arguments>
% This macro is like an in-text version of \cs{DescribeMacro}.
% The macro description stays as part of the surrounding text and is not placed inside a framed box.
% The description can be broken between lines. This can be avoided by placing it inside a |\mbox{}|.
% \cs{Macro} is equivalent to \cs{MacroArgs}\cs{AlsoMacro}.
%
% \DescribeMacro\MacroArgs<macro arguments>
% This macro formats the \meta{macro arguments} the same way as \cs{DescribeMacro} and \cs{Macro} but without a macro name.
% Like \cs{Macro} the description is placed in-text.
%
% \DescribeMacro\AlsoMacro<\textbackslash macro><further macro arguments>
% This macro can only be used inside the \meta{macro arguments} of the above macros and typesets an additional macro as part
% of the syntax of the described macro. The additional macro is normally an end- or other marker of some kind. Further macro arguments may
% follow. Macros which are not part of the syntax but normal arguments should be written as
% |<\textbackslash name>| (yielding \MacroArgs<\textbackslash name>) instead.
% The `\verb+|+' character is an abbreviation of \Macro\AlsoMacro, but only at places where this can appear.
%
% \subsubsection*{Examples:}
% |\Macro\@for<\textbackslash var> ':=' <list> \AlsoMacro\do {<code>}|\\
%  \Macro\@for<\textbackslash var> ':=' <list> \AlsoMacro\do {<code>}
%  \par\medskip\noindent
% |\Macro\pgfkeys{<key1>'='<value1>','<key2>'/.code='{<code>}}|\\
%  \Macro\pgfkeys{<key1>'='<value1>','<key2>'/.code='{<code>}}\\
%
% \DescribeMacro\MakeShortMacroArgs*{<char>}
% This macro is similar to \Macro\MakeShortVerb from the \pkg{shortvrb} package.
% It can be used to globally define one character to act like \Macro\MacroArgs till the same character is discovered again.
% Special characters must be escaped with an backslash for the definition.
% One additional benefit beside the shorter size is that the argument list is automatically terminated.
% For example |\MakeShortMacroArgs{\"}| will make `|"<arg>{<arg>}"|' act like `|\MacroArgs<arg>{<arg>}\relax|'.
% One side-effect is that should the argument list be terminated, e.g. by an unknown element or macro, then the rest of the
% text till the end-character is typeset as normal, but inside a group.
%
% The starred version will define the character equal to \Macro\Macro instead.
%
% \DescribeMacro\DeleteShortMacroArgs{<char>}
% Globally removes the special meaning from \meta{char} given to him by \Macro\MakeShortMacroArgs.
%
% Note that special characters like |`| are
% best defined |\AtBeginDocument| and deleted again |\AtEndDocument| to avoid issues if they are written to the |aux| file by some package.
%
% \begin{DescribeEnv}{DescribeMacros}
%   \Macro\Macro<\textbackslash name><arguments>\\
%   \Macro\Macro<\textbackslash name><arguments>\\
%   \ldots
% \end{DescribeEnv}
% This environment can be used to place multiple macro description into the same framed box.
% The macros are described using \cs{Macro}, which has a slightly different definition than outside of this environment, to place the
% description into a |\hbox|. The environment stacks these |\hbox|es in a |\vbox|. The macros can also be placed freely using anything
% which produces a |\hbox|, e.g.\ |\hbox{\Macro\A ~~~ \Macro\B}| or using
% a |tabular| (see also \env{DescribeMacrosTab}).
%
% \DescribeEnv[<tabular content>]{DescribeMacrosTab}{<tabular column definition>}
% This is a special version of the \env{DescribeMacros} environment which adds a tabular environment around the content.
% This is useful if a large set of small macros should be desribed at once. Placing them all below each other would result in a very bad page layout.
% The environment has one argument which is passed to \env{tabular} as the column definition. A `|@{}|' is added before and after to remove any margins.
%
%
% \begin{DescribeEnv}{DescribeEnv}{<name>}<arguments>
%   \MacroArgs<body content>\relax\space |\\|\\
%   \MacroArgs<more body content>
% \end{DescribeEnv}
% \vspace{-\beforedescskip}
% \DescribeMacro\DescribeEnv[<body content>]{<name>}<arguments>
% The \env{DescribeEnv} can be used to describe environments in the same way the \cs{DescribeMacro} macro describes macros.
% Supported \meta{arguments} are shown in Table~\ref{tab:macroargs}.
% Potential \meta{body content} can be placed between the begin and end of the environment description to explain the user what kind of material
% should be placed inside it.
% The environment also exists in macro form as \cs{DescribeEnv}, which allows to provide small \meta{body content} as an optional argument.
% Please note that for this optional argument a \cs{MacroArgs} is automatically inserted, but not for the \cs{DescribeEnv} environment content.
%
% The body content is placed into a indented |\hbox{}| stacked inside a |\vbox{}| also holding the environment begin and end line.
% The |\\| macro is redefined to create a new indented |\hbox| acting as new code line.
% Therefore this environment is similar to a one-column \env{tabular}: all macros placed into a line are only valid up to the next line end.
%
% \begin{table}
% \tablecaption{Supported `arguments' for \cs{DescribeMacro}/\cs{DescribeEnv}/\cs{MacroArgs}.}\label{tab:macroargs}
% \begin{tabular}{llll}
%  \toprule
%   Description                &  Syntax       &  Result              & Macro$^{a}$ \\
%  \midrule
%   Meta text                  &  |<text>|     & \MacroArgs<text>     & \Macro\meta{<text>} \\
%   Mandatory Argument         &  |{args}|     & \MacroArgs{args}     & \\
%   ---, with meta text        &  |{<text>}|   & \MacroArgs{<text>}   & \Macro\marg{<text>} \\
%   Optional Argument          &  |[args]|     & \MacroArgs[args]     & \\
%   ---, with meta text        &  |[<text>]|   & \MacroArgs[<text>]   & \Macro\oarg{<text>} \\
%   Picture Argument           &  |(args)|     & \MacroArgs(args)     & \\
%   ---, with meta text        &  |(<text>)|   & \MacroArgs(<text>)   & \Macro\parg{<text>} \\
%   Beamer Overlay Argument    &  |<<args>>|   & \MacroArgs<<args>>   \\
%   ---, with meta text        &  |<< <text> >>| & \MacroArgs<<<text> >>   & \Macro\aarg{<text>} \\
%   Star                       &  |*|          & \MacroArgs*          \\
%  \midrule
%   Verbatim content           &  |'$&^%_#$\'| & \MacroArgs'$&^%_#$\' \\
%   ---, produce |'| char      &  |''|         & \MacroArgs''         \\
%   Insert any \TeX\ code      &  |!\fbox{T}!| & \MacroArgs!\fbox{T}! \\
%   Unbreakable Space          &  |~|          & \\
%   Space (explicit macro)     &  |\space|     & \\
%  \midrule
%   Second macro (e.g.\ endmarker) &  |\AlsoMacro\macro| & \MacroArgs\AlsoMacro\macro \\
%   \hfill short version:        &  \verb+|\macro+     & \MacroArgs|\macro \\
%  \bottomrule
%   \multicolumn{4}{l}{{\Large\strut}$^{a}$) As alternative to be used inside normal text.} \\
%   \multicolumn{4}{l}{Note that `|args|' can itself be further macro arguments except true verbatim.}
% \end{tabular}
% \end{table}
%
% \DescribeMacro\DescribeLength<\textbackslash name>{<default value>}
% This macro can be used to describe \LaTeX\ lengths also known as dimensions. Multiple \cs{DescribeLength} macros in a row will automatically be grouped.
%
% \subsection{Format Macros}
% \begin{DescribeMacrosTab}{ll}
%     \Macro\cs{<macro name>}    & \Macro\env{<environment name>} \\
%     \Macro\pkg{<package name>} & \Macro\cls{<class name>}       \\
% \end{DescribeMacrosTab}
% This macros can be used to format names of macros, environments, packages and classes, respectively.
% At the moment they simply use |\texttt|.
%
% \DescribeMacros\hbox{\Macro\bslash~~~~\Macro\percent~~~~\Macro\braceleft~~~~\Macro\braceright}\endDescribeMacros
% This macros define expandable backslash (\texttt{\bslash}$_{12}$), percent char (\texttt{\percent}$_{12}$),
% and left (\texttt{\braceleft}$_{12}$) and right (\texttt{\braceright}$_{12}$) braces with catcode 12 (other), respectively.
% They should only be used with text-typer font when used in text, because other fonts might not have the
% correct characters. The macros must be protected when used in a moving argument.
%
% \DescribeMacrosTab{@{}ll@{}}
%     \Macro\meta{<meta text>}     &
%     \Macro\marg{<argument text>} \\
%     \Macro\oarg{<argument text>} &
%     \Macro\parg{<argument text>} \\
%     \Macro\aarg{<argument text>} &
%     \Macro\sarg                  \\
% \endDescribeMacrosTab
% This macros allow to typeset meta text and mandatory, optional, picture and beamer overlay arguments as well as a star symbol.
% They are used internally by \cs{MacroArgs} and friends. See Table~\ref{tab:macroargs} for examples.
%
% \DescribeMacrosTab{ll}
%     \Macro\metastyle &
%     \Macro\margstyle \\
%     \Macro\oargstyle &
%     \Macro\pargstyle \\
%     \Macro\aargstyle &
%     \Macro\sargstyle \\
% \endDescribeMacrosTab
% This macros are used to define the style in which the corresponding macros above are being formatted.
% They are used like \MacroArgs '{'<\textbackslash stylemacro>{<material>}'}' to allow the styles to
% use macros like |\ttfamily| or |\texttt|\marg{material}. By default the optional argument and the also optional star are
% printed in the color `|optional|' which is a 65\% gray.
%
% \subsection{Settings}
% The following macro and dimensions can be redefined by the user to adjust the layout of the package documentation.
%
% \DescribeLength\descindent    {-20pt}
% \DescribeLength\beforedescskip{~12pt plus 4pt minus 4pt}
% \DescribeLength\afterdescskip {~~6pt plus 2pt minus 2pt}
% These length define the indention and vertical distances before and after a \cs{Describe\ldots} macro or environment, respectively.
%
% \DescribeLength\descsep       {1em in \texttt{tt} font = 10.5pt}
% This macro defines the space on the left and right side between the description text and the framed box.
%
% \subsection{Macros and Environments to include LaTeX Code Examples}
% \DescribeEnv{example}
% \DescribeEnv{examplecode}
%
% (to be written)
%
% \StopEventually{}
%
% \clearpage
% \section{Implementation}
%
% \iffalse
%<*ydoc.cls>
% \fi
% \subsection{Class File}
%
%    \begin{macrocode}
\NeedsTeXFormat{LaTeX2e}[1999/12/01]
\ProvidesClass{ydoc}[%
%<!DATE>
%<!VERSION>
%<*DRIVER>
    2011/08/11 develop
%</DRIVER>
    ydoc class: document LaTeX class and packages]
%    \end{macrocode}
%
% At the moment simply load \cls{article} class with \opt{a4paper} option
% and load the \pkg{ydoc} package.
%    \begin{macrocode}
\PassOptionsToClass{a4paper}{article}
\DeclareOption*{\expandafter\PassOptionsToClass\expandafter{\CurrentOption}{article}}
\ProcessOptions\relax
\LoadClass{article}
\RequirePackage{ydoc}
%    \end{macrocode}
% \iffalse
%</ydoc.cls>
% \fi
%
% \iffalse
%<*ydoc.sty>
% \fi
% \subsection{Package File}
%
%    \begin{macrocode}
\NeedsTeXFormat{LaTeX2e}[1999/12/01]
\ProvidesPackage{ydoc}[%
%<!DATE>
%<!VERSION>
%<*DRIVER>
    2011/08/11 develop
%</DRIVER>
    ydoc package: document LaTeX class and packages]
%    \end{macrocode}
%
%    \begin{macrocode}
\RequirePackage{svn-prov}[2010/04/03]
\RequirePackage{ydoc-code}
\RequirePackage{ydoc-expl}
\RequirePackage{ydoc-desc}
\RequirePackage{ydoc-doc}

\RequirePackage{newverbs}
\MakeSpecialShortVerb{\qverb}{\"}
\AtBeginDocument{\catcode`\^^A=14\relax}

% \iffalse
%<*driver>
\begingroup
\def\latexfmt{LaTeX2e}
\ifx\latexfmt\fmtname
\long\def\skip#1\NeedsTeXFormat{\endgroup\NeedsTeXFormat}
\expandafter\skip
\else
\endgroup
\let\NeedsTeXFormat\end
\fi
%</driver>
% \fi
%
% \iffalse
%<*ydocstrip.tex>
% \fi
%
% \iffalse
%    \begin{macrocode}
\expandafter\ifx\csname ydocstripversion\endcsname\relax\else
    \endinput
\fi

\chardef\ydocstripversion=1

\input docstrip


\def\checkOption<#1{%
  \ifcase
    \ifx*#10\else \ifx/#11\else
    \ifx+#12\else \ifx-#13\else
    \ifx<#14\else
    \ifx=#15\else
    \ifx!#16\else
    10\fi\fi\fi\fi\fi\fi\fi\relax
  \expandafter\starOption\or
  \expandafter\slashOption\or
  \expandafter\plusOption\or
  \expandafter\minusOption\or
  \expandafter\verbOption\or
  \expandafter\varOption\or
  \expandafter\valueOption\else
  \expandafter\doOption\fi
  #1}

\def\varOption =#1#2>#3\endLine{{%
    \ifx*#1\relax
        \edef\varStop{\perCent<=/#2>}%
        \global\expandafter\let\csname ydocstrip@var@#2\endcsname\empty%
        \expandafter\def\csname ydocstrip@var@#2\endcsname##1{}%
        \loop
            \ifeof\inFile
                \errmessage{Source file ended while reading a multi-line variable content!}%
            \fi
            \read\inFile to \inLine
        \if 1\ifx\inLine\varStop 0\fi 1% if not inLine==varStop
            \expandafter\xdef\csname ydocstrip@var@#2\endcsname{\csname ydocstrip@var@#2\endcsname^^J\inLine}%
            \maybeMsg{.}%
        \repeat
    \else
    \ifx/#1\relax
        \errmessage{Error in expression: spurious '<=/#2>'}%
    \else
    \ifx+#1\relax
        \expandafter\ifx\csname ydocstrip@var@#2\endcsname\relax
            \expandafter\xdef\csname ydocstrip@var@#2\endcsname{#3}%
        \else
            \expandafter\xdef\csname ydocstrip@var@#2\endcsname{\csname ydocstrip@var@#2\endcsname^^J#3}%
        \fi
    \else
        \expandafter\gdef\csname ydocstrip@var@#1#2\endcsname{#3}%
    \fi
    \fi
    \fi
}}


\def\valueOption !#1>#2\endLine{%
    \begingroup
    \expandafter\let\expandafter\var\csname ydocstrip@var@#1\endcsname
    \ifx\var\relax
        \errmessage{Used variable '#1' was never defined!}%
    \else
    \ifx\var\empty\else
        \def\do##1##2##3{%
            \StreamPut##1{\csname ydocstrip@var@#1\endcsname}%
        }%
        \activefiles
    \fi
    \fi
    \endgroup
}
%    \end{macrocode}
% \fi
% \iffalse
%</ydocstrip.tex>
% \fi
%
% \iffalse
%<*ydoc.ins>
% \fi
% \iffalse COPYRIGHT
%%
%% Copyright (c) 2010-2011 by Martin Scharrer <martin@scharrer-online.de>
%% WWW: http://latex.scharrer-online.de/
%%
%% This work may be distributed and/or modified under the
%% conditions of the LaTeX Project Public License, either version 1.3
%% of this license or (at your option) any later version.
%% The latest version of this license is in
%%   http://www.latex-project.org/lppl.txt
%% and version 1.3 or later is part of all distributions of LaTeX
%% version 2005/12/01 or later.
%%
%% This work has the LPPL maintenance status `maintained'.
%%
%% The Current Maintainer of this work is Martin Scharrer.
%%
%% This work consists of the files ydoc.dtx and ydoc.ins
%% and the derived file ydoc.cls, ydoc.sty, ydoc-code.sty,
%% ydoc-expl.sty and ydoc-desc.sty.
%%
% \fi
%
% \iffalse
\input ydocstrip

\preamble
\endpreamble

\keepsilent
\askforoverwritefalse

\generate{%
  \file{ydocstrip.tex}{\from{ydoc.dtx}{ydocstrip.tex}}%
  \file{ydocincl.tex} {\from{ydoc.dtx}{ydocincl.tex}}%
  \file{ydoc.ins}     {\from{ydoc.dtx}{ydoc.ins}}%
  \file{ydoc.cls}     {\from{ydoc.dtx}{ydoc.cls}}%
  \file{ydoc.sty}     {\from{ydoc.dtx}{ydoc.sty}}%
  \file{ydoc.cfg}     {\from{ydoc.dtx}{ydoc.cfg}}%
  \file{ydoc-code.sty}{\from{ydoc.dtx}{ydoc-code.sty}}%
  \file{ydoc-desc.sty}{\from{ydoc.dtx}{ydoc-desc.sty}}%
  \file{ydoc-expl.sty}{\from{ydoc.dtx}{ydoc-expl.sty}}%
  \file{ydoc-doc.sty} {\from{ydoc.dtx}{ydoc-doc.sty}}%
}

\begingroup
\obeyspaces
\Msg{*************************************************************}%
\Msg{*                                                           *}%
\Msg{* To finish the installation you have to move the following *}%
\Msg{* file into a directory searched by TeX:                    *}%
\Msg{*                                                           *}%
\Msg{*     ydoc.cls                                              *}%
\Msg{*     ydoc.sty                                              *}%
\Msg{*     ydoc-code.sty                                         *}%
\Msg{*     ydoc-desc.sty                                         *}%
\Msg{*     ydoc-expl.sty                                         *}%
\Msg{*     ydoc-doc.sty                                          *}%
\Msg{*     ydocstrip.tex                                         *}%
\Msg{*     ydocincl.tex                                          *}%
\Msg{*                                                           *}%
\Msg{* To produce the documentation run the file ydoc.dtx        *}%
\Msg{* through LaTeX.                                            *}%
\Msg{*                                                           *}%
\Msg{* Happy TeXing!                                             *}%
\Msg{*                                                           *}%
\Msg{*************************************************************}%
\endgroup%

\endbatchfile
% vim: ft=tex
% \fi
% \iffalse
%</ydoc.ins>
% \fi
%
% \iffalse
%<*driver>
\NeedsTeXFormat{LaTeX2e}[1999/12/01]
\ProvidesFile{ydoc.dtx}[%
%<=*DATE>
    2012/05/03
%<=/DATE>
%<=*VERSION>
    v0.6alpha
%<=/VERSION>
    DTX file of ydoc]

\message{^^JRun this DTX file using 'tex' to extract all files!^^J}
\documentclass{ydoc}

\newdimen\tempskip

\def\tablecaption{%
  \tempskip=\abovecaptionskip
  \abovecaptionskip=\belowcaptionskip
  \belowcaptionskip=\tempskip
  \caption
}

%\OnlyDescription
\begin{document}
  \DocInput{ydoc.dtx}
  \PrintChanges
  \PrintIndex
\end{document}
%</driver>
% \fi
%
% \CheckSum{0}
%
% \CharacterTable
%  {Upper-case    \A\B\C\D\E\F\G\H\I\J\K\L\M\N\O\P\Q\R\S\T\U\V\W\X\Y\Z
%   Lower-case    \a\b\c\d\e\f\g\h\i\j\k\l\m\n\o\p\q\r\s\t\u\v\w\x\y\z
%   Digits        \0\1\2\3\4\5\6\7\8\9
%   Exclamation   \!     Double quote  \"     Hash (number) \#
%   Dollar        \$     Percent       \%     Ampersand     \&
%   Acute accent  \'     Left paren    \(     Right paren   \)
%   Asterisk      \*     Plus          \+     Comma         \,
%   Minus         \-     Point         \.     Solidus       \/
%   Colon         \:     Semicolon     \;     Less than     \<
%   Equals        \=     Greater than  \>     Question mark \?
%   Commercial at \@     Left bracket  \[     Backslash     \\
%   Right bracket \]     Circumflex    \^     Underscore    \_
%   Grave accent  \`     Left brace    \{     Vertical bar  \|
%   Right brace   \}     Tilde         \~}
%
%
% \changes{v0.1alpha}{2010/04/10}{First released alpha version}
%
% \DoNotIndex{\newcommand,\newenvironment}
%
% \GetFileInfoSVN{ydoc.dtx}
%
% \title{The \textsf{\textbf{ydoc}} Class and Packages}
% \author{Martin Scharrer}
% \repository{https://bitbucket.org/martin_scharrer/ydoc/}
% \email{martin@scharrer.de}
% \makeatletter
% \date{Version \expandafter\@gobble\fileversion\\[0.5ex]\filedate}
% \makeatother
%
% \maketitle
%
% \ifpdf
% \hypersetup{%
%   pdfauthor   = {Martin Scharrer <martin@scharrer-online.de>},
%   pdftitle    = {The ydoc package, \fileversion, \filedate},
%   pdfsubject  = {Documentation of LaTeX package ydoc.},
%   pdfkeywords = {ydoc, documentation, LaTeX}
% }%
% \fi
%
%
% \begin{abstract}
% {\color{red}\bfseries This package bundle is currently under development.
% All functionality, settings and macro as well as file names can change in later versions and may be incomplete!
% It is not ready yet to be used for other packages.
% }
%
% The \pkg{ydoc} class and packages provide macros to document the functionality and implementation of \LaTeX{} classes and packages.
% It is similar to the \pkg{ltxdoc} class with the \pkg{doc} package, but uses more modern features/packages by default
% (e.g.\ \pkg{xcolor}, \pkg{hyperref}, \pkg{listings}). However, some of the features like code indexing is not yet included.
% \end{abstract}
%
% \section{Introduction}
% The \pkg{ydoc} packages allow the documentation of \LaTeX\ packages and classes.
% The name stands for ``\emph{Y}et another \emph{Doc}umentation Package'' and is a pun on
% the fact that there are several documentation packages written by package developers
% to document their own packages. All these packages didn't suited the author and therefore he,
% take a guess, wrote his own documentation package.
% It (will) support(s) all macros and environments (but not necessary with full/identical features) provided by the \pkg{doc}
% package to allow the fast adaption of existing |.dtx| files.
%
% This documentation uses the \pkg{ydoc} packages itself and therefore also acts as a live example.
%
% \subsection{\pkg{ydoc} Files}
% The \pkg{ydoc} bundle consists (at the moment, subject to change) of the \cls{ydoc} class and the packages \pkg{ydoc}, \pkg{ydoc-code},
% \pkg{ydoc-desc}, \pkg{ydoc-expl} and \pkg{ydoc-doc}. The \cls{ydoc} class and package allow the user the freedom to use the functionality with other classes if wanted.
% The class will load the package. The \pkg{ydoc} package loads the packages \pkg{ydoc-code}, \pkg{ydoc-desc}, \pkg{ydoc-expl} and \pkg{ydoc-doc}, which
% provide the functionality to document \LaTeX\ code implementation, describe the user-level macro, include live code examples and provide replacements for the
% macros of the \pkg{doc} package, respectively.
% This packages can be loaded on their own in other kind of \LaTeX\ documents if required.
%
% \subsection{Similar Packages}
% Other documentation related classes and packages are \pkg{ltxdoc}, \pkg{doc}, \pkg{dox}, \pkg{xdoc}, \pkg{gmdoc}, \pkg{pauldoc}, \pkg{hypdoc},
% \pkg{codedoc}, \pkg{nicetext} and \pkg{tkz-doc}.
%
%
% \section{Usage}
% (section incomplete)
%
% \subsection{Code Documentation Environments}
%
% \begin{DescribeEnv}{macro}{<macro>}[<\# of args>]!\optional!{<arg~1~description>}!\ldots!{<arg~$n$~description>}
%  \MacroArgs<macro documentation>\\
%  \begin{DescribeEnv}{macrocode}
%     \MacroArgs<macro code>
%  \end{DescribeEnv}\\
%  \ldots
% \end{DescribeEnv}
% The implementation of macros can be documented using this environment. The actual \meta{macro code} must be placed in a \env{macrocode} environment.
% Longer macro definition can be split using multiple \env{macrocode} environments with interleaved documentation texts.
%
% The \pkg{ydoc} definition of the \env{macro} environment has an additional feature compare to \pkg{doc}. The arguments of the macro (|#1|, |#2|, \ldots) can be
% documented in a vertical list. The environment has an optional argument to declare the \meta{number of arguments} the macro implementation has.
% The descriptions of this macro arguments are read from the next arguments of the environment. If the \meta{number of arguments} is not given or zero (or less)
% no further arguments are read by the \env{macro} environment.
%
% \begin{DescribeEnv}{macrocode}
%     \MacroArgs<macro code>
% \end{DescribeEnv}
% This environment wraps around any \TeX\ code and types it verbatim. The environment end is read verbatim as well and must be written on a separate
% line beginning with a percent followed by exactly four spaces: `\verb*+%    \end{macrocode}+'.
%
% \begin{DescribeEnv}{environment}{<name>}[<\# of args>]!\optional!{<arg~1~description>}!\ldots!{<arg~$n$~description>}
%  \MacroArgs<environment documentation>\\
%  \begin{DescribeEnv}{macrocode}
%     \MacroArgs<macro code>
%  \end{DescribeEnv}\\
%  \ldots
% \end{DescribeEnv}
% This environment provides the same functionality as the \env{macro} environment above, but for environments instead.
%
%
% \subsection{Description Macros and Environments}
%
% \DescribeMacro\DescribeMacro<\textbackslash macro><macro arguments>
% The \cs{DescribeMacro} is used to describe macros included their arguments.
% It takes the to be described \meta{\textbackslash macro} as first argument (can also be enclosed in |{ }|).
% The macro name can include `|@|'. Any number of \meta{macro arguments} (in a broad sense, see Table~\ref{tab:macroargs}) following it
% are formatted as arguments of this macro.
% Any following non-argument token (normal text, macro, etc.)\ will make \cs{DescribeMacro} stop collecting arguments.
% For example, if a \TeX\ group should be started using |{ }| direct after \cs{DescribeMacro} a |\relax| (or a similar macro) should be inserted
% between them, otherwise the group will be taken as mandatory argument of the described macro.
%
% Multiple \cs{DescribeMacro} in a row will automatically stacked inside one framed box. If this is not wanted simply separate them with |\relax|
% or any other macro or token. See also the \env{DescribeMacros} environment below.
%
% \subsubsection*{Examples:}
% \par\noindent |\DescribeMacro\mymacro*[<optional>]{<meta text>}| will result in\\
% \mbox{\Macro\mymacro*[<optional>]{<meta text>}} (inside a framed box).
% \par\medskip\noindent The above syntax description of \cs{DescribeMacro} itself was typeset with\\
% |\DescribeMacro\DescribeMacro<\textbackslash macro><macro arguments>|.
% \par\medskip\noindent Special macros with have a partner macro as end marker can be typeset like this:\\
% |\DescribeMacro\csname<text>\AlsoMacro\endcsname|, which will result in\\
% \mbox{\Macro\csname<text>\AlsoMacro\endcsname}.
%
% \DescribeMacro\Macro<\textbackslash macro><macro arguments>
% This macro is like an in-text version of \cs{DescribeMacro}.
% The macro description stays as part of the surrounding text and is not placed inside a framed box.
% The description can be broken between lines. This can be avoided by placing it inside a |\mbox{}|.
% \cs{Macro} is equivalent to \cs{MacroArgs}\cs{AlsoMacro}.
%
% \DescribeMacro\MacroArgs<macro arguments>
% This macro formats the \meta{macro arguments} the same way as \cs{DescribeMacro} and \cs{Macro} but without a macro name.
% Like \cs{Macro} the description is placed in-text.
%
% \DescribeMacro\AlsoMacro<\textbackslash macro><further macro arguments>
% This macro can only be used inside the \meta{macro arguments} of the above macros and typesets an additional macro as part
% of the syntax of the described macro. The additional macro is normally an end- or other marker of some kind. Further macro arguments may
% follow. Macros which are not part of the syntax but normal arguments should be written as
% |<\textbackslash name>| (yielding \MacroArgs<\textbackslash name>) instead.
% The `\verb+|+' character is an abbreviation of \Macro\AlsoMacro, but only at places where this can appear.
%
% \subsubsection*{Examples:}
% |\Macro\@for<\textbackslash var> ':=' <list> \AlsoMacro\do {<code>}|\\
%  \Macro\@for<\textbackslash var> ':=' <list> \AlsoMacro\do {<code>}
%  \par\medskip\noindent
% |\Macro\pgfkeys{<key1>'='<value1>','<key2>'/.code='{<code>}}|\\
%  \Macro\pgfkeys{<key1>'='<value1>','<key2>'/.code='{<code>}}\\
%
% \DescribeMacro\MakeShortMacroArgs*{<char>}
% This macro is similar to \Macro\MakeShortVerb from the \pkg{shortvrb} package.
% It can be used to globally define one character to act like \Macro\MacroArgs till the same character is discovered again.
% Special characters must be escaped with an backslash for the definition.
% One additional benefit beside the shorter size is that the argument list is automatically terminated.
% For example |\MakeShortMacroArgs{\"}| will make `|"<arg>{<arg>}"|' act like `|\MacroArgs<arg>{<arg>}\relax|'.
% One side-effect is that should the argument list be terminated, e.g. by an unknown element or macro, then the rest of the
% text till the end-character is typeset as normal, but inside a group.
%
% The starred version will define the character equal to \Macro\Macro instead.
%
% \DescribeMacro\DeleteShortMacroArgs{<char>}
% Globally removes the special meaning from \meta{char} given to him by \Macro\MakeShortMacroArgs.
%
% Note that special characters like |`| are
% best defined |\AtBeginDocument| and deleted again |\AtEndDocument| to avoid issues if they are written to the |aux| file by some package.
%
% \begin{DescribeEnv}{DescribeMacros}
%   \Macro\Macro<\textbackslash name><arguments>\\
%   \Macro\Macro<\textbackslash name><arguments>\\
%   \ldots
% \end{DescribeEnv}
% This environment can be used to place multiple macro description into the same framed box.
% The macros are described using \cs{Macro}, which has a slightly different definition than outside of this environment, to place the
% description into a |\hbox|. The environment stacks these |\hbox|es in a |\vbox|. The macros can also be placed freely using anything
% which produces a |\hbox|, e.g.\ |\hbox{\Macro\A ~~~ \Macro\B}| or using
% a |tabular| (see also \env{DescribeMacrosTab}).
%
% \DescribeEnv[<tabular content>]{DescribeMacrosTab}{<tabular column definition>}
% This is a special version of the \env{DescribeMacros} environment which adds a tabular environment around the content.
% This is useful if a large set of small macros should be desribed at once. Placing them all below each other would result in a very bad page layout.
% The environment has one argument which is passed to \env{tabular} as the column definition. A `|@{}|' is added before and after to remove any margins.
%
%
% \begin{DescribeEnv}{DescribeEnv}{<name>}<arguments>
%   \MacroArgs<body content>\relax\space |\\|\\
%   \MacroArgs<more body content>
% \end{DescribeEnv}
% \vspace{-\beforedescskip}
% \DescribeMacro\DescribeEnv[<body content>]{<name>}<arguments>
% The \env{DescribeEnv} can be used to describe environments in the same way the \cs{DescribeMacro} macro describes macros.
% Supported \meta{arguments} are shown in Table~\ref{tab:macroargs}.
% Potential \meta{body content} can be placed between the begin and end of the environment description to explain the user what kind of material
% should be placed inside it.
% The environment also exists in macro form as \cs{DescribeEnv}, which allows to provide small \meta{body content} as an optional argument.
% Please note that for this optional argument a \cs{MacroArgs} is automatically inserted, but not for the \cs{DescribeEnv} environment content.
%
% The body content is placed into a indented |\hbox{}| stacked inside a |\vbox{}| also holding the environment begin and end line.
% The |\\| macro is redefined to create a new indented |\hbox| acting as new code line.
% Therefore this environment is similar to a one-column \env{tabular}: all macros placed into a line are only valid up to the next line end.
%
% \begin{table}
% \tablecaption{Supported `arguments' for \cs{DescribeMacro}/\cs{DescribeEnv}/\cs{MacroArgs}.}\label{tab:macroargs}
% \begin{tabular}{llll}
%  \toprule
%   Description                &  Syntax       &  Result              & Macro$^{a}$ \\
%  \midrule
%   Meta text                  &  |<text>|     & \MacroArgs<text>     & \Macro\meta{<text>} \\
%   Mandatory Argument         &  |{args}|     & \MacroArgs{args}     & \\
%   ---, with meta text        &  |{<text>}|   & \MacroArgs{<text>}   & \Macro\marg{<text>} \\
%   Optional Argument          &  |[args]|     & \MacroArgs[args]     & \\
%   ---, with meta text        &  |[<text>]|   & \MacroArgs[<text>]   & \Macro\oarg{<text>} \\
%   Picture Argument           &  |(args)|     & \MacroArgs(args)     & \\
%   ---, with meta text        &  |(<text>)|   & \MacroArgs(<text>)   & \Macro\parg{<text>} \\
%   Beamer Overlay Argument    &  |<<args>>|   & \MacroArgs<<args>>   \\
%   ---, with meta text        &  |<< <text> >>| & \MacroArgs<<<text> >>   & \Macro\aarg{<text>} \\
%   Star                       &  |*|          & \MacroArgs*          \\
%  \midrule
%   Verbatim content           &  |'$&^%_#$\'| & \MacroArgs'$&^%_#$\' \\
%   ---, produce |'| char      &  |''|         & \MacroArgs''         \\
%   Insert any \TeX\ code      &  |!\fbox{T}!| & \MacroArgs!\fbox{T}! \\
%   Unbreakable Space          &  |~|          & \\
%   Space (explicit macro)     &  |\space|     & \\
%  \midrule
%   Second macro (e.g.\ endmarker) &  |\AlsoMacro\macro| & \MacroArgs\AlsoMacro\macro \\
%   \hfill short version:        &  \verb+|\macro+     & \MacroArgs|\macro \\
%  \bottomrule
%   \multicolumn{4}{l}{{\Large\strut}$^{a}$) As alternative to be used inside normal text.} \\
%   \multicolumn{4}{l}{Note that `|args|' can itself be further macro arguments except true verbatim.}
% \end{tabular}
% \end{table}
%
% \DescribeMacro\DescribeLength<\textbackslash name>{<default value>}
% This macro can be used to describe \LaTeX\ lengths also known as dimensions. Multiple \cs{DescribeLength} macros in a row will automatically be grouped.
%
% \subsection{Format Macros}
% \begin{DescribeMacrosTab}{ll}
%     \Macro\cs{<macro name>}    & \Macro\env{<environment name>} \\
%     \Macro\pkg{<package name>} & \Macro\cls{<class name>}       \\
% \end{DescribeMacrosTab}
% This macros can be used to format names of macros, environments, packages and classes, respectively.
% At the moment they simply use |\texttt|.
%
% \DescribeMacros\hbox{\Macro\bslash~~~~\Macro\percent~~~~\Macro\braceleft~~~~\Macro\braceright}\endDescribeMacros
% This macros define expandable backslash (\texttt{\bslash}$_{12}$), percent char (\texttt{\percent}$_{12}$),
% and left (\texttt{\braceleft}$_{12}$) and right (\texttt{\braceright}$_{12}$) braces with catcode 12 (other), respectively.
% They should only be used with text-typer font when used in text, because other fonts might not have the
% correct characters. The macros must be protected when used in a moving argument.
%
% \DescribeMacrosTab{@{}ll@{}}
%     \Macro\meta{<meta text>}     &
%     \Macro\marg{<argument text>} \\
%     \Macro\oarg{<argument text>} &
%     \Macro\parg{<argument text>} \\
%     \Macro\aarg{<argument text>} &
%     \Macro\sarg                  \\
% \endDescribeMacrosTab
% This macros allow to typeset meta text and mandatory, optional, picture and beamer overlay arguments as well as a star symbol.
% They are used internally by \cs{MacroArgs} and friends. See Table~\ref{tab:macroargs} for examples.
%
% \DescribeMacrosTab{ll}
%     \Macro\metastyle &
%     \Macro\margstyle \\
%     \Macro\oargstyle &
%     \Macro\pargstyle \\
%     \Macro\aargstyle &
%     \Macro\sargstyle \\
% \endDescribeMacrosTab
% This macros are used to define the style in which the corresponding macros above are being formatted.
% They are used like \MacroArgs '{'<\textbackslash stylemacro>{<material>}'}' to allow the styles to
% use macros like |\ttfamily| or |\texttt|\marg{material}. By default the optional argument and the also optional star are
% printed in the color `|optional|' which is a 65\% gray.
%
% \subsection{Settings}
% The following macro and dimensions can be redefined by the user to adjust the layout of the package documentation.
%
% \DescribeLength\descindent    {-20pt}
% \DescribeLength\beforedescskip{~12pt plus 4pt minus 4pt}
% \DescribeLength\afterdescskip {~~6pt plus 2pt minus 2pt}
% These length define the indention and vertical distances before and after a \cs{Describe\ldots} macro or environment, respectively.
%
% \DescribeLength\descsep       {1em in \texttt{tt} font = 10.5pt}
% This macro defines the space on the left and right side between the description text and the framed box.
%
% \subsection{Macros and Environments to include LaTeX Code Examples}
% \DescribeEnv{example}
% \DescribeEnv{examplecode}
%
% (to be written)
%
% \StopEventually{}
%
% \clearpage
% \section{Implementation}
%
% \iffalse
%<*ydoc.cls>
% \fi
% \subsection{Class File}
%
%    \begin{macrocode}
\NeedsTeXFormat{LaTeX2e}[1999/12/01]
\ProvidesClass{ydoc}[%
%<!DATE>
%<!VERSION>
%<*DRIVER>
    2011/08/11 develop
%</DRIVER>
    ydoc class: document LaTeX class and packages]
%    \end{macrocode}
%
% At the moment simply load \cls{article} class with \opt{a4paper} option
% and load the \pkg{ydoc} package.
%    \begin{macrocode}
\PassOptionsToClass{a4paper}{article}
\DeclareOption*{\expandafter\PassOptionsToClass\expandafter{\CurrentOption}{article}}
\ProcessOptions\relax
\LoadClass{article}
\RequirePackage{ydoc}
%    \end{macrocode}
% \iffalse
%</ydoc.cls>
% \fi
%
% \iffalse
%<*ydoc.sty>
% \fi
% \subsection{Package File}
%
%    \begin{macrocode}
\NeedsTeXFormat{LaTeX2e}[1999/12/01]
\ProvidesPackage{ydoc}[%
%<!DATE>
%<!VERSION>
%<*DRIVER>
    2011/08/11 develop
%</DRIVER>
    ydoc package: document LaTeX class and packages]
%    \end{macrocode}
%
%    \begin{macrocode}
\RequirePackage{svn-prov}[2010/04/03]
\RequirePackage{ydoc-code}
\RequirePackage{ydoc-expl}
\RequirePackage{ydoc-desc}
\RequirePackage{ydoc-doc}

\RequirePackage{newverbs}
\MakeSpecialShortVerb{\qverb}{\"}
\AtBeginDocument{\catcode`\^^A=14\relax}

\input{ydoc.cfg}
%    \end{macrocode}
% \iffalse
%</ydoc.sty>
% \fi
%
% \iffalse
%<*ydoc.cfg>
% \fi
% \subsection{Config File}
%    \begin{macrocode}
%% Please delete the following line on manual changes:
\ProvidesFile{ydoc.cfg}[%
%<!DATE>
%<!VERSION>
%<*DRIVER>
   2011/08/11 develop
%</DRIVER>
   Default config file for ydoc]
%    \end{macrocode}
%
%    \begin{macrocode}
\usepackage[T1]{fontenc}
\IfFileExists{fourier.sty}{%
    \usepackage{fourier}
}{}
%    \end{macrocode}
%
% Use 'lmodern' only for the 'tt' font if fourier is installed.
%    \begin{macrocode}
\IfFileExists{lmodern.sty}{
    \IfFileExists{fourier.sty}{
        \renewcommand{\ttdefault}{lmtt}
    }{
        \usepackage{lmodern}
    }
}{}
%    \end{macrocode}
%
%
%    \begin{macrocode}
\urlstyle{sf}
%    \end{macrocode}
%
% Use micro-typesetting if pdftex is used:
%    \begin{macrocode}
\usepackage{ifpdf}
\ifpdf
\usepackage{microtype}
\fi
%    \end{macrocode}
%    \begin{macrocode}
\usepackage{array}
\usepackage{booktabs}
\usepackage{multicol}
\usepackage{xcolor}
\usepackage{listings}
\usepackage{booktabs}
\usepackage{hyperref}
%    \end{macrocode}
%
%    \begin{macrocode}
\reversemarginpar
%    \end{macrocode}
%
% \iffalse
%</ydoc.cfg>
% \fi
%
% \iffalse
%<*ydoc-code.sty>
% \fi
% \subsection{Macros and Environments to document Implementations}
%
%    \begin{macrocode}
\NeedsTeXFormat{LaTeX2e}[1999/12/01]
\ProvidesPackage{ydoc-code}[%
%<!DATE>
%<!VERSION>
%<*DRIVER>
    2011/08/11 develop
%</DRIVER>
    ydoc package to document macro code]
%    \end{macrocode}
%
%    \begin{macrocode}
\RequirePackage{hyperref}
\hypersetup{colorlinks=true,pdfborder=0 0 0,pdfborderstyle={}}
%    \end{macrocode}
%
%    \begin{macrocode}
\IfFileExists{needspace.sty}{%
    \RequirePackage{needspace}
}{%
    \def\Needspace{\@ifstar\@gobble\@gobble}
}
%    \end{macrocode}
%
% \subsubsection{Color and style definitions}
%    \begin{macrocode}
\RequirePackage{xcolor}
\definecolor{macroimpl}{rgb}{0.0,0.0,0.4}
%    \end{macrocode}
%
% \subsubsection{General Macros}
%
% \begin{macro}{\ydocwrite}
%    \begin{macrocode}
\@ifundefined{ydocwrite}{%
  \newwrite\ydocwrite
}{}
%    \end{macrocode}
% \end{macro}
%
% \begin{macro}{\ydocfname}
%    \begin{macrocode}
\@ifundefined{ydocfname}{%
  \def\ydocfname{\jobname.cod}%
}{}
%    \end{macrocode}
% \end{macro}
%
% \begin{macro}{\ydoc@catcodes}
%    \begin{macrocode}
\def\ydoc@catcodes{%
  \let\do\@makeother
  \dospecials
  \catcode`\\=\active
  \catcode`\^^M=\active
  \catcode`\ =\active
}
%    \end{macrocode}
% \end{macro}
%
% \subsubsection{Handling Macrocode}
% \begin{environment}{macrocode}
%    \begin{macrocode}
\def\macrocode{%
  \par\noindent
  \begingroup
  \ydoc@catcodes
  \macro@code
}
\def\endmacrocode{}
%    \end{macrocode}
% \end{environment}
%
% \begin{macro}{\macro@code}[1]{verbatim macro code}
%    \begin{macrocode}
\begingroup
\endlinechar\m@ne
\@firstofone{%
\catcode`\|=0\relax
\catcode`\(=1\relax
\catcode`\)=2\relax
\catcode`\*=14\relax
\catcode`\{=12\relax
\catcode`\}=12\relax
\catcode`\ =12\relax
\catcode`\%=12\relax
\catcode`\\=\active
\catcode`\^^M=\active
\catcode`\ =\active
}*
|gdef|macro@code#1^^M%    \end{macrocode}(*
|endgroup|expandafter|macro@@code|expandafter(|ydoc@removeline#1|noexpand|lastlinemacro)*
)*
|gdef|ydoc@removeline#1^^M(|noexpand|firstlinemacro)*
|gdef|ydoc@defspecialmacros(*
|def^^M(|noexpand|newlinemacro)*
|def (|noexpand|spacemacro)*
|def\(|noexpand|bslashmacro)*
)*
|gdef|ydoc@defrevspecialmacros(*
|def|newlinemacro(|noexpand^^M)*
|def|spacemacro(|noexpand )*
|def|bslashmacro(|noexpand\)*
)*
|endgroup
%    \end{macrocode}
% \end{macro}
%
% \begin{macro}{\macro@@code}[1]{verbatim macro code}
%    \begin{macrocode}
\def\macro@@code#1{%
  {\ydoc@defspecialmacros
  \xdef\themacrocode{#1}}%
  \PrintMacroCode
  \end{macrocode}%
}
%    \end{macrocode}
% \end{macro}
%
% \begin{macro}{\linenumberbox}
%    \begin{macrocode}
\def\newlinemacro{\\\null}
\def\spacemacro{\ }
\def\bslashmacro{\char92}
\def\lastlinemacro{}
\def\firstlinemacro{\linenumberbox}
\def\newlinemacro{\\\linenumberbox}
\newcounter{linenumber}
\def\linenumberbox{%
  \hbox to 1.25em{}%
  \llap{%
    \stepcounter{linenumber}%
    {\footnotesize\color{gray}\thelinenumber~}%
  }%
}
%    \end{macrocode}
% \end{macro}
%
% \begin{macro}{\PrintMacroCode}
%    \begin{macrocode}
\def\PrintMacroCode{%
  \begingroup
  \ttfamily
  \noindent\themacrocode
  \endgroup
}
%    \end{macrocode}
% \end{macro}
%
% \begin{macro}{\PrintMacroCode}
%    \begin{macrocode}
\RequirePackage{listings}
%    \end{macrocode}
%    \begin{macrocode}
\def\PrintMacroCode{%
  \begingroup
  \let\firstlinemacro\empty
  \let\lastlinemacro\empty
  \def\newlinemacro{^^J}%
  \let\bslashmacro\bslash
  \let\spacemacro\space
  \immediate\openout\ydocwrite=\ydocfname\relax
  \immediate\write\ydocwrite{\themacrocode}%
  \immediate\closeout\ydocwrite
  \@nameuse{ydoc@countbslashes}%
  \ydoclistingssettings
  \let\input\@input
  \lstinputlisting{\ydocfname}%
  \endgroup
}
%    \end{macrocode}
% \end{macro}
%
%    \begin{macrocode}
\lstdefinestyle{ydoccode}{%
    language=[latex]tex,basicstyle=\ttfamily,
    numbers=left,numberstyle=\tiny\color{gray},firstnumber=last,
    breaklines,prebreak={\mbox{\tiny$\swarrow$}},
    commentstyle=\color{black!60},
}%
%    \end{macrocode}
%
% \begin{macro}{\ydoclistingssettings}
%    \begin{macrocode}
\def\ydoclistingssettings{%
    \lstset{style=ydoccode}%
}
%    \end{macrocode}
% \end{macro}
%
% \begin{macro}{\macro@impl@args}[1]{number of macro arguments}
%    \begin{macrocode}
\def\macro@impl@args[#1]{%
  \begingroup
  \parindent=10pt\relax
  \let\macro@impl@argcnt\@tempcnta
  \let\macro@impl@curarg\@tempcntb
  \macro@impl@argcnt=#1\relax
  \macro@impl@curarg=0\relax
  \ifnum\macro@impl@curarg<\macro@impl@argcnt\relax
    \expandafter\macro@impl@arg
  \else
    \expandafter\macro@impl@endargs
  \fi
}
%    \end{macrocode}
% \end{macro}
%
% \begin{macro}{\macro@impl@endargs}
%    \begin{macrocode}
\def\macro@impl@endargs{%
  \endgroup
  \unskip\par\noindent\ignorespaces
}
%    \end{macrocode}
% \end{macro}
%
% \begin{macro}{\macro@impl@argline}[2]{argument number}{argument description}
%    \begin{macrocode}
\def\macro@impl@argline#1#2{%
  \par{\texttt{\##1}:~#2\strut}%
}
%    \end{macrocode}
% \end{macro}
%
% \begin{macro}{\macro@impl@arg}[1]{argument description}
%    \begin{macrocode}
\def\macro@impl@arg#1{%
  \advance\macro@impl@curarg by\@ne\relax
  \macro@impl@argline{\the\macro@impl@curarg}{#1}%
  \ifnum\macro@impl@curarg<\macro@impl@argcnt\relax
    \expandafter\macro@impl@arg
  \else
    \expandafter\macro@impl@endargs
  \fi
}
%    \end{macrocode}
% \end{macro}
%
% \begin{environment}{macro}[1]{implemented macro}
%    \begin{macrocode}
\def\macro#1{%
  \PrintMacroImpl{#1}%
  \@ifnextchar[%]
    {\macro@impl@args}%
    {}%
}
\def\endmacro{}
%    \end{macrocode}
% \end{environment}
%
% \begin{environment}{key}[2]{key family}{key name}
%    \begin{macrocode}
\def\key#1#2{%
  \PrintMacroImpl{KV@#1@#2}%
  \@ifnextchar[%]
    {\macro@impl@args}%
    {}%
}
\def\endkey{}
%    \end{macrocode}
% \end{environment}
%
% \begin{environment}{environment}[1]{environment name}
%    \begin{macrocode}
\def\environment#1{%
  \PrintEnvImplName{#1}%
  \@ifnextchar[%]
    {\macro@impl@args}%
    {}%
}
\def\endenvironment{}
%    \end{macrocode}
% \end{environment}
%
%
% \begin{environment}{style}[1]{style name}
%    \begin{macrocode}
\def\style#1{%
  \PrintStyleImplName{#1}%
  \@ifnextchar[%]
    {\macro@impl@args}%
    {}%
}
\def\endstyle{}
\def\PrintStyleImplName{\PrintEnvImplName}
%    \end{macrocode}
% \end{environment}
%
% \begin{macro}{\PrintMacroImpl}[1]{macro (token)}
%    \begin{macrocode}
\def\PrintMacroImpl#1{%
  \par\bigskip\noindent
  \Needspace*{3\baselineskip}%
  \hbox{%
    \edef\name{\expandafter\@gobble\string#1}%
    \global\@namedef{href@impl@\name}{}%
    \immediate\write\@mainaux{%
      \global\noexpand\@namedef{href@impl@\name}{}%
    }%
    \raisebox{4ex}[4ex]{\hypertarget{impl:\name}{}}%
    \hspace*{\descindent}\fbox{%
      \hspace*{\descsep}%
      \@ifundefined{href@desc@\name}{}{\hyperlink{desc:\name}}%
      {\PrintMacroImplName{#1}}%
      \hspace*{\descsep}%
    }%
  }%
  \par\medskip\noindent
}
%    \end{macrocode}
% \end{macro}
%
% \begin{macro}{\PrintMacroImplName}[1]{macro (token)}
%    \begin{macrocode}
\def\PrintMacroImplName#1{%
  \implstyle{\string#1\strut}%
}
%    \end{macrocode}
% \end{macro}
%
% \begin{macro}{\PrintEnvImplName}[1]{environment name}
% test
%    \begin{macrocode}
\def\PrintEnvImplName#1{%
  \par\bigskip\noindent
  \hbox{\hspace*{\descindent}\fbox{{\implstyle{#1}}}}%
  \par\medskip
}
%    \end{macrocode}
% \end{macro}
%
% \begin{macro}{\implstyle}
%    \begin{macrocode}
\def\implstyle{\ttfamily\bfseries\color{macroimpl}}
%    \end{macrocode}
% \end{macro}
%
% \begin{macro}{\bslash}
% Defines an expandable backslash with catcode 12: `\texttt{\bslash}$_{12}$'.
% The |\@firstofone| trick is used to read the |\gdef\bslash| code before changing the catcode.
%    \begin{macrocode}
{%
\@firstofone
%    \end{macrocode}
% \end{macro}
% \iffalse
%</ydoc-code.sty>
% \fi
%
% \iffalse
%<*ydoc-doc.sty>
% \fi
% \subsection{Provide \pkg{doc} macros}
%    \begin{macrocode}
\NeedsTeXFormat{LaTeX2e}[1999/12/01]
\ProvidesPackage{ydoc-doc}[%
%<!DATE>
%<!VERSION>
%<*DRIVER>
    2099/01/01 develop
%</DRIVER>
    ydoc package to provide 'doc' macros]
%    \end{macrocode}
%
% \begin{macro}{\ydoc@countbslashes}
% Reads the macro code into a temp box. The backslashes are defined to increase a counter.
%    \begin{macrocode}
\newcount\ydoc@bslashcnt
\def\ydoc@countbslashes{%
  \begingroup
    \let\firstlinemacro\empty
    \let\lastlinemacro\empty
    \let\newlinemacro\empty
    \let\spacemacro\empty
    \def\bslashmacro{\global\advance\ydoc@bslashcnt by\@ne}%
    \setbox\@tempboxa\hbox{\themacrocode}%
  \endgroup
}
%    \end{macrocode}
% \end{macro}
%
% \begin{macro}{\CheckSum}
%    \begin{macrocode}
\def\CheckSum#1{%
  \gdef\ydoc@checksum{#1}%
}
\let\ydoc@checksum\m@ne
%    \end{macrocode}
% \end{macro}
%
% \begin{macro}{\AlsoImplementation}
% \begin{macro}{\OnlyDescription}
% \begin{macro}{\StopEventually}
% \begin{macro}{\Finale}
% The first two macros modify the \cs{StopEventually} macro which either
% stores its argument in \cs{Final} or executes it itself.
%    \begin{macrocode}
\def\AlsoImplementation{%
  \gdef\StopEventually##1{%
    \@bsphack
    \gdef\Finale{##1\ydoc@checkchecksum}%
    \@esphack
  }%
}
\AlsoImplementation
\def\OnlyDescription{%
  \@bsphack
  \long\gdef\StopEventually##1{##1\endinput}%
  \@esphack
}
\let\Finale\relax
%    \end{macrocode}
% \end{macro}
% \end{macro}
% \end{macro}
% \end{macro}
%
% \begin{macro}{\MakePercentComment}
% \begin{macro}{\MakePercentIgnore}
%    \begin{macrocode}
\def\MakePercentIgnore{\catcode`\%9\relax}
\def\MakePercentComment{\catcode`\%14\relax}
%    \end{macrocode}
% \end{macro}
% \end{macro}
%
% \begin{macro}{\DocInput}
%    \begin{macrocode}
\def\DocInput#1{\MakePercentIgnore\input{#1}\MakePercentComment}
%    \end{macrocode}
% \end{macro}
%
% \begin{macro}{\CharacterTable}
%    \begin{macrocode}
\providecommand*\CharacterTable{%
    \begingroup
    \CharTableChanges
    \@CharacterTable
}
\def\@CharacterTable#1{%
    \def\ydoc@used@CharacterTable{#1}%
    \@onelevel@sanitize\ydoc@used@CharacterTable
    \ifx\ydoc@used@CharacterTable\ydoc@correct@CharacterTable
           \typeout{***************************}%
           \typeout{* Character table correct *}%
           \typeout{***************************}%
    \else
         \PackageError{ydoc}{Character table corrupted}
                           {\the\wrong@table}
         \show\ydoc@used@CharacterTable
         \show\ydoc@correct@CharacterTable
    \fi
    \endgroup
}
\newhelp\wrong@table{Some of the ASCII characters are corrupted.^^J
            I now \string\show\space you both tables for comparison.}
\newcommand*\CharTableChanges{}
%    \end{macrocode}
% \end{macro}
%
% \begin{macro}{\ydoc@correct@CharacterTable}
%    \begin{macrocode}
\def\ydoc@correct@CharacterTable
  {Upper-case    \A\B\C\D\E\F\G\H\I\J\K\L\M\N\O\P\Q\R\S\T\U\V\W\X\Y\Z
   Lower-case    \a\b\c\d\e\f\g\h\i\j\k\l\m\n\o\p\q\r\s\t\u\v\w\x\y\z
   Digits        \0\1\2\3\4\5\6\7\8\9
   Exclamation   \!     Double quote  \"     Hash (number) \#
   Dollar        \$     Percent       \%     Ampersand     \&
   Acute accent  \'     Left paren    \(     Right paren   \)
   Asterisk      \*     Plus          \+     Comma         \,
   Minus         \-     Point         \.     Solidus       \/
   Colon         \:     Semicolon     \;     Less than     \<
   Equals        \=     Greater than  \>     Question mark \?
   Commercial at \@     Left bracket  \[     Backslash     \\
   Right bracket \]     Circumflex    \^     Underscore    \_
   Grave accent  \`     Left brace    \{     Vertical bar  \|
   Right brace   \}     Tilde         \~}
\@onelevel@sanitize\ydoc@correct@CharacterTable
%
%    \end{macrocode}
% \end{macro}
%
% \begin{macro}{\DoNotIndex}
%    \begin{macrocode}
\providecommand*\DoNotIndex[1]{%
  \PackageWarning{ydoc}{Ignoring DoNotIndex - not implemented yet!}{}{}%
}
%    \end{macrocode}
% \end{macro}
%
% \begin{macro}{\changes}
%    \begin{macrocode}
\providecommand*\changes[3]{%
  \PackageWarning{ydoc}{Ignoring changes - not implemented yet!}{}{}%
}
%    \end{macrocode}
% \end{macro}
%
% \begin{macro}{\RecordChanges}
%    \begin{macrocode}
\providecommand*\RecordChanges{%
  \PackageWarning{ydoc}{List of changes not implemented yet!}{}{}%
}
%    \end{macrocode}
% \end{macro}
%
% \begin{macro}{\PrintChanges}
%    \begin{macrocode}
\providecommand*\PrintChanges{%
  \PackageWarning{ydoc}{List of changes not implemented yet!}{}{}%
}
%    \end{macrocode}
% \end{macro}
%
% \begin{macro}{\PrintIndex}
%    \begin{macrocode}
\providecommand*\PrintIndex{%
  \PackageWarning{ydoc}{Code index not implemented yet!}{}{}%
}
%    \end{macrocode}
% \end{macro}
%
% \begin{macro}{\CodelineIndex}
%    \begin{macrocode}
\providecommand*\CodelineIndex{%
  \PackageWarning{ydoc}{Code line index not implemented yet!}{}{}%
}
%    \end{macrocode}
% \end{macro}
%
% \begin{macro}{\EnableCrossrefs}
%    \begin{macrocode}
\providecommand*\EnableCrossrefs{%
  \PackageWarning{ydoc}{Cross references not implemented yet!}{}{}%
}
%    \end{macrocode}
% \end{macro}
%
% \begin{macro}{\GetFileInfo}
% Current implementation taken from \pkg{doc} package.
%    \begin{macrocode}
\providecommand*\GetFileInfo[1]{%
  \def\filename{#1}%
  \def\@tempb##1 ##2 ##3\relax##4\relax{%
    \def\filedate{##1}%
    \def\fileversion{##2}%
    \def\fileinfo{##3}}%
  \edef\@tempa{\csname ver@#1\endcsname}%
  \expandafter\@tempb\@tempa\relax? ? \relax\relax
}
%    \end{macrocode}
% \end{macro}
%
% \begin{macro}{\ydoc@checkchecksum}
%    \begin{macrocode}
\def\ydoc@checkchecksum{%
  \ifnum\ydoc@checksum=\m@ne
    \message{^^J}%
    \message{**********************^^J}%
    \message{* No checksum found! *^^J}%
    \message{**********************^^J}%
    \GenericWarning{No checksum found}{Correct checksum is \the\ydoc@bslashcnt^^J}{}{}%
  \else
  \ifnum\ydoc@checksum=\z@
    \message{^^J}%
    \message{*********************^^J}%
    \message{* Checksum disabled *^^J}%
    \message{*********************^^J}%
    \GenericWarning{Checksum disabled}{Correct checksum is \the\ydoc@bslashcnt^^J}{}{}%
  \else
  \ifnum\ydoc@checksum=\ydoc@bslashcnt
    \message{^^J}%
    \message{*******************^^J}%
    \message{* Checksum passed *^^J}%
    \message{*******************^^J}%
  \else
    \message{^^J}%
    \message{****************************^^J}%
    \message{* Checksum wrong (\ydoc@checksum<>\the\ydoc@bslashcnt) ^^J}%
    \message{****************************^^J}%
    \GenericError{Checksum wrong}{Correct checksum is \the\ydoc@bslashcnt^^J}{}{}%
  \fi
  \fi
  \fi
}
%    \end{macrocode}
% \end{macro}
%
%    \begin{macrocode}
\RequirePackage{shortvrb}
\AtBeginDocument{\MakeShortVerb{\|}}
%    \end{macrocode}
%
%    \begin{macrocode}
\RequirePackage{url}

\def\package{\def\@package}
\package{\jobname}

\def\bundle{\def\@bundle}
\let\@bundle\@empty


\def\ctanlocation{\def\@ctanlocation##1}
\ctanlocation{http://www.ctan.org/pkg/#1}

\date{Version \fileversion\space -- \filedate}

\def\@homepage{%
    \begingroup
    \edef\@tempa{%
        \endgroup
        CTAN:
        \noexpand\url
        {\@ctanlocation{\ifx\@bundle\@empty\@package\else\@bundle\fi}}%
    }%
    \@tempa
}

\let\@repository\@empty
\protected\def\repository{\urldef\@repository\url}
\protected\def\homepage{\urldef\@homepage\url}
\protected\def\email{\hyper@normalise\email@}
\def\email@#1{\def\@plainemail{#1}\def\@email{\hyper@linkurl{\Hurl{#1}}{mailto:#1}}}

\let\@email\empty

\let\@plainemail\empty
\title{The \texorpdfstring{\pkgtitle{\@package}}{\@package} Package}
\def\@bundlesubtitle{Part of the \texorpdfstring{\pkgtitle{\@bundle}}{\@bundle} bundle}

\protected\def\pkgtitle#1{%
    \texorpdfstring{\textsf{#1}}{#1}%
}



\def\@maketitle{%
    \newpage
    \null\vskip 2em
    \begin{center}%
        \let\footnote\thanks
        {\LARGE \@title \par }\vskip 1.5em%
        \ifx\@bundle\@empty\else
        {\large \@bundlesubtitle \par }\vskip 1.5em%
        \fi
        {\large \lineskip .5em%
        \begin{tabular}[t]{c}%
            \@author
        \end{tabular}%
        \par}%
        \ifx\@plainemail\empty\else
            {\large \lineskip .5em%
            \begin{tabular}[t]{c}%
                \@email
            \end{tabular}%
            \par}%
        \fi
        \vskip 1em
        {\large \lineskip .5em%
        \begin{tabular}[t]{c}%
            \@homepage
        \end{tabular}%
        \par}%
        \vskip 1em
        \ifx\@repository\@empty\else
            {\large \lineskip .5em%
            \begin{tabular}[t]{c}%
                VC: \@repository
            \end{tabular}%
            \par}%
        \fi
        \vskip 1em
        {\large \@date }%
    \end{center}%
    \par\vskip 1.5em
    \aftergroup\ydocpdfsettings
}

\ifpdf
\def\ydocpdfsettings{%
    \hypersetup{%
        pdfauthor   = {\@author\space<\@plainemail>},
        pdftitle    = {\@title},
        pdfsubject  = {Documentation of LaTeX package \@package},
        pdfkeywords = {\@package, LaTeX, TeX}
    }%
}
\else
\let\ydocpdfsettings\empty
\fi

\let\orig@maketitle\maketitle
\def\maketitle{%
    \ydocpdfsettings
    \orig@maketitle
    \let\orig@maketitle\relax
}
%    \end{macrocode}
% \iffalse
%</ydoc-doc.sty>
% \fi
%
% \iffalse
%<*ydoc-desc.sty>
% \fi
% \subsection{Description Macros and Environments}
%
%    \begin{macrocode}
\NeedsTeXFormat{LaTeX2e}[1999/12/01]
\ProvidesPackage{ydoc-desc}[%
%<!DATE>
%<!VERSION>
%<*DRIVER>
    2099/01/01 develop
%</DRIVER>
    ydoc package to describe macros, environments, options etc.]
%    \end{macrocode}
%
%    \begin{macrocode}
\IfFileExists{needspace.sty}{%
    \RequirePackage{needspace}
}{%
    \def\Needspace{\@ifstar\@gobble\@gobble}
}
%    \end{macrocode}
%
% The short verbatim code is required for the similar macros provided here.
%    \begin{macrocode}
\RequirePackage{shortvrb}
%    \end{macrocode}
%
% The \pkg{etoolbox} package is used mainly for |\newrobustcmd|.
%    \begin{macrocode}
\RequirePackage{etoolbox}
%    \end{macrocode}
%
% \subsubsection{Color and style definitions}
%    \begin{macrocode}
\RequirePackage{xcolor}
%    \end{macrocode}
% Define special no-op `none' color which does not change the color.
% This is not yet tested and may break output files, but seems to work fine with PDF.
%    \begin{macrocode}
\expandafter\def\csname\string\color@none\endcsname{%
    \xcolor@ {}{}{}{}
}
%    \end{macrocode}
%    \begin{macrocode}
\definecolor{macrodesc}{rgb}{0,0.2,0.6}
\definecolor{keydesc}{rgb}{0,0.4,0.9}
\definecolor{macroimpl}{rgb}{0,0.1,0.3}
\definecolor{meta}{rgb}{0,0.25,0.75}
\definecolor{scriptcolor}{rgb}{0.2,0.6,0.2}
\definecolor{optioncolor}{rgb}{0.3.0.2,0}
\colorlet{optional}{black!65!white}
\colorlet{metaoptional}{optional!50!meta}
\providecolor{urlcolor}{named}{blue}
\providecolor{linkcolor}{named}{blue}
\providecolor{filecolor}{named}{blue}
\providecolor{citecolor}{named}{blue}
\providecolor{anchorcolor}{named}{blue}
\providecolor{menucolor}{named}{blue}
\providecolor{runcolor}{named}{blue}

%    \end{macrocode}
%
%    \begin{macrocode}
\RequirePackage{hyperref}
\hypersetup{%
    colorlinks=true,
    pdfborder=0 0 0,
    pdfborderstyle={},
    urlcolor=urlcolor,
    linkcolor=linkcolor,
    filecolor=filecolor,
    citecolor=citecolor,
    anchorcolor=anchorcolor,
    menucolor=menucolor,
    runcolor=runcolor,
}

%    \end{macrocode}
%
% \subsubsection{Text Formatting Macros}
%
% \begin{macro}{\meta}
% Prints \meta{meta text}.
%    \begin{macrocode}
\newrobustcmd*\meta[1]{%
  {\metastyle{%
  \ensuremath\langle
    #1\/%
  \ensuremath\rangle
  }}%
}
%    \end{macrocode}
% \end{macro}
%
% \begin{macro}{\marg}
% Sets style and adds braces.
% The text is formatted as separate set of macro arguments.
%    \begin{macrocode}
\newrobustcmd*{\marg}[1]{%
  {\margstyle{%
    {\ttfamily\braceleft}%
    \meta{#1}%
    {\ttfamily\braceright}%
  }}%
}
%    \end{macrocode}
% \end{macro}
%
% \begin{macro}{\oarg}
% Sets style and adds brackets.
% The text is formatted as separate set of macro arguments.
%    \begin{macrocode}
\newrobustcmd*{\oarg}[1]{%
  {\oargstyle{%
    {\ttfamily[}%
    \meta{#1}%
    {\ttfamily]}%
  }}%
}
%    \end{macrocode}
% \end{macro}
%
% \begin{macro}{\parg}
% Sets style and adds parentheses.
%    \begin{macrocode}
\newrobustcmd*{\parg}[1]{%
  {\pargstyle{%
    {\ttfamily(}%
    \meta{#1}%
    {\ttfamily)}%
  }}%
}
%    \end{macrocode}
% \end{macro}
%
% \begin{macro}{\aarg}
% Sets style and adds angles.
%    \begin{macrocode}
\newrobustcmd*{\aarg}[1]{%
  {\aargstyle{%
    {\ttfamily<}%
    \meta{#1}%
    {\ttfamily>}%
  }}%
}
%    \end{macrocode}
% \end{macro}
%
% \begin{macro}{\sarg}
% Prints star with given style.
%    \begin{macrocode}
\newrobustcmd*{\sarg}{{\sargstyle{*}}}
%    \end{macrocode}
% \end{macro}
%
% \begin{macro}{\pkg}
% \begin{macro}{\cls}
% \begin{macro}{\lib}
% \begin{macro}{\env}
% \begin{macro}{\opt}
% \begin{macro}{\file}
%    \begin{macrocode}
\newrobustcmd*\pkg[1]{{\pkgstyle{#1}}}
\newrobustcmd*\cls[1]{{\clsstyle{#1}}}
\newrobustcmd*\lib[1]{{\libstyle{#1}}}
\newrobustcmd*\env[1]{{\envstyle{#1}}}

\newrobustcmd*\opt{\@ifstar\ys@opt\y@opt}
\def\y@opt#1{{\optstyle{#1}}}
\def\ys@opt#1{{\optstyle{#1}}\optpar{#1}}
\newrobustcmd*\optpar[1]{\marginpar{\hbox to \marginparwidth{\hss\y@opt{#1}}}}

\newrobustcmd*\file[1]{{\filestyle{#1}}}
\newcommand*\pkgstyle[1]{\texttt{\textcolor{pkg}{#1}}}
\newcommand*\clsstyle[1]{\texttt{\textcolor{cls}{#1}}}
\newcommand*\libstyle[1]{\texttt{\textcolor{lib}{#1}}}
\newcommand*\envstyle[1]{\texttt{\textcolor{env}{#1}}}
\newcommand*\optstyle[1]{\textsf{\textcolor{opt}{#1}}}
\newcommand*\filestyle[1]{\texttt{\textcolor{file}{#1}}}
\colorlet{cls}{none}
\colorlet{lib}{none}
\colorlet{env}{none}
\colorlet{file}{none}
\colorlet{pkg}{none}
\definecolor{opt}{rgb}{0.5,0.16666,0}
%    \end{macrocode}
% \end{macro}
% \end{macro}
% \end{macro}
% \end{macro}
% \end{macro}
% \end{macro}
%
% \begin{macro}{\cs}
% \begin{macro}{\cmd}
%    \begin{macrocode}
\newrobustcmd*\cs[1]{\texttt{\textbackslash #1}}
\newrobustcmd*\cmd[1]{\texttt{{\escapechar=92\string#1}}}
%    \end{macrocode}
% \end{macro}
% \end{macro}
%
% \begin{macro}{\Key}
%    \begin{macrocode}
\newrobustcmd*\Key[1]{\PrintKeyName{#1}\MacroArgs}
%    \end{macrocode}
% \end{macro}
%
% \subsubsection{Text Formatting Styles}
%
% \begin{macro}{\macrodescstyle}
% Style of described macro names.
%    \begin{macrocode}
\def\macrodescstyle{\ttfamily\bfseries\color{macrodesc}}
%    \end{macrocode}
% \end{macro}
%
% \begin{macro}{\macrodescstyle}
% Style of described macro names.
%    \begin{macrocode}
\def\keydescstyle{\ttfamily\bfseries\color{keydesc}}
%    \end{macrocode}
% \end{macro}
%
% \begin{macro}{\macroargsstyle}
% Default style for macro arguments (e.g.\ \cs{MacroArgs}).
%    \begin{macrocode}
\def\macroargsstyle{\ttfamily}
%    \end{macrocode}
% \end{macro}
%
% \begin{macro}{\envcodestyle}
% Default style for code body content in described environments.
%    \begin{macrocode}
\def\envcodestyle{\ttfamily}
%    \end{macrocode}
% \end{macro}
%
% \begin{macro}{\verbstyle}
% Style for verbatim text inside macro argument list.
%    \begin{macrocode}
\def\verbstyle{\verbatim@font}
%    \end{macrocode}
% \end{macro}
%
% \begin{macro}{\metastyle}
% Meta text style. Because \cs{macroargsstyle} might be also active a |\normalfont| reset
% the font.
%    \begin{macrocode}
\def\metastyle{\normalfont\itshape\color{meta}}
%    \end{macrocode}
% \end{macro}
%
% \begin{macro}{\margstyle}
% Style for \cs{marg}.
%    \begin{macrocode}
\def\margstyle{}
%    \end{macrocode}
% \end{macro}
%
% \begin{macro}{\Optional}
% \begin{macro}{\optional}
% \begin{macro}{\optionalstyle}
%    \begin{macrocode}
\protected\def\Optional{\optionalon\optional}
\def\optionalstyle{\blendcolors*{!60!white}\color{black!75}}
%    \end{macrocode}
% \end{macro}
% \end{macro}
% \end{macro}
%
% \begin{macro}{\optionalon}
% \begin{macro}{\optionaloff}
%    \begin{macrocode}
\def\optionalon{\protected\def\optional{\optionalstyle}}
\def\optionaloff{\let\optional\relax}
\optionalon
%    \end{macrocode}
% \end{macro}
% \end{macro}
%
% \begin{macro}{\oargstyle}
% Style for \cs{oarg}. A special color is set to show the `optional' status.
%    \begin{macrocode}
\def\oargstyle{\optional}
%    \end{macrocode}
% \end{macro}
%
% \begin{macro}{\pargstyle}
% Style for \cs{parg}.
%    \begin{macrocode}
\def\pargstyle{}
%    \end{macrocode}
% \end{macro}
%
% \begin{macro}{\aargstyle}
% Style for \cs{aarg}.
%    \begin{macrocode}
\def\aargstyle{}
%    \end{macrocode}
% \end{macro}
%
% \begin{macro}{\sargstyle}
% Style for \cs{sarg}. A special color is set to show the `optional' status.
%    \begin{macrocode}
\def\sargstyle{\ttfamily\color{optional}}
%    \end{macrocode}
% \end{macro}
%
% \subsubsection{Dimension Registers}
%
% \begin{macro}{\descindent}
%    \begin{macrocode}
\newdimen\descindent
\descindent=-\parindent
%    \end{macrocode}
% \end{macro}
%
% \begin{macro}{\beforedescskip}
%    \begin{macrocode}
\newdimen\beforedescskip
\beforedescskip=\bigskipamount
%    \end{macrocode}
% \end{macro}
%
% \begin{macro}{\afterdescskip}
%    \begin{macrocode}
\newdimen\afterdescskip
\afterdescskip=\medskipamount
%    \end{macrocode}
% \end{macro}
%
% \begin{macro}{\descsep}
% Set to |1em| in |tt| font.
%    \begin{macrocode}
\newdimen\descsep
\begingroup
\ttfamily
\global\descsep=1em\relax
\endgroup
%    \end{macrocode}
% \end{macro}
%
%
% \subsubsection{Macro Argument Reading Mechanism}
%
% \begin{macro}{\read@Macro@arg}
% Reads next token and calls second macro.
%    \begin{macrocode}
\def\read@Macro@arg{%
  \futurelet\@let@token\handle@Macro@arg
}
%    \end{macrocode}
% \end{macro}
%
% \begin{macro}{\AlsoMacro}
% Reads argument while |@| is a letter,
% prints the macro name and reads further arguments.
%    \begin{macrocode}
\newcommand*\AlsoMacro{%
  \begingroup\makeatletter
  \AlsoMacro@
}
\def\AlsoMacro@#1{%
  \endgroup
%<*DEBUG>
   %\typeout{DEBUG: Macro: \string#1}%
%</DEBUG>
  \PrintMacroName{#1}%
  \read@Macro@arg
}
%    \end{macrocode}
% \end{macro}
%
% \begin{macro}{\ydoc@short@AlsoMacro}
% Makes |&| an alias for |\AlsoMacro|.
%    \begin{macrocode}
\begingroup
\catcode`\|\active
\gdef\ydoc@short@AlsoMacro{%
  \catcode`\|\active
  \let|\AlsoMacro
}
\endgroup
%    \end{macrocode}
% \end{macro}
%
%
% \begin{macro}{\ydoc@macrocatcodes}
% Sets the catcodes inside for |read@Macro@arg| material.
%    \begin{macrocode}
\def\ydoc@macrocatcodes{%
  \ydoc@short@AlsoMacro
  \@makeother\'%
  \@makeother\!%
  \@makeother\[%
  \@makeother\]%
  \@makeother\(%
  \@makeother\)%
}
%    \end{macrocode}
% \end{macro}
%
%
% \begin{macro}{\handle@Macro@arg}
% Checks if next token is the begin of a valid macro argument and calls
% the appropriate read macro or the end macro otherwise.
%    \begin{macrocode}
\def\handle@Macro@arg{%
  \expandafter\let\expandafter\handler\csname handle@Macro@token@\meaning\@let@token\endcsname
  \ifx\handler\relax
    \def\handler{\ifhmode\unskip\fi\end@Macro@args}%
%<*DEBUG>
 %  \typeout{DEBUG: Stopped at: \expandafter\meaning\csname @let@token\endcsname}%
 %  \typeout{}%
 %\else
 %\expandafter\ifx\csname @let@token\endcsname\AlsoMacro
 %  \typeout{DEBUG: TOKEN: \string\AlsoMacro}%
 %\else
 %  \typeout{DEBUG: TOKEN: \expandafter\meaning\csname @let@token\endcsname}%
 %\fi
%</DEBUG>
  \fi
  \handler
}
\def\define@Macro@handler{%
    \begingroup
    \ydoc@macrocatcodes
    \define@Macro@handler@
}
\def\define@Macro@handler@#1{%
    \endgroup
    \@namedef{handle@Macro@token@\meaning#1}%
}
%    \end{macrocode}
% \end{macro}
%
% \begin{macro}{\end@Macro@args}
% Closes box as calls hook. Might be locally redefined by some
% macros calling \cs{read@Macro@arg}.
%    \begin{macrocode}
\def\end@Macro@args{%
  \y@egroup
  \after@Macro@args
}
%    \end{macrocode}
% \end{macro}
%
% \begin{macro}{\after@Macro@args}
% Hook to add additional commands in certain situations.
%    \begin{macrocode}
\def\after@Macro@args{%
}
%    \end{macrocode}
% \end{macro}
%
% \subsubsection*{Macro argument reading macros}
% This macros read the macro arguments and call the appropriate format macros.
%
% \begin{macro}{\read@Macro@marg}
%    \begin{macrocode}
\define@Macro@handler{\bgroup}{%
    \begingroup
        \afterassignment\read@Macro@marg@
        \let\@let@token=%
}
\def\read@Macro@marg@{%
        \bgroup
        \margstyle{}%
        \let\end@Macro@args\empty%
        {\ttfamily\braceleft}%
        \aftergroup\read@Macro@marg@@
        \read@Macro@arg
}
\def\read@Macro@marg@@{%
        {\ttfamily\braceright}%
    \endgroup
    \read@Macro@arg
}
%    \end{macrocode}
% \end{macro}
%
% \begin{macro}{\read@Macro@oarg}
%    \begin{macrocode}
\define@Macro@handler{[}[{%
    \begingroup
        \let\read@Macro@oarg@end\read@Macro@oarg@@end
        \let\end@Macro@args\read@Macro@oarg@end
        \oargstyle{}%
        {\ttfamily[}%]
        \read@Macro@arg
}
\define@Macro@handler{]}{%
    \read@Macro@oarg@end
}
\def\read@Macro@oarg@@end#1]{%
        #1%
        {\ttfamily]}%
    \endgroup
    \read@Macro@arg
}
\def\read@Macro@oarg@end{\end@Macro@args}
\let\read@Macro@aarg@end\read@Macro@oarg@end
\let\read@Macro@parg@end\read@Macro@oarg@end
%    \end{macrocode}
% \end{macro}
%
% \begin{macro}{\read@Macro@parg}
%    \begin{macrocode}
\define@Macro@handler{(}({%
    \begingroup
        \let\read@Macro@parg@end\read@Macro@parg@@end
        \let\end@Macro@args\read@Macro@parg@end
        \pargstyle{}%
        {\ttfamily(}%)
        \read@Macro@arg
}
\define@Macro@handler{)}{%
    \read@Macro@parg@end
}
\def\read@Macro@parg@@end#1){%
        #1%
        {\ttfamily)}%
    \endgroup
    \read@Macro@arg
}
%    \end{macrocode}
% \end{macro}
%
% \begin{macro}{\read@Macro@aarg}
%    \begin{macrocode}
\def\read@Macro@aarg<{%
    \begingroup
        \let\read@Macro@aarg@end\read@Macro@aarg@@end
        \let\end@Macro@args\read@Macro@aarg@end
        \aargstyle{}%
        {\ttfamily<}%
        \read@Macro@arg
}
\define@Macro@handler{>}{%
    \read@Macro@aarg@end
}
\def\read@Macro@aarg@@end#1>>{%
        #1%
        {\ttfamily>}%
    \endgroup
    \read@Macro@arg
}
%    \end{macrocode}
% \end{macro}
%
% \begin{macro}{\read@Macro@angle}
%    \begin{macrocode}
\define@Macro@handler{<}<{%
  \futurelet\@let@token\read@Macro@angle@
}
%    \end{macrocode}
% \end{macro}
%
% \begin{macro}{\read@Macro@angle@}
%    \begin{macrocode}
\def\read@Macro@angle@{%
  \ifx\@let@token<%
    \expandafter\read@Macro@aarg
  \else
    \expandafter\read@Macro@meta
  \fi
}
%    \end{macrocode}
% \end{macro}
%
% \begin{macro}{\read@Macro@meta}
%    \begin{macrocode}
\def\read@Macro@meta#1>{%
  \meta{#1}\read@Macro@arg
}
%    \end{macrocode}
% \end{macro}
%
% \begin{macro}{\read@Macro@sarg}
%    \begin{macrocode}
\define@Macro@handler**{%
  \sarg\read@Macro@arg
}
%    \end{macrocode}
% \end{macro}
%
% Allows `|=|' to be used directly without switching to verbatim
% mode. This is especially useful for keys.
%    \begin{macrocode}
\define@Macro@handler{=}={%
  =\read@Macro@arg
}
%    \end{macrocode}
%
% \begin{macro}{\read@Macro@verb}
% Sets up verbatim mode calls second macro.
%    \begin{macrocode}
\define@Macro@handler{'}'{%
  \begingroup
  \let\do\@makeother
  \dospecials
  \@noligs
  \@makeother\'%
  \obeyspaces
  \read@Macro@verb@
}
%    \end{macrocode}
% \end{macro}
%
% \begin{macro}{\read@Macro@verb@}
% Closes verbatim mode and formats text.
% If |#1| is empty (|''|) than a single |'| is printed.
%    \begin{macrocode}
\begingroup
\@makeother\'%
\gdef\read@Macro@verb@#1'{%
  \endgroup
  \ifx\relax#1\relax
    {\verbstyle{\string'}}%
  \else
    {%
     \frenchspacing
     \@noligs\verbstyle{#1}}%
  \fi
  \read@Macro@arg
}
\endgroup
%    \end{macrocode}
% \end{macro}
%
% \begin{macro}{\read@Macro@cmds}
% Simply executes given code.
%    \begin{macrocode}
\define@Macro@handler!!#1!{%
  #1\relax
  \read@Macro@arg
}
%    \end{macrocode}
% \end{macro}
%
% \begin{macro}{\read@Macro@rmspace}
% Removes space. The |\@firstofone| is used to preserve the space in the macro definition.
%    \begin{macrocode}
\define@Macro@handler{\@sptoken} {%
  \read@Macro@arg
}
%    \end{macrocode}
% \end{macro}
%
% \begin{macro}{\read@Macro@addtoken}
% Takes token over from input to output `stream'. This is used for |\space| and |~|.
%    \begin{macrocode}
\define@Macro@handler{~}#1{%
  #1\read@Macro@arg
}
\AtBeginDocument{%
\define@Macro@handler{~}#1{%
  #1\read@Macro@arg
}
}
\define@Macro@handler{\space}#1{%
  #1\read@Macro@arg
}
%    \end{macrocode}
% \end{macro}
%
% \subsubsection{Description Macros}
% \subsubsection*{For Macros}
%
% \begin{macro}{\DescribeMacro}
%    \begin{macrocode}
\@ifundefined{DescribeMacro}{}{%
  \PackageInfo{ydoc-desc}{Redefining \string\DescribeMacro}{}%
}
%    \end{macrocode}
% A \cs{DescribeMacro} places itself in a \env{DescribeMacros} environment.
% Multiple \cs{DescribeMacro} macros will stack themselves inside this environment.
% For this to work \cs{DescribeMacros} is locally defined to |\y@egroup| to close the
% |\hbox| from the previous \cs{DescribeMacro}.
%    \begin{macrocode}
\def\DescribeMacro{%
  \DescribeMacros
  \let\DescribeMacros\y@egroup
  \optionalon
  \def\after@Macro@args{\endDescribeMacros}%
  \begingroup\makeatletter
  \Describe@Macro
}
%    \end{macrocode}
% \end{macro}
%
% \begin{macro}{\DescribeScript}
%    \begin{macrocode}
\def\DescribeScript#1{%
  \DescribeMacros
  \let\DescribeMacros\y@egroup
  \optionalon
  \def\after@Macro@args{\endDescribeMacros}%
  \hbox\y@bgroup
  \texttt{#1}%
  \ydoc@macrocatcodes
  \macroargsstyle
  \read@Macro@arg~%
}
%    \end{macrocode}
% \end{macro}
%
% \begin{macro}{\DescribeKey}
%    \begin{macrocode}
\def\DescribeKey{%
  \DescribeKeys
  \let\DescribeKeys\y@egroup
  \optionalon
  \def\after@Macro@args{\endDescribeKeys}%
  \begingroup\makeatletter
  \Describe@Macro
}
%    \end{macrocode}
% \end{macro}
%
% \begin{macro}{\Describe@Macro}
%    \begin{macrocode}
\def\Describe@Macro#1{%
  \endgroup
  \edef\name{\expandafter\@gobble\string#1}%
  \global\@namedef{href@desc@\name}{}%
  \immediate\write\@mainaux{%
    \global\noexpand\@namedef{href@desc@\name}{}%
  }%
  \hbox\y@bgroup
  \@ifundefined{href@impl@\name}{}{\hyperlink{impl:\name}}%
  {%
  \hbox{\vbox to 0pt{\vss\hbox{\raisebox{4ex}{\hypertarget{desc:\name}{}}}}%
  \PrintMacroName{#1}}%
  }%
  \ydoc@macrocatcodes
  \macroargsstyle
  \read@Macro@arg
}
%    \end{macrocode}
% \end{macro}
%
% \begin{macro}{\MakeShortMacroArgs}
% Defines the given character as short version for |\MacroArgs|.
% It is first define to be a short verbatim character to take
% advantage of the house-keeping (save \& restore of the original
% catcode and definition) of |shortvrb|.
%
% The starred version define the character to act like |\Macro| instead.
%    \begin{macrocode}
\newcommand*\MakeShortMacroArgs{%
  \@ifstar
    {\@MakeShortMacroArgs\Macro}%
    {\@MakeShortMacroArgs\MacroArgs}%
}
\def\@MakeShortMacroArgs#1#2{%
  \MakeShortVerb{#2}
  \catcode`#2\active
  \begingroup
  \catcode`\~\active
  \lccode`\~`#2\relax
  \lowercase{\endgroup\gdef~{\bgroup\let~\egroup#1}}%
}
%    \end{macrocode}
% \end{macro}
%
% \begin{macro}{\DeleteShortMacroArgs}
%    \begin{macrocode}
\newcommand*\DeleteShortMacroArgs[1]{%
  \DeleteShortVerb{#1}%
}
%    \end{macrocode}
% \end{macro}
%
% \begin{macro}{\Macro}
% Simply uses the two macros below.
%    \begin{macrocode}
\newcommand*\Macro{\MacroArgs\AlsoMacro}
%    \end{macrocode}
% \end{macro}
%
% \begin{macro}{\@Macro}
% Alternative definition of \cs{Macro} inside \env{DescribeMacros} environments.
%    \begin{macrocode}
\def\@Macro{%
  \begingroup\makeatletter
  \Describe@Macro
}
%    \end{macrocode}
% \end{macro}
%
%    \begin{macrocode}
\define@Macro@handler\AlsoMacro{}
\define@Macro@handler\DescribeMacro{}
\define@Macro@handler\DescribeKey{}
\define@Macro@handler\DescribeScript{}
%    \end{macrocode}
%
% \begin{macro}{\MacroArgs}
% Uses the normal macro argument reading mechanism from \cs{DescribeMacro}.
% Instead of a box a simple group is added.
%    \begin{macrocode}
\newcommand*\MacroArgs{%
  \begingroup
  \def\end@Macro@args{\endgroup\xspace}%
  \ydoc@macrocatcodes
  \macroargsstyle
%<*DEBUG>
   %\typeout{}%
   %\typeout{DEBUG: Start MacroArgs}%
%</DEBUG>
  \read@Macro@arg
}
\RequirePackage{xspace}
%    \end{macrocode}
% \end{macro}
%
% \begin{macro}{\DescribeMacros}
%    \begin{macrocode}
\def\DescribeMacros{%
  \begingroup
  \let\Macro\@Macro
  \parindent=0pt\relax
  \setbox\descbox\vbox\y@bgroup
}
%    \end{macrocode}
% \end{macro}
%
% \begin{macro}{\endDescribeMacros}
%    \begin{macrocode}
\def\endDescribeMacros{%
  \y@egroup
  \PrintMacros
  \endgroup
}
%    \end{macrocode}
% \end{macro}
%
% \begin{macro}{\DescribeKeys}
%    \begin{macrocode}
\def\DescribeKeys{%
  \begingroup
  \let\PrintMacroName\PrintKeyName
  \let\Key\@Macro
  \parindent=0pt\relax
  \setbox\descbox\vbox\y@bgroup
}
%    \end{macrocode}
% \end{macro}
%
% \begin{macro}{\endDescribeKeys}
%    \begin{macrocode}
\def\endDescribeKeys{%
  \y@egroup
  \PrintKeys
  \endgroup
}
\def\PrintKeys{\PrintMacros}
%    \end{macrocode}
% \end{macro}
%
% \begin{macro}{\DescribeMacrosTabcolsep}
%    \begin{macrocode}
\def\DescribeMacrosTabcolsep{\tabcolsep}
%    \end{macrocode}
% \end{macro}
%
% \begin{macro}{\DescribeMacrosTab}
%    \begin{macrocode}
\def\DescribeMacrosTab{%
  \DescribeMacros
  \hbox\y@bgroup
  \tabcolsep=\DescribeMacrosTabcolsep\relax
  \DescribeMacrosTab@
}
\def\DescribeMacrosTab@#1{\tabular{@{}#1@{}}}
%    \end{macrocode}
% \end{macro}
%
% \begin{macro}{\endDescribeMacrosTab}
%    \begin{macrocode}
\def\endDescribeMacrosTab{%
  \endtabular\y@egroup
  \endDescribeMacros
}
%    \end{macrocode}
% \end{macro}
%
% \subsubsection*{For Lengths}
%
% \begin{macro}{\DescribeLength}
%    \begin{macrocode}
\newcommand*\DescribeLength{%
  \begingroup
  \let\DescribeLength\Describe@Length
  \setbox\descbox\hbox\y@bgroup
    \tabular{@{}l@{\hspace{2em}}l@{}}%
    \Describe@Length
}
%    \end{macrocode}
% \end{macro}
%
% \begin{macro}{\Describe@Length}
%    \begin{macrocode}
\newcommand*\Describe@Length[2]{%
  \PrintLengthName{#1}&
  (Default: {\macroargsstyle#2\unskip})%
  \@ifnextchar\DescribeLength
    {\\}%
    {%
      \endtabular
      \y@egroup
      \PrintLength
      \endgroup
    }%
}
%    \end{macrocode}
% \end{macro}
%
% \subsubsection*{For Environments}
%
% \begin{macro}{\DescribeEnv}
%    \begin{macrocode}
\@ifundefined{DescribeEnv}{}{%
  \PackageInfo{ydoc-desc}{Redefining \string\DescribeEnv}{}%
}
\let\DescribeEnv\relax
%    \end{macrocode}
%    \begin{macrocode}
\newcommand*\DescribeEnv[2][]{%
  \begingroup
  \def\DescribeEnv@name{#2}%
  \let\\\DescribeEnv@newline
%    \end{macrocode}
%
% Sets after-macro-arguments hook.
% First checks if the environment or macro version was used.
% The environment starts a new line only if the next token isn't |\end|,
% which is taken as end of the environment.
%    \begin{macrocode}
  \ifx\@currenvir\DescribeEnv@string
    \def\after@Macro@args{%
      \let\after@Macro@args\empty
      \setbox\@tempboxa\hbox\y@bgroup
      \@ifnextchar\end{}%
        {\DescribeEnv@newline}%
      #1%
    }%
%    \end{macrocode}
% The macro version adds the optional argument as content line if given.
%    \begin{macrocode}
  \else
    \ifx\relax#1\relax
      \def\after@Macro@args{%
        \y@bgroup
        \endDescribeEnv
      }%
    \else
      \def\after@Macro@args{%
        \setbox\@tempboxa\hbox\y@bgroup
        \DescribeEnv@newline\MacroArgs#1%
        \endDescribeEnv
      }%
    \fi
  \fi
%    \end{macrocode}
% Start |\vbox| and adds first line.
%    \begin{macrocode}
  \setbox\descbox\vbox\y@bgroup
  \envcodestyle
  \let\PrintEnv\PrintSubEnv
  \hbox\y@bgroup
  \PrintEnvName{\begin}{\DescribeEnv@name}%
  \ydoc@macrocatcodes
  \macroargsstyle
  \read@Macro@arg
}
%    \end{macrocode}
% \end{macro}
%
% \begin{macro}{\DescribeEnv@newline}
% Closes existing and starts a new horizontal box representing a indented line.
% The optional argument allows to add extra space between lines like the normal |\\|.
% Negative values are not supported.
%    \begin{macrocode}
\newcommand*\DescribeEnv@newline[1][0pt]{%
  \strut\y@egroup
  {\vskip#1}%
  \hbox\y@bgroup\strut
  \hspace*{\descsep}%
  \ignorespaces
}%
%    \end{macrocode}
% \end{macro}
%
% \begin{macro}{\DescribeEnv@string}
% Holds the environment name for comparison.
%    \begin{macrocode}
\def\DescribeEnv@string{DescribeEnv}
%    \end{macrocode}
% \end{macro}
%
% \begin{macro}{\descbox}
% Save box to store description content.
%    \begin{macrocode}
\newbox\descbox
%    \end{macrocode}
% \end{macro}
%
% \begin{macro}{\endDescribeEnv}
%    \begin{macrocode}
\def\endDescribeEnv{%
  \y@egroup
  \begingroup
  \setbox\@tempboxa\lastbox
  \ifcase0%
    \ifdim\wd\@tempboxa>\descsep1\fi
    \ifdim\ht\@tempboxa>\ht\strutbox1\fi
    \ifdim\dp\@tempboxa>\dp\strutbox1\fi
  \else
    \box\@tempboxa
  \fi
  \endgroup
  \hbox\y@bgroup
    \PrintEnvName{\end}{\DescribeEnv@name}
  \y@egroup
  \y@egroup
  \PrintEnv
  \endgroup
}
%    \end{macrocode}
% \end{macro}
%
%
% \subsubsection{Print Macros}
%
% \begin{macro}{\PrintMacroName}
% Formats macro name. The backslash is forced to |tt| font.
%    \begin{macrocode}
\def\PrintMacroName#1{%
  {\macrodescstyle{\strut
   \texttt{\char92}%
   \escapechar\m@ne
   \string#1\strut}}%
}
%    \end{macrocode}
% \end{macro}
%
% \begin{macro}{\PrintKeyName}
% Formats macro name. The backslash is forced to |tt| font.
%    \begin{macrocode}
\def\PrintKeyName#1{%
  {\keydescstyle{\strut
   #1\strut}}%
}
%    \end{macrocode}
% \end{macro}
%
% \begin{macro}{\PrintLengthName}
% Formats length register name.
%    \begin{macrocode}
\let\PrintLengthName\PrintMacroName
%    \end{macrocode}
% \end{macro}
%
%
% \begin{macro}{\PrintEnvName}
% |#1| = `|\begin|' or `|\end|', |#2| = env name.
%    \begin{macrocode}
\def\PrintEnvName#1#2{%
  \strut
  \string#1\braceleft
  {\macrodescstyle#2\strut}%
  \braceright
}
%    \end{macrocode}
% \end{macro}
%
% \begin{macro}{\PrintMacros}
% Prints macros described using \cs{DescribeMacros}.
% The actual content was stored inside \cs{descbox}.
% If it is wider than the line width it is centered.
%    \begin{macrocode}
\def\PrintMacros{%
  \par\vspace\beforedescskip
  \begingroup
  \sbox\@tempboxa{\descframe{\usebox{\descbox}}}%
  \Needspace*{\dimexpr\ht\@tempboxa+3\baselineskip\relax}%
  \par\noindent
  \ifdim\wd\@tempboxa>\dimexpr\linewidth-2\descindent\relax
    \makebox[\linewidth][c]{\usebox\@tempboxa}%
  \else
    \hspace*{\descindent}%
    \usebox\@tempboxa
  \fi
  \endgroup
  \par
  \vspace\afterdescskip
  \par\noindent
}
\def\descframe#1{%
    \fbox{\hspace*{\descsep}#1\hspace*{\descsep}}%
}
%    \end{macrocode}
% \end{macro}
%
% \begin{macro}{\PrintLength}
% Prints lengths registers described using one or multiple \cs{DescribeLength}.
%    \begin{macrocode}
\let\PrintLength\PrintMacros
%    \end{macrocode}
% \end{macro}
%
% \begin{macro}{\PrintEnv}
% Prints \env{DescribeEnv} environments.
% The actual content was stored inside \cs{descbox}.
%    \begin{macrocode}
\let\PrintEnv\PrintMacros
%    \end{macrocode}
% \end{macro}
%
% \begin{macro}{\PrintSubEnv}
% Prints sub environments, i.e.\ \env{DescribeEnv} environments
% inside the body of another \env{DescribeEnv}.
% The actual content was stored inside \cs{descbox}.
%    \begin{macrocode}
\def\PrintSubEnv{%
  \hbox{\hbox{\usebox{\descbox}}}%
}
%    \end{macrocode}
% \end{macro}
%
%
% \subsubsection{Special Character Macros}
%
% \begin{macro}{\bslash}
% Defines an expandable backslash with catcode 12: `\texttt{\bslash}$_{12}$'.
% The |\@firstofone| trick is used to read the |\gdef\bslash| code before changing the catcode.
%    \begin{macrocode}
{%
\@firstofone
%    \end{macrocode}
% \end{macro}
%
% \begin{macro}{\percent}
% Defines an expandable percent character with catcode 12: `\percent$_{12}$'.
%    \begin{macrocode}
\begingroup
\catcode`\%=12
\gdef\percent{%}
\endgroup
%    \end{macrocode}
% \end{macro}
%
% \begin{macro}{\braceleft}
% \begin{macro}{\braceright}
% Defines expandable left and right braces with catcode 12: `\texttt{\braceleft}$_{12}$' `\texttt{\braceright}$_{12}$'.
%    \begin{macrocode}
\begingroup
\catcode`\<=1
\catcode`\>=2
\catcode`\{=12
\catcode`\}=12
\gdef\braceleft <{>
\gdef\braceright<}>
\endgroup
%    \end{macrocode}
% \end{macro}
% \end{macro}
%
% \subsubsection{Other Macros}
%
% \begin{macro}{\y@bgroup}
% \begin{macro}{\y@egroup}
% These macros are used to begin and end |\vbox|/|\hbox|-es.
%    \begin{macrocode}
\def\y@bgroup{\bgroup\color@setgroup}
\def\y@egroup{\color@endgroup\egroup}
%    \end{macrocode}
% \end{macro}
% \end{macro}
%
% \begin{macro}{\codeline}
%    \begin{macrocode}
\newcommand*{\codeline}[1][c]{%
    \codelinebefore
    \hbox to \hsize\bgroup
    \ifx i#1\hspace*{\leftmargin}\else
        \ifx l#1\else\hss\fi
    \fi
    \let\xspace\relax
    \hbox\bgroup
    \aftergroup\codeline@end
    \aftergroup#1%
    \afterassignment\MacroArgs
    \let\@let@token=%
}
\def\codeline@end#1{%
    \ifx r#1\else\hss\fi
    \egroup
    \codelineafter
}
\newcommand*\codelinebefore{\par\smallskip\noindent}
\newcommand*\codelineafter {\par\smallskip\noindent}
%    \end{macrocode}
% \end{macro}
%
% \begin{environment}{codequote}
%    \begin{macrocode}
\newenvironment{codequote}{%
    \def\\{\newline\relax\MacroArgs}%
    \par\smallskip\bgroup\leftskip=\leftmargin\rightskip=\rightmargin\noindent\MacroArgs}
    {\par\egroup\smallskip\noindent\ignorespacesafterend}
%    \end{macrocode}
% \end{environment}
%
% \begin{environment}{macroquote}
%    \begin{macrocode}
\newenvironment{macroquote}{%
    \def\\{\newline\relax\Macro}%
    \par\smallskip\bgroup\leftskip=\leftmargin\rightskip=\rightmargin\noindent\Macro}
    {\par\egroup\smallskip\noindent\ignorespacesafterend}
%    \end{macrocode}
% \end{environment}
% \iffalse
%</ydoc-desc.sty>
% \fi
%
% \iffalse
%<*ydoc-expl.sty>
% \fi
% \subsection{Include Code Examples}
%
%    \begin{macrocode}
\NeedsTeXFormat{LaTeX2e}[1999/12/01]
\ProvidesPackage{ydoc-expl}[%
%<!DATE>
%<!VERSION>
%<*DRIVER>
    2011/08/11 develop
%</DRIVER>
    ydoc package to insert live examples of LaTeX code]
%    \end{macrocode}
%
% \iffalse
%<*dontinclude>
%    \begin{macrocode}
\newcounter{example}
\def\examplename{Example}
\newenvironment{example}[1]{%
  \par\bigskip\noindent
  \centerline{\examplename~\refstepcounter{example}\theexample: #1}%
  \smallskip
  \hrule
  \smallskip
}{%
  \smallskip
  \hrule
  \par\bigskip
}
%    \end{macrocode}
%
% \begin{macro}{\PrintExample}
%    \begin{macrocode}
\def\PrintExample{%
  \usebox\examplecodebox
  \hfill
  \usebox\exampleresultbox
}
%    \end{macrocode}
% \end{macro}
%</dontinclude>
% \fi
%
%    \begin{macrocode}
\RequirePackage{listings}
\lst@RequireAspects{writefile}
\def\ydoc@exafile{\jobname.exa}
%    \end{macrocode}
%
% \begin{style}{examplecode}
%    \begin{macrocode}
\lstdefinestyle{examplecode}{%
    language=[latex]tex,
    basicstyle=\ttfamily,
    columns=fullflexible,
    numbers=left,
    firstnumber=1,
    numberstyle=\tiny\color{gray}\sffamily,
    numbersep=5pt,
    breaklines,prebreak={\mbox{\tiny$\swarrow$}},
    commentstyle=\color{black!60},
}%
%    \end{macrocode}
% \end{style}
%
% \begin{style}{exampleresult}
%    \begin{macrocode}
\lstdefinestyle{exampleresult}{%
    firstnumber=1,
    gobble=0,
    basicstyle=\ttfamily,
    columns=fullflexible,
    commentstyle=\color{black!60},
}
%    \end{macrocode}
% \end{style}
%
% \begin{style}{exampleextract}
%    \begin{macrocode}
\lstdefinestyle{exampleextract}{gobble=4}%
%    \end{macrocode}
% \end{style}
%
%    \begin{macrocode}
\newbox\examplecodebox
\newbox\exampleresultbox
%    \end{macrocode}
%
% \begin{macro}{\BoxExample}
%    \begin{macrocode}
\def\BoxExample{%
  \setbox\examplecodebox\hbox{\color@setgroup
    \lstinputlisting[style=examplecode,style=thisexampleprint]%
    {\ydoc@exafile}%
  \unskip\color@endgroup}%
  \setbox\exampleresultbox\hbox{\color@setgroup
    \lstset{style=exampleresult}%
    \@@input\ydoc@exafile\relax
  \unskip\color@endgroup}%
}
%    \end{macrocode}
% \end{macro}
%
% \begin{macro}{\PrintExample}
%    \begin{macrocode}
%<*DISABLED>
\RequirePackage{showexpl}
\def\PrintExample{%
  \begingroup
  \lstset{style=examplecode}%
  \MakePercentComment
  \LTXinputExample[varwidth]{\ydoc@exafile}%
  \endgroup
}
%</DISABLED>
%    \end{macrocode}
% \end{macro}
%
%
%
% \begin{macro}{\PrintExample}
%    \begin{macrocode}
\def\PrintExample{%
  \begingroup
  \BoxExample
  \@tempdima=\textwidth
  \advance\@tempdima by -\wd\examplecodebox\relax
  \advance\@tempdima by -\wd\exampleresultbox\relax
  \advance\@tempdima by -15pt\relax
  \ifdim\@tempdima>\bigskipamount
    \hbox to \textwidth{%
     \null\hss
     \minipage[c]{\wd\exampleresultbox}\fbox{\usebox\exampleresultbox}\endminipage
     \hfill\hfill\hskip\bigskipamount\hskip15pt\hfill\hfill
     \minipage[c]{\wd\examplecodebox}\usebox\examplecodebox\endminipage
     \hss\null
     }%
  \else
    \vbox{%
        \centerline{\fbox{\usebox\exampleresultbox}}%
        \vspace{\bigskipamount}%
        \centerline{\usebox\examplecodebox}%
    }%
  \fi
  \endgroup
}
%    \end{macrocode}
% \end{macro}
%
% \begin{environment}{examplecode}
%    \begin{macrocode}
\lstnewenvironment{examplecode}[1][]{%
  \lstdefinestyle{thisexampleprint}{#1}%
  \setbox\@tempboxa\hbox\bgroup
  \lstset{style=exampleextract,#1}%
  \lst@BeginWriteFile{\ydoc@exafile}%
}
{%
  \lst@EndWriteFile
  \egroup
  \begingroup
  \MakePercentComment
  \catcode`\^^M=5\relax
  \PrintExample
  \endgroup
}
%    \end{macrocode}
% \end{environment}
%
%    \begin{macrocode}
\RequirePackage{float}
%    \end{macrocode}
%
% \begin{environment}{example}
%    \begin{macrocode}
\floatstyle{plain}
\newfloat{example}{tbhp}{loe}
\floatname{example}{\examplename}
\def\examplename{Example}
%    \end{macrocode}
% \end{environment}
%
% \begin{environment}{exampletable}
%    \begin{macrocode}
\newenvironment{exampletable}{%
  \floatstyle{plaintop}%
  \restylefloat{example}%
  \example
}{\endexample}
%    \end{macrocode}
% \end{environment}
% \iffalse
%</ydoc-expl.sty>
% \fi
%
% \iffalse
%<*ydocincl.tex>
% \fi
%
%    \begin{macrocode}
\expandafter\ifx\csname ydocinclversion\endcsname\relax\else
    \endinput
\fi

\chardef\ydocinclversion=1

\newread\inFile
\newread\subFile
\newwrite\outFile
\newif\ifContinue
\newlinechar=`^^J

\def\makeOther#1{\catcode`#1=12\relax}

\let\inLine\relax
\let\lastLine\relax

\def\includefiles#1#2{%
    \begingroup
    \immediate\openin\inFile#1\relax
    \immediate\openout\outFile#2\relax
    \makeOther\@%
    \makeOther\ \makeOther\\\makeOther\$%
    \makeOther\#\makeOther\^\makeOther\^^K%
    \makeOther\_\makeOther\^^A\makeOther\%%
    \makeOther\~\makeOther\{\makeOther\}\makeOther\&%
    \endlinechar-1\relax
    \Continuetrue
    \loop
      \let\lastLine\inLine
      \read\inFile to\inLine
      \ifeof\inFile
        \Continuefalse
      \else
        \expandafter\checkLine\inLine\empty\empty\empty\endLine
      \fi
      \ifContinue
    \repeat
    \immediate\closein\inFile
    \immediate\closeout\outFile
    \endgroup
    \end
}

\def\copyline{%
    \immediate\write\outFile{\inLine}%
}

\chardef\percentcharnum=`\%

\begingroup
\makeOther\%\makeOther\@\relax
\gdef\SubFileOptionString\relax
\catcode`\|=0
\makeOther\ \makeOther\\|relax
|gdef|IfFalseString{% \iffalse}|relax
|gdef|FiString{% \fi}|relax
|endgroup

\def\checkLine#1#2#3#4\endLine{%
    \def\firstthree{#1#2#3}%
    \ifx\firstthree\SubFileOptionString
        \readSubFile#4\endLine
    \else
        \copyline
    \fi
}

\def\readSubFile#1>#2\endLine{%
    \immediate\openin\subFile=#1\relax
    \ifeof\subFile
        % File not found
    \else
        \message{^^JIncluding subfile '#1'^^J}%
        \immediate\write\outFile{\CommentChar<*#1>}%
        \ifx\lastLine\IfFalseString
            \immediate\write\outFile{\FiString}%
        \fi
        \copySubFile
        \ifx\lastLine\IfFalseString
            \immediate\write\outFile{\IfFalseString}%
        \fi
        \immediate\write\outFile{\CommentChar</#1>}%
    \fi
    \immediate\closein\subFile
}

\def\copySubFile{%
    \read\subFile to\subLine
    \ifeof\subFile\else
        \immediate\write\outFile{\subLine}%
        \expandafter\copySubFile
    \fi
}
%    \end{macrocode}
%
% \iffalse
%</ydocincl.tex>
% \fi
%
% \Finale
% \endinput

%    \end{macrocode}
% \iffalse
%</ydoc.sty>
% \fi
%
% \iffalse
%<*ydoc.cfg>
% \fi
% \subsection{Config File}
%    \begin{macrocode}
%% Please delete the following line on manual changes:
\ProvidesFile{ydoc.cfg}[%
%<!DATE>
%<!VERSION>
%<*DRIVER>
   2011/08/11 develop
%</DRIVER>
   Default config file for ydoc]
%    \end{macrocode}
%
%    \begin{macrocode}
\usepackage[T1]{fontenc}
\IfFileExists{fourier.sty}{%
    \usepackage{fourier}
}{}
%    \end{macrocode}
%
% Use 'lmodern' only for the 'tt' font if fourier is installed.
%    \begin{macrocode}
\IfFileExists{lmodern.sty}{
    \IfFileExists{fourier.sty}{
        \renewcommand{\ttdefault}{lmtt}
    }{
        \usepackage{lmodern}
    }
}{}
%    \end{macrocode}
%
%
%    \begin{macrocode}
\urlstyle{sf}
%    \end{macrocode}
%
% Use micro-typesetting if pdftex is used:
%    \begin{macrocode}
\usepackage{ifpdf}
\ifpdf
\usepackage{microtype}
\fi
%    \end{macrocode}
%    \begin{macrocode}
\usepackage{array}
\usepackage{booktabs}
\usepackage{multicol}
\usepackage{xcolor}
\usepackage{listings}
\usepackage{booktabs}
\usepackage{hyperref}
%    \end{macrocode}
%
%    \begin{macrocode}
\reversemarginpar
%    \end{macrocode}
%
% \iffalse
%</ydoc.cfg>
% \fi
%
% \iffalse
%<*ydoc-code.sty>
% \fi
% \subsection{Macros and Environments to document Implementations}
%
%    \begin{macrocode}
\NeedsTeXFormat{LaTeX2e}[1999/12/01]
\ProvidesPackage{ydoc-code}[%
%<!DATE>
%<!VERSION>
%<*DRIVER>
    2011/08/11 develop
%</DRIVER>
    ydoc package to document macro code]
%    \end{macrocode}
%
%    \begin{macrocode}
\RequirePackage{hyperref}
\hypersetup{colorlinks=true,pdfborder=0 0 0,pdfborderstyle={}}
%    \end{macrocode}
%
%    \begin{macrocode}
\IfFileExists{needspace.sty}{%
    \RequirePackage{needspace}
}{%
    \def\Needspace{\@ifstar\@gobble\@gobble}
}
%    \end{macrocode}
%
% \subsubsection{Color and style definitions}
%    \begin{macrocode}
\RequirePackage{xcolor}
\definecolor{macroimpl}{rgb}{0.0,0.0,0.4}
%    \end{macrocode}
%
% \subsubsection{General Macros}
%
% \begin{macro}{\ydocwrite}
%    \begin{macrocode}
\@ifundefined{ydocwrite}{%
  \newwrite\ydocwrite
}{}
%    \end{macrocode}
% \end{macro}
%
% \begin{macro}{\ydocfname}
%    \begin{macrocode}
\@ifundefined{ydocfname}{%
  \def\ydocfname{\jobname.cod}%
}{}
%    \end{macrocode}
% \end{macro}
%
% \begin{macro}{\ydoc@catcodes}
%    \begin{macrocode}
\def\ydoc@catcodes{%
  \let\do\@makeother
  \dospecials
  \catcode`\\=\active
  \catcode`\^^M=\active
  \catcode`\ =\active
}
%    \end{macrocode}
% \end{macro}
%
% \subsubsection{Handling Macrocode}
% \begin{environment}{macrocode}
%    \begin{macrocode}
\def\macrocode{%
  \par\noindent
  \begingroup
  \ydoc@catcodes
  \macro@code
}
\def\endmacrocode{}
%    \end{macrocode}
% \end{environment}
%
% \begin{macro}{\macro@code}[1]{verbatim macro code}
%    \begin{macrocode}
\begingroup
\endlinechar\m@ne
\@firstofone{%
\catcode`\|=0\relax
\catcode`\(=1\relax
\catcode`\)=2\relax
\catcode`\*=14\relax
\catcode`\{=12\relax
\catcode`\}=12\relax
\catcode`\ =12\relax
\catcode`\%=12\relax
\catcode`\\=\active
\catcode`\^^M=\active
\catcode`\ =\active
}*
|gdef|macro@code#1^^M%    \end{macrocode}(*
|endgroup|expandafter|macro@@code|expandafter(|ydoc@removeline#1|noexpand|lastlinemacro)*
)*
|gdef|ydoc@removeline#1^^M(|noexpand|firstlinemacro)*
|gdef|ydoc@defspecialmacros(*
|def^^M(|noexpand|newlinemacro)*
|def (|noexpand|spacemacro)*
|def\(|noexpand|bslashmacro)*
)*
|gdef|ydoc@defrevspecialmacros(*
|def|newlinemacro(|noexpand^^M)*
|def|spacemacro(|noexpand )*
|def|bslashmacro(|noexpand\)*
)*
|endgroup
%    \end{macrocode}
% \end{macro}
%
% \begin{macro}{\macro@@code}[1]{verbatim macro code}
%    \begin{macrocode}
\def\macro@@code#1{%
  {\ydoc@defspecialmacros
  \xdef\themacrocode{#1}}%
  \PrintMacroCode
  \end{macrocode}%
}
%    \end{macrocode}
% \end{macro}
%
% \begin{macro}{\linenumberbox}
%    \begin{macrocode}
\def\newlinemacro{\\\null}
\def\spacemacro{\ }
\def\bslashmacro{\char92}
\def\lastlinemacro{}
\def\firstlinemacro{\linenumberbox}
\def\newlinemacro{\\\linenumberbox}
\newcounter{linenumber}
\def\linenumberbox{%
  \hbox to 1.25em{}%
  \llap{%
    \stepcounter{linenumber}%
    {\footnotesize\color{gray}\thelinenumber~}%
  }%
}
%    \end{macrocode}
% \end{macro}
%
% \begin{macro}{\PrintMacroCode}
%    \begin{macrocode}
\def\PrintMacroCode{%
  \begingroup
  \ttfamily
  \noindent\themacrocode
  \endgroup
}
%    \end{macrocode}
% \end{macro}
%
% \begin{macro}{\PrintMacroCode}
%    \begin{macrocode}
\RequirePackage{listings}
%    \end{macrocode}
%    \begin{macrocode}
\def\PrintMacroCode{%
  \begingroup
  \let\firstlinemacro\empty
  \let\lastlinemacro\empty
  \def\newlinemacro{^^J}%
  \let\bslashmacro\bslash
  \let\spacemacro\space
  \immediate\openout\ydocwrite=\ydocfname\relax
  \immediate\write\ydocwrite{\themacrocode}%
  \immediate\closeout\ydocwrite
  \@nameuse{ydoc@countbslashes}%
  \ydoclistingssettings
  \let\input\@input
  \lstinputlisting{\ydocfname}%
  \endgroup
}
%    \end{macrocode}
% \end{macro}
%
%    \begin{macrocode}
\lstdefinestyle{ydoccode}{%
    language=[latex]tex,basicstyle=\ttfamily,
    numbers=left,numberstyle=\tiny\color{gray},firstnumber=last,
    breaklines,prebreak={\mbox{\tiny$\swarrow$}},
    commentstyle=\color{black!60},
}%
%    \end{macrocode}
%
% \begin{macro}{\ydoclistingssettings}
%    \begin{macrocode}
\def\ydoclistingssettings{%
    \lstset{style=ydoccode}%
}
%    \end{macrocode}
% \end{macro}
%
% \begin{macro}{\macro@impl@args}[1]{number of macro arguments}
%    \begin{macrocode}
\def\macro@impl@args[#1]{%
  \begingroup
  \parindent=10pt\relax
  \let\macro@impl@argcnt\@tempcnta
  \let\macro@impl@curarg\@tempcntb
  \macro@impl@argcnt=#1\relax
  \macro@impl@curarg=0\relax
  \ifnum\macro@impl@curarg<\macro@impl@argcnt\relax
    \expandafter\macro@impl@arg
  \else
    \expandafter\macro@impl@endargs
  \fi
}
%    \end{macrocode}
% \end{macro}
%
% \begin{macro}{\macro@impl@endargs}
%    \begin{macrocode}
\def\macro@impl@endargs{%
  \endgroup
  \unskip\par\noindent\ignorespaces
}
%    \end{macrocode}
% \end{macro}
%
% \begin{macro}{\macro@impl@argline}[2]{argument number}{argument description}
%    \begin{macrocode}
\def\macro@impl@argline#1#2{%
  \par{\texttt{\##1}:~#2\strut}%
}
%    \end{macrocode}
% \end{macro}
%
% \begin{macro}{\macro@impl@arg}[1]{argument description}
%    \begin{macrocode}
\def\macro@impl@arg#1{%
  \advance\macro@impl@curarg by\@ne\relax
  \macro@impl@argline{\the\macro@impl@curarg}{#1}%
  \ifnum\macro@impl@curarg<\macro@impl@argcnt\relax
    \expandafter\macro@impl@arg
  \else
    \expandafter\macro@impl@endargs
  \fi
}
%    \end{macrocode}
% \end{macro}
%
% \begin{environment}{macro}[1]{implemented macro}
%    \begin{macrocode}
\def\macro#1{%
  \PrintMacroImpl{#1}%
  \@ifnextchar[%]
    {\macro@impl@args}%
    {}%
}
\def\endmacro{}
%    \end{macrocode}
% \end{environment}
%
% \begin{environment}{key}[2]{key family}{key name}
%    \begin{macrocode}
\def\key#1#2{%
  \PrintMacroImpl{KV@#1@#2}%
  \@ifnextchar[%]
    {\macro@impl@args}%
    {}%
}
\def\endkey{}
%    \end{macrocode}
% \end{environment}
%
% \begin{environment}{environment}[1]{environment name}
%    \begin{macrocode}
\def\environment#1{%
  \PrintEnvImplName{#1}%
  \@ifnextchar[%]
    {\macro@impl@args}%
    {}%
}
\def\endenvironment{}
%    \end{macrocode}
% \end{environment}
%
%
% \begin{environment}{style}[1]{style name}
%    \begin{macrocode}
\def\style#1{%
  \PrintStyleImplName{#1}%
  \@ifnextchar[%]
    {\macro@impl@args}%
    {}%
}
\def\endstyle{}
\def\PrintStyleImplName{\PrintEnvImplName}
%    \end{macrocode}
% \end{environment}
%
% \begin{macro}{\PrintMacroImpl}[1]{macro (token)}
%    \begin{macrocode}
\def\PrintMacroImpl#1{%
  \par\bigskip\noindent
  \Needspace*{3\baselineskip}%
  \hbox{%
    \edef\name{\expandafter\@gobble\string#1}%
    \global\@namedef{href@impl@\name}{}%
    \immediate\write\@mainaux{%
      \global\noexpand\@namedef{href@impl@\name}{}%
    }%
    \raisebox{4ex}[4ex]{\hypertarget{impl:\name}{}}%
    \hspace*{\descindent}\fbox{%
      \hspace*{\descsep}%
      \@ifundefined{href@desc@\name}{}{\hyperlink{desc:\name}}%
      {\PrintMacroImplName{#1}}%
      \hspace*{\descsep}%
    }%
  }%
  \par\medskip\noindent
}
%    \end{macrocode}
% \end{macro}
%
% \begin{macro}{\PrintMacroImplName}[1]{macro (token)}
%    \begin{macrocode}
\def\PrintMacroImplName#1{%
  \implstyle{\string#1\strut}%
}
%    \end{macrocode}
% \end{macro}
%
% \begin{macro}{\PrintEnvImplName}[1]{environment name}
% test
%    \begin{macrocode}
\def\PrintEnvImplName#1{%
  \par\bigskip\noindent
  \hbox{\hspace*{\descindent}\fbox{{\implstyle{#1}}}}%
  \par\medskip
}
%    \end{macrocode}
% \end{macro}
%
% \begin{macro}{\implstyle}
%    \begin{macrocode}
\def\implstyle{\ttfamily\bfseries\color{macroimpl}}
%    \end{macrocode}
% \end{macro}
%
% \begin{macro}{\bslash}
% Defines an expandable backslash with catcode 12: `\texttt{\bslash}$_{12}$'.
% The |\@firstofone| trick is used to read the |\gdef\bslash| code before changing the catcode.
%    \begin{macrocode}
{%
\@firstofone
%    \end{macrocode}
% \end{macro}
% \iffalse
%</ydoc-code.sty>
% \fi
%
% \iffalse
%<*ydoc-doc.sty>
% \fi
% \subsection{Provide \pkg{doc} macros}
%    \begin{macrocode}
\NeedsTeXFormat{LaTeX2e}[1999/12/01]
\ProvidesPackage{ydoc-doc}[%
%<!DATE>
%<!VERSION>
%<*DRIVER>
    2099/01/01 develop
%</DRIVER>
    ydoc package to provide 'doc' macros]
%    \end{macrocode}
%
% \begin{macro}{\ydoc@countbslashes}
% Reads the macro code into a temp box. The backslashes are defined to increase a counter.
%    \begin{macrocode}
\newcount\ydoc@bslashcnt
\def\ydoc@countbslashes{%
  \begingroup
    \let\firstlinemacro\empty
    \let\lastlinemacro\empty
    \let\newlinemacro\empty
    \let\spacemacro\empty
    \def\bslashmacro{\global\advance\ydoc@bslashcnt by\@ne}%
    \setbox\@tempboxa\hbox{\themacrocode}%
  \endgroup
}
%    \end{macrocode}
% \end{macro}
%
% \begin{macro}{\CheckSum}
%    \begin{macrocode}
\def\CheckSum#1{%
  \gdef\ydoc@checksum{#1}%
}
\let\ydoc@checksum\m@ne
%    \end{macrocode}
% \end{macro}
%
% \begin{macro}{\AlsoImplementation}
% \begin{macro}{\OnlyDescription}
% \begin{macro}{\StopEventually}
% \begin{macro}{\Finale}
% The first two macros modify the \cs{StopEventually} macro which either
% stores its argument in \cs{Final} or executes it itself.
%    \begin{macrocode}
\def\AlsoImplementation{%
  \gdef\StopEventually##1{%
    \@bsphack
    \gdef\Finale{##1\ydoc@checkchecksum}%
    \@esphack
  }%
}
\AlsoImplementation
\def\OnlyDescription{%
  \@bsphack
  \long\gdef\StopEventually##1{##1\endinput}%
  \@esphack
}
\let\Finale\relax
%    \end{macrocode}
% \end{macro}
% \end{macro}
% \end{macro}
% \end{macro}
%
% \begin{macro}{\MakePercentComment}
% \begin{macro}{\MakePercentIgnore}
%    \begin{macrocode}
\def\MakePercentIgnore{\catcode`\%9\relax}
\def\MakePercentComment{\catcode`\%14\relax}
%    \end{macrocode}
% \end{macro}
% \end{macro}
%
% \begin{macro}{\DocInput}
%    \begin{macrocode}
\def\DocInput#1{\MakePercentIgnore\input{#1}\MakePercentComment}
%    \end{macrocode}
% \end{macro}
%
% \begin{macro}{\CharacterTable}
%    \begin{macrocode}
\providecommand*\CharacterTable{%
    \begingroup
    \CharTableChanges
    \@CharacterTable
}
\def\@CharacterTable#1{%
    \def\ydoc@used@CharacterTable{#1}%
    \@onelevel@sanitize\ydoc@used@CharacterTable
    \ifx\ydoc@used@CharacterTable\ydoc@correct@CharacterTable
           \typeout{***************************}%
           \typeout{* Character table correct *}%
           \typeout{***************************}%
    \else
         \PackageError{ydoc}{Character table corrupted}
                           {\the\wrong@table}
         \show\ydoc@used@CharacterTable
         \show\ydoc@correct@CharacterTable
    \fi
    \endgroup
}
\newhelp\wrong@table{Some of the ASCII characters are corrupted.^^J
            I now \string\show\space you both tables for comparison.}
\newcommand*\CharTableChanges{}
%    \end{macrocode}
% \end{macro}
%
% \begin{macro}{\ydoc@correct@CharacterTable}
%    \begin{macrocode}
\def\ydoc@correct@CharacterTable
  {Upper-case    \A\B\C\D\E\F\G\H\I\J\K\L\M\N\O\P\Q\R\S\T\U\V\W\X\Y\Z
   Lower-case    \a\b\c\d\e\f\g\h\i\j\k\l\m\n\o\p\q\r\s\t\u\v\w\x\y\z
   Digits        \0\1\2\3\4\5\6\7\8\9
   Exclamation   \!     Double quote  \"     Hash (number) \#
   Dollar        \$     Percent       \%     Ampersand     \&
   Acute accent  \'     Left paren    \(     Right paren   \)
   Asterisk      \*     Plus          \+     Comma         \,
   Minus         \-     Point         \.     Solidus       \/
   Colon         \:     Semicolon     \;     Less than     \<
   Equals        \=     Greater than  \>     Question mark \?
   Commercial at \@     Left bracket  \[     Backslash     \\
   Right bracket \]     Circumflex    \^     Underscore    \_
   Grave accent  \`     Left brace    \{     Vertical bar  \|
   Right brace   \}     Tilde         \~}
\@onelevel@sanitize\ydoc@correct@CharacterTable
%
%    \end{macrocode}
% \end{macro}
%
% \begin{macro}{\DoNotIndex}
%    \begin{macrocode}
\providecommand*\DoNotIndex[1]{%
  \PackageWarning{ydoc}{Ignoring DoNotIndex - not implemented yet!}{}{}%
}
%    \end{macrocode}
% \end{macro}
%
% \begin{macro}{\changes}
%    \begin{macrocode}
\providecommand*\changes[3]{%
  \PackageWarning{ydoc}{Ignoring changes - not implemented yet!}{}{}%
}
%    \end{macrocode}
% \end{macro}
%
% \begin{macro}{\RecordChanges}
%    \begin{macrocode}
\providecommand*\RecordChanges{%
  \PackageWarning{ydoc}{List of changes not implemented yet!}{}{}%
}
%    \end{macrocode}
% \end{macro}
%
% \begin{macro}{\PrintChanges}
%    \begin{macrocode}
\providecommand*\PrintChanges{%
  \PackageWarning{ydoc}{List of changes not implemented yet!}{}{}%
}
%    \end{macrocode}
% \end{macro}
%
% \begin{macro}{\PrintIndex}
%    \begin{macrocode}
\providecommand*\PrintIndex{%
  \PackageWarning{ydoc}{Code index not implemented yet!}{}{}%
}
%    \end{macrocode}
% \end{macro}
%
% \begin{macro}{\CodelineIndex}
%    \begin{macrocode}
\providecommand*\CodelineIndex{%
  \PackageWarning{ydoc}{Code line index not implemented yet!}{}{}%
}
%    \end{macrocode}
% \end{macro}
%
% \begin{macro}{\EnableCrossrefs}
%    \begin{macrocode}
\providecommand*\EnableCrossrefs{%
  \PackageWarning{ydoc}{Cross references not implemented yet!}{}{}%
}
%    \end{macrocode}
% \end{macro}
%
% \begin{macro}{\GetFileInfo}
% Current implementation taken from \pkg{doc} package.
%    \begin{macrocode}
\providecommand*\GetFileInfo[1]{%
  \def\filename{#1}%
  \def\@tempb##1 ##2 ##3\relax##4\relax{%
    \def\filedate{##1}%
    \def\fileversion{##2}%
    \def\fileinfo{##3}}%
  \edef\@tempa{\csname ver@#1\endcsname}%
  \expandafter\@tempb\@tempa\relax? ? \relax\relax
}
%    \end{macrocode}
% \end{macro}
%
% \begin{macro}{\ydoc@checkchecksum}
%    \begin{macrocode}
\def\ydoc@checkchecksum{%
  \ifnum\ydoc@checksum=\m@ne
    \message{^^J}%
    \message{**********************^^J}%
    \message{* No checksum found! *^^J}%
    \message{**********************^^J}%
    \GenericWarning{No checksum found}{Correct checksum is \the\ydoc@bslashcnt^^J}{}{}%
  \else
  \ifnum\ydoc@checksum=\z@
    \message{^^J}%
    \message{*********************^^J}%
    \message{* Checksum disabled *^^J}%
    \message{*********************^^J}%
    \GenericWarning{Checksum disabled}{Correct checksum is \the\ydoc@bslashcnt^^J}{}{}%
  \else
  \ifnum\ydoc@checksum=\ydoc@bslashcnt
    \message{^^J}%
    \message{*******************^^J}%
    \message{* Checksum passed *^^J}%
    \message{*******************^^J}%
  \else
    \message{^^J}%
    \message{****************************^^J}%
    \message{* Checksum wrong (\ydoc@checksum<>\the\ydoc@bslashcnt) ^^J}%
    \message{****************************^^J}%
    \GenericError{Checksum wrong}{Correct checksum is \the\ydoc@bslashcnt^^J}{}{}%
  \fi
  \fi
  \fi
}
%    \end{macrocode}
% \end{macro}
%
%    \begin{macrocode}
\RequirePackage{shortvrb}
\AtBeginDocument{\MakeShortVerb{\|}}
%    \end{macrocode}
%
%    \begin{macrocode}
\RequirePackage{url}

\def\package{\def\@package}
\package{\jobname}

\def\bundle{\def\@bundle}
\let\@bundle\@empty


\def\ctanlocation{\def\@ctanlocation##1}
\ctanlocation{http://www.ctan.org/pkg/#1}

\date{Version \fileversion\space -- \filedate}

\def\@homepage{%
    \begingroup
    \edef\@tempa{%
        \endgroup
        CTAN:
        \noexpand\url
        {\@ctanlocation{\ifx\@bundle\@empty\@package\else\@bundle\fi}}%
    }%
    \@tempa
}

\let\@repository\@empty
\protected\def\repository{\urldef\@repository\url}
\protected\def\homepage{\urldef\@homepage\url}
\protected\def\email{\hyper@normalise\email@}
\def\email@#1{\def\@plainemail{#1}\def\@email{\hyper@linkurl{\Hurl{#1}}{mailto:#1}}}

\let\@email\empty

\let\@plainemail\empty
\title{The \texorpdfstring{\pkgtitle{\@package}}{\@package} Package}
\def\@bundlesubtitle{Part of the \texorpdfstring{\pkgtitle{\@bundle}}{\@bundle} bundle}

\protected\def\pkgtitle#1{%
    \texorpdfstring{\textsf{#1}}{#1}%
}



\def\@maketitle{%
    \newpage
    \null\vskip 2em
    \begin{center}%
        \let\footnote\thanks
        {\LARGE \@title \par }\vskip 1.5em%
        \ifx\@bundle\@empty\else
        {\large \@bundlesubtitle \par }\vskip 1.5em%
        \fi
        {\large \lineskip .5em%
        \begin{tabular}[t]{c}%
            \@author
        \end{tabular}%
        \par}%
        \ifx\@plainemail\empty\else
            {\large \lineskip .5em%
            \begin{tabular}[t]{c}%
                \@email
            \end{tabular}%
            \par}%
        \fi
        \vskip 1em
        {\large \lineskip .5em%
        \begin{tabular}[t]{c}%
            \@homepage
        \end{tabular}%
        \par}%
        \vskip 1em
        \ifx\@repository\@empty\else
            {\large \lineskip .5em%
            \begin{tabular}[t]{c}%
                VC: \@repository
            \end{tabular}%
            \par}%
        \fi
        \vskip 1em
        {\large \@date }%
    \end{center}%
    \par\vskip 1.5em
    \aftergroup\ydocpdfsettings
}

\ifpdf
\def\ydocpdfsettings{%
    \hypersetup{%
        pdfauthor   = {\@author\space<\@plainemail>},
        pdftitle    = {\@title},
        pdfsubject  = {Documentation of LaTeX package \@package},
        pdfkeywords = {\@package, LaTeX, TeX}
    }%
}
\else
\let\ydocpdfsettings\empty
\fi

\let\orig@maketitle\maketitle
\def\maketitle{%
    \ydocpdfsettings
    \orig@maketitle
    \let\orig@maketitle\relax
}
%    \end{macrocode}
% \iffalse
%</ydoc-doc.sty>
% \fi
%
% \iffalse
%<*ydoc-desc.sty>
% \fi
% \subsection{Description Macros and Environments}
%
%    \begin{macrocode}
\NeedsTeXFormat{LaTeX2e}[1999/12/01]
\ProvidesPackage{ydoc-desc}[%
%<!DATE>
%<!VERSION>
%<*DRIVER>
    2099/01/01 develop
%</DRIVER>
    ydoc package to describe macros, environments, options etc.]
%    \end{macrocode}
%
%    \begin{macrocode}
\IfFileExists{needspace.sty}{%
    \RequirePackage{needspace}
}{%
    \def\Needspace{\@ifstar\@gobble\@gobble}
}
%    \end{macrocode}
%
% The short verbatim code is required for the similar macros provided here.
%    \begin{macrocode}
\RequirePackage{shortvrb}
%    \end{macrocode}
%
% The \pkg{etoolbox} package is used mainly for |\newrobustcmd|.
%    \begin{macrocode}
\RequirePackage{etoolbox}
%    \end{macrocode}
%
% \subsubsection{Color and style definitions}
%    \begin{macrocode}
\RequirePackage{xcolor}
%    \end{macrocode}
% Define special no-op `none' color which does not change the color.
% This is not yet tested and may break output files, but seems to work fine with PDF.
%    \begin{macrocode}
\expandafter\def\csname\string\color@none\endcsname{%
    \xcolor@ {}{}{}{}
}
%    \end{macrocode}
%    \begin{macrocode}
\definecolor{macrodesc}{rgb}{0,0.2,0.6}
\definecolor{keydesc}{rgb}{0,0.4,0.9}
\definecolor{macroimpl}{rgb}{0,0.1,0.3}
\definecolor{meta}{rgb}{0,0.25,0.75}
\definecolor{scriptcolor}{rgb}{0.2,0.6,0.2}
\definecolor{optioncolor}{rgb}{0.3.0.2,0}
\colorlet{optional}{black!65!white}
\colorlet{metaoptional}{optional!50!meta}
\providecolor{urlcolor}{named}{blue}
\providecolor{linkcolor}{named}{blue}
\providecolor{filecolor}{named}{blue}
\providecolor{citecolor}{named}{blue}
\providecolor{anchorcolor}{named}{blue}
\providecolor{menucolor}{named}{blue}
\providecolor{runcolor}{named}{blue}

%    \end{macrocode}
%
%    \begin{macrocode}
\RequirePackage{hyperref}
\hypersetup{%
    colorlinks=true,
    pdfborder=0 0 0,
    pdfborderstyle={},
    urlcolor=urlcolor,
    linkcolor=linkcolor,
    filecolor=filecolor,
    citecolor=citecolor,
    anchorcolor=anchorcolor,
    menucolor=menucolor,
    runcolor=runcolor,
}

%    \end{macrocode}
%
% \subsubsection{Text Formatting Macros}
%
% \begin{macro}{\meta}
% Prints \meta{meta text}.
%    \begin{macrocode}
\newrobustcmd*\meta[1]{%
  {\metastyle{%
  \ensuremath\langle
    #1\/%
  \ensuremath\rangle
  }}%
}
%    \end{macrocode}
% \end{macro}
%
% \begin{macro}{\marg}
% Sets style and adds braces.
% The text is formatted as separate set of macro arguments.
%    \begin{macrocode}
\newrobustcmd*{\marg}[1]{%
  {\margstyle{%
    {\ttfamily\braceleft}%
    \meta{#1}%
    {\ttfamily\braceright}%
  }}%
}
%    \end{macrocode}
% \end{macro}
%
% \begin{macro}{\oarg}
% Sets style and adds brackets.
% The text is formatted as separate set of macro arguments.
%    \begin{macrocode}
\newrobustcmd*{\oarg}[1]{%
  {\oargstyle{%
    {\ttfamily[}%
    \meta{#1}%
    {\ttfamily]}%
  }}%
}
%    \end{macrocode}
% \end{macro}
%
% \begin{macro}{\parg}
% Sets style and adds parentheses.
%    \begin{macrocode}
\newrobustcmd*{\parg}[1]{%
  {\pargstyle{%
    {\ttfamily(}%
    \meta{#1}%
    {\ttfamily)}%
  }}%
}
%    \end{macrocode}
% \end{macro}
%
% \begin{macro}{\aarg}
% Sets style and adds angles.
%    \begin{macrocode}
\newrobustcmd*{\aarg}[1]{%
  {\aargstyle{%
    {\ttfamily<}%
    \meta{#1}%
    {\ttfamily>}%
  }}%
}
%    \end{macrocode}
% \end{macro}
%
% \begin{macro}{\sarg}
% Prints star with given style.
%    \begin{macrocode}
\newrobustcmd*{\sarg}{{\sargstyle{*}}}
%    \end{macrocode}
% \end{macro}
%
% \begin{macro}{\pkg}
% \begin{macro}{\cls}
% \begin{macro}{\lib}
% \begin{macro}{\env}
% \begin{macro}{\opt}
% \begin{macro}{\file}
%    \begin{macrocode}
\newrobustcmd*\pkg[1]{{\pkgstyle{#1}}}
\newrobustcmd*\cls[1]{{\clsstyle{#1}}}
\newrobustcmd*\lib[1]{{\libstyle{#1}}}
\newrobustcmd*\env[1]{{\envstyle{#1}}}

\newrobustcmd*\opt{\@ifstar\ys@opt\y@opt}
\def\y@opt#1{{\optstyle{#1}}}
\def\ys@opt#1{{\optstyle{#1}}\optpar{#1}}
\newrobustcmd*\optpar[1]{\marginpar{\hbox to \marginparwidth{\hss\y@opt{#1}}}}

\newrobustcmd*\file[1]{{\filestyle{#1}}}
\newcommand*\pkgstyle[1]{\texttt{\textcolor{pkg}{#1}}}
\newcommand*\clsstyle[1]{\texttt{\textcolor{cls}{#1}}}
\newcommand*\libstyle[1]{\texttt{\textcolor{lib}{#1}}}
\newcommand*\envstyle[1]{\texttt{\textcolor{env}{#1}}}
\newcommand*\optstyle[1]{\textsf{\textcolor{opt}{#1}}}
\newcommand*\filestyle[1]{\texttt{\textcolor{file}{#1}}}
\colorlet{cls}{none}
\colorlet{lib}{none}
\colorlet{env}{none}
\colorlet{file}{none}
\colorlet{pkg}{none}
\definecolor{opt}{rgb}{0.5,0.16666,0}
%    \end{macrocode}
% \end{macro}
% \end{macro}
% \end{macro}
% \end{macro}
% \end{macro}
% \end{macro}
%
% \begin{macro}{\cs}
% \begin{macro}{\cmd}
%    \begin{macrocode}
\newrobustcmd*\cs[1]{\texttt{\textbackslash #1}}
\newrobustcmd*\cmd[1]{\texttt{{\escapechar=92\string#1}}}
%    \end{macrocode}
% \end{macro}
% \end{macro}
%
% \begin{macro}{\Key}
%    \begin{macrocode}
\newrobustcmd*\Key[1]{\PrintKeyName{#1}\MacroArgs}
%    \end{macrocode}
% \end{macro}
%
% \subsubsection{Text Formatting Styles}
%
% \begin{macro}{\macrodescstyle}
% Style of described macro names.
%    \begin{macrocode}
\def\macrodescstyle{\ttfamily\bfseries\color{macrodesc}}
%    \end{macrocode}
% \end{macro}
%
% \begin{macro}{\macrodescstyle}
% Style of described macro names.
%    \begin{macrocode}
\def\keydescstyle{\ttfamily\bfseries\color{keydesc}}
%    \end{macrocode}
% \end{macro}
%
% \begin{macro}{\macroargsstyle}
% Default style for macro arguments (e.g.\ \cs{MacroArgs}).
%    \begin{macrocode}
\def\macroargsstyle{\ttfamily}
%    \end{macrocode}
% \end{macro}
%
% \begin{macro}{\envcodestyle}
% Default style for code body content in described environments.
%    \begin{macrocode}
\def\envcodestyle{\ttfamily}
%    \end{macrocode}
% \end{macro}
%
% \begin{macro}{\verbstyle}
% Style for verbatim text inside macro argument list.
%    \begin{macrocode}
\def\verbstyle{\verbatim@font}
%    \end{macrocode}
% \end{macro}
%
% \begin{macro}{\metastyle}
% Meta text style. Because \cs{macroargsstyle} might be also active a |\normalfont| reset
% the font.
%    \begin{macrocode}
\def\metastyle{\normalfont\itshape\color{meta}}
%    \end{macrocode}
% \end{macro}
%
% \begin{macro}{\margstyle}
% Style for \cs{marg}.
%    \begin{macrocode}
\def\margstyle{}
%    \end{macrocode}
% \end{macro}
%
% \begin{macro}{\Optional}
% \begin{macro}{\optional}
% \begin{macro}{\optionalstyle}
%    \begin{macrocode}
\protected\def\Optional{\optionalon\optional}
\def\optionalstyle{\blendcolors*{!60!white}\color{black!75}}
%    \end{macrocode}
% \end{macro}
% \end{macro}
% \end{macro}
%
% \begin{macro}{\optionalon}
% \begin{macro}{\optionaloff}
%    \begin{macrocode}
\def\optionalon{\protected\def\optional{\optionalstyle}}
\def\optionaloff{\let\optional\relax}
\optionalon
%    \end{macrocode}
% \end{macro}
% \end{macro}
%
% \begin{macro}{\oargstyle}
% Style for \cs{oarg}. A special color is set to show the `optional' status.
%    \begin{macrocode}
\def\oargstyle{\optional}
%    \end{macrocode}
% \end{macro}
%
% \begin{macro}{\pargstyle}
% Style for \cs{parg}.
%    \begin{macrocode}
\def\pargstyle{}
%    \end{macrocode}
% \end{macro}
%
% \begin{macro}{\aargstyle}
% Style for \cs{aarg}.
%    \begin{macrocode}
\def\aargstyle{}
%    \end{macrocode}
% \end{macro}
%
% \begin{macro}{\sargstyle}
% Style for \cs{sarg}. A special color is set to show the `optional' status.
%    \begin{macrocode}
\def\sargstyle{\ttfamily\color{optional}}
%    \end{macrocode}
% \end{macro}
%
% \subsubsection{Dimension Registers}
%
% \begin{macro}{\descindent}
%    \begin{macrocode}
\newdimen\descindent
\descindent=-\parindent
%    \end{macrocode}
% \end{macro}
%
% \begin{macro}{\beforedescskip}
%    \begin{macrocode}
\newdimen\beforedescskip
\beforedescskip=\bigskipamount
%    \end{macrocode}
% \end{macro}
%
% \begin{macro}{\afterdescskip}
%    \begin{macrocode}
\newdimen\afterdescskip
\afterdescskip=\medskipamount
%    \end{macrocode}
% \end{macro}
%
% \begin{macro}{\descsep}
% Set to |1em| in |tt| font.
%    \begin{macrocode}
\newdimen\descsep
\begingroup
\ttfamily
\global\descsep=1em\relax
\endgroup
%    \end{macrocode}
% \end{macro}
%
%
% \subsubsection{Macro Argument Reading Mechanism}
%
% \begin{macro}{\read@Macro@arg}
% Reads next token and calls second macro.
%    \begin{macrocode}
\def\read@Macro@arg{%
  \futurelet\@let@token\handle@Macro@arg
}
%    \end{macrocode}
% \end{macro}
%
% \begin{macro}{\AlsoMacro}
% Reads argument while |@| is a letter,
% prints the macro name and reads further arguments.
%    \begin{macrocode}
\newcommand*\AlsoMacro{%
  \begingroup\makeatletter
  \AlsoMacro@
}
\def\AlsoMacro@#1{%
  \endgroup
%<*DEBUG>
   %\typeout{DEBUG: Macro: \string#1}%
%</DEBUG>
  \PrintMacroName{#1}%
  \read@Macro@arg
}
%    \end{macrocode}
% \end{macro}
%
% \begin{macro}{\ydoc@short@AlsoMacro}
% Makes |&| an alias for |\AlsoMacro|.
%    \begin{macrocode}
\begingroup
\catcode`\|\active
\gdef\ydoc@short@AlsoMacro{%
  \catcode`\|\active
  \let|\AlsoMacro
}
\endgroup
%    \end{macrocode}
% \end{macro}
%
%
% \begin{macro}{\ydoc@macrocatcodes}
% Sets the catcodes inside for |read@Macro@arg| material.
%    \begin{macrocode}
\def\ydoc@macrocatcodes{%
  \ydoc@short@AlsoMacro
  \@makeother\'%
  \@makeother\!%
  \@makeother\[%
  \@makeother\]%
  \@makeother\(%
  \@makeother\)%
}
%    \end{macrocode}
% \end{macro}
%
%
% \begin{macro}{\handle@Macro@arg}
% Checks if next token is the begin of a valid macro argument and calls
% the appropriate read macro or the end macro otherwise.
%    \begin{macrocode}
\def\handle@Macro@arg{%
  \expandafter\let\expandafter\handler\csname handle@Macro@token@\meaning\@let@token\endcsname
  \ifx\handler\relax
    \def\handler{\ifhmode\unskip\fi\end@Macro@args}%
%<*DEBUG>
 %  \typeout{DEBUG: Stopped at: \expandafter\meaning\csname @let@token\endcsname}%
 %  \typeout{}%
 %\else
 %\expandafter\ifx\csname @let@token\endcsname\AlsoMacro
 %  \typeout{DEBUG: TOKEN: \string\AlsoMacro}%
 %\else
 %  \typeout{DEBUG: TOKEN: \expandafter\meaning\csname @let@token\endcsname}%
 %\fi
%</DEBUG>
  \fi
  \handler
}
\def\define@Macro@handler{%
    \begingroup
    \ydoc@macrocatcodes
    \define@Macro@handler@
}
\def\define@Macro@handler@#1{%
    \endgroup
    \@namedef{handle@Macro@token@\meaning#1}%
}
%    \end{macrocode}
% \end{macro}
%
% \begin{macro}{\end@Macro@args}
% Closes box as calls hook. Might be locally redefined by some
% macros calling \cs{read@Macro@arg}.
%    \begin{macrocode}
\def\end@Macro@args{%
  \y@egroup
  \after@Macro@args
}
%    \end{macrocode}
% \end{macro}
%
% \begin{macro}{\after@Macro@args}
% Hook to add additional commands in certain situations.
%    \begin{macrocode}
\def\after@Macro@args{%
}
%    \end{macrocode}
% \end{macro}
%
% \subsubsection*{Macro argument reading macros}
% This macros read the macro arguments and call the appropriate format macros.
%
% \begin{macro}{\read@Macro@marg}
%    \begin{macrocode}
\define@Macro@handler{\bgroup}{%
    \begingroup
        \afterassignment\read@Macro@marg@
        \let\@let@token=%
}
\def\read@Macro@marg@{%
        \bgroup
        \margstyle{}%
        \let\end@Macro@args\empty%
        {\ttfamily\braceleft}%
        \aftergroup\read@Macro@marg@@
        \read@Macro@arg
}
\def\read@Macro@marg@@{%
        {\ttfamily\braceright}%
    \endgroup
    \read@Macro@arg
}
%    \end{macrocode}
% \end{macro}
%
% \begin{macro}{\read@Macro@oarg}
%    \begin{macrocode}
\define@Macro@handler{[}[{%
    \begingroup
        \let\read@Macro@oarg@end\read@Macro@oarg@@end
        \let\end@Macro@args\read@Macro@oarg@end
        \oargstyle{}%
        {\ttfamily[}%]
        \read@Macro@arg
}
\define@Macro@handler{]}{%
    \read@Macro@oarg@end
}
\def\read@Macro@oarg@@end#1]{%
        #1%
        {\ttfamily]}%
    \endgroup
    \read@Macro@arg
}
\def\read@Macro@oarg@end{\end@Macro@args}
\let\read@Macro@aarg@end\read@Macro@oarg@end
\let\read@Macro@parg@end\read@Macro@oarg@end
%    \end{macrocode}
% \end{macro}
%
% \begin{macro}{\read@Macro@parg}
%    \begin{macrocode}
\define@Macro@handler{(}({%
    \begingroup
        \let\read@Macro@parg@end\read@Macro@parg@@end
        \let\end@Macro@args\read@Macro@parg@end
        \pargstyle{}%
        {\ttfamily(}%)
        \read@Macro@arg
}
\define@Macro@handler{)}{%
    \read@Macro@parg@end
}
\def\read@Macro@parg@@end#1){%
        #1%
        {\ttfamily)}%
    \endgroup
    \read@Macro@arg
}
%    \end{macrocode}
% \end{macro}
%
% \begin{macro}{\read@Macro@aarg}
%    \begin{macrocode}
\def\read@Macro@aarg<{%
    \begingroup
        \let\read@Macro@aarg@end\read@Macro@aarg@@end
        \let\end@Macro@args\read@Macro@aarg@end
        \aargstyle{}%
        {\ttfamily<}%
        \read@Macro@arg
}
\define@Macro@handler{>}{%
    \read@Macro@aarg@end
}
\def\read@Macro@aarg@@end#1>>{%
        #1%
        {\ttfamily>}%
    \endgroup
    \read@Macro@arg
}
%    \end{macrocode}
% \end{macro}
%
% \begin{macro}{\read@Macro@angle}
%    \begin{macrocode}
\define@Macro@handler{<}<{%
  \futurelet\@let@token\read@Macro@angle@
}
%    \end{macrocode}
% \end{macro}
%
% \begin{macro}{\read@Macro@angle@}
%    \begin{macrocode}
\def\read@Macro@angle@{%
  \ifx\@let@token<%
    \expandafter\read@Macro@aarg
  \else
    \expandafter\read@Macro@meta
  \fi
}
%    \end{macrocode}
% \end{macro}
%
% \begin{macro}{\read@Macro@meta}
%    \begin{macrocode}
\def\read@Macro@meta#1>{%
  \meta{#1}\read@Macro@arg
}
%    \end{macrocode}
% \end{macro}
%
% \begin{macro}{\read@Macro@sarg}
%    \begin{macrocode}
\define@Macro@handler**{%
  \sarg\read@Macro@arg
}
%    \end{macrocode}
% \end{macro}
%
% Allows `|=|' to be used directly without switching to verbatim
% mode. This is especially useful for keys.
%    \begin{macrocode}
\define@Macro@handler{=}={%
  =\read@Macro@arg
}
%    \end{macrocode}
%
% \begin{macro}{\read@Macro@verb}
% Sets up verbatim mode calls second macro.
%    \begin{macrocode}
\define@Macro@handler{'}'{%
  \begingroup
  \let\do\@makeother
  \dospecials
  \@noligs
  \@makeother\'%
  \obeyspaces
  \read@Macro@verb@
}
%    \end{macrocode}
% \end{macro}
%
% \begin{macro}{\read@Macro@verb@}
% Closes verbatim mode and formats text.
% If |#1| is empty (|''|) than a single |'| is printed.
%    \begin{macrocode}
\begingroup
\@makeother\'%
\gdef\read@Macro@verb@#1'{%
  \endgroup
  \ifx\relax#1\relax
    {\verbstyle{\string'}}%
  \else
    {%
     \frenchspacing
     \@noligs\verbstyle{#1}}%
  \fi
  \read@Macro@arg
}
\endgroup
%    \end{macrocode}
% \end{macro}
%
% \begin{macro}{\read@Macro@cmds}
% Simply executes given code.
%    \begin{macrocode}
\define@Macro@handler!!#1!{%
  #1\relax
  \read@Macro@arg
}
%    \end{macrocode}
% \end{macro}
%
% \begin{macro}{\read@Macro@rmspace}
% Removes space. The |\@firstofone| is used to preserve the space in the macro definition.
%    \begin{macrocode}
\define@Macro@handler{\@sptoken} {%
  \read@Macro@arg
}
%    \end{macrocode}
% \end{macro}
%
% \begin{macro}{\read@Macro@addtoken}
% Takes token over from input to output `stream'. This is used for |\space| and |~|.
%    \begin{macrocode}
\define@Macro@handler{~}#1{%
  #1\read@Macro@arg
}
\AtBeginDocument{%
\define@Macro@handler{~}#1{%
  #1\read@Macro@arg
}
}
\define@Macro@handler{\space}#1{%
  #1\read@Macro@arg
}
%    \end{macrocode}
% \end{macro}
%
% \subsubsection{Description Macros}
% \subsubsection*{For Macros}
%
% \begin{macro}{\DescribeMacro}
%    \begin{macrocode}
\@ifundefined{DescribeMacro}{}{%
  \PackageInfo{ydoc-desc}{Redefining \string\DescribeMacro}{}%
}
%    \end{macrocode}
% A \cs{DescribeMacro} places itself in a \env{DescribeMacros} environment.
% Multiple \cs{DescribeMacro} macros will stack themselves inside this environment.
% For this to work \cs{DescribeMacros} is locally defined to |\y@egroup| to close the
% |\hbox| from the previous \cs{DescribeMacro}.
%    \begin{macrocode}
\def\DescribeMacro{%
  \DescribeMacros
  \let\DescribeMacros\y@egroup
  \optionalon
  \def\after@Macro@args{\endDescribeMacros}%
  \begingroup\makeatletter
  \Describe@Macro
}
%    \end{macrocode}
% \end{macro}
%
% \begin{macro}{\DescribeScript}
%    \begin{macrocode}
\def\DescribeScript#1{%
  \DescribeMacros
  \let\DescribeMacros\y@egroup
  \optionalon
  \def\after@Macro@args{\endDescribeMacros}%
  \hbox\y@bgroup
  \texttt{#1}%
  \ydoc@macrocatcodes
  \macroargsstyle
  \read@Macro@arg~%
}
%    \end{macrocode}
% \end{macro}
%
% \begin{macro}{\DescribeKey}
%    \begin{macrocode}
\def\DescribeKey{%
  \DescribeKeys
  \let\DescribeKeys\y@egroup
  \optionalon
  \def\after@Macro@args{\endDescribeKeys}%
  \begingroup\makeatletter
  \Describe@Macro
}
%    \end{macrocode}
% \end{macro}
%
% \begin{macro}{\Describe@Macro}
%    \begin{macrocode}
\def\Describe@Macro#1{%
  \endgroup
  \edef\name{\expandafter\@gobble\string#1}%
  \global\@namedef{href@desc@\name}{}%
  \immediate\write\@mainaux{%
    \global\noexpand\@namedef{href@desc@\name}{}%
  }%
  \hbox\y@bgroup
  \@ifundefined{href@impl@\name}{}{\hyperlink{impl:\name}}%
  {%
  \hbox{\vbox to 0pt{\vss\hbox{\raisebox{4ex}{\hypertarget{desc:\name}{}}}}%
  \PrintMacroName{#1}}%
  }%
  \ydoc@macrocatcodes
  \macroargsstyle
  \read@Macro@arg
}
%    \end{macrocode}
% \end{macro}
%
% \begin{macro}{\MakeShortMacroArgs}
% Defines the given character as short version for |\MacroArgs|.
% It is first define to be a short verbatim character to take
% advantage of the house-keeping (save \& restore of the original
% catcode and definition) of |shortvrb|.
%
% The starred version define the character to act like |\Macro| instead.
%    \begin{macrocode}
\newcommand*\MakeShortMacroArgs{%
  \@ifstar
    {\@MakeShortMacroArgs\Macro}%
    {\@MakeShortMacroArgs\MacroArgs}%
}
\def\@MakeShortMacroArgs#1#2{%
  \MakeShortVerb{#2}
  \catcode`#2\active
  \begingroup
  \catcode`\~\active
  \lccode`\~`#2\relax
  \lowercase{\endgroup\gdef~{\bgroup\let~\egroup#1}}%
}
%    \end{macrocode}
% \end{macro}
%
% \begin{macro}{\DeleteShortMacroArgs}
%    \begin{macrocode}
\newcommand*\DeleteShortMacroArgs[1]{%
  \DeleteShortVerb{#1}%
}
%    \end{macrocode}
% \end{macro}
%
% \begin{macro}{\Macro}
% Simply uses the two macros below.
%    \begin{macrocode}
\newcommand*\Macro{\MacroArgs\AlsoMacro}
%    \end{macrocode}
% \end{macro}
%
% \begin{macro}{\@Macro}
% Alternative definition of \cs{Macro} inside \env{DescribeMacros} environments.
%    \begin{macrocode}
\def\@Macro{%
  \begingroup\makeatletter
  \Describe@Macro
}
%    \end{macrocode}
% \end{macro}
%
%    \begin{macrocode}
\define@Macro@handler\AlsoMacro{}
\define@Macro@handler\DescribeMacro{}
\define@Macro@handler\DescribeKey{}
\define@Macro@handler\DescribeScript{}
%    \end{macrocode}
%
% \begin{macro}{\MacroArgs}
% Uses the normal macro argument reading mechanism from \cs{DescribeMacro}.
% Instead of a box a simple group is added.
%    \begin{macrocode}
\newcommand*\MacroArgs{%
  \begingroup
  \def\end@Macro@args{\endgroup\xspace}%
  \ydoc@macrocatcodes
  \macroargsstyle
%<*DEBUG>
   %\typeout{}%
   %\typeout{DEBUG: Start MacroArgs}%
%</DEBUG>
  \read@Macro@arg
}
\RequirePackage{xspace}
%    \end{macrocode}
% \end{macro}
%
% \begin{macro}{\DescribeMacros}
%    \begin{macrocode}
\def\DescribeMacros{%
  \begingroup
  \let\Macro\@Macro
  \parindent=0pt\relax
  \setbox\descbox\vbox\y@bgroup
}
%    \end{macrocode}
% \end{macro}
%
% \begin{macro}{\endDescribeMacros}
%    \begin{macrocode}
\def\endDescribeMacros{%
  \y@egroup
  \PrintMacros
  \endgroup
}
%    \end{macrocode}
% \end{macro}
%
% \begin{macro}{\DescribeKeys}
%    \begin{macrocode}
\def\DescribeKeys{%
  \begingroup
  \let\PrintMacroName\PrintKeyName
  \let\Key\@Macro
  \parindent=0pt\relax
  \setbox\descbox\vbox\y@bgroup
}
%    \end{macrocode}
% \end{macro}
%
% \begin{macro}{\endDescribeKeys}
%    \begin{macrocode}
\def\endDescribeKeys{%
  \y@egroup
  \PrintKeys
  \endgroup
}
\def\PrintKeys{\PrintMacros}
%    \end{macrocode}
% \end{macro}
%
% \begin{macro}{\DescribeMacrosTabcolsep}
%    \begin{macrocode}
\def\DescribeMacrosTabcolsep{\tabcolsep}
%    \end{macrocode}
% \end{macro}
%
% \begin{macro}{\DescribeMacrosTab}
%    \begin{macrocode}
\def\DescribeMacrosTab{%
  \DescribeMacros
  \hbox\y@bgroup
  \tabcolsep=\DescribeMacrosTabcolsep\relax
  \DescribeMacrosTab@
}
\def\DescribeMacrosTab@#1{\tabular{@{}#1@{}}}
%    \end{macrocode}
% \end{macro}
%
% \begin{macro}{\endDescribeMacrosTab}
%    \begin{macrocode}
\def\endDescribeMacrosTab{%
  \endtabular\y@egroup
  \endDescribeMacros
}
%    \end{macrocode}
% \end{macro}
%
% \subsubsection*{For Lengths}
%
% \begin{macro}{\DescribeLength}
%    \begin{macrocode}
\newcommand*\DescribeLength{%
  \begingroup
  \let\DescribeLength\Describe@Length
  \setbox\descbox\hbox\y@bgroup
    \tabular{@{}l@{\hspace{2em}}l@{}}%
    \Describe@Length
}
%    \end{macrocode}
% \end{macro}
%
% \begin{macro}{\Describe@Length}
%    \begin{macrocode}
\newcommand*\Describe@Length[2]{%
  \PrintLengthName{#1}&
  (Default: {\macroargsstyle#2\unskip})%
  \@ifnextchar\DescribeLength
    {\\}%
    {%
      \endtabular
      \y@egroup
      \PrintLength
      \endgroup
    }%
}
%    \end{macrocode}
% \end{macro}
%
% \subsubsection*{For Environments}
%
% \begin{macro}{\DescribeEnv}
%    \begin{macrocode}
\@ifundefined{DescribeEnv}{}{%
  \PackageInfo{ydoc-desc}{Redefining \string\DescribeEnv}{}%
}
\let\DescribeEnv\relax
%    \end{macrocode}
%    \begin{macrocode}
\newcommand*\DescribeEnv[2][]{%
  \begingroup
  \def\DescribeEnv@name{#2}%
  \let\\\DescribeEnv@newline
%    \end{macrocode}
%
% Sets after-macro-arguments hook.
% First checks if the environment or macro version was used.
% The environment starts a new line only if the next token isn't |\end|,
% which is taken as end of the environment.
%    \begin{macrocode}
  \ifx\@currenvir\DescribeEnv@string
    \def\after@Macro@args{%
      \let\after@Macro@args\empty
      \setbox\@tempboxa\hbox\y@bgroup
      \@ifnextchar\end{}%
        {\DescribeEnv@newline}%
      #1%
    }%
%    \end{macrocode}
% The macro version adds the optional argument as content line if given.
%    \begin{macrocode}
  \else
    \ifx\relax#1\relax
      \def\after@Macro@args{%
        \y@bgroup
        \endDescribeEnv
      }%
    \else
      \def\after@Macro@args{%
        \setbox\@tempboxa\hbox\y@bgroup
        \DescribeEnv@newline\MacroArgs#1%
        \endDescribeEnv
      }%
    \fi
  \fi
%    \end{macrocode}
% Start |\vbox| and adds first line.
%    \begin{macrocode}
  \setbox\descbox\vbox\y@bgroup
  \envcodestyle
  \let\PrintEnv\PrintSubEnv
  \hbox\y@bgroup
  \PrintEnvName{\begin}{\DescribeEnv@name}%
  \ydoc@macrocatcodes
  \macroargsstyle
  \read@Macro@arg
}
%    \end{macrocode}
% \end{macro}
%
% \begin{macro}{\DescribeEnv@newline}
% Closes existing and starts a new horizontal box representing a indented line.
% The optional argument allows to add extra space between lines like the normal |\\|.
% Negative values are not supported.
%    \begin{macrocode}
\newcommand*\DescribeEnv@newline[1][0pt]{%
  \strut\y@egroup
  {\vskip#1}%
  \hbox\y@bgroup\strut
  \hspace*{\descsep}%
  \ignorespaces
}%
%    \end{macrocode}
% \end{macro}
%
% \begin{macro}{\DescribeEnv@string}
% Holds the environment name for comparison.
%    \begin{macrocode}
\def\DescribeEnv@string{DescribeEnv}
%    \end{macrocode}
% \end{macro}
%
% \begin{macro}{\descbox}
% Save box to store description content.
%    \begin{macrocode}
\newbox\descbox
%    \end{macrocode}
% \end{macro}
%
% \begin{macro}{\endDescribeEnv}
%    \begin{macrocode}
\def\endDescribeEnv{%
  \y@egroup
  \begingroup
  \setbox\@tempboxa\lastbox
  \ifcase0%
    \ifdim\wd\@tempboxa>\descsep1\fi
    \ifdim\ht\@tempboxa>\ht\strutbox1\fi
    \ifdim\dp\@tempboxa>\dp\strutbox1\fi
  \else
    \box\@tempboxa
  \fi
  \endgroup
  \hbox\y@bgroup
    \PrintEnvName{\end}{\DescribeEnv@name}
  \y@egroup
  \y@egroup
  \PrintEnv
  \endgroup
}
%    \end{macrocode}
% \end{macro}
%
%
% \subsubsection{Print Macros}
%
% \begin{macro}{\PrintMacroName}
% Formats macro name. The backslash is forced to |tt| font.
%    \begin{macrocode}
\def\PrintMacroName#1{%
  {\macrodescstyle{\strut
   \texttt{\char92}%
   \escapechar\m@ne
   \string#1\strut}}%
}
%    \end{macrocode}
% \end{macro}
%
% \begin{macro}{\PrintKeyName}
% Formats macro name. The backslash is forced to |tt| font.
%    \begin{macrocode}
\def\PrintKeyName#1{%
  {\keydescstyle{\strut
   #1\strut}}%
}
%    \end{macrocode}
% \end{macro}
%
% \begin{macro}{\PrintLengthName}
% Formats length register name.
%    \begin{macrocode}
\let\PrintLengthName\PrintMacroName
%    \end{macrocode}
% \end{macro}
%
%
% \begin{macro}{\PrintEnvName}
% |#1| = `|\begin|' or `|\end|', |#2| = env name.
%    \begin{macrocode}
\def\PrintEnvName#1#2{%
  \strut
  \string#1\braceleft
  {\macrodescstyle#2\strut}%
  \braceright
}
%    \end{macrocode}
% \end{macro}
%
% \begin{macro}{\PrintMacros}
% Prints macros described using \cs{DescribeMacros}.
% The actual content was stored inside \cs{descbox}.
% If it is wider than the line width it is centered.
%    \begin{macrocode}
\def\PrintMacros{%
  \par\vspace\beforedescskip
  \begingroup
  \sbox\@tempboxa{\descframe{\usebox{\descbox}}}%
  \Needspace*{\dimexpr\ht\@tempboxa+3\baselineskip\relax}%
  \par\noindent
  \ifdim\wd\@tempboxa>\dimexpr\linewidth-2\descindent\relax
    \makebox[\linewidth][c]{\usebox\@tempboxa}%
  \else
    \hspace*{\descindent}%
    \usebox\@tempboxa
  \fi
  \endgroup
  \par
  \vspace\afterdescskip
  \par\noindent
}
\def\descframe#1{%
    \fbox{\hspace*{\descsep}#1\hspace*{\descsep}}%
}
%    \end{macrocode}
% \end{macro}
%
% \begin{macro}{\PrintLength}
% Prints lengths registers described using one or multiple \cs{DescribeLength}.
%    \begin{macrocode}
\let\PrintLength\PrintMacros
%    \end{macrocode}
% \end{macro}
%
% \begin{macro}{\PrintEnv}
% Prints \env{DescribeEnv} environments.
% The actual content was stored inside \cs{descbox}.
%    \begin{macrocode}
\let\PrintEnv\PrintMacros
%    \end{macrocode}
% \end{macro}
%
% \begin{macro}{\PrintSubEnv}
% Prints sub environments, i.e.\ \env{DescribeEnv} environments
% inside the body of another \env{DescribeEnv}.
% The actual content was stored inside \cs{descbox}.
%    \begin{macrocode}
\def\PrintSubEnv{%
  \hbox{\hbox{\usebox{\descbox}}}%
}
%    \end{macrocode}
% \end{macro}
%
%
% \subsubsection{Special Character Macros}
%
% \begin{macro}{\bslash}
% Defines an expandable backslash with catcode 12: `\texttt{\bslash}$_{12}$'.
% The |\@firstofone| trick is used to read the |\gdef\bslash| code before changing the catcode.
%    \begin{macrocode}
{%
\@firstofone
%    \end{macrocode}
% \end{macro}
%
% \begin{macro}{\percent}
% Defines an expandable percent character with catcode 12: `\percent$_{12}$'.
%    \begin{macrocode}
\begingroup
\catcode`\%=12
\gdef\percent{%}
\endgroup
%    \end{macrocode}
% \end{macro}
%
% \begin{macro}{\braceleft}
% \begin{macro}{\braceright}
% Defines expandable left and right braces with catcode 12: `\texttt{\braceleft}$_{12}$' `\texttt{\braceright}$_{12}$'.
%    \begin{macrocode}
\begingroup
\catcode`\<=1
\catcode`\>=2
\catcode`\{=12
\catcode`\}=12
\gdef\braceleft <{>
\gdef\braceright<}>
\endgroup
%    \end{macrocode}
% \end{macro}
% \end{macro}
%
% \subsubsection{Other Macros}
%
% \begin{macro}{\y@bgroup}
% \begin{macro}{\y@egroup}
% These macros are used to begin and end |\vbox|/|\hbox|-es.
%    \begin{macrocode}
\def\y@bgroup{\bgroup\color@setgroup}
\def\y@egroup{\color@endgroup\egroup}
%    \end{macrocode}
% \end{macro}
% \end{macro}
%
% \begin{macro}{\codeline}
%    \begin{macrocode}
\newcommand*{\codeline}[1][c]{%
    \codelinebefore
    \hbox to \hsize\bgroup
    \ifx i#1\hspace*{\leftmargin}\else
        \ifx l#1\else\hss\fi
    \fi
    \let\xspace\relax
    \hbox\bgroup
    \aftergroup\codeline@end
    \aftergroup#1%
    \afterassignment\MacroArgs
    \let\@let@token=%
}
\def\codeline@end#1{%
    \ifx r#1\else\hss\fi
    \egroup
    \codelineafter
}
\newcommand*\codelinebefore{\par\smallskip\noindent}
\newcommand*\codelineafter {\par\smallskip\noindent}
%    \end{macrocode}
% \end{macro}
%
% \begin{environment}{codequote}
%    \begin{macrocode}
\newenvironment{codequote}{%
    \def\\{\newline\relax\MacroArgs}%
    \par\smallskip\bgroup\leftskip=\leftmargin\rightskip=\rightmargin\noindent\MacroArgs}
    {\par\egroup\smallskip\noindent\ignorespacesafterend}
%    \end{macrocode}
% \end{environment}
%
% \begin{environment}{macroquote}
%    \begin{macrocode}
\newenvironment{macroquote}{%
    \def\\{\newline\relax\Macro}%
    \par\smallskip\bgroup\leftskip=\leftmargin\rightskip=\rightmargin\noindent\Macro}
    {\par\egroup\smallskip\noindent\ignorespacesafterend}
%    \end{macrocode}
% \end{environment}
% \iffalse
%</ydoc-desc.sty>
% \fi
%
% \iffalse
%<*ydoc-expl.sty>
% \fi
% \subsection{Include Code Examples}
%
%    \begin{macrocode}
\NeedsTeXFormat{LaTeX2e}[1999/12/01]
\ProvidesPackage{ydoc-expl}[%
%<!DATE>
%<!VERSION>
%<*DRIVER>
    2011/08/11 develop
%</DRIVER>
    ydoc package to insert live examples of LaTeX code]
%    \end{macrocode}
%
% \iffalse
%<*dontinclude>
%    \begin{macrocode}
\newcounter{example}
\def\examplename{Example}
\newenvironment{example}[1]{%
  \par\bigskip\noindent
  \centerline{\examplename~\refstepcounter{example}\theexample: #1}%
  \smallskip
  \hrule
  \smallskip
}{%
  \smallskip
  \hrule
  \par\bigskip
}
%    \end{macrocode}
%
% \begin{macro}{\PrintExample}
%    \begin{macrocode}
\def\PrintExample{%
  \usebox\examplecodebox
  \hfill
  \usebox\exampleresultbox
}
%    \end{macrocode}
% \end{macro}
%</dontinclude>
% \fi
%
%    \begin{macrocode}
\RequirePackage{listings}
\lst@RequireAspects{writefile}
\def\ydoc@exafile{\jobname.exa}
%    \end{macrocode}
%
% \begin{style}{examplecode}
%    \begin{macrocode}
\lstdefinestyle{examplecode}{%
    language=[latex]tex,
    basicstyle=\ttfamily,
    columns=fullflexible,
    numbers=left,
    firstnumber=1,
    numberstyle=\tiny\color{gray}\sffamily,
    numbersep=5pt,
    breaklines,prebreak={\mbox{\tiny$\swarrow$}},
    commentstyle=\color{black!60},
}%
%    \end{macrocode}
% \end{style}
%
% \begin{style}{exampleresult}
%    \begin{macrocode}
\lstdefinestyle{exampleresult}{%
    firstnumber=1,
    gobble=0,
    basicstyle=\ttfamily,
    columns=fullflexible,
    commentstyle=\color{black!60},
}
%    \end{macrocode}
% \end{style}
%
% \begin{style}{exampleextract}
%    \begin{macrocode}
\lstdefinestyle{exampleextract}{gobble=4}%
%    \end{macrocode}
% \end{style}
%
%    \begin{macrocode}
\newbox\examplecodebox
\newbox\exampleresultbox
%    \end{macrocode}
%
% \begin{macro}{\BoxExample}
%    \begin{macrocode}
\def\BoxExample{%
  \setbox\examplecodebox\hbox{\color@setgroup
    \lstinputlisting[style=examplecode,style=thisexampleprint]%
    {\ydoc@exafile}%
  \unskip\color@endgroup}%
  \setbox\exampleresultbox\hbox{\color@setgroup
    \lstset{style=exampleresult}%
    \@@input\ydoc@exafile\relax
  \unskip\color@endgroup}%
}
%    \end{macrocode}
% \end{macro}
%
% \begin{macro}{\PrintExample}
%    \begin{macrocode}
%<*DISABLED>
\RequirePackage{showexpl}
\def\PrintExample{%
  \begingroup
  \lstset{style=examplecode}%
  \MakePercentComment
  \LTXinputExample[varwidth]{\ydoc@exafile}%
  \endgroup
}
%</DISABLED>
%    \end{macrocode}
% \end{macro}
%
%
%
% \begin{macro}{\PrintExample}
%    \begin{macrocode}
\def\PrintExample{%
  \begingroup
  \BoxExample
  \@tempdima=\textwidth
  \advance\@tempdima by -\wd\examplecodebox\relax
  \advance\@tempdima by -\wd\exampleresultbox\relax
  \advance\@tempdima by -15pt\relax
  \ifdim\@tempdima>\bigskipamount
    \hbox to \textwidth{%
     \null\hss
     \minipage[c]{\wd\exampleresultbox}\fbox{\usebox\exampleresultbox}\endminipage
     \hfill\hfill\hskip\bigskipamount\hskip15pt\hfill\hfill
     \minipage[c]{\wd\examplecodebox}\usebox\examplecodebox\endminipage
     \hss\null
     }%
  \else
    \vbox{%
        \centerline{\fbox{\usebox\exampleresultbox}}%
        \vspace{\bigskipamount}%
        \centerline{\usebox\examplecodebox}%
    }%
  \fi
  \endgroup
}
%    \end{macrocode}
% \end{macro}
%
% \begin{environment}{examplecode}
%    \begin{macrocode}
\lstnewenvironment{examplecode}[1][]{%
  \lstdefinestyle{thisexampleprint}{#1}%
  \setbox\@tempboxa\hbox\bgroup
  \lstset{style=exampleextract,#1}%
  \lst@BeginWriteFile{\ydoc@exafile}%
}
{%
  \lst@EndWriteFile
  \egroup
  \begingroup
  \MakePercentComment
  \catcode`\^^M=5\relax
  \PrintExample
  \endgroup
}
%    \end{macrocode}
% \end{environment}
%
%    \begin{macrocode}
\RequirePackage{float}
%    \end{macrocode}
%
% \begin{environment}{example}
%    \begin{macrocode}
\floatstyle{plain}
\newfloat{example}{tbhp}{loe}
\floatname{example}{\examplename}
\def\examplename{Example}
%    \end{macrocode}
% \end{environment}
%
% \begin{environment}{exampletable}
%    \begin{macrocode}
\newenvironment{exampletable}{%
  \floatstyle{plaintop}%
  \restylefloat{example}%
  \example
}{\endexample}
%    \end{macrocode}
% \end{environment}
% \iffalse
%</ydoc-expl.sty>
% \fi
%
% \iffalse
%<*ydocincl.tex>
% \fi
%
%    \begin{macrocode}
\expandafter\ifx\csname ydocinclversion\endcsname\relax\else
    \endinput
\fi

\chardef\ydocinclversion=1

\newread\inFile
\newread\subFile
\newwrite\outFile
\newif\ifContinue
\newlinechar=`^^J

\def\makeOther#1{\catcode`#1=12\relax}

\let\inLine\relax
\let\lastLine\relax

\def\includefiles#1#2{%
    \begingroup
    \immediate\openin\inFile#1\relax
    \immediate\openout\outFile#2\relax
    \makeOther\@%
    \makeOther\ \makeOther\\\makeOther\$%
    \makeOther\#\makeOther\^\makeOther\^^K%
    \makeOther\_\makeOther\^^A\makeOther\%%
    \makeOther\~\makeOther\{\makeOther\}\makeOther\&%
    \endlinechar-1\relax
    \Continuetrue
    \loop
      \let\lastLine\inLine
      \read\inFile to\inLine
      \ifeof\inFile
        \Continuefalse
      \else
        \expandafter\checkLine\inLine\empty\empty\empty\endLine
      \fi
      \ifContinue
    \repeat
    \immediate\closein\inFile
    \immediate\closeout\outFile
    \endgroup
    \end
}

\def\copyline{%
    \immediate\write\outFile{\inLine}%
}

\chardef\percentcharnum=`\%

\begingroup
\makeOther\%\makeOther\@\relax
\gdef\SubFileOptionString\relax
\catcode`\|=0
\makeOther\ \makeOther\\|relax
|gdef|IfFalseString{% \iffalse}|relax
|gdef|FiString{% \fi}|relax
|endgroup

\def\checkLine#1#2#3#4\endLine{%
    \def\firstthree{#1#2#3}%
    \ifx\firstthree\SubFileOptionString
        \readSubFile#4\endLine
    \else
        \copyline
    \fi
}

\def\readSubFile#1>#2\endLine{%
    \immediate\openin\subFile=#1\relax
    \ifeof\subFile
        % File not found
    \else
        \message{^^JIncluding subfile '#1'^^J}%
        \immediate\write\outFile{\CommentChar<*#1>}%
        \ifx\lastLine\IfFalseString
            \immediate\write\outFile{\FiString}%
        \fi
        \copySubFile
        \ifx\lastLine\IfFalseString
            \immediate\write\outFile{\IfFalseString}%
        \fi
        \immediate\write\outFile{\CommentChar</#1>}%
    \fi
    \immediate\closein\subFile
}

\def\copySubFile{%
    \read\subFile to\subLine
    \ifeof\subFile\else
        \immediate\write\outFile{\subLine}%
        \expandafter\copySubFile
    \fi
}
%    \end{macrocode}
%
% \iffalse
%</ydocincl.tex>
% \fi
%
% \Finale
% \endinput

%    \end{macrocode}
% \iffalse
%</ydoc.sty>
% \fi
%
% \iffalse
%<*ydoc.cfg>
% \fi
% \subsection{Config File}
%    \begin{macrocode}
%% Please delete the following line on manual changes:
\ProvidesFile{ydoc.cfg}[%
%<!DATE>
%<!VERSION>
%<*DRIVER>
   2011/08/11 develop
%</DRIVER>
   Default config file for ydoc]
%    \end{macrocode}
%
%    \begin{macrocode}
\usepackage[T1]{fontenc}
\IfFileExists{fourier.sty}{%
    \usepackage{fourier}
}{}
%    \end{macrocode}
%
% Use 'lmodern' only for the 'tt' font if fourier is installed.
%    \begin{macrocode}
\IfFileExists{lmodern.sty}{
    \IfFileExists{fourier.sty}{
        \renewcommand{\ttdefault}{lmtt}
    }{
        \usepackage{lmodern}
    }
}{}
%    \end{macrocode}
%
%
%    \begin{macrocode}
\urlstyle{sf}
%    \end{macrocode}
%
% Use micro-typesetting if pdftex is used:
%    \begin{macrocode}
\usepackage{ifpdf}
\ifpdf
\usepackage{microtype}
\fi
%    \end{macrocode}
%    \begin{macrocode}
\usepackage{array}
\usepackage{booktabs}
\usepackage{multicol}
\usepackage{xcolor}
\usepackage{listings}
\usepackage{booktabs}
\usepackage{hyperref}
%    \end{macrocode}
%
%    \begin{macrocode}
\reversemarginpar
%    \end{macrocode}
%
% \iffalse
%</ydoc.cfg>
% \fi
%
% \iffalse
%<*ydoc-code.sty>
% \fi
% \subsection{Macros and Environments to document Implementations}
%
%    \begin{macrocode}
\NeedsTeXFormat{LaTeX2e}[1999/12/01]
\ProvidesPackage{ydoc-code}[%
%<!DATE>
%<!VERSION>
%<*DRIVER>
    2011/08/11 develop
%</DRIVER>
    ydoc package to document macro code]
%    \end{macrocode}
%
%    \begin{macrocode}
\RequirePackage{hyperref}
\hypersetup{colorlinks=true,pdfborder=0 0 0,pdfborderstyle={}}
%    \end{macrocode}
%
%    \begin{macrocode}
\IfFileExists{needspace.sty}{%
    \RequirePackage{needspace}
}{%
    \def\Needspace{\@ifstar\@gobble\@gobble}
}
%    \end{macrocode}
%
% \subsubsection{Color and style definitions}
%    \begin{macrocode}
\RequirePackage{xcolor}
\definecolor{macroimpl}{rgb}{0.0,0.0,0.4}
%    \end{macrocode}
%
% \subsubsection{General Macros}
%
% \begin{macro}{\ydocwrite}
%    \begin{macrocode}
\@ifundefined{ydocwrite}{%
  \newwrite\ydocwrite
}{}
%    \end{macrocode}
% \end{macro}
%
% \begin{macro}{\ydocfname}
%    \begin{macrocode}
\@ifundefined{ydocfname}{%
  \def\ydocfname{\jobname.cod}%
}{}
%    \end{macrocode}
% \end{macro}
%
% \begin{macro}{\ydoc@catcodes}
%    \begin{macrocode}
\def\ydoc@catcodes{%
  \let\do\@makeother
  \dospecials
  \catcode`\\=\active
  \catcode`\^^M=\active
  \catcode`\ =\active
}
%    \end{macrocode}
% \end{macro}
%
% \subsubsection{Handling Macrocode}
% \begin{environment}{macrocode}
%    \begin{macrocode}
\def\macrocode{%
  \par\noindent
  \begingroup
  \ydoc@catcodes
  \macro@code
}
\def\endmacrocode{}
%    \end{macrocode}
% \end{environment}
%
% \begin{macro}{\macro@code}[1]{verbatim macro code}
%    \begin{macrocode}
\begingroup
\endlinechar\m@ne
\@firstofone{%
\catcode`\|=0\relax
\catcode`\(=1\relax
\catcode`\)=2\relax
\catcode`\*=14\relax
\catcode`\{=12\relax
\catcode`\}=12\relax
\catcode`\ =12\relax
\catcode`\%=12\relax
\catcode`\\=\active
\catcode`\^^M=\active
\catcode`\ =\active
}*
|gdef|macro@code#1^^M%    \end{macrocode}(*
|endgroup|expandafter|macro@@code|expandafter(|ydoc@removeline#1|noexpand|lastlinemacro)*
)*
|gdef|ydoc@removeline#1^^M(|noexpand|firstlinemacro)*
|gdef|ydoc@defspecialmacros(*
|def^^M(|noexpand|newlinemacro)*
|def (|noexpand|spacemacro)*
|def\(|noexpand|bslashmacro)*
)*
|gdef|ydoc@defrevspecialmacros(*
|def|newlinemacro(|noexpand^^M)*
|def|spacemacro(|noexpand )*
|def|bslashmacro(|noexpand\)*
)*
|endgroup
%    \end{macrocode}
% \end{macro}
%
% \begin{macro}{\macro@@code}[1]{verbatim macro code}
%    \begin{macrocode}
\def\macro@@code#1{%
  {\ydoc@defspecialmacros
  \xdef\themacrocode{#1}}%
  \PrintMacroCode
  \end{macrocode}%
}
%    \end{macrocode}
% \end{macro}
%
% \begin{macro}{\linenumberbox}
%    \begin{macrocode}
\def\newlinemacro{\\\null}
\def\spacemacro{\ }
\def\bslashmacro{\char92}
\def\lastlinemacro{}
\def\firstlinemacro{\linenumberbox}
\def\newlinemacro{\\\linenumberbox}
\newcounter{linenumber}
\def\linenumberbox{%
  \hbox to 1.25em{}%
  \llap{%
    \stepcounter{linenumber}%
    {\footnotesize\color{gray}\thelinenumber~}%
  }%
}
%    \end{macrocode}
% \end{macro}
%
% \begin{macro}{\PrintMacroCode}
%    \begin{macrocode}
\def\PrintMacroCode{%
  \begingroup
  \ttfamily
  \noindent\themacrocode
  \endgroup
}
%    \end{macrocode}
% \end{macro}
%
% \begin{macro}{\PrintMacroCode}
%    \begin{macrocode}
\RequirePackage{listings}
%    \end{macrocode}
%    \begin{macrocode}
\def\PrintMacroCode{%
  \begingroup
  \let\firstlinemacro\empty
  \let\lastlinemacro\empty
  \def\newlinemacro{^^J}%
  \let\bslashmacro\bslash
  \let\spacemacro\space
  \immediate\openout\ydocwrite=\ydocfname\relax
  \immediate\write\ydocwrite{\themacrocode}%
  \immediate\closeout\ydocwrite
  \@nameuse{ydoc@countbslashes}%
  \ydoclistingssettings
  \let\input\@input
  \lstinputlisting{\ydocfname}%
  \endgroup
}
%    \end{macrocode}
% \end{macro}
%
%    \begin{macrocode}
\lstdefinestyle{ydoccode}{%
    language=[latex]tex,basicstyle=\ttfamily,
    numbers=left,numberstyle=\tiny\color{gray},firstnumber=last,
    breaklines,prebreak={\mbox{\tiny$\swarrow$}},
    commentstyle=\color{black!60},
}%
%    \end{macrocode}
%
% \begin{macro}{\ydoclistingssettings}
%    \begin{macrocode}
\def\ydoclistingssettings{%
    \lstset{style=ydoccode}%
}
%    \end{macrocode}
% \end{macro}
%
% \begin{macro}{\macro@impl@args}[1]{number of macro arguments}
%    \begin{macrocode}
\def\macro@impl@args[#1]{%
  \begingroup
  \parindent=10pt\relax
  \let\macro@impl@argcnt\@tempcnta
  \let\macro@impl@curarg\@tempcntb
  \macro@impl@argcnt=#1\relax
  \macro@impl@curarg=0\relax
  \ifnum\macro@impl@curarg<\macro@impl@argcnt\relax
    \expandafter\macro@impl@arg
  \else
    \expandafter\macro@impl@endargs
  \fi
}
%    \end{macrocode}
% \end{macro}
%
% \begin{macro}{\macro@impl@endargs}
%    \begin{macrocode}
\def\macro@impl@endargs{%
  \endgroup
  \unskip\par\noindent\ignorespaces
}
%    \end{macrocode}
% \end{macro}
%
% \begin{macro}{\macro@impl@argline}[2]{argument number}{argument description}
%    \begin{macrocode}
\def\macro@impl@argline#1#2{%
  \par{\texttt{\##1}:~#2\strut}%
}
%    \end{macrocode}
% \end{macro}
%
% \begin{macro}{\macro@impl@arg}[1]{argument description}
%    \begin{macrocode}
\def\macro@impl@arg#1{%
  \advance\macro@impl@curarg by\@ne\relax
  \macro@impl@argline{\the\macro@impl@curarg}{#1}%
  \ifnum\macro@impl@curarg<\macro@impl@argcnt\relax
    \expandafter\macro@impl@arg
  \else
    \expandafter\macro@impl@endargs
  \fi
}
%    \end{macrocode}
% \end{macro}
%
% \begin{environment}{macro}[1]{implemented macro}
%    \begin{macrocode}
\def\macro#1{%
  \PrintMacroImpl{#1}%
  \@ifnextchar[%]
    {\macro@impl@args}%
    {}%
}
\def\endmacro{}
%    \end{macrocode}
% \end{environment}
%
% \begin{environment}{key}[2]{key family}{key name}
%    \begin{macrocode}
\def\key#1#2{%
  \PrintMacroImpl{KV@#1@#2}%
  \@ifnextchar[%]
    {\macro@impl@args}%
    {}%
}
\def\endkey{}
%    \end{macrocode}
% \end{environment}
%
% \begin{environment}{environment}[1]{environment name}
%    \begin{macrocode}
\def\environment#1{%
  \PrintEnvImplName{#1}%
  \@ifnextchar[%]
    {\macro@impl@args}%
    {}%
}
\def\endenvironment{}
%    \end{macrocode}
% \end{environment}
%
%
% \begin{environment}{style}[1]{style name}
%    \begin{macrocode}
\def\style#1{%
  \PrintStyleImplName{#1}%
  \@ifnextchar[%]
    {\macro@impl@args}%
    {}%
}
\def\endstyle{}
\def\PrintStyleImplName{\PrintEnvImplName}
%    \end{macrocode}
% \end{environment}
%
% \begin{macro}{\PrintMacroImpl}[1]{macro (token)}
%    \begin{macrocode}
\def\PrintMacroImpl#1{%
  \par\bigskip\noindent
  \Needspace*{3\baselineskip}%
  \hbox{%
    \edef\name{\expandafter\@gobble\string#1}%
    \global\@namedef{href@impl@\name}{}%
    \immediate\write\@mainaux{%
      \global\noexpand\@namedef{href@impl@\name}{}%
    }%
    \raisebox{4ex}[4ex]{\hypertarget{impl:\name}{}}%
    \hspace*{\descindent}\fbox{%
      \hspace*{\descsep}%
      \@ifundefined{href@desc@\name}{}{\hyperlink{desc:\name}}%
      {\PrintMacroImplName{#1}}%
      \hspace*{\descsep}%
    }%
  }%
  \par\medskip\noindent
}
%    \end{macrocode}
% \end{macro}
%
% \begin{macro}{\PrintMacroImplName}[1]{macro (token)}
%    \begin{macrocode}
\def\PrintMacroImplName#1{%
  \implstyle{\string#1\strut}%
}
%    \end{macrocode}
% \end{macro}
%
% \begin{macro}{\PrintEnvImplName}[1]{environment name}
% test
%    \begin{macrocode}
\def\PrintEnvImplName#1{%
  \par\bigskip\noindent
  \hbox{\hspace*{\descindent}\fbox{{\implstyle{#1}}}}%
  \par\medskip
}
%    \end{macrocode}
% \end{macro}
%
% \begin{macro}{\implstyle}
%    \begin{macrocode}
\def\implstyle{\ttfamily\bfseries\color{macroimpl}}
%    \end{macrocode}
% \end{macro}
%
% \begin{macro}{\bslash}
% Defines an expandable backslash with catcode 12: `\texttt{\bslash}$_{12}$'.
% The |\@firstofone| trick is used to read the |\gdef\bslash| code before changing the catcode.
%    \begin{macrocode}
{%
\@firstofone
%    \end{macrocode}
% \end{macro}
% \iffalse
%</ydoc-code.sty>
% \fi
%
% \iffalse
%<*ydoc-doc.sty>
% \fi
% \subsection{Provide \pkg{doc} macros}
%    \begin{macrocode}
\NeedsTeXFormat{LaTeX2e}[1999/12/01]
\ProvidesPackage{ydoc-doc}[%
%<!DATE>
%<!VERSION>
%<*DRIVER>
    2099/01/01 develop
%</DRIVER>
    ydoc package to provide 'doc' macros]
%    \end{macrocode}
%
% \begin{macro}{\ydoc@countbslashes}
% Reads the macro code into a temp box. The backslashes are defined to increase a counter.
%    \begin{macrocode}
\newcount\ydoc@bslashcnt
\def\ydoc@countbslashes{%
  \begingroup
    \let\firstlinemacro\empty
    \let\lastlinemacro\empty
    \let\newlinemacro\empty
    \let\spacemacro\empty
    \def\bslashmacro{\global\advance\ydoc@bslashcnt by\@ne}%
    \setbox\@tempboxa\hbox{\themacrocode}%
  \endgroup
}
%    \end{macrocode}
% \end{macro}
%
% \begin{macro}{\CheckSum}
%    \begin{macrocode}
\def\CheckSum#1{%
  \gdef\ydoc@checksum{#1}%
}
\let\ydoc@checksum\m@ne
%    \end{macrocode}
% \end{macro}
%
% \begin{macro}{\AlsoImplementation}
% \begin{macro}{\OnlyDescription}
% \begin{macro}{\StopEventually}
% \begin{macro}{\Finale}
% The first two macros modify the \cs{StopEventually} macro which either
% stores its argument in \cs{Final} or executes it itself.
%    \begin{macrocode}
\def\AlsoImplementation{%
  \gdef\StopEventually##1{%
    \@bsphack
    \gdef\Finale{##1\ydoc@checkchecksum}%
    \@esphack
  }%
}
\AlsoImplementation
\def\OnlyDescription{%
  \@bsphack
  \long\gdef\StopEventually##1{##1\endinput}%
  \@esphack
}
\let\Finale\relax
%    \end{macrocode}
% \end{macro}
% \end{macro}
% \end{macro}
% \end{macro}
%
% \begin{macro}{\MakePercentComment}
% \begin{macro}{\MakePercentIgnore}
%    \begin{macrocode}
\def\MakePercentIgnore{\catcode`\%9\relax}
\def\MakePercentComment{\catcode`\%14\relax}
%    \end{macrocode}
% \end{macro}
% \end{macro}
%
% \begin{macro}{\DocInput}
%    \begin{macrocode}
\def\DocInput#1{\MakePercentIgnore\input{#1}\MakePercentComment}
%    \end{macrocode}
% \end{macro}
%
% \begin{macro}{\CharacterTable}
%    \begin{macrocode}
\providecommand*\CharacterTable{%
    \begingroup
    \CharTableChanges
    \@CharacterTable
}
\def\@CharacterTable#1{%
    \def\ydoc@used@CharacterTable{#1}%
    \@onelevel@sanitize\ydoc@used@CharacterTable
    \ifx\ydoc@used@CharacterTable\ydoc@correct@CharacterTable
           \typeout{***************************}%
           \typeout{* Character table correct *}%
           \typeout{***************************}%
    \else
         \PackageError{ydoc}{Character table corrupted}
                           {\the\wrong@table}
         \show\ydoc@used@CharacterTable
         \show\ydoc@correct@CharacterTable
    \fi
    \endgroup
}
\newhelp\wrong@table{Some of the ASCII characters are corrupted.^^J
            I now \string\show\space you both tables for comparison.}
\newcommand*\CharTableChanges{}
%    \end{macrocode}
% \end{macro}
%
% \begin{macro}{\ydoc@correct@CharacterTable}
%    \begin{macrocode}
\def\ydoc@correct@CharacterTable
  {Upper-case    \A\B\C\D\E\F\G\H\I\J\K\L\M\N\O\P\Q\R\S\T\U\V\W\X\Y\Z
   Lower-case    \a\b\c\d\e\f\g\h\i\j\k\l\m\n\o\p\q\r\s\t\u\v\w\x\y\z
   Digits        \0\1\2\3\4\5\6\7\8\9
   Exclamation   \!     Double quote  \"     Hash (number) \#
   Dollar        \$     Percent       \%     Ampersand     \&
   Acute accent  \'     Left paren    \(     Right paren   \)
   Asterisk      \*     Plus          \+     Comma         \,
   Minus         \-     Point         \.     Solidus       \/
   Colon         \:     Semicolon     \;     Less than     \<
   Equals        \=     Greater than  \>     Question mark \?
   Commercial at \@     Left bracket  \[     Backslash     \\
   Right bracket \]     Circumflex    \^     Underscore    \_
   Grave accent  \`     Left brace    \{     Vertical bar  \|
   Right brace   \}     Tilde         \~}
\@onelevel@sanitize\ydoc@correct@CharacterTable
%
%    \end{macrocode}
% \end{macro}
%
% \begin{macro}{\DoNotIndex}
%    \begin{macrocode}
\providecommand*\DoNotIndex[1]{%
  \PackageWarning{ydoc}{Ignoring DoNotIndex - not implemented yet!}{}{}%
}
%    \end{macrocode}
% \end{macro}
%
% \begin{macro}{\changes}
%    \begin{macrocode}
\providecommand*\changes[3]{%
  \PackageWarning{ydoc}{Ignoring changes - not implemented yet!}{}{}%
}
%    \end{macrocode}
% \end{macro}
%
% \begin{macro}{\RecordChanges}
%    \begin{macrocode}
\providecommand*\RecordChanges{%
  \PackageWarning{ydoc}{List of changes not implemented yet!}{}{}%
}
%    \end{macrocode}
% \end{macro}
%
% \begin{macro}{\PrintChanges}
%    \begin{macrocode}
\providecommand*\PrintChanges{%
  \PackageWarning{ydoc}{List of changes not implemented yet!}{}{}%
}
%    \end{macrocode}
% \end{macro}
%
% \begin{macro}{\PrintIndex}
%    \begin{macrocode}
\providecommand*\PrintIndex{%
  \PackageWarning{ydoc}{Code index not implemented yet!}{}{}%
}
%    \end{macrocode}
% \end{macro}
%
% \begin{macro}{\CodelineIndex}
%    \begin{macrocode}
\providecommand*\CodelineIndex{%
  \PackageWarning{ydoc}{Code line index not implemented yet!}{}{}%
}
%    \end{macrocode}
% \end{macro}
%
% \begin{macro}{\EnableCrossrefs}
%    \begin{macrocode}
\providecommand*\EnableCrossrefs{%
  \PackageWarning{ydoc}{Cross references not implemented yet!}{}{}%
}
%    \end{macrocode}
% \end{macro}
%
% \begin{macro}{\GetFileInfo}
% Current implementation taken from \pkg{doc} package.
%    \begin{macrocode}
\providecommand*\GetFileInfo[1]{%
  \def\filename{#1}%
  \def\@tempb##1 ##2 ##3\relax##4\relax{%
    \def\filedate{##1}%
    \def\fileversion{##2}%
    \def\fileinfo{##3}}%
  \edef\@tempa{\csname ver@#1\endcsname}%
  \expandafter\@tempb\@tempa\relax? ? \relax\relax
}
%    \end{macrocode}
% \end{macro}
%
% \begin{macro}{\ydoc@checkchecksum}
%    \begin{macrocode}
\def\ydoc@checkchecksum{%
  \ifnum\ydoc@checksum=\m@ne
    \message{^^J}%
    \message{**********************^^J}%
    \message{* No checksum found! *^^J}%
    \message{**********************^^J}%
    \GenericWarning{No checksum found}{Correct checksum is \the\ydoc@bslashcnt^^J}{}{}%
  \else
  \ifnum\ydoc@checksum=\z@
    \message{^^J}%
    \message{*********************^^J}%
    \message{* Checksum disabled *^^J}%
    \message{*********************^^J}%
    \GenericWarning{Checksum disabled}{Correct checksum is \the\ydoc@bslashcnt^^J}{}{}%
  \else
  \ifnum\ydoc@checksum=\ydoc@bslashcnt
    \message{^^J}%
    \message{*******************^^J}%
    \message{* Checksum passed *^^J}%
    \message{*******************^^J}%
  \else
    \message{^^J}%
    \message{****************************^^J}%
    \message{* Checksum wrong (\ydoc@checksum<>\the\ydoc@bslashcnt) ^^J}%
    \message{****************************^^J}%
    \GenericError{Checksum wrong}{Correct checksum is \the\ydoc@bslashcnt^^J}{}{}%
  \fi
  \fi
  \fi
}
%    \end{macrocode}
% \end{macro}
%
%    \begin{macrocode}
\RequirePackage{shortvrb}
\AtBeginDocument{\MakeShortVerb{\|}}
%    \end{macrocode}
%
%    \begin{macrocode}
\RequirePackage{url}

\def\package{\def\@package}
\package{\jobname}

\def\bundle{\def\@bundle}
\let\@bundle\@empty


\def\ctanlocation{\def\@ctanlocation##1}
\ctanlocation{http://www.ctan.org/pkg/#1}

\date{Version \fileversion\space -- \filedate}

\def\@homepage{%
    \begingroup
    \edef\@tempa{%
        \endgroup
        CTAN:
        \noexpand\url
        {\@ctanlocation{\ifx\@bundle\@empty\@package\else\@bundle\fi}}%
    }%
    \@tempa
}

\let\@repository\@empty
\protected\def\repository{\urldef\@repository\url}
\protected\def\homepage{\urldef\@homepage\url}
\protected\def\email{\hyper@normalise\email@}
\def\email@#1{\def\@plainemail{#1}\def\@email{\hyper@linkurl{\Hurl{#1}}{mailto:#1}}}

\let\@email\empty

\let\@plainemail\empty
\title{The \texorpdfstring{\pkgtitle{\@package}}{\@package} Package}
\def\@bundlesubtitle{Part of the \texorpdfstring{\pkgtitle{\@bundle}}{\@bundle} bundle}

\protected\def\pkgtitle#1{%
    \texorpdfstring{\textsf{#1}}{#1}%
}



\def\@maketitle{%
    \newpage
    \null\vskip 2em
    \begin{center}%
        \let\footnote\thanks
        {\LARGE \@title \par }\vskip 1.5em%
        \ifx\@bundle\@empty\else
        {\large \@bundlesubtitle \par }\vskip 1.5em%
        \fi
        {\large \lineskip .5em%
        \begin{tabular}[t]{c}%
            \@author
        \end{tabular}%
        \par}%
        \ifx\@plainemail\empty\else
            {\large \lineskip .5em%
            \begin{tabular}[t]{c}%
                \@email
            \end{tabular}%
            \par}%
        \fi
        \vskip 1em
        {\large \lineskip .5em%
        \begin{tabular}[t]{c}%
            \@homepage
        \end{tabular}%
        \par}%
        \vskip 1em
        \ifx\@repository\@empty\else
            {\large \lineskip .5em%
            \begin{tabular}[t]{c}%
                VC: \@repository
            \end{tabular}%
            \par}%
        \fi
        \vskip 1em
        {\large \@date }%
    \end{center}%
    \par\vskip 1.5em
    \aftergroup\ydocpdfsettings
}

\ifpdf
\def\ydocpdfsettings{%
    \hypersetup{%
        pdfauthor   = {\@author\space<\@plainemail>},
        pdftitle    = {\@title},
        pdfsubject  = {Documentation of LaTeX package \@package},
        pdfkeywords = {\@package, LaTeX, TeX}
    }%
}
\else
\let\ydocpdfsettings\empty
\fi

\let\orig@maketitle\maketitle
\def\maketitle{%
    \ydocpdfsettings
    \orig@maketitle
    \let\orig@maketitle\relax
}
%    \end{macrocode}
% \iffalse
%</ydoc-doc.sty>
% \fi
%
% \iffalse
%<*ydoc-desc.sty>
% \fi
% \subsection{Description Macros and Environments}
%
%    \begin{macrocode}
\NeedsTeXFormat{LaTeX2e}[1999/12/01]
\ProvidesPackage{ydoc-desc}[%
%<!DATE>
%<!VERSION>
%<*DRIVER>
    2099/01/01 develop
%</DRIVER>
    ydoc package to describe macros, environments, options etc.]
%    \end{macrocode}
%
%    \begin{macrocode}
\IfFileExists{needspace.sty}{%
    \RequirePackage{needspace}
}{%
    \def\Needspace{\@ifstar\@gobble\@gobble}
}
%    \end{macrocode}
%
% The short verbatim code is required for the similar macros provided here.
%    \begin{macrocode}
\RequirePackage{shortvrb}
%    \end{macrocode}
%
% The \pkg{etoolbox} package is used mainly for |\newrobustcmd|.
%    \begin{macrocode}
\RequirePackage{etoolbox}
%    \end{macrocode}
%
% \subsubsection{Color and style definitions}
%    \begin{macrocode}
\RequirePackage{xcolor}
%    \end{macrocode}
% Define special no-op `none' color which does not change the color.
% This is not yet tested and may break output files, but seems to work fine with PDF.
%    \begin{macrocode}
\expandafter\def\csname\string\color@none\endcsname{%
    \xcolor@ {}{}{}{}
}
%    \end{macrocode}
%    \begin{macrocode}
\definecolor{macrodesc}{rgb}{0,0.2,0.6}
\definecolor{keydesc}{rgb}{0,0.4,0.9}
\definecolor{macroimpl}{rgb}{0,0.1,0.3}
\definecolor{meta}{rgb}{0,0.25,0.75}
\definecolor{scriptcolor}{rgb}{0.2,0.6,0.2}
\definecolor{optioncolor}{rgb}{0.3.0.2,0}
\colorlet{optional}{black!65!white}
\colorlet{metaoptional}{optional!50!meta}
\providecolor{urlcolor}{named}{blue}
\providecolor{linkcolor}{named}{blue}
\providecolor{filecolor}{named}{blue}
\providecolor{citecolor}{named}{blue}
\providecolor{anchorcolor}{named}{blue}
\providecolor{menucolor}{named}{blue}
\providecolor{runcolor}{named}{blue}

%    \end{macrocode}
%
%    \begin{macrocode}
\RequirePackage{hyperref}
\hypersetup{%
    colorlinks=true,
    pdfborder=0 0 0,
    pdfborderstyle={},
    urlcolor=urlcolor,
    linkcolor=linkcolor,
    filecolor=filecolor,
    citecolor=citecolor,
    anchorcolor=anchorcolor,
    menucolor=menucolor,
    runcolor=runcolor,
}

%    \end{macrocode}
%
% \subsubsection{Text Formatting Macros}
%
% \begin{macro}{\meta}
% Prints \meta{meta text}.
%    \begin{macrocode}
\newrobustcmd*\meta[1]{%
  {\metastyle{%
  \ensuremath\langle
    #1\/%
  \ensuremath\rangle
  }}%
}
%    \end{macrocode}
% \end{macro}
%
% \begin{macro}{\marg}
% Sets style and adds braces.
% The text is formatted as separate set of macro arguments.
%    \begin{macrocode}
\newrobustcmd*{\marg}[1]{%
  {\margstyle{%
    {\ttfamily\braceleft}%
    \meta{#1}%
    {\ttfamily\braceright}%
  }}%
}
%    \end{macrocode}
% \end{macro}
%
% \begin{macro}{\oarg}
% Sets style and adds brackets.
% The text is formatted as separate set of macro arguments.
%    \begin{macrocode}
\newrobustcmd*{\oarg}[1]{%
  {\oargstyle{%
    {\ttfamily[}%
    \meta{#1}%
    {\ttfamily]}%
  }}%
}
%    \end{macrocode}
% \end{macro}
%
% \begin{macro}{\parg}
% Sets style and adds parentheses.
%    \begin{macrocode}
\newrobustcmd*{\parg}[1]{%
  {\pargstyle{%
    {\ttfamily(}%
    \meta{#1}%
    {\ttfamily)}%
  }}%
}
%    \end{macrocode}
% \end{macro}
%
% \begin{macro}{\aarg}
% Sets style and adds angles.
%    \begin{macrocode}
\newrobustcmd*{\aarg}[1]{%
  {\aargstyle{%
    {\ttfamily<}%
    \meta{#1}%
    {\ttfamily>}%
  }}%
}
%    \end{macrocode}
% \end{macro}
%
% \begin{macro}{\sarg}
% Prints star with given style.
%    \begin{macrocode}
\newrobustcmd*{\sarg}{{\sargstyle{*}}}
%    \end{macrocode}
% \end{macro}
%
% \begin{macro}{\pkg}
% \begin{macro}{\cls}
% \begin{macro}{\lib}
% \begin{macro}{\env}
% \begin{macro}{\opt}
% \begin{macro}{\file}
%    \begin{macrocode}
\newrobustcmd*\pkg[1]{{\pkgstyle{#1}}}
\newrobustcmd*\cls[1]{{\clsstyle{#1}}}
\newrobustcmd*\lib[1]{{\libstyle{#1}}}
\newrobustcmd*\env[1]{{\envstyle{#1}}}

\newrobustcmd*\opt{\@ifstar\ys@opt\y@opt}
\def\y@opt#1{{\optstyle{#1}}}
\def\ys@opt#1{{\optstyle{#1}}\optpar{#1}}
\newrobustcmd*\optpar[1]{\marginpar{\hbox to \marginparwidth{\hss\y@opt{#1}}}}

\newrobustcmd*\file[1]{{\filestyle{#1}}}
\newcommand*\pkgstyle[1]{\texttt{\textcolor{pkg}{#1}}}
\newcommand*\clsstyle[1]{\texttt{\textcolor{cls}{#1}}}
\newcommand*\libstyle[1]{\texttt{\textcolor{lib}{#1}}}
\newcommand*\envstyle[1]{\texttt{\textcolor{env}{#1}}}
\newcommand*\optstyle[1]{\textsf{\textcolor{opt}{#1}}}
\newcommand*\filestyle[1]{\texttt{\textcolor{file}{#1}}}
\colorlet{cls}{none}
\colorlet{lib}{none}
\colorlet{env}{none}
\colorlet{file}{none}
\colorlet{pkg}{none}
\definecolor{opt}{rgb}{0.5,0.16666,0}
%    \end{macrocode}
% \end{macro}
% \end{macro}
% \end{macro}
% \end{macro}
% \end{macro}
% \end{macro}
%
% \begin{macro}{\cs}
% \begin{macro}{\cmd}
%    \begin{macrocode}
\newrobustcmd*\cs[1]{\texttt{\textbackslash #1}}
\newrobustcmd*\cmd[1]{\texttt{{\escapechar=92\string#1}}}
%    \end{macrocode}
% \end{macro}
% \end{macro}
%
% \begin{macro}{\Key}
%    \begin{macrocode}
\newrobustcmd*\Key[1]{\PrintKeyName{#1}\MacroArgs}
%    \end{macrocode}
% \end{macro}
%
% \subsubsection{Text Formatting Styles}
%
% \begin{macro}{\macrodescstyle}
% Style of described macro names.
%    \begin{macrocode}
\def\macrodescstyle{\ttfamily\bfseries\color{macrodesc}}
%    \end{macrocode}
% \end{macro}
%
% \begin{macro}{\macrodescstyle}
% Style of described macro names.
%    \begin{macrocode}
\def\keydescstyle{\ttfamily\bfseries\color{keydesc}}
%    \end{macrocode}
% \end{macro}
%
% \begin{macro}{\macroargsstyle}
% Default style for macro arguments (e.g.\ \cs{MacroArgs}).
%    \begin{macrocode}
\def\macroargsstyle{\ttfamily}
%    \end{macrocode}
% \end{macro}
%
% \begin{macro}{\envcodestyle}
% Default style for code body content in described environments.
%    \begin{macrocode}
\def\envcodestyle{\ttfamily}
%    \end{macrocode}
% \end{macro}
%
% \begin{macro}{\verbstyle}
% Style for verbatim text inside macro argument list.
%    \begin{macrocode}
\def\verbstyle{\verbatim@font}
%    \end{macrocode}
% \end{macro}
%
% \begin{macro}{\metastyle}
% Meta text style. Because \cs{macroargsstyle} might be also active a |\normalfont| reset
% the font.
%    \begin{macrocode}
\def\metastyle{\normalfont\itshape\color{meta}}
%    \end{macrocode}
% \end{macro}
%
% \begin{macro}{\margstyle}
% Style for \cs{marg}.
%    \begin{macrocode}
\def\margstyle{}
%    \end{macrocode}
% \end{macro}
%
% \begin{macro}{\Optional}
% \begin{macro}{\optional}
% \begin{macro}{\optionalstyle}
%    \begin{macrocode}
\protected\def\Optional{\optionalon\optional}
\def\optionalstyle{\blendcolors*{!60!white}\color{black!75}}
%    \end{macrocode}
% \end{macro}
% \end{macro}
% \end{macro}
%
% \begin{macro}{\optionalon}
% \begin{macro}{\optionaloff}
%    \begin{macrocode}
\def\optionalon{\protected\def\optional{\optionalstyle}}
\def\optionaloff{\let\optional\relax}
\optionalon
%    \end{macrocode}
% \end{macro}
% \end{macro}
%
% \begin{macro}{\oargstyle}
% Style for \cs{oarg}. A special color is set to show the `optional' status.
%    \begin{macrocode}
\def\oargstyle{\optional}
%    \end{macrocode}
% \end{macro}
%
% \begin{macro}{\pargstyle}
% Style for \cs{parg}.
%    \begin{macrocode}
\def\pargstyle{}
%    \end{macrocode}
% \end{macro}
%
% \begin{macro}{\aargstyle}
% Style for \cs{aarg}.
%    \begin{macrocode}
\def\aargstyle{}
%    \end{macrocode}
% \end{macro}
%
% \begin{macro}{\sargstyle}
% Style for \cs{sarg}. A special color is set to show the `optional' status.
%    \begin{macrocode}
\def\sargstyle{\ttfamily\color{optional}}
%    \end{macrocode}
% \end{macro}
%
% \subsubsection{Dimension Registers}
%
% \begin{macro}{\descindent}
%    \begin{macrocode}
\newdimen\descindent
\descindent=-\parindent
%    \end{macrocode}
% \end{macro}
%
% \begin{macro}{\beforedescskip}
%    \begin{macrocode}
\newdimen\beforedescskip
\beforedescskip=\bigskipamount
%    \end{macrocode}
% \end{macro}
%
% \begin{macro}{\afterdescskip}
%    \begin{macrocode}
\newdimen\afterdescskip
\afterdescskip=\medskipamount
%    \end{macrocode}
% \end{macro}
%
% \begin{macro}{\descsep}
% Set to |1em| in |tt| font.
%    \begin{macrocode}
\newdimen\descsep
\begingroup
\ttfamily
\global\descsep=1em\relax
\endgroup
%    \end{macrocode}
% \end{macro}
%
%
% \subsubsection{Macro Argument Reading Mechanism}
%
% \begin{macro}{\read@Macro@arg}
% Reads next token and calls second macro.
%    \begin{macrocode}
\def\read@Macro@arg{%
  \futurelet\@let@token\handle@Macro@arg
}
%    \end{macrocode}
% \end{macro}
%
% \begin{macro}{\AlsoMacro}
% Reads argument while |@| is a letter,
% prints the macro name and reads further arguments.
%    \begin{macrocode}
\newcommand*\AlsoMacro{%
  \begingroup\makeatletter
  \AlsoMacro@
}
\def\AlsoMacro@#1{%
  \endgroup
%<*DEBUG>
   %\typeout{DEBUG: Macro: \string#1}%
%</DEBUG>
  \PrintMacroName{#1}%
  \read@Macro@arg
}
%    \end{macrocode}
% \end{macro}
%
% \begin{macro}{\ydoc@short@AlsoMacro}
% Makes |&| an alias for |\AlsoMacro|.
%    \begin{macrocode}
\begingroup
\catcode`\|\active
\gdef\ydoc@short@AlsoMacro{%
  \catcode`\|\active
  \let|\AlsoMacro
}
\endgroup
%    \end{macrocode}
% \end{macro}
%
%
% \begin{macro}{\ydoc@macrocatcodes}
% Sets the catcodes inside for |read@Macro@arg| material.
%    \begin{macrocode}
\def\ydoc@macrocatcodes{%
  \ydoc@short@AlsoMacro
  \@makeother\'%
  \@makeother\!%
  \@makeother\[%
  \@makeother\]%
  \@makeother\(%
  \@makeother\)%
}
%    \end{macrocode}
% \end{macro}
%
%
% \begin{macro}{\handle@Macro@arg}
% Checks if next token is the begin of a valid macro argument and calls
% the appropriate read macro or the end macro otherwise.
%    \begin{macrocode}
\def\handle@Macro@arg{%
  \expandafter\let\expandafter\handler\csname handle@Macro@token@\meaning\@let@token\endcsname
  \ifx\handler\relax
    \def\handler{\ifhmode\unskip\fi\end@Macro@args}%
%<*DEBUG>
 %  \typeout{DEBUG: Stopped at: \expandafter\meaning\csname @let@token\endcsname}%
 %  \typeout{}%
 %\else
 %\expandafter\ifx\csname @let@token\endcsname\AlsoMacro
 %  \typeout{DEBUG: TOKEN: \string\AlsoMacro}%
 %\else
 %  \typeout{DEBUG: TOKEN: \expandafter\meaning\csname @let@token\endcsname}%
 %\fi
%</DEBUG>
  \fi
  \handler
}
\def\define@Macro@handler{%
    \begingroup
    \ydoc@macrocatcodes
    \define@Macro@handler@
}
\def\define@Macro@handler@#1{%
    \endgroup
    \@namedef{handle@Macro@token@\meaning#1}%
}
%    \end{macrocode}
% \end{macro}
%
% \begin{macro}{\end@Macro@args}
% Closes box as calls hook. Might be locally redefined by some
% macros calling \cs{read@Macro@arg}.
%    \begin{macrocode}
\def\end@Macro@args{%
  \y@egroup
  \after@Macro@args
}
%    \end{macrocode}
% \end{macro}
%
% \begin{macro}{\after@Macro@args}
% Hook to add additional commands in certain situations.
%    \begin{macrocode}
\def\after@Macro@args{%
}
%    \end{macrocode}
% \end{macro}
%
% \subsubsection*{Macro argument reading macros}
% This macros read the macro arguments and call the appropriate format macros.
%
% \begin{macro}{\read@Macro@marg}
%    \begin{macrocode}
\define@Macro@handler{\bgroup}{%
    \begingroup
        \afterassignment\read@Macro@marg@
        \let\@let@token=%
}
\def\read@Macro@marg@{%
        \bgroup
        \margstyle{}%
        \let\end@Macro@args\empty%
        {\ttfamily\braceleft}%
        \aftergroup\read@Macro@marg@@
        \read@Macro@arg
}
\def\read@Macro@marg@@{%
        {\ttfamily\braceright}%
    \endgroup
    \read@Macro@arg
}
%    \end{macrocode}
% \end{macro}
%
% \begin{macro}{\read@Macro@oarg}
%    \begin{macrocode}
\define@Macro@handler{[}[{%
    \begingroup
        \let\read@Macro@oarg@end\read@Macro@oarg@@end
        \let\end@Macro@args\read@Macro@oarg@end
        \oargstyle{}%
        {\ttfamily[}%]
        \read@Macro@arg
}
\define@Macro@handler{]}{%
    \read@Macro@oarg@end
}
\def\read@Macro@oarg@@end#1]{%
        #1%
        {\ttfamily]}%
    \endgroup
    \read@Macro@arg
}
\def\read@Macro@oarg@end{\end@Macro@args}
\let\read@Macro@aarg@end\read@Macro@oarg@end
\let\read@Macro@parg@end\read@Macro@oarg@end
%    \end{macrocode}
% \end{macro}
%
% \begin{macro}{\read@Macro@parg}
%    \begin{macrocode}
\define@Macro@handler{(}({%
    \begingroup
        \let\read@Macro@parg@end\read@Macro@parg@@end
        \let\end@Macro@args\read@Macro@parg@end
        \pargstyle{}%
        {\ttfamily(}%)
        \read@Macro@arg
}
\define@Macro@handler{)}{%
    \read@Macro@parg@end
}
\def\read@Macro@parg@@end#1){%
        #1%
        {\ttfamily)}%
    \endgroup
    \read@Macro@arg
}
%    \end{macrocode}
% \end{macro}
%
% \begin{macro}{\read@Macro@aarg}
%    \begin{macrocode}
\def\read@Macro@aarg<{%
    \begingroup
        \let\read@Macro@aarg@end\read@Macro@aarg@@end
        \let\end@Macro@args\read@Macro@aarg@end
        \aargstyle{}%
        {\ttfamily<}%
        \read@Macro@arg
}
\define@Macro@handler{>}{%
    \read@Macro@aarg@end
}
\def\read@Macro@aarg@@end#1>>{%
        #1%
        {\ttfamily>}%
    \endgroup
    \read@Macro@arg
}
%    \end{macrocode}
% \end{macro}
%
% \begin{macro}{\read@Macro@angle}
%    \begin{macrocode}
\define@Macro@handler{<}<{%
  \futurelet\@let@token\read@Macro@angle@
}
%    \end{macrocode}
% \end{macro}
%
% \begin{macro}{\read@Macro@angle@}
%    \begin{macrocode}
\def\read@Macro@angle@{%
  \ifx\@let@token<%
    \expandafter\read@Macro@aarg
  \else
    \expandafter\read@Macro@meta
  \fi
}
%    \end{macrocode}
% \end{macro}
%
% \begin{macro}{\read@Macro@meta}
%    \begin{macrocode}
\def\read@Macro@meta#1>{%
  \meta{#1}\read@Macro@arg
}
%    \end{macrocode}
% \end{macro}
%
% \begin{macro}{\read@Macro@sarg}
%    \begin{macrocode}
\define@Macro@handler**{%
  \sarg\read@Macro@arg
}
%    \end{macrocode}
% \end{macro}
%
% Allows `|=|' to be used directly without switching to verbatim
% mode. This is especially useful for keys.
%    \begin{macrocode}
\define@Macro@handler{=}={%
  =\read@Macro@arg
}
%    \end{macrocode}
%
% \begin{macro}{\read@Macro@verb}
% Sets up verbatim mode calls second macro.
%    \begin{macrocode}
\define@Macro@handler{'}'{%
  \begingroup
  \let\do\@makeother
  \dospecials
  \@noligs
  \@makeother\'%
  \obeyspaces
  \read@Macro@verb@
}
%    \end{macrocode}
% \end{macro}
%
% \begin{macro}{\read@Macro@verb@}
% Closes verbatim mode and formats text.
% If |#1| is empty (|''|) than a single |'| is printed.
%    \begin{macrocode}
\begingroup
\@makeother\'%
\gdef\read@Macro@verb@#1'{%
  \endgroup
  \ifx\relax#1\relax
    {\verbstyle{\string'}}%
  \else
    {%
     \frenchspacing
     \@noligs\verbstyle{#1}}%
  \fi
  \read@Macro@arg
}
\endgroup
%    \end{macrocode}
% \end{macro}
%
% \begin{macro}{\read@Macro@cmds}
% Simply executes given code.
%    \begin{macrocode}
\define@Macro@handler!!#1!{%
  #1\relax
  \read@Macro@arg
}
%    \end{macrocode}
% \end{macro}
%
% \begin{macro}{\read@Macro@rmspace}
% Removes space. The |\@firstofone| is used to preserve the space in the macro definition.
%    \begin{macrocode}
\define@Macro@handler{\@sptoken} {%
  \read@Macro@arg
}
%    \end{macrocode}
% \end{macro}
%
% \begin{macro}{\read@Macro@addtoken}
% Takes token over from input to output `stream'. This is used for |\space| and |~|.
%    \begin{macrocode}
\define@Macro@handler{~}#1{%
  #1\read@Macro@arg
}
\AtBeginDocument{%
\define@Macro@handler{~}#1{%
  #1\read@Macro@arg
}
}
\define@Macro@handler{\space}#1{%
  #1\read@Macro@arg
}
%    \end{macrocode}
% \end{macro}
%
% \subsubsection{Description Macros}
% \subsubsection*{For Macros}
%
% \begin{macro}{\DescribeMacro}
%    \begin{macrocode}
\@ifundefined{DescribeMacro}{}{%
  \PackageInfo{ydoc-desc}{Redefining \string\DescribeMacro}{}%
}
%    \end{macrocode}
% A \cs{DescribeMacro} places itself in a \env{DescribeMacros} environment.
% Multiple \cs{DescribeMacro} macros will stack themselves inside this environment.
% For this to work \cs{DescribeMacros} is locally defined to |\y@egroup| to close the
% |\hbox| from the previous \cs{DescribeMacro}.
%    \begin{macrocode}
\def\DescribeMacro{%
  \DescribeMacros
  \let\DescribeMacros\y@egroup
  \optionalon
  \def\after@Macro@args{\endDescribeMacros}%
  \begingroup\makeatletter
  \Describe@Macro
}
%    \end{macrocode}
% \end{macro}
%
% \begin{macro}{\DescribeScript}
%    \begin{macrocode}
\def\DescribeScript#1{%
  \DescribeMacros
  \let\DescribeMacros\y@egroup
  \optionalon
  \def\after@Macro@args{\endDescribeMacros}%
  \hbox\y@bgroup
  \texttt{#1}%
  \ydoc@macrocatcodes
  \macroargsstyle
  \read@Macro@arg~%
}
%    \end{macrocode}
% \end{macro}
%
% \begin{macro}{\DescribeKey}
%    \begin{macrocode}
\def\DescribeKey{%
  \DescribeKeys
  \let\DescribeKeys\y@egroup
  \optionalon
  \def\after@Macro@args{\endDescribeKeys}%
  \begingroup\makeatletter
  \Describe@Macro
}
%    \end{macrocode}
% \end{macro}
%
% \begin{macro}{\Describe@Macro}
%    \begin{macrocode}
\def\Describe@Macro#1{%
  \endgroup
  \edef\name{\expandafter\@gobble\string#1}%
  \global\@namedef{href@desc@\name}{}%
  \immediate\write\@mainaux{%
    \global\noexpand\@namedef{href@desc@\name}{}%
  }%
  \hbox\y@bgroup
  \@ifundefined{href@impl@\name}{}{\hyperlink{impl:\name}}%
  {%
  \hbox{\vbox to 0pt{\vss\hbox{\raisebox{4ex}{\hypertarget{desc:\name}{}}}}%
  \PrintMacroName{#1}}%
  }%
  \ydoc@macrocatcodes
  \macroargsstyle
  \read@Macro@arg
}
%    \end{macrocode}
% \end{macro}
%
% \begin{macro}{\MakeShortMacroArgs}
% Defines the given character as short version for |\MacroArgs|.
% It is first define to be a short verbatim character to take
% advantage of the house-keeping (save \& restore of the original
% catcode and definition) of |shortvrb|.
%
% The starred version define the character to act like |\Macro| instead.
%    \begin{macrocode}
\newcommand*\MakeShortMacroArgs{%
  \@ifstar
    {\@MakeShortMacroArgs\Macro}%
    {\@MakeShortMacroArgs\MacroArgs}%
}
\def\@MakeShortMacroArgs#1#2{%
  \MakeShortVerb{#2}
  \catcode`#2\active
  \begingroup
  \catcode`\~\active
  \lccode`\~`#2\relax
  \lowercase{\endgroup\gdef~{\bgroup\let~\egroup#1}}%
}
%    \end{macrocode}
% \end{macro}
%
% \begin{macro}{\DeleteShortMacroArgs}
%    \begin{macrocode}
\newcommand*\DeleteShortMacroArgs[1]{%
  \DeleteShortVerb{#1}%
}
%    \end{macrocode}
% \end{macro}
%
% \begin{macro}{\Macro}
% Simply uses the two macros below.
%    \begin{macrocode}
\newcommand*\Macro{\MacroArgs\AlsoMacro}
%    \end{macrocode}
% \end{macro}
%
% \begin{macro}{\@Macro}
% Alternative definition of \cs{Macro} inside \env{DescribeMacros} environments.
%    \begin{macrocode}
\def\@Macro{%
  \begingroup\makeatletter
  \Describe@Macro
}
%    \end{macrocode}
% \end{macro}
%
%    \begin{macrocode}
\define@Macro@handler\AlsoMacro{}
\define@Macro@handler\DescribeMacro{}
\define@Macro@handler\DescribeKey{}
\define@Macro@handler\DescribeScript{}
%    \end{macrocode}
%
% \begin{macro}{\MacroArgs}
% Uses the normal macro argument reading mechanism from \cs{DescribeMacro}.
% Instead of a box a simple group is added.
%    \begin{macrocode}
\newcommand*\MacroArgs{%
  \begingroup
  \def\end@Macro@args{\endgroup\xspace}%
  \ydoc@macrocatcodes
  \macroargsstyle
%<*DEBUG>
   %\typeout{}%
   %\typeout{DEBUG: Start MacroArgs}%
%</DEBUG>
  \read@Macro@arg
}
\RequirePackage{xspace}
%    \end{macrocode}
% \end{macro}
%
% \begin{macro}{\DescribeMacros}
%    \begin{macrocode}
\def\DescribeMacros{%
  \begingroup
  \let\Macro\@Macro
  \parindent=0pt\relax
  \setbox\descbox\vbox\y@bgroup
}
%    \end{macrocode}
% \end{macro}
%
% \begin{macro}{\endDescribeMacros}
%    \begin{macrocode}
\def\endDescribeMacros{%
  \y@egroup
  \PrintMacros
  \endgroup
}
%    \end{macrocode}
% \end{macro}
%
% \begin{macro}{\DescribeKeys}
%    \begin{macrocode}
\def\DescribeKeys{%
  \begingroup
  \let\PrintMacroName\PrintKeyName
  \let\Key\@Macro
  \parindent=0pt\relax
  \setbox\descbox\vbox\y@bgroup
}
%    \end{macrocode}
% \end{macro}
%
% \begin{macro}{\endDescribeKeys}
%    \begin{macrocode}
\def\endDescribeKeys{%
  \y@egroup
  \PrintKeys
  \endgroup
}
\def\PrintKeys{\PrintMacros}
%    \end{macrocode}
% \end{macro}
%
% \begin{macro}{\DescribeMacrosTabcolsep}
%    \begin{macrocode}
\def\DescribeMacrosTabcolsep{\tabcolsep}
%    \end{macrocode}
% \end{macro}
%
% \begin{macro}{\DescribeMacrosTab}
%    \begin{macrocode}
\def\DescribeMacrosTab{%
  \DescribeMacros
  \hbox\y@bgroup
  \tabcolsep=\DescribeMacrosTabcolsep\relax
  \DescribeMacrosTab@
}
\def\DescribeMacrosTab@#1{\tabular{@{}#1@{}}}
%    \end{macrocode}
% \end{macro}
%
% \begin{macro}{\endDescribeMacrosTab}
%    \begin{macrocode}
\def\endDescribeMacrosTab{%
  \endtabular\y@egroup
  \endDescribeMacros
}
%    \end{macrocode}
% \end{macro}
%
% \subsubsection*{For Lengths}
%
% \begin{macro}{\DescribeLength}
%    \begin{macrocode}
\newcommand*\DescribeLength{%
  \begingroup
  \let\DescribeLength\Describe@Length
  \setbox\descbox\hbox\y@bgroup
    \tabular{@{}l@{\hspace{2em}}l@{}}%
    \Describe@Length
}
%    \end{macrocode}
% \end{macro}
%
% \begin{macro}{\Describe@Length}
%    \begin{macrocode}
\newcommand*\Describe@Length[2]{%
  \PrintLengthName{#1}&
  (Default: {\macroargsstyle#2\unskip})%
  \@ifnextchar\DescribeLength
    {\\}%
    {%
      \endtabular
      \y@egroup
      \PrintLength
      \endgroup
    }%
}
%    \end{macrocode}
% \end{macro}
%
% \subsubsection*{For Environments}
%
% \begin{macro}{\DescribeEnv}
%    \begin{macrocode}
\@ifundefined{DescribeEnv}{}{%
  \PackageInfo{ydoc-desc}{Redefining \string\DescribeEnv}{}%
}
\let\DescribeEnv\relax
%    \end{macrocode}
%    \begin{macrocode}
\newcommand*\DescribeEnv[2][]{%
  \begingroup
  \def\DescribeEnv@name{#2}%
  \let\\\DescribeEnv@newline
%    \end{macrocode}
%
% Sets after-macro-arguments hook.
% First checks if the environment or macro version was used.
% The environment starts a new line only if the next token isn't |\end|,
% which is taken as end of the environment.
%    \begin{macrocode}
  \ifx\@currenvir\DescribeEnv@string
    \def\after@Macro@args{%
      \let\after@Macro@args\empty
      \setbox\@tempboxa\hbox\y@bgroup
      \@ifnextchar\end{}%
        {\DescribeEnv@newline}%
      #1%
    }%
%    \end{macrocode}
% The macro version adds the optional argument as content line if given.
%    \begin{macrocode}
  \else
    \ifx\relax#1\relax
      \def\after@Macro@args{%
        \y@bgroup
        \endDescribeEnv
      }%
    \else
      \def\after@Macro@args{%
        \setbox\@tempboxa\hbox\y@bgroup
        \DescribeEnv@newline\MacroArgs#1%
        \endDescribeEnv
      }%
    \fi
  \fi
%    \end{macrocode}
% Start |\vbox| and adds first line.
%    \begin{macrocode}
  \setbox\descbox\vbox\y@bgroup
  \envcodestyle
  \let\PrintEnv\PrintSubEnv
  \hbox\y@bgroup
  \PrintEnvName{\begin}{\DescribeEnv@name}%
  \ydoc@macrocatcodes
  \macroargsstyle
  \read@Macro@arg
}
%    \end{macrocode}
% \end{macro}
%
% \begin{macro}{\DescribeEnv@newline}
% Closes existing and starts a new horizontal box representing a indented line.
% The optional argument allows to add extra space between lines like the normal |\\|.
% Negative values are not supported.
%    \begin{macrocode}
\newcommand*\DescribeEnv@newline[1][0pt]{%
  \strut\y@egroup
  {\vskip#1}%
  \hbox\y@bgroup\strut
  \hspace*{\descsep}%
  \ignorespaces
}%
%    \end{macrocode}
% \end{macro}
%
% \begin{macro}{\DescribeEnv@string}
% Holds the environment name for comparison.
%    \begin{macrocode}
\def\DescribeEnv@string{DescribeEnv}
%    \end{macrocode}
% \end{macro}
%
% \begin{macro}{\descbox}
% Save box to store description content.
%    \begin{macrocode}
\newbox\descbox
%    \end{macrocode}
% \end{macro}
%
% \begin{macro}{\endDescribeEnv}
%    \begin{macrocode}
\def\endDescribeEnv{%
  \y@egroup
  \begingroup
  \setbox\@tempboxa\lastbox
  \ifcase0%
    \ifdim\wd\@tempboxa>\descsep1\fi
    \ifdim\ht\@tempboxa>\ht\strutbox1\fi
    \ifdim\dp\@tempboxa>\dp\strutbox1\fi
  \else
    \box\@tempboxa
  \fi
  \endgroup
  \hbox\y@bgroup
    \PrintEnvName{\end}{\DescribeEnv@name}
  \y@egroup
  \y@egroup
  \PrintEnv
  \endgroup
}
%    \end{macrocode}
% \end{macro}
%
%
% \subsubsection{Print Macros}
%
% \begin{macro}{\PrintMacroName}
% Formats macro name. The backslash is forced to |tt| font.
%    \begin{macrocode}
\def\PrintMacroName#1{%
  {\macrodescstyle{\strut
   \texttt{\char92}%
   \escapechar\m@ne
   \string#1\strut}}%
}
%    \end{macrocode}
% \end{macro}
%
% \begin{macro}{\PrintKeyName}
% Formats macro name. The backslash is forced to |tt| font.
%    \begin{macrocode}
\def\PrintKeyName#1{%
  {\keydescstyle{\strut
   #1\strut}}%
}
%    \end{macrocode}
% \end{macro}
%
% \begin{macro}{\PrintLengthName}
% Formats length register name.
%    \begin{macrocode}
\let\PrintLengthName\PrintMacroName
%    \end{macrocode}
% \end{macro}
%
%
% \begin{macro}{\PrintEnvName}
% |#1| = `|\begin|' or `|\end|', |#2| = env name.
%    \begin{macrocode}
\def\PrintEnvName#1#2{%
  \strut
  \string#1\braceleft
  {\macrodescstyle#2\strut}%
  \braceright
}
%    \end{macrocode}
% \end{macro}
%
% \begin{macro}{\PrintMacros}
% Prints macros described using \cs{DescribeMacros}.
% The actual content was stored inside \cs{descbox}.
% If it is wider than the line width it is centered.
%    \begin{macrocode}
\def\PrintMacros{%
  \par\vspace\beforedescskip
  \begingroup
  \sbox\@tempboxa{\descframe{\usebox{\descbox}}}%
  \Needspace*{\dimexpr\ht\@tempboxa+3\baselineskip\relax}%
  \par\noindent
  \ifdim\wd\@tempboxa>\dimexpr\linewidth-2\descindent\relax
    \makebox[\linewidth][c]{\usebox\@tempboxa}%
  \else
    \hspace*{\descindent}%
    \usebox\@tempboxa
  \fi
  \endgroup
  \par
  \vspace\afterdescskip
  \par\noindent
}
\def\descframe#1{%
    \fbox{\hspace*{\descsep}#1\hspace*{\descsep}}%
}
%    \end{macrocode}
% \end{macro}
%
% \begin{macro}{\PrintLength}
% Prints lengths registers described using one or multiple \cs{DescribeLength}.
%    \begin{macrocode}
\let\PrintLength\PrintMacros
%    \end{macrocode}
% \end{macro}
%
% \begin{macro}{\PrintEnv}
% Prints \env{DescribeEnv} environments.
% The actual content was stored inside \cs{descbox}.
%    \begin{macrocode}
\let\PrintEnv\PrintMacros
%    \end{macrocode}
% \end{macro}
%
% \begin{macro}{\PrintSubEnv}
% Prints sub environments, i.e.\ \env{DescribeEnv} environments
% inside the body of another \env{DescribeEnv}.
% The actual content was stored inside \cs{descbox}.
%    \begin{macrocode}
\def\PrintSubEnv{%
  \hbox{\hbox{\usebox{\descbox}}}%
}
%    \end{macrocode}
% \end{macro}
%
%
% \subsubsection{Special Character Macros}
%
% \begin{macro}{\bslash}
% Defines an expandable backslash with catcode 12: `\texttt{\bslash}$_{12}$'.
% The |\@firstofone| trick is used to read the |\gdef\bslash| code before changing the catcode.
%    \begin{macrocode}
{%
\@firstofone
%    \end{macrocode}
% \end{macro}
%
% \begin{macro}{\percent}
% Defines an expandable percent character with catcode 12: `\percent$_{12}$'.
%    \begin{macrocode}
\begingroup
\catcode`\%=12
\gdef\percent{%}
\endgroup
%    \end{macrocode}
% \end{macro}
%
% \begin{macro}{\braceleft}
% \begin{macro}{\braceright}
% Defines expandable left and right braces with catcode 12: `\texttt{\braceleft}$_{12}$' `\texttt{\braceright}$_{12}$'.
%    \begin{macrocode}
\begingroup
\catcode`\<=1
\catcode`\>=2
\catcode`\{=12
\catcode`\}=12
\gdef\braceleft <{>
\gdef\braceright<}>
\endgroup
%    \end{macrocode}
% \end{macro}
% \end{macro}
%
% \subsubsection{Other Macros}
%
% \begin{macro}{\y@bgroup}
% \begin{macro}{\y@egroup}
% These macros are used to begin and end |\vbox|/|\hbox|-es.
%    \begin{macrocode}
\def\y@bgroup{\bgroup\color@setgroup}
\def\y@egroup{\color@endgroup\egroup}
%    \end{macrocode}
% \end{macro}
% \end{macro}
%
% \begin{macro}{\codeline}
%    \begin{macrocode}
\newcommand*{\codeline}[1][c]{%
    \codelinebefore
    \hbox to \hsize\bgroup
    \ifx i#1\hspace*{\leftmargin}\else
        \ifx l#1\else\hss\fi
    \fi
    \let\xspace\relax
    \hbox\bgroup
    \aftergroup\codeline@end
    \aftergroup#1%
    \afterassignment\MacroArgs
    \let\@let@token=%
}
\def\codeline@end#1{%
    \ifx r#1\else\hss\fi
    \egroup
    \codelineafter
}
\newcommand*\codelinebefore{\par\smallskip\noindent}
\newcommand*\codelineafter {\par\smallskip\noindent}
%    \end{macrocode}
% \end{macro}
%
% \begin{environment}{codequote}
%    \begin{macrocode}
\newenvironment{codequote}{%
    \def\\{\newline\relax\MacroArgs}%
    \par\smallskip\bgroup\leftskip=\leftmargin\rightskip=\rightmargin\noindent\MacroArgs}
    {\par\egroup\smallskip\noindent\ignorespacesafterend}
%    \end{macrocode}
% \end{environment}
%
% \begin{environment}{macroquote}
%    \begin{macrocode}
\newenvironment{macroquote}{%
    \def\\{\newline\relax\Macro}%
    \par\smallskip\bgroup\leftskip=\leftmargin\rightskip=\rightmargin\noindent\Macro}
    {\par\egroup\smallskip\noindent\ignorespacesafterend}
%    \end{macrocode}
% \end{environment}
% \iffalse
%</ydoc-desc.sty>
% \fi
%
% \iffalse
%<*ydoc-expl.sty>
% \fi
% \subsection{Include Code Examples}
%
%    \begin{macrocode}
\NeedsTeXFormat{LaTeX2e}[1999/12/01]
\ProvidesPackage{ydoc-expl}[%
%<!DATE>
%<!VERSION>
%<*DRIVER>
    2011/08/11 develop
%</DRIVER>
    ydoc package to insert live examples of LaTeX code]
%    \end{macrocode}
%
% \iffalse
%<*dontinclude>
%    \begin{macrocode}
\newcounter{example}
\def\examplename{Example}
\newenvironment{example}[1]{%
  \par\bigskip\noindent
  \centerline{\examplename~\refstepcounter{example}\theexample: #1}%
  \smallskip
  \hrule
  \smallskip
}{%
  \smallskip
  \hrule
  \par\bigskip
}
%    \end{macrocode}
%
% \begin{macro}{\PrintExample}
%    \begin{macrocode}
\def\PrintExample{%
  \usebox\examplecodebox
  \hfill
  \usebox\exampleresultbox
}
%    \end{macrocode}
% \end{macro}
%</dontinclude>
% \fi
%
%    \begin{macrocode}
\RequirePackage{listings}
\lst@RequireAspects{writefile}
\def\ydoc@exafile{\jobname.exa}
%    \end{macrocode}
%
% \begin{style}{examplecode}
%    \begin{macrocode}
\lstdefinestyle{examplecode}{%
    language=[latex]tex,
    basicstyle=\ttfamily,
    columns=fullflexible,
    numbers=left,
    firstnumber=1,
    numberstyle=\tiny\color{gray}\sffamily,
    numbersep=5pt,
    breaklines,prebreak={\mbox{\tiny$\swarrow$}},
    commentstyle=\color{black!60},
}%
%    \end{macrocode}
% \end{style}
%
% \begin{style}{exampleresult}
%    \begin{macrocode}
\lstdefinestyle{exampleresult}{%
    firstnumber=1,
    gobble=0,
    basicstyle=\ttfamily,
    columns=fullflexible,
    commentstyle=\color{black!60},
}
%    \end{macrocode}
% \end{style}
%
% \begin{style}{exampleextract}
%    \begin{macrocode}
\lstdefinestyle{exampleextract}{gobble=4}%
%    \end{macrocode}
% \end{style}
%
%    \begin{macrocode}
\newbox\examplecodebox
\newbox\exampleresultbox
%    \end{macrocode}
%
% \begin{macro}{\BoxExample}
%    \begin{macrocode}
\def\BoxExample{%
  \setbox\examplecodebox\hbox{\color@setgroup
    \lstinputlisting[style=examplecode,style=thisexampleprint]%
    {\ydoc@exafile}%
  \unskip\color@endgroup}%
  \setbox\exampleresultbox\hbox{\color@setgroup
    \lstset{style=exampleresult}%
    \@@input\ydoc@exafile\relax
  \unskip\color@endgroup}%
}
%    \end{macrocode}
% \end{macro}
%
% \begin{macro}{\PrintExample}
%    \begin{macrocode}
%<*DISABLED>
\RequirePackage{showexpl}
\def\PrintExample{%
  \begingroup
  \lstset{style=examplecode}%
  \MakePercentComment
  \LTXinputExample[varwidth]{\ydoc@exafile}%
  \endgroup
}
%</DISABLED>
%    \end{macrocode}
% \end{macro}
%
%
%
% \begin{macro}{\PrintExample}
%    \begin{macrocode}
\def\PrintExample{%
  \begingroup
  \BoxExample
  \@tempdima=\textwidth
  \advance\@tempdima by -\wd\examplecodebox\relax
  \advance\@tempdima by -\wd\exampleresultbox\relax
  \advance\@tempdima by -15pt\relax
  \ifdim\@tempdima>\bigskipamount
    \hbox to \textwidth{%
     \null\hss
     \minipage[c]{\wd\exampleresultbox}\fbox{\usebox\exampleresultbox}\endminipage
     \hfill\hfill\hskip\bigskipamount\hskip15pt\hfill\hfill
     \minipage[c]{\wd\examplecodebox}\usebox\examplecodebox\endminipage
     \hss\null
     }%
  \else
    \vbox{%
        \centerline{\fbox{\usebox\exampleresultbox}}%
        \vspace{\bigskipamount}%
        \centerline{\usebox\examplecodebox}%
    }%
  \fi
  \endgroup
}
%    \end{macrocode}
% \end{macro}
%
% \begin{environment}{examplecode}
%    \begin{macrocode}
\lstnewenvironment{examplecode}[1][]{%
  \lstdefinestyle{thisexampleprint}{#1}%
  \setbox\@tempboxa\hbox\bgroup
  \lstset{style=exampleextract,#1}%
  \lst@BeginWriteFile{\ydoc@exafile}%
}
{%
  \lst@EndWriteFile
  \egroup
  \begingroup
  \MakePercentComment
  \catcode`\^^M=5\relax
  \PrintExample
  \endgroup
}
%    \end{macrocode}
% \end{environment}
%
%    \begin{macrocode}
\RequirePackage{float}
%    \end{macrocode}
%
% \begin{environment}{example}
%    \begin{macrocode}
\floatstyle{plain}
\newfloat{example}{tbhp}{loe}
\floatname{example}{\examplename}
\def\examplename{Example}
%    \end{macrocode}
% \end{environment}
%
% \begin{environment}{exampletable}
%    \begin{macrocode}
\newenvironment{exampletable}{%
  \floatstyle{plaintop}%
  \restylefloat{example}%
  \example
}{\endexample}
%    \end{macrocode}
% \end{environment}
% \iffalse
%</ydoc-expl.sty>
% \fi
%
% \iffalse
%<*ydocincl.tex>
% \fi
%
%    \begin{macrocode}
\expandafter\ifx\csname ydocinclversion\endcsname\relax\else
    \endinput
\fi

\chardef\ydocinclversion=1

\newread\inFile
\newread\subFile
\newwrite\outFile
\newif\ifContinue
\newlinechar=`^^J

\def\makeOther#1{\catcode`#1=12\relax}

\let\inLine\relax
\let\lastLine\relax

\def\includefiles#1#2{%
    \begingroup
    \immediate\openin\inFile#1\relax
    \immediate\openout\outFile#2\relax
    \makeOther\@%
    \makeOther\ \makeOther\\\makeOther\$%
    \makeOther\#\makeOther\^\makeOther\^^K%
    \makeOther\_\makeOther\^^A\makeOther\%%
    \makeOther\~\makeOther\{\makeOther\}\makeOther\&%
    \endlinechar-1\relax
    \Continuetrue
    \loop
      \let\lastLine\inLine
      \read\inFile to\inLine
      \ifeof\inFile
        \Continuefalse
      \else
        \expandafter\checkLine\inLine\empty\empty\empty\endLine
      \fi
      \ifContinue
    \repeat
    \immediate\closein\inFile
    \immediate\closeout\outFile
    \endgroup
    \end
}

\def\copyline{%
    \immediate\write\outFile{\inLine}%
}

\chardef\percentcharnum=`\%

\begingroup
\makeOther\%\makeOther\@\relax
\gdef\SubFileOptionString\relax
\catcode`\|=0
\makeOther\ \makeOther\\|relax
|gdef|IfFalseString{% \iffalse}|relax
|gdef|FiString{% \fi}|relax
|endgroup

\def\checkLine#1#2#3#4\endLine{%
    \def\firstthree{#1#2#3}%
    \ifx\firstthree\SubFileOptionString
        \readSubFile#4\endLine
    \else
        \copyline
    \fi
}

\def\readSubFile#1>#2\endLine{%
    \immediate\openin\subFile=#1\relax
    \ifeof\subFile
        % File not found
    \else
        \message{^^JIncluding subfile '#1'^^J}%
        \immediate\write\outFile{\CommentChar<*#1>}%
        \ifx\lastLine\IfFalseString
            \immediate\write\outFile{\FiString}%
        \fi
        \copySubFile
        \ifx\lastLine\IfFalseString
            \immediate\write\outFile{\IfFalseString}%
        \fi
        \immediate\write\outFile{\CommentChar</#1>}%
    \fi
    \immediate\closein\subFile
}

\def\copySubFile{%
    \read\subFile to\subLine
    \ifeof\subFile\else
        \immediate\write\outFile{\subLine}%
        \expandafter\copySubFile
    \fi
}
%    \end{macrocode}
%
% \iffalse
%</ydocincl.tex>
% \fi
%
% \Finale
% \endinput

%    \end{macrocode}
% \iffalse
%</ydoc.sty>
% \fi
%
% \iffalse
%<*ydoc.cfg>
% \fi
% \subsection{Config File}
%    \begin{macrocode}
%% Please delete the following line on manual changes:
\ProvidesFile{ydoc.cfg}[%
%<!DATE>
%<!VERSION>
%<*DRIVER>
   2011/08/11 develop
%</DRIVER>
   Default config file for ydoc]
%    \end{macrocode}
%
%    \begin{macrocode}
\usepackage[T1]{fontenc}
\IfFileExists{fourier.sty}{%
    \usepackage{fourier}
}{}
%    \end{macrocode}
%
% Use 'lmodern' only for the 'tt' font if fourier is installed.
%    \begin{macrocode}
\IfFileExists{lmodern.sty}{
    \IfFileExists{fourier.sty}{
        \renewcommand{\ttdefault}{lmtt}
    }{
        \usepackage{lmodern}
    }
}{}
%    \end{macrocode}
%
%
%    \begin{macrocode}
\urlstyle{sf}
%    \end{macrocode}
%
% Use micro-typesetting if pdftex is used:
%    \begin{macrocode}
\usepackage{ifpdf}
\ifpdf
\usepackage{microtype}
\fi
%    \end{macrocode}
%    \begin{macrocode}
\usepackage{array}
\usepackage{booktabs}
\usepackage{multicol}
\usepackage{xcolor}
\usepackage{listings}
\usepackage{booktabs}
\usepackage{hyperref}
%    \end{macrocode}
%
%    \begin{macrocode}
\reversemarginpar
%    \end{macrocode}
%
% \iffalse
%</ydoc.cfg>
% \fi
%
% \iffalse
%<*ydoc-code.sty>
% \fi
% \subsection{Macros and Environments to document Implementations}
%
%    \begin{macrocode}
\NeedsTeXFormat{LaTeX2e}[1999/12/01]
\ProvidesPackage{ydoc-code}[%
%<!DATE>
%<!VERSION>
%<*DRIVER>
    2011/08/11 develop
%</DRIVER>
    ydoc package to document macro code]
%    \end{macrocode}
%
%    \begin{macrocode}
\RequirePackage{hyperref}
\hypersetup{colorlinks=true,pdfborder=0 0 0,pdfborderstyle={}}
%    \end{macrocode}
%
%    \begin{macrocode}
\IfFileExists{needspace.sty}{%
    \RequirePackage{needspace}
}{%
    \def\Needspace{\@ifstar\@gobble\@gobble}
}
%    \end{macrocode}
%
% \subsubsection{Color and style definitions}
%    \begin{macrocode}
\RequirePackage{xcolor}
\definecolor{macroimpl}{rgb}{0.0,0.0,0.4}
%    \end{macrocode}
%
% \subsubsection{General Macros}
%
% \begin{macro}{\ydocwrite}
%    \begin{macrocode}
\@ifundefined{ydocwrite}{%
  \newwrite\ydocwrite
}{}
%    \end{macrocode}
% \end{macro}
%
% \begin{macro}{\ydocfname}
%    \begin{macrocode}
\@ifundefined{ydocfname}{%
  \def\ydocfname{\jobname.cod}%
}{}
%    \end{macrocode}
% \end{macro}
%
% \begin{macro}{\ydoc@catcodes}
%    \begin{macrocode}
\def\ydoc@catcodes{%
  \let\do\@makeother
  \dospecials
  \catcode`\\=\active
  \catcode`\^^M=\active
  \catcode`\ =\active
}
%    \end{macrocode}
% \end{macro}
%
% \subsubsection{Handling Macrocode}
% \begin{environment}{macrocode}
%    \begin{macrocode}
\def\macrocode{%
  \par\noindent
  \begingroup
  \ydoc@catcodes
  \macro@code
}
\def\endmacrocode{}
%    \end{macrocode}
% \end{environment}
%
% \begin{macro}{\macro@code}[1]{verbatim macro code}
%    \begin{macrocode}
\begingroup
\endlinechar\m@ne
\@firstofone{%
\catcode`\|=0\relax
\catcode`\(=1\relax
\catcode`\)=2\relax
\catcode`\*=14\relax
\catcode`\{=12\relax
\catcode`\}=12\relax
\catcode`\ =12\relax
\catcode`\%=12\relax
\catcode`\\=\active
\catcode`\^^M=\active
\catcode`\ =\active
}*
|gdef|macro@code#1^^M%    \end{macrocode}(*
|endgroup|expandafter|macro@@code|expandafter(|ydoc@removeline#1|noexpand|lastlinemacro)*
)*
|gdef|ydoc@removeline#1^^M(|noexpand|firstlinemacro)*
|gdef|ydoc@defspecialmacros(*
|def^^M(|noexpand|newlinemacro)*
|def (|noexpand|spacemacro)*
|def\(|noexpand|bslashmacro)*
)*
|gdef|ydoc@defrevspecialmacros(*
|def|newlinemacro(|noexpand^^M)*
|def|spacemacro(|noexpand )*
|def|bslashmacro(|noexpand\)*
)*
|endgroup
%    \end{macrocode}
% \end{macro}
%
% \begin{macro}{\macro@@code}[1]{verbatim macro code}
%    \begin{macrocode}
\def\macro@@code#1{%
  {\ydoc@defspecialmacros
  \xdef\themacrocode{#1}}%
  \PrintMacroCode
  \end{macrocode}%
}
%    \end{macrocode}
% \end{macro}
%
% \begin{macro}{\linenumberbox}
%    \begin{macrocode}
\def\newlinemacro{\\\null}
\def\spacemacro{\ }
\def\bslashmacro{\char92}
\def\lastlinemacro{}
\def\firstlinemacro{\linenumberbox}
\def\newlinemacro{\\\linenumberbox}
\newcounter{linenumber}
\def\linenumberbox{%
  \hbox to 1.25em{}%
  \llap{%
    \stepcounter{linenumber}%
    {\footnotesize\color{gray}\thelinenumber~}%
  }%
}
%    \end{macrocode}
% \end{macro}
%
% \begin{macro}{\PrintMacroCode}
%    \begin{macrocode}
\def\PrintMacroCode{%
  \begingroup
  \ttfamily
  \noindent\themacrocode
  \endgroup
}
%    \end{macrocode}
% \end{macro}
%
% \begin{macro}{\PrintMacroCode}
%    \begin{macrocode}
\RequirePackage{listings}
%    \end{macrocode}
%    \begin{macrocode}
\def\PrintMacroCode{%
  \begingroup
  \let\firstlinemacro\empty
  \let\lastlinemacro\empty
  \def\newlinemacro{^^J}%
  \let\bslashmacro\bslash
  \let\spacemacro\space
  \immediate\openout\ydocwrite=\ydocfname\relax
  \immediate\write\ydocwrite{\themacrocode}%
  \immediate\closeout\ydocwrite
  \@nameuse{ydoc@countbslashes}%
  \ydoclistingssettings
  \let\input\@input
  \lstinputlisting{\ydocfname}%
  \endgroup
}
%    \end{macrocode}
% \end{macro}
%
%    \begin{macrocode}
\lstdefinestyle{ydoccode}{%
    language=[latex]tex,basicstyle=\ttfamily,
    numbers=left,numberstyle=\tiny\color{gray},firstnumber=last,
    breaklines,prebreak={\mbox{\tiny$\swarrow$}},
    commentstyle=\color{black!60},
}%
%    \end{macrocode}
%
% \begin{macro}{\ydoclistingssettings}
%    \begin{macrocode}
\def\ydoclistingssettings{%
    \lstset{style=ydoccode}%
}
%    \end{macrocode}
% \end{macro}
%
% \begin{macro}{\macro@impl@args}[1]{number of macro arguments}
%    \begin{macrocode}
\def\macro@impl@args[#1]{%
  \begingroup
  \parindent=10pt\relax
  \let\macro@impl@argcnt\@tempcnta
  \let\macro@impl@curarg\@tempcntb
  \macro@impl@argcnt=#1\relax
  \macro@impl@curarg=0\relax
  \ifnum\macro@impl@curarg<\macro@impl@argcnt\relax
    \expandafter\macro@impl@arg
  \else
    \expandafter\macro@impl@endargs
  \fi
}
%    \end{macrocode}
% \end{macro}
%
% \begin{macro}{\macro@impl@endargs}
%    \begin{macrocode}
\def\macro@impl@endargs{%
  \endgroup
  \unskip\par\noindent\ignorespaces
}
%    \end{macrocode}
% \end{macro}
%
% \begin{macro}{\macro@impl@argline}[2]{argument number}{argument description}
%    \begin{macrocode}
\def\macro@impl@argline#1#2{%
  \par{\texttt{\##1}:~#2\strut}%
}
%    \end{macrocode}
% \end{macro}
%
% \begin{macro}{\macro@impl@arg}[1]{argument description}
%    \begin{macrocode}
\def\macro@impl@arg#1{%
  \advance\macro@impl@curarg by\@ne\relax
  \macro@impl@argline{\the\macro@impl@curarg}{#1}%
  \ifnum\macro@impl@curarg<\macro@impl@argcnt\relax
    \expandafter\macro@impl@arg
  \else
    \expandafter\macro@impl@endargs
  \fi
}
%    \end{macrocode}
% \end{macro}
%
% \begin{environment}{macro}[1]{implemented macro}
%    \begin{macrocode}
\def\macro#1{%
  \PrintMacroImpl{#1}%
  \@ifnextchar[%]
    {\macro@impl@args}%
    {}%
}
\def\endmacro{}
%    \end{macrocode}
% \end{environment}
%
% \begin{environment}{key}[2]{key family}{key name}
%    \begin{macrocode}
\def\key#1#2{%
  \PrintMacroImpl{KV@#1@#2}%
  \@ifnextchar[%]
    {\macro@impl@args}%
    {}%
}
\def\endkey{}
%    \end{macrocode}
% \end{environment}
%
% \begin{environment}{environment}[1]{environment name}
%    \begin{macrocode}
\def\environment#1{%
  \PrintEnvImplName{#1}%
  \@ifnextchar[%]
    {\macro@impl@args}%
    {}%
}
\def\endenvironment{}
%    \end{macrocode}
% \end{environment}
%
%
% \begin{environment}{style}[1]{style name}
%    \begin{macrocode}
\def\style#1{%
  \PrintStyleImplName{#1}%
  \@ifnextchar[%]
    {\macro@impl@args}%
    {}%
}
\def\endstyle{}
\def\PrintStyleImplName{\PrintEnvImplName}
%    \end{macrocode}
% \end{environment}
%
% \begin{macro}{\PrintMacroImpl}[1]{macro (token)}
%    \begin{macrocode}
\def\PrintMacroImpl#1{%
  \par\bigskip\noindent
  \Needspace*{3\baselineskip}%
  \hbox{%
    \edef\name{\expandafter\@gobble\string#1}%
    \global\@namedef{href@impl@\name}{}%
    \immediate\write\@mainaux{%
      \global\noexpand\@namedef{href@impl@\name}{}%
    }%
    \raisebox{4ex}[4ex]{\hypertarget{impl:\name}{}}%
    \hspace*{\descindent}\fbox{%
      \hspace*{\descsep}%
      \@ifundefined{href@desc@\name}{}{\hyperlink{desc:\name}}%
      {\PrintMacroImplName{#1}}%
      \hspace*{\descsep}%
    }%
  }%
  \par\medskip\noindent
}
%    \end{macrocode}
% \end{macro}
%
% \begin{macro}{\PrintMacroImplName}[1]{macro (token)}
%    \begin{macrocode}
\def\PrintMacroImplName#1{%
  \implstyle{\string#1\strut}%
}
%    \end{macrocode}
% \end{macro}
%
% \begin{macro}{\PrintEnvImplName}[1]{environment name}
% test
%    \begin{macrocode}
\def\PrintEnvImplName#1{%
  \par\bigskip\noindent
  \hbox{\hspace*{\descindent}\fbox{{\implstyle{#1}}}}%
  \par\medskip
}
%    \end{macrocode}
% \end{macro}
%
% \begin{macro}{\implstyle}
%    \begin{macrocode}
\def\implstyle{\ttfamily\bfseries\color{macroimpl}}
%    \end{macrocode}
% \end{macro}
%
% \begin{macro}{\bslash}
% Defines an expandable backslash with catcode 12: `\texttt{\bslash}$_{12}$'.
% The |\@firstofone| trick is used to read the |\gdef\bslash| code before changing the catcode.
%    \begin{macrocode}
{%
\@firstofone
%    \end{macrocode}
% \end{macro}
% \iffalse
%</ydoc-code.sty>
% \fi
%
% \iffalse
%<*ydoc-doc.sty>
% \fi
% \subsection{Provide \pkg{doc} macros}
%    \begin{macrocode}
\NeedsTeXFormat{LaTeX2e}[1999/12/01]
\ProvidesPackage{ydoc-doc}[%
%<!DATE>
%<!VERSION>
%<*DRIVER>
    2099/01/01 develop
%</DRIVER>
    ydoc package to provide 'doc' macros]
%    \end{macrocode}
%
% \begin{macro}{\ydoc@countbslashes}
% Reads the macro code into a temp box. The backslashes are defined to increase a counter.
%    \begin{macrocode}
\newcount\ydoc@bslashcnt
\def\ydoc@countbslashes{%
  \begingroup
    \let\firstlinemacro\empty
    \let\lastlinemacro\empty
    \let\newlinemacro\empty
    \let\spacemacro\empty
    \def\bslashmacro{\global\advance\ydoc@bslashcnt by\@ne}%
    \setbox\@tempboxa\hbox{\themacrocode}%
  \endgroup
}
%    \end{macrocode}
% \end{macro}
%
% \begin{macro}{\CheckSum}
%    \begin{macrocode}
\def\CheckSum#1{%
  \gdef\ydoc@checksum{#1}%
}
\let\ydoc@checksum\m@ne
%    \end{macrocode}
% \end{macro}
%
% \begin{macro}{\AlsoImplementation}
% \begin{macro}{\OnlyDescription}
% \begin{macro}{\StopEventually}
% \begin{macro}{\Finale}
% The first two macros modify the \cs{StopEventually} macro which either
% stores its argument in \cs{Final} or executes it itself.
%    \begin{macrocode}
\def\AlsoImplementation{%
  \gdef\StopEventually##1{%
    \@bsphack
    \gdef\Finale{##1\ydoc@checkchecksum}%
    \@esphack
  }%
}
\AlsoImplementation
\def\OnlyDescription{%
  \@bsphack
  \long\gdef\StopEventually##1{##1\endinput}%
  \@esphack
}
\let\Finale\relax
%    \end{macrocode}
% \end{macro}
% \end{macro}
% \end{macro}
% \end{macro}
%
% \begin{macro}{\MakePercentComment}
% \begin{macro}{\MakePercentIgnore}
%    \begin{macrocode}
\def\MakePercentIgnore{\catcode`\%9\relax}
\def\MakePercentComment{\catcode`\%14\relax}
%    \end{macrocode}
% \end{macro}
% \end{macro}
%
% \begin{macro}{\DocInput}
%    \begin{macrocode}
\def\DocInput#1{\MakePercentIgnore\input{#1}\MakePercentComment}
%    \end{macrocode}
% \end{macro}
%
% \begin{macro}{\CharacterTable}
%    \begin{macrocode}
\providecommand*\CharacterTable{%
    \begingroup
    \CharTableChanges
    \@CharacterTable
}
\def\@CharacterTable#1{%
    \def\ydoc@used@CharacterTable{#1}%
    \@onelevel@sanitize\ydoc@used@CharacterTable
    \ifx\ydoc@used@CharacterTable\ydoc@correct@CharacterTable
           \typeout{***************************}%
           \typeout{* Character table correct *}%
           \typeout{***************************}%
    \else
         \PackageError{ydoc}{Character table corrupted}
                           {\the\wrong@table}
         \show\ydoc@used@CharacterTable
         \show\ydoc@correct@CharacterTable
    \fi
    \endgroup
}
\newhelp\wrong@table{Some of the ASCII characters are corrupted.^^J
            I now \string\show\space you both tables for comparison.}
\newcommand*\CharTableChanges{}
%    \end{macrocode}
% \end{macro}
%
% \begin{macro}{\ydoc@correct@CharacterTable}
%    \begin{macrocode}
\def\ydoc@correct@CharacterTable
  {Upper-case    \A\B\C\D\E\F\G\H\I\J\K\L\M\N\O\P\Q\R\S\T\U\V\W\X\Y\Z
   Lower-case    \a\b\c\d\e\f\g\h\i\j\k\l\m\n\o\p\q\r\s\t\u\v\w\x\y\z
   Digits        \0\1\2\3\4\5\6\7\8\9
   Exclamation   \!     Double quote  \"     Hash (number) \#
   Dollar        \$     Percent       \%     Ampersand     \&
   Acute accent  \'     Left paren    \(     Right paren   \)
   Asterisk      \*     Plus          \+     Comma         \,
   Minus         \-     Point         \.     Solidus       \/
   Colon         \:     Semicolon     \;     Less than     \<
   Equals        \=     Greater than  \>     Question mark \?
   Commercial at \@     Left bracket  \[     Backslash     \\
   Right bracket \]     Circumflex    \^     Underscore    \_
   Grave accent  \`     Left brace    \{     Vertical bar  \|
   Right brace   \}     Tilde         \~}
\@onelevel@sanitize\ydoc@correct@CharacterTable
%
%    \end{macrocode}
% \end{macro}
%
% \begin{macro}{\DoNotIndex}
%    \begin{macrocode}
\providecommand*\DoNotIndex[1]{%
  \PackageWarning{ydoc}{Ignoring DoNotIndex - not implemented yet!}{}{}%
}
%    \end{macrocode}
% \end{macro}
%
% \begin{macro}{\changes}
%    \begin{macrocode}
\providecommand*\changes[3]{%
  \PackageWarning{ydoc}{Ignoring changes - not implemented yet!}{}{}%
}
%    \end{macrocode}
% \end{macro}
%
% \begin{macro}{\RecordChanges}
%    \begin{macrocode}
\providecommand*\RecordChanges{%
  \PackageWarning{ydoc}{List of changes not implemented yet!}{}{}%
}
%    \end{macrocode}
% \end{macro}
%
% \begin{macro}{\PrintChanges}
%    \begin{macrocode}
\providecommand*\PrintChanges{%
  \PackageWarning{ydoc}{List of changes not implemented yet!}{}{}%
}
%    \end{macrocode}
% \end{macro}
%
% \begin{macro}{\PrintIndex}
%    \begin{macrocode}
\providecommand*\PrintIndex{%
  \PackageWarning{ydoc}{Code index not implemented yet!}{}{}%
}
%    \end{macrocode}
% \end{macro}
%
% \begin{macro}{\CodelineIndex}
%    \begin{macrocode}
\providecommand*\CodelineIndex{%
  \PackageWarning{ydoc}{Code line index not implemented yet!}{}{}%
}
%    \end{macrocode}
% \end{macro}
%
% \begin{macro}{\EnableCrossrefs}
%    \begin{macrocode}
\providecommand*\EnableCrossrefs{%
  \PackageWarning{ydoc}{Cross references not implemented yet!}{}{}%
}
%    \end{macrocode}
% \end{macro}
%
% \begin{macro}{\GetFileInfo}
% Current implementation taken from \pkg{doc} package.
%    \begin{macrocode}
\providecommand*\GetFileInfo[1]{%
  \def\filename{#1}%
  \def\@tempb##1 ##2 ##3\relax##4\relax{%
    \def\filedate{##1}%
    \def\fileversion{##2}%
    \def\fileinfo{##3}}%
  \edef\@tempa{\csname ver@#1\endcsname}%
  \expandafter\@tempb\@tempa\relax? ? \relax\relax
}
%    \end{macrocode}
% \end{macro}
%
% \begin{macro}{\ydoc@checkchecksum}
%    \begin{macrocode}
\def\ydoc@checkchecksum{%
  \ifnum\ydoc@checksum=\m@ne
    \message{^^J}%
    \message{**********************^^J}%
    \message{* No checksum found! *^^J}%
    \message{**********************^^J}%
    \GenericWarning{No checksum found}{Correct checksum is \the\ydoc@bslashcnt^^J}{}{}%
  \else
  \ifnum\ydoc@checksum=\z@
    \message{^^J}%
    \message{*********************^^J}%
    \message{* Checksum disabled *^^J}%
    \message{*********************^^J}%
    \GenericWarning{Checksum disabled}{Correct checksum is \the\ydoc@bslashcnt^^J}{}{}%
  \else
  \ifnum\ydoc@checksum=\ydoc@bslashcnt
    \message{^^J}%
    \message{*******************^^J}%
    \message{* Checksum passed *^^J}%
    \message{*******************^^J}%
  \else
    \message{^^J}%
    \message{****************************^^J}%
    \message{* Checksum wrong (\ydoc@checksum<>\the\ydoc@bslashcnt) ^^J}%
    \message{****************************^^J}%
    \GenericError{Checksum wrong}{Correct checksum is \the\ydoc@bslashcnt^^J}{}{}%
  \fi
  \fi
  \fi
}
%    \end{macrocode}
% \end{macro}
%
%    \begin{macrocode}
\RequirePackage{shortvrb}
\AtBeginDocument{\MakeShortVerb{\|}}
%    \end{macrocode}
%
%    \begin{macrocode}
\RequirePackage{url}

\def\package{\def\@package}
\package{\jobname}

\def\bundle{\def\@bundle}
\let\@bundle\@empty


\def\ctanlocation{\def\@ctanlocation##1}
\ctanlocation{http://www.ctan.org/pkg/#1}

\date{Version \fileversion\space -- \filedate}

\def\@homepage{%
    \begingroup
    \edef\@tempa{%
        \endgroup
        CTAN:
        \noexpand\url
        {\@ctanlocation{\ifx\@bundle\@empty\@package\else\@bundle\fi}}%
    }%
    \@tempa
}

\let\@repository\@empty
\protected\def\repository{\urldef\@repository\url}
\protected\def\homepage{\urldef\@homepage\url}
\protected\def\email{\hyper@normalise\email@}
\def\email@#1{\def\@plainemail{#1}\def\@email{\hyper@linkurl{\Hurl{#1}}{mailto:#1}}}

\let\@email\empty

\let\@plainemail\empty
\title{The \texorpdfstring{\pkgtitle{\@package}}{\@package} Package}
\def\@bundlesubtitle{Part of the \texorpdfstring{\pkgtitle{\@bundle}}{\@bundle} bundle}

\protected\def\pkgtitle#1{%
    \texorpdfstring{\textsf{#1}}{#1}%
}



\def\@maketitle{%
    \newpage
    \null\vskip 2em
    \begin{center}%
        \let\footnote\thanks
        {\LARGE \@title \par }\vskip 1.5em%
        \ifx\@bundle\@empty\else
        {\large \@bundlesubtitle \par }\vskip 1.5em%
        \fi
        {\large \lineskip .5em%
        \begin{tabular}[t]{c}%
            \@author
        \end{tabular}%
        \par}%
        \ifx\@plainemail\empty\else
            {\large \lineskip .5em%
            \begin{tabular}[t]{c}%
                \@email
            \end{tabular}%
            \par}%
        \fi
        \vskip 1em
        {\large \lineskip .5em%
        \begin{tabular}[t]{c}%
            \@homepage
        \end{tabular}%
        \par}%
        \vskip 1em
        \ifx\@repository\@empty\else
            {\large \lineskip .5em%
            \begin{tabular}[t]{c}%
                VC: \@repository
            \end{tabular}%
            \par}%
        \fi
        \vskip 1em
        {\large \@date }%
    \end{center}%
    \par\vskip 1.5em
    \aftergroup\ydocpdfsettings
}

\ifpdf
\def\ydocpdfsettings{%
    \hypersetup{%
        pdfauthor   = {\@author\space<\@plainemail>},
        pdftitle    = {\@title},
        pdfsubject  = {Documentation of LaTeX package \@package},
        pdfkeywords = {\@package, LaTeX, TeX}
    }%
}
\else
\let\ydocpdfsettings\empty
\fi

\let\orig@maketitle\maketitle
\def\maketitle{%
    \ydocpdfsettings
    \orig@maketitle
    \let\orig@maketitle\relax
}
%    \end{macrocode}
% \iffalse
%</ydoc-doc.sty>
% \fi
%
% \iffalse
%<*ydoc-desc.sty>
% \fi
% \subsection{Description Macros and Environments}
%
%    \begin{macrocode}
\NeedsTeXFormat{LaTeX2e}[1999/12/01]
\ProvidesPackage{ydoc-desc}[%
%<!DATE>
%<!VERSION>
%<*DRIVER>
    2099/01/01 develop
%</DRIVER>
    ydoc package to describe macros, environments, options etc.]
%    \end{macrocode}
%
%    \begin{macrocode}
\IfFileExists{needspace.sty}{%
    \RequirePackage{needspace}
}{%
    \def\Needspace{\@ifstar\@gobble\@gobble}
}
%    \end{macrocode}
%
% The short verbatim code is required for the similar macros provided here.
%    \begin{macrocode}
\RequirePackage{shortvrb}
%    \end{macrocode}
%
% The \pkg{etoolbox} package is used mainly for |\newrobustcmd|.
%    \begin{macrocode}
\RequirePackage{etoolbox}
%    \end{macrocode}
%
% \subsubsection{Color and style definitions}
%    \begin{macrocode}
\RequirePackage{xcolor}
%    \end{macrocode}
% Define special no-op `none' color which does not change the color.
% This is not yet tested and may break output files, but seems to work fine with PDF.
%    \begin{macrocode}
\expandafter\def\csname\string\color@none\endcsname{%
    \xcolor@ {}{}{}{}
}
%    \end{macrocode}
%    \begin{macrocode}
\definecolor{macrodesc}{rgb}{0,0.2,0.6}
\definecolor{keydesc}{rgb}{0,0.4,0.9}
\definecolor{macroimpl}{rgb}{0,0.1,0.3}
\definecolor{meta}{rgb}{0,0.25,0.75}
\definecolor{scriptcolor}{rgb}{0.2,0.6,0.2}
\definecolor{optioncolor}{rgb}{0.3.0.2,0}
\colorlet{optional}{black!65!white}
\colorlet{metaoptional}{optional!50!meta}
\providecolor{urlcolor}{named}{blue}
\providecolor{linkcolor}{named}{blue}
\providecolor{filecolor}{named}{blue}
\providecolor{citecolor}{named}{blue}
\providecolor{anchorcolor}{named}{blue}
\providecolor{menucolor}{named}{blue}
\providecolor{runcolor}{named}{blue}

%    \end{macrocode}
%
%    \begin{macrocode}
\RequirePackage{hyperref}
\hypersetup{%
    colorlinks=true,
    pdfborder=0 0 0,
    pdfborderstyle={},
    urlcolor=urlcolor,
    linkcolor=linkcolor,
    filecolor=filecolor,
    citecolor=citecolor,
    anchorcolor=anchorcolor,
    menucolor=menucolor,
    runcolor=runcolor,
}

%    \end{macrocode}
%
% \subsubsection{Text Formatting Macros}
%
% \begin{macro}{\meta}
% Prints \meta{meta text}.
%    \begin{macrocode}
\newrobustcmd*\meta[1]{%
  {\metastyle{%
  \ensuremath\langle
    #1\/%
  \ensuremath\rangle
  }}%
}
%    \end{macrocode}
% \end{macro}
%
% \begin{macro}{\marg}
% Sets style and adds braces.
% The text is formatted as separate set of macro arguments.
%    \begin{macrocode}
\newrobustcmd*{\marg}[1]{%
  {\margstyle{%
    {\ttfamily\braceleft}%
    \meta{#1}%
    {\ttfamily\braceright}%
  }}%
}
%    \end{macrocode}
% \end{macro}
%
% \begin{macro}{\oarg}
% Sets style and adds brackets.
% The text is formatted as separate set of macro arguments.
%    \begin{macrocode}
\newrobustcmd*{\oarg}[1]{%
  {\oargstyle{%
    {\ttfamily[}%
    \meta{#1}%
    {\ttfamily]}%
  }}%
}
%    \end{macrocode}
% \end{macro}
%
% \begin{macro}{\parg}
% Sets style and adds parentheses.
%    \begin{macrocode}
\newrobustcmd*{\parg}[1]{%
  {\pargstyle{%
    {\ttfamily(}%
    \meta{#1}%
    {\ttfamily)}%
  }}%
}
%    \end{macrocode}
% \end{macro}
%
% \begin{macro}{\aarg}
% Sets style and adds angles.
%    \begin{macrocode}
\newrobustcmd*{\aarg}[1]{%
  {\aargstyle{%
    {\ttfamily<}%
    \meta{#1}%
    {\ttfamily>}%
  }}%
}
%    \end{macrocode}
% \end{macro}
%
% \begin{macro}{\sarg}
% Prints star with given style.
%    \begin{macrocode}
\newrobustcmd*{\sarg}{{\sargstyle{*}}}
%    \end{macrocode}
% \end{macro}
%
% \begin{macro}{\pkg}
% \begin{macro}{\cls}
% \begin{macro}{\lib}
% \begin{macro}{\env}
% \begin{macro}{\opt}
% \begin{macro}{\file}
%    \begin{macrocode}
\newrobustcmd*\pkg[1]{{\pkgstyle{#1}}}
\newrobustcmd*\cls[1]{{\clsstyle{#1}}}
\newrobustcmd*\lib[1]{{\libstyle{#1}}}
\newrobustcmd*\env[1]{{\envstyle{#1}}}

\newrobustcmd*\opt{\@ifstar\ys@opt\y@opt}
\def\y@opt#1{{\optstyle{#1}}}
\def\ys@opt#1{{\optstyle{#1}}\optpar{#1}}
\newrobustcmd*\optpar[1]{\marginpar{\hbox to \marginparwidth{\hss\y@opt{#1}}}}

\newrobustcmd*\file[1]{{\filestyle{#1}}}
\newcommand*\pkgstyle[1]{\texttt{\textcolor{pkg}{#1}}}
\newcommand*\clsstyle[1]{\texttt{\textcolor{cls}{#1}}}
\newcommand*\libstyle[1]{\texttt{\textcolor{lib}{#1}}}
\newcommand*\envstyle[1]{\texttt{\textcolor{env}{#1}}}
\newcommand*\optstyle[1]{\textsf{\textcolor{opt}{#1}}}
\newcommand*\filestyle[1]{\texttt{\textcolor{file}{#1}}}
\colorlet{cls}{none}
\colorlet{lib}{none}
\colorlet{env}{none}
\colorlet{file}{none}
\colorlet{pkg}{none}
\definecolor{opt}{rgb}{0.5,0.16666,0}
%    \end{macrocode}
% \end{macro}
% \end{macro}
% \end{macro}
% \end{macro}
% \end{macro}
% \end{macro}
%
% \begin{macro}{\cs}
% \begin{macro}{\cmd}
%    \begin{macrocode}
\newrobustcmd*\cs[1]{\texttt{\textbackslash #1}}
\newrobustcmd*\cmd[1]{\texttt{{\escapechar=92\string#1}}}
%    \end{macrocode}
% \end{macro}
% \end{macro}
%
% \begin{macro}{\Key}
%    \begin{macrocode}
\newrobustcmd*\Key[1]{\PrintKeyName{#1}\MacroArgs}
%    \end{macrocode}
% \end{macro}
%
% \subsubsection{Text Formatting Styles}
%
% \begin{macro}{\macrodescstyle}
% Style of described macro names.
%    \begin{macrocode}
\def\macrodescstyle{\ttfamily\bfseries\color{macrodesc}}
%    \end{macrocode}
% \end{macro}
%
% \begin{macro}{\macrodescstyle}
% Style of described macro names.
%    \begin{macrocode}
\def\keydescstyle{\ttfamily\bfseries\color{keydesc}}
%    \end{macrocode}
% \end{macro}
%
% \begin{macro}{\macroargsstyle}
% Default style for macro arguments (e.g.\ \cs{MacroArgs}).
%    \begin{macrocode}
\def\macroargsstyle{\ttfamily}
%    \end{macrocode}
% \end{macro}
%
% \begin{macro}{\envcodestyle}
% Default style for code body content in described environments.
%    \begin{macrocode}
\def\envcodestyle{\ttfamily}
%    \end{macrocode}
% \end{macro}
%
% \begin{macro}{\verbstyle}
% Style for verbatim text inside macro argument list.
%    \begin{macrocode}
\def\verbstyle{\verbatim@font}
%    \end{macrocode}
% \end{macro}
%
% \begin{macro}{\metastyle}
% Meta text style. Because \cs{macroargsstyle} might be also active a |\normalfont| reset
% the font.
%    \begin{macrocode}
\def\metastyle{\normalfont\itshape\color{meta}}
%    \end{macrocode}
% \end{macro}
%
% \begin{macro}{\margstyle}
% Style for \cs{marg}.
%    \begin{macrocode}
\def\margstyle{}
%    \end{macrocode}
% \end{macro}
%
% \begin{macro}{\Optional}
% \begin{macro}{\optional}
% \begin{macro}{\optionalstyle}
%    \begin{macrocode}
\protected\def\Optional{\optionalon\optional}
\def\optionalstyle{\blendcolors*{!60!white}\color{black!75}}
%    \end{macrocode}
% \end{macro}
% \end{macro}
% \end{macro}
%
% \begin{macro}{\optionalon}
% \begin{macro}{\optionaloff}
%    \begin{macrocode}
\def\optionalon{\protected\def\optional{\optionalstyle}}
\def\optionaloff{\let\optional\relax}
\optionalon
%    \end{macrocode}
% \end{macro}
% \end{macro}
%
% \begin{macro}{\oargstyle}
% Style for \cs{oarg}. A special color is set to show the `optional' status.
%    \begin{macrocode}
\def\oargstyle{\optional}
%    \end{macrocode}
% \end{macro}
%
% \begin{macro}{\pargstyle}
% Style for \cs{parg}.
%    \begin{macrocode}
\def\pargstyle{}
%    \end{macrocode}
% \end{macro}
%
% \begin{macro}{\aargstyle}
% Style for \cs{aarg}.
%    \begin{macrocode}
\def\aargstyle{}
%    \end{macrocode}
% \end{macro}
%
% \begin{macro}{\sargstyle}
% Style for \cs{sarg}. A special color is set to show the `optional' status.
%    \begin{macrocode}
\def\sargstyle{\ttfamily\color{optional}}
%    \end{macrocode}
% \end{macro}
%
% \subsubsection{Dimension Registers}
%
% \begin{macro}{\descindent}
%    \begin{macrocode}
\newdimen\descindent
\descindent=-\parindent
%    \end{macrocode}
% \end{macro}
%
% \begin{macro}{\beforedescskip}
%    \begin{macrocode}
\newdimen\beforedescskip
\beforedescskip=\bigskipamount
%    \end{macrocode}
% \end{macro}
%
% \begin{macro}{\afterdescskip}
%    \begin{macrocode}
\newdimen\afterdescskip
\afterdescskip=\medskipamount
%    \end{macrocode}
% \end{macro}
%
% \begin{macro}{\descsep}
% Set to |1em| in |tt| font.
%    \begin{macrocode}
\newdimen\descsep
\begingroup
\ttfamily
\global\descsep=1em\relax
\endgroup
%    \end{macrocode}
% \end{macro}
%
%
% \subsubsection{Macro Argument Reading Mechanism}
%
% \begin{macro}{\read@Macro@arg}
% Reads next token and calls second macro.
%    \begin{macrocode}
\def\read@Macro@arg{%
  \futurelet\@let@token\handle@Macro@arg
}
%    \end{macrocode}
% \end{macro}
%
% \begin{macro}{\AlsoMacro}
% Reads argument while |@| is a letter,
% prints the macro name and reads further arguments.
%    \begin{macrocode}
\newcommand*\AlsoMacro{%
  \begingroup\makeatletter
  \AlsoMacro@
}
\def\AlsoMacro@#1{%
  \endgroup
%<*DEBUG>
   %\typeout{DEBUG: Macro: \string#1}%
%</DEBUG>
  \PrintMacroName{#1}%
  \read@Macro@arg
}
%    \end{macrocode}
% \end{macro}
%
% \begin{macro}{\ydoc@short@AlsoMacro}
% Makes |&| an alias for |\AlsoMacro|.
%    \begin{macrocode}
\begingroup
\catcode`\|\active
\gdef\ydoc@short@AlsoMacro{%
  \catcode`\|\active
  \let|\AlsoMacro
}
\endgroup
%    \end{macrocode}
% \end{macro}
%
%
% \begin{macro}{\ydoc@macrocatcodes}
% Sets the catcodes inside for |read@Macro@arg| material.
%    \begin{macrocode}
\def\ydoc@macrocatcodes{%
  \ydoc@short@AlsoMacro
  \@makeother\'%
  \@makeother\!%
  \@makeother\[%
  \@makeother\]%
  \@makeother\(%
  \@makeother\)%
}
%    \end{macrocode}
% \end{macro}
%
%
% \begin{macro}{\handle@Macro@arg}
% Checks if next token is the begin of a valid macro argument and calls
% the appropriate read macro or the end macro otherwise.
%    \begin{macrocode}
\def\handle@Macro@arg{%
  \expandafter\let\expandafter\handler\csname handle@Macro@token@\meaning\@let@token\endcsname
  \ifx\handler\relax
    \def\handler{\ifhmode\unskip\fi\end@Macro@args}%
%<*DEBUG>
 %  \typeout{DEBUG: Stopped at: \expandafter\meaning\csname @let@token\endcsname}%
 %  \typeout{}%
 %\else
 %\expandafter\ifx\csname @let@token\endcsname\AlsoMacro
 %  \typeout{DEBUG: TOKEN: \string\AlsoMacro}%
 %\else
 %  \typeout{DEBUG: TOKEN: \expandafter\meaning\csname @let@token\endcsname}%
 %\fi
%</DEBUG>
  \fi
  \handler
}
\def\define@Macro@handler{%
    \begingroup
    \ydoc@macrocatcodes
    \define@Macro@handler@
}
\def\define@Macro@handler@#1{%
    \endgroup
    \@namedef{handle@Macro@token@\meaning#1}%
}
%    \end{macrocode}
% \end{macro}
%
% \begin{macro}{\end@Macro@args}
% Closes box as calls hook. Might be locally redefined by some
% macros calling \cs{read@Macro@arg}.
%    \begin{macrocode}
\def\end@Macro@args{%
  \y@egroup
  \after@Macro@args
}
%    \end{macrocode}
% \end{macro}
%
% \begin{macro}{\after@Macro@args}
% Hook to add additional commands in certain situations.
%    \begin{macrocode}
\def\after@Macro@args{%
}
%    \end{macrocode}
% \end{macro}
%
% \subsubsection*{Macro argument reading macros}
% This macros read the macro arguments and call the appropriate format macros.
%
% \begin{macro}{\read@Macro@marg}
%    \begin{macrocode}
\define@Macro@handler{\bgroup}{%
    \begingroup
        \afterassignment\read@Macro@marg@
        \let\@let@token=%
}
\def\read@Macro@marg@{%
        \bgroup
        \margstyle{}%
        \let\end@Macro@args\empty%
        {\ttfamily\braceleft}%
        \aftergroup\read@Macro@marg@@
        \read@Macro@arg
}
\def\read@Macro@marg@@{%
        {\ttfamily\braceright}%
    \endgroup
    \read@Macro@arg
}
%    \end{macrocode}
% \end{macro}
%
% \begin{macro}{\read@Macro@oarg}
%    \begin{macrocode}
\define@Macro@handler{[}[{%
    \begingroup
        \let\read@Macro@oarg@end\read@Macro@oarg@@end
        \let\end@Macro@args\read@Macro@oarg@end
        \oargstyle{}%
        {\ttfamily[}%]
        \read@Macro@arg
}
\define@Macro@handler{]}{%
    \read@Macro@oarg@end
}
\def\read@Macro@oarg@@end#1]{%
        #1%
        {\ttfamily]}%
    \endgroup
    \read@Macro@arg
}
\def\read@Macro@oarg@end{\end@Macro@args}
\let\read@Macro@aarg@end\read@Macro@oarg@end
\let\read@Macro@parg@end\read@Macro@oarg@end
%    \end{macrocode}
% \end{macro}
%
% \begin{macro}{\read@Macro@parg}
%    \begin{macrocode}
\define@Macro@handler{(}({%
    \begingroup
        \let\read@Macro@parg@end\read@Macro@parg@@end
        \let\end@Macro@args\read@Macro@parg@end
        \pargstyle{}%
        {\ttfamily(}%)
        \read@Macro@arg
}
\define@Macro@handler{)}{%
    \read@Macro@parg@end
}
\def\read@Macro@parg@@end#1){%
        #1%
        {\ttfamily)}%
    \endgroup
    \read@Macro@arg
}
%    \end{macrocode}
% \end{macro}
%
% \begin{macro}{\read@Macro@aarg}
%    \begin{macrocode}
\def\read@Macro@aarg<{%
    \begingroup
        \let\read@Macro@aarg@end\read@Macro@aarg@@end
        \let\end@Macro@args\read@Macro@aarg@end
        \aargstyle{}%
        {\ttfamily<}%
        \read@Macro@arg
}
\define@Macro@handler{>}{%
    \read@Macro@aarg@end
}
\def\read@Macro@aarg@@end#1>>{%
        #1%
        {\ttfamily>}%
    \endgroup
    \read@Macro@arg
}
%    \end{macrocode}
% \end{macro}
%
% \begin{macro}{\read@Macro@angle}
%    \begin{macrocode}
\define@Macro@handler{<}<{%
  \futurelet\@let@token\read@Macro@angle@
}
%    \end{macrocode}
% \end{macro}
%
% \begin{macro}{\read@Macro@angle@}
%    \begin{macrocode}
\def\read@Macro@angle@{%
  \ifx\@let@token<%
    \expandafter\read@Macro@aarg
  \else
    \expandafter\read@Macro@meta
  \fi
}
%    \end{macrocode}
% \end{macro}
%
% \begin{macro}{\read@Macro@meta}
%    \begin{macrocode}
\def\read@Macro@meta#1>{%
  \meta{#1}\read@Macro@arg
}
%    \end{macrocode}
% \end{macro}
%
% \begin{macro}{\read@Macro@sarg}
%    \begin{macrocode}
\define@Macro@handler**{%
  \sarg\read@Macro@arg
}
%    \end{macrocode}
% \end{macro}
%
% Allows `|=|' to be used directly without switching to verbatim
% mode. This is especially useful for keys.
%    \begin{macrocode}
\define@Macro@handler{=}={%
  =\read@Macro@arg
}
%    \end{macrocode}
%
% \begin{macro}{\read@Macro@verb}
% Sets up verbatim mode calls second macro.
%    \begin{macrocode}
\define@Macro@handler{'}'{%
  \begingroup
  \let\do\@makeother
  \dospecials
  \@noligs
  \@makeother\'%
  \obeyspaces
  \read@Macro@verb@
}
%    \end{macrocode}
% \end{macro}
%
% \begin{macro}{\read@Macro@verb@}
% Closes verbatim mode and formats text.
% If |#1| is empty (|''|) than a single |'| is printed.
%    \begin{macrocode}
\begingroup
\@makeother\'%
\gdef\read@Macro@verb@#1'{%
  \endgroup
  \ifx\relax#1\relax
    {\verbstyle{\string'}}%
  \else
    {%
     \frenchspacing
     \@noligs\verbstyle{#1}}%
  \fi
  \read@Macro@arg
}
\endgroup
%    \end{macrocode}
% \end{macro}
%
% \begin{macro}{\read@Macro@cmds}
% Simply executes given code.
%    \begin{macrocode}
\define@Macro@handler!!#1!{%
  #1\relax
  \read@Macro@arg
}
%    \end{macrocode}
% \end{macro}
%
% \begin{macro}{\read@Macro@rmspace}
% Removes space. The |\@firstofone| is used to preserve the space in the macro definition.
%    \begin{macrocode}
\define@Macro@handler{\@sptoken} {%
  \read@Macro@arg
}
%    \end{macrocode}
% \end{macro}
%
% \begin{macro}{\read@Macro@addtoken}
% Takes token over from input to output `stream'. This is used for |\space| and |~|.
%    \begin{macrocode}
\define@Macro@handler{~}#1{%
  #1\read@Macro@arg
}
\AtBeginDocument{%
\define@Macro@handler{~}#1{%
  #1\read@Macro@arg
}
}
\define@Macro@handler{\space}#1{%
  #1\read@Macro@arg
}
%    \end{macrocode}
% \end{macro}
%
% \subsubsection{Description Macros}
% \subsubsection*{For Macros}
%
% \begin{macro}{\DescribeMacro}
%    \begin{macrocode}
\@ifundefined{DescribeMacro}{}{%
  \PackageInfo{ydoc-desc}{Redefining \string\DescribeMacro}{}%
}
%    \end{macrocode}
% A \cs{DescribeMacro} places itself in a \env{DescribeMacros} environment.
% Multiple \cs{DescribeMacro} macros will stack themselves inside this environment.
% For this to work \cs{DescribeMacros} is locally defined to |\y@egroup| to close the
% |\hbox| from the previous \cs{DescribeMacro}.
%    \begin{macrocode}
\def\DescribeMacro{%
  \DescribeMacros
  \let\DescribeMacros\y@egroup
  \optionalon
  \def\after@Macro@args{\endDescribeMacros}%
  \begingroup\makeatletter
  \Describe@Macro
}
%    \end{macrocode}
% \end{macro}
%
% \begin{macro}{\DescribeScript}
%    \begin{macrocode}
\def\DescribeScript#1{%
  \DescribeMacros
  \let\DescribeMacros\y@egroup
  \optionalon
  \def\after@Macro@args{\endDescribeMacros}%
  \hbox\y@bgroup
  \texttt{#1}%
  \ydoc@macrocatcodes
  \macroargsstyle
  \read@Macro@arg~%
}
%    \end{macrocode}
% \end{macro}
%
% \begin{macro}{\DescribeKey}
%    \begin{macrocode}
\def\DescribeKey{%
  \DescribeKeys
  \let\DescribeKeys\y@egroup
  \optionalon
  \def\after@Macro@args{\endDescribeKeys}%
  \begingroup\makeatletter
  \Describe@Macro
}
%    \end{macrocode}
% \end{macro}
%
% \begin{macro}{\Describe@Macro}
%    \begin{macrocode}
\def\Describe@Macro#1{%
  \endgroup
  \edef\name{\expandafter\@gobble\string#1}%
  \global\@namedef{href@desc@\name}{}%
  \immediate\write\@mainaux{%
    \global\noexpand\@namedef{href@desc@\name}{}%
  }%
  \hbox\y@bgroup
  \@ifundefined{href@impl@\name}{}{\hyperlink{impl:\name}}%
  {%
  \hbox{\vbox to 0pt{\vss\hbox{\raisebox{4ex}{\hypertarget{desc:\name}{}}}}%
  \PrintMacroName{#1}}%
  }%
  \ydoc@macrocatcodes
  \macroargsstyle
  \read@Macro@arg
}
%    \end{macrocode}
% \end{macro}
%
% \begin{macro}{\MakeShortMacroArgs}
% Defines the given character as short version for |\MacroArgs|.
% It is first define to be a short verbatim character to take
% advantage of the house-keeping (save \& restore of the original
% catcode and definition) of |shortvrb|.
%
% The starred version define the character to act like |\Macro| instead.
%    \begin{macrocode}
\newcommand*\MakeShortMacroArgs{%
  \@ifstar
    {\@MakeShortMacroArgs\Macro}%
    {\@MakeShortMacroArgs\MacroArgs}%
}
\def\@MakeShortMacroArgs#1#2{%
  \MakeShortVerb{#2}
  \catcode`#2\active
  \begingroup
  \catcode`\~\active
  \lccode`\~`#2\relax
  \lowercase{\endgroup\gdef~{\bgroup\let~\egroup#1}}%
}
%    \end{macrocode}
% \end{macro}
%
% \begin{macro}{\DeleteShortMacroArgs}
%    \begin{macrocode}
\newcommand*\DeleteShortMacroArgs[1]{%
  \DeleteShortVerb{#1}%
}
%    \end{macrocode}
% \end{macro}
%
% \begin{macro}{\Macro}
% Simply uses the two macros below.
%    \begin{macrocode}
\newcommand*\Macro{\MacroArgs\AlsoMacro}
%    \end{macrocode}
% \end{macro}
%
% \begin{macro}{\@Macro}
% Alternative definition of \cs{Macro} inside \env{DescribeMacros} environments.
%    \begin{macrocode}
\def\@Macro{%
  \begingroup\makeatletter
  \Describe@Macro
}
%    \end{macrocode}
% \end{macro}
%
%    \begin{macrocode}
\define@Macro@handler\AlsoMacro{}
\define@Macro@handler\DescribeMacro{}
\define@Macro@handler\DescribeKey{}
\define@Macro@handler\DescribeScript{}
%    \end{macrocode}
%
% \begin{macro}{\MacroArgs}
% Uses the normal macro argument reading mechanism from \cs{DescribeMacro}.
% Instead of a box a simple group is added.
%    \begin{macrocode}
\newcommand*\MacroArgs{%
  \begingroup
  \def\end@Macro@args{\endgroup\xspace}%
  \ydoc@macrocatcodes
  \macroargsstyle
%<*DEBUG>
   %\typeout{}%
   %\typeout{DEBUG: Start MacroArgs}%
%</DEBUG>
  \read@Macro@arg
}
\RequirePackage{xspace}
%    \end{macrocode}
% \end{macro}
%
% \begin{macro}{\DescribeMacros}
%    \begin{macrocode}
\def\DescribeMacros{%
  \begingroup
  \let\Macro\@Macro
  \parindent=0pt\relax
  \setbox\descbox\vbox\y@bgroup
}
%    \end{macrocode}
% \end{macro}
%
% \begin{macro}{\endDescribeMacros}
%    \begin{macrocode}
\def\endDescribeMacros{%
  \y@egroup
  \PrintMacros
  \endgroup
}
%    \end{macrocode}
% \end{macro}
%
% \begin{macro}{\DescribeKeys}
%    \begin{macrocode}
\def\DescribeKeys{%
  \begingroup
  \let\PrintMacroName\PrintKeyName
  \let\Key\@Macro
  \parindent=0pt\relax
  \setbox\descbox\vbox\y@bgroup
}
%    \end{macrocode}
% \end{macro}
%
% \begin{macro}{\endDescribeKeys}
%    \begin{macrocode}
\def\endDescribeKeys{%
  \y@egroup
  \PrintKeys
  \endgroup
}
\def\PrintKeys{\PrintMacros}
%    \end{macrocode}
% \end{macro}
%
% \begin{macro}{\DescribeMacrosTabcolsep}
%    \begin{macrocode}
\def\DescribeMacrosTabcolsep{\tabcolsep}
%    \end{macrocode}
% \end{macro}
%
% \begin{macro}{\DescribeMacrosTab}
%    \begin{macrocode}
\def\DescribeMacrosTab{%
  \DescribeMacros
  \hbox\y@bgroup
  \tabcolsep=\DescribeMacrosTabcolsep\relax
  \DescribeMacrosTab@
}
\def\DescribeMacrosTab@#1{\tabular{@{}#1@{}}}
%    \end{macrocode}
% \end{macro}
%
% \begin{macro}{\endDescribeMacrosTab}
%    \begin{macrocode}
\def\endDescribeMacrosTab{%
  \endtabular\y@egroup
  \endDescribeMacros
}
%    \end{macrocode}
% \end{macro}
%
% \subsubsection*{For Lengths}
%
% \begin{macro}{\DescribeLength}
%    \begin{macrocode}
\newcommand*\DescribeLength{%
  \begingroup
  \let\DescribeLength\Describe@Length
  \setbox\descbox\hbox\y@bgroup
    \tabular{@{}l@{\hspace{2em}}l@{}}%
    \Describe@Length
}
%    \end{macrocode}
% \end{macro}
%
% \begin{macro}{\Describe@Length}
%    \begin{macrocode}
\newcommand*\Describe@Length[2]{%
  \PrintLengthName{#1}&
  (Default: {\macroargsstyle#2\unskip})%
  \@ifnextchar\DescribeLength
    {\\}%
    {%
      \endtabular
      \y@egroup
      \PrintLength
      \endgroup
    }%
}
%    \end{macrocode}
% \end{macro}
%
% \subsubsection*{For Environments}
%
% \begin{macro}{\DescribeEnv}
%    \begin{macrocode}
\@ifundefined{DescribeEnv}{}{%
  \PackageInfo{ydoc-desc}{Redefining \string\DescribeEnv}{}%
}
\let\DescribeEnv\relax
%    \end{macrocode}
%    \begin{macrocode}
\newcommand*\DescribeEnv[2][]{%
  \begingroup
  \def\DescribeEnv@name{#2}%
  \let\\\DescribeEnv@newline
%    \end{macrocode}
%
% Sets after-macro-arguments hook.
% First checks if the environment or macro version was used.
% The environment starts a new line only if the next token isn't |\end|,
% which is taken as end of the environment.
%    \begin{macrocode}
  \ifx\@currenvir\DescribeEnv@string
    \def\after@Macro@args{%
      \let\after@Macro@args\empty
      \setbox\@tempboxa\hbox\y@bgroup
      \@ifnextchar\end{}%
        {\DescribeEnv@newline}%
      #1%
    }%
%    \end{macrocode}
% The macro version adds the optional argument as content line if given.
%    \begin{macrocode}
  \else
    \ifx\relax#1\relax
      \def\after@Macro@args{%
        \y@bgroup
        \endDescribeEnv
      }%
    \else
      \def\after@Macro@args{%
        \setbox\@tempboxa\hbox\y@bgroup
        \DescribeEnv@newline\MacroArgs#1%
        \endDescribeEnv
      }%
    \fi
  \fi
%    \end{macrocode}
% Start |\vbox| and adds first line.
%    \begin{macrocode}
  \setbox\descbox\vbox\y@bgroup
  \envcodestyle
  \let\PrintEnv\PrintSubEnv
  \hbox\y@bgroup
  \PrintEnvName{\begin}{\DescribeEnv@name}%
  \ydoc@macrocatcodes
  \macroargsstyle
  \read@Macro@arg
}
%    \end{macrocode}
% \end{macro}
%
% \begin{macro}{\DescribeEnv@newline}
% Closes existing and starts a new horizontal box representing a indented line.
% The optional argument allows to add extra space between lines like the normal |\\|.
% Negative values are not supported.
%    \begin{macrocode}
\newcommand*\DescribeEnv@newline[1][0pt]{%
  \strut\y@egroup
  {\vskip#1}%
  \hbox\y@bgroup\strut
  \hspace*{\descsep}%
  \ignorespaces
}%
%    \end{macrocode}
% \end{macro}
%
% \begin{macro}{\DescribeEnv@string}
% Holds the environment name for comparison.
%    \begin{macrocode}
\def\DescribeEnv@string{DescribeEnv}
%    \end{macrocode}
% \end{macro}
%
% \begin{macro}{\descbox}
% Save box to store description content.
%    \begin{macrocode}
\newbox\descbox
%    \end{macrocode}
% \end{macro}
%
% \begin{macro}{\endDescribeEnv}
%    \begin{macrocode}
\def\endDescribeEnv{%
  \y@egroup
  \begingroup
  \setbox\@tempboxa\lastbox
  \ifcase0%
    \ifdim\wd\@tempboxa>\descsep1\fi
    \ifdim\ht\@tempboxa>\ht\strutbox1\fi
    \ifdim\dp\@tempboxa>\dp\strutbox1\fi
  \else
    \box\@tempboxa
  \fi
  \endgroup
  \hbox\y@bgroup
    \PrintEnvName{\end}{\DescribeEnv@name}
  \y@egroup
  \y@egroup
  \PrintEnv
  \endgroup
}
%    \end{macrocode}
% \end{macro}
%
%
% \subsubsection{Print Macros}
%
% \begin{macro}{\PrintMacroName}
% Formats macro name. The backslash is forced to |tt| font.
%    \begin{macrocode}
\def\PrintMacroName#1{%
  {\macrodescstyle{\strut
   \texttt{\char92}%
   \escapechar\m@ne
   \string#1\strut}}%
}
%    \end{macrocode}
% \end{macro}
%
% \begin{macro}{\PrintKeyName}
% Formats macro name. The backslash is forced to |tt| font.
%    \begin{macrocode}
\def\PrintKeyName#1{%
  {\keydescstyle{\strut
   #1\strut}}%
}
%    \end{macrocode}
% \end{macro}
%
% \begin{macro}{\PrintLengthName}
% Formats length register name.
%    \begin{macrocode}
\let\PrintLengthName\PrintMacroName
%    \end{macrocode}
% \end{macro}
%
%
% \begin{macro}{\PrintEnvName}
% |#1| = `|\begin|' or `|\end|', |#2| = env name.
%    \begin{macrocode}
\def\PrintEnvName#1#2{%
  \strut
  \string#1\braceleft
  {\macrodescstyle#2\strut}%
  \braceright
}
%    \end{macrocode}
% \end{macro}
%
% \begin{macro}{\PrintMacros}
% Prints macros described using \cs{DescribeMacros}.
% The actual content was stored inside \cs{descbox}.
% If it is wider than the line width it is centered.
%    \begin{macrocode}
\def\PrintMacros{%
  \par\vspace\beforedescskip
  \begingroup
  \sbox\@tempboxa{\descframe{\usebox{\descbox}}}%
  \Needspace*{\dimexpr\ht\@tempboxa+3\baselineskip\relax}%
  \par\noindent
  \ifdim\wd\@tempboxa>\dimexpr\linewidth-2\descindent\relax
    \makebox[\linewidth][c]{\usebox\@tempboxa}%
  \else
    \hspace*{\descindent}%
    \usebox\@tempboxa
  \fi
  \endgroup
  \par
  \vspace\afterdescskip
  \par\noindent
}
\def\descframe#1{%
    \fbox{\hspace*{\descsep}#1\hspace*{\descsep}}%
}
%    \end{macrocode}
% \end{macro}
%
% \begin{macro}{\PrintLength}
% Prints lengths registers described using one or multiple \cs{DescribeLength}.
%    \begin{macrocode}
\let\PrintLength\PrintMacros
%    \end{macrocode}
% \end{macro}
%
% \begin{macro}{\PrintEnv}
% Prints \env{DescribeEnv} environments.
% The actual content was stored inside \cs{descbox}.
%    \begin{macrocode}
\let\PrintEnv\PrintMacros
%    \end{macrocode}
% \end{macro}
%
% \begin{macro}{\PrintSubEnv}
% Prints sub environments, i.e.\ \env{DescribeEnv} environments
% inside the body of another \env{DescribeEnv}.
% The actual content was stored inside \cs{descbox}.
%    \begin{macrocode}
\def\PrintSubEnv{%
  \hbox{\hbox{\usebox{\descbox}}}%
}
%    \end{macrocode}
% \end{macro}
%
%
% \subsubsection{Special Character Macros}
%
% \begin{macro}{\bslash}
% Defines an expandable backslash with catcode 12: `\texttt{\bslash}$_{12}$'.
% The |\@firstofone| trick is used to read the |\gdef\bslash| code before changing the catcode.
%    \begin{macrocode}
{%
\@firstofone
%    \end{macrocode}
% \end{macro}
%
% \begin{macro}{\percent}
% Defines an expandable percent character with catcode 12: `\percent$_{12}$'.
%    \begin{macrocode}
\begingroup
\catcode`\%=12
\gdef\percent{%}
\endgroup
%    \end{macrocode}
% \end{macro}
%
% \begin{macro}{\braceleft}
% \begin{macro}{\braceright}
% Defines expandable left and right braces with catcode 12: `\texttt{\braceleft}$_{12}$' `\texttt{\braceright}$_{12}$'.
%    \begin{macrocode}
\begingroup
\catcode`\<=1
\catcode`\>=2
\catcode`\{=12
\catcode`\}=12
\gdef\braceleft <{>
\gdef\braceright<}>
\endgroup
%    \end{macrocode}
% \end{macro}
% \end{macro}
%
% \subsubsection{Other Macros}
%
% \begin{macro}{\y@bgroup}
% \begin{macro}{\y@egroup}
% These macros are used to begin and end |\vbox|/|\hbox|-es.
%    \begin{macrocode}
\def\y@bgroup{\bgroup\color@setgroup}
\def\y@egroup{\color@endgroup\egroup}
%    \end{macrocode}
% \end{macro}
% \end{macro}
%
% \begin{macro}{\codeline}
%    \begin{macrocode}
\newcommand*{\codeline}[1][c]{%
    \codelinebefore
    \hbox to \hsize\bgroup
    \ifx i#1\hspace*{\leftmargin}\else
        \ifx l#1\else\hss\fi
    \fi
    \let\xspace\relax
    \hbox\bgroup
    \aftergroup\codeline@end
    \aftergroup#1%
    \afterassignment\MacroArgs
    \let\@let@token=%
}
\def\codeline@end#1{%
    \ifx r#1\else\hss\fi
    \egroup
    \codelineafter
}
\newcommand*\codelinebefore{\par\smallskip\noindent}
\newcommand*\codelineafter {\par\smallskip\noindent}
%    \end{macrocode}
% \end{macro}
%
% \begin{environment}{codequote}
%    \begin{macrocode}
\newenvironment{codequote}{%
    \def\\{\newline\relax\MacroArgs}%
    \par\smallskip\bgroup\leftskip=\leftmargin\rightskip=\rightmargin\noindent\MacroArgs}
    {\par\egroup\smallskip\noindent\ignorespacesafterend}
%    \end{macrocode}
% \end{environment}
%
% \begin{environment}{macroquote}
%    \begin{macrocode}
\newenvironment{macroquote}{%
    \def\\{\newline\relax\Macro}%
    \par\smallskip\bgroup\leftskip=\leftmargin\rightskip=\rightmargin\noindent\Macro}
    {\par\egroup\smallskip\noindent\ignorespacesafterend}
%    \end{macrocode}
% \end{environment}
% \iffalse
%</ydoc-desc.sty>
% \fi
%
% \iffalse
%<*ydoc-expl.sty>
% \fi
% \subsection{Include Code Examples}
%
%    \begin{macrocode}
\NeedsTeXFormat{LaTeX2e}[1999/12/01]
\ProvidesPackage{ydoc-expl}[%
%<!DATE>
%<!VERSION>
%<*DRIVER>
    2011/08/11 develop
%</DRIVER>
    ydoc package to insert live examples of LaTeX code]
%    \end{macrocode}
%
% \iffalse
%<*dontinclude>
%    \begin{macrocode}
\newcounter{example}
\def\examplename{Example}
\newenvironment{example}[1]{%
  \par\bigskip\noindent
  \centerline{\examplename~\refstepcounter{example}\theexample: #1}%
  \smallskip
  \hrule
  \smallskip
}{%
  \smallskip
  \hrule
  \par\bigskip
}
%    \end{macrocode}
%
% \begin{macro}{\PrintExample}
%    \begin{macrocode}
\def\PrintExample{%
  \usebox\examplecodebox
  \hfill
  \usebox\exampleresultbox
}
%    \end{macrocode}
% \end{macro}
%</dontinclude>
% \fi
%
%    \begin{macrocode}
\RequirePackage{listings}
\lst@RequireAspects{writefile}
\def\ydoc@exafile{\jobname.exa}
%    \end{macrocode}
%
% \begin{style}{examplecode}
%    \begin{macrocode}
\lstdefinestyle{examplecode}{%
    language=[latex]tex,
    basicstyle=\ttfamily,
    columns=fullflexible,
    numbers=left,
    firstnumber=1,
    numberstyle=\tiny\color{gray}\sffamily,
    numbersep=5pt,
    breaklines,prebreak={\mbox{\tiny$\swarrow$}},
    commentstyle=\color{black!60},
}%
%    \end{macrocode}
% \end{style}
%
% \begin{style}{exampleresult}
%    \begin{macrocode}
\lstdefinestyle{exampleresult}{%
    firstnumber=1,
    gobble=0,
    basicstyle=\ttfamily,
    columns=fullflexible,
    commentstyle=\color{black!60},
}
%    \end{macrocode}
% \end{style}
%
% \begin{style}{exampleextract}
%    \begin{macrocode}
\lstdefinestyle{exampleextract}{gobble=4}%
%    \end{macrocode}
% \end{style}
%
%    \begin{macrocode}
\newbox\examplecodebox
\newbox\exampleresultbox
%    \end{macrocode}
%
% \begin{macro}{\BoxExample}
%    \begin{macrocode}
\def\BoxExample{%
  \setbox\examplecodebox\hbox{\color@setgroup
    \lstinputlisting[style=examplecode,style=thisexampleprint]%
    {\ydoc@exafile}%
  \unskip\color@endgroup}%
  \setbox\exampleresultbox\hbox{\color@setgroup
    \lstset{style=exampleresult}%
    \@@input\ydoc@exafile\relax
  \unskip\color@endgroup}%
}
%    \end{macrocode}
% \end{macro}
%
% \begin{macro}{\PrintExample}
%    \begin{macrocode}
%<*DISABLED>
\RequirePackage{showexpl}
\def\PrintExample{%
  \begingroup
  \lstset{style=examplecode}%
  \MakePercentComment
  \LTXinputExample[varwidth]{\ydoc@exafile}%
  \endgroup
}
%</DISABLED>
%    \end{macrocode}
% \end{macro}
%
%
%
% \begin{macro}{\PrintExample}
%    \begin{macrocode}
\def\PrintExample{%
  \begingroup
  \BoxExample
  \@tempdima=\textwidth
  \advance\@tempdima by -\wd\examplecodebox\relax
  \advance\@tempdima by -\wd\exampleresultbox\relax
  \advance\@tempdima by -15pt\relax
  \ifdim\@tempdima>\bigskipamount
    \hbox to \textwidth{%
     \null\hss
     \minipage[c]{\wd\exampleresultbox}\fbox{\usebox\exampleresultbox}\endminipage
     \hfill\hfill\hskip\bigskipamount\hskip15pt\hfill\hfill
     \minipage[c]{\wd\examplecodebox}\usebox\examplecodebox\endminipage
     \hss\null
     }%
  \else
    \vbox{%
        \centerline{\fbox{\usebox\exampleresultbox}}%
        \vspace{\bigskipamount}%
        \centerline{\usebox\examplecodebox}%
    }%
  \fi
  \endgroup
}
%    \end{macrocode}
% \end{macro}
%
% \begin{environment}{examplecode}
%    \begin{macrocode}
\lstnewenvironment{examplecode}[1][]{%
  \lstdefinestyle{thisexampleprint}{#1}%
  \setbox\@tempboxa\hbox\bgroup
  \lstset{style=exampleextract,#1}%
  \lst@BeginWriteFile{\ydoc@exafile}%
}
{%
  \lst@EndWriteFile
  \egroup
  \begingroup
  \MakePercentComment
  \catcode`\^^M=5\relax
  \PrintExample
  \endgroup
}
%    \end{macrocode}
% \end{environment}
%
%    \begin{macrocode}
\RequirePackage{float}
%    \end{macrocode}
%
% \begin{environment}{example}
%    \begin{macrocode}
\floatstyle{plain}
\newfloat{example}{tbhp}{loe}
\floatname{example}{\examplename}
\def\examplename{Example}
%    \end{macrocode}
% \end{environment}
%
% \begin{environment}{exampletable}
%    \begin{macrocode}
\newenvironment{exampletable}{%
  \floatstyle{plaintop}%
  \restylefloat{example}%
  \example
}{\endexample}
%    \end{macrocode}
% \end{environment}
% \iffalse
%</ydoc-expl.sty>
% \fi
%
% \iffalse
%<*ydocincl.tex>
% \fi
%
%    \begin{macrocode}
\expandafter\ifx\csname ydocinclversion\endcsname\relax\else
    \endinput
\fi

\chardef\ydocinclversion=1

\newread\inFile
\newread\subFile
\newwrite\outFile
\newif\ifContinue
\newlinechar=`^^J

\def\makeOther#1{\catcode`#1=12\relax}

\let\inLine\relax
\let\lastLine\relax

\def\includefiles#1#2{%
    \begingroup
    \immediate\openin\inFile#1\relax
    \immediate\openout\outFile#2\relax
    \makeOther\@%
    \makeOther\ \makeOther\\\makeOther\$%
    \makeOther\#\makeOther\^\makeOther\^^K%
    \makeOther\_\makeOther\^^A\makeOther\%%
    \makeOther\~\makeOther\{\makeOther\}\makeOther\&%
    \endlinechar-1\relax
    \Continuetrue
    \loop
      \let\lastLine\inLine
      \read\inFile to\inLine
      \ifeof\inFile
        \Continuefalse
      \else
        \expandafter\checkLine\inLine\empty\empty\empty\endLine
      \fi
      \ifContinue
    \repeat
    \immediate\closein\inFile
    \immediate\closeout\outFile
    \endgroup
    \end
}

\def\copyline{%
    \immediate\write\outFile{\inLine}%
}

\chardef\percentcharnum=`\%

\begingroup
\makeOther\%\makeOther\@\relax
\gdef\SubFileOptionString\relax
\catcode`\|=0
\makeOther\ \makeOther\\|relax
|gdef|IfFalseString{% \iffalse}|relax
|gdef|FiString{% \fi}|relax
|endgroup

\def\checkLine#1#2#3#4\endLine{%
    \def\firstthree{#1#2#3}%
    \ifx\firstthree\SubFileOptionString
        \readSubFile#4\endLine
    \else
        \copyline
    \fi
}

\def\readSubFile#1>#2\endLine{%
    \immediate\openin\subFile=#1\relax
    \ifeof\subFile
        % File not found
    \else
        \message{^^JIncluding subfile '#1'^^J}%
        \immediate\write\outFile{\CommentChar<*#1>}%
        \ifx\lastLine\IfFalseString
            \immediate\write\outFile{\FiString}%
        \fi
        \copySubFile
        \ifx\lastLine\IfFalseString
            \immediate\write\outFile{\IfFalseString}%
        \fi
        \immediate\write\outFile{\CommentChar</#1>}%
    \fi
    \immediate\closein\subFile
}

\def\copySubFile{%
    \read\subFile to\subLine
    \ifeof\subFile\else
        \immediate\write\outFile{\subLine}%
        \expandafter\copySubFile
    \fi
}
%    \end{macrocode}
%
% \iffalse
%</ydocincl.tex>
% \fi
%
% \Finale
% \endinput
