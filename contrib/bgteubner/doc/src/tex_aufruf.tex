%
% bgteubner class bundle
%
% tex_aufruf.tex
% Copyright 2003--2012 Harald Harders
%
% This program may be distributed and/or modified under the
% conditions of the LaTeX Project Public License, either version 1.3
% of this license or (at your opinion) any later version.
% The latest version of this license is in
%    http://www.latex-project.org/lppl.txt
% and version 1.3 or later is part of all distributions of LaTeX
% version 1999/12/01 or later.
%
% This program consists of all files listed in manifest.txt.
% ===================================================================
%\documentclass[ngerman]{...}
% ===================================================================
\chapter{Aufruf der Programme}%
\index{Programmaufruf}%

Da der Teubner Verlag \acro{PDF}"=Dateien entgegennimmt und die
besten Resultate durch die Verwendung von \pdfLaTeX\ erzielt werden
konnten, wird vorgeschrieben, \pdfLaTeX\ zu verwenden.

Nehmen wir an, dass die Hauptdatei des Dokuments \verb|buch.tex| hei�t.
Das Programm \pdfLaTeX\ wird mit
\begin{verbatim}[\small]
pdflatex buch
\end{verbatim}
\index{Programmaufruf!pdfLaTeX@\pdfLaTeX}%
\index{pdfLaTeX@\pdfLaTeX!Programmaufruf}%
aufgerufen.
Um im Stichwortverzeichnis das richtige Layout zu erhalten, muss
Makeindex folgenderma�en aufgerufen werden:
\begin{verbatim}[\small]
makeindex -c -g -s bgteubner.ist buch
\end{verbatim}
\index{Programmaufruf!makeindex@\texttt{makeindex}}%
\index{makeindex@\texttt{makeindex}!Programmaufruf}%
Au�erdem muss das Literaturverzeichnis, das mit \BibTeX\
erstellt werden soll, mit
\begin{verbatim}[\small]
bibtex buch
\end{verbatim}
\index{Programmaufruf!bibtex@\texttt{bibtex}}%
\index{bibtex@\texttt{bibtex}!Programmaufruf}%
erzeugt werden.

\begingroup
\feinschliff{\def\Typ{\meta{T}}}{\def\Typ{\meta{Typ}}}%
  {\def\Typ{\meta{T}}}{\def\Typ{\meta{T}}}%
Glossar�hnliche Umgebungen werden ebenfalls mit dem Programm
\verb|makeindex| erzeugt. 
F�r einen Glossartypen \Typ\ lautet der Aufruf:
\begin{verbatim}[\small\makeescape\|\makebgroup\[\makeegroup\]]
makeindex -c -g -s bgteuglo.ist -t buch.glg|Typ -o buch.gls|Typ buch.glo|Typ
\end{verbatim}
\endgroup
Statt \verb|bgteuglo.ist| kann auch \verb|bgteuglochar.ist| verwendet
werden, wenn der Glossar �berschriften f�r die Buchstaben erhalten
soll.
Beispielsweise f�r einen mit dem Befehl \cs{makeglossary\marg{cmd}}
eingerichteten Glossar:
\begin{verbatim}[\small\makeescape\|\makebgroup\[\makeegroup\]]
makeindex -c -g -s bgteuglo.ist -t buch.glgcmd -o buch.glscmd buch.glocmd
\end{verbatim}

Um auf jeden Fall ein richtig formatiertes Buch mit korrekten
Querverweisen zu erhalten, muss \pdfLaTeX\ mehrmals aufgerufen werden.
Folgende Aufruf"|folge ist denkbar:
\begin{verbatim}[\small]
pdflatex buch
pdflatex buch
makeindex -c -g -s bgteuglo.ist -t buch.glgcmd -o buch.glscmd buch.glocmd
bibtex buch
pdflatex buch
pdflatex buch
makeindex -c -g -s bgteubner.ist buch
pdflatex buch
\end{verbatim}
In seltenen F�llen kann sogar noch h�ufigeres �bersetzen notwendig
sein.

Auch wenn in den meisten F�llen diese Prozedur �bertrieben ist,
sollte man sie zur Erstellung der endg�ltigen Datei, die an die
Druckerei weitergegeben wird, einmal vollst�ndig durchf�hren.
W�hrend der Erstellung des Buches ist das nat�rlich nicht notwendig.


% ===================================================================

%%% Local Variables: 
%%% mode: latex
%%% TeX-master: "bgteubner"
%%% End: 
