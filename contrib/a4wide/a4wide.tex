\documentclass[parskip=true, pagesize=auto]{scrartcl}

\usepackage{fixltx2e}
\usepackage{lmodern}
\usepackage[T1]{fontenc}
\usepackage[utf8]{inputenc}
\usepackage{booktabs}
\usepackage{siunitx}

\addtokomafont{disposition}{\rmfamily}
\sisetup{obeybold}

\newunit{\point}{pt}
\newunit{\inch}{in}

\title{The \textsf{a4wide} package}
\author{Jean-François Lamy \and Alexander Holt}
\date{August 1994}


\begin{document}

\maketitle

The \textsf{a4wide} package redefines the margins so that they are more in line with what we are used to see.
It was written by Jean-François Lamy in July 1986 and minimally modified for \LaTeXe\ by Alexander Holt, August 1994.

The package is released under the LPPL, version 1, or (at your option)
any later version.

Depending on the base font size, it modifies the margins as follows:
%
\begin{center}
  \begin{tabular}{@{}S[tabformat=2]S[tabformat=2]S[tabformat=1.3]S[tabformat=1.3]S[tabformat=1.2]@{}}
    \toprule
    {Base font}          & {Number of lines} & {Width of}               & {Width of left and}         & {Width of}                    \\
    {size (\si{\point})} & {in page body}    & {text line (\si{\inch})} & {right margin (\si{\inch})} & {marginal notes (\si{\inch})} \\
    \midrule
    10                   & 53                & 5.875                    & 0.25                        & 0.75                          \\
    11                   & 46                & 6.125                    & 0.125                       & 0.75                          \\
    12                   & 42                & 6.375                    & 0                           & 0.75                          \\
    \bottomrule
  \end{tabular}
\end{center}

\end{document}
