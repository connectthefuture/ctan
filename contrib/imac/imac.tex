%%%%%%%%%%%%%%%%%%%%%%%%%%% ASME.tex %%%%%%%%%%%%%%%%%%%%%%%%%%%%%%%%%
%%% This is a template for the IMAC (International Modal Analysis 
%%% Conference) style file imac.sty.  No subsubsection format is 
%%% defined for IMAC, so I used this command for the nomenclature command.
%%% With this you should also find imac.bst which applies the closest
%%% thing I can determine to be the ``Standard'' for IMAC.
%%% The 9 point text size has already been defined in the style file imac.sty.
%%%
%%% Version 1.01, 5/18/99
%%%
\documentclass[twocolumn]{article}
\usepackage{imac}
\usepackage{helvet}   % Uncomment this line if you have ps helvetica 
%                       fonts available on your system. Things will work better.
%                       Trust me.
%                       Comment it out if you do not.


% You also need to have the packages: cite, citesort, and ifthen 
% (ifthen comes with the standard distribution of LaTeX).
% They can be obtained from any CTAN server. Probably where you found 
% this.

\newcommand{\degrees}{$^{\circ}$~}
                         
\begin{document}

%%% Don't want date printed
\date{}

\title{\Large\textbf{IMAC version 1.01: A \LaTeX\ PACKAGE FOR WRITING
PAPERS IN\\
INTERNATIONAL MODAL ANALYSIS CONFERENCE (IMAC) FORMAT}}


\author{\vspace{.25in}\\
\textbf{Joseph C. Slater}\\
\\
 {\normalsize Department of Mechanical and Materials Engineering} \\
 {\normalsize Wright State University}\\
 {\normalsize Colonel Glenn Highway}\\
 {\normalsize Dayton, OH 45435}\\
}
\maketitle

%%%%%%%%%%%%%%%%%%%%%%%%%%%%%%%%%%%%%%%%%%%%%%%%%%%%%%%%%

\begin{abstract}
Just like in any other \LaTeX\  document, you begin the abstract, make 
the title, and define the author as shown in the example. Not that 
IMAC requires that the author names be in bold, so you have to do this 
yourself. In addition, I haven't yet hacked the style definition for 
the title. Maybe I will later, but that's a minor inconvenience. 
\end{abstract}


\subsubsection*{Nomenclature}
\begin{tabular}{lll}
\matr{M} &&  matrix\\

\dmat{M}&& diagonal matrix\\

\vect{\Phi} && vector \\

\elem{M_{11}}&& single element of \matr{M}\\

\pnorm{x} && p-norm of \vect{x}\\

$\begin{bmatrix}\matr{M}_{11}&:&\matr{M}_{12}\\
	\hdotsfor[2]{1}&:&\hdotsfor[2]{1}\\
	\matr{M}_{21}&:&\matr{M}_{22}
\end{bmatrix}$ 
&& partitioned matrix\\

\end{tabular}

\section{Changes}
The only change since 1.00 is the inclusion of the GNU public license.

\section{Typing Your Document}
\indent First thing to note is that the IMAC style indents the first 
line of each paragraph, including the paragraph immediately following 
a section header. You can override the \LaTeX\ default of not 
indenting by using the \indent command each time after a 
\verb*a\sectiona command or by using the package \verb*zindentfirstz 
which can be downloaded from any CTAN location. 

The second thing which you may need to do is obtain and install the 
packages: \verb*zcitez, \verb*zcitesortz, \verb*zifthenz, and \verb*zamsmathz. Things will also 
work out a little nicer if you have the \verb*zhelvetz package, but 
it is not necessary. If you do have the \verb*zhelvetz package, you 
should uncomment the line \verb*z\usepackage{helvet}z above.

The PostScript\textregistered\ file \verb*zimac.psz can be viewed in 
\verb*zGhostScriptz or printed to a ps printer for comparison to the results 
you obtain from \LaTeX ing this document. 

A few macros have been defined below for conforming to the 
IMAC\cite{ewins}
notation convention (See Table \ref{dm}). If you use them, you can simply redefine the 
macros according to the journal requirements when the time comes for submission.
I've defined only those that I thought were either difficult or do not 
conform to normal textbook standards. You may find it useful to define 
your own macros at the end of the file \verb*zimac.styz, but please 
make sure that you don't delete them the next time you update!

In Table \ref{dm} you'll also see some code for making a partitioned 
matrix. Sorry, but I don't know how to turn this into an 
environment. If you figure it out, let me know and I'll incorporate it. 
The colons form the horizontal delimiters and the \verb*z\hdotsforz 
command forms the vertical delimiters. The first argument represents 
a spacing of the dots, and the second required argument is the number 
of columns that the dotted line should span. 

\begin{table*}
	\begin{tabular}{lll}
	\matr{M} &\verb*c\matr{M}c&  matrix\\
	
	\dmat{M}&\verb*c\dmat{M}c& diagonal matrix\\
	
	\vect{\Phi} & \verb*a\vect{\Phi}a & vector \\
	
	\elem{M_{11}}&\verb*z\elem{M_{11}}z& single element of \matr{M}\\
	
	\pnorm{x} &\verb*a\pnorm{x}a & p-norm of \vect{x}\\
%%%%  Look between these lines for the partitioned matrix code
	$\begin{bmatrix}\matr{M}_{11}&:&\matr{M}_{12}\\
		\hdotsfor[2]{1}&:&\hdotsfor[2]{1}\\
		\matr{M}_{21}&:&\matr{M}_{22}
	\end{bmatrix}$ 
%%%%  Look between these lines for the partitioned matrix code
%%%%  The stuff below this is for displaying the code. Ignore it.
	&\begin{minipage}{3.5in}
	{\verb*z\begin{bmatrix}z\\\verb*z\matr{M}_{11}&:&\matr{M}_{12}\\z\\
\verb*z\hdotsfor[2]{1}&:&\hdotsfor[2]{1}\\z\\
\verb*z\matr{M}_{21}&:&\matr{M}_{22}z\\\verb*z\end{bmatrix}z}
\end{minipage}
	& partitioned matrix\\
	\end{tabular}
	\caption{\label{dm}Defined macros}

\end{table*} 

Please contact me at \verb*zjslater@cs.wright.eduz if you find bugs in this. 
I'll do my best to fix 
them in s timely fashion. Please don't contact me with respect to general \LaTeX\ 
questions. I don't have the time for that kind of inundation. 
For help with \LaTeX\ please consult Lamport\cite{lamport}, Goossens 
et al\cite{goossens}, and/or Kopka and Daly\cite{kopka}. Additional 
resources are available through the \TeX\ newsgroup \verb*zcomp.text.texz, 
the 
CTAN archives at \verb*zhttp://www.ucc.ie/cgi-bin/ctanz, and the \TeX 
users group (TUG) home page (\verb*zhttp://www.tug.org/z).

From here on is just some examples to give you some 
continuing examples of how to do things. The table would have been 
better off if I have use the \verb*zhhlinez package.


%%%%%%%%%%%%%%%%%%%%%%%%%%%%%%%%%%%%%%%%%%%%%%%%%%%%%%%%%

\section{Here is a section}

\indent In bladed disk assemblies, the disk acts as a coupling device between the
blades.  As the stiffness of the disk increases, blade coupling
decreases. It has been shown that weak interblade coupling leads to
high levels of mode localization when blades are 
mistuned.  


\begin{figure}
\begin{center}
\fbox{There once was a figure here}
\end{center}
\caption{\label{undeformed}Here is a figure.}
\end{figure}

\subsection{This is a subsection}

\indent All models used in this study were variations of the symmetric,
constant stiffness system referred to as the baseline model (Figure
\ref{undeformed}).  Yada yada.



\begin{table}
\begin{center}
\begin{tabular}{|c|c|c|c|}
\hline
    & Random 1      &     Random 2  &    Random 3\\
\cline{2-4}
\raisebox{2ex}{Blade Number}  & $10^{-3}$ Kg   &    $10^{-3}$ Kg &    $10^{-3}$  Kg\\
\hline
\hline

1   &    0.0168  & 0.3343  & 0.3413\\
2    &   0.0260  & 0.2867  & 0.4431\\
3    &   0.2579  & 0.4529  & 0.3710\\

\hline

\end{tabular}
\end{center}
\caption{\label{random} Simple table.}
\end{table}

... and that's it! If you can't get it to work, I can be reached at 
\verb*zjslater@cs.wright.eduz.




\section*{Acknowledgments}
Note that for the Acknowledgments you need to use the 
\verb*z\sectionz command in the starred form to avoid getting the 
section number.

Thanks to Leslie Lamport\cite{lamport}, Goossens, Mittelbach, and 
Samarin\cite{goossens} and all others  who've built \LaTeX\ into what it is 
today.

\section*{Copyright}


The IMAC Package for generating International Modal Analysis Conference 
formatted papers in \LaTeX. 
Copyright (C) 1998 Joseph C. Slater

This program is free software; you can redistribute it and/or
modify it under the terms of the GNU General Public License
as published by the Free Software Foundation; either version 2
of the License, or (at your option) any later version.

This program is distributed in the hope that it will be useful,
but WITHOUT ANY WARRANTY; without even the implied warranty of
MERCHANTABILITY or FITNESS FOR A PARTICULAR PURPOSE.  See the
GNU General Public License for more details.

You should have received a copy of the GNU General Public License
along with this program; if not, write to the Free Software
Foundation, Inc., 59 Temple Place - Suite 330, Boston, MA  02111-1307, USA.




\bibliographystyle{imac}

\bibliography{imac}

\end{document}
