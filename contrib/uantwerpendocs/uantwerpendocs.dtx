% \iffalse meta-comment
%
% Copyright (C) 2013-2016 by Walter Daems <walter.daems@uantwerpen.be>
%
% This work may be distributed and/or modified under the conditions of
% the LaTeX Project Public License, either version 1.3 of this license
% or (at your option) any later version.  The latest version of this
% license is in:
% 
%    http://www.latex-project.org/lppl.txt
% 
% and version 1.3 or later is part of all distributions of LaTeX
% version 2005/12/01 or later.
%
% This work has the LPPL maintenance status `maintained'.
% 
% The Current Maintainer of this work is Walter Daems.
%
% This work consists of the files listed in the file manifest.txt.
%
% \fi
%
% \iffalse
%<*driver>
\ProvidesFile{uantwerpendocs.dtx}
%</driver>
%<ct|mt>\NeedsTeXFormat{LaTeX2e}[1999/12/01]
%<ct>\ProvidesClass{uantwerpencoursetext}
%<mt>\ProvidesClass{uantwerpenmasterthesis}
%<le>\ProvidesClass{uantwerpenletter}
%<ct|mt>    [2016/05/01 v1.7 .dtx skeleton file]
%
\def\fileversion{1.8}
\def\filedate{2017/01/08}
%<*driver>
\documentclass{ltxdoc}
\usepackage{makeidx}
\usepackage{alltt}
\usepackage{booktabs}
\IfFileExists{tocbibind.sty}{\usepackage{tocbibind}}{}
\IfFileExists{hyperref.sty}{\usepackage[bookmarksopen]{hyperref}}{}
\EnableCrossrefs
\CodelineIndex
\RecordChanges
\begin{document}
  \DocInput{uantwerpendocs.dtx}
\end{document}
%</driver>
% \fi
%
% \CheckSum{0}
%
% \CharacterTable
%  {Upper-case    \A\B\C\D\E\F\G\H\I\J\K\L\M\N\O\P\Q\R\S\T\U\V\W\X\Y\Z
%   Lower-case    \a\b\c\d\e\f\g\h\i\j\k\l\m\n\o\p\q\r\s\t\u\v\w\x\y\z
%   Digits        \0\1\2\3\4\5\6\7\8\9
%   Exclamation   \!     Double quote  \"     Hash (number) \#
%   Dollar        \$     Percent       \%     Ampersand     \&
%   Acute accent  \'     Left paren    \(     Right paren   \)
%   Asterisk      \*     Plus          \+     Comma         \,
%   Minus         \-     Point         \.     Solidus       \/
%   Colon         \:     Semicolon     \;     Less than     \<
%   Equals        \=     Greater than  \>     Question mark \?
%   Commercial at \@     Left bracket  \[     Backslash     \\
%   Right bracket \]     Circumflex    \^     Underscore    \_
%   Grave accent  \`     Left brace    \{     Vertical bar  \|
%   Right brace   \}     Tilde         \~}
%
%
% \changes{v1.0}{2013/05/11}{\@ Consolidated uacoursetext class and
% uamasterthesis class}
% \changes{v1.1}{2013/05/29}{\@ Small bugfixes and 'filled' option}
% \changes{v1.2}{2014/08/22}{\@ Added lmodern package and increased
% headheight to 13.7pt to please Fancyhdr}
% \changes{v1.3}{2015/12/31}{\@ Minor bugfixes, abondonment of a few
% packages, font freedom}
% \changes{v1.4}{2016/01/07}{\@ Implemented uantwerpenletter class}
% \changes{v1.5}{2016/01/11}{\@ Replaced bottom arcs in footer of
% letter by official PDF versions}
% \changes{v1.6}{2016/02/04}{\@ Added diploma codes for Faculty of
% Applied Economics}
% \changes{v1.7}{2016/05/01}{\@ Added babel translations of elements
% of master's thesis title page}
% \changes{v1.8}{2018/01/08}{\@ Corrected minor typographic mistakes,
% added signature and solved problems with shell escape}
%
% \DoNotIndex{\newcommand,\newenvironment}
% \setlength{\parindent}{0em}
% \addtolength{\parskip}{0.5\baselineskip}
%
% \title{The |uantwerpendocs| classes\thanks{This document
% corresponds to \texttt{uantwerpendocs}~\fileversion, dated
% \filedate.}~\thanks{Thanks to Paul Levrie for testing and proofreading.}}
% \author{Walter Daems (|walter.daems@uantwerpen.be|)} 
% \date{\filedate}
%
% \maketitle
%
% \section{Introduction}
%
% This package implements the house style of Universiteit Antwerpen
% for course texts, master's theses and letters.
% Using these class files will make it easy for you to make and keep
% your course texts and master's theses compliant to this version and
% future versions of the UAntwerpen house style.
%
% If you think
% \begin{itemize}
%   \item there's an error in compliancy w.r.t. the house style,
%   \item there's a feature missing in this class file,
%   \item there's a bug in this class file,
% \end{itemize}
% please, contact us through e-mail (|walter.daems@uantwerpen.be|).
% We'll provide you with an answer
% and if (and as soon as) possible with a solution to the problem
% you spotted.
%
% Do you like these class files? You're welcome to send us beer, wine,
% or just kind words.
%
% \section{Synopsis}
% The |uantwerpencoursetext| and |uantwerpenmasterthesis| classes are
% an extension of the standard \LaTeX{} |book| class.  They are
% intended 
% to be used for writing course texts and master's theses. They
% provides a title page that is compliant to the UAntwerpen house
% style, and they also typeset the rest of your document
% appropriately. 
%
% The |uantwerpenletter| class is derived from the standard \LaTeX{}
% |letter| class. It is intended to be used for writing business
% letters. It is compliant to the house style and allows for using
% windowed envelopes of the DL format, with right-aligned window.
%
% They require (and use) the following packages:
% \begin{itemize}
%   \item the |ifthen| package
%   \item the |ifmtarg| package
%   \item the |geometry| package
%   \item teh |atbegshi| package
%   \item the |hyperref| package
%   \item the |graphicx| package
%   \item the |background| package
%   \item the |color| package
%   \item the |tikz| package
%   \item the |fancyhdr| package
%   \item the |pst-barcode| package
%   \item the |auto-pst-pdf| package
% \end{itemize}
% and optionally
% \begin{itemize}
%   \item the |varioref| package.
% \end{itemize}
% So make sure these packages are available to your
% \LaTeX{} compiler.
%
% \section{Portability}
% This class file should be ready to use with all common \LaTeX{}
% compilers (PDF\LaTeX{}, \LaTeX{}, Xe\LaTeX{},\ldots) from the major
% \TeX{}-distributions (TeTeX, TexLive, MikTeX). If you experience
% problems, please inform the authors.
%
% \section{Usage}
%
% \subsection{Basic Usage}
%
% Use the templates provided below. Remember to \LaTeX{} your source
% file twice in order to have the title and final page correctly
% aligned.
%
% \subsubsection{\texttt{uantwerpencoursetext} class}
% Use the following harness for your \LaTeX{} course text:
% \begin{verbatim}
% \documentclass[a4paper]{uantwerpencoursetext}
%
% \usepackage{<include any packages you require here>}
%
% \facultyacronym{<put your faculty's acronym here}
%
% \title{<put your title here>}
% \subtitle{<put your subtitle here>}
% \author{<put your name here>}
%
% \courseversion{<put a version identifier here>}
% \versionyear{<the publication date of the course here>}
% 
% \lecturer{<person teaching the course>}
% \programme{<descriptor of first programme>}
% \course{<course code>}{<name of the course>}% 
%
% \academicyear{<XXXX-YYYY>}
%
% \begin{document}
%
%   \maketitle
%
%   % put your LaTeX code here
%
%   \finalpage
%
% \end{document}
% \end{verbatim}
%
% The available faculty acronyms are listed in a table on page
% \pageref{acronyms}. 
%
% \subsubsection{\texttt{uantwerpenmasterthesis} class}
% Use the following harness for your \LaTeX{} master's thesis:
% \begin{verbatim}
% \documentclass[a4paper]{uantwerpenmasterthesis}
%
% \usepackage{<include any packages you require here>}
%
% \facultyacronym{<put your faculty's acronym here>}
%
% \title{<put your title here>}
% \author{<put your name here>}
% \supervisori{<put the first supervisor's name(s) here}
% \supervisorii<put the first supervisor's name(s) here}
% \supervisoriii{<put the first supervisor's name(s) here}
% \supervisoriv{<put the first supervisor's name(s) here}
%
% % classmarker
% \academicyear{<XXXX-YYYY>}
%
% \begin{document}
%
%   \maketitle
%
%   % put your LaTeX code here
%
%   \finalpage
%
% \end{document}
% \end{verbatim}
%
% The available faculty acronyms are listed in a table on page
% \pageref{acronyms}. 
% 
% \subsubsection{\texttt{uantwerpenletter} class}
% Use the following harness for your \LaTeX{} letter:
% \begin{verbatim}
% \documentclass[a4paper]{uantwerpenletter}
%
% % setup fonts according to your specific TeX compiler setup
%
% \usepackage{<include any packages you require here>}
%
% % \logo{} only specify if you want to use your unit's logo
%
% \sender{<put your name here>}{<put your title/role here>}
% \facultyacronym{<put your faculty's acronym here>}
% \unit{<put your unit here>}
% \address{<put your multi-line address here>}
% \email{<user name>}{<domain name>}
% \phone{<put your phone number here, start with +32>}
% \fax{<put your fax number here, start with +32>}
% \mobile{<put your mobile number here, start with +32>}
% \returnaddress{<put your single-line return address here>}
%
% \to{<name of the addressee goes here>}
% \toorganization{<name of the organization goes here>}
% \toaddress{<multi-line address of the addressee goes here>}
% 
% \date{<specify date - otherwise today>}
% \subject{<specify subject>}
%
% \begin{document}
%
%   \maketitle % generates top of the letter
%
%   \opening{Dear <name>}
% 
%   <write your letter here>
%
%   \closing{Kind regards,}
%
%   \carboncopy{<put CC people here>}
%   \enclosed{<put reference to enclosed documents here>}
%
% \end{document}
% \end{verbatim}
%
% The available faculty acronyms are listed in a table on page
% \pageref{acronyms}. You may use lists in the |\carboncopy| and
% |\enclosed| commands. The spacing will be compact.
%
% \subsection{The class options explained}
%
% The classes have several options. They are listed below.
% After every option, it has been indicated to which class the option
% applies (between square brackets, without prefix uantwerpen).
% \changes{v1.1}{2013/05/28}{Added option user documentation}
%
% \DescribeMacro{copyright} [coursetext]\\
%   This option forces printing a watermark on every page. For the
%   paper version of your document, this is inappropriate, but for any
%   e-copy you make available, this may be appropriate;
%
% \DescribeMacro{filled} [letter / coursetext /
% masterthesis]\\ 
%   This option causes the text to be filled (simultaneous left and
%   right alignment). Though this setting is not recommended, it is
%   provided because the default |\raggedright| cannot be undone. The
%   |filled| option prevents the |\raggedright| from being
%   issued. However, if you care about the readability of your text,
%   you shouldn't use this option.
%
% \DescribeMacro{titlepagenoartwork} [coursetext /
% masterthesis]\\
%   This option forces the title pages to be typeset without circle graphics and
%   logo. This allows for printing on a pre-printed color sheet that
%   already contains circle graphics and logo;
%
% \DescribeMacro{titlepagetableonly} [coursetext /
% masterthesis]\\ 
%   This option forces the title-page data to be printed in table form
%   as first page. Some publishers require the manuscript to be
%   delivered in this form. They perform the entire typesetting of the
%   title page.
%
% \DescribeMacro{qr} [coursetext]\\
%   This option allows you to generate a QR code containing the details of
%   the course on the title page or the table-only title page. For
%   this option to work, the package pstricks is loaded. It will not
%   work with pdf\LaTeX{} unless you enable shell escape. Read your
%   pdf\LaTeX{}-package documentation on how to do that.
%
% Common sets of options depend on the purpose:
% \begin{itemize}
% \item to make a text ready for electronic distribution:
% |a4paper|, |copyright|.
% \item to make a camera-ready text (for printing) in case
% the cover is printed on a pre-printed color artwork cover sheet is:
% |a4paper|, |qr|, |titlepagenoartwork|.
% \item to make a camera-ready text (for printing) in case the cover
% is typeset based on table data:
% |a4paper|, |qr|, |titlepagetableonly|. 
% \item to make a letter:
% no options (filling a letter is discouraged)
% \end{itemize}
%
% \subsection{The macros explained}
%
% \subsubsection{Common macros}
%
% After every macro, it has been indicated to which class the macro
% applies (between square brackets), and whether it is mandatory or not.
%
% \DescribeMacro{\facultyacronym} [coursetext /
% masterthesis] (mandatory)\\ 
% This macro sets the acronym of the faculty.
% This macro also sets the faculty name according to the specified
% acronym.
% If you're missing a faculty or institute, please ask the
% authors to complete the list.
% 
% The available acronyms are:
% \label{acronyms}
% \begin{center}
% \begin{tabular}{cl}
%   \toprule
%   Acronym & Faculty name \\
%   \midrule
%   CPG
%   & Centrum Pieter Gillis\\
%   FBD 
%   & Faculteit Farmaceutische, Biomedische en Diergeneeskundige Wetenschappen\\
%   GGW    
%   & Faculteit Geneeskunde en Gezondheidswetenschappen\\
%   IOB
%   & Instituut voor Ontwikkelingsbeleid- en beheer\\
%   IOIW
%   & Instituut voor Onderwijs- en Informatiewetenschappen\\
%   LW
%   & Faculteit Letteren en Wijsbegeerte\\
%   OW
%   & Faculteit Ontwerpwetenschappen\\
%   SW
%   & Faculteit Sociale Wetenschappen\\
%   REC
%   & Faculteit Rechten\\
%   TEW
%   & Faculteit Toegepaste Economische Wetenschappen\\
%   TI     
%   & Faculteit Toegepaste Ingenieurswetenschappen\\
%   WET    
%   & Faculteit Wetenschappen\\
%   \bottomrule
% \end{tabular}
% \end{center}
%
% \subsubsection{Macros for the coursetext and masterthesis classes}
%
% \DescribeMacro{\title} [coursetext /
% masterthesis] (mandatory)\\ 
% This macro sets the title of the document.
% It also sets the |pdftitle| tag of the hyperref package, so that
% the PDF-document meta-information is correct.
%
% \DescribeMacro{\subtitle} [coursetext] (optional)\\
% This macro sets the title of the document. You may use this 
% \begin{itemize}
% \item to further clarify the title
% \item to indicate the nature of this document
% \end{itemize}
% The latter is to be considered when you want to provide multiple
% documents as parts of the full course text (e.g., Course Notes,
% Formula Collection, Exercise Book, Solution Book).
% This macro also sets the |subject| tag of the hyperref package,
% so that the PDF-document meta-information is correct.
%
% \DescribeMacro{\author} [coursetext /
% masterthesis] (mandatory)\\ 
% This macro sets the author of the document.
% It also sets the |pdfauthor| tag of the hyperref package, so that
% the PDF-document meta-information is correct.
%
% \DescribeMacro{\publisher} [coursetext] (mandatory)\\
% This macro sets the publisher information of the document.
% It is printed on the front page. It defaults to the repographic
% service of campus Groenenborger, one of the standard printing
% services of Universiteit Antwerpen.
%
% \DescribeMacro{\publishercode} [coursetext] (mandatory)\\
% This macro sets the publisher code of the document.
% It is printed on the front page. This is code that the publisher
% uses for its internal administration. It may be a proprietary code,
% or an ISBN number.
%
% \DescribeMacro{\courseversion} [coursetext] (optional)\\
% This macro indicates which version of the course this is.
%
% \DescribeMacro{\versionyear} [coursetext] (mandatory)\\
% This is to be the year in which you published the current version of
% the course in the form YYYY.
%
% \DescribeMacro{\lecturer} [coursetext] (mandatory)\\
% This is the name of the person that actually teaches the course (in
% Dutch: titularis). If there are multiple persons, please, use the
% macros |\lectureri|, |\lecturerii|, |\lectureriii|,
% |\lectureriv|. 
%
% \DescribeMacro{\programme} [coursetext] (mandatory)\\
% This macro takes three arguments (for the time being, only
% applicable to the faculty of applied engineering):
% \begin{itemize}
% \item the type of the programme: BA, SP, VP or MA
% \item the domain of the programme: IW
% \item the qualifier of the programme: BK, CH, BCH, EM, EI
% \end{itemize}
% If you need more programme classes or qualifiers, ask the authors to
% complete the available codes.
% Correct usage of the macro will result in error-free descriptions on
% your title page.
% You can overrule the standard descriptions, by specifying 'FREE' as
% first argument and a free text description as second, leaving the third
% one empty. However, we strongly advise against taking this route.
% Instead, ask the authors to complete the available codes.
%
% \DescribeMacro{\supervisor} [masterthesis] (mandatory)\\
% This is the name of the person that promotes the thesis.
% If there are multiple persons, please, use the
% macros |\supervisori|, |\supervisorii|, |\supervisoriii|,
% |\supervisoriv|. 
%
% \DescribeMacro{\course} [coursetext] (mandatory)\\
% Code (first argument) and name (second argument) of the curriculum
% course this coursematerial belongs to. The code should be of the form:\\
% |TNNNFFFAAA|,
% with:
% \begin{center}
%   \begin{tabular}{cp{10cm}}
%     \toprule
%       Code & Explanation \\
%     \midrule
%       |T|    & a number indicating the type of programme \\
%              & (1 for Bachelor courses, 2 for Master courses , 5 for
%                specific courses of preparatory programmes) \\
%       |NNN|  & a number assigned by the Faculty's administration\\
%       |FFF|  & the acronym of your Faculty, e.g., FTI\\
%       |AAA|  & an alphanumeric code assigned by the Faculty's
%                administration \\
%     \bottomrule
%   \end{tabular}
% \end{center}
%
% An example of such a code: 1001FTIWIS, for the first-semester
% mathematics course of the Faculty of Applied Engineering.
%
% For courses of the Faculty of Applied Engineering, the name should
% be of the form |x-YYYYYYYY| with |x| the number of the
% semester and |YYYYYYYY| the official name of the course.
%
% In case the course's name contains accented characters, one should
% also provide a qr version, containing utf8-characters only.
% The macro for this purpose takes only one argument, i.e. the
% course's name! This is to avoid inconsistencies in the course codes.
%
% \DescribeMacro{\academicyear} [coursetext /
% masterthesis] (mandatory)\\ 
% Use this macro to specify the academic year in full, i.e. in the
% form |XXXX-YYYY|. 
% 
% \changes{v1.6}{2016/02/04}{Added diploma code documentation}
% \DescribeMacro{\diploma} [masterthesis] (mandatory)\\
% This must be the official title, in Dutch. To avoid errors, we chose
% to use specific codes, that will expand to the correct description.
% These codes are specific for the Faculty of Applied Engineering. If
% you want the author of the package to add codes for your faculty,
% just ask!
% \begin{center}
%   \begin{tabular}{lp{10cm}}
%     \toprule
%     Code & Description (in Dutch!) \\
%     \midrule
%       \multicolumn{2}{c}{Faculteit Toegepaste
%       Ingenieurswetenschappen}\\ 
%     \midrule
%     |MA-IW-BK| 
%     & Master of Science in de industri\"ele wetenschappen: bouwkunde\\
%     |MA-IW-BCH| 
%     & Master of Science in de industri\"ele wetenschappen: biochemie\\
%     |MA-IW-CH|
%     & Master of Science in de industri\"ele wetenschappen: chemie\\
%     |MA-IW-EI|
%     & Master of Science in de industri\"ele wetenschappen: elektronica-ICT\\
%     |MA-IW-EI-AE|
%     & Master of Science in de industri\"ele wetenschappen: elektronica-ICT,
%     afstudeerrichting, Automotive Engineering\\
%     |MA-IW-EI-ICT|
%     & Master of Science in de industri\"ele wetenschappen: elektronica-ICT,
%     afstudeerrichting ICT\\
%     |MA-IW-EM-AE|
%     & Master of Science in de industri\"ele wetenschappen: Elektromechanica,
%     afstudeerrichting Automotive Engineering\\
%     |MA-IW-EM-AU|
%     & Master of Science in de industri\"ele wetenschappen: Elektromechanica,
%     afstudeerrichting Automatisering\\
%     |MA-IW-EM-EM|
%     & Master of Science in de industri\"ele wetenschappen: Elektromechanica,
%     afstudeerrichting Elektromechanica\\
%     |MA-IW-EM-EN|
%     & Master of Science in de industri\"ele wetenschappen: Elektromechanica,
%     afstudeerrichting Energie\\
%     \midrule
%       \multicolumn{2}{c}{Faculteit Toegepaste Economische
%       Wetenschappen}\\
%     \midrule
%     |MA-TEW-HI| 
%     & Master of Science in de toegepaste economische wetenschappen:
%     handelsingenieur\\
%     |MA-TEW-HIBI| 
%     & Master of Science in de toegepaste economische wetenschappen:
%     handelsingenieur in de beleidsinformatica\\
%     |MA-TEW-EB|
%     & Master of Science in de toegepaste economische wetenschappen:
%     economische beleid\\
%     |MA-TEW-BK|
%     & Master of Science in de toegepaste economische wetenschappen:
%     bedrijfskunde\\
%     \bottomrule
%   \end{tabular}
% \end{center}
% 
% \DescribeMacro{\defensedate} [masterthesis] (mandatory)\\
% Date of the defense in Dutch, in the form 'month year', e.g. ``juni 2012''.
% 
% \DescribeMacro{\defenselocation} [masterthesis]
% (optional)\\ 
% Location of the defense. Defaults to ``Antwerpen''.
% 
% \DescribeMacro{\copyrightnotices} [coursetext]
% (optional)\\ 
% Use this macro to specify additional copyright notice messages to
% appear in the copyright notice on the bottom of page 2 of your
% course text.
% 
% \subsubsection{Macros for the letter class}
%
% \DescribeMacro{\sender} [letter] (mandatory)\\
% Description of the person writing the letter.
% \begin{itemize}
% \item first argument: name of the person writing the letter
% \item second argument: title / role of the person
% \end{itemize}
% Newlines are not allowed in the arguments.
%
% \DescribeMacro{\logo} [letter] (optional)\\
% file name of an alternative logo to use. The file name must be the
% name of a file in the search path of type PDF.
% If this macro is not used, The default logo of the university will
% be used.
%
% \DescribeMacro{\unit} [letter] (optional)\\
% Name of the unit to which the person belongs. This can be a
% research group, a laboratory, an administrative division, etc.
% Newlines are allowed.
%
% \DescribeMacro{\address} [letter] (mandatory)\\
% Address of the sending unit (or faculty). This can be different from
% the return address. Newlines are allowed and encouraged.
%
% \DescribeMacro{\email} [letter] (optional)\\
% E-mail address of the sending person, or the administrative person
% tracking the letter. This must definitely be someone that can answer
% questions related to this letter.
% \begin{itemize}
% \item first argument: user name
% \item second argument: domain name
% \end{itemize}
%
% \DescribeMacro{\phone} [letter] (optional)\\
% Phone number of the sending person. See also |\email|.
%
% \DescribeMacro{\fax} [letter] (optional)\\
% Probably facsimile is not used anymore, but anyway: fax number of
% the sending person. See also |\email|.
%
% \DescribeMacro{\mobile} [letter] (optional)\\
% Mobile phone number of the sending person. See also |\email|.
%
% \DescribeMacro{\returnaddress} [letter] (mandatory)\\
% This is a short return address (listed in small font on top of the
% destination address (such that it is visible in a windowed envelope
% (European format)). It should fit on a single line. Typically we list
% an acronym for the unit, a room number, a campus name and address.
% The goal is to get the undelivered letter back to the person that
% can take action accordingly.
%
% \DescribeMacro{\to} [letter] (mandatory)\\
% Name of the addressee. Newlines are allowed. 
% Preferably name and role are split over two lines.
%
% \DescribeMacro{\toorganization} [letter] (optional)\\
% Name of the organization that employs the addressee.
%
% \DescribeMacro{\toaddress} [letter] (mandatory)\\
% Address of the addressee. Newlines are allowed. The address should
% fit on max. 3 lines.
%
% \DescribeMacro{\date} [letter] (optional) \\
% Date of the letter. If not specified today's date (at the time of
% running \LaTeX{}) will be used.
%
% \DescribeMacro{\subject} [letter] (mandatory) \\
% Short descriptive subject that describes the contents of the
% letter.
%
% \DescribeMacro{\opening}  [letter] (mandatory) \\
% Opening address of the letter. E.g. 'Dear X,'.
%
% \DescribeMacro{\closing}  [letter] (mandatory) \\
% Closing clause of the letter. E.g. 'Best regards,'.
%
% \DescribeMacro{\signature}  [letter] (optional) \\
% Add a signature (in between the closing statement of the
% letter and the sender's name. This might be a text message or a
% picture of your signature.
%
% \DescribeMacro{\carboncopy} [letter] (optional)\\
% List of persons receiving a copy of this letter. Format at will.
%
% \DescribeMacro{\enclosed} [letter] (optional)\\
% List of enclosed documents. Format at will.
% 
% \subsection{Examples}
% \subsubsection{\texttt{uantwerpencoursetext}}
%
% This example uses the |qr| option (that invokes the |auto-pst-pdf|
% package) so enable 'write18' or 'shell-escape' for your \LaTeX{}
% compiler.
%
% \begin{verbatim}
%<*ct-example> 
\documentclass[a4paper,11pt,oneside,openright,english,qr,copyright]{uantwerpencoursetext}

\usepackage[english,dutch]{babel}

\title{Z\'agen, zoeken en zuchten}
\qrtitle{Zágen, zoeken en zuchten}
\subtitle{Cursusnota's}
\author{Walter Daems en Paul Levrie}

\courseversion{1.3}
\versionyear{2016}

\lectureri{Zoltan Zo\"ekers}
\qrlectureri{Zoltan Zoëkers}
\lecturerii{Siana Sigh}
\lectureriii{Zeger de Z\'ager}
\qrlectureriii{Zeger de Záger}

\facultyacronym{TI}
\programme{MA}{IW}{EI}
\coursei{2023FTIZZZ}{5-Zoekmachines in een zaagperspectief}
\courseii{2045FTIIII}{6-Zaagmachines in \'e\'en zuchtperspectief}
\qrcourseii{6-Zaagmachines in één zuchtperspectief}

\academicyear{2015-2016}


\publisher{Universiteit Antwerpen\\
  Cursusdienst en reprografie\\
  Campus Groenenborger, G.U.027\\
  Groenenborgerlaan 171\\
  2020 Antwerpen\\
  T +32 3 265 32 15\\
  F + 32 3 233 32 27\\
  E cursusdienst.cgb@uantwerpen.be}

\publishercode{C11111102}

\copyrightnotices{
  The graphics in this document have been typeset using \texttt{TikZ}.\\
  This document has been \TeX-ed on a GNU/Linux workstation.
}

\begin{document}
\selectlanguage{dutch} % or english if your text is in English

\maketitle

\frontmatter

\tableofcontents

\mainmatter
\chapter*{Inleiding}
Lorem ipsum dolor sit amet, consectetur adipisicing elit, sed do
eiusmod tempor incididunt ut labore et dolore magna aliqua. Ut enim ad
minim veniam, quis nostrud exercitation ullamco laboris nisi ut
aliquip ex ea commodo consequat. Duis aute irure dolor in
reprehenderit in voluptate velit esse cillum dolore eu fugiat nulla
pariatur. Excepteur sint occaecat cupidatat non proident, sunt in
culpa qui officia deserunt mollit anim id est laborum.


\chapter{Onzin voor dummies}

\section{Een beetje Cicero}
Sed ut perspiciatis unde omnis iste natus error sit voluptatem
accusantium doloremque laudantium, totam rem aperiam, eaque ipsa quae
ab illo inventore veritatis et quasi architecto beatae vitae dicta
sunt explicabo. Nemo enim ipsam voluptatem quia voluptas sit
aspernatur aut odit aut fugit, sed quia consequuntur magni dolores eos
qui ratione voluptatem sequi nesciunt. Neque porro quisquam est, qui
dolorem ipsum quia dolor sit amet, consectetur, adipisci velit, sed
quia non numquam eius modi tempora incidunt ut labore et dolore magnam
aliquam quaerat voluptatem. Ut enim ad minima veniam, quis nostrum
exercitationem ullam corporis suscipit laboriosam, nisi ut aliquid ex
ea commodi consequatur? Quis autem vel eum iure reprehenderit qui in
ea voluptate velit esse quam nihil molestiae consequatur, vel illum
qui dolorem eum fugiat quo voluptas nulla pariatur?

\begin{equation}
  e^{-j\pi} + 1 = 0
\end{equation}

At vero eos et accusamus et iusto odio dignissimos ducimus qui
blanditiis praesentium voluptatum deleniti atque corrupti quos dolores
et quas molestias excepturi sint occaecati cupiditate non provident,
similique sunt in culpa qui officia deserunt mollitia animi, id est
laborum et dolorum fuga. Et harum quidem rerum facilis est et expedita
distinctio. Nam libero tempore, cum soluta nobis est eligendi optio
cumque nihil impedit quo minus id quod maxime placeat facere possimus,
omnis voluptas assumenda est, omnis dolor repellendus. Temporibus
autem quibusdam et aut officiis debitis aut rerum necessitatibus saepe
eveniet ut et voluptates repudiandae sint et molestiae non
recusandae. Itaque earum rerum hic tenetur a sapiente delectus, ut aut
reiciendis voluptatibus maiores alias consequatur aut perferendis
doloribus asperiores repellat.

\section{En waartoe het geleid heeft}

Lorem ipsum dolor sit amet, consectetur adipisicing elit, sed do
eiusmod tempor incididunt ut labore et dolore magna aliqua. Ut enim ad
minim veniam, quis nostrud exercitation ullamco laboris nisi ut
aliquip ex ea commodo consequat. Duis aute irure dolor in
reprehenderit in voluptate velit esse cillum dolore eu fugiat nulla
pariatur. Excepteur sint occaecat cupidatat non proident, sunt in
culpa qui officia deserunt mollit anim id est laborum.

\subsection{Herhaling}
Sed ut perspiciatis unde omnis iste natus error sit voluptatem
accusantium doloremque laudantium, totam rem aperiam, eaque ipsa quae
ab illo inventore veritatis et quasi architecto beatae vitae dicta
sunt explicabo. Nemo enim ipsam voluptatem quia voluptas sit
aspernatur aut odit aut fugit, sed quia consequuntur magni dolores eos
qui ratione voluptatem sequi nesciunt. Neque porro quisquam est, qui
dolorem ipsum quia dolor sit amet, consectetur, adipisci velit, sed
quia non numquam eius modi tempora incidunt ut labore et dolore magnam
aliquam quaerat voluptatem. Ut enim ad minima veniam, quis nostrum
exercitationem ullam corporis suscipit laboriosam, nisi ut aliquid ex
ea commodi consequatur? Quis autem vel eum iure reprehenderit qui in
ea voluptate velit esse quam nihil molestiae consequatur, vel illum
qui dolorem eum fugiat quo voluptas nulla pariatur?

\subsection{Begint vervelend te worden}
At vero eos et accusamus et iusto odio dignissimos ducimus qui
blanditiis praesentium voluptatum deleniti atque corrupti quos dolores
et quas molestias excepturi sint occaecati cupiditate non provident,
similique sunt in culpa qui officia deserunt mollitia animi, id est
laborum et dolorum fuga. Et harum quidem rerum facilis est et expedita
distinctio. Nam libero tempore, cum soluta nobis est eligendi optio
cumque nihil impedit quo minus id quod maxime placeat facere possimus,
omnis voluptas assumenda est, omnis dolor repellendus. Temporibus
autem quibusdam et aut officiis debitis aut rerum necessitatibus saepe
eveniet ut et voluptates repudiandae sint et molestiae non
recusandae. Itaque earum rerum hic tenetur a sapiente delectus, ut aut
reiciendis voluptatibus maiores alias consequatur aut perferendis
doloribus asperiores repellat.

\newpage

\subsection{Begint echt vervelend te worden}
At vero eos et accusamus et iusto odio dignissimos ducimus qui
blanditiis praesentium voluptatum deleniti atque corrupti quos dolores
et quas molestias excepturi sint occaecati cupiditate non provident,
similique sunt in culpa qui officia deserunt mollitia animi, id est
laborum et dolorum fuga. Et harum quidem rerum facilis est et expedita
distinctio. Nam libero tempore, cum soluta nobis est eligendi optio
cumque nihil impedit quo minus id quod maxime placeat facere possimus,
omnis voluptas assumenda est, omnis dolor repellendus. Temporibus
autem quibusdam et aut officiis debitis aut rerum necessitatibus saepe
eveniet ut et voluptates repudiandae sint et molestiae non
recusandae. Itaque earum rerum hic tenetur a sapiente delectus, ut aut
reiciendis voluptatibus maiores alias consequatur aut perferendis
doloribus asperiores repellat.


\chapter{Besluit}

\backmatter
\appendix

\chapter{Symbolen}
\chapter{Romeinse sprekers}
\chapter{Referentielijst}

\makefinalpage

\end{document}
%</ct-example> 
% \end{verbatim}
% 
% 
% \subsubsection{\texttt{uantwerpenmasterthesis}}
% 
% \begin{verbatim}
%<*mt-example> 
\documentclass[a4paper,11pt,twoside,openright,english]{uantwerpenmasterthesis}

\usepackage[english]{babel} % or dutch if your text is in Dutch

\title{Minimax optimisatie voor performantieruimtemodellering}
\author{Bert Bibber}

\supervisori{Prof. dr. ir. Kumulus (Universiteit Antwerpen)}
\supervisorii{Prof. dr. Hilarius Warwinkel (TNT-Bang, N.V.)}
\supervisoriii{ing. Piet Pienter (POM)}

\facultyacronym{TI}
\academicyear{2015-2016}
\diploma{MA-IW-EI-ICT}
\defenselocation{Antwerpen}
\defensedate{juni 2016}

\begin{document}

\maketitle

\frontmatter

\tableofcontents

\mainmatter
\chapter*{Inleiding}
Lorem ipsum dolor sit amet, consectetur adipisicing elit, sed do
eiusmod tempor incididunt ut labore et dolore magna aliqua. Ut enim ad
minim veniam, quis nostrud exercitation ullamco laboris nisi ut
aliquip ex ea commodo consequat. Duis aute irure dolor in
reprehenderit in voluptate velit esse cillum dolore eu fugiat nulla
pariatur. Excepteur sint occaecat cupidatat non proident, sunt in
culpa qui officia deserunt mollit anim id est laborum.

\chapter{Onderzoeksvraag}

\section{Een beetje Cicero}
Sed ut perspiciatis unde omnis iste natus error sit voluptatem
accusantium doloremque laudantium, totam rem aperiam, eaque ipsa quae
ab illo inventore veritatis et quasi architecto beatae vitae dicta
sunt explicabo. Nemo enim ipsam voluptatem quia voluptas sit
aspernatur aut odit aut fugit, sed quia consequuntur magni dolores eos
qui ratione voluptatem sequi nesciunt. Neque porro quisquam est, qui
dolorem ipsum quia dolor sit amet, consectetur, adipisci velit, sed
quia non numquam eius modi tempora incidunt ut labore et dolore magnam
aliquam quaerat voluptatem. Ut enim ad minima veniam, quis nostrum
exercitationem ullam corporis suscipit laboriosam, nisi ut aliquid ex
ea commodi consequatur? Quis autem vel eum iure reprehenderit qui in
ea voluptate velit esse quam nihil molestiae consequatur, vel illum
qui dolorem eum fugiat quo voluptas nulla pariatur?

\begin{equation}
  e^{-j\pi} + 1 = 0
\end{equation}

At vero eos et accusamus et iusto odio dignissimos ducimus qui
blanditiis praesentium voluptatum deleniti atque corrupti quos dolores
et quas molestias excepturi sint occaecati cupiditate non provident,
similique sunt in culpa qui officia deserunt mollitia animi, id est
laborum et dolorum fuga. Et harum quidem rerum facilis est et expedita
distinctio. Nam libero tempore, cum soluta nobis est eligendi optio
cumque nihil impedit quo minus id quod maxime placeat facere possimus,
omnis voluptas assumenda est, omnis dolor repellendus. Temporibus
autem quibusdam et aut officiis debitis aut rerum necessitatibus saepe
eveniet ut et voluptates repudiandae sint et molestiae non
recusandae. Itaque earum rerum hic tenetur a sapiente delectus, ut aut
reiciendis voluptatibus maiores alias consequatur aut perferendis
doloribus asperiores repellat.

\chapter{Literatuurstudie}

\chapter{Theoretische achtergrond}

\chapter{Eigen realisatie}

\chapter{Besluit}

\backmatter
\appendix

\chapter{Symbolen}
\chapter{Referentielijst}

\makefinalpage

\end{document}
%</mt-example> 
% \end{verbatim}
% 
%

% \subsubsection{\texttt{uantwerpenletter}}
% 
% \paragraph{Plain example}
%
% \begin{verbatim}
%<*le-example> 
\documentclass[a4paper]{uantwerpenletter}

%% As a good UAntwerpen citizen, you would use the calibri font.
%% As this only works for XeLaTeX or LuaLaTeX, we chose to include
%% cmbright instead. So for ease of use, we include:
\usepackage{cmbright}
%% But if you have XeLaTeX or LuaLaTeX, use the following instead:
%%\usepackage{fontspec}
%%\setmainfont{Calibri}

\usepackage[english]{babel}

\sender{Prof. Walter Daems}{Senior Lecturer}
\facultyacronym{TI}
\unit{CoSys-Lab}
\address{
  Campus Groenenborger\\
  Groenenborgerlaan 171\\
  B-2020 Antwerpen\\
  BELGIUM}
\email{walter.daems}{uantwerpen.be}
\phone{+32 3 265 98 43}
\mobile{+32 499 355 115}
\returnaddress{FTI - U.301 -- Groenenborgerlaan 171, 2020 Antwerpen, BELGIUM}

\to{Prof. B. Bonette}
\toorganization{Mumford University}
\toaddress{
  450 Morning Mall\\
  Mumford, DX 94305-2004\\
  USA}

\date{January 3, 2016}
\subject{Congratulations for online video lectures}

\begin{document}
  \maketitle

  \opening{Dear Prof. Bonette,}

  I'd like to congratulate you and the other professors of your
  university on the very instructive video lectures
  provided by your University. They are valued very
  highly. 

  You inspired many a professor at our university to provide more
  technical content beyond classical paper courses.
  Based on your inspiring lectures, some students desire to candidate
  themselves for taking an internship at your university. You can find
  their details enclosed.

  Below, you can find a few more paragraphs to illustrate that this
  class can generate multipage letters.

  Lorem ipsum dolor sit amet, consectetur adipisicing elit, sed do
  eiusmod tempor incididunt ut labore et dolore magna aliqua. Ut enim ad
  minim veniam, quis nostrud exercitation ullamco laboris nisi ut
  aliquip ex ea commodo consequat. Duis aute irure dolor in
  reprehenderit in voluptate velit esse cillum dolore eu fugiat nulla
  pariatur. Excepteur sint occaecat cupidatat non proident, sunt in
  culpa qui officia deserunt mollit anim id est laborum.

  Sed ut perspiciatis unde omnis iste natus error sit voluptatem
  accusantium doloremque laudantium, totam rem aperiam, eaque ipsa quae
  ab illo inventore veritatis et quasi architecto beatae vitae dicta
  sunt explicabo. Nemo enim ipsam voluptatem quia voluptas sit
  aspernatur aut odit aut fugit, sed quia consequuntur magni dolores eos
  qui ratione voluptatem sequi nesciunt. Neque porro quisquam est, qui
  dolorem ipsum quia dolor sit amet, consectetur, adipisci velit, sed
  quia non numquam eius modi tempora incidunt ut labore et dolore magnam
  aliquam quaerat voluptatem. Ut enim ad minima veniam, quis nostrum
  exercitationem ullam corporis suscipit laboriosam, nisi ut aliquid ex
  ea commodi consequatur? Quis autem vel eum iure reprehenderit qui in
  ea voluptate velit esse quam nihil molestiae consequatur, vel illum
  qui dolorem eum fugiat quo voluptas nulla pariatur?

  At vero eos et accusamus et iusto odio dignissimos ducimus qui
  blanditiis praesentium voluptatum deleniti atque corrupti quos dolores
  et quas molestias excepturi sint occaecati cupiditate non provident,
  similique sunt in culpa qui officia deserunt mollitia animi, id est
  laborum et dolorum fuga. Et harum quidem rerum facilis est et expedita
  distinctio. Nam libero tempore, cum soluta nobis est eligendi optio
  cumque nihil impedit quo minus id quod maxime placeat facere possimus,
  omnis voluptas assumenda est, omnis dolor repellendus. Temporibus
  autem quibusdam et aut officiis debitis aut rerum necessitatibus saepe
  eveniet ut et voluptates repudiandae sint et molestiae non
  recusandae. Itaque earum rerum hic tenetur a sapiente delectus, ut aut
  reiciendis voluptatibus maiores alias consequatur aut perferendis
  doloribus asperiores repellat.

  \closing{Kind regards,}
  % you might want to insert a signature picture or text:
  % \signature{\includegraphics{signature.jpg}}
  \carboncopy{Prof. S. Mariotte, Mumford University}
  \enclosed{
    \begin{enumerate}
    \item list of course numbers that are most fequently viewed at
      our university (1pp)
    \item a list of students desiring to take an internship at
      Mumford University (2pp)
    \end{enumerate}
  }
\end{document}
%</le-example> 
% \end{verbatim}
%
% \paragraph{Example with configuration file}~\\
% Probably, one has to write many letters. The sender details will be
% most certainly valid for many an occasion. Therefore, you might want
% to consider putting this default setup in a configuration file,
% e.g. \texttt{uantwerpenletter.cfg}:
%
% \begin{verbatim}
%<*le-cfg>
%% configuration file for uantwerpenletter class
\usepackage{fontspec} % XeLaTeX/LauTeX specific, replace by e.g.
\setmainfont{Calibri} % \usepackage{cmbright}
\sender{Prof. Walter Daems}{Senior Lecturer}
\facultyacronym{TI}
\unit{CoSys-Lab}
\address{
  Campus Groenenborger\\
  Groenenborgerlaan 171\\
  B-2020 Antwerpen\\
  BELGIUM}
\email{walter.daems}{uantwerpen.be}
\phone{+32 3 265 98 43}
\mobile{+32 499 355 115}
\returnaddress{FTI - U.301 -- Groenenborgerlaan 171, 2020 Antwerpen, BELGIUM}
%</le-cfg>
% \end{verbatim}
%
% The file can then be loaded in the preamble of your letter:
% \begin{verbatim}
% \input{uantwerpenletter.cfg}
% \end{verbatim}
%
% After loading this configuration file, you may override some
% elements if this is appropriate.
%
% You may also consider using multiple configuration files in case you
% have multiple roles in the university. Just make sure they are on your
% \LaTeX\ search path.
%
% \StopEventually{\clearpage\PrintChanges\clearpage\PrintIndex}
% 
% \section{Implementation}
% 
% \subsection{Class inheritance}
% 
% 
% For convenience, we'll derive from the standard \LaTeX{} |book| and 
% |letter| class. 
% 
% \changes{v1.0}{2013/05/11}{Added option titlepagetableonly}
% \changes{v1.0}{2013/05/11}{Added option titlepagenoartwork}
% \changes{v1.0}{2013/05/11}{Added option qr}
% \changes{v1.4}{2016/01/07}{Implemented letter class}
%
% Before loading the class, we provide the extra options.
% 
%    \begin{macrocode}
%<*ct>      
\newif\if@copyright
\DeclareOption{copyright}{\@copyrighttrue}
\newif\if@qr
\DeclareOption{qr}{\@qrtrue}
%</ct>
%    \end{macrocode}
%
%    \begin{macrocode}
%<*ct|mt>
\newif\if@titlepagenoartwork
\DeclareOption{titlepagenoartwork}{\@titlepagenoartworktrue}
\newif\if@titlepagetableonly
\DeclareOption{titlepagetableonly}{\@titlepagetableonlytrue}
%</ct|mt>
%<*ct|mt|le>
\newif\if@filled
\DeclareOption{filled}{\@filledtrue}
%</ct|mt|le>
%    \end{macrocode}
%
% We execute some standard options:
% We load the |book| class.
%    \begin{macrocode}
%<*le> 
\ExecuteOptions{a4paper,10pt,final,oneside,openright}
\ProcessOptions
\LoadClassWithOptions{letter}
\newcommand\tat{\makeatletter @\makeatother}
\newcommand\tbs{\textbackslash}
%</le>
%<*ct|mt> 
\ExecuteOptions{a4paper,11pt,final,oneside,openright}
\ProcessOptions
\LoadClassWithOptions{book}
%</ct|mt>
%    \end{macrocode}
% 
% \subsection{Modern typesetting}
% Let's force some modern typesetting without paragraph indentation
% and with a decent paragraph spacing.
% 
%    \begin{macrocode}
%<*ct|mt|le>      
\setlength{\parindent}{0pt}
\addtolength{\parskip}{0.75\baselineskip}
\setcounter{secnumdepth}{3}
%</ct|mt|le> 
%    \end{macrocode}
% 
% \subsection{Auxiliary packages}
% Reinventing the wheel is a waste of time, let's preload some
% appropriate auxiliary packages that have proven their value.
%
% \subsubsection{Geometry}
% Let's reduce the margins to 1 inch each.
%    \begin{macrocode}
%<*ct|mt>      
\RequirePackage[top=1in, bottom=1in, left=1in, right=1in]{geometry}
%</ct|mt> 
%<*le>      
\RequirePackage[top=1in, bottom=1in, left=1.34in, right=1in]{geometry}
\RequirePackage[normalem]{ulem}
\RequirePackage{atbegshi}
%</le> 
%    \end{macrocode}
% 
% \subsubsection{Font packages}
% Note that the use of cmbright is no
% longer imposed (as of v1.3). Using a good font is now up to the
% user. The packages 'mathpazo' and 'cmbright' are highly recommended.
% For writing letters, 'Calibri' is the official font of the
% University of Antwerp.
% \changes{v1.2}{2014/08/22}{Added lmodern package to please MikTeX}
% \changes{v1.3}{2015/12/31}{Abandoned use of cmbright - no more
% font dictatorship for theses and courses}
%    \begin{macrocode}
%<*ct|mt|le>      
% no more font code
%</ct|mt|le>
%    \end{macrocode}
%
% \subsubsection{Boilerplate packages}
% 
% Some boilerplate packages and an empty macro to test against
% (using|\ifx|)
% \changes{v1.8}{2017/01/08}{Added missing packages ifmtarg and shellesc}
%    \begin{macrocode}
%<*ct|mt|le>      
\RequirePackage{ifthen}
\RequirePackage{ifmtarg}
\RequirePackage{shellesc}
\newcommand{\@emptymacro}{}
%</ct|mt|le> 
%    \end{macrocode}
% 
% \subsubsection{Graphics packages}
% 
% Graphics packages that are required for the title page, but may come
% in handy for regular use as well.
%
% Some packages for coursetext and masterthesis:
% \changes{v1.3}{2015/12/31}{Added inclusion of background package}
%    \begin{macrocode}
%<*ct|mt|le>      
\RequirePackage{graphicx}
\RequirePackage{color}
\RequirePackage{tikz}
%</ct|mt|le>
%<*ct>
\if@copyright
\RequirePackage[firstpage=false,contents={Copyright University of Antwerp, All Rights Reserved},color=lightgray,scale=3]{background}
\fi
%</ct>
%    \end{macrocode}
%
% In uantwerpencoursetext we also want to generate a qr code.
% Therefore we load the |pst-barcode| and |auto-pst-pdf| package.
% In this case you must enable 'write18' or 'shell-escape' for your
% \LaTeX{} compiler. Check your documention on how to do so!
%
%    \begin{macrocode}
%<*ct>
\if@qr
\RequirePackage{auto-pst-pdf}
\RequirePackage{pst-barcode}
\fi
%</ct>
%    \end{macrocode}
%
% \subsubsection{Header/Footer}
% 
% The de-facto standard for headers and footers:
%    \begin{macrocode}
%<*ct|mt|le>      
\RequirePackage{fancyhdr}
%</ct|mt|le> 
%    \end{macrocode}
% 
% \subsection{Colors}
%
%    \begin{macrocode}
%<*ct|mt|le> 
\definecolor{uacorpbord}{cmyk}     {0.00,1.00,0.60,0.37}
\definecolor{uacorpblue}{cmyk}     {1.00,0.25,0.00,0.50}
\definecolor{uacorplightblue}{cmyk}{1.00,0.00,0.08,0.13}
\definecolor{uacorporange}{cmyk}   {0.00,0.32,1.00,0.09}
\definecolor{uaftifresh}{cmyk}     {0.34,1.00,0.00,0.00}
\definecolor{uaftisober}{cmyk}     {0.10,1.00,0.00,0.49}
\definecolor{lightgray}{cmyk}      {0.00,0.00,0.00,0.05}
%</ct|mt|le>
%    \end{macrocode}
%
% \subsection{Babel provisions}
%
% \changes{v1.7}{2016/05/01}{Added babel tags of elements
% of master's thesis title page}
%    \begin{macrocode}
%<*ct|mt|le>
\newcommand{\uaname}{University of Antwerp}
\newcommand{\logoname}{UA_HOR_ENG_CMYK}
\newcommand{\footername}{4E_PMS302_BR_ENG_RGB}
\newcommand{\orname}{of}
\newcommand{\domainname}{uantwerp.be}
\newcommand{\datename}{Date}
\newcommand{\subjectname}{Subject}
\newcommand{\academicyearname}{Academic year}
\newcommand{\masterthesisname}{Master's thesis}
\newcommand{\promotorsname}{Promoters}
\newcommand{\thesisname}{Thesis to obtain the degree of}
\AtBeginDocument{
  \@ifpackageloaded{babel}{
    \addto\captionsdutch{%
      \renewcommand{\uaname}{Universiteit Antwerpen}
      \renewcommand{\logoname}{UA_HOR_NED_CMYK}
      \renewcommand{\footername}{4E_PMS302_BR_NED_RGB}
      \renewcommand{\orname}{van}
      \renewcommand{\domainname}{uantwerpen.be}
      \renewcommand{\subjectname}{Onderwerp}%
      \renewcommand{\datename}{Datum}%
      \renewcommand{\academicyearname}{Academiejaar}
      \renewcommand{\masterthesisname}{Masterproef}
      \renewcommand{\promotorsname}{Promotoren}
      \renewcommand{\thesisname}{Proefschrift tot het behalen van de
        graad van}
    }
    \addto\captionsgerman{%
      \renewcommand{\uaname}{Universit\"at Antwerpen}
      \renewcommand{\logoname}{UA_HOR_DUI_CMYK}
      \renewcommand{\footername}{4E_PMS302_BR_NED_RGB}
      \renewcommand{\orname}{von}
      \renewcommand{\domainname}{uantwerpen.be}
      \renewcommand{\subjectname}{Betreff}%
      \renewcommand{\datename}{Datum}%
      \renewcommand{\academicyearname}{Akademisches Jahr}
      \renewcommand{\masterthesisname}{Masterdissertation}
      \renewcommand{\promotorsname}{Veranstalter}
      \renewcommand{\thesisname}{Dissertation zur Erreichung des
        Grades der}
    }
    \addto\captionsfrench{%
      \renewcommand{\uaname}{Universit\'e d'Anvers}
      \renewcommand{\logoname}{UA_HOR_FRA_CMYK}
      \renewcommand{\footername}{4E_PMS302_BR_ENG_RGB}
      \renewcommand{\orname}{de}
      \renewcommand{\domainname}{uanvers.be}
      \renewcommand{\subjectname}{Objet}%
      \renewcommand{\datename}{Date}%
      \renewcommand{\academicyearname}{Ann\'ee acad\'emique}
      \renewcommand{\masterthesisname}{Th\`ese de master}
      \renewcommand{\promotorsname}{Promoteurs}
      \renewcommand{\thesisname}{Th\`ese \`a l'atteinte du degr\'e de}
    }
    \addto\captionsspanish{%
      \renewcommand{\uaname}{Universidad de Amberes}
      \renewcommand{\logoname}{UA_HOR_SPA_CMYK}
      \renewcommand{\footername}{4E_PMS302_BR_ENG_RGB}
      \renewcommand{\orname}{de}
      \renewcommand{\domainname}{uantwerp.be}
      \renewcommand{\subjectname}{Asunto}%
      \renewcommand{\datename}{Fecha}%
      \renewcommand{\academicyearname}{A\~no acad\'emico}
      \renewcommand{\masterthesisname}{Tesis de maestr\'\i{}a}
      \renewcommand{\promotorsname}{Promotores}
      \renewcommand{\thesisname}{Disertaci\'on a la consecuci\'on del
        grado de}
    }
  }
  {}
}
%</ct|mt|le>
%    \end{macrocode}
%
% \subsection{Tags}
% 
% \begin{macro}{\facultyacronym}
%   The |facultyacronym| sets the faculty acronym tag
%   |\@facultyacronym| that is used in the header/footer
%   information. The correct acronym also sets the faculty's name
%   correctly.
%
%    \begin{macrocode}
%<*ct|mt|le>      
\newcommand{\@facultyacronym}{}
\newcommand{\@faculty}{< Specify faculty using \tbs{}facultyacronym\{ABC\} >}
\newcommand{\facultyacronym}[1]{
  \renewcommand{\@facultyacronym}{#1}
  \ifthenelse{\equal{#1}{CPG}}{\renewcommand\@faculty{Centrum
      Pieter Gillis}}{
  \ifthenelse{\equal{#1}{FBD}}{\renewcommand\@faculty{Faculteit
      Farmaceutische, Biomedische en Diergeneeskundige Wetenschappen}}{
  \ifthenelse{\equal{#1}{GGW}}{\renewcommand\@faculty{Faculteit
      Geneeskunde en Gezondheidswetenschappen}}{
  \ifthenelse{\equal{#1}{IOB}}{\renewcommand\@faculty{Instituut
      voor Ontwikkelingsbeleid- en beheer}}{
  \ifthenelse{\equal{#1}{IOIW}}{\renewcommand\@faculty{Instituut
      voor Onderwijs- en Informatiewetenschappen}}{
  \ifthenelse{\equal{#1}{LW}}{\renewcommand\@faculty{Faculteit
      Letteren en Wijsbegeerte}}{
  \ifthenelse{\equal{#1}{OW}}{\renewcommand\@faculty{Faculteit
      Ontwerpwetenschappen}}{
  \ifthenelse{\equal{#1}{SW}}{\renewcommand\@faculty{Faculteit
      Sociale Wetenschappen}}{
  \ifthenelse{\equal{#1}{REC}}{\renewcommand\@faculty{Faculteit
      Rechten}}{
  \ifthenelse{\equal{#1}{TEW}}{\renewcommand\@faculty{Faculteit
      Toegepaste Economische Wetenschappen}}{
  \ifthenelse{\equal{#1}{TI}}{\renewcommand\@faculty{Faculteit
      Toegepaste Ingenieurswetenschappen}}{
  \ifthenelse{\equal{#1}{WET}}{\renewcommand\@faculty{Faculteit
      Wetenschappen}}{
    \errmessage{Error: wrong faculty acronym; choose one of FBD, GGW,
      LW, OW, PSW, REC, TEW, TI, WET}}}}}}}}}}}}}}
%</ct|mt|le>
%    \end{macrocode}
% \end{macro}
% 
% \begin{macro}{\title}
%   The |title| tag is native to \LaTeX{}. It sets the |\@title| tag
%   that will be used on the title page.
%   However, in view of the qr trouble, we fiddle a little with it.
%   In case the title contains accented characters, you also
%   need to provide a qr version in full unicode (so without the
%   traditional \LaTeX{} accented characters.)
%
%    \begin{macrocode}
%<*ct>      
\newcommand{\@qrtitle}{}
\renewcommand{\title}[1]{%
  \renewcommand\@title{#1}
  \ifx\@qrtitle\@emptymacro
  \renewcommand\@qrtitle{#1}
  \fi
}
\newcommand{\qrtitle}[1]{%
  \renewcommand\@qrtitle{#1}
}
%</ct> 
%    \end{macrocode}
% \end{macro}
% 
% \begin{macro}{\subtitle}
%   This macro sets the |\@subtitle| tag that later will be used on
%   the title page, in the header/footer and to set the appropriate
%   |hyperref| tag.
%    \begin{macrocode}
%<*ct>      
\newcommand{\@subtitle}{}
\newcommand{\@qrsubtitle}{}
\newcommand{\subtitle}[1]{%
  \renewcommand\@subtitle{#1}
  \ifx\@qrsubtitle\@emptymacro
    \renewcommand\@qrsubtitle{#1}
  \fi
}
\newcommand{\qrsubtitle}[1]{%
  \renewcommand\@qrsubtitle{#1}
}
%</ct> 
%    \end{macrocode}
% \end{macro}
% 
% \begin{macro}{\author}
%   The |author| tag is native to \LaTeX{}. It sets the |\@author|
%   tag that will be used on the title page.
%   However, in view of the qr trouble, we fiddle a little with it.
%   In case the title contains accented characters, you also
%   need to provide a qr version in full unicode (so without the
%   traditional \LaTeX{} accented characters.)
%
%    \begin{macrocode}
%<*ct>      
\newcommand{\@qrauthor}{}
\renewcommand{\author}[1]{%
  \renewcommand\@author{#1}
  \ifx\@qrauthor\@emptymacro
  \renewcommand\@qrauthor{#1}
  \fi
}
\newcommand{\qrauthor}[1]{%
  \renewcommand\@qrauthor{#1}
}
%</ct> 
%    \end{macrocode}
% \end{macro}
% 
% \begin{macro}{\courseversion}
%   This macro sets the |\@courseversion| tag that later will be used
%   on the title page and in the header/footer.
%    \begin{macrocode}
%<*ct>      
\newcommand{\@courseversion}{}
\newcommand{\courseversion}[1]{\renewcommand{\@courseversion}{#1}}
%</ct> 
%    \end{macrocode}
% \end{macro}
% 
% \begin{macro}{\versionyear}
%   This macro sets the |\@versionyear| tag that later will be used on
%   the title page and in the copyright message.
%    \begin{macrocode}
%<*ct>      
\newcommand{\@versionyear}{}
\newcommand{\versionyear}[1]{\renewcommand{\@versionyear}{#1}}
%</ct> 
%    \end{macrocode}
% \end{macro}
% 
% \begin{macro}{\publisher}
%   This macro sets the |\@publisher| tag that later will be used on
%   the title page.
%    \begin{macrocode}
%<*ct>      
\newcommand{\@publisher}{\uaname\\
Cursusdienst en reprografie\\
Campus Groenenborger, G.U.027\\
Groenenborgerlaan 171\\
2020 Antwerpen\\
T +32 3 265 32 15\\
F + 32 3 233 32 27\\
E cursusdienst.cgb@uantwerpen.be}
\newcommand{\publisher}[1]{\renewcommand{\@publisher}{#1}}
%</ct> 
%    \end{macrocode}
% \end{macro}
% 
% \begin{macro}{\publishercode}
%   This macro sets the |\@publishercode| tag that later will be used on
%   the title page.
%    \begin{macrocode}
%<*ct>      
\newcommand{\@publishercode}{}
\newcommand{\publishercode}[1]{\renewcommand{\@publishercode}{#1}}
%</ct> 
%    \end{macrocode}
% \end{macro}
% 
% \begin{macro}{\lecturer}
%   This macro sets many |\@lecturer| tags (max. 4) that later will be used on
%   the title page. If there is only one teaching lecturer one can
%   use the convenient shorthand without counter.
%   In case the lecturer's name contains accented characters, you also
%   need to provide a qr version in full unicode (so without the
%   traditional \LaTeX{} accented characters.)
%    \begin{macrocode}
%<*ct>
\newcommand{\@lectureri}{}
\newcommand{\@lecturerii}{}
\newcommand{\@lectureriii}{}
\newcommand{\@lectureriv}{}
\newcommand{\@qrlectureri}{}
\newcommand{\@qrlecturerii}{}
\newcommand{\@qrlectureriii}{}
\newcommand{\@qrlectureriv}{}
\newcommand{\lecturer}[1]{
  \renewcommand{\@lectureri}{#1}
  \ifx\@qrlectureri\@emptymacro
  \renewcommand\@qrlectureri{#1}
  \fi
}
\newcommand{\qrlecturer}[1]{
  \renewcommand\@qrlectureri{#1}
}
\newcommand{\lectureri}[1]{
  \renewcommand{\@lectureri}{#1}
  \ifx\@qrlectureri\@emptymacro
  \renewcommand\@qrlectureri{#1}
  \fi
}
\newcommand{\qrlectureri}[1]{
  \renewcommand\@qrlectureri{#1}
}
\newcommand{\lecturerii}[1]{
  \renewcommand{\@lecturerii}{#1}
  \ifx\@qrlecturerii\@emptymacro
  \renewcommand\@qrlecturerii{#1}
  \fi
}
\newcommand{\qrlecturerii}[1]{
  \renewcommand\@qrlecturerii{#1}
}
\newcommand{\lectureriii}[1]{
  \renewcommand{\@lectureriii}{#1}
  \ifx\@qrlectureriii\@emptymacro
  \renewcommand\@qrlectureriii{#1}
  \fi
}
\newcommand{\qrlectureriii}[1]{
  \renewcommand\@qrlectureriii{#1}
}
\newcommand{\lectureriv}[1]{
  \renewcommand{\@lectureriv}{#1}
  \ifx\@qrlectureriv\@emptymacro
  \renewcommand\@qrlectureriv{#1}
  \fi
}
\newcommand{\qrlectureriv}[1]{
  \renewcommand\@qrlectureriv{#1}
}
%</ct> 
%    \end{macrocode}
% \end{macro}
% 
% \begin{macro}{\supervisor}
%   This macro sets many |\@supervisor| tags (max. 4) that later will be used on
%   the title page. If there is only one supervisor one can
%   use the convenient shorthand without counter.
%    \begin{macrocode}
%<*mt>      
\newcommand{\@supervisori}{}
\newcommand{\@supervisorii}{}
\newcommand{\@supervisoriii}{}
\newcommand{\@supervisoriv}{}
\newcommand{\supervisor}[1]{\renewcommand{\@supervisori}{#1}}
\newcommand{\supervisori}[1]{\renewcommand{\@supervisori}{#1}}
\newcommand{\supervisorii}[1]{\renewcommand{\@supervisorii}{#1}}
\newcommand{\supervisoriii}[1]{\renewcommand{\@supervisoriii}{#1}}
\newcommand{\supervisoriv}[1]{\renewcommand{\@supervisoriv}{#1}}
%</mt> 
%    \end{macrocode}
% \end{macro}
% 
% 
% \begin{macro}{\programme}
%   This macro sets the |\@programme| tags that later will
%   be used on the title page. The involved way of repeatedly calling
%   the renewcommand to set the tags is required for inclusion of the
%   data as QR data.
%
%    \begin{macrocode}
%<*ct>      
\newcommand{\@programmet}{} % type
\newcommand{\@programmec}{} % class
\newcommand{\@programmecqr}{} % class for qr code
\newcommand{\@programmes}{} % class
\newcommand{\@programmeq}{} % qualifier
\newcommand{\programme}[3]{%
    \ifthenelse{\equal{#1}{BA}}%
    {\renewcommand{\@programmet}{Bachelor of Science in de }}{%
    \ifthenelse{\equal{#1}{MA}}%
    {\renewcommand{\@programmet}{Master of Science in de }}{%
    \ifthenelse{\equal{#1}{VP}}%
    {\renewcommand{\@programmet}{Voorbereidingsprogramma voor Master of Science in de }}{%
    \ifthenelse{\equal{#1}{SP}}%
    {\renewcommand{\@programmet}{Schakelprogramma voor Master of Science in de }}{%
    \ifthenelse{\equal{#1}{FREE}}%
    {}{
    \errmessage{Error in 1st arg of macro programme[3]: invalid
      programme type!}}}}}}%
    %
    \ifthenelse{\equal{#2}{IW}}%
    {\renewcommand{\@programmec}{industri\"ele wetenschappen}
     \renewcommand{\@programmecqr}{industriële wetenschappen}}{
    \ifthenelse{\equal{#2}{}}%
    {}{
    \errmessage{{Error in 2nd arg of macro programme[3]: invalid 
      programme class! }}}}%
    %
    \ifthenelse{\equal{#3}{BK}}%
    {\renewcommand{\@programmeq}{bouwkunde}}{%
    \ifthenelse{\equal{#3}{CH}}%
    {\renewcommand{\@programmeq}{chemie}}{%
    \ifthenelse{\equal{#3}{BCH}}%
    {\renewcommand{\@programmeq}{biochemie}}{%
    \ifthenelse{\equal{#3}{EM}}%
    {\renewcommand{\@programmeq}{elektromechanica}}{%
    \ifthenelse{\equal{#3}{EI}}%
    {\renewcommand{\@programmeq}{elektronica-ICT}}{%
    \ifthenelse{\equal{#3}{}}%
    {}{%
    \ifthenelse{\equal{#1}{FREE}}
    {\renewcommand{\@programmeq}{#3}}{
    \errmessage{Error in 3rd arg to macro programme[3]: invalid
      programme qualifier}}}}}}}}% 
    %
    \ifthenelse{\equal{#2}{IW}\and\not\equal{#3}{}}
    {\renewcommand{\@programmes}{: }}{}
}
%</ct> 
%    \end{macrocode}
% \end{macro}
% 
% \begin{macro}{\course}
%   This macro sets many |\@coursecode| and |\@course| tags (max. 4)
%   that later will 
%   be used on the title page. If there is only one course code
%   one can use the convenient shorthand without counter.
%    \begin{macrocode}
%<*ct>      
\newcommand{\@coursecodei}{}
\newcommand{\@coursecodeii}{}
\newcommand{\@coursecodeiii}{}
\newcommand{\@coursecodeiv}{}
\newcommand{\@coursei}{}
\newcommand{\@courseii}{}
\newcommand{\@courseiii}{}
\newcommand{\@courseiv}{}
\newcommand{\@qrcoursei}{}
\newcommand{\@qrcourseii}{}
\newcommand{\@qrcourseiii}{}
\newcommand{\@qrcourseiv}{}
\newcommand{\course}[2]{
  \renewcommand{\@coursecodei}{#1}
  \renewcommand{\@coursei}{#2}
  \ifx\@qrcoursei\@emptymacro
  \renewcommand{\@qrcoursei}{#2}
  \fi
}
\newcommand{\qrcourse}[1]{
  \renewcommand{\@qrcoursei}{#1}
}
\newcommand{\coursei}[2]{
  \renewcommand{\@coursecodei}{#1}
  \renewcommand{\@coursei}{#2}
  \ifx\@qrcoursei\@emptymacro
  \renewcommand{\@qrcoursei}{#2}
  \fi
}
\newcommand{\qrcoursei}[1]{
  \renewcommand{\@qrcoursei}{#1}
}
\newcommand{\courseii}[2]{
  \renewcommand{\@coursecodeii}{#1}
  \renewcommand{\@courseii}{#2}
  \ifx\@qrcourseii\@emptymacro
  \renewcommand{\@qrcourseii}{#2}
  \fi
}
\newcommand{\qrcourseii}[1]{
  \renewcommand{\@qrcourseii}{#1}
}
\newcommand{\courseiii}[2]{
  \renewcommand{\@coursecodeiii}{#1}
  \renewcommand{\@courseiii}{#2}
  \ifx\@qrcourseiii\@emptymacro
  \renewcommand{\@qrcourseiii}{#2}
  \fi
}
\newcommand{\qrcourseiii}[1]{
  \renewcommand{\@qrcourseiii}{#1}
}
\newcommand{\courseiv}[2]{
  \renewcommand{\@coursecodeiv}{#1}
  \renewcommand{\@courseiv}{#2}
  \ifx\@qrcourseiv\@emptymacro
  \renewcommand{\@qrcourseiv}{#2}
  \fi
}
\newcommand{\qrcourseiv}[1]{
  \renewcommand{\@qrcourseiv}{#1}
}
%</ct> 
%    \end{macrocode}
% \end{macro}
% 
% \begin{macro}{\diploma}
%   This macro sets the |\@diploma| tags that later will
%   be used on the title page. 
%   \changes{v1.1}{2013/05/21}{Fixed typo on programme's name
%   (e-umlaut)}
% \changes{v1.6}{2016/02/04}{Added diploma codes}
%    \begin{macrocode}
%<*mt>
\newcommand{\@diploma}{ERROR}
\newcommand{\diploma}[1]{
  \newcommand{\MoSIW}{Master of Science in de industri\"ele wetenschappen}
  \newcommand{\MoSTEW}{Master of Science in de toegepaste economische wetenschappen}
  \renewcommand{\@diploma}{
    \ifthenelse{\equal{#1}{MA-IW-BK}}
                          {\MoSIW: bouwkunde}{
    \ifthenelse{\equal{#1}{MA-IW-BCH}}
                          {\MoSIW: biochemie}{
    \ifthenelse{\equal{#1}{MA-IW-CH}}
                          {\MoSIW: chemie}{
    \ifthenelse{\equal{#1}{MA-IW-EI}}
                          {\MoSIW: elektronica-ICT}{
    \ifthenelse{\equal{#1}{MA-IW-EI-AE}}
                          {\MoSIW:\\elektronica-ICT, afstudeerrichting automotive engineering}{
    \ifthenelse{\equal{#1}{MA-IW-EI-ICT}}
                          {\MoSIW:\\elektronica-ICT, afstudeerrichting ICT}{
    \ifthenelse{\equal{#1}{MA-IW-EM-AE}}
                          {\MoSIW:\\elektromechanica, afstudeerrichting automotive engineering}{
    \ifthenelse{\equal{#1}{MA-IW-EM-AU}}
                          {\MoSIW:\\elektromechanica, afstudeerrichting automatisering}{
    \ifthenelse{\equal{#1}{MA-IW-EM-EM}}
                          {\MoSIW:\\elektromechanica, afstudeerrichting elektromechanica}{
    \ifthenelse{\equal{#1}{MA-IW-EM-EN}}
                          {\MoSIW:\\elektromechanica, afstudeerrichting energie}{
    \ifthenelse{\equal{#1}{MA-TEW-HI}}
                          {\MoSTEW: handelsingenieur}{
    \ifthenelse{\equal{#1}{MA-TEW-HIBI}}
                          {\MoSTEW:\\handelsingenieur in de beleidsinformatica}{
    \ifthenelse{\equal{#1}{MA-TEW-EB}}
                          {\MoSTEW: economisch beleid}{
    \ifthenelse{\equal{#1}{MA-TEW-BK}}
                          {\MoSTEW: bedrijfskunde}
    {\errmessage{Error in argument to macro diploma: must be one of
        MA-IW-XXX with XXX one of BCH, CH, EI, EI-AE, EI-ICT, EM-AE,
        EM-AU, EM-EM, EM-EN, MA-TEW-YYY with YYY one of HI, HIBI, EB, BK! <<}}}}}}}}}}}}}}}
}
}
%</mt> 
%    \end{macrocode}
% \end{macro}
% 
% \begin{macro}{\defensedate}
%   This macro sets the |\@defensedate| tags that later will
%   be used on the title page. 
%    \begin{macrocode}
%<*mt>      
\newcommand{\@defensedate}{ERROR}
\newcommand{\defensedate}[1]{\renewcommand{\@defensedate}{#1}}
%</mt> 
%    \end{macrocode}
% \end{macro}
% 
% \begin{macro}{\defenselocation}
%   This macro sets the |\@defenselocation| tags that later will
%   be used on the title page. 
%    \begin{macrocode}
%<*mt>      
\newcommand{\@defenselocation}{Antwerpen}
\newcommand{\defenselocation}[1]{\renewcommand{\@defenselocation}{#1}}
%</mt> 
%    \end{macrocode}
% \end{macro}
% 
% \begin{macro}{\academicyear}
%   This macro sets the |\@academicyear| tag that later will be used on
%   the title page.
%    \begin{macrocode}
%<*ct|mt>      
\newcommand{\@academicyear}{XXX-YYYY}
\newcommand{\academicyear}[1]{\renewcommand{\@academicyear}{#1}}
%</ct|mt> 
%    \end{macrocode}
% \end{macro}
% 
% \begin{macro}{\copyrightnotices}
%   This macro sets the |\@copyrightnotices| tag that later will be
%   used on the back of the title page.
%    \begin{macrocode}
%<*ct>      
\newcommand{\@copyrightnotices}{}
\newcommand{\copyrightnotices}[1]{\renewcommand{\@copyrightnotices}{#1}}
%</ct> 
%    \end{macrocode}
% \end{macro}
%
% \begin{macro}{\sender}
% This macro sets the |\@sender| and |\@senderrole| tags that will be
% used in the letter's heading text.
%    \begin{macrocode}
%<*le>
\newcommand{\@sender}{< Specify sender using
  \tbs{}sender\{name\}\{role\} >}
\newcommand{\@senderrole}{~}
\newcommand{\sender}[2]{\renewcommand{\@sender}{#1}\renewcommand{\@senderrole}{#2}}
%</le>
%    \end{macrocode}
% \end{macro}
%
% \begin{macro}{\logo}
%   This macro sets the |\@logo| tag that will be used to load a
%   graphics file with that name.
%    \begin{macrocode}
%<*le>
\newcommand{\@logo}{\logoname}
\newcommand{\logo}[1]{\renewcommand{\@unit}{#1}}
%</le>
%    \end{macrocode}
% \end{macro}
%
% \begin{macro}{\unit}
%   This macro sets the |\@unit| tag that will be used in the letter's
%   heading text.
%    \begin{macrocode}
%<*le>
\newcommand{\@unit}{}
\newcommand{\unit}[1]{\renewcommand{\@unit}{#1}}
%</le>
%    \end{macrocode}
% \end{macro}
%
% \begin{macro}{\email}
%   This macro sets the |\@emailuser| and |\@emaildomain| tags that
%   will be used in the letter's heading text. This split construction
%   was used to overcome problems with the |@| sign.
%    \begin{macrocode}
%<*le>
\newcommand{\@emailuser}{}
\newcommand{\@emaildomain}{}
\newcommand{\email}[2]{\renewcommand{\@emailuser}{#1}\renewcommand{\@emaildomain}{#2}}
%</le>
%    \end{macrocode}
% \end{macro}
%
% \begin{macro}{\phone}
%   This macro sets the |\@phone| tag that will be used in the letter's
%   heading text.
%    \begin{macrocode}
%<*le>
\newcommand{\@phone}{}
\newcommand{\phone}[1]{\renewcommand{\@phone}{#1}}
%</le>
%    \end{macrocode}
% \end{macro}
%
% \begin{macro}{\fax}
%   This macro sets the |\@fax| tag that will be used in the letter's
%   heading text.
%    \begin{macrocode}
%<*le>
\newcommand{\@fax}{}
\newcommand{\fax}[1]{\renewcommand{\@fax}{#1}}
%</le>
%    \end{macrocode}
% \end{macro}
%
%
% \begin{macro}{\mobile}
%   This macro sets the |\@mobile| tag that will be used in the letter's
%   heading text.
%    \begin{macrocode}
%<*le>
\newcommand{\@mobile}{}
\newcommand{\mobile}[1]{\renewcommand{\@mobile}{#1}}
%</le>
%    \end{macrocode}
% \end{macro}
%
%
% \begin{macro}{\returnaddress}
%   This macro sets the |\@returnaddress| tag that will be used in the letter's
%   heading text (in the area of the envelope's window).
%    \begin{macrocode}
%<*le>
\newcommand{\@returnaddress}{<specify return-address using \tbs\{single-line-return-address\}>}
\renewcommand{\returnaddress}[1]{\renewcommand{\@returnaddress}{#1}}
%</le>
%    \end{macrocode}
% \end{macro}
%
%
% \begin{macro}{\to}
%   This macro sets the |\@to| tag that will be used in the letter's
%   heading text (in the area of the envelope's window).
%    \begin{macrocode}
%<*le>
\newcommand{\@to}{<Specify addressee using \tbs to\{name\}>}
\renewcommand{\to}[1]{\renewcommand{\@to}{#1}}
%</le>
%    \end{macrocode}
% \end{macro}
%
%
% \begin{macro}{\toorganization}
%   This macro sets the |\@toorganization| tag that will be used in
%   the letter's heading text (in the area of the envelope's window).
%    \begin{macrocode}
%<*le>
\newcommand{\@toorganization}{<Specify organization using
  \tbs toorganization\{\}>}
\newcommand{\toorganization}[1]{\renewcommand{\@toorganization}{#1}}
%</le>
%    \end{macrocode}
% \end{macro}
%
% \begin{macro}{\toaddress}
%   This macro sets the |\@toaddress| tag that will be used in
%   the letter's heading text (in the area of the envelope's window).
%    \begin{macrocode}
%<*le>
\newcommand{\@toaddress}{<Specify (multiline) destination address\\using \tbs toaddress\{\}>}
\newcommand{\toaddress}[1]{\renewcommand{\@toaddress}{#1}}
%</le>
%    \end{macrocode}
% \end{macro}
%
% \begin{macro}{\subject}
%   This macro sets the |\@subject| tag that will be used in
%   the letter's heading text.
%    \begin{macrocode}
%<*le>
\newcommand{\@subject}{-}
\newcommand*{\subject}[1]{\renewcommand{\@subject}{#1}}
%</le>
%    \end{macrocode}
% \end{macro}
%
% \begin{macro}{\opening}
%   This macro is much a do about nothing, but I prefer to do it this
%   way for historic reasons.
%    \begin{macrocode}
%<*le>
\renewcommand*{\opening}[1]{#1}
%</le>
%    \end{macrocode}
% \end{macro}
%
% \begin{macro}{\closing}
%   This macro sets the |\@closing| tag that will be used to finish
%   the letter.
%    \begin{macrocode}
%<*le>
\newcommand{\@closing}{<specify a closing formula using \tbs closing\{\}>}
\renewcommand*{\closing}[1]{\renewcommand{\@closing}{#1}}
%</le>
%    \end{macrocode}
% \end{macro}
%
% \begin{macro}{\signature}
% \changes{v1.8}{2017/01/08}{Added signature}
%   This macro sets the |\@signature| tag that will be used to finish
%   the letter. By default this corresponds to a decent amount of vertical white space
%    \begin{macrocode}
%<*le>
\newcommand{\@signature}{\vspace*{8ex}}
\renewcommand*{\signature}[1]{\renewcommand{\@signature}{#1}}
%</le>
%    \end{macrocode}
% \end{macro}
% 
%
% \begin{macro}{\carboncopy}
%   This macro will set the |\@carboncopy| tag that will be used in
%   the trailer of the letter.
%    \begin{macrocode}
%<*le>
\newcommand{\@carboncopy}{}
\newcommand{\carboncopy}[1]{\renewcommand{\@carboncopy}{#1}}
%</le>
%    \end{macrocode}
% \end{macro}
%
% \begin{macro}{\enclosed}
%   This macro will set the |\@enclosed| tag that will be used in
%   the trailer of the letter.
%    \begin{macrocode}
%<*le>
\newcommand{\@enclosed}{}
\newcommand{\enclosed}[1]{\renewcommand{\@enclosed}{#1}}
%</le>
%    \end{macrocode}
% \end{macro}
%
% \begin{macro}{\address}
%   This macro will set the |\@address| tag that will be used in
%   the letter's heading text (in the area of the envelope's window).
%    \begin{macrocode}
%<*le>
\newcommand{\@address}{< Put your multi-line address here\\using
  \tbs address\{\} >}
\renewcommand{\address}[1]{\renewcommand{\@address}{#1}}
%</le>
%    \end{macrocode}
% \end{macro}
%
% % \DescribeMacro{\address} [uantwerpenletter] (mandatory)\\
% % Address of the sending unit (or faculty). This can be different from
% % the return address. Newlines are allowed and encouraged.
% %
% % \DescribeMacro{\date} [uantwerpenletter] (optional) \\
% % Date of the letter. If not specified today's date (at the time of
% % running \LaTeX{}) will be used.
% %
% \subsection{Header and Footer}
% The |fancyhdr| package is used to make a decent header and footer.
% The header and footer of the |uantwerpencoursetext| package are defined to be:
%    \begin{macrocode}
%<*ct>      
\if@twoside
\lhead[\thepage]{\slshape\rightmark}
\chead[]{}
\rhead[\slshape\leftmark]{\thepage}
\lfoot[\uaname{} -- \@facultyacronym]{\@courseversion}
\cfoot[]{}
\rfoot[]{\@title{}\@ifmtarg{\@subtitle}{}{ --- \@subtitle}}
\else
\lhead[]{\leftmark}
\chead[]{}
\rhead[]{\thepage}
\lfoot[]{\@courseversion}
\cfoot[]{UAntwerpen--\@facultyacronym}
\rfoot[]{\@title{}}
\fi
%</ct> 
%    \end{macrocode}
% 
% The header and footer of the |uantwerpenmasterthesis| package are
% defined to be: 
%    \begin{macrocode}
%<*mt>
\if@twoside
  \lhead[\thepage]{\slshape\rightmark}
  \chead[]{}
  \rhead[\slshape\leftmark]{\thepage}
  \lfoot[\uaname{} -- \@facultyacronym]{}
  \cfoot[]{}
  \rfoot[]{\@title{}}
\else
  \lhead[]{\leftmark}
  \chead[]{}
  \rhead[]{\thepage}
  \lfoot[]{}
  \cfoot[]{UAntwerpen--\@facultyacronym}
  \rfoot[]{\@title{}}
\fi
%</mt>
%    \end{macrocode}
%
% The header and footer of the |uantwerpenletter| package are
% defined to be: 
%    \begin{macrocode}
%<*le>
\lhead[]{}
\chead[]{}
\rhead[]{}
\lfoot[\small\textcolor{gray}{\@date}]{\textcolor{gray}{\@date}}
\cfoot[]{}
\rfoot[\small\textcolor{gray}{\pagename~\thepage~\orname~\pageref{lastpage}}]{\small\textcolor{gray}{\pagename~\thepage~\orname~\pageref{lastpage}}}
%</le>
%    \end{macrocode}
%
% Some common code remains:
% \changes{v1.1}{2013/05/28}{Made raggedright conditional on option
% 'filled', because it can't be undone}
% \changes{v1.2}{2014/08/22}{Increased headheight to please Fancyhdr}
%    \begin{macrocode}
%<*ct|mt>
\setlength{\headheight}{13.7pt}
\renewcommand{\headrulewidth}{1pt}
\renewcommand{\footrulewidth}{1pt}
\pagenumbering{arabic}
%</ct|mt>
%<*le>
\renewcommand{\headrulewidth}{0pt}
\renewcommand{\footrulewidth}{0pt}
%</le>
%    \end{macrocode}
%
% Remains to take care of filling
% 
%    \begin{macrocode}
%<*ct|mt|le>
\if@filled\else
  \raggedright
\fi
\raggedbottom
\onecolumn
%</ct|mt|le>
%    \end{macrocode}
%
% \subsection{Copyright notice}
%
% \begin{macro}{\@crnotice}
%   This is the standard text that will be used for the |\@crnotice| tag.
%    \begin{macrocode}
%<*ct>
\newcommand{\@crnotice}{
  This document has been typeset using \LaTeX{} and the
  \texttt{uantwerpencoursetext} class.\\
  \@copyrightnotices

  \@courseversion

  CONFIDENTIAL AND PROPRIETARY.

  \copyright{} \@versionyear{} University of Antwerp, All rights reserved.
}
%</ct>
%    \end{macrocode}
% \end{macro}
%
%
% \subsection{Title page for the masterthesis and coursetext clases}
%
% The title page is generated using the |\maketitle| command. As the 
% book class from which we inherit already defines this command, we
% need to renew it.
%
% The UAntwerpen house style works with large-radius circles.
% We need some math in order to calculate center points an radiuses
% 
% \paragraph{Header circle:} the blue header-circle on top of the
% page. We calculate the center point and the radius based on:
% \begin{itemize}
% \item the fact that the center point is above the page's left edge;
% \item the distance of the intersection points at the left and right
%   edge of the page with respect to the top of the page, $X$ and $Y$
%   respectively;
% \item the width of the page $W$.
% \end{itemize}
% Some simple trigonometry leads to the elevation of the center point
% above the top of the page $Q$ and the radius $R$:
% \begin{eqnarray}
%   R &=& \frac{W^2 + (X-Y)^2}{2(X-Y)}\\
%   Q &=& R - X
% \end{eqnarray}
% We labeled the variables in the code below with a prefix
% |ua@| and a suffix h (from 'header'). We also took the paperheight into account (the page's
% coordinate system origins at the left bottom.
% Therefore:
%    \begin{macrocode}
%<*ct|mt>
\pgfmathsetmacro{\ua@Wh}{\paperwidth}
\pgfmathsetmacro{\ua@Xh}{0.2\paperheight}
\pgfmathsetmacro{\ua@Yh}{0.125\paperheight}
\pgfmathsetmacro{\ua@XMYh}{\ua@Xh-\ua@Yh}
\pgfmathsetmacro{\ua@Rh}{0.5*\ua@Wh/\ua@XMYh*\ua@Wh+0.5*\ua@XMYh}
\pgfmathsetmacro{\ua@Qh}{\ua@Rh-\ua@Xh+\paperheight}
%</ct|mt>
%    \end{macrocode}
%
% \paragraph{Bottom circle A:} the bottom circle with the largest
% radius. We calculate the center point and the radius based on:
% \begin{itemize}
% \item the fact that the center point is at a distance of 3/5 page
% width from the right page edge. We denote that distance by $W$;
% \item the fact that the horizontal bottom tangent line of the circle
% is at a specific height above the bottom page edge, labeled $S$;
% \item the distance of the intersection points at the right edge of
% the page with respect to the horizontal bottom tangent line of the
% circle, denoted as $X-Y$.
% \end{itemize}
% Given these definitions, almost the same equations as for the header
% circle hold:
% \begin{eqnarray}
%   R &=& \frac{W^2 + (X-Y)^2}{2(X-Y)}\\
%   Q &=& R + S
% \end{eqnarray}
% with $R$ the circle's radius, and $Q$ the elevation of the center
% point above the page's bottom.
%
% We labeled the variables in the code below with a prefix
% |ua@| and a suffix ba (from 'bottom a').
% Therefore:
%    \begin{macrocode}
%<*ct|mt>
\pgfmathsetmacro{\ua@Wba}{0.4*\paperwidth}
\pgfmathsetmacro{\ua@Sba}{0.125*\paperheight}
\pgfmathsetmacro{\ua@XMYba}{0.02\paperheight}
\pgfmathsetmacro{\ua@Rba}{0.5*\ua@Wba/\ua@XMYba*\ua@Wba+0.5*\ua@XMYba}
\pgfmathsetmacro{\ua@Qba}{\ua@Rba+\ua@Sba}
%</ct|mt>
%    \end{macrocode}
%
%
% \paragraph{Bottom circle B:} the bottom circle with the smallest
% radius. The same reasoning leads to:
%
%    \begin{macrocode}
%<*ct|mt>
\pgfmathsetmacro{\ua@Wbb}{0.65*\paperwidth}
\pgfmathsetmacro{\ua@Sbb}{0.14*\paperheight}
\pgfmathsetmacro{\ua@XMYbb}{0.06\paperheight}
\pgfmathsetmacro{\ua@Rbb}{0.5*\ua@Wbb/\ua@XMYbb*\ua@Wbb+0.5*\ua@XMYbb}
\pgfmathsetmacro{\ua@Qbb}{\ua@Rbb+\ua@Sbb}
%</ct|mt>
%    \end{macrocode}
%
% \paragraph{Text alignment:} The text is aligned around an imaginary
% vertical line around 2/5 page width distance from the left edge.
%  
%    \begin{macrocode}
%<*ct|mt>
\pgfmathsetmacro{\ua@ll}{0.15*\paperwidth}
\pgfmathsetmacro{\ua@l}{0.4*\paperwidth}
\pgfmathsetmacro{\ua@d}{0.1in}
\pgfmathsetmacro{\ua@rr}{0.85*\paperwidth}
%</ct|mt>
%    \end{macrocode}
%
% Below, one can find the code for the title page of the
% |uantwerpencoursetext| class. We start with setting up some things
% for the (optional) QR mark.
% \changes{v1.3}{2015/12/31}{Changed eso-pic for tikz and background}
%    \begin{macrocode}
%<*ct>
\newcommand\sprtr{ / }
\newcommand\myqrdata{
I: Universiteit Antwerpen\string\n
F: \@faculty\string\n 
P: \@programmet\ \@programmecqr\ \@programmeq\string\n
C: \@coursecodei\ \@qrcoursei
\sprtr
\@coursecodeii\ \@qrcourseii  
\sprtr
\@coursecodeiii\ \@qrcourseiii 
\sprtr
\@coursecodeiv\ \@qrcourseiv 
\string\n
Y: \@academicyear\string\n
T: \@qrtitle\string\n
S: \@qrsubtitle\string\n
A: \@qrauthor\string\n
L: \@qrlectureri
\sprtr
\@qrlecturerii
\sprtr
\@qrlectureriii
\sprtr
\@qrlectureriv}
\newsavebox{\myqrcode}
%</ct>
%    \end{macrocode}
% \begin{macro}{\maketitle}
% And finally, here is the |\maketitle| macro:
%   \changes{v1.7}{2016/05/01}{Embedded babel translations of keywords
%   into title page}
%    \begin{macrocode}
%<*ct>
\renewcommand\maketitle{%
  \pagestyle{empty}
  \if@qr
  \savebox{\myqrcode}[2.7in][t]{
    \begin{pspicture}(2.7in,2.7in)
      \psbarcode{\myqrdata}{width=1.0 height=1.0 encoding=byte eclevel=M}{qrcode}
    \end{pspicture}
  }
  \fi
  \begin{titlepage}
    \if@titlepagetableonly
    Dit is een cursustekst van Universiteit Antwerpen.\\
    Het titelblad dient opgemaakt te worden met de volgende gegevens:
    \begin{center}
      \begin{tabular}{|l|l|}
        \hline
        \bfseries I & Universiteit Antwerpen \\\hline
        \bfseries F & \@faculty \\\hline
        \bfseries P & \parbox[t]{14cm}{\@programmet
          \@programmec\@programmes \@programmeq}  \\\hline
        \bfseries C & \parbox[t]{14cm}{
          \texttt{\@coursecodei} \@coursei\\
          \texttt{\@coursecodeii} \@courseii\\
          \texttt{\@coursecodeiii} \@courseiii\\
          \texttt{\@coursecodeiv} \@courseiv} \\\hline
        \bfseries Y & \@academicyear \\\hline
        \bfseries T & \parbox[t]{14cm}{\@title}\\\hline
        \bfseries S & \parbox[t]{14cm}{\@subtitle}\\\hline
        \bfseries A & \parbox[t]{14cm}{\@author}\\\hline
        \bfseries L & \parbox[t]{14cm}{\@lectureri\\
                                       \@lecturerii\\
                                       \@lectureriii\\
                                       \@lectureriv} \\\hline
      \end{tabular}
    \end{center}
    ~\\
    Waarbij: I = instelleng, F = faculteit, P = programma, C =
    cursusmodule, T = titel, S = subtitel, A = auteur(s), L =
    lesgever(s)\\~\\
    \if@qr
    Deze informatie is ook gecodeerd in de onderstaande QR-code.\\
    \vspace*{1cm}
    \usebox{\myqrcode}
    \fi
    \else
    \begin{tikzpicture}[remember picture,overlay]
      \node at (current page.center) {
        \begin{tikzpicture}[inner sep=0pt]
          \clip (0,0) rectangle(\paperwidth,\paperheight);
          \if@titlepagenoartwork\else
          \filldraw [uacorpbord] (0.55\paperwidth,\ua@Qba pt) circle (\ua@Rba pt);
          \filldraw [white] (0.35\paperwidth,\ua@Qbb pt) circle (\ua@Rbb pt);
          \filldraw [uacorpblue] (0,\ua@Qh pt) circle (\ua@Rh pt);
          \path 
          (0.95\paperwidth,0.1\paperheight) 
          node [anchor=north east] {
            \includegraphics[width=0.25\paperwidth]{\logoname}};
          \fi
          \path
          (\ua@l pt -\ua@d pt,0.77\paperheight) 
          node [anchor=north east, text width=0.35\paperwidth] {
            \begin{flushright}
              \uppercase\expandafter{\uaname}\\~\\
              \academicyearname{} \@academicyear
            \end{flushright}}
          (\ua@l pt +\ua@d pt,0.65\paperheight)
          node [anchor = north west, text width = 0.55\paperwidth] {
            {\large \@faculty{}}\\[0.05\paperheight]
            {\huge \bf \@title{}}\\[2ex]
            {\Large \bf \@subtitle{}}\\[3ex]
            {\Large \bf \@author{}}}
          (\ua@l pt +\ua@d pt,0.45\paperheight)
          node [anchor = north west, text width = 0.55\paperwidth] {
            \@lectureri~\\
            \@lecturerii~\\
            \@lectureriii~\\
            \@lectureriv}
          (\ua@l pt +\ua@d pt,0.35\paperheight)
          node [anchor = north west, text width = 0.55\paperwidth] {
            {\bf \@programmet \\ \@programmec\@programmes  \@programmeq}}
          (\ua@l pt +\ua@d pt,0.30\paperheight)
          node [anchor = north west, text width = 0.55\paperwidth] {
            \texttt{\@publishercode}~\\~\\
            \texttt{\@coursecodei{}} \@coursei~\\
            \texttt{\@coursecodeii{}} \@courseii~\\
            \texttt{\@coursecodeiii{}} \@courseiii~\\
            \texttt{\@coursecodeiv{}} \@courseiv}
          (\ua@l pt - 2\ua@d pt, 0.368\paperheight)
          node [anchor = north east, text width = 0.35\paperwidth] {
            {\normalsize\begin{flushright}
                \@publisher{}
              \end{flushright}}};
        \end{tikzpicture}
      };
    \end{tikzpicture}
    \fi
  \end{titlepage}%
  \clearpage
  \vspace*{\stretch{1}}
  \@crnotice
  \clearpage
  \setcounter{footnote}{0}%
  \global\let\thanks\relax
  \global\let\maketitle\relax
  \global\let\@thanks\@empty
  \global\let\title\relax
  \global\let\author\relax
  \global\let\date\relax
  \global\let\and\relax
  \pagestyle{fancy}
  \thispagestyle{empty}
}
%</ct> 
%    \end{macrocode}
% \end{macro}
%
% \begin{macro}{\makefinalpage}
%   Below, one can find the code for the final page of the
%   |uantwerpencoursetext| class:
%   \changes{v1.3}{2015/12/31}{Changed eso-pic for tikz}
%   \changes{v1.8}{2016/07/06}{Inserted blank one but last page}
%    \begin{macrocode}
%<*ct>
\newcommand\makefinalpage{
  \if@titlepagetableonly
  \else
  \cleardoublepage
  \thispagestyle{empty}
  ~% intentionally blank page
  \clearpage
  \thispagestyle{empty}
  \begin{tikzpicture}[remember picture,overlay]
    \node at (current page.center) {
      \begin{tikzpicture}[inner sep=0pt]
        \clip (0,0) rectangle(\paperwidth,\paperheight);
        \if@titlepagenoartwork\else
        \filldraw [uacorpblue] (\paperwidth,\ua@Qh pt) circle (\ua@Rh pt);
        \fi
        \path 
        (0.1\paperwidth,0.1\paperheight)
        node [anchor = south west, text width = 0.6\paperwidth] {
          CONFIDENTIAL AND PROPRIETARY\\~\\

          \copyright{} \@versionyear{} \uaname, 
          All rights reserved.}
        (0.9\paperwidth,0.1\paperwidth)
        node [anchor = south east] {
          \usebox{\myqrcode}
        };
      \end{tikzpicture}
    };
  \end{tikzpicture}
  \fi
}
%</ct> 
%    \end{macrocode}
% \end{macro}
% 
% \begin{macro}{\maketitle}
%   And next, the code for the title page of the
%   |uantwerpenmasterthesis| class: 
%   \changes{v1.3}{2015/12/31}{Changed eso-pic for tikz}
%   \changes{v1.7}{2016/05/01}{Embedded babel translations of keywords
%   into title page}
%    \begin{macrocode}
%<*mt>      
\renewcommand\maketitle{%
  \pagestyle{empty}
  \begin{titlepage}
    \begin{tikzpicture}[remember picture,overlay]
      \node at (current page.center) {
        \begin{tikzpicture}[inner sep=0pt]
          \clip (0,0) rectangle(\paperwidth,\paperheight);
          \if@titlepagenoartwork\else
          \filldraw [lightgray] (0.55\paperwidth,\ua@Qba pt) circle (\ua@Rba pt);
          \filldraw [white] (0.35\paperwidth,\ua@Qbb pt) circle (\ua@Rbb pt);
          \filldraw [lightgray] (0,\ua@Qh pt) circle (\ua@Rh pt);
          \path 
          (0.95\paperwidth,0.1\paperheight) 
          node [anchor=north east] {
            \includegraphics[width=0.25\paperwidth]{\logoname}};
          \fi
          \path
          (\ua@ll pt,0.77\paperheight) 
          node [anchor=north west, text width=0.7\paperwidth] {
            \uppercase\expandafter{\uaname}\\~\\
            \academicyearname{} \@academicyear\\~\\
            {\large \@faculty{}}\\~\\
            \masterthesisname{}\\[0.05\paperheight]
            {\Large \bf \@title{}}}
          (\ua@ll pt,0.55\paperheight)
          node [anchor = north west, text width = 0.7\paperwidth] {
            {\large \bf \@author{}}\\~\\~\\
            \begin{tabular}{@{}p{2.7cm}p{10.8cm}}
              \textbf{\promotorsname{}:} 
              & \@supervisori \\
              & \@supervisorii \\
              & \@supervisoriii \\
              & \@supervisoriv
            \end{tabular}
          }
          (\ua@ll pt,0.3\paperheight) node [anchor=north west, text width = 0.7\paperwidth]{%
            \thesisname\\
            \@diploma\\
            \@defenselocation, \@defensedate
          };
        \end{tikzpicture}
      };
    \end{tikzpicture}
  \end{titlepage}%
  \clearpage
  \setcounter{footnote}{0}%
  \global\let\thanks\relax
  \global\let\maketitle\relax
  \global\let\@thanks\@empty
  \global\let\title\relax
  \global\let\author\relax
  \global\let\date\relax
  \global\let\and\relax
  \pagestyle{fancy}
  \thispagestyle{empty}
}
%</mt>
%    \end{macrocode}
% \end{macro}
%
% \begin{macro}{\makefinalpage}
%   Below, one can find the code for the final page of the
%   |uantwerpenmasterthesis| class: 
%   \changes{v1.3}{2015/12/31}{Changed eso-pic for tikz}
%    \begin{macrocode}
%<*mt>
\newcommand\makefinalpage{
  \cleardoublepage
  \thispagestyle{empty}
  \begin{tikzpicture}[remember picture,overlay]
    \node at (current page.center) {
      \begin{tikzpicture}[inner sep=0pt]
        \clip (0,0) rectangle(\paperwidth,\paperheight);
        \if@titlepagenoartwork\else
        \filldraw [lightgray] (\paperwidth,\ua@Qh pt) circle (\ua@Rh pt);
        \fi
      \end{tikzpicture}
    };
  \end{tikzpicture}
}
%</mt> 
%    \end{macrocode}
% \end{macro}
%
% \subsection{Letter}
%
% \subsubsection{Bottom cords}
% \changes{v1.5}{2016/01/11}{Removed documentation}
% The graphical footer of the page is provided through PDF/PS
% includes in a Dutch and a non-dutch version.
%
% \subsubsection{Title Page}
%
% \begin{macro}{\maketitle}
% The top of the letter is generated using the |\maketitle|
% command. 
%
% \changes{v1.5}{2016/01/11}{Implemented new footer}
% \changes{v1.8}{2016/01/08}{Allowed for empty senderrole}
%    \begin{macrocode}
%<*le>
\newcommand\maketitle{%
  \pagestyle{fancy}
  \thispagestyle{empty}
  \begin{tikzpicture}[overlay,remember picture]
    \path (current page.north west) +(1.8cm,-1.2cm) 
    node[anchor=north west] {
      \includegraphics[height=1.1cm]{\@logo} 
    };
    \draw (current page.north west)
    +(1cm,-98mm) -- +(1.5cm,-98mm);
    % 90 x 45
    % pos: 20mm van kant, 15mm van onderkant
    \path (current page.north east) +(-100mm,-65mm) 
    node[anchor=west,text width=80mm,align=left] {
      \scriptsize\textcolor{gray}{\uline{\@returnaddress}}\\*
      \normalsize\@to\\*
      \@toorganization\\*
      \@toaddress
    };
  \end{tikzpicture}
  \begin{tikzpicture}[overlay,remember picture]
    \path (current page.south west) +(0,-0.7cm) 
    node[anchor=south west,inner sep=0pt, outer sep=0pt] 
    {\includegraphics{\footername}};
  \end{tikzpicture}
  ~\\[3ex]
  \textcolor{uacorpblue}{\bf \@sender}
  \ifx\@senderrole\@emptymacro\\[1.75ex]\else \\*\@senderrole\\[1.75ex]\fi
  \@faculty
  \ifx\@unit\@emptymacro\else\\* \@unit\fi~\\[1.75ex]
  \@address\\[1.75ex]
  \ifx\@email\@emptymacro\else E \@emailuser\tat{}\@emaildomain\\\fi
  \ifx\@phone\@emptymacro\else T \@phone\\\fi
  \ifx\@fax\@emptymacro\else F \@fax\\\fi
  \ifx\@mobile\@emptymacro\else M \@mobile\\\fi
  ~\\[4ex]
  \begin{tikzpicture}[anchor=north west,align=left,outer sep=0,inner sep=0]
    \path
    (0,0) node { \scriptsize \strut
      \textcolor{gray}{\uppercase\expandafter{\datename} } }
    (4cm,0) node { \scriptsize \strut
      \textcolor{gray}{\uppercase\expandafter{\subjectname} } }
    (0,-2.5ex) node { \strut \@date }
    (4cm,-2.5ex) node[align=left] { \strut \@subject };
  \end{tikzpicture}~\\[4ex]
}
%</le>
%    \end{macrocode}
% \end{macro}
%
% The trailer of the page is automatically generated at the end of the document:
% \changes{v1.8}{2016/01/08}{Allowed for empty senderrole}
%    \begin{macrocode}
%<*le>
\AtEndDocument{
  \@closing\\*[3ex]\@signature~\\[3ex]
  \@sender
  \ifx\@senderrole\@emptymacro\\[8ex]\else\\*\@senderrole\\[8ex]\fi
  \setlength{\parskip}{0em}
  \ifx\@carboncopy\@emptymacro\else CC: \@carboncopy\\[4ex]\fi
  \ifx\@enclosed\@emptymacro\else ENCL: \@enclosed\fi
  \label{lastpage}
}
%</le> 
%    \end{macrocode}
%
% \subsection{References}
%    \begin{macrocode}
%<*ct|mt>
\IfFileExists{varioref.sty}{\RequirePackage{varioref}}{}
%</ct|mt>
%
%<*ct>
\IfFileExists{hyperref.sty}{
    \RequirePackage{hyperref}
    \hypersetup{
      backref=true,
      breaklinks=true,
      colorlinks=true,
      citecolor=black,
      filecolor=black,
      hyperindex=true,
      linkcolor=black,
      pageanchor=true, 
      pagebackref=true,
      pagecolor=black,
      pdfpagemode=UseOutlines,
      urlcolor=black
    }
    \AtBeginDocument{
      \hypersetup{
        pdftitle={\@title},
        pdfsubject={\@subtitle},
        pdfauthor={\@author}
      }
    }
}{}
%</ct>
%
%
%<*mt>
\IfFileExists{hyperref.sty}{
    \RequirePackage{hyperref}
    \hypersetup{
      backref=true,
      breaklinks=true,
      colorlinks=true,
      citecolor=black,
      filecolor=black,
      hyperindex=true,
      linkcolor=black,
      pageanchor=true, 
      pagebackref=true,
      pagecolor=black,
      pdfpagemode=UseOutlines,
      urlcolor=black
    }
    \AtBeginDocument{
      \hypersetup{
        pdftitle={\@title},
        pdfsubject={Master's Thesis},
        pdfauthor={\@author}
      }
    }
}{}
%</mt> 
%    \end{macrocode}
%
% \Finale
\endinput
