% \iffalse\begin{macrocode}
%<*driver>

\documentclass{ltxdoc}
\usepackage[utf8]{inputenc} % this file uses UTF-8
\usepackage[english]{babel}
\usepackage{tgpagella}
\usepackage{tabularx}
\usepackage{hologo}
\usepackage{booktabs}
\usepackage[scaled=0.86]{berasans}
\usepackage[scaled=1.03]{inconsolata}
\usepackage[resetfonts]{cmap}
\usepackage[T1]{fontenc} % use 8bit fonts
\emergencystretch 2dd
\usepackage{hypdoc}
\usepackage{microtype}
\usepackage{ragged2e}
\usepackage{paralist}
\usepackage{multicol}

% Making paragraphs numbered
\makeatletter
\renewcommand\paragraph{\@startsection{paragraph}{4}{\z@}%
            {-2.5ex\@plus -1ex \@minus -.25ex}%
            {1.25ex \@plus .25ex}%
            {\normalfont\normalsize\bfseries}}
\makeatother
\setcounter{secnumdepth}{4} % how many sectioning levels to assign
\setcounter{tocdepth}{4}    % how many sectioning levels to show

% ltxdoc class options
\CodelineIndex
\MakeShortVerb{|}
\EnableCrossrefs
\DoNotIndex{}
\makeatletter
  \c@IndexColumns=2
  \c@GlossaryColumns=2
\makeatother

\begin{document}
  \RecordChanges
  \DocInput{fithesis.dtx}
  \PrintIndex
  \RaggedRight
  \PrintChanges
\end{document}

%</driver>
%    \end{macrocode}
%<*class>
\NeedsTeXFormat{LaTeX2e}
% Define `\thesis@version` and store it in the `VERSION.tex` file \fi
{\def\thesis@versiondef#1#2{
  \gdef\thesis@version@number{#1}
  \gdef\thesis@version@date{#2}
  \gdef\thesis@version{#2 #1 fithesis3 MU thesis class}}
\thesis@versiondef{v0.3.46}{2017/06/02}}
% {\newwrite\f\openout\f=VERSION\write\f{\thesis@version}\closeout\f}
%
%%%%%%%%%%%%%%%%%%%%%%%%%%%%%%%%%%%%%%%%%%%%%%%%%%%%%%%%%%%%%%%%%%%%%%%%%%%%%%%
%
% ^^A Since 0.3.45, all changes are documented at the change sites.
% \changes{v0.3.44}  {2017/05/18}{Fixed the color in the logo of FI
%   MU. [VN]}
% \changes{v0.3.44}  {2017/05/18}{The captions in the examples for
%   MU are now above tables. [VN]}
% \changes{v0.3.44}  {2017/05/18}{The style file for FSpS, MU now
%   uses a 14 cm wide type area. [VN]}
% \changes{v0.3.44}  {2017/05/18}{The style file for FSpS, MU now
%   uses 1.5 spacing outside bibliography. [VN]}
% \changes{v0.3.44}  {2017/05/18}{The style file for FSpS, MU now
%   uses different title page layout. [VN]}
% \changes{v0.3.44}  {2017/05/18}{The style file for FSpS, MU now
%   uses different declaration text. [VN]}
% \changes{v0.3.44}  {2017/05/18}{Fixed wrong / missing
%   non-breaking spaces in Czech / Slovak localization. [VN]}
% \changes{v0.3.44}  {2017/05/18}{The style file for FSpS, MU now
%   includes place, date, and signature field next to the
%   declaration. [VN]}
% \changes{v0.3.44}  {2017/05/18}{Added an additional hyphenation
%   hint to the example document. [VN]}
% \changes{v0.3.44}  {2017/05/18}{The style file for FSpS, MU now
%   complies with updated requirements. [VN]}
% \changes{v0.3.43}  {2017/05/07}{Added a mention about the thesis
%   proposal thesis type to the user guide for the Faculty of
%   Informatics at the Masaryk University, Brno and fixed a bad
%   citation in the user guides for the Masaryk University, Brno.
%   Added an extra english option for babel in the examples for the
%   Masaryk University, Brno. \cs{thesis@blocks@thanks} no longer
%   gobbles leading spaces of \cs{thesis@thanks}. A fix in the
%   Slovak locale by \texttt{kiraacorsac} at GitHub. [VN]}
% \changes{v0.3.42}  {2017/01/28}{Documented that all color
%   settings are done in the \textsc{rgb} colorspace, which makes
%   the color option suitable mostly for the digital versions of
%   fithesis documents rather than for printing. Added the thesis
%   proposal thesis type. [VN]}
% \changes{v0.3.41}  {2016/08/17}{Removed the arbitrary limitation
%   of \cs{thesissetup}, which prevented it from accepting
%   multi-paragraph values. The \cs{thesislong} macro is therefore
%   no longer necessary, but kept around for backwards
%   compatibility. [VN]}
% \changes{v0.3.40}  {2016/06/06}{Fixed \cs{thesis@pages} not
%   working properly when the \texttt{autoLayout} key is set to
%   \texttt{false}. Added a section on the \textsf{markdown}
%   package to the example documents of the Masaryk University in
%   Brno. Added the \cs{ifthesis@blocks@assignment@hideIfDigital@}
%   conditional to the \texttt{style/mu/fithesis-base.sty} style
%   file. [VN]}
% \changes{v0.3.39:2}{2016/05/26}{The \texttt{assignment} key no
%   longer affects the page numbers and takes into account the
%   \texttt{digital} and \texttt{printed} class options in the
%   style files of the Masaryk University in Brno. [VN]}
% \changes{v0.3.39:1}{2016/05/26}{The
%   \cs{thesis@blocks@bibliography} now uses an emergency stretch
%   of 3\,em in the style files of the Masaryk University in Brno.
%   [VN]}
% \changes{v0.3.38:2}{2016/05/15}{Fixed \cs{thesis@pages} not
%   taking \cs{thesis@postamble} into account. Fixed the chapter
%   number being printed regardless of secnumdepth by the style
%   files of the Masaryk University in Brno. Fixed
%   \cs{thesis@blocks@assignment} being typeset in Czech only
%   by the style file of the Faculty of Science at the Masaryk
%   University in Brno. [VN]}
% \changes{v0.3.38:1}{2016/04/18}{The \texttt{bib} key was added
%   on the \textsf{fithesis3} class level and is supported by the
%   style files of the Masaryk University in Brno. [VN]}
% \changes{v0.3.37}  {2016/04/12}{The \cs{tableofcontents} in the
%   style files of the Masaryk University in Brno now correctly
%   handles all tocdepth values. The expansion of \cs{part} also no
%   longer results in a \textsf{hyperref}-related error in the
%   style files of the Masaryk University in Brno. [VN]}
% \changes{v0.3.36}  {2016/03/26}{The \cs{thesis@load} macro has
%   been lifted to the public API as \cs{thesisload}. [VN]}
% \changes{v0.3.35:3}{2016/03/23}{Fixed \cs{l@}\textit{locale}
%   being possibly undefined in a LuaLaTeX run. [VN]}
% \changes{v0.3.35:2}{2016/03/22}{Added support for seminar papers
%   on the \textsf{fithesis3} class level. The style files for the
%   faculties of the Masaryk University in Brno do not provide any
%   special handling of this thesis type yet. [VN]}
% \changes{v0.3.35:1}{2016/03/22}{Added support for the Division of
%   Information and Library Studies of the Faculty of Arts at the
%   Masaryk University in Brno. This support is enabled by
%   specifying \cs{thesis@department} to be \texttt{kisk}. [VN]}
% \changes{v0.3.34}  {2016/02/24}{Added visual tests of output
%   PDFs. The \texttt{test/} directory is now also uses the
%   \textit{university}\texttt{/}\textit{faculty} path scheme
%   employed by the rest of the package. [VN]}
% \changes{v0.3.33}  {2016/02/19}{Added \cs{thesis@patch}. [VN]}
% \changes{v0.3.32}  {2016/02/18}{Fixed a regression from 
%   v0.3.27:2 -- The margins in \cs{thesis@blocks@mainMatter} were
%   corrected in the style files of the Faculty of Economics and
%   Administration and the Faculty of Medicine at the Masaryk
%   University in Brno. [VN]}
% \changes{v0.3.31:3}{2016/01/13}{The
%   \texttt{style/mu/fithesis-sci.sty} style file redefines
%   \cs{thesis@blocks}\texttt{\discretionary{@}{@}{@}declaration}
%   to include a formatted date and an author's signature field.
%   The locale file \texttt{style/mu/sci/czech.def} contains a new
%   string \cs{thesis@czech\discretionary{@}{@}{@}formattedDate}.
%   [VN]}
% \changes{v0.3.31:2}{2016/01/07}{All trailing \texttt{\%}s were
%   removed. Some of the inherited \textsf{fithesis2} code in the
%   \texttt{style/mu/fithesis-1*.clo} and
%   \texttt{style/mu/fithesis-base.sty} files was refactored and
%   reformatted. Alternative templates for the Faculty of Science
%   at the Masaryk University in Brno are now mentioned in the
%   respective user guide. [VN]}
% \changes{v0.3.31:1}{2016/01/07}{The redefinitions of
%   \cs{appendix} from the
%   \texttt{style/mu\discretionary{/}{/}{/}fithesis-1*.clo} files,
%   which broke hyperref links to appendices, were removed. [VN]}
% \changes{v0.3.30}  {2016/01/05}{The \cs{thesis@seasonYear} macro
%   has been added, which, unlike \cs{thesis@year} takes into
%   account the fact that January and Fabruary of the year $n$
%   still belong to the fall semester of the year $n-1$. [VN]}
% \changes{v0.3.29:3}{2015/12/09}{The \texttt{table} class option
%   for the style files of the Masaryk University in Brno now loads
%   all the required packages and changes the table measurements
%   even when the \texttt{color} option is not specified. [VN]}
% \changes{v0.3.29:2}{2015/12/09}{The \texttt{printed} and
%   \texttt{digital} class options, which set all the options
%   appropriate for either the printed or the digital versions of a
%   document, are now available for the style files of the Masaryk
%   University in Brno. [VN]}
% \changes{v0.3.29:1}{2015/12/08}{The initial pages of the styles
%   of the Masaryk University in Brno are no longer page-numbered,
%   so that hyperref links work correctly. [VN]}
% \changes{v0.3.28:2}{2015/12/03}{If the thesis locale and the
%   document locale is the same (the default behaviour), the
%   \cs{thesis@selectLocale} macro is applied globally at the
%   beginning of the document. As a result, the \textsf{csquotes}
%   style is automatically set for the entire document. [VN]}
% \changes{v0.3.28:1}{2015/12/02}{The \texttt{draft} option has no
%   longer an effect on the \textsf{microtype} package. The
%   \cs{thesis@require} command now also takes an optional
%   argument and the \cs{thesis@require\-WithOption} command has
%   become \cs{thesis@require\-IfExists}. [VN]}
% \changes{v0.3.27:5}{2015/11/30}{The PDF bookmarks are no longer
%   garbled, when the \Hologo{LuaTeX} engine is used. [VN]}
% \changes{v0.3.27:4}{2015/11/30}{\cs{thesis@english@declaration}
%   now uses the correct idiom (by one's own $\to$ on
%   one's own). [VN]}
% \changes{v0.3.27:3}{2015/11/29}{As a preparation for the future
%   inclusion of bibliography support, the \textsf{csquotes}
%   package is loaded, \texttt{@csquotesStyle} is a new part of
%   the locale interface that sets the \textsf{csquotes} style
%   of a locale, and \cs{thesis@selectLocale} now switches the
%   \textsf{csquotes} style. [VN]}
% \changes{v0.3.27:2}{2015/11/29}{\cs{thesis@selectLocale} is no
%   longer performed globally for the entire document during
%   \cs{thesis@load} (effectively overriding the user's hyphenation
%   settings, if Babel's or Polyglossia's \cs{languagename}
%   differs from \cs{thesis@locale}). Instead, \cs{thesis@preamble}
%   and \texttt{@postamble} now locally switch the locale and
%   expand \cs{thesis@blocks@preamble} and \texttt{@postamble},
%   which are the new redefinables. Since this breaks the behaviour
%   of \cs{thesis@blocks@mainMatter}, whose effects would also be
%   local, \texttt{@mainMatter} is now executed directly by
%   \cs{thesis@preamble} after closing the group and becomes a new
%   part of the interface between the class and the style files.
%   [VN]}
% \changes{v0.3.27:1}{2015/11/29}{Fixed a typo in the guide.
%   Added a compatibility layer for \cs{title}, \cs{author}, and
%   \cs{maketitle}. Minor changes of the documentation. [VN]}
% \changes{v0.3.26}  {2015/11/21}{Updated the example documents and
%   the user guide. Fixed a typo in the description of
%   \cs{thesis@season}. [VN]}
% \changes{v0.3.25}  {2015/11/20}{The example documents from the
%   \texttt{example} directory are now a part of the CTAN archive.
%   Additional information were inserted into the guide and to the
%   example files. [VN]}
% \changes{v0.3.24}  {2015/11/17}{Added the \cs{thesis@backend}
%   tunable. The hyphenation pattern switching now uses
%   \textsf{polyglossia} instead of crude \cs{language} switching
%   wherenever possible. Added the opt-out \texttt{microtype}
%   class option, which loads the microtypographic extension. The
%   \cs{thesis@}\textit{locale}\texttt{@summer} and
%   \texttt{@winter} locale macros were renamed to \texttt{@spring}
%   and \texttt{@fall}. The \cs{thesis@parseDate} now uses more
%   realistic month ranges to set \cs{thesis@season} and
%   \cs{thesis@academicYear}. Removed the extraneous indent in the
%   \cs{thesis@blocks@declaration} macro definition within the
%   \texttt{style/mu/fithesis-fi.sty} style file. \cs{paragraph}s
%   are not included in the table of contents by default. The
%   \texttt{table} class option now supports the \texttt{tabu}
%   environment. The list of tables and the list of figures now
%   have an entry in the table of contents for the
%   \texttt{style/mu/fithesis-econ.sty} style file. [VN]}
% \changes{v0.3.23}  {2015/10/14}{Fixed a typo in the Slovak
%   locale. [VN]}
% \changes{v0.3.22}  {2015/10/09}{Updated the link colors in the
%   style of the Faculty of Economics and Administration at the
%   Masaryk University in Brno and fixed the title page leading
%   in the style of the Faculty of Science at the Masaryk
%   University in Brno. [VN]}
% \changes{v0.3.21}  {2015/08/26}{Fixed an invalid font name. [VN]}
% \changes{v0.3.20}  {2015/07/07}{Removed an extraneous
%   \cs{hypersetup} option to eliminate a warning. Performed
%   several minor Makefile updates. Updated the technical
%   documentation. [VN]}
% \changes{v0.3.19}  {2015/06/27}{Updated the license. Added the
%   \texttt{fithesis-} prefix to locale files. Proof-read and
%   updated the documentation. Encapsulated the
%   \texttt{localeInheritance} and \texttt{styleInheritance}
%   setters. Added the \cs{thesis@selectLocale}\texttt{\{...\}},
%   which acts as a replacement for
%   \cs{def}\cs{thesis@locale}\texttt{\{...\}}, which also switches
%   hyphenation patterns. The macro definitions inside locale files
%   are now global to account for the fact that it now makes sense
%   to include locale files on-site (and therefore possibly inside
%   a group) using the \cs{thesis@selectLocale}. The class files
%   are now generated using the XeTeX engine, which
%   preserves the characters outside ASCII. [VN]}
% \changes{v0.3.18}  {2015/06/24}{A bulk of changes required to submit
%   the document class to CTAN: Changed the structure of the output
%   \texttt{fithesis3.ctan.zip} archive. Updated the license
%   notice. Added a \texttt{README} file. Canonicalized a url
%   within the user guides. Renamed the root directory from
%   \texttt{fithesis3/} to \texttt{fithesis/}. Refactored the
%   makefiles. Added developer example files. Renamed
%   \texttt{docstrip.cfg} to \texttt{LICENSE.tex} to better
%   describe its role. The \texttt{fithesis.dtx} file now
%   generates a \texttt{VERSION.tex} file containing the version of
%   the package, when it's being typeset. Flattened the
%   \texttt{logo/} directory structure. [VN]}
% \changes{v0.3.17}  {2015/06/24}{Changed a forgotten
%   \cs{thesis@@lower}\texttt{\{...\}} invocation in the definition
%   of \cs{thesis@czech@declaration} for the Faculty of Arts into
%   \cs{thesis@@lower}\texttt{\{czech@...\}}, so that the macro
%   always expands to the correct output regardless of the current
%   locale. This is merely a matter of consistency, since the style
%   file of the Faculty of Arts only uses Czech strings within the
%   Czech locale. Removed an extraneous comment. Fixed a unit test.
%   Fixed a changelog entry. [VN]}
% \changes{v0.3.16}  {2015/06/21}{Clubs and widows are now set to
%   be infinitely bad. The \texttt{assignment} key has weaker, but
%   more robust semantics now. [VN]}
% \changes{v0.3.15}  {2015/06/14}{Renamed \cs{thesis@requireStyle}
%   to \cs{thesis@requireWithOptions} and moved the style loader
%   from the \cs{thesis@load} routine to a new
%   \cs{thesis@requireStyle} macro to make the semantics of
%   \cs{thesis@requireLocale} and \cs{thesis@requireStyle} more
%   similar. Changed the \texttt{basepath}, \texttt{logopath},
%   \texttt{localepath} and \texttt{stylepath} keys to match the
%   lower camelcasing of the rest of the keys. Added further
%   description regarding the use of the \texttt{assignment} key.
%   [VN]}
% \changes{v0.3.14}  {2015/06/07}{Updated the documentation. [VN]}
% \changes{v0.3.13}  {2015/05/30}{Fixed an inconsistency in the
%   example code. Removed an extraneous \cs{thesis@blocks@clear}
%   block withing the definition of \cs{thesis@blocks@frontMatter}
%   in the fss style file. Added comments, fixed clubs and widows
%   and removed text overflows within the user guides. Adjusted the
%   colors of various style files. Removed the trailing dot in the
%   bibliographic identification within the med and ped style
%   files. Fixed a typo within the technical documentation. Fixed
%   the twoside alignment of the \cs{thesis@blocks@bibEntry} and
%   the \cs{thesis@blocks@bibEntryEn} blocks within the sci style
%   file.  The \cs{thesis@blocks@assignment} block no longer clears
%   a page when nothing is inserted. It is also no longer
%   hard-coded to be hidden for rigorous theses. Instead, the
%   \cs{ifthesis@blocks@assignment} conditional can be set either
%   by the subsequently loaded style files or by the user. So far,
%   only the fi and sci style files set the conditional. [VN]}
% \changes{v0.3.12}  {2015/05/24}{The subsections and
%   subsubsections now use the correct \texttt{tocdepth}. [VN]}
% \changes{v0.3.11}  {2015/05/15}{Added hyphenation into the
%   technical documentation. Fixed an unterminated group. Polished
%   the text of the guide. Added the \texttt{palatino} and
%   \texttt{nopalatino} options. Stylistic changes to the text of
%   the technical documentation. \cs{thesis@subdir} is now robust
%   against relative paths. Accounted for French spacing in the
%   guide. Fixed the \texttt{thesis@english@facultyName} string.
%   Documentation refinements. [VN]}
% \changes{v0.3.10}  {2015/05/09}{Fixed a typo in the technical
%   documentation. Updated the \emph{Advanced usage} chapter of the
%   user guide. The required packaged listed in Section 2.2 of the
%   user guide are now always correct. Adjusted the footer spacing
%   in the styles of econ and fi. Added \emph{Advanced usage}
%   chapter to the user guide. Added the description of basic
%   options into the user guide. Added the \texttt{table} and
%   \texttt{oldtable} options. Added the \texttt{type} field to the
%   guide for completeness. [VN]}
% \changes{v0.3.09}  {2015/04/26}{A complete refactoring of the class. The class
%   was decomposed into a base class, locale files and style files. [VN]}
% \changes{v0.3.08}  {2015/03/04}{Fixed a non-terminated \cs{if} condition.
%   [VN] (backport of v0.2.18)\\Fixed mostly documentation errors reported
%   at the new fithesis discussion forum (-ti, eco$\to$econ, implicit
%   twocolumn, example extended (font setup), etc.). [PS] (backport of v0.2.17)}
% \changes{v0.3.07}  {2015/02/03}{Replaced the \cs{thesiswoman} command with
%   \cs{thesisgender}. [VN]}
% \changes{v0.3.06}  {2015/01/26}{Added the colorx package and the base colors
%   for each faculty. If the color option is specified, the tabular environment
%   gets redefined and uses the faculty colors to color alternating table rows
%   to improve readability. The hyperref links in the e-version are now likewise
%   colored according to the chosen faculty, in this case regardless of the
%   presence of the color option. Dropped the support for typesetting theses
%   outside MU. [VN]}
% \changes{v0.3.05}  {2015/01/21}{Added support for change typesetting.
%   Restructured the code to make it more amenable to literal programming.
%   Added support for \cs{CodelineIndex} typesetting. Added information about
%   the usage of \textsf{fithesis1} and \textsf{fithesis2} on the FI unix
%   machines. (backport of v0.2.16) [VN] Minor changes throughout the text,
%   added a link to the the fithesis forums [PS] (backport of v0.2.15@r14:15)}
% \changes{v0.3.04}  {2015/01/14}{Import the url package to allow for the use of
%   \cs{url} within the documentation. (backport of v0.2.15@r13) [VN]}
% \changes{v0.3.03}  {2015/01/14}{Small fixes (added \cs{relax} at
%   \cs{MainMatter}), generating both fithesis.cls (obsolete, loading
%   \texttt{fithesis2.cls}) and \texttt{fithesis2.cls}, minor doc edits,
%   version numbering of \texttt{.clo} fixed, switch to utf8 and ensuring that
%   \texttt{.dtx} compiles. Documentation adjusted to the status quo, added
%   link to discussion forum (backport of v0.2.14) [PS]}
% \changes{v0.3.02}  {2015/01/13}{PDF metadata stamping added for
%   \cs{thesistitle} and \cs{thesisstudent} [VN]}
% \changes{v0.3.01}  {2015/01/09}{documentation now uses babel and cmap
%   packages. the entire file was transcoded into utf8, \cs{thesiscolor} was
%   replaced by color class option, added PDF metadata stamping support [VN]}
% \changes{v0.3.00}  {2015/01/01}{fi logo is no longer special-cased (added eps
%   and PDF), \cs{thesislogopath} added to set the logo directory path,
%   \cs{thesiscolor} added to enable colorful typo elements [VN]}
% \changes{v0.2.12a}{2008--2011}{fork fithesis2 by Mr. Filipčík and Janoušek;
%   cf. \protect\url{http://github.com/liskin/fithesis}}
% \changes{v0.2.12} {2008/07/27}{Licence change to the LPPL [JP]}
% \changes{v0.2.11} {2008/01/07}{fix missing \texttt{fi-logo.mf} [JP,PS]}
% \changes{v0.2.10} {2006/05/12}{fix EN name of Acknowledgement [JP]}
% \changes{v0.2.09}  {2006/05/08}{add EN version of University name [JP]}
% \changes{v0.2.08}  {2006/01/20}{add change of University name [JP]}
% \changes{v0.2.07}  {2005/05/10}{escape all Czech letters [JP]
%   babel is used instead of stupid package czech [JP]
%   \cs{MainMatter} should be placed after \cs{tablesofcontents} [PS]}
% \changes{v0.2.06}  {2004/12/22}{fix : behind Advisor [JP]}
% \changes{v0.2.05}  {2004/05/13}{add English abstract [JP]}
% \changes{v0.2.04}  {2004/05/13}{fix SK declaration [Peter Cerensky, JP]}
% \changes{v0.2.03}  {2004/05/13}{fix title spacing [PS, JP]}
% \changes{v0.2.02}  {2004/05/12}{fix encoding bug [JP]}
% \changes{v0.2.01}  {2004/05/11}{add subsubsection to toc [JP]}
% \changes{v0.2.00}  {2004/05/03}{add sk lang [JP, Peter Cerensky]
%   set default cls class to \textsf{rapport3} [JP]}
% \changes{v0.1g}   {2004/04/01}{change of default size (12pt$\to$11pt) [JP]}
% \changes{v0.1f}   {2004/01/24}{add documentation for hyperref [JP]}
% \changes{v0.1e}   {2004/01/07}{add Brno to MU title [JP]}
% \changes{v0.1d}   {2003/03/24}{removed def schapter from fit1*.clo [JP]}
% \changes{v0.1c}   {2003/02/21}{default values of \cs{facultyname} and
%   \\\cs{@thesissubtitle} set for backward compatibility) [PS]}
% \changes{v0.1b}   {2003/02/14}{change of default size (11pt$\to$12pt) [JP]}
% \changes{v0.1a}   {2003/02/12}{minor documentation changes (CZ only,
%   sorry) [PS]}
% \changes{v0.1}    {2003/02/11}{new release, documentation editing (CZ only,
%   sorry) [PS]}
% \changes{v0.0a}   {2002}{changes by Jan Pavlovič to allow fithesis being
%   backend of docbook based system for thesis writing}
% \changes{v0.0}    {1998}{bachelor project of Daniel Marek under
%   supervision of Petr Sojka}
%
%%%%%%%%%%%%%%%%%%%%%%%%%%%%%%%%%%%%%%%%%%%%%%%%%%%%%%%%%%%%%%%%%%%%%%%%%%%%%%%
%
% \title{The \textsf{fithesis3} class for the typesetting of theses written
%   at the Masaryk University in Brno}
% \author{Daniel Marek, Jan Pavlovič, Vít Novotný, Petr Sojka}
% \date{\today}
% \maketitle
%
% \begin{abstract}
% \noindent This document details the design and the implementation
% of the \textsf{fithesis3} document class. It contains technical
% information for anyone who wishes to extend the class with their
% locale or style files. Users who only wish to use the class are
% advised to consult the guides distributed along with the class,
% which only document the parts of the public API relevant to the
% given style files.
% \end{abstract}
%
% \tableofcontents
%
% \section{Required classes and packages}
% \begin{macro}{\thesis@backend}
% The class requires the class specified in |\thesis@backend|,
% whose default value is |[a4paper]{rapport3}|. If a different
% base class is desired, it can be specified by redefining
% |\thesis@backend| prior to loading the \textsf{fithesis3} class.
%    \begin{macrocode}
\ProvidesClass{fithesis3}[\thesis@version]
\ifx\thesis@backend\undefined
  \def\thesis@backend{[a4paper]{rapport3}}
\fi\expandafter\LoadClass\thesis@backend
%    \end{macrocode}
% \end{macro}
% The class also requires the following packages:
% \begin{itemize}
%   \item\textsf{keyval} -- Adds support for parsing
%     comma-delimited lists of key-value pairs.
%   \item\textsf{etoolbox} -- Adds support for expanding
%     code after the preamble using the |\AtPreamble| hook.
%   \item\textsf{ltxcmds} -- Implements several commands from
%     the \LaTeX\ kernel. Used for the |\ltx@ifpackageloaded|
%     command, which -- unlike its |\@ifpackageloaded| counterpart
%     -- can be used outside the preamble.
%   \item\textsf{ifxetex} -- Used to detect the \Hologo{XeTeX}
%     engine.
%   \item\textsf{ifluatex} -- Used to detect the \Hologo{LuaTeX}
%     engine.
%   \item\textsf{inputenc} -- Used to enable the input UTF-8
%     encoding. This package does not get loaded under
%     the \Hologo{XeTeX} and \Hologo{LuaTeX} engines.
% \end{itemize}
% The \texttt{hyperref} package is also conditionally loaded during
% the expansion of the |\thesis@load| macro (see Section
% \ref{sec:thesisload}). Other packages may be required by the
% style files (see Section \ref{sec:style-files}) you are using.
%    \begin{macrocode}
\RequirePackage{keyval}
\RequirePackage{etoolbox}
\RequirePackage{ltxcmds}
\RequirePackage{ifxetex}
\RequirePackage{ifluatex}
\ifxetex\else\ifluatex\else
  \RequirePackage[utf8]{inputenc}
\fi\fi
%    \end{macrocode}
% \section{Public API}
% \label{sec:public-api}
% \subsection{Options}
% Any \oarg{options} passed to the class will be handed down to the
% loaded style files. The supported options are therefore documented
% in the subsections of Section \ref{sec:style-files} dedicated to
% the respective style files.
%
% The class options specify the \emph{form} of the document.
%
% \subsection{The \cs{thesissetup} macro}
% \begin{macro}{\thesissetup}
% The main public macro is the |\thesissetup|\marg{keyvals}
% command, where \textit{keyvals} is a comma-delimited list of
% \textit{key}=\textit{value} pairs as defined by the
% \textsf{keyval} package. This macro needs to be included prior to
% the beginning of a \LaTeX\ document. When the macro is expanded,
% the \textit{key}=\textit{value} pairs are processed.
%
% Contrary to the class options, the \textit{key}=\textit{value}
% pairs of the \cs{thesissetup} macro specify metainformation about
% the document.
%    \begin{macrocode}
\long\def\thesissetup#1{%
  \setkeys{thesis}{#1}}
%    \end{macrocode}
% \subsubsection{The \texttt{basePath} key}
% \begin{macro}{\thesis@basepath}
% The \marg{\texttt{basePath}=path} pair sets the \textit{path}
% containing the class files. The \textit{path} is prepended to
% every other path (|\thesis@logopath|, |\thesis@stylepath| and
% |\thesis@localepath|) used by the class. If non-empty, the
% \textit{path} gets normalized to \textit{path/}. The normalized
% \textit{path} is stored within the |\thesis@basepath| macro,
% whose implicit value is |fithesis/|.
%    \begin{macrocode}
\def\thesis@basepath{fithesis/}
\define@key{thesis}{basePath}{%
  \ifx\thesis@empty#1\thesis@empty
    \def\thesis@basepath{}%
  \else
    \def\thesis@basepath{#1/}%
  \fi}
%    \end{macrocode}
% \end{macro}
% \begin{macro}{\thesis@logopath}
% \subsubsection{The \texttt{logoPath} key}
% The \marg{\texttt{logoPath}=path} pair sets the \textit{path}
% containing the logo files, which is used by the style files to
% load the university and faculty logos. The \textit{path} is
% normalized using the |\thesis@subdir| macro and stored
% within the |\thesis@logopath| macro, whose implicit value
% is |\thesis@basepath| followed by |logo/\thesis@university/|. By
% default, this expands to \texttt{fithesis/logo/mu/}.
%    \begin{macrocode}
\def\thesis@logopath{\thesis@basepath logo/\thesis@university/}
\define@key{thesis}{logoPath}{%
  \def\thesis@logopath{\thesis@subdir#1%
    \empty\empty\empty\empty}}
%    \end{macrocode}
% \end{macro}
% \begin{macro}{\thesis@stylepath}
% \subsubsection{The \texttt{stylePath} key}
% The \marg{\texttt{stylePath}=path} pair sets the \textit{path}
% containing the style files. The \textit{path} is normalized using
% the |\thesis@subdir| macro and stored within the
% |\thesis@stylepath| macro, whose implicit value is
% |\thesis@basepath style/|. By default, this expands to
% \texttt{fithesis/style/}.
%    \begin{macrocode}
\def\thesis@stylepath{\thesis@basepath style/}
\define@key{thesis}{stylePath}{%
  \def\thesis@stylepath{\thesis@subdir#1%
    \empty\empty\empty\empty}}
%    \end{macrocode}
% \end{macro}
% \begin{macro}{\thesis@localepath}
% \subsubsection{The \texttt{localePath} key}
% The \marg{\texttt{localePath}=path} pair sets the \textit{path}
% containing the locale files. The \textit{path} is normalized
% using the |\thesis@subdir| macro and stored within the
% |\thesis@localepath| macro, whose implicit value is
% |\thesis@basepath| followed by |locale/|.  By default, this
% expands to \texttt{fithesis/locale/}.
%    \begin{macrocode}
\def\thesis@localepath{\thesis@basepath locale/}
\define@key{thesis}{localePath}{%
  \def\thesis@localepath{\thesis@subdir#1%
    \empty\empty\empty\empty}}
%    \end{macrocode}
% \end{macro}
% \begin{macro}{\thesis@subdir}
% The |\thesis@subdir| macro returns |/| unchanged, coerces
% |.|, |..|, |/|\textit{path}, |./|\textit{path} and
% |../|\textit{path} to |./|, |../|, |/|\textit{path}|/|,
% |./|\textit{path}|/| and |../|\textit{path}|/|, respectively, and
% prefixes any other \textit{path} with |\thesis@basepath|.
%    \begin{macrocode}
\def\thesis@subdir#1#2#3#4\empty{%
  \ifx#1\empty%           <empty> -> <basepath>
    \thesis@basepath
  \else
    \if#1/%
      \ifx#2\empty%             / -> /
        /%
      \else%              /<path> -> /<path>/
        #1#2#3#4/%
      \fi
    \else
      \if#1.%
        \ifx#2\empty%           . -> ./
          ./%
        \else
          \if#2.%
            \ifx#3\empty%      .. -> ../
              ../%
            \else
              \if#3/%   ../<path> -> ../<path>/
                ../#4/%
              \else
                \thesis@basepath#1#2#3#4/%
              \fi
            \fi
          \else
            \if#2/%      ./<path> -> ./<path>/
              ./#3#4/%
            \else
              \thesis@basepath#1#2#3#4/%
            \fi
          \fi
        \fi
      \else
        \thesis@basepath#1#2#3#4/%
      \fi
    \fi
  \fi}
%    \end{macrocode}
% \end{macro}
% \begin{macro}{\thesis@def}
% The |\thesis@def|\oarg{key}\marg{name} macro defines
% the |\thesis@|\textit{name} macro to expand
% to either <<\textit{key}>>, if specified, or to
% <<\textit{name}>>. The macro serves to provide placeholder
% strings for macros with no default value.
%    \begin{macrocode}
\newcommand{\thesis@def}[2][]{%
  \expandafter\def\csname thesis@#2\endcsname{%
    <<\ifx\thesis@empty#1\thesis@empty#2\else#1\fi>>}}
%    \end{macrocode}
% \end{macro}
% \begin{macro}{\thesis@declaration}
% \subsubsection{The \texttt{declaration} key}
% The \marg{\texttt{declaration}=text} pair sets the
% declaration \textit{text} to be included into the document.
% The \textit{text} is stored within the |\thesis@declaration|
% macro, whose implicit value is
% |\thesis@@{declaration}|.
%    \begin{macrocode}
\def\thesis@declaration{\thesis@@{declaration}}
\long\def\KV@thesis@declaration#1{%
  \long\def\thesis@declaration{#1}}
%    \end{macrocode}
% \end{macro}
% \begin{macro}{\ifthesis@woman}
% \subsubsection{The \texttt{gender} key}
% The \marg{\texttt{gender}=char} pair sets the author's gender to
% either a male, if \textit{char} is the character \texttt{m}, or
% to a female. The gender can be tested using the
% |\ifthesis@woman| \ldots |\else| \ldots |\fi| conditional. The
% implicit gender is male.
%    \begin{macrocode}
\newif\ifthesis@woman\thesis@womanfalse
\define@key{thesis}{gender}{%
  \def\thesis@male{m}%
  \def\thesis@arg{#1}%
  \ifx\thesis@male\thesis@arg
    \thesis@womanfalse
  \else
    \thesis@womantrue
  \fi}
%    \end{macrocode}
% \end{macro}
% \begin{macro}{\thesis@author}
% \subsubsection{The \texttt{author} key}
% The \marg{\texttt{author}=name} pair sets the author's full name
% to \textit{name}. The \textit{name} is parsed using the
% \DescribeMacro{\thesis@parseAuthor}|\thesis@parseAuthor| macro
% and stored within the following macros:
% \begin{itemize}
%   \item\DescribeMacro{\thesis@author}|\thesis@author|
%     -- The full name of the author.
%   \item\DescribeMacro{\thesis@author@head}|\thesis@author@head|
%     -- The first space-delimited part of the name. This
%     corresponds to the author's first name.
%   \item\DescribeMacro{\thesis@author@tail}|\thesis@author@tail|
%     -- The full name without the first space-delimited part of
%     the name. This corresponds to the author's surname.
% \end{itemize}
% The standard \LaTeX\ \DescribeMacro{\author}|\author| macro also
% sets this key.
%    \begin{macrocode}
\def\thesis@parseAuthor#1{%
  \def\thesis@author{#1}%
  \def\thesis@author@head{\expandafter\expandafter\expandafter
    \@gobble\thesis@head#1 \relax}%
  \def\thesis@author@tail{\thesis@tail#1 \relax}}
\thesis@def{author}%
\thesis@def[author]{author@head}%
\thesis@def[author]{author@tail}%
\define@key{thesis}{author}{%
  \thesis@parseAuthor{#1}}
\let\author\thesis@parseAuthor
%    \end{macrocode}
% \end{macro}
% \begin{macro}{\thesis@id}
% \subsubsection{The \texttt{id} key}
% The \marg{\texttt{id}=identifier} pair sets the identifier
% of the thesis author to \textit{identifier}. This usually
% corresponds to the unique identifier of the author within the
% information system of the given university.
%    \begin{macrocode}
\thesis@def{id}
\define@key{thesis}{id}{%
  \def\thesis@id{#1}}
%    \end{macrocode}
% \end{macro}
% \begin{macro}{\thesis@type}
% \subsubsection{The \texttt{type} key}
% The \marg{\texttt{type}=type} pair sets the type of the thesis
% to \textit{type}. The following types of theses are recognized:
% \begin{center}\begin{tabular}{lc}\toprule
%   The thesis type   & The value of \textit{type} \\\midrule
%   Seminar paper     & \texttt{sem} \\
%   Bachelor's thesis & \texttt{bc} \\
%   Master's thesis   & \texttt{mgr} \\
%   Thesis proposal   & \texttt{prop} \\
%   Doctoral thesis   & \texttt{d} \\
%   Rigorous thesis   & \texttt{r} \\\bottomrule
% \end{tabular}\end{center}
% The \textit{type} is stored within the |\thesis@type| macro,
% whose implicit value is |bc|. For the ease of testing of the
% thesis type via |\ifx| conditions within style and locale files,
% the \DescribeMacro{\thesis@sempaper}|\thesis@sempaper|,
% \DescribeMacro{\thesis@bachelors}|\thesis@bachelors|,
% \DescribeMacro{\thesis@masters}|\thesis@masters|,
% \DescribeMacro{\thesis@proposal}|\thesis@proposal|,
% \DescribeMacro{\thesis@doctoral}|\thesis@doctoral| and
% \DescribeMacro{\thesis@rigorous}|\thesis@rigorous| macros
% containing the corresponding \textit{type} values are available
% as a part of the private API.
%    \begin{macrocode}
\def\thesis@sempaper{sem}
\def\thesis@bachelors{bc}
\def\thesis@masters{mgr}
\def\thesis@proposal{prop}
\def\thesis@doctoral{d}
\def\thesis@rigorous{r}
\let\thesis@type\thesis@bachelors
\define@key{thesis}{type}{%
  \def\thesis@type{#1}}
%    \end{macrocode}
% \end{macro}
% \begin{macro}{\thesis@university}
% \subsubsection{The \texttt{university} key}
% The \marg{\texttt{university}=identifier} pair sets the
% identifier of the university, at which the thesis is being
% written, to \textit{identifier}. The \textit{identifier} is
% stored within the |\thesis@university| macro, whose
% implicit value is \texttt{mu}. This value corresponds to the
% Masaryk University in Brno.
%    \begin{macrocode}
\def\thesis@university{mu}
\define@key{thesis}{university}{%
  \def\thesis@university{#1}}
%    \end{macrocode}
% \end{macro}
% \begin{macro}{\thesis@faculty}
% \subsubsection{The \texttt{faculty} key}
% The \marg{\texttt{faculty}=identifier} pair sets the faculty, at
% which the thesis is being written, to \textit{domain}. The
% following faculty \textit{identifier}s are recognized at the
% Masaryk University in Brno:
% \begin{center}\begin{tabularx}{\textwidth}{Xc}\toprule
%   The faculty & The \textit{domain} name \\\midrule
%   The Faculty of Informatics & \texttt{fi} \\
%   The Faculty of Science & \texttt{sci} \\
%   The Faculty of Law & \texttt{law} \\
%   The Faculty of Economics and Administration & \texttt{econ} \\
%   The Faculty of Social Studies & \texttt{fss} \\
%   The Faculty of Medicine & \texttt{med} \\
%   The Faculty of Education & \texttt{ped} \\
%   The Faculty of Arts & \texttt{phil} \\
%   The Faculty of Sports Studies & \texttt{fsps} \\\bottomrule
% \end{tabularx}\end{center}
% The \textit{identifier} is stored within the |\thesis@faculty|
% macro, whose implicit value is \texttt{fi}.
%    \begin{macrocode}
\def\thesis@faculty{fi}
\define@key{thesis}{faculty}{%
  \def\thesis@faculty{#1}}
%    \end{macrocode}
% \end{macro}
% \begin{macro}{\thesis@department}
% \subsubsection{The \texttt{department} key}
% The \marg{\texttt{department}=name} pair sets the name of the
% department, at which the thesis is being written, to
% \textit{name}. Unlike the university and faculty identifiers,
% \textsf{fithesis3} does not prescribe the format of the
% \textit{name}; the style files may internally parse it, or
% typeset it as-is. The \textit{name} is stored within the
% |\thesis@department| macro.
%    \begin{macrocode}
\thesis@def{department}
\define@key{thesis}{department}{%
  \def\thesis@department{#1}}
%    \end{macrocode}
% \end{macro}
% \begin{macro}{\thesis@departmentEn}
% \subsubsection{The \texttt{departmentEn} key}
% The \marg{\texttt{departmentEn}=name} pair sets the English
% name of the department, at which the thesis is being written, to
% \textit{name}. The \textit{name} is stored within the
% |\thesis@departmentEn| macro.
%    \begin{macrocode}
\thesis@def{departmentEn}
\define@key{thesis}{departmentEn}{%
  \def\thesis@departmentEn{#1}}
%    \end{macrocode}
% \end{macro}
% \begin{macro}{\thesis@programme}
% \subsubsection{The \texttt{programme} key}
% The \marg{\texttt{programme}=name} pair sets the name of the
% author's study programme to \textit{name}. Unlike the university
% and faculty identifiers, the programme \textit{name} is only used
% for typesetting and it should therefore be specified in plain
% language with optional \TeX\ macros.  The \textit{name} is stored
% within the |\thesis@programme| macro.
%    \begin{macrocode}
\thesis@def{programme}
\define@key{thesis}{programme}{%
  \def\thesis@programme{#1}}
%    \end{macrocode}
% \end{macro}
% \begin{macro}{\thesis@programmeEn}
% \subsubsection{The \texttt{programmeEn} key}
% The \marg{\texttt{programmeEn}=name} pair sets the English name
% of the author's study programme to \textit{name}. The
% \textit{name} is stored within the |\thesis@programmeEn| macro.
%    \begin{macrocode}
\thesis@def{programmeEn}
\define@key{thesis}{programmeEn}{%
  \def\thesis@programmeEn{#1}}
%    \end{macrocode}
% \end{macro}
% \begin{macro}{\thesis@field}
% \subsubsection{The \texttt{field} key}
% The \marg{\texttt{field}=name} pair sets the name of the author's
% field of study to \textit{name}. Unlike the university and
% faculty identifiers, the \textit{name} of the field of study is
% only used for typesetting and it should therefore be specified in
% plain language with optional \TeX\ macros. The \textit{name} is
% stored within the |\thesis@field| macro.
%    \begin{macrocode}
\thesis@def{field}
\define@key{thesis}{field}{%
  \def\thesis@field{#1}}
%    \end{macrocode}
% \end{macro}
% \begin{macro}{\thesis@fieldEn}
% \subsubsection{The \texttt{fieldEn} key}
% The \marg{\texttt{fieldEn}=name} pair sets the English name of
% the author's field of stufy to \textit{name}. The \textit{name}
% is stored within the |\thesis@fieldEn| macro.
%    \begin{macrocode}
\thesis@def{fieldEn}
\define@key{thesis}{fieldEn}{%
  \def\thesis@fieldEn{#1}}
%    \end{macrocode}
% \end{macro}
% \begin{macro}{\thesis@universityLogo}
% \subsubsection{The \texttt{universityLogo} key}
% The \marg{\texttt{universityLogo}=filename} pair sets the
% filename of the logo file to be used as the university logo to
% \textit{filename}. The \textit{filename} is stored within the
% |\thesis@universityLogo| macro, whose implicit value is
% \texttt{fithesis-base}. The \texttt{fithesis-} prefix serves to
% prevent package clashes with other similarly named files within
% the \TeX\ directory structure. The logo file is loaded from the
% |\thesis@logopath|\discretionary{}{}{}|\thesis@universityLogo|
% path.
%    \begin{macrocode}
\def\thesis@universityLogo{fithesis-base}
\define@key{thesis}{universityLogo}{%
  \def\thesis@universityLogo{#1}}
%    \end{macrocode}
% \end{macro}
% \begin{macro}{\thesis@facultyLogo}
% \subsubsection{The \texttt{facultyLogo} key}
% The \marg{\texttt{facultyLogo}=filename} pair sets the filename
% of the logo file to be used as the faculty logo to
% \textit{filename}. The \textit{filename} is stored within the
% |\thesis@|\discretionary{}{}{}|facultyLogo| macro, whose
% implicit value is |fithesis-\thesis@faculty|. The
% \texttt{fithesis-} prefix serves to prevent package clashes with
% other similarly named files within the \TeX\ directory structure.
% The logo file is loaded from the
% |\thesis@logopath\thesis@facultyLogo| path.
%    \begin{macrocode}
\def\thesis@facultyLogo{fithesis-\thesis@faculty}
\define@key{thesis}{facultyLogo}{%
  \def\thesis@facultyLogo{#1}}
%    \end{macrocode}
% \end{macro}
% \begin{macro}{\thesis@style}
% \subsubsection{The \texttt{style} key}
% The \marg{\texttt{style}=filename} pair sets the filename of the
% style file to be used to \textit{filename}. The \textit{filename}
% is stored within the |\thesis@style| macro, whose
% implicit value is |\thesis@university/fithesis-\thesis@faculty|.
% When the \textit{filename} is an empty token string, no style
% files will be loaded during the main routine (see Section
% \ref{sec:thesisload}).
%    \begin{macrocode}
\def\thesis@style{\thesis@university/fithesis-\thesis@faculty}
\define@key{thesis}{style}{%
  \def\thesis@style{#1}}
%    \end{macrocode}
% \end{macro}
% \begin{macro}{\thesis@style@inheritance}
% \subsubsection{The \texttt{styleInheritance} key}
% The \marg{\texttt{styleInheritance}=bool} pair either enables,
% if \textit{bool} is \texttt{true} or unspecified, or disables the
% inheritance for style files. The setting affects the function of
% the |\thesis@requireStyle| macro (see Section
% \ref{sec:reflection}) and can be tested using the
% |\ifthesis@style@inheritance| \ldots |\else| \ldots |\fi|
% conditional. Inheritance is enabled for style files by default.
%    \begin{macrocode}
\newif\ifthesis@style@inheritance\thesis@style@inheritancetrue
\define@key{thesis}{styleInheritance}[true]{%
  \begingroup
  \def\@true{true}%
  \def\@arg{#1}%
  \ifx\@true\@arg
    \endgroup\thesis@style@inheritancetrue
  \else
    \endgroup\thesis@style@inheritancefalse
  \fi}
%    \end{macrocode}
% \end{macro}
% \begin{macro}{\thesis@locale}
% \subsubsection{The \texttt{locale} key}
% The \marg{\texttt{locale}=name} pair sets the name of the main
% locale to \textit{name}. The \textit{name} is stored within the
% |\thesis@locale| macro, whose implicit value is the main
% language of either the \textsf{babel} or the \textsf{polyglossia}
% package, or \texttt{english}, when undefined. When the
% \textit{name} is an empty token string, no locale files will be
% loaded during the main routine (see Section
% \ref{sec:thesisload}).
%    \begin{macrocode}
\def\thesis@locale{%
  % Babel / polyglossia detection
  \ifx\languagename\undefined
  english\else\languagename\fi}
\define@key{thesis}{locale}{%
  \def\thesis@locale{#1}}
%    \end{macrocode}
% \end{macro}
% \begin{macro}{\ifthesis@english}
% The English locale is special. Several parts of the document will
% typically be typeset in both the current locale and English.
% However, if the current locale is English, this would result in
% duplicity. To avoid this, the |\ifthesis@english| \ldots |\else|
% \ldots |\fi| conditional is made available for testing, whether
% or not the current locale is English.
%    \begin{macrocode}
\def\ifthesis@english{
  \expandafter\def\expandafter\@english\expandafter{\string
  \english}%
  \expandafter\expandafter\expandafter\def\expandafter
  \expandafter\expandafter\@locale\expandafter\expandafter
  \expandafter{\expandafter\string\csname\thesis@locale\endcsname}%
  \expandafter\csname\expandafter i\expandafter f\ifx\@locale
  \@english
    true%
  \else
    false%
  \fi\endcsname}
%    \end{macrocode}
% \end{macro}
% \begin{macro}{\thesis@locale@inheritance}
% \subsubsection{The \texttt{localeInheritance} key}
% The \marg{\texttt{localeInheritance}=bool} pair either enables,
% if \textit{bool} is \texttt{true} or unspecified, or disables the
% inheritance. The setting affects the function of
% the |\thesis@requireLocale| macro (see Section
% \ref{sec:reflection}) and can be tested using the
% |\ifthesis@locale@inheritance| \ldots |\else| \ldots |\fi|
% conditional. Inheritance is enabled for locale files by default.
%    \begin{macrocode}
\newif\ifthesis@locale@inheritance\thesis@locale@inheritancetrue
\define@key{thesis}{localeInheritance}[true]{%
  \begingroup
  \def\@true{true}%
  \def\@arg{#1}%
  \ifx\@true\@arg
    \endgroup\thesis@locale@inheritancetrue
  \else
    \endgroup\thesis@locale@inheritancefalse
  \fi}
%    \end{macrocode}
% \end{macro}
% \subsubsection{The \texttt{date} key}
% The \marg{\texttt{date}=date} pair sets the date of the thesis
% submission to \textit{date}, where \textit{date} is a string
% in the \texttt{YYYY/MM/DD} format, where \texttt{YYYY} stands
% for full year, \texttt{MM} stands for month and \texttt{DD}
% stands for day. The \textit{date} is parsed and stored using
% the \DescribeMacro{\thesis@parseDate}|\thesis@parseDate|
% macro within the following macros:
% \DescribeMacro{\thesis@date}
% \DescribeMacro{\thesis@year}
% \DescribeMacro{\thesis@month}
% \DescribeMacro{\thesis@day}
% \begin{multicols}{2}
% \begin{itemize}
%   \item|\thesis@date| -- The entire \textit{date}
%   \item|\thesis@year| -- The \texttt{YYYY} of \textit{date}
%   \item|\thesis@month| -- The \texttt{MM} of \textit{date}
%   \item|\thesis@day| -- The \texttt{DD} of \textit{date}
% \end{itemize}
% \end{multicols}
% \begin{itemize}
%   \item\DescribeMacro{\thesis@season}|\thesis@season| -- Expands
%     to either:
%     \begin{itemize}
%       \item\texttt{spring} if $2<{}$\texttt{MM}${}<9$,
%       \item\texttt{fall} if \texttt{MM}${}\leq2$ or \texttt{MM}${}\geq9$.
%     \end{itemize}
%   \item\DescribeMacro{\thesis@seasonYear}|\thesis@seasonYear|
%     -- The year of the given semester:
%     \begin{itemize}
%       \item\texttt{YYYY}${}-1$ if \texttt{MM}${}\leq2$.
%       \item\texttt{YYYY} if \texttt{MM}${}>2$
%     \end{itemize}
%   \item\DescribeMacro{\thesis@academicYear}|\thesis@academicYear|
%     -- The academic year of the given semester:
%     \begin{itemize}
%       \item\texttt{YYYY}${}-1$\texttt{/YYYY} if \texttt{MM}${}<9$.
%       \item\texttt{YYYY/YYYY}${}+1$ if \texttt{MM}${}\geq9$
%     \end{itemize}
% \end{itemize}
% To set up the default values, the |\thesis@parseDate| macro is
% called with the fully expanded |\the\year/\the\month/\the\day|
% string, which equals the current date.
%    \begin{macrocode}
\def\thesis@parseDate#1/#2/#3|{{
  % Set the basic macros
  \gdef\thesis@date{#1/#2/#3}%
  \gdef\thesis@year{#1}%
  \gdef\thesis@month{#2}%
  \gdef\thesis@day{#3}%
  
  % Set the season
  \newcount\@month\expandafter\@month\thesis@month\relax
       \ifnum\@month>8\gdef\thesis@season{fall}
  \else\ifnum\@month<3\gdef\thesis@season{fall}
  \else               \gdef\thesis@season{spring}
  \fi\fi

  % Set the academic year
  \newcount\@year\expandafter\@year\thesis@year\relax
  \ifnum\@month>8%
                    \edef\@yearA{\the\@year}%
    \advance\@year 1\edef\@yearB{\the\@year}%
    \advance\@year-1
  \else
    \advance\@year-1\edef\@yearA{\the\@year}%
    \advance\@year 1\edef\@yearB{\the\@year}%
  \fi
  \global\edef\thesis@academicYear{\@yearA/\@yearB}
  
  % Set the season year
  \ifnum\@month>2\else
    \advance\@year-1
  \fi
  \global\edef\thesis@seasonYear{\the\@year}}}

\edef\thesis@date{\the\year/\the\month/\the\day}%
\expandafter\thesis@parseDate\thesis@date|%

\define@key{thesis}{date}{{%
  \edef\@date{#1}%
  \expandafter\thesis@parseDate\@date|}}
%    \end{macrocode}
% \begin{macro}{\thesis@place}
% \subsubsection{The \texttt{place} key}
% The \marg{\texttt{place}=place} pair sets the location of the
% faculty, at which the thesis is being prepared, to \textit{place}.
% The \textit{place} is stored within the |\thesis@place|
% macro, whose implicit value is \texttt{Brno}.
%    \begin{macrocode}
\def\thesis@place{Brno}
\define@key{thesis}{place}{%
  \def\thesis@place{#1}}
%    \end{macrocode}
% \end{macro}
% \begin{macro}{\thesis@title}
% \subsubsection{The \texttt{title} key}
% The \marg{\texttt{title}=title} pair sets the title of the
% thesis to \textit{title}. The \textit{title} is stored within the
% |\thesis@title| macro. The standard \LaTeX\
% \DescribeMacro{\title}|\title| macro also sets this key.
%    \begin{macrocode}
\thesis@def{title}
\define@key{thesis}{title}{%
  \def\thesis@title{#1}}
\def\title#1{\def\thesis@title{#1}}
%    \end{macrocode}
% \end{macro}
% \begin{macro}{\maketitle}
% The standard \LaTeX\ |\maketitle| macro is defined, but disabled.
%    \begin{macrocode}
\let\maketitle\relax
%    \end{macrocode}
% \end{macro}
% \begin{macro}{\thesis@TeXtitle}
% \subsubsection{The \texttt{TeXtitle} key}
% The \marg{\texttt{TeXtitle}=title} pair sets the \TeX\ title of
% the thesis to \textit{title}. The \textit{title} is used, when
% typesetting the title, whereas |\thesis@title| is a plain text,
% which gets included in the PDF header of the resulting document.
% The \textit{title} is stored within the |\thesis@TeXtitle| macro,
% whose implicit value is |\thesis@title|.
%    \begin{macrocode}
\def\thesis@TeXtitle{\thesis@title}
\define@key{thesis}{TeXtitle}{%
  \def\thesis@TeXtitle{#1}}
%    \end{macrocode}
% \end{macro}
% \begin{macro}{\thesis@titleEn}
% \subsubsection{The \texttt{titleEn} key}
% The \marg{\texttt{titleEn}=title} pair sets the English title of
% the thesis to \textit{title}. The \textit{title} is stored within
% the |\thesis@titleEn| macro.
%    \begin{macrocode}
\thesis@def{titleEn}
\define@key{thesis}{titleEn}{%
  \def\thesis@titleEn{#1}}
%    \end{macrocode}
% \end{macro}
% \begin{macro}{\thesis@TeXtitleEn}
% \subsubsection{The \texttt{TeXtitleEn} key}
% The \marg{\texttt{TeXtitleEn}=title} pair sets the English \TeX\
% title of the thesis to \textit{title}. The \textit{title} is
% used, when typesetting the title, whereas |\thesis@titleEn| is a
% plain text, which gets included in the PDF header of the
% resulting document.  The \textit{title} is stored within the
% |\thesis@TeXtitleEn| macro, whose implicit value is
% |\thesis@titleEn|.
%    \begin{macrocode}
\def\thesis@TeXtitleEn{\thesis@titleEn}
\define@key{thesis}{TeXtitleEn}{%
  \def\thesis@TeXtitleEn{#1}}
%    \end{macrocode}
% \end{macro}
% \begin{macro}{\thesis@keywords}
% \subsubsection{The \texttt{keywords} key}
% The \marg{\texttt{keywords}=list} pair sets the keywords of the
% thesis to the comma-delimited \textit{list}. The \textit{list}
% is stored within the |\thesis@keywords| macro.
%    \begin{macrocode}
\thesis@def{keywords}
\define@key{thesis}{keywords}{%
  \def\thesis@keywords{#1}}
%    \end{macrocode}
% \end{macro}
% \begin{macro}{\thesis@TeXkeywords}
% \subsubsection{The \texttt{TeXkeywords} key}
% The \marg{\texttt{TeXkeywords}=list} pair sets the \TeX\ keywords
% of the thesis to the comma-delimited \textit{list}. The
% \textit{list} is used, when typesetting the keywords, whereas
% |\thesis@|\discretionary{}{}{}|keywords| is a plain text, which
% gets included in the PDF header of the resulting document. The
% \textit{list} is stored within the |\thesis@TeXkeywords| macro.
%    \begin{macrocode}
\def\thesis@TeXkeywords{\thesis@keywords}
\define@key{thesis}{TeXkeywords}{%
  \def\thesis@TeXkeywords{#1}}
%    \end{macrocode}
% \end{macro}
% \begin{macro}{\thesis@keywordsEn}
% \subsubsection{The \texttt{keywordsEn} key}
% The \marg{\texttt{keywordsEn}=list} pair sets the English
% keywords of the thesis to the comma-delimited \textit{list}. The
% \textit{list} is stored within the |\thesis@keywordsEn| macro.
%    \begin{macrocode}
\thesis@def{keywordsEn}
\define@key{thesis}{keywordsEn}{%
  \def\thesis@keywordsEn{#1}}
%    \end{macrocode}
% \end{macro}
% \begin{macro}{\thesis@TeXkeywordsEn}
% \subsubsection{The \texttt{TeXkeywordsEn} key}
% The \marg{\texttt{TeXkeywordsEn}=list} pair sets the English
% \TeX\ keywords of the thesis to the comma-delimited
% \textit{list}.  The \textit{list} is used, when typesetting the
% keywords, whereas |\thesis@keywordsEn| is a plain text, which
% gets included in the PDF header of the resulting document. The
% \textit{list} is stored within the |\thesis@TeXkeywordsEn| macro.
%    \begin{macrocode}
\def\thesis@TeXkeywordsEn{\thesis@keywordsEn}
\define@key{thesis}{TeXkeywordsEn}{%
  \def\thesis@TeXkeywordsEn{#1}}
%    \end{macrocode}
% \end{macro}
% \begin{macro}{\thesis@abstract}
% \subsubsection{The \texttt{abstract} key}
% The \marg{\texttt{abstract}=text} pair sets the abstract of the
% thesis to \textit{text}. The \textit{text} is stored within the
% |\thesis@abstract| macro.
%    \begin{macrocode}
\thesis@def{abstract}
\long\def\KV@thesis@abstract#1{%
  \long\def\thesis@abstract{#1}}
%    \end{macrocode}
% \end{macro}
% \begin{macro}{\thesis@abstractEn}
% \subsubsection{The \texttt{abstractEn} key}
% The \marg{\texttt{abstractEn}=text} pair sets the English
% abstract of the thesis to \textit{text}. The \textit{text}
% is stored within the |\thesis@abstractEn| macro.
%    \begin{macrocode}
\thesis@def{abstractEn}
\long\def\KV@thesis@abstractEn#1{%
  \long\def\thesis@abstractEn{#1}}
%    \end{macrocode}
% \end{macro}
% \begin{macro}{\thesis@advisor}
% \subsubsection{The \texttt{advisor} key}
% The \marg{\texttt{advisor}=name} pair sets the thesis advisor's
% full name to \textit{name}. The \textit{name} is stored within
% the |\thesis@advisor| macro.
%    \begin{macrocode}
\thesis@def{advisor}
\define@key{thesis}{advisor}{\def\thesis@advisor{#1}}
%    \end{macrocode}
% \end{macro}
% \begin{macro}{\thesis@thanks}
% \subsubsection{The \texttt{thanks} key}
% The \marg{\texttt{thanks}=text} pair sets the acknowledgement
% text to \textit{text}. The \textit{text} is stored within
% the |\thesis@thanks| macro.
%    \begin{macrocode}
\long\def\KV@thesis@thanks#1{%
  \long\def\thesis@thanks{#1}}
%    \end{macrocode}
% \end{macro}
% \begin{macro}{\thesis@assignmentFiles}
% \subsubsection{The \texttt{assignment} key}
% The \marg{\texttt{assignment}=list} pair sets the comma-delimited
% list of paths to the PDF files containing the thesis assignment
% to \textit{list}. The \textit{list} is stored within the
% |\thesis@assignmentFiles| macro.
%    \begin{macrocode}
\define@key{thesis}{assignment}{%
  \def\thesis@assignmentFiles{#1}}
%    \end{macrocode}
% \end{macro}
% When the |\thesis@assignmentFiles| macro is defined and
% non-empty, the style files should take that as a cue that the
% user wishes to typeset the thesis assignment.
% \begin{macro}{\thesis@bibFiles}
% \subsubsection{The \texttt{bib} key}
% The \marg{\texttt{bib}=list} pair sets the comma-delimited
% list of paths to the BIB files containing the bibliography
% databases to \textit{list}. The \textit{list} is stored within
% the |\thesis@bibFiles| macro.
%    \begin{macrocode}
\define@key{thesis}{bib}{%
  \def\thesis@bibFiles{#1}}
%    \end{macrocode}
% \end{macro}
% When the |\thesis@bibFiles| macro is defined and non-empty, the
% style files should take that as a cue that the user wishes to
% typeset the bibliography.
% \begin{macro}{\ifthesis@auto}
% \subsubsection{The \texttt{autoLayout} key}
% The \marg{\texttt{autoLayout}=bool} pair either enables,
% if \textit{bool} is \texttt{true} or unspecified, or disables
% autolayout. Autolayout injects the
% |\thesis@preamble| and |\thesis@postamble| macros at the
% beginning and at the end of the document, respectively. The
% setting can be tested using the |\ifthesis@auto| \ldots |\else|
% \ldots |\fi| conditional. The autolayout is enabled by default.
%    \begin{macrocode}
\newif\ifthesis@auto\thesis@autotrue
\define@key{thesis}{autoLayout}[true]{%
  \def\@true{true}%
  \def\@arg{#1}%
  \ifx\@true\@arg
    \thesis@autotrue
  \else
    \thesis@autofalse
  \fi}
%    \end{macrocode}
% \end{macro}\begin{macro}{\thesis@pages@preamble}
% The \cs{thesis@pages@preamble} macro contains the last page
% number within the preamble of the document. During the first
% \TeX{} compilation, the macro expands to ??.
% \changes{v0.3.45}{2017/05/24}{Defined the
%   \cs{thesis@pages@preamble} and \cs{thesis@pages@postamble}
%   macros. The patch was submitted by Juraj Pálenik. [VN]}
% \begin{macrocode}
\ifx\thesis@pages@preamble\undefined
  \def\thesis@pages@preamble{??}\fi
%    \end{macrocode}
% \end{macro}\begin{macro}{\thesis@pages@postamble}
% The \cs{thesis@pages@postamble} macro contains the last page
% number prior to the postamble of the document. During the first
% \TeX{} compilation, the macro expands to ??.
% \begin{macrocode}
\ifx\thesis@pages@postamble\undefined
  \def\thesis@pages@postamble{??}\fi
%    \end{macrocode}
% \end{macro}
% The \DescribeMacro{\thesis@preamble}|\thesis@preamble|
% and \DescribeMacro{\thesis@postamble}|\thesis@postamble|
% macros temporarily switch to the hyphenation patterns and the
% \textsf{csquotes} style of the main locale and typeset the
% contents of the
% \DescribeMacro{\thesis@blocks@preamble}|\thesis@blocks@preamble|
% or
% \DescribeMacro{\thesis@blocks@postamble}|\thesis@blocks@postamble|
% macros, respectively; the latter two macros are to be redefined
% by the loaded style files.
%
% After expanding |\thesis@blocks@preamble| inside a \TeX{} group,
% the |\thesis@preamble| macro defines the
% \cs{thesis@pages@preamble} macro, writes the definition to the
% auxiliary file, and  clears the page. After leaving the group,
% the |\thesis@preamble| sets up the style of the main matter by
% expanding the
% \DescribeMacro{\thesis@blocks@mainMatter}|\thesis@blocks@mainMatter|
% macro.
%    \begin{macrocode}
\def\thesis@preamble{%
  {\thesis@selectLocale{\thesis@locale}%
  \thesis@blocks@preamble
  \gdef\thesis@pages@preamble{\thepage}
  \write\@auxout{\noexpand\gdef\noexpand
    \thesis@pages@preamble{\thepage}}
  \clearpage}
  \thesis@blocks@mainMatter}

\let\thesis@blocks@preamble\relax
\let\thesis@blocks@mainMatter\relax
% \end{macro}
% Before expanding |\thesis@blocks@postamble| inside a \TeX{}
% group, the |\thesis@postamble| macro defines the
% \cs{thesis@pages@postamble} macro, writes the definition to the
% auxiliary file, and  clears the page.
%    \begin{macrocode}
\def\thesis@postamble{%
  \gdef\thesis@pages@postamble{\thepage}
  \write\@auxout{\noexpand\gdef\noexpand
    \thesis@pages@postamble{\thepage}}
  {\thesis@selectLocale{\thesis@locale}%
  \thesis@blocks@postamble}}

\let\thesis@blocks@postamble\relax
%    \end{macrocode}
% \subsubsection{The \texttt{extra} key}
% \changes{v0.3.45}{2017/05/29}{Added the \texttt{extra} key
%   to \cs{thesissetup} and defined the helper
%   \cs{thesis@def@extra} macro. [VN]}
% The \marg{\texttt{extra}=\marg{keyvals}} pair enables the
% definition of extra data fields, where \textit{keyvals} is a
% comma-delimited list of \textit{key}=\textit{value} pairs as
% defined by the \textsf{keyval} package. For each
% \textit{key}=\textit{value} pair, a |\thesis@extra@|\textit{key}
% is defined to be \textit{value}. These extra data fields are
% provided as a unified interface for passing additional data to
% the style and locale files.
%    \begin{macrocode}
\def\thesis@extra@KV@prefix{KV@thesis@extra@}
\def\thesis@extra@XKV@fams{thesis@extra}
\long\def\KV@thesis@extra#1{%
%    \end{macrocode}
% Patch the \textsc{xkeyval} package to support unknown keys.
%    \begin{macrocode}
  \long\def\XKV@s@tk@ys##1=##2=##3\@nil{%
    \XKV@g@tkeyname##1=\@nil\XKV@tkey
    \expandafter\KV@@sp@def\expandafter\XKV@tkey\expandafter{\XKV@tkey}%
    \ifx\XKV@tkey\@empty
      \XKV@toks{##2}%
      \ifcat$\the\XKV@toks$\else
        \XKV@err{no key specified for value `\the\XKV@toks'}%
      \fi
    \else
      \@expandtwoargs\in@{,\XKV@tkey,}{,\XKV@na,}%
      \ifin@\else
        \XKV@knftrue
        \KV@@sp@def\XKV@tempa{##2}%
        \ifXKV@preset\XKV@s@tk@ys@{##3}\else
          \ifXKV@pl
            \XKV@for@eo\XKV@fams\XKV@tfam{%
              \XKV@makehd\XKV@tfam
              \XKV@s@tk@ys@{##3}%
            }%
          \else
            \XKV@whilist\XKV@fams\XKV@tfam\ifXKV@knf\fi{%
              \XKV@makehd\XKV@tfam
              \XKV@s@tk@ys@{##3}%
            }%
          \fi
        \fi
        \ifXKV@knf
          \ifXKV@inpox
            \ifx\XKV@doxs\relax
              \ifx\@currext\@clsextension\else
                \let\CurrentOption\XKV@tkey\@unknownoptionerror
              \fi
            \else\XKV@doxs\fi
          \else
            \ifXKV@st
              \XKV@addtolist@o\XKV@rm\CurrentOption
            \else
              \ifx\XKV@fams\thesis@extra@XKV@fams
                \expandafter\long\expandafter\def\csname%
                  thesis@extra@\XKV@tkey\endcsname{##2}%
              \else
                \XKV@err{`\XKV@tkey' undefined in families
                         `\XKV@fams'}%
              \fi
            \fi
          \fi
        \else
          \ifXKV@inpox\ifx\XKV@testclass\XKV@documentclass
            \expandafter\XKV@useoption\expandafter{\CurrentOption}%
          \fi\fi
        \fi
      \fi
    \fi
  }%
  \setkeys{thesis@extra}{#1}%
  \def\KV@prefix{KV@thesis@}}
%    \end{macrocode}
% Patch the \textsc{keyval} package to support unknown keys.
%    \begin{macrocode}
\long\def\KV@split#1=#2=#3\relax{%
  \KV@@sp@def\@tempa{#1}%
  \ifx\@tempa\@empty\else
    \expandafter\let\expandafter\@tempc
      \csname\KV@prefix\@tempa\endcsname
    \ifx\@tempc\relax
      \ifx\KV@prefix\thesis@extra@KV@prefix
        \KV@@sp@def\@tempb{#2}%
        \expandafter\let\csname thesis@extra@\@tempa\endcsname
          \@tempb%
      \else
        \KV@errx
         {\@tempa\space undefined}%
      \fi
    \else
      \ifx\@empty#3\@empty
        \KV@default
      \else
        \KV@@sp@def\@tempb{#2}%
        \expandafter\@tempc\expandafter{\@tempb}\relax
      \fi
    \fi
  \fi}
%    \end{macrocode}
% \begin{macro}{\thesis@def@extra}
% The |\thesis@def@extra|\oarg{definition}\marg{name} macro defines
% the |\thesis@extra@|\textit{name} macro to expand
% to either \textit{definition}, if specified, or to
% |\thesis@placeholder@extra@|\textit{name}, where
% |\thesis@placeholder@extra@|\textit{name} is defined to expand to
% <<extra@\textit{name}>>. If |\thesis@extra@|\textit{name} has
% already been defined by the user, |\thesis@def@extra| has no
% effect.
%    \begin{macrocode}
\newcommand{\thesis@def@extra}[2][]{%
  \expandafter\ifx\csname thesis@extra@#2\endcsname\relax
    \def\thesis@placeholder@extra{<<extra@#2>>}%
    \expandafter\let\csname thesis@placeholder@extra@#2\endcsname
      \thesis@placeholder@extra
    \def\thesis@arg{#1}%
    \ifx\empty\thesis@arg
      \expandafter\let\csname thesis@extra@#2\endcsname
        \thesis@placeholder@extra
    \else
      \expandafter\def\csname thesis@extra@#2\endcsname{#1}%
    \fi
  \fi}
%    \end{macrocode}
% \end{macro}
% \end{macro} ^^A The \thesissetup macro definition
% \subsection{The \cs{thesislong} macro}\label{sec:thesislong}
% \begin{macro}{\thesislong}
% The public macro |\thesislong|\marg{key}\marg{value}, can be
% used as an alternative to the |\thesissetup{|\meta{key}%
% | = |\marg{value}|}| public macro:
%    \begin{macrocode}
\long\def\thesislong#1#2{%
  \csname KV@thesis@#1\endcsname{#2}}
%    \end{macrocode}
% This macro is a relict of the time when |\thesissetup| did not
% accept multi-paragraph input.
% \end{macro}
% \subsection{The \cs{thesisload} macro}\label{sec:thesisload}
% \begin{macro}{\thesisload}
% The |\thesisload| macro is responsible for preparing the
% environment for, and consequently loading, the necessary locale
% and style files. By default, the |\thesisload| macro gets
% expanded at the end of the preamble,
% but it can be expanded manually prior to that point, if necessary
% to prevent package clashes. The \DescribeMacro{\ifthesis@loaded}
% |\ifthesis@loaded| macro ensures that the expansion is only
% performed once. For backwards compatibility, the
% \DescribeMacro{\thesis@load}|\thesis@load| macro can be used to
% the same effect.
%    \begin{macrocode}
\newif\ifthesis@loaded\thesis@loadedfalse
\BeforeBeginEnvironment{document}{\thesisload}
\def\thesis@load{\thesisload}
\def\thesisload{%
  \ifthesis@loaded\else
    \thesis@loadedtrue
    \makeatletter
%    \end{macrocode}
% First, the name of the main locale file is fully expanded and
% loaded using the |\thesis@requireLocale| macro. If the user
% specified an explicit empty string as the value of
% |\thesis@locale|, do nothing.
%    \begin{macrocode}
      \ifx\thesis@locale\empty\else
        \edef\thesis@locale{\thesis@locale}
        \thesis@requireLocale{\thesis@locale}
      \fi
%    \end{macrocode}
% Coerce LuaTeX into defining |\l@|\textit{locale} for
% \textit{locale}s with known hyphenation patterns, unless
% \textsf{babel} has been loaded. In that case
% |\l@|\textit{locale} has already been defined.
%    \begin{macrocode}
    \ifluatex
      \ltx@ifpackageloaded{babel}{}{
        % See <article.gmane.org/gmane.comp.tex.luatex.user/5680>
        \RequirePackage[base]{babel}}
    \fi
%    \end{macrocode}
% Fix the value of the |\ifthesis@english| macro.
% \changes{v0.3.45}{2017/05/23}{Updated the \cs{ifthesis@english}
%   macro, so that it no longer dynamically reacts to changes of
%   the locale. Instead, it is now based on the main locale during
%   the expansion of \cs{thesisload}.}
%    \begin{macrocode}
\ifthesis@english
  \expandafter\expandafter\expandafter\let\expandafter\expandafter
    \csname ifthesis@english\endcsname\csname iftrue\endcsname
\else
  \expandafter\expandafter\expandafter\let\expandafter\expandafter
    \csname ifthesis@english\endcsname\csname iffalse\endcsname
\fi
%    \end{macrocode}
% Consequently, the style files are loaded. If the user specified an
% explicit empty string as the value of |\thesis@style|, do nothing.
%    \begin{macrocode}
      \ifx\thesis@style\empty\else
        \thesis@requireStyle{\thesis@style}
      \fi
%    \end{macrocode}
% If the \textsf{babel} or \textsf{polyglossia} locale is identical
% to the thesis locale, the |\thesis@selectLocale| macro will be
% used to globally set up the \textsf{csquotes} style appropriate for
% the given locale.
%    \begin{macrocode}
    \ifx\languagename\empty\else
      \begingroup
      \edef\@doclocale{\languagename}%
      \ifx\@doclocale\thesis@locale
        \endgroup
        \AtBeginDocument{%
          \thesis@selectLocale{\thesis@locale}}%
      \else
        \endgroup
      \fi
    \fi
%    \end{macrocode}
% With the placeholder strings loaded from the locale files, we
% can now inject metadata into the resulting PDF file. To this
% end, the \textsf{hyperref} package is conditionally included with
% the \texttt{unicode} option. Consequently, the following values
% are assigned to the PDF headers:\begin{itemize}
%   \item\texttt{Title} is set to |\thesis@title|.
%   \item\texttt{Author} is set to |\thesis@author|.
%   \item\texttt{Keywords} is set to |\thesis@keywords|.
%   \item\texttt{Creator} is set to \texttt{\thesis@version}.
% \end{itemize}
%    \begin{macrocode}
       \thesis@require{hyperref}
       \hypersetup{
         unicode=true,
         pdfencoding=auto,
         pdftitle=\thesis@title,
         pdfauthor=\thesis@author,
         pdfkeywords=\thesis@keywords,
         pdfcreator=\thesis@version}
%    \end{macrocode}
% If autolayout is enabled, the |\thesis@preamble| and
% |\thesis@postamble| macros are scheduled for expansion at the
% beginning and at the end of the document, respectively. The
% definition of the |\thesis@pages| macro is also scheduled to be
% written to the auxiliary file at the end of the document.
%    \begin{macrocode}
      \ifthesis@auto
        \AtBeginDocument{\thesis@preamble}
        \AtEndDocument{%
          \thesis@postamble
          \write\@auxout{\noexpand\gdef\noexpand\thesis@pages{\thepage}}}
      \else
        \AtEndDocument{%
          \write\@auxout{\noexpand\gdef\noexpand\thesis@pages{\thepage}}}
      \fi
    \makeatother
  \fi}
%    \end{macrocode}
% \end{macro}
% \section{Private API}
% \subsection{File manipulation macros}\label{sec:reflection}
% \begin{macro}{\thesis@exists}
% The |\thesis@exists|\marg{file}\marg{tokens} macro is
% used to test for the existence of a given \textit{file}. If the
% \textit{file} exists, the macro expands to \textit{tokens}.
% Otherwise, a class warning is written to the output.
%    \begin{macrocode}
\def\thesis@exists#1#2{%
  \IfFileExists{#1}{#2}{%
  \ClassWarning{fithesis3}{File #1 doesn't exist}}}
%    \end{macrocode}
% \end{macro}\begin{macro}{\thesis@input}
% The |\thesis@input|\marg{file} macro inputs the given
% \textit{file}, if it exists.
%    \begin{macrocode}
\def\thesis@input#1{%
  \thesis@exists{#1}{\input{#1}}}
%    \end{macrocode}
% \end{macro}\begin{macro}{\thesis@require}
% The |\thesis@require|\oarg{options}\marg{package} expands to
% |\RequirePackage|\oarg{options}\marg{package}, if the specified
% \textit{package} has not yet been loaded.
%    \begin{macrocode}
\newcommand\thesis@require[2][]{%
  \@ifpackageloaded{#2}{}{\RequirePackage[#1]{#2}}}
%    \end{macrocode}
% \end{macro}\begin{macro}{\thesis@requireIfExists}
% The |\thesis@requireIfExists|\oarg{options}\marg{package} expands
% to |\thesis@require|\oarg{options}\marg{package}, if the
% specified \textit{package} exists and has not yet been loaded.
%    \begin{macrocode}
\newcommand\thesis@requireIfExists[2][]{%
  \thesis@exists{#2.sty}{\thesis@require[#1]{#2}}}
%    \end{macrocode}
% \end{macro}\begin{macro}{\thesis@requireStyle}
% If inheritance is enabled for style files, then the
% |\thesis@requireStyle|\marg{style} macro sequentially
% loads each of the following files, provided they exist:
% \begin{enumerate}
%   \item|\thesis@stylepath fithesis-base.sty|
%   \item|\thesis@stylepath\thesis@university/fithesis-base.sty|
%   \item|\thesis@stylepath| \textit{style}|.sty|
% \end{enumerate}If inheritance is disabled for style files, then
% only the last listed file is loaded. The \texttt{fithesis-}
% prefix serves to prevent package clashes with other similarly
% named package files within the \TeX\ directory structure.
%    \begin{macrocode}
\def\thesis@requireStyle#1{%
  \ifthesis@style@inheritance
    \thesis@requireIfExists{\thesis@stylepath fithesis-base}%
    \thesis@requireIfExists{\thesis@stylepath\thesis@university
      /fithesis-base}
  \fi
  \thesis@requireIfExists{\thesis@stylepath#1}}
%    \end{macrocode}
% \end{macro}\begin{macro}{\thesis@requireLocale}
% If inheritance is enabled for style files, then the
% |\thesis@requireLocale|\marg{locale} macro sequentially
% loads each of the following locale files, provided they exist:
% \begin{enumerate}
%   \item|\thesis@localepath fithesis-|\textit{locale}|.def|
%   \item|\thesis@localepath\thesis@university/fithesis-|^^A
%     \textit{locale}|.def|
%   \item|\thesis@localepath\thesis@university/\thesis@faculty/|^^A
%     |fithesis-|\textit{locale}|.def|
% \end{enumerate} If inheritance is disabled for locale files, then
% only the first listed file is loaded. The \texttt{fithesis-}
% prefix serves to prevent clashes with other similarly named files
% within the \TeX\ directory structure. To prevent undesirable side
% effects from locale files being loaded multiple times, the
% |\thesis@|\textit{locale}|@required| macro is defined as a flag,
% which prevents future invocations with the same \textit{locale}.
% The macro can be used within both locale and style files,
% although the usage within locale files is strongly discouraged to
% prevent circular dependencies.
%
% If the \textsf{polyglossia} package is being used, its
% definitions for the respective locale get loaded as well. As a
% consequence, this command may not be used within the document,
% but only in the preamble.
%    \begin{macrocode}
\def\thesis@requireLocale#1{%
  % Ignore redundant requests
  \expandafter\ifx\csname thesis@#1@required\endcsname\relax
    \expandafter\def\csname thesis@#1@required\endcsname{}%
    \@ifpackageloaded{polyglossia}{\setotherlanguage{#1}}{}
    \thesis@input{\thesis@localepath fithesis-#1.def}%
    \ifthesis@locale@inheritance
      \thesis@input{\thesis@localepath\thesis@university/%
        fithesis-#1.def}%
      \thesis@input{\thesis@localepath\thesis@university/%
        \thesis@faculty/fithesis-#1.def}%
    \fi
  \fi}
%    \end{macrocode}\end{macro}
% \subsection{String manipulation macros}
% \begin{macro}{\thesis@}
% The |\thesis@|\marg{name} macro expands to |\thesis@|^^A
% \textit{name}, where \textit{name} gets fully expanded and can
% therefore contain active characters and command sequences.
%    \begin{macrocode}
\def\thesis@#1{\csname thesis@#1\endcsname}
%    \end{macrocode}
% \end{macro}\begin{macro}{\thesis@@}
% The |\thesis@@|\marg{name} macro expands to |\thesis@|^^A
% \textit{locale}|@|\textit{name}, where \textit{locale}
% corresponds to the name of the current locale.  The \textit{name}
% gets fully expanded and can therefore contain active characters
% and command sequences.
%    \begin{macrocode}
\def\thesis@@#1{\thesis@{\thesis@locale @#1}}
%    \end{macrocode}
% \end{macro}
% The \DescribeMacro{\thesis@lower}|\thesis@lower|
% and \DescribeMacro{\thesis@upper}|\thesis@upper|
% macros are used for upper- and lowercasing within
% locale files. To cast the |\thesis@|\textit{name} macro
% to the lower- or uppercase, |\thesis@lower{|\textit{name}|}| or
% |\thesis@upper{|\textit{name}|}| would be used, respectively.
% The \textit{name} gets fully expanded and can therefore contain
% active characters and command sequences.
%    \begin{macrocode}
\def\thesis@lower#1{{%
  \let\ea\expandafter
  \ea\ea\ea\ea\ea\ea\ea\ea\ea\ea\ea\ea\ea\ea\ea\lowercase\ea\ea\ea
  \ea\ea\ea\ea\ea\ea\ea\ea\ea\ea\ea\ea{\ea\ea\ea\ea\ea\ea\ea\ea\ea
  \ea\ea\ea\ea\ea\ea\@gobble\ea\ea\ea\string\ea\csname\csname the%
  sis@#1\endcsname\endcsname}}}
\def\thesis@upper#1{{%
  \let\ea\expandafter
  \ea\ea\ea\ea\ea\ea\ea\ea\ea\ea\ea\ea\ea\ea\ea\uppercase\ea\ea\ea
  \ea\ea\ea\ea\ea\ea\ea\ea\ea\ea\ea\ea{\ea\ea\ea\ea\ea\ea\ea\ea\ea
  \ea\ea\ea\ea\ea\ea\@gobble\ea\ea\ea\string\ea\csname\csname the%
  sis@#1\endcsname\endcsname}}}
%    \end{macrocode}
% The \DescribeMacro{\thesis@@lower}|\thesis@@lower|
% and \DescribeMacro{\thesis@@upper}|\thesis@@upper|
% macros are used for upper- and lowercasing current
% \textit{locale} strings within style files. To cast the
% |\thesis@|\textit{locale}|@|\textit{name} macro to the
% lower- or uppercase, |\thesis@@lower{|\textit{name}|}| or
% |\thesis@@upper{|\textit{name}|}| would be used,
% respectively. The \textit{name} gets fully expanded and can
% therefore contain active characters and command sequences.
%    \begin{macrocode}
\def\thesis@@lower#1{\thesis@lower{\thesis@locale @#1}}
\def\thesis@@upper#1{\thesis@upper{\thesis@locale @#1}}
%    \end{macrocode}
% The \DescribeMacro{\thesis@head}|\thesis@head|
% and \DescribeMacro{\thesis@tail}|\thesis@tail|
% macros are used for retrieving the head or the tail of
% space-separated token sequences that end with |\relax|.
%    \begin{macrocode}
\def\thesis@head#1 #2{%
  \ifx\relax#2%
    \expandafter\@gobbletwo
  \else
    \ #1%
  \fi
  \thesis@head#2}%
\def\thesis@tail#1 #2{%
  \ifx\relax#2%
    #1%
    \expandafter\@gobbletwo
  \fi
  \thesis@tail#2}%
%    \end{macrocode}
% \subsection{General purpose macros}
% The \DescribeMacro{\thesis@pages}|\thesis@pages| macro contains
% the last page number within the document. During the first \TeX\
% compilation, the macro expands to \texttt{??}.
%    \begin{macrocode}
\ifx\thesis@pages\undefined\def\thesis@pages{??}\fi
%    \end{macrocode}
% \DescribeMacro{\thesis@selectLocale}|\thesis@selectLocale|\marg{locale}
% macro redefines the |\thesis@locale| macro to \textit{locale},
% switches to the hyphenation patterns of \textit{locale}, and
% starts using the |\thesis@|\textit{locale}|@csquotesStyle| style
% of the \textsf{csquotes} package. The respective locale files and
% polyglossia locale definitions should be loaded beforehand using
% the |\thesis@requireLocale| macro.
%
% This macro should always be used within a group, so that the
% locale, \textsf{csquotes}, and hyphenation settings return back to
% what the user has specified after the localized blocks of
% typographic material.
%    \begin{macrocode}
\def\thesis@selectLocale#1{%
  \edef\thesis@locale{#1}%
  \ltx@ifpackageloaded{csquotes}{%
    \csq@setstyle{\thesis@@{csquotesStyle}}%
  }{}%
  \ltx@ifpackageloaded{polyglossia}{%
    \selectlanguage{\thesis@locale}
  }{%
    \language\csname l@\thesis@locale\endcsname
  }}
%    \end{macrocode}
% \begin{macro}{\thesis@patch}
% The |\thesis@patch|\oarg{versions}\oarg{patch} macro expands
% \textit{patch}, if |\thesis@version|\texttt{\discretionary{@}^^A
% {@}{@}}|number| (defined at the top of the file
% \texttt{fithesis3.cls}) matches any of the comma-delimited
% \textit{versions}. This macro enables the simple deployment of
% version-targeted patches.
%    \begin{macrocode}
\def\thesis@patch#1#2{%
  \def\thesis@patch@versions{#1}%
  \def\thesis@patch@action{#2}%
  \def\thesis@patch@next##1,{%
    \def\thesis@patch@arg{##1}%
    \def\thesis@patch@relax{\relax}%
    \ifx\thesis@patch@arg\thesis@version@number
      \def\thesis@patch@next####1\relax,{}%
      \expandafter\thesis@patch@action
      \expandafter\thesis@patch@next
    \else\ifx\thesis@patch@arg\thesis@patch@relax\else
      \expandafter\expandafter\expandafter\thesis@patch@next
    \fi\fi}%
  \expandafter\expandafter\expandafter\thesis@patch@next
  \expandafter\thesis@patch@versions\expandafter,\relax,}
%    \end{macrocode}
% \end{macro}
% \iffalse
%</class>
% ^^A Old fithesis classes
%<*oldclass1>

\NeedsTeXFormat{LaTeX2e}
\ProvidesClass{oldfithesis1}[2015/03/04 old fithesis will load fithesis3 MU thesis class]

\ClassWarning{oldfithesis1}{%
  You are using the fithesis class, which has been deprecated.
  The fithesis3 class will be used instead.
  For more information, see <http://www.fi.muni.cz/tech/unix/tex/fithesis.xhtml>%
}\LoadClass{fithesis3}

%</oldclass1>
%
%<*oldclass2>

\NeedsTeXFormat{LaTeX2e}
\ProvidesClass{oldfithesis2}[2015/03/04 old fithesis2 will load fithesis3 MU thesis class]

\ClassWarning{oldfithesis2}{%
  You are using the fithesis2 class, which has been deprecated.
  The fithesis3 class will be used instead.
  For more information, see <http://www.fi.muni.cz/tech/unix/tex/fithesis.xhtml>%
}\LoadClass{fithesis3}

%</oldclass2>
% \fi
%
% \subsection{Locale files}
% \label{sec:locale-files}
% Locale files contain macro definitions for various locales. They
% live in the \texttt{locale/} subtree and they are loaded during
% the main routine (see Section \ref{sec:thesisload}).
%
% When creating a new locale file, it is advisable to create one
% self-contained \texttt{dtx} file, which is then partitioned into
% locale files via the \textsf{docstrip} tool based on the
% respective \texttt{ins} file. A \DescribeMacro{\file} macro
% |\file|\marg{filename} is available for the sectioning of the
% documentation of various files within the \texttt{dtx} file.  For
% more information about \texttt{dtx} files and the
% \textsf{docstrip} tool, consult the \textsf{dtxtut, docstrip,
% doc} and \textsf{ltxdoc} manuals.
%
% Mind that the name of the locale is also used to load hyphenation
% patterns, which is why it shouldn't be arbitrary. To see the
% names of the hyphenation patterns, consult the \textsf{hyph-utf8}
% manual.
%
% \subsubsection{Interface}
% The union of locale files loaded via the locale file inheritance
% scheme (see the definition of the |\thesis@requireLocale| macro
% in Section \ref{sec:reflection}) needs to globally define the
% following macros:
% \begin{itemize}
%   \item|\thesis@|\textit{locale}|@csquotesStyle| -- The name of
%     the style of the \textsf{csquotes} package that matches this
%     locale
%   \item|\thesis@|\textit{locale}|@universityName| -- The name of
%     the university
%   \item|\thesis@|\textit{locale}|@facultyName| -- The name of the
%     faculty
%   \item|\thesis@|\textit{locale}|@assignment| -- The instructions
%   to replace the current page with the official thesis assignment
%   \item|\thesis@|\textit{locale}|@declaration| -- The thesis
%     declaration text
%   \item|\thesis@|\textit{locale}|@fieldTitle| -- The title of
%     the field of study entry
%   \item|\thesis@|\textit{locale}|@advisorTitle| -- The title of
%     the advisor entry
%   \item|\thesis@|\textit{locale}|@authorTitle| -- The title of
%     the author entry
%   \item|\thesis@|\textit{locale}|@abstractTitle| -- The title of
%     the abstract section
%   \item|\thesis@|\textit{locale}|@keywordsTitle| -- The title of
%     the keywords section
%   \item|\thesis@|\textit{locale}|@thanksTitle| -- The title of
%     the acknowledgement section
%   \item|\thesis@|\textit{locale}|@declarationTitle| -- The title
%     of the declaration section
%   \item|\thesis@|\textit{locale}|@idTitle| -- The title of the
%     thesis author's identifier field
%   \item|\thesis@|\textit{locale}|@spring| -- The name of the
%     spring semester
%   \item|\thesis@|\textit{locale}|@fall| -- The name of the
%     fall semester
%   \item|\thesis@|\textit{locale}|@semester| -- The full name of
%     the current semester
%   \item|\thesis@|\textit{locale}|@typeName| -- The name of the
%     thesis type
% \changes{v0.3.46}{2017/06/02}{Lifted the \texttt{authorSignature}
%   and \texttt{formattedDate} strings to the global locale file
%   interface. [VN]}
%   \item|\thesis@|\textit{locale}|@authorSignature| -- The label
%     of the author's signature field
%   \item|\thesis@|\textit{locale}|@formattedDate| -- A formatted
%     date
% \end{itemize} where \textit{locale} is the name of the locale.
%
% \def\file#1{\paragraph{The \texttt{#1} file}}
% \subsubsection{English locale files}
% % \iffalse meta-comment
%
% Copyright 1989-2005 Johannes L. Braams and any individual authors
% listed elsewhere in this file.  All rights reserved.
%    2013-2017 Javier Bezos, Johannes L. Braams
% This file is part of the Babel system.
% --------------------------------------
% 
% It may be distributed and/or modified under the
% conditions of the LaTeX Project Public License, either version 1.3
% of this license or (at your option) any later version.
% The latest version of this license is in
%   http://www.latex-project.org/lppl.txt
% and version 1.3 or later is part of all distributions of LaTeX
% version 2003/12/01 or later.
% 
% This work has the LPPL maintenance status "maintained".
% 
% The Current Maintainer of this work is Javier Bezos.
% 
% The list of all files belonging to the Babel system is
% given in the file `manifest.bbl. See also `legal.bbl' for additional
% information.
% 
% The list of derived (unpacked) files belonging to the distribution
% and covered by LPPL is defined by the unpacking scripts (with
% extension .ins) which are part of the distribution.
% \fi
% \iffalse
%    Tell the \LaTeX\ system who we are and write an entry on the
%    transcript.
%<*dtx>
\ProvidesFile{english.dtx}
%</dtx>
%<english>\ProvidesLanguage{english}
%<american>\ProvidesLanguage{american}
%<usenglish>\ProvidesLanguage{USenglish}
%<british>\ProvidesLanguage{british}
%<ukenglish>\ProvidesLanguage{UKenglish}
%<australian>\ProvidesLanguage{australian}
%<newzealand>\ProvidesLanguage{newzealand}
%<canadian>\ProvidesLanguage{canadian}
%\fi
%\ProvidesFile{english.dtx}
        [2017/06/06 v3.3r English support from the babel system]
%\iffalse
%% File 'english.dtx'
%% Babel package for LaTeX version 2e
%% Copyright (C) 1989 - 2005
%%           by Johannes Braams, TeXniek
%%           2013-2017 Javier Bezos, Johannes Braams
%
%
%    This file is part of the babel system, it provides the source
%    code for the English language definition file.
%<*filedriver>
\documentclass{ltxdoc}
\newcommand*\TeXhax{\TeX hax}
\newcommand*\babel{\textsf{babel}}
\newcommand*\langvar{$\langle \mathit lang \rangle$}
\newcommand*\note[1]{}
\newcommand*\Lopt[1]{\textsf{#1}}
\newcommand*\file[1]{\texttt{#1}}
\begin{document}
 \DocInput{english.dtx}
\end{document}
%</filedriver>
%\fi
% \GetFileInfo{english.dtx}
%
% \changes{english-2.0a}{1990/04/02}{Added checking of format}
% \changes{english-2.1}{1990/04/24}{Reflect changes in babel 2.1}
% \changes{english-2.1a}{1990/05/14}{Incorporated Nico's comments}
% \changes{english-2.1b}{1990/05/14}{merged \file{USenglish.sty} into
%    this file}
% \changes{english-2.1c}{1990/05/22}{fixed typo in definition for
%    american language found by Werenfried Spit (nspit@fys.ruu.nl)}
% \changes{english-2.1d}{1990/07/16}{Fixed some typos}
% \changes{english-3.0}{1991/04/23}{Modified for babel 3.0}
% \changes{english-3.0a}{1991/05/29}{Removed bug found by van der Meer}
% \changes{english-3.0c}{1991/07/15}{Renamed \file{babel.sty} in
%    \file{babel.com}}
% \changes{english-3.1}{1991/11/05}{Rewrote parts of the code to use
%    the new features of babel version 3.1}
% \changes{english-3.3}{1994/02/08}{Update or \LaTeXe}
% \changes{english-3.3c}{1994/06/26}{Removed the use of \cs{filedate}
%    and moved the identification after the loading of
%    \file{babel.def}}
% \changes{english-3.3g}{1996/07/10}{Replaced \cs{undefined} with
%    \cs{@undefined} and \cs{empty} with \cs{@empty} for consistency
%    with \LaTeX} 
% \changes{english-3.3h}{1996/10/10}{Moved the definition of
%    \cs{atcatcode} right to the beginning.} 
% \changes{english-3.3q}{2017/01/10}{Added the proxy files for the
%    dialects}
%
%  \section{The English language}
%
%    The file \file{\filename}\footnote{The file described in this
%    section has version number \fileversion\ and was last revised on
%    \filedate.} defines all the language definition macros for the
%    English language as well as for the American and Australian
%    version of this language. For the Australian version the British
%    hyphenation patterns will be used, if available, for the Canadian
%    variant the American patterns are selected.
%
%    For this language currently no special definitions are needed or
%    available.
%
% \StopEventually{}
%
%    The macro |\LdfInit| takes care of preventing that this file is
%    loaded more than once, checking the category code of the
%    \texttt{@} sign, etc.
% \changes{english-3.3h}{1996/11/02}{Now use \cs{LdfInit} to perform
%    initial checks} 
%    \begin{macrocode}
%<*code>
\LdfInit\CurrentOption{date\CurrentOption}
%    \end{macrocode}
%
%    When this file is read as an option, i.e. by the |\usepackage|
%    command, \texttt{english} could be an `unknown' language in which
%    case we have to make it known.  So we check for the existence of
%    |\l@english| to see whether we have to do something here.
%
% \changes{english-3.0}{1991/04/23}{Now use \cs{adddialect} if
%    language undefined}
% \changes{english-3.0d}{1991/10/22}{removed use of \cs{@ifundefined}}
% \changes{english-3.3c}{1994/06/26}{Now use \cs{@nopatterns} to
%    produce the warning}
% \changes{english-3.3g}{1996/07/10}{Allow british as the name of the
%    UK patterns}
% \changes{english-3.3j}{2000/01/21}{Also allow american english
%    hyphenation patterns to be used for `english'}
%    We allow for the british english patterns to be loaded as either
%    `british', or `UKenglish'. When neither of those is
%    known we try to define |\l@english| as an alias for |\l@american|
%    or |\l@USenglish|.
% \changes{english-3.3k}{2001/02/07}{Added support for canadian}
% \changes{english-3.3n}{2004/06/12}{Added support for australian and
%    newzealand} 
%    \begin{macrocode}
\ifx\l@english\@undefined
  \ifx\l@UKenglish\@undefined
    \ifx\l@british\@undefined
      \ifx\l@american\@undefined
        \ifx\l@USenglish\@undefined
          \ifx\l@canadian\@undefined
            \ifx\l@australian\@undefined
              \ifx\l@newzealand\@undefined
                \@nopatterns{English}
                \adddialect\l@english0
              \else
                \let\l@english\l@newzealand
              \fi
            \else
              \let\l@english\l@australian
            \fi
          \else
            \let\l@english\l@canadian
          \fi
        \else
          \let\l@english\l@USenglish
        \fi
      \else
        \let\l@english\l@american
      \fi
    \else
      \let\l@english\l@british
    \fi 
  \else
    \let\l@english\l@UKenglish
  \fi
\fi
%    \end{macrocode}
%    Because we allow `british' to be used as the babel option we need
%    to make sure that it will be recognised by |\selectlanguage|. In
%    the code above we have made sure that |\l@english| was defined.
%    Now we want to make sure that |\l@british| and |\l@UKenglish| are
%    defined as well. When either of them is we make them equal to
%    each other, when neither is we fall back to the default,
%    |\l@english|. 
% \changes{english-3.3o}{2004/06/14}{Make sure that british patterns
%    are used if they were loaded}
%    \begin{macrocode}
\ifx\l@british\@undefined
  \ifx\l@UKenglish\@undefined
    \adddialect\l@british\l@english
    \adddialect\l@UKenglish\l@english
  \else
    \let\l@british\l@UKenglish
  \fi
\else
  \let\l@UKenglish\l@british
\fi
%    \end{macrocode}
%    `American' is a version of `English' which can have its own
%    hyphenation patterns. The default english patterns are in fact
%    for american english. We allow for the patterns to be loaded as
%    `english' `american' or `USenglish'.
% \changes{english-3.0}{1990/04/23}{Now use \cs{adddialect} for
%    american}
% \changes{english-3.0b}{1991/06/06}{Removed \cs{global} definitions}
% \changes{english-3.3d}{1995/02/01}{Only define american as a
%    dialect when no separate patterns have been loaded}
% \changes{english-3.3g}{1996/07/10}{Allow USenglish as the name of
%    the american patterns} 
%    \begin{macrocode}
\ifx\l@american\@undefined
  \ifx\l@USenglish\@undefined
%    \end{macrocode}
%    When the patterns are not know as `american' or `USenglish' we
%    add a ``dialect''.
%    \begin{macrocode}
    \adddialect\l@american\l@english
  \else
    \let\l@american\l@USenglish
  \fi
\else
%    \end{macrocode}
%    Make sure that USenglish is known, even if the patterns were
%    loaded as `american'.
% \changes{english-3.3j}{2000/01/21}{Ensure that \cs{l@USenglish} is
%    alway defined}
% \changes{english-3.3l}{2001/04/15}{Added missing backslash}
%    \begin{macrocode}
  \ifx\l@USenglish\@undefined
    \let\l@USenglish\l@american
  \fi
\fi
%    \end{macrocode}
%
% \changes{english-3.3k}{2001/02/07}{Added support for canadian}
%    `Canadian' english spelling is a hybrid of British and American
%    spelling. Although so far no special `translations' have been
%    reported we allow this file to be loaded by the option
%    \Lopt{candian} as well.
%    \begin{macrocode}
\ifx\l@canadian\@undefined
  \adddialect\l@canadian\l@american
\fi
%    \end{macrocode}
%
% \changes{english-3.3n}{2004/06/12}{Added support for australian and
%   newzealand}
%    `Australian' and `New Zealand' english spelling seem to be the
%    same as British spelling. Although so far no special
%    `translations' have been reported we allow this file to be loaded
%    by the options \Lopt{australian} and \Lopt{newzealand} as well.
%    \begin{macrocode}
\ifx\l@australian\@undefined
  \adddialect\l@australian\l@british
\fi
\ifx\l@newzealand\@undefined
  \adddialect\l@newzealand\l@british
\fi
%    \end{macrocode}
%
 
%  \begin{macro}{\englishhyphenmins}
% \changes{english-3.3m}{2003/11/17}{Added default for setting of
%    hyphenmin parameters} 
%    This macro is used to store the correct values of the hyphenation
%    parameters |\lefthyphenmin| and |\righthyphenmin|.
%    \begin{macrocode}
\providehyphenmins{\CurrentOption}{\tw@\thr@@}
%    \end{macrocode}
%  \end{macro}
%
%    The next step consists of defining commands to switch to (and
%    from) the English language.
% \begin{macro}{\captionsenglish}
%    The macro |\captionsenglish| defines all strings used
%    in the four standard document classes provided with \LaTeX.
% \changes{english-3.0b}{1991/06/06}{Removed \cs{global} definitions}
% \changes{english-3.0b}{1991/06/06}{\cs{pagename} should be
%    \cs{headpagename}}
% \changes{english-3.1a}{1991/11/11}{added \cs{seename} and
%    \cs{alsoname}}
% \changes{english-3.1b}{1992/01/26}{added \cs{prefacename}}
% \changes{english-3.2}{1993/07/15}{\cs{headpagename} should be
%    \cs{pagename}}
% \changes{english-3.3e}{1995/07/04}{Added \cs{proofname} for
%    AMS-\LaTeX}
% \changes{english-3.3g}{1996/07/10}{Construct control sequence on the
%    fly} 
% \changes{english-3.3j}{2000/09/19}{Added \cs{glossaryname}}
%    \begin{macrocode}
\@namedef{captions\CurrentOption}{%
  \def\prefacename{Preface}%
  \def\refname{References}%
  \def\abstractname{Abstract}%
  \def\bibname{Bibliography}%
  \def\chaptername{Chapter}%
  \def\appendixname{Appendix}%
  \def\contentsname{Contents}%
  \def\listfigurename{List of Figures}%
  \def\listtablename{List of Tables}%
  \def\indexname{Index}%
  \def\figurename{Figure}%
  \def\tablename{Table}%
  \def\partname{Part}%
  \def\enclname{encl}%
  \def\ccname{cc}%
  \def\headtoname{To}%
  \def\pagename{Page}%
  \def\seename{see}%
  \def\alsoname{see also}%
  \def\proofname{Proof}%
  \def\glossaryname{Glossary}%
  }
%    \end{macrocode}
% \end{macro}
%
% \begin{macro}{\dateenglish}
%    In order to define |\today| correctly we need to know whether it
%    should be `english', `australian', or `american'. We can find
%    this out by checking the value of |\CurrentOption|.
% \changes{english-3.3j}{2000/01/21}{Make sure that the value of
%    \cs{today} is correct for both options `american' and
%    `USenglish'}
% \changes{english-3.3n}{2004/06/12}{Added support for `Australian'
%    and `Newzealand'}
% \changes{english-3.3o}{2004/06/14}{Explicitly choose the UK form of
%    date} 
% \changes{english-3.3p}{2012/11/07}{Warning if `english' is used with
%    other options} 
%    \begin{macrocode}
\def\bbl@tempa{british}
\ifx\CurrentOption\bbl@tempa\def\bbl@tempb{UK}\fi
\def\bbl@tempa{UKenglish}
\ifx\CurrentOption\bbl@tempa\def\bbl@tempb{UK}\fi
\def\bbl@tempa{american}
\ifx\CurrentOption\bbl@tempa\def\bbl@tempb{US}\fi
\def\bbl@tempa{USenglish}
\ifx\CurrentOption\bbl@tempa\def\bbl@tempb{US}\fi
\def\bbl@tempa{canadian}
\ifx\CurrentOption\bbl@tempa\def\bbl@tempb{US}\fi
\def\bbl@tempa{australian}
\ifx\CurrentOption\bbl@tempa\def\bbl@tempb{AU}\fi
\def\bbl@tempa{newzealand}
\ifx\CurrentOption\bbl@tempa\def\bbl@tempb{AU}\fi
\def\bbl@tempa{english}
\ifx\CurrentOption\bbl@tempa
  \AtEndOfPackage{\@nameuse{bbl@englishwarning}}
\else
  \edef\bbl@englishwarning{%
    \let\noexpand\bbl@englishwarning\relax
    \noexpand\PackageWarning{Babel}{%
      The package option `english' should not be used\noexpand\MessageBreak
      with a more specific one (like `\CurrentOption')}}
\fi
%    \end{macrocode}
%
%    The macro |\dateenglish| redefines the command |\today| to
%    produce English dates.
% \changes{english-3.0b}{1991/06/06}{Removed \cs{global} definitions}
% \changes{english-3.3g}{1996/07/10}{Construct control sequence on the
%    fly}
% \changes{english-3.3i}{1997/10/01}{Use \cs{edef} to define \cs{today}
%    to save memory}
% \changes{english-3.3i}{1998/03/28}{use \cs{def} instead of
%    \cs{edef}}
%    \begin{macrocode}
\def\bbl@tempa{UK}
\ifx\bbl@tempa\bbl@tempb
  \@namedef{date\CurrentOption}{%
    \def\today{\ifcase\day\or
      1st\or 2nd\or 3rd\or 4th\or 5th\or
      6th\or 7th\or 8th\or 9th\or 10th\or
      11th\or 12th\or 13th\or 14th\or 15th\or
      16th\or 17th\or 18th\or 19th\or 20th\or
      21st\or 22nd\or 23rd\or 24th\or 25th\or
      26th\or 27th\or 28th\or 29th\or 30th\or
      31st\fi~\ifcase\month\or
      January\or February\or March\or April\or May\or June\or
      July\or August\or September\or October\or November\or 
      December\fi\space \number\year}}
%    \end{macrocode}
% \end{macro}
%
% \begin{macro}{\dateaustralian}
%    Now, test for `australian' or `american'.
% \changes{english-3.3n}{2004/06/12}{Add australian date}
%    \begin{macrocode}
\else
%    \end{macrocode}
%
%    The macro |\dateaustralian| redefines the command |\today| to
%    produce Australian resp.\ New Zealand dates.
%    \begin{macrocode}
  \def\bbl@tempa{AU}
  \ifx\bbl@tempa\bbl@tempb
    \@namedef{date\CurrentOption}{%
      \def\today{\number\day~\ifcase\month\or
        January\or February\or March\or April\or May\or June\or
        July\or August\or September\or October\or November\or 
        December\fi\space \number\year}}
%    \end{macrocode}
% \end{macro}
%
% \begin{macro}{\dateamerican}
%    The macro |\dateamerican| redefines the command |\today| to
%    produce American dates.
% \changes{english-3.0b}{1991/06/06}{Removed \cs{global} definitions}
% \changes{english-3.3i}{1997/10/01}{Use \cs{edef} to define
%    \cs{today} to save memory}
% \changes{english-3.3i}{1998/03/28}{use \cs{def} instead of
%    \cs{edef}}
%    \begin{macrocode}
  \else
    \@namedef{date\CurrentOption}{%
      \def\today{\ifcase\month\or
        January\or February\or March\or April\or May\or June\or
        July\or August\or September\or October\or November\or
        December\fi \space\number\day, \number\year}}
  \fi
\fi
%    \end{macrocode}
% \end{macro}
%
% \begin{macro}{\extrasenglish}
% \begin{macro}{\noextrasenglish}
%    The macro |\extrasenglish| will perform all the extra definitions
%    needed for the English language. The macro |\noextrasenglish| is
%    used to cancel the actions of |\extrasenglish|.  For the moment
%    these macros are empty but they are defined for compatibility
%    with the other language definition files.
%
% \changes{english-3.3g}{1996/07/10}{Construct control sequences on
%    the fly} 
%    \begin{macrocode}
\@namedef{extras\CurrentOption}{}
\@namedef{noextras\CurrentOption}{}
%    \end{macrocode}
% \end{macro}
% \end{macro}
%
%    The macro |\ldf@finish| takes care of looking for a
%    configuration file, setting the main language to be switched on
%    at |\begin{document}| and resetting the category code of
%    \texttt{@} to its original value.
% \changes{english-3.3h}{1996/11/02}{Now use \cs{ldf@finish} to wrap
%    up} 
%    \begin{macrocode}
\ldf@finish\CurrentOption
%</code>
%    \end{macrocode}
%
% Finally, We create  a few proxy files, which just load english.ldf.
%
%    \begin{macrocode}
%<*american|usenglish|british|ukenglish|australian|newzealand|canadian>
\input english.ldf\relax
%</american|usenglish|british|ukenglish|australian|newzealand|canadian>
%    \end{macrocode}
%
% \Finale
%%
%% \CharacterTable
%%  {Upper-case    \A\B\C\D\E\F\G\H\I\J\K\L\M\N\O\P\Q\R\S\T\U\V\W\X\Y\Z
%%   Lower-case    \a\b\c\d\e\f\g\h\i\j\k\l\m\n\o\p\q\r\s\t\u\v\w\x\y\z
%%   Digits        \0\1\2\3\4\5\6\7\8\9
%%   Exclamation   \!     Double quote  \"     Hash (number) \#
%%   Dollar        \$     Percent       \%     Ampersand     \&
%%   Acute accent  \'     Left paren    \(     Right paren   \)
%%   Asterisk      \*     Plus          \+     Comma         \,
%%   Minus         \-     Point         \.     Solidus       \/
%%   Colon         \:     Semicolon     \;     Less than     \<
%%   Equals        \=     Greater than  \>     Question mark \?
%%   Commercial at \@     Left bracket  \[     Backslash     \\
%%   Right bracket \]     Circumflex    \^     Underscore    \_
%%   Grave accent  \`     Left brace    \{     Vertical bar  \|
%%   Right brace   \}     Tilde         \~}
%%
\endinput

% \subsubsection{Czech locale files}
% % \file{locale/fithesis-czech.def}
% This is the base file of the Czech locale.\iffalse
%<*base>
% \fi\begin{macrocode}
\ProvidesFile{fithesis/locale/fithesis-czech.def}[2017/05/15]
%    \end{macrocode}
% The locale file defines all the private macros mandated by the
% locale file interface.
% \begin{macro}{\thesis@czech@gender@koncovka}
% The locale file also defines the |\thesis@czech@gender@koncovka|
% macro, which expands to the correct verb ending based on the
% value of the |\thesis@ifwoman| macro and the
% \end{macro}\begin{macro}{\thesis@czech@typeName@akuzativ}
% |\thesis@czech@typeName@akuzativ| containing the accusative case
% of the thesis type name.
% \end{macro}\begin{macrocode}

% Pomocná makra
\gdef\thesis@czech@gender@koncovka{%
  \ifthesis@woman a\fi}

% Csquotes styl
\gdef\thesis@czech@csquotesStyle{german}

% Zástupné texty
\gdef\thesis@czech@universityName{Název univerzity}
\gdef\thesis@czech@facultyName{Název fakulty}
\gdef\thesis@czech@assignment{%
  \ifthesis@digital@
    Na tomto místě se v~tištěné práci nachází oficiální podepsané
    zadání práce.
  \else
    Místo tohoto listu vložte kopii oficiálního podepsaného zadání
    práce.
  \fi}
\gdef\thesis@czech@declaration{Text prohlášení ...}

% Časové údaje
\gdef\thesis@czech@spring{jaro}
\gdef\thesis@czech@fall{podzim}
\gdef\thesis@czech@semester{%
  \thesis@{czech@\thesis@season} \thesis@seasonYear}
\gdef\thesis@czech@formattedDate{{%
  \thesis@day.
  \newcount\@month\expandafter\@month\thesis@month\relax
  \ifnum\@month=1%
    ledna
  \else\ifnum\@month=2%
    února
  \else\ifnum\@month=3%
    března
  \else\ifnum\@month=4%
    dubna
  \else\ifnum\@month=5%
    května
  \else\ifnum\@month=6%
    června
  \else\ifnum\@month=7%
    července
  \else\ifnum\@month=8%
    srpna
  \else\ifnum\@month=9%
    září
  \else\ifnum\@month=10%
    října
  \else\ifnum\@month=11%
    listopadu
  \else\ifnum\@month=12%
    prosince
  \else
    <<neznámý měsíc (\the\@month)>>
  \fi\fi\fi\fi\fi\fi
  \fi\fi\fi\fi\fi\fi
  \thesis@year}}

% Různé
\gdef\thesis@czech@authorSignature{Podpis autora}
\gdef\thesis@czech@fieldTitle{Obor}
\gdef\thesis@czech@advisorTitle{Vedoucí práce}
\gdef\thesis@czech@authorTitle{Autor}
\gdef\thesis@czech@abstractTitle{Shrnutí}
\gdef\thesis@czech@keywordsTitle{Klíčová slova}
\gdef\thesis@czech@thanksTitle{Poděkování}
\gdef\thesis@czech@declarationTitle{Prohlášení}
\gdef\thesis@czech@idTitle{ID}
\gdef\thesis@czech@typeName@sempaper{Seminární práce}
\gdef\thesis@czech@typeName@bachelors{Bakalářská práce}
\gdef\thesis@czech@typeName@masters{Diplomová práce}
\gdef\thesis@czech@typeName@proposal{Teze závěrečné práce}
\gdef\thesis@czech@typeName@doctoral{Disertační práce}
\gdef\thesis@czech@typeName@rigorous{Rigorózní práce}
\gdef\thesis@czech@typeName{%
  \ifx\thesis@type\thesis@sempaper
    \thesis@czech@typeName@sempaper
  \else\ifx\thesis@type\thesis@bachelors
    \thesis@czech@typeName@bachelors
  \else\ifx\thesis@type\thesis@masters
    \thesis@czech@typeName@masters
  \else\ifx\thesis@type\thesis@proposal
    \thesis@czech@typeName@proposal
  \else\ifx\thesis@type\thesis@doctoral
    \thesis@czech@typeName@doctoral
  \else\ifx\thesis@type\thesis@rigorous
    \thesis@czech@typeName@rigorous
  \else
    <<Neznámý typ práce (\thesis@type)>>%
  \fi\fi\fi\fi\fi\fi}
\gdef\thesis@czech@typeName@akuzativ@sempaper{Seminární práci}
\gdef\thesis@czech@typeName@akuzativ@bachelors{Bakalářskou práci}
\gdef\thesis@czech@typeName@akuzativ@masters{Diplomovou práci}
\gdef\thesis@czech@typeName@akuzativ@proposal{Tezi závěrečné práce}
\gdef\thesis@czech@typeName@akuzativ@doctoral{Disertační práci}
\gdef\thesis@czech@typeName@akuzativ@rigorous{Rigorózní práci}
\gdef\thesis@czech@typeName@akuzativ{%
  \ifx\thesis@type\thesis@sempaper
    \thesis@czech@typeName@akuzativ@sempaper
  \else\ifx\thesis@type\thesis@bachelors
    \thesis@czech@typeName@akuzativ@bachelors
  \else\ifx\thesis@type\thesis@masters
    \thesis@czech@typeName@akuzativ@masters
  \else\ifx\thesis@type\thesis@proposal
    \thesis@czech@typeName@akuzativ@proposal
  \else\ifx\thesis@type\thesis@doctoral
    \thesis@czech@typeName@akuzativ@doctoral
  \else\ifx\thesis@type\thesis@rigorous
    \thesis@czech@typeName@akuzativ@rigorous
  \else
    <<Neznámý typ práce (\thesis@type)>>%
  \fi\fi\fi\fi\fi\fi}
%    \end{macrocode}\iffalse
%</base>
% \fi\file{locale/mu/fithesis-czech.def}
% This is the Czech locale file specific to the Masaryk
% University in Brno. It replaces the \texttt{universityName}
% placeholder with the correct value and defines the
% \texttt{declaration} and \texttt{idTitle} strings.
% \iffalse
%<*mu>
% \fi\begin{macrocode}
\ProvidesFile{fithesis/locale/mu/fithesis-czech.def}[2017/06/02]

% Zástupné texty
\gdef\thesis@czech@universityName{Masarykova univerzita}
\gdef\thesis@czech@declaration{Prohlašuji, že jsem
  \thesis@lower{czech@typeName@akuzativ} zpracoval%
  \thesis@czech@gender@koncovka\ samostatně a~%
  použil\thesis@czech@gender@koncovka\ jen prameny
  uvedené v~seznamu literatury.}

% Bibliografický záznam
\gdef\thesis@czech@bib@title{Bibliografický záznam}
\gdef\thesis@czech@bib@pages{str}
%    \end{macrocode}
% \changes{v0.3.46}{2017/06/02}{Lifted the \texttt{bib@author},
%   \texttt{bib@thesisTitle}, and \texttt{bib@advisor} strings from
%   \texttt{locale/mu/sci/*.def} to \texttt{locale/mu/*.def},
%   so that they can be shared with \texttt{locale/mu/econ/*.def}.
%   [VN]}
%    \begin{macrocode}
\global\let\thesis@czech@bib@author\thesis@czech@authorTitle
\gdef\thesis@czech@bib@thesisTitle{Název práce}
\global\let\thesis@czech@bib@advisor\thesis@czech@advisorTitle

% Různé
\gdef\thesis@czech@idTitle{UČO}
%    \end{macrocode}\iffalse
%</mu>
% \fi\file{locale/mu/law/fithesis-czech.def}
% This is the Czech locale file specific to the Faculty of Law at
% the Masaryk University in Brno. It replaces the
% \texttt{facultyName} placeholder with the correct value, defines
% the \texttt{facultyLongName} required by the
% |\thesis@blocks@cover| and the |\thesis@blocks@titlePage| blocks
% and replaces the \texttt{abstractTitle} string in accordance
% with the requirements of the faculty.
% \iffalse
%<*mu/law>
% \fi\begin{macrocode}
\ProvidesFile{fithesis/locale/mu/law/fithesis-czech.def}[2015/06/26]

% Různé
\gdef\thesis@czech@abstractTitle{Abstrakt}

% Zástupné texty
\gdef\thesis@czech@facultyName{Právnická fakulta}
\gdef\thesis@czech@facultyLongName{Právnická fakulta Masarykovy
  univerzity}
%    \end{macrocode}\iffalse
%</mu/law>
% \fi\file{locale/mu/fsps/fithesis-czech.def}
% This is the Czech locale file specific to the Faculty of Sports
% Studies at the Masaryk University in Brno. It replaces the
% \texttt{facultyName} placeholder with the correct value and
% redefines the \texttt{fieldTitle} string in accordance with the
% common usage at the faculty. The locale file also redefines the
% \texttt{declaration} string in accordance with the requirements
% of the faculty.
% \iffalse
%<*mu/fsps>
% \fi\begin{macrocode}
\ProvidesFile{fithesis/locale/mu/fsps/fithesis-czech.def}[2017/05/15]

% Zástupné texty
\gdef\thesis@czech@facultyName{Fakulta sportovních studií}
\gdef\thesis@czech@declaration{Prohlašuji, že jsem
  \thesis@lower{czech@typeName@akuzativ} vypracoval%
  \thesis@czech@gender@koncovka\ samostatně a~na základě
  literatury a~pramenů uvedených v~použitých zdrojích.}

% Různé
\gdef\thesis@czech@fieldTitle{Specializace}
%    \end{macrocode}\iffalse
%</mu/fsps>
% \fi\file{locale/mu/fss/fithesis-czech.def}
% This is the Czech locale file specific to the Faculty of Social
% Studies at the Masaryk University in Brno. It replaces the
% \texttt{facultyName} and \texttt{assignment} placeholders with
% the correct values.
% \iffalse
%<*mu/fss>
% \fi\begin{macrocode}
\ProvidesFile{fithesis/locale/mu/fss/fithesis-czech.def}[2016/05/25]

% Zástupné texty
\gdef\thesis@czech@facultyName{Fakulta sociálních studií}
\gdef\thesis@czech@assignment{%
  \ifthesis@digital@
    Na tomto místě se v~tištěné práci nachází oficiální podepsané
    zadání práce, prohlášení autora školního díla nebo obojí.
  \else
    Místo tohoto listu vložte kopie oficiálního podepsaného zadání
    práce nebo prohlášení autora školního díla nebo obojí
    v~závislosti na požadavcích příslušné katedry.
  \fi}

%    \end{macrocode}\iffalse
%</mu/fss>
% \fi\file{locale/mu/econ/fithesis-czech.def}
% This is the Czech locale file specific to the Faculty of
% Economics and Administration at the Masaryk University in Brno.
% It replaces the \texttt{facultyName} and \texttt{abstractTitle}
% placeholders with the correct values. The locale file also
% redefines the \texttt{declaration} string in accordance with
% the requirements of the faculty and defines the private macros
% required by the |\thesis@blocks@|\discretionary{}{}{}|bibEntry|
% block defined within the \texttt{style/mu/fithesis-econ.sty}
% style file.
% \iffalse
%<*mu/econ>
% \fi\begin{macrocode}
\ProvidesFile{fithesis/locale/mu/econ/fithesis-czech.def}[2017/06/02]

% Zástupné texty
\gdef\thesis@czech@facultyName{Ekonomicko-správní fakulta}

% Bibliografický záznam
%    \end{macrocode}
% \changes{v0.3.46}{2017/06/02}{Defined strings required by
%   \cs{thesis@blocks@bibEntry} from
%   \texttt{style/mu/fithesis-econ.sty} in
%   \texttt{locale/mu/econ/*.def}. [VN]}
%    \begin{macrocode}
\gdef\thesis@czech@bib@thesisTitleEn{Název práce v angličtině}
\gdef\thesis@czech@bib@department{Katedra}
\gdef\thesis@czech@bib@year{Rok obhajoby}

% Různé
%    \end{macrocode}
% \changes{v0.3.46}{2017/06/02}{Updated the
%   \cs{abstractTitle} string in \texttt{locale/mu/econ/*.def} in
%   accordance with the 2/2017 dean's directive. The patch was
%   submitted by Jana Ratajská. [VN]}
%    \begin{macrocode}
\gdef\thesis@czech@abstractTitle{Anotace}
%    \end{macrocode}
% \changes{v0.3.46}{2017/06/02}{Updated the \cs{declaration} string
%   in \texttt{locale/mu/econ/*.def} in accordance with the 2/2017
%   dean's directive. [VN]}
% The following extra data field is defined for
% \texttt{declaration} string: \begin{itemize}
%   \item|advisorCsGenitiv| -- the advisor's name in
%     genitive following Czech morphology.
% \end{itemize}
%    \begin{macrocode}
\thesis@def@extra{advisorCsGenitiv}
\gdef\thesis@czech@declaration{Prohlašuji, že jsem
  \thesis@lower{czech@typeName@akuzativ} zpracoval%
  \thesis@czech@gender@koncovka\ samostatně pod vedením
  \thesis@extra@advisorCsGenitiv\ 
  a~uvedl\thesis@czech@gender@koncovka\ v~ní všechny
  odborné zdroje v~souladu s~právními předpisy, vnitřními
  předpisy Masarykovy univerzity a~vnitřními akty řízení
  Masarykovy univerzity a~Ekonomicko-správní fakulty MU.}
%    \end{macrocode}\iffalse
%</mu/econ>
% \fi\file{locale/mu/med/fithesis-czech.def}
% This is the Czech locale file specific to the Faculty of
% Medicine at the Masaryk University in Brno.
% It replaces the \texttt{facultyName} placeholder with the
% correct value and redefines the \texttt{abstractTitle} string in
% accordance with the common usage at the faculty. The file also
% defines the \texttt{bib@title} and \texttt{bib@pages} strings
% required by the |\thesis@blocks@bibEntry| block defined within
% the \texttt{style/mu/fithesis-med.sty} style file.
% \iffalse
%<*mu/med>
% \fi\begin{macrocode}
\ProvidesFile{fithesis/locale/mu/med/fithesis-czech.def}[2016/03/23]

% Zástupné texty
\gdef\thesis@czech@facultyName{Lékařská fakulta}

% Různé
\gdef\thesis@czech@abstractTitle{Anotace}
%    \end{macrocode}\iffalse
%</mu/med>
% \fi\file{locale/mu/fi/fithesis-czech.def}
% This is the Czech locale file specific to the Faculty of
% Informatics at the Masaryk University in Brno.
% It replaces the \texttt{facultyName} placeholder with the
% correct value and redefines the \texttt{declaration} string in
% accordance with the requirements of the faculty. The file also
% defines the \texttt{advisorSignature} string required by the
% |\thesis@blocks@titlePage| block defined within the
% \texttt{style/mu/\discretionary{}{}{}fithesis-fi.sty}
% style file.
% \iffalse
%<*mu/fi>
% \fi\begin{macrocode}
\ProvidesFile{fithesis/locale/mu/fi/fithesis-czech.def}[2016/05/25]

% Zástupné texty
\gdef\thesis@czech@facultyName{Fakulta informatiky}
\gdef\thesis@czech@assignment{%
  \ifthesis@digital@
    Na tomto místě se v~tištěné práci nachází oficiální podepsané
    zadání práce a prohlášení autora školního díla.
  \else
    Místo tohoto listu vložte kopie oficiálního podepsaného zadání
    práce a prohlášení autora školního díla.
  \fi}
\gdef\thesis@czech@declaration{%
  Prohlašuji, že tato \thesis@lower{czech@typeName} je mým
  původním autorským dílem, které jsem vypracoval%
  \thesis@czech@gender@koncovka\ samostatně. Všechny zdroje,
  prameny a~literaturu, které jsem při vypracování
  používal\thesis@czech@gender@koncovka\ nebo z~nich
  čerpal\thesis@czech@gender@koncovka, v~práci řádně cituji
  s~uvedením úplného odkazu na příslušný zdroj.}

% Ostatní
\gdef\thesis@czech@advisorSignature{Podpis vedoucího}
\gdef\thesis@czech@typeName@proposal{Teze disertační práce}
\gdef\thesis@czech@typeName@akuzativ@proposal{Tezi disertační práce}
%    \end{macrocode}\iffalse
%</mu/fi>
% \fi\file{locale/mu/phil/fithesis-czech.def}
% This is the Czech locale file specific to the Faculty of
% Arts at the Masaryk University in Brno.
% It replaces the \texttt{facultyName} placeholder with the
% correct value. It also redefines the \texttt{declaration},
% \texttt{typeName} and \texttt{typeName@akuzativ} strings in
% accordance with the requirements of the faculty.
% 
% The locale file also defines the \texttt{departmentName}
% string, which is used by the \texttt{style/mu/fithesis-phil^^A
% .sty} style file, when typesetting the names of known
% departments.
% \iffalse
%<*mu/phil>
% \fi\begin{macrocode}
\ProvidesFile{fithesis/locale/mu/phil/fithesis-czech.def}[2016/03/22]

% Zástupné texty
\gdef\thesis@czech@facultyName{Filozofická fakulta}
\gdef\thesis@czech@departmentName{%
  \ifx\thesis@department\thesis@departments@kisk
    Kabinet informačních studií a knihovnictví%
  \else
    <<Neznámé oddělení (\thesis@department)>>%
  \fi}
\gdef\thesis@czech@declaration{%
  \ifx\thesis@department\thesis@departments@kisk
    Prohlašuji, že jsem předkládanou práci zpracoval%
    \thesis@czech@gender@koncovka\ samostatně a~použil%
    \thesis@czech@gender@koncovka\ jen uvedené prameny a~%
    literaturu. Současně dávám svolení k~tomu, aby elektronická
    verze této práce byla zpřístupněna přes informační systém
    Masarykovy univerzity.%
  \else
    Prohlašuji, že jsem \thesis@lower{czech@typeName@akuzativ}
    vypracoval\thesis@czech@gender@koncovka\ samostatně s~využitím
    uvedené literatury.%
  \fi}

% Ostatní
\global\let\thesis@czech@typeName@super
  \thesis@czech@typeName
\gdef\thesis@czech@typeName{%
  \ifx\thesis@type\thesis@bachelors
    Bakalářská diplomová práce%
  \else\ifx\thesis@type\thesis@masters
    Magisterská diplomová práce%
  \else
    \thesis@czech@typeName@super
  \fi\fi}

\global\let\thesis@czech@typeName@akuzativ@super
  \thesis@czech@typeName@akuzativ
\gdef\thesis@czech@typeName@akuzativ{%
  \ifx\thesis@type\thesis@bachelors
    Diplomovou práci%
  \else\ifx\thesis@type\thesis@masters
    Diplomovou práci%
  \else
    \thesis@czech@typeName@akuzativ@super
  \fi\fi}
%    \end{macrocode}\iffalse
%</mu/phil>
% \fi\file{locale/mu/ped/fithesis-czech.def}
% This is the Czech locale file specific to the Faculty of
% Education at the Masaryk University in Brno.
% It replaces the \texttt{facultyName} placeholder with the
% correct value. The file also defines the
% \texttt{bib@title} and \texttt{bib@pages} strings required by the
% |\thesis@blocks@bibEntry| block defined within the
% \texttt{style/mu/\discretionary{}{}{}fithesis-ped.sty}
% style file.
% \iffalse
%<*mu/ped>
% \fi\begin{macrocode}
\ProvidesFile{fithesis/locale/mu/ped/fithesis-czech.def}[2016/03/22]

% Zástupné texty
\gdef\thesis@czech@facultyName{Pedagogická fakulta}
%    \end{macrocode}\iffalse
%</mu/ped>
% \fi\file{locale/mu/sci/fithesis-czech.def}
% This is the Czech locale file specific to the Faculty of Science
% at the Masaryk University in Brno.  It defines the private macros
% required by the |\thesis@blocks@|\discretionary{}{}{}|bibEntryCs|
% block defined within the
% \texttt{style/mu/fithesis-sci.sty} style file.  It also
% replaces the \texttt{facultyName} placeholder with the correct
% value and redefines the \texttt{abstractTitle} and
% \texttt{declaration} strings in accordance with the formal
% requirements of the faculty.
% \iffalse
%<*mu/sci>
% \fi\begin{macrocode}
\ProvidesFile{fithesis/locale/mu/sci/fithesis-czech.def}[2017/06/02]

% Zástupné texty
\gdef\thesis@czech@facultyName{Přírodovědecká fakulta}

% Ostatní
\gdef\thesis@czech@abstractTitle{Abstrakt}
\gdef\thesis@czech@declaration{%
  Prohlašuji, že jsem svoji \thesis@lower{czech@typeName@%
  akuzativ} vypracoval\thesis@czech@gender@koncovka\ samo%
  statně s~využitím informačních zdrojů, které jsou v~práci
  citovány.}

% Bibliografický záznam
\gdef\thesis@czech@bib@programme{Studijní program}
\global\let\thesis@czech@bib@field\thesis@czech@fieldTitle
\gdef\thesis@czech@bib@academicYear{Akademický rok}
\gdef\thesis@czech@bib@pages{Počet stran}
\global\let\thesis@czech@bib@keywords\thesis@czech@keywordsTitle
%    \end{macrocode}\iffalse
%</mu/sci>
% \fi

% \subsubsection{Slovak locale files}
% % \file{locale/fithesis-slovak.def}
% This is the base file of the Slovak locale.\iffalse
%<*base>
% \fi\begin{macrocode}
\ProvidesFile{fithesis/locale/fithesis-slovak.def}[2017/05/15]
%    \end{macrocode}
% The locale file defines all the private macros mandated by the
% locale file interface.
% \begin{macro}{\thesis@slovak@gender@koncovka}
% The locale file also defines the |\thesis@slovak@gender@koncovka|
% macro, which expands to the correct verb ending based on the
% value of the |\thesis@ifwoman| macro and the
% \end{macro}\begin{macro}{\thesis@slovak@typeName@akuzativ}
% |\thesis@slovak@typeName@akuzativ| containing the accusative case
% of the thesis type name.
% \end{macro}\begin{macrocode}

% Pomocná makrá
\gdef\thesis@slovak@gender@koncovka{%
  \ifthesis@woman a\fi}

% Csquotes štýl
\gdef\thesis@slovak@csquotesStyle{german}

% Zástupné texty
\gdef\thesis@slovak@universityName{Názov univerzity}
\gdef\thesis@slovak@facultyName{Názov fakulty}
\gdef\thesis@slovak@assignment{%
  \ifthesis@digital@
    Na tomto mieste sa v~tlačenej práci nachádza oficiálne
    podpísané zadanie práce.
  \else
    Namiesto tejto stránky vložte kópiu oficiálneho podpísaného
    zadania práce.
  \fi}
\gdef\thesis@slovak@declaration{Text vyhlásenie ...}

% Časové údaje
\gdef\thesis@slovak@spring{jar}
\gdef\thesis@slovak@fall{jeseň}
\gdef\thesis@slovak@semester{%
  \thesis@{slovak@\thesis@season} \thesis@seasonYear}
\gdef\thesis@slovak@formattedDate{{%
  \thesis@day.
  \newcount\@month\expandafter\@month\thesis@month\relax
  \ifnum\@month=1%
    januára
  \else\ifnum\@month=2%
    februára
  \else\ifnum\@month=3%
    marca
  \else\ifnum\@month=4%
    apríla
  \else\ifnum\@month=5%
    mája
  \else\ifnum\@month=6%
    júna
  \else\ifnum\@month=7%
    júla
  \else\ifnum\@month=8%
    augusta
  \else\ifnum\@month=9%
    septembra
  \else\ifnum\@month=10%
    októbra
  \else\ifnum\@month=11%
    novembra
  \else\ifnum\@month=12%
    decembra
  \else
    <<neznámy mesiac (\the\@month)>>
  \fi\fi\fi\fi\fi\fi
  \fi\fi\fi\fi\fi\fi
  \thesis@year}}

% Rôzne
\gdef\thesis@slovak@authorSignature{Podpis autora}
\gdef\thesis@slovak@fieldTitle{Odbor}
\gdef\thesis@slovak@advisorTitle{Vedúci práce}
\gdef\thesis@slovak@authorTitle{Autor}
\gdef\thesis@slovak@abstractTitle{Zhrnutie}
\gdef\thesis@slovak@keywordsTitle{Kľúčové slová}
\gdef\thesis@slovak@thanksTitle{Poďakovanie}
\gdef\thesis@slovak@declarationTitle{Vyhlásenie}
\gdef\thesis@slovak@idTitle{ID}
\gdef\thesis@slovak@typeName@sempaper{Seminárna práca}
\gdef\thesis@slovak@typeName@bachelors{Bakalárska práca}
\gdef\thesis@slovak@typeName@masters{Diplomová práca}
\gdef\thesis@slovak@typeName@proposal{Tézy záverečnej práce}
\gdef\thesis@slovak@typeName@doctoral{Dizertačná práca}
\gdef\thesis@slovak@typeName@rigorous{Rigorózna práca}
\gdef\thesis@slovak@typeName{%
  \ifx\thesis@type\thesis@sempaper
    \thesis@slovak@typeName@sempaper
  \else\ifx\thesis@type\thesis@bachelors
    \thesis@slovak@typeName@bachelors
  \else\ifx\thesis@type\thesis@masters
    \thesis@slovak@typeName@masters
  \else\ifx\thesis@type\thesis@proposal
    \thesis@slovak@typeName@proposal
  \else\ifx\thesis@type\thesis@doctoral
    \thesis@slovak@typeName@doctoral
  \else\ifx\thesis@type\thesis@rigorous
    \thesis@slovak@typeName@rigorous
  \else
    <<Neznámy typ práce (\thesis@type)>>%
  \fi\fi\fi\fi\fi\fi}
\gdef\thesis@slovak@typeName@akuzativ@sempaper{Seminárnu prácu}
\gdef\thesis@slovak@typeName@akuzativ@bachelors{Bakalársku prácu}
\gdef\thesis@slovak@typeName@akuzativ@masters{Diplomovú prácu}
\gdef\thesis@slovak@typeName@akuzativ@proposal{Tézy záverečnej práce}
\gdef\thesis@slovak@typeName@akuzativ@doctoral{Dizertačnú prácu}
\gdef\thesis@slovak@typeName@akuzativ@rigorous{Rigoróznu prácu}
\gdef\thesis@slovak@typeName@akuzativ{%
  \ifx\thesis@type\thesis@sempaper
    \thesis@slovak@typeName@akuzativ@sempaper
  \else\ifx\thesis@type\thesis@bachelors
    \thesis@slovak@typeName@akuzativ@bachelors
  \else\ifx\thesis@type\thesis@masters
    \thesis@slovak@typeName@akuzativ@masters
  \else\ifx\thesis@type\thesis@proposal
    \thesis@slovak@typeName@akuzativ@proposal
  \else\ifx\thesis@type\thesis@doctoral
    \thesis@slovak@typeName@akuzativ@doctoral
  \else\ifx\thesis@type\thesis@rigorous
    \thesis@slovak@typeName@akuzativ@rigorous
  \else
    <<Neznámy typ práce (\thesis@type)>>%
  \fi\fi\fi\fi\fi\fi}
%    \end{macrocode}\iffalse
%</base>
% \fi\file{locale/mu/fithesis-slovak.def}
% This is the Slovak locale file specific to the Masaryk
% University in Brno. It replaces the \texttt{universityName}
% placeholder with the correct value and defines the
% \texttt{declaration} and \texttt{idTitle} strings.
% \iffalse
%<*mu>
% \fi\begin{macrocode}
\ProvidesFile{fithesis/locale/mu/fithesis-slovak.def}[2017/06/02]

% Zástupné texty
\gdef\thesis@slovak@universityName{Masarykova univerzita}
\gdef\thesis@slovak@declaration{%
  Vyhlasujem, že som predloženú \thesis@lower{%
  slovak@typeName@akuzativ} vypracoval%
  \thesis@slovak@gender@koncovka\ samostatne len s~použitím
  uvedenej literatúry a prameňov.}

% Bibliografický záznam
\gdef\thesis@slovak@bib@title{Bibliografický záznam}
\gdef\thesis@slovak@bib@pages{str}
%    \end{macrocode}
% \changes{v0.3.46}{2017/06/02}{Lifted the \texttt{bib@author},
%   \texttt{bib@thesisTitle}, and \texttt{bib@advisor} strings from
%   \texttt{locale/mu/sci/*.def} to \texttt{locale/mu/*.def},
%   so that they can be shared with \texttt{locale/mu/econ/*.def}.
%   [VN]}
%    \begin{macrocode}
\global\let\thesis@slovak@bib@author\thesis@slovak@authorTitle
\gdef\thesis@slovak@bib@thesisTitle{Názov práce}
\global\let\thesis@slovak@bib@advisor\thesis@slovak@advisorTitle

% Rôzne
\gdef\thesis@slovak@idTitle{UČO}
%    \end{macrocode}\iffalse
%</mu>
% \fi\file{locale/mu/law/fithesis-slovak.def}
% This is the Slovak locale file specific to the Faculty of Law at
% the Masaryk University in Brno. It replaces the
% \texttt{facultyName} placeholder with the correct value, defines
% the \texttt{facultyLongName} required by the
% |\thesis@blocks@cover| and the |\thesis@blocks@titlePage| blocks
% and replaces the \texttt{abstractTitle} string in accordance
% with the requirements of the faculty.
% \iffalse
%<*mu/law>
% \fi\begin{macrocode}
\ProvidesFile{fithesis/locale/mu/law/fithesis-slovak.def}[2015/06/26]

% Rôzne
\gdef\thesis@slovak@abstractTitle{Abstrakt}

% Zástupné texty
\gdef\thesis@slovak@facultyName{Právnická fakulta}
\gdef\thesis@slovak@facultyLongName{Právnická fakulta Masarykovej
  univerzity}
%    \end{macrocode}\iffalse
%</mu/law>
% \fi\file{locale/mu/fsps/fithesis-slovak.def}
% This is the Slovak locale file specific to the Faculty of Sports
% Studies at the Masaryk University in Brno. It replaces the
% \texttt{facultyName} placeholder with the correct value and
% redefines the \texttt{fieldTitle} string in accordance with the
% common usage at the faculty. The locale file also redefines the
% \texttt{declaration} string in accordance with the requirements
% of the faculty.
% \iffalse
%<*mu/fsps>
% \fi\begin{macrocode}
\ProvidesFile{fithesis/locale/mu/fsps/fithesis-slovak.def}[2017/05/15]

% Zástupné texty
\gdef\thesis@slovak@facultyName{Fakulta športových štúdií}
\gdef\thesis@slovak@declaration{%
  Vyhlasujem, že som \thesis@lower{%
  slovak@typeName@akuzativ} vypracoval%
  \thesis@slovak@gender@koncovka\ samostatne a~na základe
  literatúry a~prameňov uvedených v~použitých zdrojoch.}

% Rôzne
\gdef\thesis@slovak@fieldTitle{Špecializácie}
%    \end{macrocode}\iffalse
%</mu/fsps>
% \fi\file{locale/mu/fss/fithesis-slovak.def}
% This is the Slovak locale file specific to the Faculty of Social
% Studies at the Masaryk University in Brno. It replaces the
% \texttt{facultyName} and \texttt{assignment} strings with the
% correct values.
% \iffalse
%<*mu/fss>
% \fi\begin{macrocode}
\ProvidesFile{fithesis/locale/mu/fss/fithesis-slovak.def}[2016/05/25]

% Zástupné texty
\gdef\thesis@slovak@facultyName{Fakulta sociálnych štúdií}
\gdef\thesis@slovak@assignment{%
  \ifthesis@digital@
    Na tomto mieste sa v~tlačenej práci nachádza oficiálne
    podpísané zadanie práce alebo vyhlásenie autora školského
    diela alebo obidve.
  \else
    Namiesto tejto stránky vložte kópiu oficiálneho podpísaného
    zadania práce alebo vyhlásenie autora školského diela alebo
    obidve v~závislosti na požiadavkách príslušnej katedry.
  \fi}
%    \end{macrocode}\iffalse
%</mu/fss>
% \fi\file{locale/mu/econ/fithesis-slovak.def}
% This is the Slovak locale file specific to the Faculty of
% Economics and Administration at the Masaryk University in Brno.
% It replaces the \texttt{facultyName} and \texttt{abstractTitle}
% placeholders with the correct values. The locale file also
% redefines the \texttt{declaration} string in accordance with
% the requirements of the faculty and defines the private macros
% required by the |\thesis@blocks@|\discretionary{}{}{}|bibEntry|
% block defined within the \texttt{style/mu/fithesis-econ.sty}
% style file.
% \iffalse
%<*mu/econ>
% \fi\begin{macrocode}
\ProvidesFile{fithesis/locale/mu/econ/fithesis-slovak.def}[2017/06/02]

% Zástupné texty
\gdef\thesis@slovak@facultyName{Ekonomicko-správna fakulta}

% Bibliografický záznam
%    \end{macrocode}
% \changes{v0.3.46}{2017/06/02}{Defined strings required by
%   \cs{thesis@blocks@bibEntry} from
%   \texttt{style/mu/fithesis-econ.sty} in
%   \texttt{locale/mu/econ/*.def}. [VN]}
%    \begin{macrocode}
\gdef\thesis@slovak@bib@thesisTitleEn{Názov práce v angličtine}
\gdef\thesis@slovak@bib@department{Katedra}
\gdef\thesis@slovak@bib@year{Rok obhajoby}

% Rôzne
%    \end{macrocode}
% \changes{v0.3.46}{2017/06/02}{Updated the
%   \cs{abstractTitle} string in \texttt{locale/mu/econ/*.def} in
%   accordance with the 2/2017 dean's directive. The patch was
%   submitted by Jana Ratajská. [VN]}
%    \begin{macrocode}
\gdef\thesis@slovak@abstractTitle{Anotácie}
%    \end{macrocode}
% \changes{v0.3.46}{2017/06/02}{Updated the \cs{declaration} string
%   in \texttt{locale/mu/econ/*.def} in accordance with the 2/2017
%   dean's directive. [VN]}
% The following extra data field is defined for the
% \texttt{declaration} string: \begin{itemize}
%   \item|advisorSkGenitiv| -- the advisor's name in
%     genitive following Slovak morphology.
% \end{itemize}
%    \begin{macrocode}
\thesis@def@extra{advisorSkGenitiv}
\gdef\thesis@slovak@declaration{Vyhlasujem, že som
  \thesis@lower{slovak@typeName@akuzativ} spracoval%
  \thesis@slovak@gender@koncovka\ samostatne pod vedením
  \thesis@extra@advisorSkGenitiv\ 
  a~uved\ifthesis@woman la\else iol\fi\ v~nej všetky
  odborné zdroje v~súlade s~právnymi predpismi, vnútornými
  předpismi Masarykovej univerzity a~vnútornými aktmi riadenia
  Masarykovej univerzity a~Ekonomicko-správnej fakulty MU.}
%    \end{macrocode}\iffalse
%</mu/econ>
% \fi\file{locale/mu/med/fithesis-slovak.def}
% This is the Slovak locale file specific to the Faculty of
% Medicine at the Masaryk University in Brno.
% It replaces the \texttt{facultyName} placeholder with the
% correct value and redefines the \texttt{abstractTitle}
% string in accordance with the common usage at the faculty.
% The file also defines the \texttt{bib@title} and
% \texttt{bib@pages} strings required by the
% |\thesis@blocks@bibEntry| block defined within the
% \texttt{style/mu/fithesis-med.sty} style file.

% \iffalse
%<*mu/med>
% \fi\begin{macrocode}
\ProvidesFile{fithesis/locale/mu/med/fithesis-slovak.def}[2016/03/23]

% Zástupné texty
\gdef\thesis@slovak@facultyName{Lekárska fakulta}

% Rôzne
\gdef\thesis@slovak@abstractTitle{Anotácie}
%    \end{macrocode}\iffalse
%</mu/med>
% \fi\file{locale/mu/fi/fithesis-slovak.def}
% This is the Slovak locale file specific to the Faculty of
% Informatics at the Masaryk University in Brno.  It replaces the
% \texttt{facultyName} placeholder with the correct value and
% redefines the \texttt{declaration} string in accordance with the
% requirements of the faculty.  The file also defines the
% \texttt{advisorSignature} string required by the
% |\thesis@blocks@titlePage| block defined within the
% \texttt{style/mu/\discretionary{}{}{}fithesis-fi.sty}
% style file.
% \iffalse
%<*mu/fi>
% \fi\begin{macrocode}
\ProvidesFile{fithesis/locale/mu/fi/fithesis-slovak.def}[2016/05/25]

% Zástupné texty
\gdef\thesis@slovak@facultyName{Fakulta informatiky}
\gdef\thesis@slovak@assignment{%
  \ifthesis@digital@
    Na tomto mieste sa v~tlačenej práci nachádza oficiálne
    podpísané zadanie práce a vyhlásenie autora školského diela.
  \else
    Namiesto tejto stránky vložte kópiu oficiálneho podpísaného
    zadania práce a vyhlásenie autora školského diela.
  \fi}
\gdef\thesis@slovak@declaration{%
  vyhlasujem, že táto \thesis@lower{slovak@typeName} je mojím
  pôvodným autorským dielom, ktoré som vypracoval%
  \thesis@slovak@gender@koncovka\ samostatne. Všetky zdroje,
  pramene a literatúru, ktoré som pri vypracovaní
  používal\thesis@slovak@gender@koncovka\ alebo z~nich
  čerpal\thesis@slovak@gender@koncovka, v~práci riadne citujem
  s~uvedením úplného odkazu na príslušný zdroj.}

% Rôzne
\gdef\thesis@slovak@advisorSignature{Podpis vedúceho}
\gdef\thesis@slovak@typeName@proposal{Tézy dizertačnej práce}
\gdef\thesis@slovak@typeName@akuzativ@proposal{Tézy dizertačnej práce}
%    \end{macrocode}\iffalse
%</mu/fi>
% \fi\file{locale/mu/phil/fithesis-slovak.def}
% This is the Slovak locale file specific to the Faculty of
% Arts at the Masaryk University in Brno.
% It replaces the \texttt{facultyName} placeholder with the
% correct value. It also defines the \texttt{declaration} string
% and redefines the \texttt{typeName} and
% \texttt{typeName@akuzativ} strings in accordance with the
% requirements of the faculty.
% 
% The locale file also defines the \texttt{departmentName}
% string, which is used by the \texttt{style/mu/fithesis-phil^^A
% .sty} style file, when typesetting the names of known
% departments.
% \iffalse
%<*mu/phil>
% \fi\begin{macrocode}
\ProvidesFile{fithesis/locale/mu/phil/fithesis-slovak.def}[2016/03/22]

% Zástupné texty
\gdef\thesis@slovak@facultyName{Filozofická fakulta}
\gdef\thesis@slovak@departmentName{%
  \ifx\thesis@department\thesis@departments@kisk
    Kabinet informačných štúdií a knihovníctva%
  \else
    <<Neznáme oddělenie (\thesis@department)>>%
  \fi}
\gdef\thesis@czech@declaration{%
  \ifx\thesis@department\thesis@departments@kisk
    Vyhlasujem, že som predkladanú prácu spracoval%
    \thesis@slovak@gender@koncovka\ samostatne~a použil%
    \thesis@slovak@gender@koncovka\ len uvedené pramene~a
    literatúru. Súčasne dávam súhlas k~tomu, aby elektronická
    verzia tejto práce bola sprístupnená cez informačný
    systém Masarykovej univerzity.%
  \else
    Vyhlasujem, že som predloženú \thesis@lower{%
    slovak@typeName@akuzativ} vypracoval%
    \thesis@slovak@gender@koncovka\ samostatne na základe vlastných
    zistení a len s~použitím uvedenej literatúry a prameňov.%
  \fi}

% Rôzne
\global\let\thesis@slovak@typeName@super
  \thesis@slovak@typeName
\gdef\thesis@slovak@typeName{%
  \ifx\thesis@type\thesis@bachelors
    Bakalárska diplomová práca%
  \else\ifx\thesis@type\thesis@masters
    Magisterská diplomová práca%
  \else
    \thesis@slovak@typeName@super
  \fi\fi}

\global\let\thesis@slovak@typeName@akuzativ@super
  \thesis@slovak@typeName@akuzativ
\gdef\thesis@slovak@typeName@akuzativ{%
  \ifx\thesis@type\thesis@bachelors
    Diplomovú prácu%
  \else\ifx\thesis@type\thesis@masters
    Diplomovú prácu%
  \else
    \thesis@slovak@typeName@akuzativ@super
  \fi\fi}
%    \end{macrocode}\iffalse
%</mu/phil>
% \fi\file{locale/mu/ped/fithesis-slovak.def}
% This is the Slovak locale file specific to the Faculty of
% Education at the Masaryk University in Brno.
% It replaces the \texttt{facultyName} placeholder with the
% correct value. The file also defines the
% \texttt{bib@title} and \texttt{bib@pages} strings required by the
% |\thesis@blocks@bibEntry| block defined within the
% \texttt{style/mu/\discretionary{}{}{}fithesis-ped.sty}
% style file.
% \iffalse
%<*mu/ped>
% \fi\begin{macrocode}
\ProvidesFile{fithesis/locale/mu/ped/fithesis-slovak.def}[2017/06/02]

% Zástupné texty
\gdef\thesis@slovak@facultyName{Pedagogická fakulta}
%    \end{macrocode}\iffalse
%</mu/ped>
% \fi\file{locale/mu/sci/fithesis-slovak.def}
% This is the Slovak locale file specific to the Faculty of
% Science at the Masaryk University in Brno.
% The locale file also defines the private macros
% required by the |\thesis@blocks@|\discretionary{}{}{}|bibEntry|
% block defined within the \texttt{style/mu/fithesis-sci.sty} style
% file. It also replaces the \texttt{facultyName} placeholder with
% the correct value and redefines the \texttt{abstractTitle} and
% \texttt{declaration} strings in accordance with the formal
% requirements of the faculty.
% \changes{v0.3.45}{2017/05/21}{Added Slovak localization for the
%   \texttt{style/mu/fithesis-sci.sty} style file. The localization
%   was submitted by Juraj Pálenik. [VN]}
% \iffalse
%<*mu/sci>
% \fi\begin{macrocode}
\ProvidesFile{fithesis/locale/mu/sci/fithesis-slovak.def}[2017/05/21]

% Zástupné texty
\gdef\thesis@slovak@facultyName{Prírodovedecká fakulta}

% Bibliografický záznam
\gdef\thesis@slovak@bib@programme{Študijný program}
\global\let\thesis@slovak@bib@field\thesis@slovak@fieldTitle
\gdef\thesis@slovak@bib@academicYear{Akademický rok}
\gdef\thesis@slovak@bib@pages{Počet strán}
\global\let\thesis@slovak@bib@keywords\thesis@slovak@keywordsTitle
%    \end{macrocode}\iffalse
%</mu/sci>
% \fi

%
% \subsection{Style files}
% \label{sec:style-files}
% Style files define the structure and the look of the resulting
% document. They live in the \texttt{style/} subtree and they are
% loaded during the main routine (see Section
% \ref{sec:thesisload}).
%
% When creating a new style file, it is advisable to create one
% self-contained \texttt{dtx} file, which can contain several
% files to be extracted via the \textsf{docstrip} tool based on the
% respective \texttt{ins} file. A \DescribeMacro{\file} macro
% |\file|\marg{filename} is available for the sectioning of the
% documentation of various files within the \texttt{dtx} file.
% For more information about \texttt{dtx} files and the
% \textsf{docstrip} tool, consult the \textsf{dtxtut, docstrip,
% doc} and \textsf{ltxdoc} manuals.
%
% \subsubsection{Interface}
% The union of style files loaded via the style file inheritance
% scheme (see the definition of the |\thesis@requireStyle| macro in
% Section \ref{sec:reflection}) should globally define at least one
% of the following macros:
% \begin{itemize}
%   \item\DescribeMacro{\thesis@blocks@preamble}^^A
%     |\thesis@blocks@preamble| -- If autolayout is enabled, then
%     this macro is expanded at the very beginning of the document.
%   \item\DescribeMacro{\thesis@blocks@postamble}^^A
%     |\thesis@blocks@postamble| -- If autolayout is enabled, then
%     this macro is expanded at the very end of the document.
%   \item\DescribeMacro{\thesis@blocks@mainMatter}^^A
%     |\thesis@blocks@mainMatter| -- If autolayout is enabled, then
%     this macro is expanded at the beginning of the document right
%     after |\thesis@blocks@preamble|. This macro sets the style of
%     the main matter of the thesis.
% \end{itemize}
%
% \subsubsection{Base style files}
% % \iffalse
%<*color>
% \fi\file{theme/mu/beamercolorthemefibeamer-mu.sty}
% This is the base color theme for presentations written at the
% Masaryk University in Brno.
%    \begin{macrocode}
\NeedsTeXFormat{LaTeX2e}
\ProvidesPackage{fibeamer/theme/mu/%
  beamercolorthemefibeamer-mu}[2016/05/06]
%    \end{macrocode}
% \begin{macro}{\darkframes}
% The |darkframes| environment switches the color definitions to
% render the enclosed frames in dark colors. This is a dummy
% definition, which will be overridden by the subsequently loaded
% color theme in the presentation mode.
%    \begin{macrocode}
\newenvironment{darkframes}{}{}
%    \end{macrocode}
% \end{macro}
% The rest of the theme will be ignored outside the presentation
% mode.
%    \begin{macrocode}
\mode<presentation>
%    \end{macrocode}
% The theme loads the following packages, which will be used by the
% subsequently loaded color theme specific to a faculty:
% \begin{itemize}
%   \item\textsf{listings} -- This package is used for code
%     listings. The subsequently loaded color theme will specify
%     source code coloring for the package.
%   \item\textsf{ifthen} -- This package is used to construct
%     compound conditionals.
%   \item\textsf{tikz} -- This package is used to create gradient
%     background for dark slides.
% \end{itemize}
%    \begin{macrocode}
  \RequirePackage{listings}
  \RequirePackage{ifthen}
  \RequirePackage{tikz}
%    \end{macrocode}
% \begin{macro}{\iffibeamer@dark}
% The |\iffibeamer@dark| conditional will be switched on and off by
% the subsequently loaded color theme based on whether or not the
% given frame is being typeset in light or dark colors. This
% information will be used by outer themes to insert the correct
% logo into each frame.
%    \begin{macrocode}
  \newif\iffibeamer@dark\fibeamer@darkfalse
%    \end{macrocode}
% \end{macro}
% A frame that is either title or dark, as specified by the value
% of the |\iffibeamer@dark| conditional, will have a gradient
% background as specified by the |fibeamer@dark@backgroundInner|
% and |fibeamer@light@backgroundOuter| colors that shall be defined
% by the subsequently loaded color theme.
%
% A frame that is neither title nor dark, as specified by the value
% of the |\iffibeamer@dark| conditional, will have a gradient
% background as specified by the |fibeamer@light@backgroundInner|
% and |fibeamer@light@backgroundOuter| colors that shall be defined
% by the subsequently loaded color theme.
%    \begin{macrocode}
  \defbeamertemplate*{background canvas}{fibeamer}{%
    \ifthenelse{%
      \boolean{fibeamer@dark} \OR \c@framenumber=0
    }{%
      \begin{tikzpicture}
        \clip (0,\fibeamer@lengths@clipbottom) rectangle
          (\paperwidth,\fibeamer@lengths@cliptop);
        \path [inner color = fibeamer@dark@backgroundInner,
               outer color = fibeamer@dark@backgroundOuter]
          (0,0) rectangle (\paperwidth,\paperwidth);
      \end{tikzpicture}
    }{%
      \begin{tikzpicture}
        \clip (0,\fibeamer@lengths@clipbottom) rectangle
          (\paperwidth,\fibeamer@lengths@cliptop);
        \path [inner color = fibeamer@light@backgroundInner,
               outer color = fibeamer@light@backgroundOuter]
          (0,0) rectangle (\paperwidth,\paperwidth);
      \end{tikzpicture}
    }}
%    \end{macrocode}
% The |\qed| symbol inserted at the end of proofs will have the
% same color as the rest of the proof.
% \changes{v1.1.1:2}{2016/01/14}{Added proper coloring of
%   \cs{qed} to the themes of the Masaryk University in Brno. [VN]}
%    \begin{macrocode}
  \setbeamercolor{qed symbol}{%
    use=block body,
    fg=block body.fg,
    bg=block body.bg}
%    \end{macrocode}
% The links can be colored by the subsequently loaded color themes.
% \changes{v1.1.4:3}{2016/05/06}{Added proper link coloring for the
%   color themes of the Masaryk University in Brno. [VN]}
%    \begin{macrocode}
  \hypersetup{colorlinks,linkcolor=}
\mode
<all>
%    \end{macrocode}
% \iffalse
%</color>
%<*font>
% \fi\file{theme/mu/beamerfontthemefibeamer-mu.sty}
% This is the base font theme for presentations written at the
% Masaryk University in Brno. The theme has no effect outside the
% presentation mode.
%    \begin{macrocode}
\NeedsTeXFormat{LaTeX2e}
\ProvidesPackage{fibeamer/theme/mu/%
  beamerfontthemefibeamer-mu}[2016/01/12]
\mode<presentation>
  \setbeamerfont{normal text}{size=\normalsize}
  \setbeamerfont{title}{size=\LARGE, series=\bfseries}
  \setbeamerfont{subtitle}{parent=normal text, size=\Large}
  \setbeamerfont{frametitle}{size=\Large}
  \setbeamerfont{framesubtitle}{size=\large, shape=\itshape}
  \setbeamerfont{description item}{series=\bfseries}
  \setbeamerfont{author}{size=\large}
\mode
<all>
%    \end{macrocode}
% \iffalse
%</font>
%<*inner>
% \fi\file{theme/mu/beamerinnerthemefibeamer-mu.sty}
% This is the base inner theme for presentations written at the
% Masaryk University in Brno. The theme has no effect outside the
% presentation mode.
%    \begin{macrocode}
\NeedsTeXFormat{LaTeX2e}
\ProvidesPackage{fibeamer/theme/mu/%
  beamerinnerthemefibeamer-mu}[2016/01/14]
\mode<presentation>
%    \end{macrocode}
% This part of the inner theme defines the design of lists.
%    \begin{macrocode}
\defbeamertemplate*{itemize item}{fibeamer}{$\bullet$}
\defbeamertemplate*{itemize subitem}{fibeamer}{\---}
\defbeamertemplate*{itemize subsubitem}{fibeamer}{\guillemotright}
%    \end{macrocode}
% This part of the inner theme defines the design of bibliography
% items and citations.^^A
% \changes{v1.1.0:8}{2016/01/12}{Added support for colored
%   citations to the themes of the Masaryk University in Brno.
%   [VN]}
%    \begin{macrocode}
\defbeamertemplate*{bibliography item}{fibeamer}{\insertbiblabel}
\AtBeginDocument{%
  \let\fibeamer@oldcite\cite
  \def\cite#1{{%
    \usebeamercolor[fg]{item}%
    \fibeamer@oldcite{#1}}}}
%    \end{macrocode}
% This part of the inner theme defines the design of the table of
% contents.
% \changes{v1.1.0:6}{2016/01/11}{Added support for the
%   \cs{tableofcontents} to the themes of the Masaryk University in
%   Brno. [VN]}
%    \begin{macrocode}
\defbeamertemplate*{section in toc}{fibeamer}{%
  \usebeamercolor[fg]{item}%
    \inserttocsectionnumber.%
  \usebeamercolor[fg]{structure}%
  \kern1.25ex\inserttocsection\par}
\defbeamertemplate*{subsection in toc}{fibeamer}{%
  \hspace\leftmargini
  \usebeamercolor[fg]{item}%
    \inserttocsectionnumber.\inserttocsubsectionnumber%
  \usebeamercolor[fg]{structure}%
  \kern1.25ex\inserttocsubsection\par}
\defbeamertemplate*{subsubsection in toc}{fibeamer}{%
  \hspace\leftmargini
  \hspace\leftmarginii
  \usebeamercolor[fg]{item}%
    \inserttocsectionnumber.\inserttocsubsectionnumber.%
    \inserttocsubsubsectionnumber%
  \usebeamercolor[fg]{structure}%
  \kern1.25ex\inserttocsubsubsection\par}
\mode
<all>
%    \end{macrocode}
% \iffalse
%</inner>
%<*outer>
% \fi\file{theme/mu/beamerouterthemefibeamer-mu.sty}
% This is the base outer theme for presentations written at the
% Masaryk University in Brno. The theme has no effect outside the
% presentation mode.
%    \begin{macrocode}
\NeedsTeXFormat{LaTeX2e}
\ProvidesPackage{fibeamer/theme/mu/%
  beamerouterthemefibeamer-mu}[2016/01/12]
\mode<presentation>
%    \end{macrocode}
% The theme uses the following packages:
% \begin{itemize}
%   \item\textsf{ifthen} -- This package is used to construct
%     compound conditionals.
%   \item\textsf{ifpdf} -- This package is used to check, whether
%     the document is being typeset in DVI mode. If it is, then
%     the |\pdfpagewidth| and |\pdfpageheight| dimensions are
%     defined, so that positioning in TikZ works correctly.
%     \changes{v1.0.1}{2015/10/03}{Added DVI output support. [VN]}
%     ^^A <http://tex.stackexchange.com/a/246631/70941>
%   \item\textsf{tikz} -- This package is used to position the
%     logo and the frame number on a frame.
%   \item\textsf{pgfcore} -- This package is used to draw the
%     dashed line at the title frame.
% \end{itemize}
%    \begin{macrocode}
  \RequirePackage{ifthen}
  \RequirePackage{ifpdf}
  \ifpdf\else
    \@ifundefined{pdfpagewidth}{\newdimen\pdfpagewidth}{}
    \@ifundefined{pdfpageheight}{\newdimen\pdfpageheight}{}
    \pdfpagewidth=\paperwidth
    \pdfpageheight=\paperheight
  \fi
  \RequirePackage{tikz}
  \RequirePackage{pgfcore}
%    \end{macrocode}
% This part of the outer theme defines the geometry of the frames
% along with other dimensions.
% \changes{v1.1.0:4}{2016/01/11}{Length definitions within the
%   themes of the Masaryk University in Brno are no longer based on
%   the dimensions of the (now unused) logo in the upper right
%   corner. [VN]}
%    \begin{macrocode}
  \newlength\fibeamer@lengths@baseunit
  \fibeamer@lengths@baseunit=3.75mm
  % The footer padding
  \newlength\fibeamer@lengths@footerpad
  \setlength\fibeamer@lengths@footerpad{%
    \fibeamer@lengths@baseunit}
  % The side margins
  \newlength\fibeamer@lengths@margin
  \setlength\fibeamer@lengths@margin{%
    3\fibeamer@lengths@baseunit}
  \setbeamersize{
    text margin left=\fibeamer@lengths@margin,
    text margin right=\fibeamer@lengths@margin}
  % The upper margin
  \newlength\fibeamer@lengths@titleline
  \setlength\fibeamer@lengths@titleline{%
    3\fibeamer@lengths@baseunit}
  % The background clipping
  \newlength\fibeamer@lengths@clipbottom
  \setlength\fibeamer@lengths@clipbottom\paperwidth
  \addtolength\fibeamer@lengths@clipbottom{-\paperheight}
  \setlength\fibeamer@lengths@clipbottom{%
    0.5\fibeamer@lengths@clipbottom}
  \newlength\fibeamer@lengths@cliptop
  \setlength\fibeamer@lengths@cliptop\paperwidth
  \addtolength\fibeamer@lengths@cliptop{%
    -\fibeamer@lengths@clipbottom}
%    \end{macrocode}
% \changes{v1.1.0:6}{2016/01/11}{Added the logo dimension
% definitions to the themes of the Masaryk University in Brno.
% [VN]}
%    \begin{macrocode}
  % The logo size
  \newlength\fibeamer@lengths@logowidth
  \setlength\fibeamer@lengths@logowidth{%
    14\fibeamer@lengths@baseunit}
  \newlength\fibeamer@lengths@logoheight
  \setlength\fibeamer@lengths@logoheight{%
    0.4\fibeamer@lengths@logowidth}
%    \end{macrocode}
% The outer theme completely culls the bottom navigation.
%    \begin{macrocode}
  \defbeamertemplate*{navigation symbols}{fibeamer}{}
%    \end{macrocode}
% The outer theme also culls the headline.
% \changes{v1.1.0:1}{2015/11/24}{The faculty logos are no longer
%   displayed on regular slides, as per the new unified design of
%   the Masaryk University in Brno. [VN]}
%    \begin{macrocode}
  \defbeamertemplate*{headline}{fibeamer}{}
%    \end{macrocode}
% The frame title.
%    \begin{macrocode}
  \defbeamertemplate*{frametitle}{fibeamer}{%
    \vskip-1em % Align the text with the top border
    \vskip\fibeamer@lengths@titleline
    \usebeamercolor[fg]{frametitle}%
    \usebeamerfont{frametitle}%
      \insertframetitle\par%
    \usebeamercolor[fg]{framesubtitle}%
    \usebeamerfont{framesubtitle}%
      \insertframesubtitle}
%    \end{macrocode}
% The footline contains the frame number. It is flushed right.
%    \begin{macrocode}
  \defbeamertemplate*{footline}{fibeamer}{%
    \ifnum\c@framenumber=0\else%
      \begin{tikzpicture}[overlay]
        \node[anchor=south east,
          yshift=\fibeamer@lengths@footerpad,
          xshift=-\fibeamer@lengths@footerpad] at
          (current page.south east) {
            \usebeamercolor[fg]{framenumber}%
            \usebeamerfont{framenumber}%
            \insertframenumber/\inserttotalframenumber};
      \end{tikzpicture}
    \fi}
%    \end{macrocode}
% The title frame contains the main logo, the |\title|, the
% |\subtitle|, and the |\author|.
%    \begin{macrocode}
  \defbeamertemplate*{title page}{fibeamer}{%
%    \end{macrocode}
% \changes{v1.1.6}{2016/09/27}{The title frame now properly loads the
%   color theme definitions for dark frames. As a result, the
%   \textsf{hyperref} \textsc{url} color for light frames is no
%   longer used in the title frame. [VN]}
%    \begin{macrocode}
    \begin{darkframes}

    % This is slide 0
    \setcounter{framenumber}{0}

    % Input the university logo
    \begin{tikzpicture}[
      remember picture,
      overlay,
      xshift=0.5\fibeamer@lengths@logowidth,
      yshift=0.5\fibeamer@lengths@logoheight
    ]
      \node at (0,0) {
        \fibeamer@includeLogo[
          width=\fibeamer@lengths@logowidth,
          height=\fibeamer@lengths@logoheight
        ]};
    \end{tikzpicture}

    % Input the title
    \usebeamerfont{title}%
    \usebeamercolor[fg]{title}%
    \begin{minipage}[b][2\baselineskip][b]{\textwidth}%
      \raggedright\inserttitle
    \end{minipage}
    \vskip-.5\baselineskip

    % Input the dashed line
    \begin{pgfpicture}
      \pgfsetlinewidth{2pt}
      \pgfsetroundcap
      \pgfsetdash{{0pt}{4pt}}{0cm}

      \pgfpathmoveto{\pgfpoint{0mm}{0mm}}
      \pgfpathlineto{\pgfpoint{\textwidth}{0mm}}

      \pgfusepath{stroke}
    \end{pgfpicture}
    \vfill
%    \end{macrocode}
% \changes{v1.1.4:4}{2016/05/06}{Added support for subtitle and
%   author name coloring within the color themes of the Masaryk
%   University in Brno. [VN]}
%    \begin{macrocode}
    % Input the subtitle
    \usebeamerfont{subtitle}%
    \usebeamercolor[fg]{subtitle}%
    \begin{minipage}{\textwidth}
      \raggedright%
      \insertsubtitle%
    \end{minipage}\vskip.25\baselineskip

    % Input the author's name
    \usebeamerfont{author}%
    \usebeamercolor[fg]{author}%
    \begin{minipage}{\textwidth}
      \raggedright%
      \insertauthor%
    \end{minipage}
    \end{darkframes}}

\mode
<all>
%    \end{macrocode}
% \iffalse
%</outer>
% \fi

% % \iffalse
%<*color>
% \fi\file{theme/mu/beamercolorthemefibeamer-mu.sty}
% This is the base color theme for presentations written at the
% Masaryk University in Brno.
%    \begin{macrocode}
\NeedsTeXFormat{LaTeX2e}
\ProvidesPackage{fibeamer/theme/mu/%
  beamercolorthemefibeamer-mu}[2016/05/06]
%    \end{macrocode}
% \begin{macro}{\darkframes}
% The |darkframes| environment switches the color definitions to
% render the enclosed frames in dark colors. This is a dummy
% definition, which will be overridden by the subsequently loaded
% color theme in the presentation mode.
%    \begin{macrocode}
\newenvironment{darkframes}{}{}
%    \end{macrocode}
% \end{macro}
% The rest of the theme will be ignored outside the presentation
% mode.
%    \begin{macrocode}
\mode<presentation>
%    \end{macrocode}
% The theme loads the following packages, which will be used by the
% subsequently loaded color theme specific to a faculty:
% \begin{itemize}
%   \item\textsf{listings} -- This package is used for code
%     listings. The subsequently loaded color theme will specify
%     source code coloring for the package.
%   \item\textsf{ifthen} -- This package is used to construct
%     compound conditionals.
%   \item\textsf{tikz} -- This package is used to create gradient
%     background for dark slides.
% \end{itemize}
%    \begin{macrocode}
  \RequirePackage{listings}
  \RequirePackage{ifthen}
  \RequirePackage{tikz}
%    \end{macrocode}
% \begin{macro}{\iffibeamer@dark}
% The |\iffibeamer@dark| conditional will be switched on and off by
% the subsequently loaded color theme based on whether or not the
% given frame is being typeset in light or dark colors. This
% information will be used by outer themes to insert the correct
% logo into each frame.
%    \begin{macrocode}
  \newif\iffibeamer@dark\fibeamer@darkfalse
%    \end{macrocode}
% \end{macro}
% A frame that is either title or dark, as specified by the value
% of the |\iffibeamer@dark| conditional, will have a gradient
% background as specified by the |fibeamer@dark@backgroundInner|
% and |fibeamer@light@backgroundOuter| colors that shall be defined
% by the subsequently loaded color theme.
%
% A frame that is neither title nor dark, as specified by the value
% of the |\iffibeamer@dark| conditional, will have a gradient
% background as specified by the |fibeamer@light@backgroundInner|
% and |fibeamer@light@backgroundOuter| colors that shall be defined
% by the subsequently loaded color theme.
%    \begin{macrocode}
  \defbeamertemplate*{background canvas}{fibeamer}{%
    \ifthenelse{%
      \boolean{fibeamer@dark} \OR \c@framenumber=0
    }{%
      \begin{tikzpicture}
        \clip (0,\fibeamer@lengths@clipbottom) rectangle
          (\paperwidth,\fibeamer@lengths@cliptop);
        \path [inner color = fibeamer@dark@backgroundInner,
               outer color = fibeamer@dark@backgroundOuter]
          (0,0) rectangle (\paperwidth,\paperwidth);
      \end{tikzpicture}
    }{%
      \begin{tikzpicture}
        \clip (0,\fibeamer@lengths@clipbottom) rectangle
          (\paperwidth,\fibeamer@lengths@cliptop);
        \path [inner color = fibeamer@light@backgroundInner,
               outer color = fibeamer@light@backgroundOuter]
          (0,0) rectangle (\paperwidth,\paperwidth);
      \end{tikzpicture}
    }}
%    \end{macrocode}
% The |\qed| symbol inserted at the end of proofs will have the
% same color as the rest of the proof.
% \changes{v1.1.1:2}{2016/01/14}{Added proper coloring of
%   \cs{qed} to the themes of the Masaryk University in Brno. [VN]}
%    \begin{macrocode}
  \setbeamercolor{qed symbol}{%
    use=block body,
    fg=block body.fg,
    bg=block body.bg}
%    \end{macrocode}
% The links can be colored by the subsequently loaded color themes.
% \changes{v1.1.4:3}{2016/05/06}{Added proper link coloring for the
%   color themes of the Masaryk University in Brno. [VN]}
%    \begin{macrocode}
  \hypersetup{colorlinks,linkcolor=}
\mode
<all>
%    \end{macrocode}
% \iffalse
%</color>
%<*font>
% \fi\file{theme/mu/beamerfontthemefibeamer-mu.sty}
% This is the base font theme for presentations written at the
% Masaryk University in Brno. The theme has no effect outside the
% presentation mode.
%    \begin{macrocode}
\NeedsTeXFormat{LaTeX2e}
\ProvidesPackage{fibeamer/theme/mu/%
  beamerfontthemefibeamer-mu}[2016/01/12]
\mode<presentation>
  \setbeamerfont{normal text}{size=\normalsize}
  \setbeamerfont{title}{size=\LARGE, series=\bfseries}
  \setbeamerfont{subtitle}{parent=normal text, size=\Large}
  \setbeamerfont{frametitle}{size=\Large}
  \setbeamerfont{framesubtitle}{size=\large, shape=\itshape}
  \setbeamerfont{description item}{series=\bfseries}
  \setbeamerfont{author}{size=\large}
\mode
<all>
%    \end{macrocode}
% \iffalse
%</font>
%<*inner>
% \fi\file{theme/mu/beamerinnerthemefibeamer-mu.sty}
% This is the base inner theme for presentations written at the
% Masaryk University in Brno. The theme has no effect outside the
% presentation mode.
%    \begin{macrocode}
\NeedsTeXFormat{LaTeX2e}
\ProvidesPackage{fibeamer/theme/mu/%
  beamerinnerthemefibeamer-mu}[2016/01/14]
\mode<presentation>
%    \end{macrocode}
% This part of the inner theme defines the design of lists.
%    \begin{macrocode}
\defbeamertemplate*{itemize item}{fibeamer}{$\bullet$}
\defbeamertemplate*{itemize subitem}{fibeamer}{\---}
\defbeamertemplate*{itemize subsubitem}{fibeamer}{\guillemotright}
%    \end{macrocode}
% This part of the inner theme defines the design of bibliography
% items and citations.^^A
% \changes{v1.1.0:8}{2016/01/12}{Added support for colored
%   citations to the themes of the Masaryk University in Brno.
%   [VN]}
%    \begin{macrocode}
\defbeamertemplate*{bibliography item}{fibeamer}{\insertbiblabel}
\AtBeginDocument{%
  \let\fibeamer@oldcite\cite
  \def\cite#1{{%
    \usebeamercolor[fg]{item}%
    \fibeamer@oldcite{#1}}}}
%    \end{macrocode}
% This part of the inner theme defines the design of the table of
% contents.
% \changes{v1.1.0:6}{2016/01/11}{Added support for the
%   \cs{tableofcontents} to the themes of the Masaryk University in
%   Brno. [VN]}
%    \begin{macrocode}
\defbeamertemplate*{section in toc}{fibeamer}{%
  \usebeamercolor[fg]{item}%
    \inserttocsectionnumber.%
  \usebeamercolor[fg]{structure}%
  \kern1.25ex\inserttocsection\par}
\defbeamertemplate*{subsection in toc}{fibeamer}{%
  \hspace\leftmargini
  \usebeamercolor[fg]{item}%
    \inserttocsectionnumber.\inserttocsubsectionnumber%
  \usebeamercolor[fg]{structure}%
  \kern1.25ex\inserttocsubsection\par}
\defbeamertemplate*{subsubsection in toc}{fibeamer}{%
  \hspace\leftmargini
  \hspace\leftmarginii
  \usebeamercolor[fg]{item}%
    \inserttocsectionnumber.\inserttocsubsectionnumber.%
    \inserttocsubsubsectionnumber%
  \usebeamercolor[fg]{structure}%
  \kern1.25ex\inserttocsubsubsection\par}
\mode
<all>
%    \end{macrocode}
% \iffalse
%</inner>
%<*outer>
% \fi\file{theme/mu/beamerouterthemefibeamer-mu.sty}
% This is the base outer theme for presentations written at the
% Masaryk University in Brno. The theme has no effect outside the
% presentation mode.
%    \begin{macrocode}
\NeedsTeXFormat{LaTeX2e}
\ProvidesPackage{fibeamer/theme/mu/%
  beamerouterthemefibeamer-mu}[2016/01/12]
\mode<presentation>
%    \end{macrocode}
% The theme uses the following packages:
% \begin{itemize}
%   \item\textsf{ifthen} -- This package is used to construct
%     compound conditionals.
%   \item\textsf{ifpdf} -- This package is used to check, whether
%     the document is being typeset in DVI mode. If it is, then
%     the |\pdfpagewidth| and |\pdfpageheight| dimensions are
%     defined, so that positioning in TikZ works correctly.
%     \changes{v1.0.1}{2015/10/03}{Added DVI output support. [VN]}
%     ^^A <http://tex.stackexchange.com/a/246631/70941>
%   \item\textsf{tikz} -- This package is used to position the
%     logo and the frame number on a frame.
%   \item\textsf{pgfcore} -- This package is used to draw the
%     dashed line at the title frame.
% \end{itemize}
%    \begin{macrocode}
  \RequirePackage{ifthen}
  \RequirePackage{ifpdf}
  \ifpdf\else
    \@ifundefined{pdfpagewidth}{\newdimen\pdfpagewidth}{}
    \@ifundefined{pdfpageheight}{\newdimen\pdfpageheight}{}
    \pdfpagewidth=\paperwidth
    \pdfpageheight=\paperheight
  \fi
  \RequirePackage{tikz}
  \RequirePackage{pgfcore}
%    \end{macrocode}
% This part of the outer theme defines the geometry of the frames
% along with other dimensions.
% \changes{v1.1.0:4}{2016/01/11}{Length definitions within the
%   themes of the Masaryk University in Brno are no longer based on
%   the dimensions of the (now unused) logo in the upper right
%   corner. [VN]}
%    \begin{macrocode}
  \newlength\fibeamer@lengths@baseunit
  \fibeamer@lengths@baseunit=3.75mm
  % The footer padding
  \newlength\fibeamer@lengths@footerpad
  \setlength\fibeamer@lengths@footerpad{%
    \fibeamer@lengths@baseunit}
  % The side margins
  \newlength\fibeamer@lengths@margin
  \setlength\fibeamer@lengths@margin{%
    3\fibeamer@lengths@baseunit}
  \setbeamersize{
    text margin left=\fibeamer@lengths@margin,
    text margin right=\fibeamer@lengths@margin}
  % The upper margin
  \newlength\fibeamer@lengths@titleline
  \setlength\fibeamer@lengths@titleline{%
    3\fibeamer@lengths@baseunit}
  % The background clipping
  \newlength\fibeamer@lengths@clipbottom
  \setlength\fibeamer@lengths@clipbottom\paperwidth
  \addtolength\fibeamer@lengths@clipbottom{-\paperheight}
  \setlength\fibeamer@lengths@clipbottom{%
    0.5\fibeamer@lengths@clipbottom}
  \newlength\fibeamer@lengths@cliptop
  \setlength\fibeamer@lengths@cliptop\paperwidth
  \addtolength\fibeamer@lengths@cliptop{%
    -\fibeamer@lengths@clipbottom}
%    \end{macrocode}
% \changes{v1.1.0:6}{2016/01/11}{Added the logo dimension
% definitions to the themes of the Masaryk University in Brno.
% [VN]}
%    \begin{macrocode}
  % The logo size
  \newlength\fibeamer@lengths@logowidth
  \setlength\fibeamer@lengths@logowidth{%
    14\fibeamer@lengths@baseunit}
  \newlength\fibeamer@lengths@logoheight
  \setlength\fibeamer@lengths@logoheight{%
    0.4\fibeamer@lengths@logowidth}
%    \end{macrocode}
% The outer theme completely culls the bottom navigation.
%    \begin{macrocode}
  \defbeamertemplate*{navigation symbols}{fibeamer}{}
%    \end{macrocode}
% The outer theme also culls the headline.
% \changes{v1.1.0:1}{2015/11/24}{The faculty logos are no longer
%   displayed on regular slides, as per the new unified design of
%   the Masaryk University in Brno. [VN]}
%    \begin{macrocode}
  \defbeamertemplate*{headline}{fibeamer}{}
%    \end{macrocode}
% The frame title.
%    \begin{macrocode}
  \defbeamertemplate*{frametitle}{fibeamer}{%
    \vskip-1em % Align the text with the top border
    \vskip\fibeamer@lengths@titleline
    \usebeamercolor[fg]{frametitle}%
    \usebeamerfont{frametitle}%
      \insertframetitle\par%
    \usebeamercolor[fg]{framesubtitle}%
    \usebeamerfont{framesubtitle}%
      \insertframesubtitle}
%    \end{macrocode}
% The footline contains the frame number. It is flushed right.
%    \begin{macrocode}
  \defbeamertemplate*{footline}{fibeamer}{%
    \ifnum\c@framenumber=0\else%
      \begin{tikzpicture}[overlay]
        \node[anchor=south east,
          yshift=\fibeamer@lengths@footerpad,
          xshift=-\fibeamer@lengths@footerpad] at
          (current page.south east) {
            \usebeamercolor[fg]{framenumber}%
            \usebeamerfont{framenumber}%
            \insertframenumber/\inserttotalframenumber};
      \end{tikzpicture}
    \fi}
%    \end{macrocode}
% The title frame contains the main logo, the |\title|, the
% |\subtitle|, and the |\author|.
%    \begin{macrocode}
  \defbeamertemplate*{title page}{fibeamer}{%
%    \end{macrocode}
% \changes{v1.1.6}{2016/09/27}{The title frame now properly loads the
%   color theme definitions for dark frames. As a result, the
%   \textsf{hyperref} \textsc{url} color for light frames is no
%   longer used in the title frame. [VN]}
%    \begin{macrocode}
    \begin{darkframes}

    % This is slide 0
    \setcounter{framenumber}{0}

    % Input the university logo
    \begin{tikzpicture}[
      remember picture,
      overlay,
      xshift=0.5\fibeamer@lengths@logowidth,
      yshift=0.5\fibeamer@lengths@logoheight
    ]
      \node at (0,0) {
        \fibeamer@includeLogo[
          width=\fibeamer@lengths@logowidth,
          height=\fibeamer@lengths@logoheight
        ]};
    \end{tikzpicture}

    % Input the title
    \usebeamerfont{title}%
    \usebeamercolor[fg]{title}%
    \begin{minipage}[b][2\baselineskip][b]{\textwidth}%
      \raggedright\inserttitle
    \end{minipage}
    \vskip-.5\baselineskip

    % Input the dashed line
    \begin{pgfpicture}
      \pgfsetlinewidth{2pt}
      \pgfsetroundcap
      \pgfsetdash{{0pt}{4pt}}{0cm}

      \pgfpathmoveto{\pgfpoint{0mm}{0mm}}
      \pgfpathlineto{\pgfpoint{\textwidth}{0mm}}

      \pgfusepath{stroke}
    \end{pgfpicture}
    \vfill
%    \end{macrocode}
% \changes{v1.1.4:4}{2016/05/06}{Added support for subtitle and
%   author name coloring within the color themes of the Masaryk
%   University in Brno. [VN]}
%    \begin{macrocode}
    % Input the subtitle
    \usebeamerfont{subtitle}%
    \usebeamercolor[fg]{subtitle}%
    \begin{minipage}{\textwidth}
      \raggedright%
      \insertsubtitle%
    \end{minipage}\vskip.25\baselineskip

    % Input the author's name
    \usebeamerfont{author}%
    \usebeamercolor[fg]{author}%
    \begin{minipage}{\textwidth}
      \raggedright%
      \insertauthor%
    \end{minipage}
    \end{darkframes}}

\mode
<all>
%    \end{macrocode}
% \iffalse
%</outer>
% \fi

% \subsubsection{The style files of the Faculty of Informatics}
% % \file{style/mu/fithesis-fi.sty}
% This is the style file for the theses written at the Faculty of
% Informatics at the Masaryk University in Brno. It has been
% prepared in accordance with the formal requirements published at
% the website of the faculty\footnote{See
% \url{http://www.fi.muni.cz/docs/BP_DP_na_FI.pdf}}.
%    \begin{macrocode}
\NeedsTeXFormat{LaTeX2e}
\ProvidesPackage{fithesis/style/mu/fithesis-fi}[2016/04/18]
%    \end{macrocode}
% The file defines the color scheme of the respective faculty. Note
% the the color definitions are in RGB, which makes the resulting
% files generally unsuitable for printing.
%    \begin{macrocode}
\thesis@color@setup{
  links={HTML}{FFD451},
  tableEmph={HTML}{FFD451},
  tableOdd={HTML}{FFF9E5},
  tableEven={HTML}{FFECB3}}
%    \end{macrocode}
% The bibliography support is enabled. The |numeric| citations are
% used and the bibliography is sorted in citation order.
%    \begin{macrocode}
\thesis@bibliography@setup{
  style=iso-numeric,
  sorting=none}
\thesis@bibliography@load
%    \end{macrocode}
% In case of rigorous and doctoral theses, the style file hides the
% thesis assignment in accordance with the formal requirements of
% the faculty.
% \begin{macrocode}
\ifx\thesis@type\thesis@rigorous
  \thesis@blocks@assignment@false
\else\ifx\thesis@type\thesis@doctoral
  \thesis@blocks@assignment@false
\fi\fi
%    \end{macrocode}
% \begin{macro}{\thesis@blocks@titlePage}
% The style file redefines the cover and title page footers to
% include the thesis advisor's name and signature in case of a
% rigorous thesis. Along with the macros required by the
% locale file interface, the locale files need to define the
% following strings:
% \begin{itemize}
%   \item\texttt{advisorSignature} -- The label of the advisor
%     signature field typeset in the case of rigorous theses
% \end{itemize}
% \begin{macrocode}
\def\thesis@blocks@advisor{%
  {\thesis@titlePage@large\\[0.3in]
    {\bf\thesis@@{advisorTitle}:} \thesis@advisor}}
\def\thesis@blocks@titlePage@content{%
    {\thesis@titlePage@Huge\bf\thesis@TeXtitle\par\vfil}\vskip 0.8in
    {\thesis@titlePage@large\sc\thesis@@{typeName}\\[0.3in]}
    {\thesis@titlePage@Large\bf\thesis@author}
    % If this is a rigorous thesis or a PhD thesis proposal,
    % typeset the name of the thesis advisor.
    \ifx\thesis@type\thesis@rigorous
      \thesis@blocks@advisor
    \else\ifx\thesis@type\thesis@proposal
      \thesis@blocks@advisor
    \fi\fi}%
\def\thesis@blocks@advisorSignature{%
  \let\@A\relax\newlength{\@A}
    \settowidth{\@A}{\thesis@@{advisorSignature}}
    \setlength{\@A}{\@A+1cm}
  \hfill\raisebox{-0.5em}{\parbox{\@A}{
    \centering
    \rule{\@A}{1pt}\\
    \thesis@@{advisorSignature}
  }}}%
\def\thesis@blocks@titlePage@footer{%
  {\thesis@titlePage@large\thesis@place, \thesis@@{semester}
  % If this is a rigorous thesis or a PhD thesis proposal,
  % create space for the advisor's signature.
  \ifx\thesis@type\thesis@rigorous
    \thesis@blocks@advisorSignature
  \else\ifx\thesis@type\thesis@proposal
    \thesis@blocks@advisorSignature
  \fi\fi}}
%    \end{macrocode}
% \end{macro}\begin{macro}{\thesis@blocks@declaration}
% The |\thesis@blocks@declaration| macro typesets the
% declaration text. Compared to the definition within the
% \texttt{style/mu/base.sty} file, this macro also
% typesets the advisor's name at the bottom of the page.
% \begin{macrocode}
\def\thesis@blocks@declaration{%
  \thesis@blocks@clear
  \begin{alwayssingle}%
    \chapter*{\thesis@@{declarationTitle}}%
    \thesis@declaration
    \vskip 2cm%
    \hfill\thesis@author
    \par\vfill\noindent
    \textbf{\thesis@@{advisorTitle}:} \thesis@advisor
    \par\vfil
  \end{alwayssingle}}
%    \end{macrocode}
% \end{macro}
% In Ph.D. theses, only the table of contents will be typeset in
% the front matter as per the formal requirements of the
% faculty\footnote{See
% \url{http://is.muni.cz/www/2575/dtedi/index_en.html}}.
%
% Note that there is no direct support for the seminar paper type.
% If you would like to change the contents of the preamble and the
% postamble, you should modify the |\thesis@blocks@preamble| and
% |\thesis@blocks@postamble| macros.
%
% All blocks within the autolayout preamble and postamble that are
% not defined within this file are defined in the
% \texttt{style/mu/fithesis-base.sty} file.
%    \begin{macrocode}
\def\thesis@blocks@preamble{%
  \thesis@blocks@coverMatter
    \thesis@blocks@cover
    \thesis@blocks@titlePage
  \thesis@blocks@frontMatter
    \ifx\thesis@type\thesis@proposal
      \thesis@blocks@toc
    \else
      \thesis@blocks@assignment
      \thesis@blocks@declaration
      \thesis@blocks@thanks
      \thesis@blocks@clearRight
        \thesis@blocks@abstract
        \thesis@blocks@keywords
      \thesis@blocks@tables
    \fi}
\def\thesis@blocks@postamble{%
  \thesis@blocks@bibliography}
%    \end{macrocode}

% \subsubsection{The style files of the Faculty of Science}
% % \file{style/mu/fithesis-sci.sty}
% This is the style file for the theses written at the Faculty of
% Science at the Masaryk University in Brno. It has been
% prepared in accordance with the formal requirements published at
% the website of the faculty\footnote{See
% \url{http://www.sci.muni.cz/NW/predpisy/od/OD-2014-05.pdf}}.
%    \begin{macrocode}
\NeedsTeXFormat{LaTeX2e}
\ProvidesPackage{fithesis/style/mu/fithesis-sci}[2017/06/02]
%    \end{macrocode}
% The file defines the color scheme of the respective faculty. Note
% the the color definitions are in RGB, which makes the resulting
% files generally unsuitable for printing.
%    \begin{macrocode}
\thesis@color@setup{
  links={HTML}{20E366},
  tableEmph={HTML}{8EDEAA},
  tableOdd={HTML}{EDF7F1},
  tableEven={HTML}{CCEDD8}}
%    \end{macrocode}
% The bibliography support is enabled. The |numeric| citations are
% used and the bibliography is sorted in citation order.
%    \begin{macrocode}
\thesis@bibliography@setup{
  style=iso-numeric,
  sorting=none}
\thesis@bibliography@load
%    \end{macrocode}
% The file uses Czech locale strings within the macros.
%    \begin{macrocode}
\thesis@requireLocale{czech}
%    \end{macrocode}
% \begin{macro}{\ifthesis@czech}
% The |\ifthesis@czech| \ldots |\else| \ldots |\fi| conditional is made
% available for testing, whether or not the current locale is Czech.
% \changes{v0.3.45}{2017/05/23}{Defined the
%   \cs{ifthesis@czech} macro in
%   \texttt{style/mu/fithesis-sci.sty}. The patch was submitted by
%   Juraj Pálenik. [VN]}
%    \begin{macrocode}
\def\ifthesis@czech{
  \expandafter\def\expandafter\@czech\expandafter{\string
  \czech}%
  \expandafter\expandafter\expandafter\def\expandafter
  \expandafter\expandafter\@locale\expandafter\expandafter
  \expandafter{\expandafter\string\csname\thesis@locale\endcsname}%
  \expandafter\csname\expandafter i\expandafter f\ifx\@locale
  \@czech
    true%
  \else
    false%
  \fi\endcsname}
\ifthesis@czech
  \expandafter\expandafter\expandafter\let\expandafter\expandafter
    \csname ifthesis@czech\endcsname\csname iftrue\endcsname
\else
  \expandafter\expandafter\expandafter\let\expandafter\expandafter
    \csname ifthesis@czech\endcsname\csname iffalse\endcsname
\fi
%    \end{macrocode}
% The file recognizes the following options: \begin{itemize}
%   \item\texttt{abstractonsinglepage} -- The abstracts are going
%   to be typeset on a single page as opposed to being spread
%   across several pages. This is the default for theses whose main
%   locale is neither Czech nor English.
% \end{itemize}
% \changes{v0.3.45}{2017/05/24}{Defined the
%   \texttt{abstractonsinglepage} option in
%   \texttt{style/mu/fithesis-sci.sty}. The patch was submitted by
%   Juraj Pálenik. [VN]}
%    \begin{macrocode}
\newif\ifthesis@abstractonsinglepage@
\DeclareOption{abstractonsinglepage}{\thesis@abstractonsinglepage@true}
\ifthesis@czech\else\ifthesis@english\else
  \ExecuteOptions{abstractonsinglepage}
\fi\fi
\ProcessOptions*
%    \end{macrocode}
% \end{macro}
% The file loads the following packages:
% \begin{itemize}
%   \item\textsf{tikz} -- Used for dimension arithmetic.
%   \item\textsf{changepage} -- Used for width adjustments.
% \end{itemize}
%    \begin{macrocode}
\thesis@require{tikz}
\thesis@require{changepage}
%    \end{macrocode}
% In case of rigorous and doctoral theses, the style file hides the
% thesis assignment in accordance with the formal requirements of
% the faculty.
% \begin{macrocode}
\ifx\thesis@type\thesis@bachelors\else
\ifx\thesis@type\thesis@masters\else
  \thesis@blocks@assignment@false
\fi\fi
%    \end{macrocode}
% Enable the inclusion of the scanned assignment inside the digital
% version of the document.
% \begin{macrocode}
\thesis@blocks@assignment@hideIfDigital@false
%    \end{macrocode}
% \begin{macro}{\thesis@blocks@bibEntry}
% The |\thesis@blocks@bibEntry| macro typesets a bibliographical
% entry. Along with the macros required by the locale file
% interface, the locale files need to define the following macros:
% \begin{itemize}
%   \item|\thesis@|\textit{locale}|@bib@title| -- The title of the
%     entire block
%   \item|\thesis@|\textit{locale}|@bib@author| -- The label of the
%     author name entry
%   \item|\thesis@|\textit{locale}|@bib@title| -- The label of the
%     title name entry
%   \item|\thesis@|\textit{locale}|@bib@programme| -- The label of
%     the programme name entry
%   \item|\thesis@|\textit{locale}|@bib@field| -- The label of the
%     field of study name entry
%   \item|\thesis@|\textit{locale}|@bib@advisor| -- The label of
%     the advisor name entry
%   \item|\thesis@|\textit{locale}|@bib@academicYear| -- The label
%     of the academic year entry
%   \item|\thesis@|\textit{locale}|@bib@pages| -- The label of the
%     number of pages entry
%   \item|\thesis@|\textit{locale}|@bib@keywords| -- The label of
%     the keywords entry
% \end{itemize}
% \changes{v0.3.45}{2017/05/26}{Bibliographical entries in
%   \texttt{style/mu/fithesis-sci.sty} now face each other when the
%   main locale is either Czech or English. [VN]}
%    \begin{macrocode}
\def\thesis@blocks@bibEntry{%
  \begin{alwayssingle}%
    % Clear only the right page, if the main locale is Czech.
    \ifthesis@czech
      \begingroup
      \let\thesis@blocks@clear\thesis@blocks@clearRight
    \fi
    \chapter*{\thesis@@{bib@title}}%
    \ifthesis@czech
      \endgroup
    \fi
    {% Calculate the width of the columns
    \let\@A\relax\newlength{\@A}\settowidth{\@A}{{%
      \bf\thesis@@{bib@author}:}}
    \let\@B\relax\newlength{\@B}\settowidth{\@B}{{%
      \bf\thesis@@{bib@thesisTitle}:}}
    \let\@C\relax\newlength{\@C}\settowidth{\@C}{{%
      \bf\thesis@@{bib@programme}:}}
    \let\@D\relax\newlength{\@D}\settowidth{\@D}{{%
      \bf\thesis@@{bib@field}:}}
    % Unless this is a rigorous thesis, we will be typesetting the
    % name of the thesis advisor.
    \let\@E\relax\newlength{\@E}
      \ifx\thesis@type\thesis@rigorous
        \setlength{\@E}{0pt}%
      \else
        \settowidth{\@E}{{\bf\thesis@@{bib@advisor}:}}
      \fi
    \let\@F\relax\newlength{\@F}\settowidth{\@F}{{%
      \bf\thesis@@{bib@academicYear}:}}
    \let\@G\relax\newlength{\@G}\settowidth{\@G}{{%
      \bf\thesis@@{bib@pages}:}}
    \let\@H\relax\newlength{\@H}\settowidth{\@H}{{%
      \bf\thesis@@{bib@keywords}:}}
    \let\@skip\relax\newlength{\@skip}\setlength{\@skip}{16pt}
    \let\@left\relax\newlength{\@left}\pgfmathsetlength{\@left}{%
      max(\@A,\@B,\@C,\@D,\@E,\@F,\@G,\@H)}
    \let\@right\relax\newlength{\@right}\setlength{\@right}{%
      \textwidth-\@left-\@skip}
    % Typeset the table
    \renewcommand{\arraystretch}{2}
    \noindent\begin{thesis@newtable@old}%
      {@{}p{\@left}@{\hskip\@skip}p{\@right}@{}}
      \textbf{\thesis@@{bib@author}:} &
        \noindent\parbox[t]{\@right}{
          \thesis@author\\
          \thesis@@{facultyName},
          \thesis@@{universityName}\\
          \thesis@department@name
        }\\
      \textbf{\thesis@@{bib@thesisTitle}:}
        & \thesis@title \\
      \textbf{\thesis@@{bib@programme}:}
        & \thesis@programme \\
      \textbf{\thesis@@{bib@field}:}
        & \thesis@field@name \\
      % Unless this is a rigorous thesis, typeset the name of the
      % thesis advisor.
      \ifx\thesis@type\thesis@rigorous\else
        \textbf{\thesis@@{bib@advisor}:}
          & \thesis@advisor \\
      \fi
      \textbf{\thesis@@{bib@academicYear}:}
        & \thesis@academicYear \\
      \textbf{\thesis@@{bib@pages}:}
        & \thesis@pages@preamble{} + \thesis@pages \\
      \textbf{\thesis@@{bib@keywords}:}
        & \thesis@TeXkeywords \\
    \end{thesis@newtable@old}}
  \end{alwayssingle}}
%    \end{macrocode}
% \end{macro}\begin{macro}{\thesis@blocks@bibEntryEn}
% The |\thesis@blocks@bibEntryEn| macro typesets a bibliographical
% entry in English unless the current locale is English.
%    \begin{macrocode}
\def\thesis@blocks@bibEntryEn{%
  \ifthesis@english\else
    {\thesis@selectLocale{english}
    \begin{alwayssingle}
      \chapter*{\thesis@english@bib@title}%
      {% Calculate the width of the columns
      \let\@A\relax\newlength{\@A}\settowidth{\@A}{{%
        \bf\thesis@english@bib@author:}}
      \let\@B\relax\newlength{\@B}\settowidth{\@B}{{%
        \bf\thesis@english@bib@thesisTitle:}}
      \let\@C\relax\newlength{\@C}\settowidth{\@C}{{%
        \bf\thesis@english@bib@programme:}}
      \let\@D\relax\newlength{\@D}\settowidth{\@D}{{%
        \bf\thesis@english@bib@field:}}
      % Unless this is a rigorous thesis, we will be typesetting
      % the name of the thesis advisor.
      \let\@E\relax\newlength{\@E}
        \ifx\thesis@type\thesis@rigorous
          \setlength{\@E}{0pt}%
        \else
          \settowidth{\@E}{{\bf\thesis@english@bib@advisor:}}
        \fi
      \let\@F\relax\newlength{\@F}\settowidth{\@F}{{%
        \bf\thesis@english@bib@academicYear:}}
      \let\@G\relax\newlength{\@G}\settowidth{\@G}{{%
        \bf\thesis@english@bib@pages:}}
      \let\@H\relax\newlength{\@H}\settowidth{\@H}{{%
        \bf\thesis@english@bib@keywords:}}
      \let\@skip\relax\newlength{\@skip}\setlength{\@skip}{16pt}
      \let\@left\relax\newlength{\@left}\pgfmathsetlength{\@left}{%
        max(\@A,\@B,\@C,\@D,\@E,\@F,\@G,\@H)}
      \let\@right\relax\newlength{\@right}\setlength{\@right}{%
        \textwidth-\@left-\@skip}
      % Typeset the table
      \renewcommand{\arraystretch}{2}
      \noindent\begin{thesis@newtable@old}%
        {@{}p{\@left}@{\hskip\@skip}p{\@right}@{}}
          \textbf{\thesis@english@bib@author:} &
            \noindent\parbox[t]{\@right}{
              \thesis@author\\
              \thesis@english@facultyName,
              \thesis@english@universityName\\
              \thesis@departmentEn@name
            }\\
          \textbf{\thesis@english@bib@thesisTitle:}
            & \thesis@titleEn \\
          \textbf{\thesis@english@bib@programme:}
            & \thesis@programmeEn \\
          \textbf{\thesis@english@bib@field:}
            & \thesis@fieldEn@name \\
          % Unless this is a rigorous thesis, typeset the name of the
          % thesis advisor.
          \ifx\thesis@type\thesis@rigorous\else
            \textbf{\thesis@english@bib@advisor:}
              & \thesis@advisor \\
          \fi
          \textbf{\thesis@english@bib@academicYear:}
            & \thesis@academicYear \\
          \textbf{\thesis@english@bib@pages:}
            & \thesis@pages@preamble{} + \thesis@pages \\
          \textbf{\thesis@english@bib@keywords:}
            & \thesis@TeXkeywordsEn \\
        \end{thesis@newtable@old}}
      \end{alwayssingle}
    }%
  \fi}
%    \end{macrocode}
% \end{macro}\begin{macro}{\thesis@blocks@abstractCs}
% The |\thesis@blocks@abstractCs| macro typesets the
% abstract in Czech. If the current locale is Czech, the
% macro produces no output. The following extra data field is
% defined for the macro: \begin{itemize}
%   \item|abstractCs| -- the Czech title of the thesis used for the
%     typesetting. This extra data field will expand to
%     |\thesis@abstract| if the current locale of the thesis
%     is Czech.
% \end{itemize}
% \changes{v0.3.45}{2017/05/28}{Defined the
%   \cs{thesis@blocks@abstractCs} macro in
%   \texttt{style/mu/fithesis-sci.sty}. The patch was submitted by
%   Juraj Pálenik. [VN]}
%    \begin{macrocode}
\thesis@def@extra[{
  \ifthesis@czech
    \thesis@abstract
  \else
    \thesis@placeholder@extra@abstractCs
  \fi
}]{abstractCs}
\def\thesis@blocks@abstractCs{%
  \ifthesis@czech\else
    {\thesis@selectLocale{czech}%
    \begin{alwayssingle}%
      \ifthesis@abstractonsinglepage@
        \thesis@blocks@clear
      \else
        % Start the new chapter without clearing the left page.
        \thesis@blocks@clearRight
      \fi
      {\let\thesis@blocks@clear\relax
      \chapter*{\thesis@czech@abstractTitle}%
      \thesis@extra@abstractCs}%
      \par\vfil\null
    \end{alwayssingle}}%
  \fi}
%    \end{macrocode}
% \end{macro}\begin{macro}{\thesis@blocks@bibEntryCs}
% The |\thesis@blocks@bibEntryCs| macro typesets a bibliographical
% entry in English unless the current locale is Czech. The
% macro uses the following extra data fields:\begin{itemize}
%   \item|programmeCs| -- the Czech name of the author's study
%     programme. This extra data field will expand to
%     |\thesis@programme| if the current locale of the thesis
%     is Czech.
%   \item|fieldCs| -- the Czech name of the author's field of
%     study. This extra data field will expand to
%     |\thesis@field@name| if the current locale of the thesis
%     is Czech.
%   \item|keywordsCs| -- the Czech keywords of the thesis.
%     This extra data field will expand to |\thesis@keywords| if
%     the current locale of the thesis is Czech.
%   \item|TeXkeywordsCs| -- the Czech \TeX{} keywords of the thesis.
%     This extra data field will expand to |\thesis@TeXkeywords| if
%     the current locale of the thesis is Czech.
% \end{itemize}
% \changes{v0.3.45}{2017/05/21}{Defined the
%   \cs{thesis@blocks@bibEntryCs} macro in
%   \texttt{style/mu/fithesis-sci.sty}. The patch was submitted by
%   Juraj Pálenik. [VN]}
%    \begin{macrocode}
\thesis@def@extra[{
  \ifthesis@czech
    \thesis@programme
  \else
    \thesis@placeholder@extra@programmeCs
  \fi
}]{programmeCs}
\thesis@def@extra[{
  \ifthesis@czech
    \thesis@field@name
  \else
    \thesis@placeholder@extra@fieldCs
  \fi
}]{fieldCs}
\thesis@def@extra[{
  \ifthesis@czech
    \thesis@title
  \else
    \thesis@placeholder@extra@titleCs
  \fi
}]{titleCs}
\thesis@def@extra[{
  \ifthesis@czech
    \thesis@keywords
  \else
    \thesis@placeholder@extra@keywordsCs
  \fi
}]{keywordsCs}
\thesis@def@extra[{
  \ifthesis@czech
    \thesis@TeXkeywords
  \else
    \thesis@placeholder@extra@keywordsCs
  \fi
}]{TeXkeywordsCs}
%    \end{macrocode}
% \changes{v0.3.45}{2017/05/26}{Bibliographical entries in
%   \texttt{style/mu/fithesis-sci.sty} now face each other when the
%   main locale is either Czech or English. [VN]}
%    \begin{macrocode}
\def\thesis@blocks@bibEntryCs{%
  \ifthesis@czech\else
    {\thesis@selectLocale{czech}
    \begin{alwayssingle}
      % Clear only the right page, if the main locale is English.
      \ifthesis@english
        \begingroup
        \let\thesis@blocks@clear\thesis@blocks@clearRight
      \fi
      \chapter*{\thesis@czech@bib@title}%
      \ifthesis@english
        \endgroup
      \fi
      {% Calculate the width of the columns
      \let\@A\relax\newlength{\@A}\settowidth{\@A}{{%
        \bf\thesis@czech@bib@author:}}
      \let\@B\relax\newlength{\@B}\settowidth{\@B}{{%
        \bf\thesis@czech@bib@thesisTitle:}}
      \let\@C\relax\newlength{\@C}\settowidth{\@C}{{%
        \bf\thesis@czech@bib@programme:}}
      \let\@D\relax\newlength{\@D}\settowidth{\@D}{{%
        \bf\thesis@czech@bib@field:}}
      % Unless this is a rigorous thesis, we will be typesetting
      % the name of the thesis advisor.
      \let\@E\relax\newlength{\@E}
        \ifx\thesis@type\thesis@rigorous
          \setlength{\@E}{0pt}%
        \else
          \settowidth{\@E}{{\bf\thesis@czech@bib@advisor:}}
        \fi
      \let\@F\relax\newlength{\@F}\settowidth{\@F}{{%
        \bf\thesis@czech@bib@academicYear:}}
      \let\@G\relax\newlength{\@G}\settowidth{\@G}{{%
        \bf\thesis@czech@bib@pages:}}
      \let\@H\relax\newlength{\@H}\settowidth{\@H}{{%
        \bf\thesis@czech@bib@keywords:}}
      \let\@skip\relax\newlength{\@skip}\setlength{\@skip}{16pt}
      \let\@left\relax\newlength{\@left}\pgfmathsetlength{\@left}{%
        max(\@A,\@B,\@C,\@D,\@E,\@F,\@G,\@H)}
      \let\@right\relax\newlength{\@right}\setlength{\@right}{%
        \textwidth-\@left-\@skip}
      % Typeset the table
      \renewcommand{\arraystretch}{2}
      \noindent\begin{thesis@newtable@old}%
        {@{}p{\@left}@{\hskip\@skip}p{\@right}@{}}
          \textbf{\thesis@czech@bib@author:} &
            \noindent\parbox[t]{\@right}{
              \thesis@author\\
              \thesis@czech@facultyName,
              \thesis@czech@universityName\\
              \thesis@extra@departmentCs
            }\\
          \textbf{\thesis@czech@bib@thesisTitle:}
            & \thesis@extra@titleCs \\
          \textbf{\thesis@czech@bib@programme:}
            & \thesis@extra@programmeCs \\
          \textbf{\thesis@czech@bib@field:}
            & \thesis@extra@fieldCs \\
          % Unless this is a rigorous thesis, typeset the name of the
          % thesis advisor.
          \ifx\thesis@type\thesis@rigorous\else
            \textbf{\thesis@czech@bib@advisor:}
              & \thesis@advisor \\
          \fi
          \textbf{\thesis@czech@bib@academicYear:}
            & \thesis@academicYear \\
          \textbf{\thesis@czech@bib@pages:}
            & \thesis@pages@preamble{} + \thesis@pages \\
          \textbf{\thesis@czech@bib@keywords:}
            & \thesis@extra@TeXkeywordsCs \\
        \end{thesis@newtable@old}}
      \end{alwayssingle}
    }%
  \fi}
%    \end{macrocode}
% \end{macro}\begin{macro}{\thesis@blocks@frontMatter}
% The |\thesis@blocks@frontMatter| macro sets up the style
% of the front matter front matter of the thesis. The front matter
% is typeset without any visible numbering, as mandated by the
% formal requirements of the faculty.
%    \begin{macrocode}
\def\thesis@blocks@frontMatter{%
  \thesis@blocks@clear
  \pagestyle{empty}
  \parindent 1.5em
  \setcounter{page}{1}
  \pagenumbering{roman}}
%    \end{macrocode}
% \end{macro}\begin{macro}{\thesis@blocks@cover}
% The |\thesis@blocks@cover| macro typesets the thesis
% cover. The following extra data field is defined for the macro:
% \begin{itemize}
%   \item|departmentCs| -- the Czech name of the department at
%     which the thesis is being written. This extra data field will
%     expand to |\thesis@department@name| if the main locale of the
%     thesis is Czech.
% \end{itemize}
% \begin{macrocode}
\thesis@def@extra[{
  \ifthesis@czech
    \thesis@department@name
  \else
    \thesis@placeholder@extra@departmentCs
  \fi
}]{departmentCs}
\def\thesis@blocks@cover{{%
  \thesis@selectLocale{czech}
  \ifthesis@cover@
    \thesis@blocks@clear
    \begin{alwayssingle}
      \begin{center}
      {\sc\thesis@titlePage@LARGE\thesis@czech@universityName\\%
          \thesis@titlePage@Large\thesis@czech@facultyName\\[0.3em]%
          \thesis@titlePage@normalsize\thesis@extra@departmentCs}
      \vfill
      {\bf\thesis@titlePage@Huge\thesis@czech@typeName}
      \vfill
      {\thesis@titlePage@large\thesis@place
       \ \thesis@year\hfill\thesis@author}
      \end{center}
    \end{alwayssingle}
  \fi}}
%    \end{macrocode}
% \end{macro}\begin{macro}{\thesis@blocks@titlePage}
% The |\thesis@blocks@titlePage| macro typesets the thesis
% title page. Depending on the value of the |\ifthesis@color@|
% conditional, the faculty logo is loaded from either
% |\thesis@logopath|, if \texttt{false}, or from
% |\thesis@logopath color/|, if \texttt{true}.
% The following extra data field is defined for the macro:
% \begin{itemize}
%   \item|TeXtitleCs| -- the Czech title of the thesis used for the
%     typesetting. This extra data field will expand to
%     |\thesis@TeXtitle| if the main locale of the thesis is Czech.
% \end{itemize}
% \begin{macrocode}
\thesis@def@extra[{
  \ifthesis@czech
    \thesis@TeXtitle
  \else
    \thesis@placeholder@extra@titleCs
  \fi
}]{TeXtitleCs}
\def\thesis@blocks@titlePage{{%
  \thesis@blocks@clear
  \thesis@selectLocale{czech}
  \begin{alwayssingle}
    % The top of the page
    \begin{adjustwidth}{-12mm}{}
      \begin{minipage}{30mm}
        \thesis@blocks@universityLogo@color[width=30mm]
      \end{minipage}\begin{minipage}{89mm}
        \begin{center}
          {\sc\thesis@titlePage@LARGE\thesis@czech@universityName\\%
              \thesis@titlePage@Large\thesis@czech@facultyName\\[0.3em]%
              \thesis@titlePage@normalsize\thesis@extra@departmentCs}
          \rule{\textwidth}{2pt}\vspace*{2mm}
        \end{center}
      \end{minipage}\begin{minipage}{30mm}
        \thesis@blocks@facultyLogo@color[width=30mm]
      \end{minipage}
    \end{adjustwidth}
    % The middle of the page
    \vfill
    \parbox\textwidth{% Prevent vfills from squashing the leading
      \bf\thesis@titlePage@Huge\thesis@extra@TeXtitleCs}
    {\thesis@titlePage@Huge\\[0.8em]}
    {\thesis@titlePage@large\thesis@czech@typeName\\[1em]}
    {\bf\thesis@titlePage@LARGE\thesis@author\\}
    \vfill\noindent
    % The bottom of the page
    {\bf\thesis@titlePage@normalsize
      % Unless this is a rigorous thesis, typeset the name of the
      % thesis advisor.
      \ifx\thesis@type\thesis@rigorous\else
          \thesis@czech@advisorTitle: \thesis@advisor\hfill
      \fi
      \thesis@place\ \thesis@year}
  \end{alwayssingle}}}
%    \end{macrocode}
% \end{macro}\begin{macro}{\thesis@blocks@thanks}
% The |\thesis@blocks@thanks| macro typesets the
% acknowledgement, if the |\thesis@thanks| macro is
% defined. Otherwise, the macro produces no output.
% As per the faculty requirements, the acknowledgement is
% positioned at the top of the page.
% \changes{v0.3.45}{2017/05/24}{Redefined the
%   \cs{thesis@blocks@thanks} and \cs{thesis@blocks@declaration}
%   macros in \texttt{style/mu/fithesis-sci.sty}. The patch was
%   submitted by Juraj Pálenik. [VN]}
%    \begin{macrocode}
\def\thesis@blocks@thanks{%
  \thesis@blocks@clear
  \ifx\thesis@thanks\undefined\else
    \begin{alwayssingle}%
      \chapter*{\thesis@@{thanksTitle}}%
      \leavevmode\thesis@thanks
    \end{alwayssingle}%
  \fi}
%    \end{macrocode}
% \end{macro}\begin{macro}{\thesis@blocks@declaration}
% The |\thesis@blocks@declaration| macro typesets the declaration
% text. Unlike the generic |\thesis@blocks@declaration| macro from
% the \texttt{style/mu/fithesis-sci.sty} file, this definition
% includes the date and a blank line for the author's signature, as
% per the requirements of the faculty.
%    \begin{macrocode}
\def\thesis@blocks@declaration{%
  \begin{alwayssingle}%
    \leavevmode\vfill
    % Start the new chapter without clearing any page.
    {\let\thesis@blocks@clear\relax
    \chapter*{\thesis@@{declarationTitle}}}%
    \thesis@declaration
    \vskip 2cm%
    {\let\@A\relax\newlength{\@A}
      \settowidth{\@A}{\thesis@@{authorSignature}}
      \setlength{\@A}{\@A+1cm}
    \noindent\thesis@place, \thesis@@{formattedDate}\hfill
    \begin{minipage}[t]{\@A}%
      \centering\rule{\@A}{1pt}\\
      \thesis@@{authorSignature}\par
    \end{minipage}}
  \end{alwayssingle}}
%    \end{macrocode}
% \end{macro}
% Note that there is no direct support for the seminar paper and
% thesis proposal types.  If you would like to change the contents
% of the preamble and the postamble, you should modify the
% |\thesis@blocks@preamble| and |\thesis@blocks@postamble| macros.
%
% All blocks within the autolayout preamble and postamble that are
% not defined within this file are defined in the
% \texttt{style/mu/fithesis-base.sty} file. The entire front matter
% is typeset as though the locale were Czech in accordance with the
% formal requirements of the faculty.
%    \begin{macrocode}
\def\thesis@blocks@preamble{
  \thesis@blocks@coverMatter
    \thesis@blocks@cover
  \thesis@blocks@frontMatter
    \thesis@blocks@titlePage
    \thesis@blocks@clearRight
      \thesis@blocks@bibEntryCs
      \thesis@blocks@bibEntry
      \thesis@blocks@bibEntryEn
      \thesis@blocks@abstractCs
      \ifthesis@abstractonsinglepage@
        \begingroup
          \let\clearpage\relax
      \fi
          \thesis@blocks@abstract
          \thesis@blocks@abstractEn
      \ifthesis@abstractonsinglepage@
        \endgroup
      \fi
    \thesis@blocks@assignment
    {\thesis@selectLocale{czech}%
    \thesis@blocks@thanks
    \thesis@blocks@declaration
    \thesis@blocks@clear
      \pagestyle{plain}%
      \thesis@blocks@tables}}
\def\thesis@blocks@postamble{%
  \thesis@blocks@bibliography}
%    \end{macrocode}

% \subsubsection{The style files of the Faculty of Arts}
% % \file{theme/mu/beamercolorthemefibeamer-phil.sty}
% This is the color theme for presentations written at the Faculty
% of Arts at the Masaryk University in Brno. This theme has no
% effect outside the presentation mode.
%    \begin{macrocode}
\NeedsTeXFormat{LaTeX2e}
\ProvidesPackage{fibeamer/theme/mu/%
  beamercolorthemefibeamer-mu-phil}[2016/05/06]
\mode<presentation>
%    \end{macrocode}
% This color theme uses the combination of yellow and shades of gray.  The
% |fibeamer@{dark,|\-|light}@background{Inner,|\-|Outer}| colors are used
% within the background canvas template, which is defined within the base
% color theme of the Masaryk University and which draws the gradient
% background of the frames.
%    \begin{macrocode}
  \definecolor{fibeamer@black}{HTML}{000000}
  \definecolor{fibeamer@white}{HTML}{FFFFFF}
  \definecolor{fibeamer@blue}{HTML}{0071B2}
  \colorlet{fibeamer@lightBlue}{fibeamer@blue!30!fibeamer@white}
  \colorlet{fibeamer@darkBlue}{fibeamer@blue!60!fibeamer@black}
  \definecolor{fibeamer@gray}{HTML}{999999}
  \definecolor{fibeamer@lightOrange}{HTML}{FFA25E}
  \colorlet{fibeamer@orange}{fibeamer@lightOrange!80!fibeamer@darkBlue}
%    \end{macrocode}
% \changes{v1.1.4:2}{2016/05/06}{Removed gradient backgrounds from
%   the color themes of the Masaryk University in Brno. [VN]}
%    \begin{macrocode}
  %% Background gradients
  \colorlet{fibeamer@dark@backgroundInner}{fibeamer@darkBlue}
  \colorlet{fibeamer@dark@backgroundOuter}{fibeamer@darkBlue}
  \colorlet{fibeamer@light@backgroundInner}{fibeamer@white}
  \colorlet{fibeamer@light@backgroundOuter}{fibeamer@white}
%    \end{macrocode}
% The |darkframes| environment switches the |\iffibeamer@darktrue|
% conditional on and sets a dark color theme.
%    \begin{macrocode}
  \renewenvironment{darkframes}{%
    \begingroup
      \fibeamer@darktrue
      %% Structures
      \setbeamercolor*{frametitle}{fg=fibeamer@lightBlue}
      \setbeamercolor*{framesubtitle}{fg=fibeamer@white}
      %% Text
      \setbeamercolor*{normal text}{fg=fibeamer@white, bg=fibeamer@blue}
      \setbeamercolor*{structure}{fg=fibeamer@white, bg=fibeamer@blue}
%    \end{macrocode}
% \changes{v1.1.0:7}{2016/01/12}{Added support for \cs{alert} to
%   the themes of the Masaryk University in Brno. [VN]}
% \changes{v1.1.4:5}{2016/05/06}{Unified the alert colors in the
%   color themes of the Masaryk University in Brno. [VN]}
%    \begin{macrocode}
      \setbeamercolor*{alerted text}{fg=fibeamer@lightOrange}
%    \end{macrocode}
% \changes{v1.1.4:3}{2016/05/06}{Added proper link coloring for the
%   color themes of the Masaryk University in Brno. [VN]}
% \changes{v1.1.6}{2017/04/23}{Added proper citation coloring for the
%   color themes of the Masaryk University in Brno. [VN]}
%    \begin{macrocode}
      %% Items, footnotes and links
      \setbeamercolor*{item}{fg=fibeamer@lightBlue}
      \setbeamercolor*{footnote mark}{fg=fibeamer@lightBlue}
      \hypersetup{urlcolor=fibeamer@lightBlue, citecolor=fibeamer@lightBlue}
      %% Blocks
      \setbeamercolor*{block title}{%
        fg=fibeamer@white, bg=fibeamer@blue!60!fibeamer@white}
      \setbeamercolor*{block title example}{%
        fg=fibeamer@white, bg=fibeamer@blue!60!fibeamer@white}
      \setbeamercolor*{block title alerted}{%
        fg=fibeamer@darkBlue, bg=fibeamer@lightOrange}
      \setbeamercolor*{block body}{%
        fg=fibeamer@blue, 
        bg=fibeamer@gray!15!fibeamer@white}
      \usebeamercolor*{normal text}
      % Code listings
      \lstset{%
        commentstyle=\color{green!30!fibeamer@white},
        keywordstyle=\color{fibeamer@lightBlue},
        stringstyle=\color{red!30!fibeamer@white}}
      }{%
    \endgroup}
%    \end{macrocode}
% Outside the |darkframes| environment, the light theme is used.
%    \begin{macrocode}
  %% Structures
  \setbeamercolor{frametitle}{fg=fibeamer@blue}
  \setbeamercolor{framesubtitle}{fg=fibeamer@black!75!fibeamer@white}
  %% Text
  \setbeamercolor{normal text}{fg=fibeamer@black, bg=fibeamer@white}
  \setbeamercolor{structure}{fg=fibeamer@black, bg=fibeamer@white}
%    \end{macrocode}
% \changes{v1.1.0:7}{2016/01/12}{Added support for \cs{alert} to
%   the themes of the Masaryk University in Brno. [VN]}
% \changes{v1.1.4:5}{2016/05/06}{Unified the alert colors in the
%   color themes of the Masaryk University in Brno. [VN]}
%    \begin{macrocode}
  \setbeamercolor{alerted text}{fg=red}
  \addtobeamertemplate{block begin}{%
    \iffibeamer@dark % alerted text in plain block at dark slides
      \setbeamercolor{alerted text}{fg=fibeamer@orange}%
    \fi}{}
%    \end{macrocode}
% \changes{v1.1.4:3}{2016/05/06}{Added proper link coloring for the
%   color themes of the Masaryk University in Brno. [VN]}
% \changes{v1.1.6}{2017/04/23}{Added proper citation coloring for the
%   color themes of the Masaryk University in Brno. [VN]}
%    \begin{macrocode}
  %% Items, footnotes and links
  \setbeamercolor*{item}{fg=fibeamer@blue}
  \setbeamercolor*{footnote mark}{fg=fibeamer@blue}
  \hypersetup{urlcolor=fibeamer@blue, citecolor=fibeamer@blue}
  %% Blocks
  \setbeamercolor{block title}{%
    fg=fibeamer@white, bg=fibeamer@blue}
  \setbeamercolor{block title example}{%
    fg=fibeamer@white, bg=fibeamer@blue}
  \setbeamercolor{block title alerted}{%
    fg=fibeamer@white, bg=red}
  \setbeamercolor{block body}{%
    fg=fibeamer@blue, bg=fibeamer@gray!20!fibeamer@white}
  %% Title
  \setbeamercolor{title}{fg=fibeamer@white, bg=fibeamer@blue}
  % Code listings
  \lstset{%
    basicstyle=\footnotesize\ttfamily,
    breakatwhitespace=false,
    breaklines=true,
    commentstyle=\color{green!60!fibeamer@black},
    extendedchars=true,
    keywordstyle=\color{fibeamer@blue},
    showspaces=false,
    showstringspaces=false,
    showtabs=false,
    stringstyle=\color{violet}}
\mode
<all>
%    \end{macrocode}

% \subsubsection{The style files of the Faculty of Education}
% % \file{style/mu/fithesis-ped.sty}
% This is the style file for the theses written at the Faculty of
% Education at the Masaryk University in Brno. It has been prepared
% in accordance with the formal requirements published at the
% of the faculty\footnote{See \url{https://is.muni.cz/auth/do/p^^A
% ed/VPAN/pokdek/Pokyn_dekana_zaverecne_prace_finale__1_.pdf}}.
%    \begin{macrocode}
\NeedsTeXFormat{LaTeX2e}
\ProvidesPackage{fithesis/style/mu/fithesis-ped}[2016/04/18]
%    \end{macrocode}
% The file defines the color scheme of the respective faculty. Note
% the the color definitions are in RGB, which makes the resulting
% files generally unsuitable for printing.
%    \begin{macrocode}
\thesis@color@setup{
  links={HTML}{FFA02F},
  tableEmph={HTML}{FFBB6B},
  tableOdd={HTML}{FFF1E0},
  tableEven={HTML}{FFDEB7}}
%    \end{macrocode}
% The bibliography support is enabled. The |numeric| citations are
% used and the bibliography is sorted by name, title, and year.
%    \begin{macrocode}
\thesis@bibliography@setup{
  style=iso-numeric,
  sorting=nty}
\thesis@bibliography@load
%    \end{macrocode}
% The style file configures the title page header to include the
% department name and the title page content to include the
% advisor's name.
%    \begin{macrocode}
\thesis@blocks@titlePage@department@true
\def\thesis@blocks@titlePage@content{%
    {\thesis@titlePage@Huge\bf\thesis@TeXtitle\par\vfil}\vskip 0.8in
    {\thesis@titlePage@large\sc\thesis@@{typeName}\\[0.3in]}
    {\thesis@titlePage@Large\bf\thesis@author}
    % Typeset the name of the thesis advisor.
    {\thesis@titlePage@large\\[0.3in]
      {\bf\thesis@@{advisorTitle}:} \thesis@advisor}}
%    \end{macrocode}
% Note that there is no direct support for the seminar paper and
% thesis proposal types.  If you would like to change the contents
% of the preamble and the postamble, you should modify the
% |\thesis@blocks@preamble| and |\thesis@blocks@postamble| macros.
%
% All blocks within the autolayout preamble and postamble that are
% not defined within this file are defined in the
% \texttt{style/mu/fithesis-base.sty} file.
%    \begin{macrocode}
\def\thesis@blocks@preamble{%
  \thesis@blocks@coverMatter
    \thesis@blocks@cover
    \thesis@blocks@titlePage
  \thesis@blocks@frontMatter
    \thesis@blocks@bibEntry
    \thesis@blocks@abstract
    \thesis@blocks@abstractEn
    \thesis@blocks@keywords
    \thesis@blocks@keywordsEn
    \thesis@blocks@declaration
    \thesis@blocks@thanks
    \thesis@blocks@tables}
\def\thesis@blocks@postamble{%
  \thesis@blocks@bibliography}
%    \end{macrocode}

% \subsubsection{The style files of the Faculty of Social Studies}
% % \file{theme/mu/beamercolorthemefibeamer-fss.sty}
% This is the color theme for presentations written at the Faculty
% of Social Studies at the Masaryk University in Brno.  This theme
% has no effect outside the presentation mode.
%    \begin{macrocode}
\NeedsTeXFormat{LaTeX2e}
\ProvidesPackage{fibeamer/theme/mu/%
  beamercolorthemefibeamer-mu-fss}[2016/05/06]
\mode<presentation>
%    \end{macrocode}
% This color theme uses the combination of yellow and shades of gray.  The
% |fibeamer@{dark,|\-|light}@background{Inner,|\-|Outer}| colors are used
% within the background canvas template, which is defined within the base
% color theme of the Masaryk University and which draws the gradient
% background of the frames.
%    \begin{macrocode}
  \definecolor{fibeamer@black}{HTML}{000000}
  \definecolor{fibeamer@white}{HTML}{FFFFFF}
  \definecolor{fibeamer@cyan}{HTML}{00796E}
  \colorlet{fibeamer@darkCyan}{fibeamer@cyan!70!fibeamer@black}
  \colorlet{fibeamer@lightCyan}{fibeamer@cyan!30!fibeamer@white}
  \definecolor{fibeamer@gray}{HTML}{999999}
  \definecolor{fibeamer@lightOrange}{HTML}{FFA25E}
  \colorlet{fibeamer@orange}{fibeamer@lightOrange!80!fibeamer@darkCyan}
%    \end{macrocode}
% \changes{v1.1.4:2}{2016/05/06}{Removed gradient backgrounds from
%   the color themes of the Masaryk University in Brno. [VN]}
%    \begin{macrocode}
  %% Background gradients
  \colorlet{fibeamer@dark@backgroundInner}{fibeamer@darkCyan}
  \colorlet{fibeamer@dark@backgroundOuter}{fibeamer@darkCyan}
  \colorlet{fibeamer@light@backgroundInner}{fibeamer@white}
  \colorlet{fibeamer@light@backgroundOuter}{fibeamer@white}
%    \end{macrocode}
% The |darkframes| environment switches the |\iffibeamer@darktrue|
% conditional on and sets a dark color theme.
%    \begin{macrocode}
  \renewenvironment{darkframes}{%
    \begingroup
      \fibeamer@darktrue
      %% Structures
      \setbeamercolor*{frametitle}{fg=fibeamer@lightCyan}
      \setbeamercolor*{framesubtitle}{fg=fibeamer@white}
      %% Text
      \setbeamercolor*{normal text}{fg=fibeamer@white, bg=fibeamer@cyan}
      \setbeamercolor*{structure}{fg=fibeamer@white, bg=fibeamer@cyan}
%    \end{macrocode}
% \changes{v1.1.0:7}{2016/01/12}{Added support for \cs{alert} to
%   the themes of the Masaryk University in Brno. [VN]}
% \changes{v1.1.4:5}{2016/05/06}{Unified the alert colors in the
%   color themes of the Masaryk University in Brno. [VN]}
%    \begin{macrocode}
      \setbeamercolor*{alerted text}{fg=fibeamer@lightOrange}
%    \end{macrocode}
% \changes{v1.1.4:3}{2016/05/06}{Added proper link coloring for the
%   color themes of the Masaryk University in Brno. [VN]}
% \changes{v1.1.6}{2017/04/23}{Added proper citation coloring for the
%   color themes of the Masaryk University in Brno. [VN]}
%    \begin{macrocode}
      %% Items, footnotes and links
      \setbeamercolor{item}{fg=fibeamer@lightCyan}
      \setbeamercolor{footnote mark}{fg=fibeamer@lightCyan}
      \hypersetup{urlcolor=fibeamer@lightCyan, citecolor=fibeamer@lightCyan}
      %% Blocks
      \setbeamercolor*{block title}{%
        fg=fibeamer@white, bg=fibeamer@cyan!60!fibeamer@white}
      \setbeamercolor*{block title example}{%
        fg=fibeamer@white, bg=fibeamer@cyan!60!fibeamer@white}
      \setbeamercolor*{block title alerted}{%
        fg=fibeamer@darkCyan, bg=fibeamer@lightOrange}
      \setbeamercolor*{block body}{%
        fg=fibeamer@cyan, 
        bg=fibeamer@gray!15!fibeamer@white}
      \usebeamercolor*{normal text}
      % Code listings
      \lstset{%
        commentstyle=\color{green!30!fibeamer@white},
        keywordstyle=\color{blue!30!fibeamer@white},
        stringstyle=\color{red!30!fibeamer@white}}
      }{%
    \endgroup}
%    \end{macrocode}
% Outside the |darkframes| environment, the light theme is used.
%    \begin{macrocode}
  %% Structures
  \setbeamercolor{frametitle}{fg=fibeamer@cyan}
  \setbeamercolor{framesubtitle}{fg=fibeamer@black!75!fibeamer@white}
  %% Text
  \setbeamercolor{normal text}{fg=fibeamer@black, bg=fibeamer@white}
  \setbeamercolor{structure}{fg=fibeamer@black, bg=fibeamer@white}
%    \end{macrocode}
% \changes{v1.1.0:7}{2016/01/12}{Added support for \cs{alert} to
%   the themes of the Masaryk University in Brno. [VN]}
% \changes{v1.1.4:5}{2016/05/06}{Unified the alert colors in the
%   color themes of the Masaryk University in Brno. [VN]}
%    \begin{macrocode}
  \setbeamercolor{alerted text}{fg=red}
  \addtobeamertemplate{block begin}{%
    \iffibeamer@dark % alerted text in plain block at dark slides
      \setbeamercolor{alerted text}{fg=fibeamer@orange}%
    \fi}{}
%    \end{macrocode}
% \changes{v1.1.4:3}{2016/05/06}{Added proper link coloring for the
%   color themes of the Masaryk University in Brno. [VN]}
% \changes{v1.1.6}{2017/04/23}{Added proper citation coloring for the
%   color themes of the Masaryk University in Brno. [VN]}
%    \begin{macrocode}
  %% Items, footnotes and links
  \setbeamercolor{item}{fg=fibeamer@cyan}
  \setbeamercolor{footnote mark}{fg=fibeamer@cyan}
  \hypersetup{urlcolor=fibeamer@cyan, citecolor=fibeamer@cyan}
  %% Blocks
  \setbeamercolor{block title}{%
    fg=fibeamer@white, bg=fibeamer@cyan}
  \setbeamercolor{block title example}{%
    fg=fibeamer@white, bg=fibeamer@cyan}
  \setbeamercolor{block title alerted}{%
    fg=fibeamer@white, bg=red}
  \setbeamercolor{block body}{%
    fg=fibeamer@cyan, bg=fibeamer@gray!20!fibeamer@white}
  %% Title
  \setbeamercolor{title}{fg=fibeamer@white, bg=fibeamer@cyan}
  % Code listings
  \lstset{%
    basicstyle=\footnotesize\ttfamily,
    breakatwhitespace=false,
    breaklines=true,
    commentstyle=\color{green!60!fibeamer@black},
    extendedchars=true,
    keywordstyle=\color{blue},
    showspaces=false,
    showstringspaces=false,
    showtabs=false,
    stringstyle=\color{violet}}
\mode
<all>
%    \end{macrocode}

% \subsubsection{The style files of the Faculty of Law}
% % \iffalse
%<*driver>
\documentclass{ltxdoc}
\usepackage{camel}
\usepackage{array}

\def\fileversion{0.1d}
\def\filedate{1997/01/09}
\def\docdate {1997/01/09}

\EnableCrossrefs
\RecordChanges                  % Gather update information
\CodelineIndex                  % Index code by line number
%\OnlyDescription  % comment out for implementation details
\let\lawhline=\hline
\begin{document}
\citationstyle{law}
\citationdata{camel}
\title{The {\sc Law} module for the {\sc Camel} Bibliography
  Engine\thanks{%
   This file has version number \fileversion{} dated
   \filedate{}.
   The documentation was last revised on \docdate.
   The documentation and the code for {\sc Camel}
   are \copyright{} 1992--95 Frank Bennett, Jr.  Distribution
   and use are freely welcomed, on the sole condition that
   acknowledgement of the {\sc Camel} package, its {\sc Law}
   module and of their author
   be made in any published using these utilities.
      }}
\author{Frank G Bennett, Jr.\thanks{Lecturer in the Commercial
    Laws of the Far East, School of
    Oriental and African Studies, University of London.
    Acknowledgements of the numerous people who have provided
    comments and suggestions in the development of this module
    and of the {\sc Camel} package itself are listed in the users'
    guide to that software.}}
\maketitle
\DocInput{law.dtx}
\end{document}
%</driver>
% \fi
% \CheckSum{741}
% 
%\iffalse
%
% Copyright (C) 1992-1995 by Frank Bennett.  All rights reserved.
%
% IMPORTANT NOTICE:
% 
% You are not allowed to change this file.  You may however copy
% this file to a file with a different name and then change the
% copy if (a) you do not charge for the modified code, (b) you
% acknowledge Camel and its author(s) in the new file, if it
% is distributed to others, and (c) you attach these same
% conditions to the new file.
% 
% You are not allowed to distribute this file alone.  You are not
% allowed to take money for the distribution or use of this file
% (or a changed version) except for a nominal charge for copying
% etc.
% 
% You are allowed to distribute this file under the condition that
% it is distributed with all of its contents, intact.
% 
% For error reports, or offers to help make Camel a more powerful,
% friendlier, and altogether more thrilling package, please contact me on
% fb@soas.ac.uk
% 
%\fi
%
%
% \DoNotIndex{\@,\@@par,\@beginparpenalty,\@empty}
% \DoNotIndex{\@flushglue,\@gobble,\@input}
% \DoNotIndex{\@makefnmark,\@makeother,\@maketitle}
% \DoNotIndex{\@namedef,\@ne,\@spaces,\@tempa}
% \DoNotIndex{\@tempb,\@tempswafalse,\@tempswatrue}
% \DoNotIndex{\@thanks,\@thefnmark,\@topnum}
% \DoNotIndex{\@@,\@elt,\@forloop,\@fortmp,\@gtempa,\@totalleftmargin}
% \DoNotIndex{\",\/,\@ifundefined,\@nil,\@verbatim,\@vobeyspaces}
% \DoNotIndex{\|,\~,\ ,\active,\advance,\aftergroup,\begingroup,\bgroup}
% \DoNotIndex{\cal,\csname,\def,\documentstyle,\dospecials,\edef}
% \DoNotIndex{\egroup}
% \DoNotIndex{\else,\endcsname,\endgroup,\endinput,\endtrivlist}
% \DoNotIndex{\expandafter,\fi,\fnsymbol,\futurelet,\gdef,\global}
% \DoNotIndex{\hbox,\hss,\if,\if@inlabel,\if@tempswa,\if@twocolumn}
% \DoNotIndex{\ifcase}
% \DoNotIndex{\ifcat,\iffalse,\ifx,\ignorespaces,\index,\input,\item}
% \DoNotIndex{\jobname,\kern,\leavevmode,\leftskip,\let,\llap,\lower}
% \DoNotIndex{\m@ne,\next,\newpage,\nobreak,\noexpand,\nonfrenchspacing}
% \DoNotIndex{\obeylines,\or,\protect,\raggedleft,\rightskip,\rm,\sc}
% \DoNotIndex{\setbox,\setcounter,\small,\space,\string,\strut}
% \DoNotIndex{\strutbox}
% \DoNotIndex{\thefootnote,\thispagestyle,\topmargin,\trivlist,\tt}
% \DoNotIndex{\twocolumn,\typeout,\vss,\vtop,\xdef,\z@}
% \DoNotIndex{\,,\@bsphack,\@esphack,\@noligs,\@vobeyspaces,\@xverbatim}
% \DoNotIndex{\`,\catcode,\end,\escapechar,\frenchspacing,\glossary}
% \DoNotIndex{\hangindent,\hfil,\hfill,\hskip,\hspace,\ht,\it,\langle}
% \DoNotIndex{\leaders,\long,\makelabel,\marginpar,\markboth,\mathcode}
% \DoNotIndex{\mathsurround,\mbox,\newcount,\newdimen,\newskip}
% \DoNotIndex{\nopagebreak}
% \DoNotIndex{\parfillskip,\parindent,\parskip,\penalty,\raise,\rangle}
% \DoNotIndex{\section,\setlength,\TeX,\topsep,\underline,\unskip,\verb}
% \DoNotIndex{\vskip,\vspace,\widetilde,\\,\%,\@date,\@defpar}
% \DoNotIndex{\[,\{,\},\]}
% \DoNotIndex{\count@,\ifnum,\loop,\today,\uppercase,\uccode}
% \DoNotIndex{\baselineskip,\begin,\tw@}
% \DoNotIndex{\a,\b,\c,\d,\e,\f,\g,\h,\i,\j,\k,\l,\m,\n,\o,\p,\q}
% \DoNotIndex{\r,\s,\t,\u,\v,\w,\x,\y,\z,\A,\B,\C,\D,\E,\F,\G,\H}
% \DoNotIndex{\I,\J,\K,\L,\M,\N,\O,\P,\Q,\R,\S,\T,\U,\V,\W,\X,\Y,\Z}
% \DoNotIndex{\1,\2,\3,\4,\5,\6,\7,\8,\9,\0}
% \DoNotIndex{\!,\#,\$,\&,\',\(,\),\+,\.,\:,\;,\<,\=,\>,\?,\_}
% \DoNotIndex{\discretionary,\immediate,\makeatletter,\makeatother}
% \DoNotIndex{\meaning,\newenvironment,\par,\relax,\renewenvironment}
% \DoNotIndex{\repeat,\scriptsize,\selectfont,\the,\undefined}
% \DoNotIndex{\arabic,\do,\makeindex,\null,\number,\show,\write,\@ehc}
% \DoNotIndex{\@author,\@ehc,\@ifstar,\@sanitize,\@title,\everypar}
% \DoNotIndex{\if@minipage,\if@restonecol,\ifeof,\ifmmode}
% \DoNotIndex{\lccode,\newtoks,\onecolumn,\openin,\p@,\SelfDocumenting}
% \DoNotIndex{\settowidth,\@resetonecoltrue,\@resetonecolfalse,\bf}
% \DoNotIndex{\clearpage,\closein,\lowercase,\@inlabelfalse}
% \DoNotIndex{\selectfont,\mathcode,\newmathalphabet,\rmdefault}
% \DoNotIndex{\bfdefault}
% \setcounter{StandardModuleDepth}{1}
% \MakeShortVerb{\"}
% 
% \begin{abstract}
% \noindent The {\tt Law} module for the {\sc Camel} bibliography
% package and \BibTeX{} attempts to implement fully automated typesetting
% of citations in the so-called Blue Book style used in
% the publication of legal materials.  This
% demanding style requires context sensitive
% in-footnote cross-referencing
% between citations.
% \end{abstract}
%
% \noindent An adage in office management is that you should
% only touch incoming paper once; to respond to it, to file
% it, to forward it, or to destroy it.  A number of 
% commercial citation
% database managers provide a facility for ``filing'' citations
% in a flexible form, the idea being to extend this principle
% to citations as well as paper.  A hanging point for this
% strategy as been in-text context-sensitive citation styles.
% Database managers are at their best in exporting 
% entire bibliographies
% and lists of authorities.  Some packages are
% capable of scanning a document for citation ``tags'', 
% which eliminates the need to separately select bibliography
% items in the database.  Some, too, can replace ``tags''
% in the document with the formatted text of a citation.
% But once the text is replaced, the format of citations
% added in this way is fixed; conversion is a one-way process.
%
% A more serious problem is that, while the database manager
% can easily identify the {\em tag}, it is far more difficult,
% without logical markup, to identify its 
% {\em context\/}---whether it occurs in a footnote, how many
% items were in the preceding footnote, how many articles
% by the same author are cited in the document, and so forth.
% As a result, citation formatting of cross-referenced styles
% is still generally done by hand.
%
% One of the most demanding cross-referenced styles is that
% laid down in {\sl A Uniform System of Citation}, or ``the Blue
% Book'', for the citation of legal materials.  This style
% has survived the era of computerization largely because
% most U.S. law journals using the style are edited by highly
% competitive law students.  Staff members contribute their
% editorial time to their journal free of charge, because of
% the value of listing law journal membership on their resume.
% The Blue Book style minimizes the bulk of citation
% text, while conveying sufficient information to the reader
% for the location of cited material.  It is also designed
% to provide all information required for the location of
% {\em primary\/} legal authority from the face of any citation
% to it, without tracing down cross-references.
%
% A first attempt to address this problem was made in 
% the complementary \LexiTeX{}
% and \LexiBib{} packages, for \LaTeX{} and \BibTeX{}
% respectively.  Lessons learned from developing these
% packages suggested that a comprehensive formatting package
% for legal citations could serve as the foundation for
% a modular, highly generalized citation and bibliography
% formatting system.  The logical code from these packages
% was excised and used in the drafting of the {\sc Camel}
% bibliography engine.  The typesetting code was then used
% to assemble the {\sc Law} module for {\sc Camel}.
%
% The implications of such a system are particularly interesting
% if it comes to be widely used in the publication of
% law journals or, even better, in the publication of court
% judgements.  The efficiency with which
% citations can be reported to citation services would be
% increased, since \BibTeX-format database entries
% are in a standard format that can
% be processed electronically.  These same lists could
% be made available as text-searchable databases of authority.
% Network discussions of legal issues could be accompanied by
% growing lists of annotated authority, available for all to
% use.
%
% To use this package, all you should need to do is unpack
% the files by running {\tt law.ins} (thanks to Robin Fairbairns
% for pushing me toward a standard-ish method of installation!),
% copy the style files to their respective homes, and follow the
% markup conventions described in the {\sc Camel} manual.  The
% same should apply to any other bibliography
% modules produced for {\sc Camel} in the future, no matter what
% sort of contortions they put your citation data through before
% they end up on the page.  If it has extensive documentation,
% it's {\em not\/} a {\sc Camel} module!
%
% You are not under an obligation to enjoy this package.  If you
% have complaints, you can contact me on {\tt fb@soas.ac.uk}.
%
%\StopEventually{}
% \section{The style code}
% \subsubsection{Entry type specific functions}
% The functions below should not be used as general utilities;
% they are designed specifically for use with a particular entry
% type.  While they could be placed directly into the entry
% functions to which they apply, defining them separately
% helps improve the transparency of the code.
%
% \DescribeEnv{build.bridges}
% I'm not actually sure whether this function belongs in {\tt
% camel.dtx} or in {\tt law.dtx}.  What it does is to insert (or
% not insert) an appropriate "\bridges" declaration into the {\tt
% .bbl} file for the current citation.  It is governed by a
% string existing on the stack when it is invoked, and leaves
% nothing on the stack.  Toggles are {\tt schedules}, {\tt sections} and {\tt
% articles}.  Season to taste.
%    \begin{macrocode}
%<*bstfunctions>
FUNCTION {build.bridges}
  { duplicate$ empty$
    { pop$ skip$ }
    { duplicate$ "sections" =
      { pop$ "\bridges{\ \S~}{}{\ \S\S~}{\ }{\ \S~}{\ \S\S~}"
      "" "" must.must.must
      }

      { duplicate$ "articles" =
        { pop$
          "\bridges{\ art.~}{}{\ arts.~}{\ }{\ art.~}{\ arts.~}"
          "" "" must.must.must
        }
        { "schedules" =
          { "\bridges{\ sched.~}{}{\ scheds.~}{\ }{\ sched.~}{\ scheds.~}"
            "" "" must.must.must
          }
          'skip$
          if$
        }if$
      }if$
    }if$
 }
%    \end{macrocode}
% \DescribeEnv{get.a.kinda.sort.key}
% It is early days for the sorting of bibliographies in {\sc
% Camel}.  This will give a rough key.
%    \begin{macrocode}
FUNCTION { get.a.kinda.sort.key }
{     author empty$
     { title empty$
       { "0000" }
       { title "*" "forward" gather.chars pop$
         duplicate$ "l" change.letter.case "the" =
         { pop$ "*" "forward" gather.chars pop$ swap$ pop$ }
         { duplicate$ "l" change.letter.case "a" =
           { pop$ "*" "forward" gather.chars pop$ swap$ pop$ }
           { swap$ pop$
           }if$
         }if$
       }if$
     }
     { author #1 "{ll}" format.name$
     }if$
}
%    \end{macrocode}
% \DescribeEnv{j.format.division}
% This function assembles the various fields relevant to the
% division of a Japanese court into a single string.
%    \begin{macrocode}
FUNCTION {j.format.division}
 { division empty$
   'skip$
   { dc.. court "end" first.in.second
     { pop$ " No.~" 's := "endlabel" 't := }
     { pop$ sc.. court "start" first.in.second
       { pop$ "No.\ " 's := "frontlabel" 't := }
       {  pop$ " " 's :=  "endlabel" 't :=
       }if$
     }if$
   }if$
   division s divno t field.tag.no.combine
 }
%    \end{macrocode}
%
% \subsubsection{Entry type functions}
% Now we define the type functions for all entry types that may
% appear
% in the {\tt*.bib} file---e.g., functions like `"article"' and 
% `"book"'.  These
% are the routines that actually generate the "*.bbl" file output
% for
% the entry.  These must all precede the "READ" command.  In
% addition, the
% style designer should have a function `"default.type"' for
% unknown types.\footnote{This comment by Oren Patashnik.}
% 
% \DescribeEnv{article}
% This function performs the necessary operations for
% exporting a valid \LexiTeX\ citation to an article.
% For this and for all citation types defined in the
% \LexiBib{} style, the goal is to provide reasonably
% complete commentary, so that anyone wanting to
% alter the behaviour of the style can set to work
% with a fair degree of confidence about what needs to
% be done to achieve a particular result.
%    \begin{macrocode}
FUNCTION {article}
%    \end{macrocode}
% The "article" entry is used for all kinds of material, so
% it ends up as one of the most complex entries.  Before we
% do anything, we have to check whether the default {\sc Camel}
% bridges are acceptable.  There are two types of situations to watch out
% for.  First, if the volume and the number are both
% non-empty, we need to add a special set of substitute bridges.
% Alternatively, if both "volume" and "number" are empty,
% but the date is more specific than just a year, we
% assume we are dealing with a newspaper.  In either case,
% we use the same special set of bridges.
%    \begin{macrocode}
{ volume empty$ not number empty$ not and
  volume empty$ number empty$ year "mo.dd.yy" format.date
  pop$ itemcount #1 = not and and or
  { "\bridges{,\ p.~}{,\ }{,\ }{\ }{\ at~}{\ at~}" 
    "" "" must.must.must 
    newline$
  }
%    \end{macrocode}
% Second, the {\tt @article} entry type, like many entry types in
% the Blue Book style normally places 
% white space between the title
% and a pinpoint page number.  If the title ends in a numeral
% this will be confusing, so the Blue Book requires that we
% separate the two with a comma in this case.  The following
% adjustment to bridging punctuation accomplishes this purpose.
%    \begin{macrocode}
  { title type.last.char "numeral" =
    { "\bridges{,\ }{,\ }{,\ }{\ }{\ at~}{\ at~}"
      "" "" must.must.must
      newline$
    }
    'skip$
    if$
  }if$
%    \end{macrocode}
% \changes{1995/07/08}{v0.1a}{Function {\tt type.last.char} and
% {\tt\protect\string\protect\bridges} declaration used to handle titles ending
% in a numeral correctly (by the insertion of a comma).  Added to
% the {\tt @article} and {@book} entries as a trial; will
% propagate to other entry types once this change is trusted.}
% Then we write the citation leader, to prepare for outputting the
% actual content of the citation text.
%    \begin{macrocode}
 "\lexibib{article}{" cite$ "}{" must.must.must
 get.a.kinda.sort.key 
 "}{" "" must.must.must 
%    \end{macrocode}
% The author name is pushed to the stack, followed by a toggle
% to trigger last-name-only formatting.  Then the "format.names"
% reduces this to a single, appropriately-formatted string,
% possibly the null string.  All that is left to do is push a
% set a braces, a null string to make up three arguments to the
% export routine, and write the lot on the output
% file unconditionally, using "must.must.must".
%    \begin{macrocode}
author  "lastonly" format.names
"}{"
""
must.must.must
%    \end{macrocode}
% Next comes the title of the article.  This is not specially
% formatted; we simply push the title field, then a warning
% string
% followed by a check for whether it is empty or not, then
% braces and a null string, and write again.
%    \begin{macrocode}
title "title" check
"}{"
""
must.must.must
%    \end{macrocode}
% The next bit is actually rather thorny.  There are three
% possible cases.  The first is where both a volume number
% and an issue number exist.  In this case, we use a
% verbose form of reference. In the second, there is no
% volume number, but possibly an issue number.  This is
% the proper form for Commonwealth legal materials,
% and requires the year in brackets, followed by the 
% issue number to show the volume within the year.
% Third, we may simply have a volume number by itself.
% This calls for the citation form for most journals,
% and U.S. case reporters.
%    \begin{macrocode}
  volume empty$ not number empty$ not and
  { journal "journal" check ", v.~" volume must.must.must
    ", n.~" number "}{" must.must.must }
  { volume empty$ not
    { volume "\ " journal "journal" check empty.to.null
      might.ifone.must
      "}{" "" "" must.must.must}
    { year "mo.dd.yy" format.date itemcount #1 =
      { "[" swap$ "] " iftwo.might.iftwo
      number "\ " journal "journal" check might.ifone.must
      "}{" "" "" must.must.must}
      { pop$ number empty$
        { "\\" journal "journal" check "\\" must.must.must
          "}{" "" "" must.must.must }
        { number "\ " journal "journal" check might.ifone.must
          "}{" "" "" must.must.must
        }if$
      }if$
    }if$
  }if$ 
%    \end{macrocode}
% The Blue Book does not like page ranges
% so we need to clean
% out anything following a dash in the "pages" field.  The
% "short" option
% toggles this behaviour on.  We also check to see that the page
% is not empty.  This is followed by braces, a null string, and
% output.
%    \begin{macrocode}
pages "short" format.pages "pages" check
"}{"
""
must.must.must
%    \end{macrocode}
% We add the year next, but only if the "volume" field is
% non-empty (if "volume" is empty, we'll have put the year
% is as a bracketed volume number, Commonwealth-style.
% We could use just the year (this is normal for Blue Book
% style, but we'll add the month for good measure, if it's
% been provided in the "year" field.
%    \begin{macrocode}
volume empty$ not
  { "(" "" "" must.must.must
    publisher ":\ " "" might.ifone.must
    "" year "mo.dd.yy" format.date ")" must.must.must }
  { year "mo.dd.yy" format.date itemcount #1 =
    'pop$
    { "" "" must.must.must
    }if$
  }if$
%    \end{macrocode}
% Finally, we have to tangle with cross-references.
% Yuck.  (Not a complaint about \BibTeX, just a general
% response to the design problems inherent in the task).
% Formatting depends on whether there is a "crossref" entry.
% Hear that, guys?  If you don't use the "crossref" field,
% we'll short-change you on formatting service.
%    \begin{macrocode}
crossref empty$
%    \end{macrocode}
% If there is no "crossref", we just push a brace and a couple
% of nulls, and write.  Done!  Hurray!
%    \begin{macrocode}
{ "}" "" "" must.must.must }
%    \end{macrocode}
% But if there {\em was\/} a "crossref", we've got work to do.
% Darn.  The first thing we do is have a look at "booktitle".
% This should be non-null in this situation; there's no sense
% setting up a cross-reference to an individual volume of
% a journal unless there's something special to be said
% about it.
%    \begin{macrocode}
{ booktitle empty$
%    \end{macrocode}
% So if no "booktitle" is found, we whinge and format
% as for a no-cross-reference entry.
%    \begin{macrocode}
  { "no booktitle (name of special issue) for "
     cite$ "/" crossref ".  Why a crossref?" * * * * warning$
    "}" "" "" must.must.must }
%    \end{macrocode}
% If there {\em is\/} a "booktitle", though, 
% and if the "volume" field is non-empty (which
% means that we just printed, or at least should
% have printed, the year), we close the
% parens following the year (which is opened by \LexiTeX),
% and open another (which will be closed by \LexiTeX).
% That's it for conditional punctuation; we follow
% with "booktitle", which should be 
% the title or subject description of the special
% issue of the journal.
%    \begin{macrocode}
  { volume empty$ {""} {") ("} if$ booktitle "}" must.must.must}
  if$
}if$
%    \end{macrocode}
% Add a fresh new line in the export file, and we're done!  Whew!
%    \begin{macrocode}
newline$
}
%    \end{macrocode}
%
% \DescribeEnv{book}
% This is the entry for books, which includes individual
% volumes in a series, and multi-volume works with a single
% title.  Correct me if I'm wrong, but I think this latter
% citation type is not supported by \BibTeX.  \LexiBib{}
% manages it by allowing the user to specify the volume
% number in the text using the optional ":<number>:" argument to
% the "\lexicite" tag.
%    \begin{macrocode}
FUNCTION {book}
%    \end{macrocode}
% This entry type normally places white space between the title
% and a pinpoint page number.  If the title ends in a numeral
% this will be confusing, so the Blue Book requires that we
% separate the two with a comma in this case.  The following
% adjustment to bridging punctuation accomplishes this purpose.
%    \begin{macrocode}
{ title type.last.char "numeral" =
    { "\bridges{,\ }{}{,\ }{\ }{\ at~}{\ at~}"
      "" "" must.must.must
      newline$
    }
    { units empty.to.null "paras" =
      { "\bridges{,\ para.~}{}{,\ paras.~}{\ }{,\ para.~}{,\ paras.}"
        "" "" must.must.must
        newline$
      }
      'skip$
      if$
    }if$
%    \end{macrocode}
% Next, after the opener, we push the opening macro tag for a book
% entry, the nickname of the citation, and a couple of
% braces.  This is all mandatory and can safely be given
% unconditional export.
%    \begin{macrocode}
  "\lexibib{book}{" cite$ "}{" must.must.must
   get.a.kinda.sort.key "" "}{" must.must.must
%    \end{macrocode}
% A non-empty "volume" field means we need a leading
% volume number in the Blue Book style.  If a volume
% number (or anything else) is found in the "volume"
% field of a book entry, we replace it with a 
% "volno" macro.  This will expand in the document to
% whatever the author has specified using the optional
% ":<number>" argument to "\lexicite".  For example,
% volume 8 of "holdsworth" would be: "\lexicite:8:{holdsworth}".
%    \begin{macrocode}
  volume empty$
  { "" }
  { "\volno\ " }
  if$
%    \end{macrocode}
% The author comes first.  We push the contents of the
% "author" field, then the toggle string ``"firstinitial"'',
% and run  the "format.names" function to produce the name
% formatted properly for a book entry.  Then we push
% fa couple of braces
% and force all three items ("\volno" or null, "author", and
% braces) onto the output.
%    \begin{macrocode}
  author
  "firstinitial" format.names "}{" must.must.must
%    \end{macrocode}
% The title may be a simple name for a single volume,
% or it may be in title/subtitle format.  Title/subtitle
% can be written into the "title" entry directly as
% ``\meta{title}: \meta{subtitle}''.  We provide for
% an alternative form here, just in case "booktitle"
% is used for the main title --- conservation of
% data.  If there is no "booktitle", we push the title, then
% check to be sure it's non-empty, then write it onto output.
% If "booktitle" exists, we push it, then a colon, then
% "title", followed by a conditional write on output that
% will suppress the colon if "title" is empty.
% Finally, we push the closing braces and an opening
% parens (for the year info) and write that stuff on output.
%    \begin{macrocode}
  booktitle empty$
    { title "title" check "" ""  must.must.must }
    { booktitle ": " title must.ifthree.might
    }if$
  type empty.to.null "cmnd" =
  type empty.to.null "command" =
  type empty.to.null "command papers" =
  or or
  { ", " year "mo.dd.yy" format.date
    ", " "Cmnd. "
    * * *
    series empty$
    { "no.~" }
    { "Ser.~" series ", No.~" * *
    }if$
  number "number" check
   * *
   "" "" must.must.must
   }
   'skip$
   if$
  "}{}{}{" "(" "" must.must.must
%    \end{macrocode}
% We're now in the final ``field'' of the \LexiTeX{} entry.
% This is mainly for the year, but we also give the 
% name of the editor(s) or translator(s) 
% if present.  If there is no
% editor or translator, we'll put a series name here, to help identify
% the source.  We don't put both, because this would
% confuse things (there can be book editors and
% series editors too, and the Blue Book style is 
% too streamlined to distinguish the two elegantly).
% So our first task is to see if there is an editor\ldots
%    \begin{macrocode}
    editor empty$ translator empty$ and
%    \end{macrocode}
% \ldots{} and if there is none, we put in a
% series name if it exists.  
% The "series" \BibTeX{} field should be used for the
% name of the series of which a volume forms a part,
% but some folks might accidentally use "booktitle".
% We'll be forgiving and accept it anyway.
%    \begin{macrocode}
    { series empty$
%    \end{macrocode}
%  If both
% "series" is empty, we've nothing to do.
%    \begin{macrocode}
      'skip$
%    \end{macrocode}
%
% If "series" exists, we push it, followed by
% a bridge and a series item number,
% and a toggle for the "field.tag.no.combine" function.
% The "endlabel" toggle causes this function to put
% the bridge and the number after the series name,
% if a number exists, and push the lot back as a single
% item on the stack.  Otherwise it leaves just the series
% name.
%
% And last, we put up a comma to close, and do a mandatory
% export of the lot.
%    \begin{macrocode}
      { series " No.~" number
      "endlabel" field.tag.no.combine ", " "" must.must.must }
      if$
    }
%    \end{macrocode}
% If either the editor or the translator fields were not empty,
% we format the editor or translator name
% instead, and put those details here.
%    \begin{macrocode}
    { editor translator either.or "firstinitial" format.names
%    \end{macrocode}
% We need to append the correct designator, either ``ed.''\ or
% ``trans.''.  The "either.or" function will use the second
% item pushed if both are non-empty, so we take advantage of
% this ``feature'' in making our choice of designators; the
% "ed." or "eds." strings are only used if the "translator"
% field is empty.  And finally, we push a null string to
% round out, and do a compulsory export.
%    \begin{macrocode}
      translator empty$
      { editor num.names$ #1 >
        { " eds.\ " }
        { " ed.\ " }
        if$
      }
      { " trans.\ " }
      if$
      "" must.must.must
    }if$
%    \end{macrocode}
% We also need to indicate the edition, if any.
%    \begin{macrocode}
  edition " ed.\ " "" might.ifone.must
%    \end{macrocode}
% The year itself is easy.  We push the year, do a check to issue
% a warning if necessary, then run format.date over it,
% which yields the year in "theyear", which can be pushed
% back onto the stack.  Then we fill out to six
% \LexiTeX{} fields in
% all, and do a compulsory export.
%    \begin{macrocode}
  type empty.to.null "cmnd" =
  type empty.to.null "command" =
  type empty.to.null "command papers" =
  or or
  { "" }
  { year "mo.dd.yy" format.date
  }if$
  ")}" "" must.must.must
%    \end{macrocode}
% A new line for a new macro, and we're done!  Rejoice!
% On to the next function definition!
%    \begin{macrocode}
  newline$
}
%    \end{macrocode}
%
% \DescribeEnv{incollection}
% This is for those nasty entries that are created when someone
% publishes an article in a collection of essays edited by someone
% else.
%    \begin{macrocode}
FUNCTION{incollection}
{ "\lexibib{incollection}{" cite$ "}{" must.must.must
   get.a.kinda.sort.key "" "}{" must.must.must
  author "lastonly" format.names "author" check
  "}{" "" must.must.must
  title "title" check
  "}{" "" must.must.must
  chapter empty$
    { "\\in \\" }
    { "\\" type empty$
      { "chapter " chapter "chapter" check " of \\" * * * }
      { type " " chapter "chapter" check " of \\" * * * * 
      }if$
    }if$
  booktitle "booktitle" check
  "}{" must.must.must
  pages "short" format.pages "pages" check
  "}{" "(" must.must.must
% We're now in the final ``field'' of the \LexiTeX{} entry.
% The coding here is the same as for a "book" entry; the
% reader is referred to that entry for the commentary
% on the following code.
%    \begin{macrocode}
    editor empty$ translator empty$ and
    { series empty$
      'skip$
      { series " No.~" number
      "endlabel" field.tag.no.combine ", " "" must.must.must }
      if$
    }
    { editor booktranslator either.or "firstinitial" format.names
      booktranslator empty$
      { editor num.names$ #1 >
        { " eds.\ " }
        { " ed.\ " }
        if$
      }
      { " trans.\ " }
      if$
      "" must.must.must
    }if$
  edition " ed.\ " "" might.ifone.must
  year "mo.dd.yy" format.date ")}" "" must.must.must
  newline$
}
%    \end{macrocode}
% \DescribeEnv{inbook}
% There is significant massaging of the use to which
% "title" and "booktitle" is put here.  If both occur,
% but are different, the form is ``"title" in "booktitle"'',
% or ``"title", chapter X of "booktitle"''.
% If both occur, but are the same, the treatment depends
% on whether "chapter" occurs.  If it does not occur,
% the style complains --- an "@inbook" entry should always
% refer to a chapter or other specific unit if the
% "title" and "booktitle" are the same.  If it does occur,
% then the chapter or other specifier is treated as a
% volume number inserted as an option to "\source" using "v=",
% the "title" being inserted as the title of the work.
%    \begin{macrocode}
FUNCTION{inbook}
{ "\lexibib{inbook}{" cite$ "}{" must.must.must
  get.a.kinda.sort.key "" "}{" must.must.must
  author "firstinitial" format.names "author" check
  "}{" "" must.must.must
  title booktitle =
    { chapter empty$
      { "no chapter or separate title for inbook entry " cite$ * warning$
      }
      { "{\rm " "Chapter" type either.or.nowarning 
        " " chapter " of} " * * * *
      }if$
    }
    { ""
    }if$
  title "title" check
  "}{" must.must.must
  title booktitle =
    { "" }
    { chapter empty$
      { "{\rm in} " }
      { type empty$
        { "{\rm Chapter " chapter " of} " * * }
        { "{\rm " type " " chapter " of} " * * * *
        }if$
      }if$
    }if$
  booktitle "booktitle" check
  "}{" must.must.must
  pages "short" format.pages "pages" check
  "}{" "(" must.must.must
% We're now in the final ``field'' of the \LexiTeX{} entry.
% The coding here is the same as for a "book" entry; the
% reader is referred to that entry for the commentary
% on the following code.
%    \begin{macrocode}
    translator empty$
    { series empty$
      'skip$
      { series " No.~" number
      "endlabel" field.tag.no.combine ", " "" must.must.must }
      if$
    }
    { translator "firstinitial" format.names
      " trans.\ "
      "" must.must.must
    }if$
  edition " ed.\ " "" might.ifone.must
  year "mo.dd.yy" format.date ")" "}" must.must.must
  newline$
}
%    \end{macrocode}
%
%    \begin{macrocode}
FUNCTION{booklet}
{ "\lexibib{booklet}{" cite$ "}{" must.must.must
  get.a.kinda.sort.key "" "}{" must.must.must
  author "full" format.names "}{" "" must.must.must
  "\\" title "\\}{}{}{(" must.must.must
  howpublished ", " "" might.ifone.must
  year "mo.dd.yy" format.date
  ")}" "" must.must.must
  newline$
}
%    \end{macrocode}
% \changes{1994/12/12}{1.0g}{Added the `techreport' function,
% to support draft article sent to Pedro Aphalo for comments.}
%    \begin{macrocode}
FUNCTION {techreport}
{"\lexibib{techreport}{" cite$ "}{" must.must.must
  get.a.kinda.sort.key "" "}{" must.must.must
  institution author either.or.nowarning
  "full" format.names "author & institution" check
  "}{" title "title" check must.must.must
  "}{}{}{" "(" "" must.must.must
  author empty$
    'skip$
    {institution "\ "  "" might.ifone.must
    }if$
  type empty$
    { "Technical report" }
    { type
    }if$
  type empty.to.null "Cmnd" =
  { "\ " }
  { " No.~" 
  }if$
  number "endlabel" field.tag.no.combine
  ", " "" must.must.must
  year "mo.dd.yy" format.date
  "" ")}" must.must.must
  newline$
  }
%    \end{macrocode}
%    \begin{macrocode}
FUNCTION {mastersthesis}
{"\lexibib{mastersthesis}{" cite$ "}{" must.must.must
   get.a.kinda.sort.key "}{" "" must.must.must
  author
  "full" format.names "author" check
  "}{" title "title" check must.must.must
  "}{}{}{("
  type empty$
    { "Master's Thesis" }
    { type
    }if$
  ", " must.must.must
  institution "institution" check ", " "" might.ifone.must
  year "mo.dd.yy" format.date
  ")" "}" must.must.must
  newline$
  }
%    \end{macrocode}
%
% \paragraph{Cases} Law cases are all entered using the
% {\tt @CASE} entry type.  The formatting of citations
% varies from jurisdiction to jurisdiction, so the behaviour
% of citations of this type is controlled via a "jurisdiction"
% field.  Supported jurisdictions are listed in the user
% guide.  Below, the functions for each jurisdiction are
% defined first, followed by the "case" function itself.
%
%    \begin{macrocode}
FUNCTION {clear.cite.vars}
 { 
"" 'volume.var :=
"" 'number.var :=
"" 'journal.var :=
"" 'pages.var :=
"" 'year.var :=
 }
FUNCTION {case}
%    \end{macrocode}
% Like the "article" entry, the "case" entry must make some
% pretty fine decisions about how to format material
% fed to it.  As a consequence, it is complex, but similar
% to the "article" entry in many particulars.
%    \begin{macrocode}
{ cites empty.to.null "=" *
  journal empty$
  { parse.one.cite }
  { volume empty.to.null 'volume.var :=
    number empty.to.null 'number.var :=
    journal 'journal.var :=
    pages empty.to.null 'pages.var :=
    year empty.to.null 'year.var :=
  }if$
  volume.var empty$ not number.var empty$ not and
  { "\bridges{,\ p.~}{,\ }{,\ }{\ }{\ at~}{\ at~}"
    "" "" must.must.must
    newline$ }
  'skip$
  if$
 "\lexibib{case}{" cite$ "}{" must.must.must
  get.a.kinda.sort.key "" "}{}{" must.must.must
  title empty$
  {  "Decision of the " court "court" check "" must.must.must
      " (" j.format.division ")" iftwo.might.iftwo
      ", " "" "" must.must.must
      casedate "month.dd.yy" format.date "" "" must.must.must
      "}{" "" "" must.must.must
  }
  { title "}{" "" must.must.must }
  if$
  volume.var empty$ not number.var empty$ not and
  { journal.var "journal" check ", v.~" volume.var must.must.must
    ", n.~" number.var "}{" must.must.must }
  { volume.var empty$
    { "[" year.var "mo.dd.yy" format.date "] " iftwo.might.iftwo
      number.var "\ " journal.var "journal" check might.ifone.must
      "}{" "" "" must.must.must }
    { volume.var "\ " journal.var "journal" check empty.to.null
      might.ifone.must
      "}{" "" "" must.must.must
    }if$
  }if$
pages.var "short" format.pages "pages" check
"}{"
"("
must.must.must
volume.var empty$
  'skip$
  { year.var "mo.dd.yy" format.date "" "" must.must.must
  }if$
crossref empty$
{ ")}" "" "" must.must.must }
{ booktitle empty$
  { "no booktitle (name of special issue) for "
     cite$ "/" crossref * * * warning$
    ")}" "" "" must.must.must }
  { volume.var empty$ {""} {") ("} if$ booktitle "}" must.must.must}
  if$
}if$
{ duplicate$ "=" = not }
{ parse.one.cite
  "={" "" "" must.must.must
  volume.var empty$ not number.var empty$ not and
  { journal.var "journal" check ", v.~" volume.var must.must.must
    ", n.~" number.var "}{" must.must.must }
  { volume.var empty$
    { "[" year.var "mo.dd.yy" format.date "] " iftwo.might.iftwo
      number.var "\ " journal.var "journal" check might.ifone.must
      "}{" "" "" must.must.must }
    { volume.var "\ " journal.var "journal" check empty.to.null
      might.ifone.must
      "}{" "" "" must.must.must
    }if$
  }if$
pages.var "short" format.pages "pages" check
"}{"
""
must.must.must
volume.var empty$
  'skip$
  { "(" year.var "mo.dd.yy" format.date ")" must.must.must
  }if$
crossref empty$
{ "}" "" "" must.must.must }
{ booktitle empty$
  { "no booktitle (name of special issue) for "
     cite$ "/" crossref * * * warning$
    "}" "" "" must.must.must }
  { volume.var empty$ {""} {") ("} if$ booktitle "}" must.must.must}
  if$
}if$
}while$
pop$
clear.cite.vars
newline$
annote empty.to.null write$ newline$
}
%    \end{macrocode}
% The following item adds annotations; this may be eliminated by
% stripping with "noannotes".
%    \begin{macrocode}
%<*!noannotes>
% annote empty.to.null write$ newline$
%</!noannotes>
% }
%    \end{macrocode}
%
% \DescribeEnv{j.statute}
% This function applies to Japanese statutory materials.
% \changes{1994/08/03}{v0.1c}{Added support for Japanese statutes.}
%    \begin{macrocode}
FUNCTION {j.statute}
 { "\lexibib{jstatute}{" cite$ "}{" must.must.must
   get.a.kinda.sort.key "" "}{}{" must.must.must
   title "title" check empty.to.null "}{}{}{" "" must.must.must
   title empty$ {""} {"("} if$
   "" "" must.must.must
   number empty$
     'skip$
     { type empty$
       { "Law" }
       { type
       }if$
       "\ no.~" *
       number "number" check " of "
       iftwo.might.iftwo
     }
     if$
   year "yy" format.date 
   title empty$ {""} {")"} if$
   "}" must.must.must
   newline$
 }
%    \end{macrocode}
%
% \DescribeEnv{s.statute}
% This function formats a statute entry for Singapore.
%    \begin{macrocode}
FUNCTION { s.statute }
{ "\lexibib{statute}{" cite$ "}{" must.must.must
   get.a.kinda.sort.key "" "}{}{" must.must.must
  title "title" check ", No.~" number
  "endlabel" field.tag.no.combine
  number empty$
    { "\ " * }
    { "\ of " * }
    if$
  year "mo.dd.yy" format.date "}{}{}{}" must.must.must
  newline$
}
%    \end{macrocode}
%    \begin{macrocode}
FUNCTION { e.statute }
{ "\lexibib{statute}{" cite$ "}{" must.must.must
   get.a.kinda.sort.key "" "}{}{" must.must.must
  title "title" check "\ " 
  year "mo.dd.yy" format.date must.must.must
  "}{}{}{}" "" "" must.must.must
  newline$
}
%    \end{macrocode}
%
% \DescribeEnv{statute}
% This function selects the correct statute entry function.
%    \begin{macrocode}
FUNCTION { statute }
 { type build.bridges
   jurisdiction empty.to.null duplicate$
   "japan" =
   { pop$ j.statute }
   { duplicate$ "singapore" =
     { pop$ s.statute }
     { duplicate$ "england" =
       { pop$ e.statute }
       { pop$ "IMPORTANT: unknown jurisdiction for " cite$ * warning$
       }if$
     }if$
   }if$
 }
%    \end{macrocode}
%
% \DescribeEnv{unpublished}
%
% A hastily drafted function type for unpublished materials.
%
%    \begin{macrocode}
FUNCTION {unpublished}
{ "\lexibib{book}{" cite$ "}{" must.must.must
   get.a.kinda.sort.key "" "}{" must.must.must
  author "firstinitial" format.names "}{" "" must.must.must
  title "title" check "" ""  must.must.must
  "}{}{}{" "(" "" must.must.must
   note
  ")}" "" must.must.must
  newline$
}
%    \end{macrocode}
%
% \DescribeEnv{default.type}
% We use the "book" type as our default type.  When "manual" is
% completed, we should probably use that type instead.
%    \begin{macrocode}
FUNCTION {default.type} { book }
%</bstfunctions>
%    \end{macrocode}
%
% 
% \subsection{Macro definitions}
% We don't define any macros for abbreviating law journal
% names.  Instead, we will use Blue Book abbreviations
% ``native'', with a special syntax (probably the full
% form in syntax: 
% ``"\gobble{Accountant}{}"'' immediately after
% the abbreviation) for resolving
% ambiguous abbreviations.  Meanwhile, trust me: use the
% Blue Book abbreviations and take this upcoming facility
% on faith.  And besides, do you ever {\em need\/} to spell
% out journal and reporter names?
%
% \subsection{Execution}
% With all preliminaries out of the way, our first act is
% to read in the entries from {\tt *.bib}..
%    \begin{macrocode}
%<*bstfunctions>
READ
%    \end{macrocode}
% Then we say ``Hi'' to the user.  It would be nice to make this
% the first message, but the structure of \BibTeX{} style files
% dictates that it will follow any warnings about missing
% entries.
%    \begin{macrocode}
EXECUTE {hello}
%</bstfunctions>
%    \end{macrocode}
% \section{Camel style code}
%    \begin{macrocode}
%<*lawcitestyle>
\ProvidesFile{law.cst}[1995/01/08]
%    \end{macrocode}
% \subsection{Word list}
% A list of inter-words and their corresponding expansions
% must be provided.
%    \begin{macrocode}
{\catcode`\_=13%
 \catcode`\^=13%
\gdef\@law@wordlist{%
  \\{\item}{\item}%
  \\{and}{_\ ^And }%
  \\{but-see}{_\ ^But see }%
  \\{,}{_; }%
  \\{;}{_; }%
  \\{:}{_; }%
  \\{eg}{_; ^E.g.~}%
  \\{accord}{_; ^Accord }%
  \\{see}{_; ^See }%
  \\{see-also}{_; ^See also }%
  \\{cited-in}{ cited in }%
  \\{citing}{ citing }%
  \\{cf}{_; ^Cf.~}%
  \\{compare}{_; ^Compare }%
  \\{with}{ with }%
  \\{contra}{_; ^Contra }%
  \\{but-cf}{_\ ^But cf.~}%
  \\{see-generally}{_. See generally }%
  \\{affirmed}{_, ^aff'd }%
  \\{reprinted-in}{_; ^Reprinted in }%
  \@law@nomatch}
}
%    \end{macrocode}
%
% \subsection{Print routines}
% Every ".cst" file must define a "\@law@print" macro.
% This is the macro that prints the citation, both in the
% text of the document and in the bibliography.
% The toggles set by the {\sc Camel} engine allow
% a high degree of refinement in the formatting of
% citations.  For the styles currently used in
% publishing with \LaTeX{}, relatively little of this
% power is required.  In-document citation styles
% are much more demanding; if you digest the operation
% the following "\@law@print" macro, which is suitable for
% formatting legal citations, you will find the drafting
% of author-date styles and the like quite straightforward
% by comparison.
%    \begin{macrocode}
\gdef\@law@print{%
%    \end{macrocode}
% In this particular style, a long citation (for the
% bibliography, for example) and a first in-text citation
% are identical.  These forms have separate toggles, so
% we start by equating them.
%    \begin{macrocode}
  \xdef\@law@argtwolist{\the\@ltok@argtwo}%
  \if@law@firstuseofcite\@law@longcitetrue\fi%
%    \end{macrocode}
% This toggles the printing on and off.  This toggle is
% set by the "n" option to the "\source" command.
%    \begin{macrocode}
\if@law@printcite%
  \begingroup%
  \def\@law@firstslash{\begingroup\def\\{\@law@secondslash}%
   \the\ltokspecialface}%
  \def\@law@secondslash{\endgroup\def\\{\@law@firstslash}}%
  \def\\{\@law@firstslash}%
%    \end{macrocode}
% There are two halves to the macro; one for long cite forms,
% the other for short.  Long cite forms are in the first
% half.
%    \begin{macrocode}
\if@law@longcite%
%    \end{macrocode}
% Long citations are pretty straightforward; we've gathered all
% the information we needed; now we just need to plunk it
% all down in order, pretty much.
% First to print is the author field (the second argument to the
% citation declaration command), followed by the
% author-to-title punctuation bridge.  The enclosing
% braces limit the scope of the special active character
% definitions of "^", "_" and "|".
%    \begin{macrocode}
 \global\ltokspecialface=\@ltok@authoroptionface%
 {\the\@ltok@authormainface%
 \@law@barinfull\the\@ltok@author}\the\@ltok@atot%
%    \end{macrocode}
%  Next comes similar treatment for the title.
%    \begin{macrocode}
 \global\ltokspecialface=\@ltok@titleoptionface%
 {\the\@ltok@titlemainface%
 \@law@barinfull\the\@ltok@name}\the\@ltok@ttocone%
%    \end{macrocode}
%    \begin{macrocode}
  \@law@longrecurse%
%    \end{macrocode}
% This else marks the boundary between long-form citations
% (which we have seen are relatively simple to print)
% and short-form citations (which are rather complex).
% This "\else" matches the "\if@law@longcite" conditional.
%    \begin{macrocode}
\else%
%    \end{macrocode}
% The footnote number and the accompanying bridge
% should not appear in a short-form citation if
% the citation being printed first occurred in the
% current footnote.  Rather than suppressing printing,
% we just set the relevant tokens to nil.
%    \begin{macrocode}
  \ifnum\the\@ltok@pageorfootno=\the\c@law@footnote\relax%
    \xdef\@law@temp{\the\@ltok@whereitsat}%
    \xdef\@law@temptwo{\the\@ltok@infoot}%
    \ifx\@law@temp\@law@temptwo%
     \global\@ltok@whereitsat{}%
     \global\@ltok@pageorfootno{}%
    \fi%
  \fi%
%    \end{macrocode}
% Special forms of tidying-up may be appropriate to
% each of the four classes of citation.  An appropriate
% cleaning macro is called here.
%    \begin{macrocode}
\csname @law@\the\@ltok@citetype preformat\endcsname%
%    \end{macrocode}
% If it has been found that the same citation has been
% used immediately previous to this instance,
% we use {\em Id.}  If the "\@justabove" test showed
% that {\em Id.} by itself is appropriate, we prepare
% to eat any period immediately following the citation
% macro.  Otherwise, we tack on the pinpoint reference
% and its accompanying bridge.  A fail-safe conditional
% reverts to the original plan if the pinpoint reference
% is empty.
%    \begin{macrocode}
  \if@justabove%
   {\def\,{,}%
    \Id%
     \if@l@quiteexact%
      \gdef\@law@gobble{\@ifnextchar.{\@gobble}{}}%
     \else%
      \ifcat$\the\@ltok@argtwo$%
        \gdef\@law@gobble{\@ifnextchar.{\@gobble}{}}%
      \else%

        \the\@ltok@atbridge{\@law@barkill\the\@ltok@argtwo\relax}%
        \@law@fetchparas%
        \@law@shiftparas%
        \@law@shortrecurse%
      \fi%
     \fi}%
%    \end{macrocode}
% If {\em Id.} was not called for, we have to create
% a proper short-form reference.  This "\else" corresponds
% with the "\@justabove" toggle, above.
%    \begin{macrocode}
  \else%
%    \end{macrocode}
% The author will be used in any case.  Note that the author
% and its following bridge may have been set to nil, above.
%    \begin{macrocode}
   \global\ltokspecialface=\@ltok@authoroptionface%
   {\the\@ltok@authormainface%
%  print whatever there if for the author
   \@law@barinshort\the\@ltok@author}\the\@ltok@atot%
%    \end{macrocode}
% The title information is also used, if it is present.
%    \begin{macrocode}
   \global\ltokspecialface=\@ltok@titleoptionface%
   {\the\@ltok@titlemainface%
%    \end{macrocode}
% If we have specified a nickname for this cite for
% `hereinafter' references, we use it instead of the
% title of the work.
%<*parabeta>
[Some day this might be hooked up.]
%</parabeta>
%    \begin{macrocode}
   \@law@barinshort\the\@ltok@name}%
%    \end{macrocode}
% At this point, we must make decisions concerning whether
% to use the {\em supra\/} cross-referencing form.  This
% depends on the type of citation we are working on.
% The "\@nosupra" condition is set to true if we are working
% on a case or a statute, otherwise it is set to false.
%    \begin{macrocode}
   \xdef\@law@temp{\the\@ltok@citetype}%
    \ifx\@law@temp\@law@case%
     \@nosupratrue%
    \else%
      \ifx\@law@temp\@law@statute%
       \@nosupratrue%
      \else%
       \@nosuprafalse%
      \fi%
    \fi%
%    \end{macrocode}
% Now we put the result of the above test to work.
% First, the citation form for cases or statutes, which
% do not permit the {\em supra\/} form.
%    \begin{macrocode}
    \if@nosupra%
%    \end{macrocode}
%    \begin{macrocode}
     \the\@ltok@ttocone%
     \@law@shortprint%
     \@law@shortrecurse%
    \else%
     \supra\the\@ltok@whereitsat\the\@ltok@pageorfootno%
     \the\@ltok@atbridge%
     {\def\,{,}%
     {\@law@barkill\the\@ltok@argtwo\relax}}%
    \fi%
%    \end{macrocode}
% In order, these close "\if@justabove", "\if@law@longcite"
% and "\if@law@printcite".
%    \begin{macrocode}
     \xdef\@law@temp{\the\@ltok@citetype}%
      \ifx\@law@temp\@law@statute%
       \ifcat$\the\@ltok@citefirst$%
         \the\@ltok@ptoctwo{\@law@barinshort\the\@ltok@citelast}%
       \fi%
      \fi%
  \fi%
\fi%
\endgroup%
\fi%  <- end of if@law@printcite
%    \end{macrocode}
% The "\@law@gobble" here will consume a period inserted
% automatically after a forced footnote, if the printed
% form turned out to be an {\em Id.}.  See the concluding
% code of "\@law@setup".
% \changes{1994/07/26}{2.0c}{Moved {\tt\protect\string%
% \protect\@law@forcingfalse} outside of a conditional
% expression at the end of the print routine, to correct
% the failure of the forcing mechanism to work more than once.}
% \changes{1994/12/14}{2.0l}{Fairly drastic simplification
% of the forcing mechanism, in the course of providing for
% unified handling of citation strings.  Code now much
% easier to follow.}
% \changes{1996/03/09}{0.1c}{Corrected a fatal bug in the
% parallel pinpointing mechanmism.  This required changes to both
% Camel and the Law module; the list macro of pinpoints was
% being parsed in both!}
%    \begin{macrocode}
\global\@law@longcitefalse\gdef\volno{\message{(No volume %
    for \the\@ltok@nickname)}}}%
%    \end{macrocode}
% \begin{macro}{\@law@longrecurse}
% This macro prints the tail end of long citations until
% parallels are exhausted.
%    \begin{macrocode}
\def\@law@longrecurse{%
  \@law@longprint%
  \ifnum\the\c@law@parapin>0\relax%
    \loop%
      \addtocounter{law@paracounter}{1}%
    \ifnum\the\c@law@paracounter<\the\c@law@parapin\relax%
      ; \@law@longprint%
    \repeat%
  \else%
    \loop%
      \addtocounter{law@paracounter}{1}%
    \ifnum\the\c@law@paracounter>\the\c@law@paranormal\relax%
    \else%
      ; \@law@longprint%
    \repeat%
  \fi%
%    \end{macrocode}
%
% \textit{Finally, we print the last date item, which might
% be blank if it's already been printed inside "\@law@longprint".}
% \changes{1997/01/09}{v2.0m}{Fixed loss of citelast in parallel citations.}
%
%    \begin{macrocode}
  \@law@temp}
\def\@law@shortrecurse{%
  \ifnum\the\c@law@parapin>0\relax%
    \loop%
      \addtocounter{law@paracounter}{1}%
    \ifnum\the\c@law@paracounter<\the\c@law@parapin\relax%
      \@law@pincut\@ltok@argtwo\frompinlist%
      ; \@law@shortprint%
    \repeat%
  \else%
    \loop%
      \addtocounter{law@paracounter}{1}%
    \ifnum\the\c@law@paracounter>\the\c@law@paranormal\relax%
    \else%
      ; \@law@shortprint%
    \repeat%
  \fi}
%    \end{macrocode}
% \end{macro}
%
% \begin{macro}{\@law@longprint}
%    \begin{macrocode}
\def\@law@longprint{%
  \begingroup%
  \@law@fetchparas%
  \@law@tidybridges%
%    \end{macrocode}
% If for some reason either the pinpoint argument is nil,
% or it {\em and\/} the location page token register is nil,
% we set the accompanying bridges to nil. 
%    \begin{macrocode}
  \ifcat$\the\@ltok@argtwo$%
   \global\@ltok@ptop{}%
   \global\@ltok@atbridge{}%
   \ifcat$\the\@ltok@citepage$%
     \global\@ltok@conetop{}%
    \fi%
   \fi%
%    \end{macrocode}
%    \begin{macrocode}
 \global\ltokspecialface=\@ltok@citefirstoptionface%
 {\the\@ltok@citefirstmainface%
 \@law@barinfull\the\@ltok@citefirst}%
 \@law@conetopsetup%
 \the\@ltok@conetop%
%    \end{macrocode}
% Next comes the location page and its following 
% bridge.  Both of these may be blank.  No funny
% business with the shorthand active characters is
% required, since we assume this will not contain
% any special text for which they will be required.
%    \begin{macrocode}
 \the\@ltok@citepage\the\@ltok@ptop%
%    \end{macrocode}
% A special use of "\," is defined before we expand
% the optional argument stuff.
%    \begin{macrocode}
  {\def\,{,}\@law@barkill%
   \expandafter\the\@ltok@argtwo%
   \relax}%
%    \end{macrocode}
% \textit{And finally, we print the final portion of the citation.
%  We package the citation part together with its preceding bridge.
%  If it is printed, we blank it; this gives us a simple command that
%  we can put at the end of the recursive routine that invokes this
%  code, so that the tail end of a string cite will be printed if
%  necessary, and only once.}
%  \changes{1997/01/09}{v2.0m}{Modified current-and-last comparison
%   algorithm for the last portion of the cite in parallel citations.}
%
%    \begin{macrocode}
  \xdef\@law@temp{%
    \the\@ltok@ptoctwo%
    \noexpand\@law@barinfull%
    \the\@ltok@citelast}%
  \xdef\@law@temptwo{%
    \the\@ltok@ptoctwo%
    \noexpand\@law@barinfull%
    \the\@ltok@usercitelast}%
  \ifx\@law@temp\@law@temptwo%
  \else%
    {\@law@temp}%
    \gdef\@law@temp{}%
  \fi%
    \@law@shiftparas%
  \endgroup}%
%    \end{macrocode}
% \end{macro}
%
% \begin{macro}{\@law@shortprint}
%    \begin{macrocode}
\def\@law@shortprint{%
% If {\sc Camel} sees more than one pinpoint page or section,
% the bridges preceding
% the page references must be set to their plural form.
% This change applies to both long and short form citations.
% It has to be cloned in each, however, because the decision has
% to be taken with respect to each cite in a string of parallels.
%    \begin{macrocode}
  \begingroup%
  \@law@fetchparas%
  \@law@tidybridges%
%    \end{macrocode}
% The actual recursive printing works.
%    \begin{macrocode}
     \global\ltokspecialface=\@ltok@citefirstoptionface%
     {\the\@ltok@citefirstmainface%
     \@law@barinshort\the\@ltok@citefirst}%
%    \end{macrocode}
% If the pinpoint reference is empty, we tack on the
% location page after the appropriate bridge.  Otherwise,
% we do nothing for the present; any pinpoint reference
% will be produced later on.
%  Note that the bridge
% used here will only be in plural form if something was
% given for use as a pinpoint reference.
% NB: THIS IS A TEST!!!! ******** BRACKETS TO BE PULLED!!!!
%    \begin{macrocode}
     \@law@conetopsetup%
     \the\@ltok@conetop%
     \ifcat$\the\@ltok@argtwo$%
       \the\@ltok@citepage%
     \else%
       {\@law@barkill\the\@ltok@argtwo}%
     \fi%
    \@law@shiftparas%
  \endgroup}%
%    \end{macrocode}
% \end{macro}
%
% \paragraph{Print format subroutines}
% \begin{macro}{\@law@casepreformat}
% This is empty; no special preparations are necessary
% for the printing of a case citation.
%    \begin{macrocode}
\gdef\@law@casepreformat{}%
%    \end{macrocode}
% \end{macro}
% \begin{macro}{\@law@statutepreformat}
% If short statutory references are in force,
% the title and the following bridge are set to nil,
% if some reference is provided in the first citation
% part.
%    \begin{macrocode}
\gdef\@law@statutepreformat{%
\if@law@statuteverbose%
\else%
  \ifcat$\the\@ltok@citefirst$%
  \else%
   \global\@ltok@name{}%
   \global\@ltok@ttocone{}%
  \fi% 
\fi%
}%
%    \end{macrocode}
% \end{macro}
% \begin{macro}{\@law@articlepreformat}
% If a work by the identical author has been cited more than
% once, we leave
% everything intact (the print routine does the necessary
% culling);
% otherwise, we cut the title here.  The use of
% "\@law@authortracing"
% is explained above.
%    \begin{macrocode}
\gdef\@law@articlepreformat{%
\ifcat$\the\@ltok@author$%
\else%
{\@law@clean\@ltok@author\@law@authortracing%
\expandafter\expandafter\expandafter\if\expandafter%
\csname\@law@authortracing\endcsname2%
 \else%
  \global\@ltok@atot{}\global\@ltok@name{}%
\fi}\fi}%
%    \end{macrocode}
% \end{macro}
% \begin{macro}{\@law@bookpreformat}
% This provides the same treatment given to articles.
%    \begin{macrocode}
\gdef\@law@bookpreformat{%
\ifcat$\the\@ltok@author$%
\else%
{\@law@clean\@ltok@author\@law@authortracing%
\expandafter\expandafter\expandafter\if\expandafter%
\csname\@law@authortracing\endcsname2%
 \else%
  \global\@ltok@atot{}\global\@ltok@name{}%
 \fi}\fi}%
%</lawcitestyle>
%    \end{macrocode}
% \end{macro}
%
% \subsection{Camel citation formats}
% The style definitions are stored in a separate file,
% to make it easier and less risky for users to play with
% the styles to produce desired output.  Table 3
% provides a guide to the macro arguments and their functions.
% \subsubsection{Books} The styles used for citing books and
% book-like things follow.
%    \begin{macrocode}
%<*lawcite>
\ProvidesFile{law.cit}[1994/12/07]
\newcitestyle{book}%
 {srsrrrB}
 {[a],\ [t][c]\ [p](pl)\ [rp]\ [e]:[id]\ at~(pl)\ at~[xrf]}
%
\newcitestyle{booklet}%
 {riririB}
 {[a],\ [t][c]\ [p](pl)\ [rp]\ [e]:[id]\ at~(pl)\ at~[xrf]}
%
\newcitestyle{techreport}%
 {riririB}
 {[a],\ [t][c]\ [p](pl)\ [rp]\ [e]:[id]\ at~(pl)\ at~[xrf]}
%
\newcitestyle{mastersthesis}%
 {riririB}
 {[a],\ [t][c]\ [p](pl)\ [rp]\ [e]:[id]\ at~(pl)\ at~[xrf]}
%    \end{macrocode}
% \subsubsection{Articles} Styles used for articles 
% and similar shortish things follow.  They're all
% the same, but the clones help keep things clear when following
% the action between \LaTeX\ and \BibTeX.
%    \begin{macrocode}
\newcitestyle{article}%
 {rsirsrA}
 {[a],\ [t],\ [c]\ [p],\ (pl),\ [rp]\ [e]:[id]\ at~(pl)\
at~[xrf]}
\newcitestyle{incollection}%
 {rsirsrA}
 {[a],\ [t],\ [c]\ [p],\ (pl),\ [rp]\ [e]:[id]\ at~(pl)\
at~[xrf]}
\newcitestyle{inbook}%
 {srirsrA}
 {[a],\ [t],\ [c]\ [p],\ (pl),\ [rp]\ [e]:[id]\ at~(pl)\
at~[xrf]}
%    \end{macrocode}
% \subsubsection{Cases} There are several styles for cases; we pretty
% much need a separate style for each jurisdiction.
%    \begin{macrocode}
\newcitestyle{case}%
 {rrirrsC}
 {[a][t],\ [c]\ [p],\ (pl),\ [rp]\ [e]:[id]\ at~(pl)\ at~[xrf]}
%    \end{macrocode}
% \subsubsection{Statutes} There are several of these.  More may
% need to be added on an {\em ad hoc\/} basis.
%    \begin{macrocode}
\newcitestyle{statute}%
 {rrrsrsS}
 {[a][t],\ [c]\ \S~[p](pl)\ \S\S~[rp]\ [e]:[id]\ \S~(pl)\
\S\S~[xrf]}
%
\newcitestyle{jstatute}%
 {rrrsrsS}
 {[a][t][c]\ \S~[p](pl)\ \S\S~[rp]\ [e]:[id]\ \S~(pl)\
\S\S~[xrf]}
%</lawcite>
%    \end{macrocode}
%
% \section{Extraction utilities}
% \subsection{The Driver}
% Here is a simple driver for extracting the files in the
% package.
% \changes{1997/01/09}{v2.0m}{Brought installation file up to
%   date for use with the new docstrip utility.}
%
%    \begin{macrocode}
%<*installer>
\def\batchfile{law.ins}
\input docstrip.tex
\preamble

Copyright (C) 1992--97 Frank Bennett, Jr.
All rights reserved.

This file is part of the Law module for the Camel package.
\endpreamble

\def\batchfile{camel.dst}      % ignored in distribution
\input docstrip.tex              % ignored in distribution

\keepsilent

\preamble
This file is part of the Law module of the Camel package.
---------------------------------------------------------
This is a generated file.  
IMPORTANT NOTICE:

You are allowed to change this file, subject to the following
conditions.  Under any circumstances, new macro definitions
should not be added to this file.  You are welcome to modify
the macro definitions contained in this file for your own
use.  If you pass a copy of the modified version to someone
else, you should (a) let me know about the change on
fb@soas.ac.uk, and (b) put a note of the changes and of your
own contact details in the file.  Furthermore, you must
acknowledge Camel and its author(s) in the new file (if it
is distributed to others), and you must attach these same
conditions to the new file.

You are not allowed to distribute this file alone.  You are not
allowed to take money for the distribution or use of this file
(or a changed version) except for a nominal charge for copying
etc.

You are allowed to distribute this file under the condition that
it is distributed with all of its contents, intact.

For error reports, or offers to help make this a more powerful,
friendlier, and better package, please contact me on
`fb' at soas.ac.uk

\endpreamble


\generate{\file{law.cst} {\from{law.dtx}{lawcitestyle}}
          \file{law.cit} {\from{law.dtx}{lawcite}}}


\postamble
\endpostamble

%<installer>% Change eng to jse below for JBibTeX support

\generate{\file{law.bst} {\from{camel.dtx}{bstheader,eng}
                          \from{law.dtx}{bstheader,eng}
                          \from{camel.dtx}{bstlibrary,eng}
                          \from{law.dtx}{bstfunctions,eng}
                          \from{camel.dtx}{bsttrailer,eng}}}

\keepsilent


\ifToplevel{
\Msg{***********************************************************}
\Msg{*}
\Msg{* To finish the installation you have to move the following}
\Msg{* file into a directory searched by TeX:}
\Msg{*}
\Msg{* \space\space law.cst}
\Msg{* \space\space law.cit}
\Msg{*}
\Msg{* You should also move the following file into a directory}
\Msg{* searched for style files by BibTeX:}
\Msg{*}
\Msg{* \space\space law.bst}
\Msg{*}
\Msg{* Other style modules can be found on CTAN in the `styles'}
\Msg{* subdirectory below Camel itself.}
\Msg{*}
\Msg{***********************************************************}
}
%</installer>
%    \end{macrocode}
% \Finale \PrintIndex \PrintChanges

% \subsubsection{The style files of the Faculty of Economics and
%   Administration}
% % \file{style/mu/fithesis-econ.sty}
% This is the style file for the theses written at the Faculty of
% Economics and Administration at the Masaryk University in Brno.
% It has been prepared in accordance with the formal requirements
% \changes{v0.3.46}{2017/06/02}{The documentation now points to the
%   2/2017 dean's directive for the Faculty of Economics and
%   Administration, Masaryk University, Brno. [VN]}
% published at the website of the faculty\footnote{See \url{ht^^A
% tps://is.muni.cz/auth/do/econ/predpisy/smernice/prehled/6715^^A
% 9928/SmerniceDekana2017-c.2-o_zaverecnych_pracich_2017.pdf}}.
%    \begin{macrocode}
\NeedsTeXFormat{LaTeX2e}
\ProvidesPackage{fithesis/style/mu/fithesis-econ}[2017/05/21]
%    \end{macrocode}
% The file defines the color scheme of the respective faculty. Note
% the the color definitions are in RGB, which makes the resulting
% files generally unsuitable for printing.
%    \begin{macrocode}
\thesis@color@setup{
  links={HTML}{F27995},
  tableEmph={HTML}{E8B88B},
  tableOdd={HTML}{F5ECEB},
  tableEven={HTML}{EBD8D5}}
%    \end{macrocode}
% The bibliography support is enabled. The |authoryear| citations
% are used and the bibliography is sorted by name, title, and year.
%    \begin{macrocode}
\thesis@bibliography@setup{
  style=iso-authoryear,
  sorting=nty}
\thesis@bibliography@load
%    \end{macrocode}
% The file loads the following packages:
% \begin{itemize}
%   \item\textsf{tikz} -- Used for dimension arithmetic.
%   \item\textsf{geometry} -- Allows for modifications of the type
%     area dimensions.
%   \item\textsf{array} -- Enables |<{decl.}| and |>{decl.}|
%     declarations in table preambles.
% \end{itemize}
% In addition to this, the type area width is set to
% 16\,cm in accordance with the formal requirements of the faculty.
% This leads to overfull lines and is against the good conscience
% of the author of this style.
%    \begin{macrocode}
\thesis@require{tikz}
\thesis@require{geometry}
\thesis@require{array}
\geometry{top=25mm,bottom=20mm,left=25mm,right=25mm,includeheadfoot}
%    \end{macrocode}
% \begin{macro}{\thesis@blocks@cover}
% The |\thesis@blocks@cover| macro typesets the thesis
% cover.
%    \begin{macrocode}
\def\thesis@blocks@cover{%
  \ifthesis@cover@
    \thesis@blocks@clear
    \begin{alwayssingle}
      \thispagestyle{empty}
      \begin{center}
      {\sc\thesis@titlePage@LARGE\thesis@@{universityName}\\%
          \thesis@titlePage@Large\thesis@@{facultyName}\\}
      \vfill
      {\bf\thesis@titlePage@Huge\thesis@@{typeName}}
      \vfill
      {\thesis@titlePage@large\thesis@place
       \ \thesis@year\hfill\thesis@author}
      \end{center}
    \end{alwayssingle}
  \fi}
%    \end{macrocode}
% \end{macro}
% The style file configures the title page header to include the
% name of the field of study and redefines the title page content
% not to include the author's name and the title page footer
% to include both the author's and advisor's name, the year and
% place of the thesis defense in accordance with the formal
% requirements of the faculty.
%    \begin{macrocode}
\thesis@blocks@titlePage@field@true
\def\thesis@blocks@titlePage@content{%
  {\thesis@titlePage@Huge\bf\thesis@TeXtitle}
  \ifthesis@english\else
    {\\[0.1in]\thesis@titlePage@Large\bf\thesis@TeXtitleEn}
  \fi {\\[0.3in]\thesis@titlePage@large\sc\thesis@@{typeName}\\}}
\def\thesis@blocks@titlePage@footer{%
  {\thesis@titlePage@large
    {% Calculate the width of the thesis author and advisor boxes
     \let\@A\relax\newlength{\@A}\settowidth{\@A}{{%
       \bf\thesis@@{advisorTitle}:}}
     \let\@B\relax\newlength{\@B}\settowidth{\@B}{\thesis@advisor}
     \let\@C\relax\newlength{\@C}\settowidth{\@C}{{%
       \bf\thesis@@{authorTitle}:}}
     \let\@D\relax\newlength{\@D}\settowidth{\@D}{\thesis@author}
    \let\@left\relax\newlength{\@left}\pgfmathsetlength{\@left}{%
      max(\@A,\@B)}
    \let\@right\relax\newlength{\@right}\pgfmathsetlength{\@right}{%
      max(\@C,\@D)}
    % Typeset the thesis author and advisor boxes
    \vskip 2in\begin{minipage}[t]{\@left}
      {\bf\thesis@@{advisorTitle}:}\\\thesis@advisor
    \end{minipage}\hfill\begin{minipage}[t]{\@right}
      {\bf\thesis@@{authorTitle}:}\\\thesis@author
    \end{minipage}}\\[4em]\thesis@place, \thesis@year}}
%    \end{macrocode}
% \begin{macro}{\thesis@blocks@frontMatter}
% The |\thesis@blocks@frontMatter| macro sets up the style
% of the front matter of the thesis. The page numbering is arabic
% as per the formal requirements and it is hidden. In case of
% double-sided typesetting, the geometry is altered according to
% the requirements of the faculty.
% \begin{macrocode}
\def\thesis@blocks@frontMatter{%
  \thesis@blocks@clear
  % In case of double-sided typesetting, change the geometry
  \ifthesis@twoside@
    \newgeometry{top=25mm,bottom=20mm,left=35mm,
      right=15mm, includeheadfoot}
  \fi\pagestyle{empty}
  \parindent 1.5em
  \setcounter{page}{1}
  \pagenumbering{arabic}}
%    \end{macrocode}
% \end{macro}\begin{macro}{\thesis@blocks@mainMatter}
% The |\thesis@blocks@mainMatter| macro sets up the style
% of the main matter of the thesis. The page numbering doesn't
% reset at the beginning of the main thesis as per the formal
% requirements.
% \begin{macrocode}
\def\thesis@blocks@mainMatter{%
  \thesis@blocks@clear
  % In case of double-sided typesetting, change the geometry
  \ifthesis@twoside@
    \newgeometry{top=25mm,bottom=20mm,left=35mm,
      right=15mm, includeheadfoot}
  \fi\pagestyle{thesisheadings}
  \parindent 1.5em\relax}
%    \end{macrocode}
% \end{macro}\begin{macro}{\thesis@blocks@tables}
% The |\thesis@blocks@tables| macro optionally typesets the
% |\listoftables| and |\listoffigures|.
% \begin{macrocode}
\def\thesis@blocks@tables{%
  \thesis@blocks@lot
  \thesis@blocks@lof}
%    \end{macrocode}
% \end{macro}
% If the |nolot| and |nolof| options haven't been specified, the
% |\thesis@blocks@lot| and |\thesis@blocks@lof| macros are
% redefined to create an entry in the table of contents.
% \begin{macrocode}
\ifx\thesis@blocks@lot\relax\else
  \def\thesis@blocks@lot{%
    \thesis@blocks@clear
    \phantomsection
    \addcontentsline{toc}{chapter}{\listtablename}%
    \listoftables}
\fi

\ifx\thesis@blocks@lof\relax\else
  \def\thesis@blocks@lof{%
    \thesis@blocks@clear
    \phantomsection
    \addcontentsline{toc}{chapter}{\listfigurename}%
    \listoffigures}
\fi
%    \end{macrocode}
% \begin{macro}{\thesis@blocks@declaration}
% The |\thesis@blocks@declaration| macro typesets the declaration
% text. Unlike the generic |\thesis@blocks@declaration| macro from
% the \texttt{style/mu/fithesis-sci.sty} file, this definition
% includes the date and a blank line for the author's signature, as
% per the requirements of the faculty.
% \changes{v0.3.46}{2017/06/02}{Redefined
%   \cs{thesis@blocks@declaration} in
%   \texttt{style/mu/fithesis-econ.sty} in accordance with the
%   example documents. The patch was submitted by Jana Ratajská.
%   [VN]}
%    \begin{macrocode}
\def\thesis@blocks@declaration{%
  \begin{alwayssingle}%
    \thesis@blocks@clear
    \leavevmode\vfill
    % Start the new chapter without clearing any page.
    {\let\thesis@blocks@clear\relax
    \chapter*{\thesis@@{declarationTitle}}}%
    \thesis@declaration
    \vskip 2cm%
    {\let\@A\relax\newlength{\@A}
      \settowidth{\@A}{\thesis@@{authorSignature}}
      \setlength{\@A}{\@A+1cm}
    \noindent\thesis@place, \thesis@@{formattedDate}\hfill
    \begin{minipage}[t]{\@A}%
      \centering\rule{\@A}{1pt}\\
      \thesis@@{authorSignature}\par
    \end{minipage}}
  \end{alwayssingle}}
%    \end{macrocode}
% \end{macro}\begin{macro}{\thesis@blocks@abstract}
% \changes{v0.3.46}{2017/06/02}{Redefined
%   \cs{thesis@blocks@abstract}, \cs{thesis@blocks@abstractEn},
%   \cs{thesis@blocks@keywords}, and \cs{thesis@blocks@keywordsEn}
%   in \texttt{style/mu/fithesis-econ.sty} in accordance with the
%   example documents. The patch was submitted by Jana Ratajská.
%   [VN]}
% The |\thesis@blocks@abstract| macro typesets the
% abstract. This definition typesets the abstract on the same page.
% \begin{macrocode}
\def\thesis@blocks@abstract{%
  \begin{alwayssingle}%
    \vskip 40\p@
    {\let\thesis@blocks@clear\relax
    \chapter*{\thesis@@{abstractTitle}}}%
    \noindent\thesis@abstract
  \end{alwayssingle}}
%    \end{macrocode}
% \end{macro}\begin{macro}{\thesis@blocks@abstractEn}
% The |\thesis@blocks@abstractEn| macro typesets the abstract in
% English. If the current locale is English, the macro produces no
% output. This macro typesets the abstract on the same page.
% \begin{macrocode}
\def\thesis@blocks@abstractEn{%
  \ifthesis@english\else
    {\thesis@selectLocale{english}%
    \begin{alwayssingle}%
      \vskip 20\p@
      {\let\thesis@blocks@clear\relax
      \chapter*{\thesis@english@abstractTitle}}%
      \noindent\thesis@abstractEn
    \end{alwayssingle}}%
  \fi}
%    \end{macrocode}
% \end{macro}\begin{macro}{\thesis@blocks@keywords}
% The |\thesis@blocks@keywords| macro typesets the keywords. This
% definition typesets the keywords on the same page.
% \begin{macrocode}
\def\thesis@blocks@keywords{%
  \begin{alwayssingle}%
    \vskip 40\p@
    {\let\thesis@blocks@clear\relax
    \chapter*{\thesis@@{keywordsTitle}}%
    \noindent\thesis@TeXkeywords}%
  \end{alwayssingle}}
%    \end{macrocode}
% \end{macro}\begin{macro}{\thesis@blocks@keywordsEn}
% The |\thesis@blocks@keywordsEn| macro typesets the keywords in
% English. If the current locale is English, the macro produces no
% output.
% \begin{macrocode}
\def\thesis@blocks@keywordsEn{%
  \ifthesis@english\else
    {\thesis@selectLocale{english}%
    \begin{alwayssingle}%
      \vskip 20\p@
      {\let\thesis@blocks@clear\relax%
      \chapter*{\thesis@english@keywordsTitle}}%
      \noindent\thesis@TeXkeywordsEn
    \end{alwayssingle}}%
  \fi}
%    \end{macrocode}
% \end{macro}\begin{macro}{\thesis@blocks@bibEntry}
% The |\thesis@blocks@bibEntry| macro typesets a bibliographical
% entry. Along with the macros required by the locale file
% interface, the locale files need to define the following macros:
% \begin{itemize}
%   \item|\thesis@|\textit{locale}|@bib@author| -- The label of the
%     author name entry
%   \item|\thesis@|\textit{locale}|@bib@title| -- The label of the
%     title name entry
%   \item|\thesis@|\textit{locale}|@bib@titleEn| -- The label of the
%     English title name entry (\cs{thesis@english@bib@titleEn}
%     does not need to be defined)
%   \item|\thesis@|\textit{locale}|@bib@department| -- The label of
%     the department name entry
%   \item|\thesis@|\textit{locale}|@bib@advisor| -- The label of
%     the advisor name entry
%   \item|\thesis@|\textit{locale}|@bib@year| -- The label of the
%     year entry
% \end{itemize}
% \changes{v0.3.46}{2017/06/02}{Defined \cs{thesis@blocks@bibEntry}
%   in \texttt{style/mu/fithesis-econ.sty} in accordance with the
%   example documents. The patch was submitted by Jana Ratajská.
%   [VN]}
%    \begin{macrocode}
\def\thesis@blocks@bibEntry{%
  \thesis@blocks@clear
  \noindent\begin{thesis@newtable@old}{@{}>{\bfseries}ll@{}}
    \thesis@@{bib@author}:        & \thesis@author     \\
    \thesis@@{bib@thesisTitle}:   & \thesis@title      \\
  \ifthesis@english\else
    \thesis@@{bib@thesisTitleEn}: & \thesis@titleEn    \\
  \fi
    \thesis@@{bib@department}:    & \thesis@department \\
    \thesis@@{bib@advisor}:       & \thesis@advisor    \\
    \thesis@@{bib@year}:          & \thesis@year       \\
  \end{thesis@newtable@old}}
%    \end{macrocode}
% \end{macro}
% Note that there is no direct support for the seminar paper and
% thesis proposal types.  If you would like to change the contents
% of the preamble and the postamble, you should modify the
% |\thesis@blocks@preamble| and |\thesis@blocks@postamble| macros.
%
% All blocks within the autolayout preamble that are not defined
% within this file are defined in the
% \texttt{style/mu/fithesis-base.sty} file.
%    \begin{macrocode}
\def\thesis@blocks@preamble{%
  \thesis@blocks@coverMatter
    \thesis@blocks@cover
  \thesis@blocks@frontMatter
    \thesis@blocks@titlePage
    \thesis@blocks@assignment
    \thesis@blocks@bibEntry
    \thesis@blocks@abstract
    \thesis@blocks@abstractEn
    \thesis@blocks@keywords
    \thesis@blocks@keywordsEn
    \thesis@blocks@declaration
    \thesis@blocks@thanks
    \thesis@blocks@toc}
%    \end{macrocode}
% All blocks within the autolayout postamble that are not defined
% within this file are defined in the \texttt{style/mu/base.sty}
% file.
%    \begin{macrocode}
\def\thesis@blocks@postamble{%
  \thesis@blocks@bibliography
  \thesis@blocks@tables}
%    \end{macrocode}

% \subsubsection{The style files of the Faculty of Medicine}
% % \file{theme/mu/beamercolorthemefibeamer-med.sty}
% This is the color theme for presentations written at the Faculty
% of Medicine at the Masaryk University in Brno. This theme has no
% effect outside the presentation mode.
%    \begin{macrocode}
\NeedsTeXFormat{LaTeX2e}
\ProvidesPackage{fibeamer/theme/mu/%
  beamercolorthemefibeamer-mu-med}[2016/05/06]
\mode<presentation>
%    \end{macrocode}
% This color theme uses the combination of yellow and shades of gray.  The
% |fibeamer@{dark,|\-|light}@background{Inner,|\-|Outer}| colors are used
% within the background canvas template, which is defined within the base
% color theme of the Masaryk University and which draws the gradient
% background of the frames.
%    \begin{macrocode}
  \definecolor{fibeamer@black}{HTML}{000000}
  \definecolor{fibeamer@white}{HTML}{FFFFFF}
  \definecolor{fibeamer@red}{HTML}{c82600}
  \colorlet{fibeamer@lightRed}{fibeamer@red!30!fibeamer@white}
  \colorlet{fibeamer@darkRed}{fibeamer@red!60!fibeamer@black}
  \definecolor{fibeamer@gray}{HTML}{999999}
  \definecolor{fibeamer@lightOrange}{HTML}{FFA25E}
  \colorlet{fibeamer@orange}{fibeamer@lightOrange!80!fibeamer@darkRed}
%    \end{macrocode}
% \changes{v1.1.4:2}{2016/05/06}{Removed gradient backgrounds from
%   the color themes of the Masaryk University in Brno. [VN]}
%    \begin{macrocode}
  %% Background gradients
  \colorlet{fibeamer@dark@backgroundInner}{fibeamer@darkRed}
  \colorlet{fibeamer@dark@backgroundOuter}{fibeamer@darkRed}
  \colorlet{fibeamer@light@backgroundInner}{fibeamer@white}
  \colorlet{fibeamer@light@backgroundOuter}{fibeamer@white}
%    \end{macrocode}
% The |darkframes| environment switches the |\iffibeamer@darktrue|
% conditional on and sets a dark color theme.
%    \begin{macrocode}
  \renewenvironment{darkframes}{%
    \begingroup
      \fibeamer@darktrue
      %% Structures
      \setbeamercolor*{frametitle}{fg=fibeamer@lightRed}
      \setbeamercolor*{framesubtitle}{fg=fibeamer@white}
      %% Text
      \setbeamercolor*{normal text}{fg=fibeamer@white, bg=fibeamer@red}
      \setbeamercolor*{structure}{fg=fibeamer@white, bg=fibeamer@red}
%    \end{macrocode}
% \changes{v1.1.0:7}{2016/01/12}{Added support for \cs{alert} to
%   the themes of the Masaryk University in Brno. [VN]}
% \changes{v1.1.4:5}{2016/05/06}{Unified the alert colors in the
%   color themes of the Masaryk University in Brno. [VN]}
%    \begin{macrocode}
      \setbeamercolor*{alerted text}{fg=fibeamer@lightOrange}
%    \end{macrocode}
% \changes{v1.1.4:3}{2016/05/06}{Added proper link coloring for the
%   color themes of the Masaryk University in Brno. [VN]}
% \changes{v1.1.6}{2017/04/23}{Added proper citation coloring for the
%   color themes of the Masaryk University in Brno. [VN]}
%    \begin{macrocode}
      %% Items, footnotes and links
      \setbeamercolor*{item}{fg=fibeamer@lightRed}
      \setbeamercolor*{footnote mark}{fg=fibeamer@lightRed}
      \hypersetup{urlcolor=fibeamer@lightRed, citecolor=fibeamer@lightRed}
      %% Blocks
      \setbeamercolor*{block title}{%
        fg=fibeamer@white, bg=fibeamer@red!60!fibeamer@white}
      \setbeamercolor*{block title example}{%
        fg=fibeamer@white, bg=fibeamer@red!60!fibeamer@white}
      \setbeamercolor*{block title alerted}{%
        fg=fibeamer@darkRed, bg=fibeamer@lightOrange}
      \setbeamercolor*{block body}{%
        fg=fibeamer@red, 
        bg=fibeamer@gray!15!fibeamer@white}
      \usebeamercolor*{normal text}
      % Code listings
      \lstset{%
        commentstyle=\color{green!30!fibeamer@white},
        keywordstyle=\color{blue!30!fibeamer@white},
        stringstyle=\color{fibeamer@lightRed}}
      }{%
    \endgroup}
%    \end{macrocode}
% Outside the |darkframes| environment, the light theme is used.
%    \begin{macrocode}
  %% Structures
  \setbeamercolor{frametitle}{fg=fibeamer@red}
  \setbeamercolor{framesubtitle}{fg=fibeamer@black!75!fibeamer@white}
  %% Text
  \setbeamercolor{normal text}{fg=fibeamer@black, bg=fibeamer@white}
  \setbeamercolor{structure}{fg=fibeamer@black, bg=fibeamer@white}
%    \end{macrocode}
% \changes{v1.1.0:7}{2016/01/12}{Added support for \cs{alert} to
%   the themes of the Masaryk University in Brno. [VN]}
% \changes{v1.1.4:5}{2016/05/06}{Unified the alert colors in the
%   color themes of the Masaryk University in Brno. [VN]}
%    \begin{macrocode}
  \setbeamercolor{alerted text}{fg=orange}
  \addtobeamertemplate{block begin}{%
    \iffibeamer@dark % alerted text in plain block at dark slides
      \setbeamercolor{alerted text}{fg=fibeamer@orange}%
    \fi}{}
%    \end{macrocode}
% \changes{v1.1.4:3}{2016/05/06}{Added proper link coloring for the
%   color themes of the Masaryk University in Brno. [VN]}
% \changes{v1.1.6}{2017/04/23}{Added proper citation coloring for the
%   color themes of the Masaryk University in Brno. [VN]}
%    \begin{macrocode}
  %% Items, footnotes and links
  \setbeamercolor*{item}{fg=fibeamer@red}
  \setbeamercolor*{footnote mark}{fg=fibeamer@red}
  \hypersetup{urlcolor=fibeamer@red, citecolor=fibeamer@red}
  %% Blocks
  \setbeamercolor{block title}{%
    fg=fibeamer@white, bg=fibeamer@red}
  \setbeamercolor{block title example}{%
    fg=fibeamer@white, bg=fibeamer@red}
  \setbeamercolor{block title alerted}{%
    fg=fibeamer@white, bg=orange}
  \setbeamercolor{block body}{%
    fg=fibeamer@red!90!fibeamer@black, 
    bg=fibeamer@gray!20!fibeamer@white}
  %% Title
  \setbeamercolor{title}{fg=fibeamer@white, bg=fibeamer@red}
  % Code listings
  \lstset{%
    basicstyle=\footnotesize\ttfamily,
    breakatwhitespace=false,
    breaklines=true,
    commentstyle=\color{green!60!fibeamer@black},
    extendedchars=true,
    keywordstyle=\color{blue},
    showspaces=false,
    showstringspaces=false,
    showtabs=false,
    stringstyle=\color{violet}}
\mode
<all>
%    \end{macrocode}

% \subsubsection{The style files of the Faculty of Sports Studies}
% % \file{style/mu/fithesis-fsps.sty}
% This is the style file for the theses written at the Faculty of
% Sports Studies at the Masaryk University in Brno. It has been
% prepared in accordance with the formal requirements published at
% the website of the faculty\footnote{See \url{https://is.muni.cz/^^A
% auth/do/fsps/studijni/info-stud/SZZ/pokyny_ZP_13-5-2013.pdf}}.
%    \begin{macrocode}
\NeedsTeXFormat{LaTeX2e}
\ProvidesPackage{fithesis/style/mu/fithesis-fsps}[2017/05/21]
%    \end{macrocode}
% The file defines the color scheme of the respective faculty. Note
% the the color definitions are in RGB, which makes the resulting
% files generally unsuitable for printing.
%    \begin{macrocode}
\thesis@color@setup{
  links={HTML}{93BCF5},
  tableEmph={HTML}{A8BDE3},
  tableOdd={HTML}{EBEFF5},
  tableEven={HTML}{D1DAEB}}
%    \end{macrocode}
% The bibliography support is enabled. The |authoryear| citations
% are used and the bibliography is sorted by name, title, and year.
%    \begin{macrocode}
\thesis@bibliography@setup{
  style=iso-authoryear,
  sorting=nty}
\thesis@bibliography@load
%    \end{macrocode}
% The file loads the following packages:
% \begin{itemize}
%   \item\textsf{tikz} -- Used for dimension arithmetic.
%   \item\textsf{geometry} -- Allows for modifications of the type
%     area dimensions.
%   \item\textsf{setspace} -- Allows for line height modifications.
% \end{itemize}
% In addition to this, the type area width is set to
% 14\,cm in accordance with the formal requirements of the faculty.
%    \begin{macrocode}
\thesis@require{tikz}
\thesis@require{geometry}
\thesis@require{setspace}
\geometry{top=30mm,bottom=30mm,left=40mm,right=30mm,includeheadfoot}
%    \end{macrocode}
% The paragraph indentation is 1.25\,cm as per the requirements of the faculty.
%    \begin{macrocode}
\setlength{\parindent}{1.25cm}
%    \end{macrocode}
% The style file redefines the title page content
% not to include the author's name and the title page footer
% to include both the author's and advisor's name, the year and
% place of the thesis defense in accordance with the formal
% requirements of the faculty.
%    \begin{macrocode}
\def\thesis@blocks@titlePage@footer{%
  {\thesis@titlePage@large
    {% Calculate the width of the thesis author and advisor boxes
     \let\@A\relax\newlength{\@A}\settowidth{\@A}{{%
       \bf\thesis@@{advisorTitle}:}}
     \let\@B\relax\newlength{\@B}\settowidth{\@B}{\thesis@advisor}
     \let\@C\relax\newlength{\@C}\settowidth{\@C}{{%
       \bf\thesis@@{authorTitle}:}}
     \let\@D\relax\newlength{\@D}\settowidth{\@D}{\thesis@author}
     \let\@E\relax\newlength{\@E}\settowidth{\@E}{\thesis@field}
     \let\@F\relax\newlength{\@F}\pgfmathsetlength{\@F}{max(\@D,\@E)}
    \let\@left\relax\newlength{\@left}\pgfmathsetlength{\@left}{%
      max(\@A,\@B)}
    \let\@right\relax\newlength{\@right}\pgfmathsetlength{\@right}{%
      max(\@C,\@F)}
    % Typeset the thesis author and advisor boxes
    \vskip 2in\begin{minipage}[t]{\@left}
      {\bf\thesis@@{advisorTitle}:}\\\thesis@advisor
    \end{minipage}\hfill\begin{minipage}[t]{\@right}
      {\bf\thesis@@{authorTitle}:}\\\thesis@author\\\thesis@field
    \end{minipage}}\\[4em]\thesis@place, \thesis@year}}
%    \end{macrocode}
% \begin{macro}{\thesis@blocks@frontMatter}
% The |\thesis@blocks@frontMatter| macro sets up the style of the
% front matter of the thesis. The leading is adjusted in
% accordance with the requirements of the faculty.
% \begin{macrocode}
\def\thesis@blocks@frontMatter{%
  \thesis@blocks@clear
  \pagestyle{plain}
  \parindent 1.5em
  \setcounter{page}{1}
  \pagenumbering{roman}
  \onehalfspacing}
%    \end{macrocode}
% \end{macro}\begin{macro}{\thesis@blocks@mainMatter}
% The |\thesis@blocks@mainMatter| macro sets up the style
% of the main matter of the thesis. The leading is adjusted in
% accordance with the requirements of the faculty.
% \begin{macrocode}
\def\thesis@blocks@mainMatter{%
  \thesis@blocks@clear
  \setcounter{page}{1}
  \pagenumbering{arabic}
  \pagestyle{thesisheadings}
  \parindent 1.5em
  \onehalfspacing}
%    \end{macrocode}
% \end{macro}\begin{macro}{\thesis@blocks@bibliography}
% The |\thesis@blocks@bibliography| macro typesets the
% bibliography. The leading is adjusted in accordance
% with the requirements of the faculty.
% \begin{macrocode}
\def\thesis@blocks@bibliography{%
  \ifthesis@bibliography@loaded@
    \ifthesis@bibliography@included@\else
      \singlespacing
      \thesis@blocks@clear
      {\emergencystretch=3em%
      \printbibliography[heading=bibintoc]}%
    \fi
  \fi}
%    \end{macrocode}
% \end{macro}\begin{macro}{\thesis@blocks@declaration}
% The |\thesis@blocks@declaration| macro typesets the declaration
% text. Unlike the generic |\thesis@blocks@declaration| macro from
% the \texttt{style/mu/fithesis-sci.sty} file, this definition
% includes the date and a blank line for the author's signature, as
% per the requirements of the faculty.
%
% Along with the macros required by the locale file interface, the
% locale files need to define the following macros:
% \begin{itemize}
%   \item|\thesis@|\textit{locale}|@authorSignature| -- The
%     label of the author's signature field
%   \item|\thesis@|\textit{locale}|@formattedDate| -- A
%     formatted date
% \end{itemize}
%    \begin{macrocode}
\def\thesis@blocks@declaration{%
  \thesis@blocks@clear
  \begin{alwayssingle}%
    \chapter*{\thesis@@{declarationTitle}}%
    \thesis@declaration
    \vskip 2cm%
    {\let\@A\relax\newlength{\@A}
      \settowidth{\@A}{\thesis@@{authorSignature}}
      \setlength{\@A}{\@A+1cm}
    \noindent\thesis@place, \thesis@@{formattedDate}\hfill
    \begin{minipage}[t]{\@A}%
      \centering\rule{\@A}{1pt}\\
      \thesis@@{authorSignature}\par
    \end{minipage}}
  \end{alwayssingle}}
%    \end{macrocode}
% \end{macro}
% Note that there is no direct support for the seminar paper and
% thesis proposal types.  If you would like to change the contents
% of the preamble and the postamble, you should modify the
% |\thesis@blocks@preamble| and |\thesis@blocks@postamble| macros.
%
% All blocks within the autolayout preamble and postamble that are
% not defined within this file are defined in the
% \texttt{style/mu/fithesis-base.sty} file.
%    \begin{macrocode}
\def\thesis@blocks@preamble{%
  \thesis@blocks@coverMatter
    \thesis@blocks@cover
    \thesis@blocks@titlePage
  \thesis@blocks@frontMatter
    \thesis@blocks@declaration
    \thesis@blocks@thanks
    \thesis@blocks@tables}
\def\thesis@blocks@postamble{%
  \thesis@blocks@bibliography}
%    \end{macrocode}

