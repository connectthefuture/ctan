%%%%%%%%%%%%%%%%%%%%%%%%%%%%%%%%%%%%%%%%%%%%%%%%%%%%%%%%%%%%%%%%%%%%%%%%%%%%%%%%%%%%%%%%%
% ------------------------------------------------------------------------------------- %
% - chemexec - chemexec_de.tex -------------------------------------------------------- %
% - creating (chemical) exercise sheets, chemical macros ------------------------------ %
% ------------------------------------------------------------------------------------- %
% - Clemens Niederberger -------------------------------------------------------------- %
% - 2011/03/06 ------------------------------------------------------------------------ %
% ------------------------------------------------------------------------------------- %
% - http://www.niederberger-berlin.net/2010/08/latex-chemische-formeln-erstellen-iiv/ - %
% - kontakt@niederberger-berlin.net --------------------------------------------------- %
% ------------------------------------------------------------------------------------- %
% - If you have any ideas, questions, suggestions or bugs to report, please feel free - %
% - to contact me. -------------------------------------------------------------------- %
% ------------------------------------------------------------------------------------- %
% Copyright 2011 Clemens Niederberger                                                   %
%                                                                                       %
% This work may be distributed and/or modified under the                                %
% conditions of the LaTeX Project Public License, either version 1.3                    %
% of this license or (at your option) any later version.                                %
% The latest version of this license is in                                              %
%   http://www.latex-project.org/lppl.txt                                               %
% and version 1.3 or later is part of all distributions of LaTeX                        %
% version 2005/12/01 or later.                                                          %
%                                                                                       %
% This work has the LPPL maintenance status `maintained'.                               %
%                                                                                       %
% The Current Maintainer of this work is Clemens Niederberger.                          %
%                                                                                       %
% This work consists of the files chemexec.sty, chemexec_de.tex                         %
% and chemexec_en.tex                                                                   %
% ------------------------------------------------------------------------------------- %
%%%%%%%%%%%%%%%%%%%%%%%%%%%%%%%%%%%%%%%%%%%%%%%%%%%%%%%%%%%%%%%%%%%%%%%%%%%%%%%%%%%%%%%%%
\documentclass[a4paper,10pt,headsepline]{scrartcl}
%%%%%%%%%%%%%%%%%%%%%%%%%%%%%%%%%%%%%%%%%%%%%%%%%%%%%%%%%%%%%%%%%%%%%%%%%%%%%%%%%%%%%%%%%
% = PAKETE/EINSTELLUNGEN ============================================================== %
%%%%%%%%%%%%%%%%%%%%%%%%%%%%%%%%%%%%%%%%%%%%%%%%%%%%%%%%%%%%%%%%%%%%%%%%%%%%%%%%%%%%%%%%%
\usepackage[ngerman]{babel}                      % deutsche Sprache
\usepackage[utf8x]{inputenc}
\usepackage[T1]{fontenc}
\usepackage{listings}
\usepackage{array,slashbox}
\setlength\extrarowheight{7pt}
\usepackage{graphicx}
\usepackage[exercise,exersize=large]{chemexec}
  \usetikzlibrary{fadings,patterns}
\usepackage{xcolor,wrapfig}
\definecolor{dunkelblau}{rgb}{0,0.33,0.62}
\definecolor{dunkelrot}{rgb}{0.4392,0.0627,0.0627}
\usepackage{url,paralist}
\usepackage[ngerman]{varioref}                   % flexibles Referenzieren
  \labelformat{section}{Abschnitt #1}
  \labelformat{subsection}{Abschnitt #1}
  \labelformat{subsubsection}{Abschnitt #1}
% = Hyperlinks ======================================================================== %
\usepackage{hyperref}
  \hypersetup{colorlinks=true,                   % Farben der pdf-Links verändern
              plainpages=false,
              linkcolor=black,
              urlcolor=black,
              citecolor=black,
              bookmarksopen=true,
              bookmarksopenlevel=2,
              bookmarksnumbered=true,
              pdfstartview=FitH,
              pdfauthor={Clemens Niederberger},
              pdftitle={chemexec},
              pdfsubject={customizing you (chemical) exercise sheets},
              pdfkeywords={chemexec},
              pdfcreator={LaTeX}
  }
\usepackage{times,chemfig,xspace}%,stree}
% = Listings einstellen =============================================================== %
\lstset{
   basicstyle={\ttfamily\footnotesize},          % Grundstil
   extendedchars=true,
   numbers=left,                                 % Zeilennummern
   numberstyle=\tiny,                            % Größe des Zeilennummern
   numberblanklines=true,                        % Leerzeilen nummerieren
   gobble=1,                                     % das erste Leerzeichen abschneiden
   xleftmargin=20pt,                             % Einrückung links
   breaklines=true,                              % Zeilenumbruch
   moredelim=[is][\color{red}]{|}{|}                    % Hervorhebung
   }

% = Kopfzeile ========================================================================= %
\usepackage{scrpage2}
\pagestyle{scrheadings}
\setheadwidth{textwithmarginpar}
\automark{section}
\ihead{\CEx \CEversion}
\ifoot{\small\color{gray}-~Seite~\thepage~-}
\cfoot{}
\ofoot{}

% = Überschriften ===================================================================== %
\setkomafont{disposition}{\rmfamily\bfseries}    % Gewicht fett und Schriftart roman

% ===================================================================================== %
\newcommand{\option}[1]{`\texttt{#1}'\xspace}
\newcommand{\TikZ}{\mbox{Ti{\bfseries\itshape k}Z}\xspace}
%\newcommand{\eg}{\mbox{e.\,g.}\xspace}
\newcommand{\zB}{\mbox{z.\,B.}\xspace}
\newlength{\chemx}
\newlength{\chemy}
\newcommand{\CEx}[1][8]{%
{\color{dunkelrot}\fontfamily{pag}\fontsize{#1}{#1}\selectfont chemexec}\xspace
}
\let\saveversion\CEversion
\def\CEversion{\saveversion\xspace}

\begin{document}
\begin{titlepage}
  % inspired by the titlepage of chemfig's documentation
  \begin{tikzpicture}[remember picture,overlay]
    \shade [color=dunkelrot,right color=white](current page.south west) rectangle ([yshift=3cm,xshift=-3cm]current page.center);
    \shade[top color=black,bottom color=dunkelrot]([yshift=7cm]current page.east)rectangle([yshift=2.5cm]current page.west);
  \end{tikzpicture}
  \begin{center}
    \vspace*{-1.5cm}
    \CEx[45]\par
    \Large\CEversion\par\bigskip
    \footnotesize\CEdate{de}\par
    \normalsize Clemens Niederberger\par\vskip1.5cm
    \color{white}\huge (chemische) \"Ubungsaufgabenbl\"atter%
  \end{center}
  \vskip3cm
  
\end{titlepage}

\tableofcontents

\section{Lizenz}
\CEx \CEversion steht unter der LaTeX Project Public License Version 1.3 oder sp\"ater.\newline(\url{http://www.latex-project.org/lppl.txt})

\section{\"Uber}
Das \verb=chemexec= Paket stellt einige kleine Umgebungen und Befehle zur Ver\-f\"u\-gung, die ich f"ur die Verwendung in "Ubungsbl"attern und Unterrichtsskripten brauchte. So gibt es nun die \verb=definition=-Umgebung, die \verb=beispiel=-Um\-ge\-bung u.\"a. Au\ss{}erdem den einen oder anderen n\"utzlichen Befehl, der einem Schreibarbeit abnimmt.\\
F\"ur Arbeitsbl\"atter haben sich die Aufgaben/L\"osungs-Befehle in \ref{sec:aufgaben} als recht n\"utzlich erwiesen.\\
Das Paket ersetzt \verb=echem.sty= f"ur OCHEM von Ingo Kl"ockl\footnote{\url{http://www.2k-software.de/ingo/ochem.html}}.

\section{Neu in Version \CEversion}
\CEx ist neu \"uberarbeitet und dabei etwas verschlankt worden. die Befehle \verb=\lw=,\linebreak\verb=\lwbar=, \verb=\atomconnect= und die \verb=Schema=-Umgebung sind herausgeflogen. Dafür funktionieren jetzt alle Befehle auch mit \verb=pdflatex=, da alle Zeichungen (siehe etwa \ref{sssec:stereo}) nun mit \TikZ und nicht mehr mit pstricks erstellt werden.

\section{Paket-Optionen}
Folgende Optionen k"onnen ausgew"ahlt werden:
\begin{itemize}
 \item Die Option \option{chapter} "andert die Z"ahler-Einstellung f"ur die Aufgaben und L"osungen (\ref{sec:aufgaben}) und die Beispiele (\ref{subsec:beispiele}).% und die Schemas (Abschnitt \ref{subsec:schema}).
 \item Die Option \option{color=farbe} "andert die Farbe der Nummern, mit denen die Aufgaben und L"osungen durchnummeriert werden, der Linien, die die \texttt{beispiel}-Umgebung einrahmen, und der "Uberschrift der \texttt{definition}-Umgebung (\ref{sec:definition}) in \texttt{far\-be}.\\ Default ist Dunkelblau:\\\verb=\xdefinecolor{dunkelblau}{rgb}{0,0.33,0.62}=.
 \item Die Option \option{english} "andert die "Uberschriften der Aufgaben (Exercise), L"osungen (Solution), Beispiele (Example) und den Exkurs (Excursus).
 \item Die Option \option{exercise} erm"oglicht das Verwenden der Befehle f"ur die Aufgaben und L"osungen.
 \item Mit der Option \option{exersize=groesse} l"asst sich die Schriftgr"o\ss e der "Uberschriften der Aufgaben und L"osungen einstellen.
 \item Die Option \option{here} legt f"ur die \texttt{Schema}-Gleitumgebung als Positionierung \texttt{H} (genau hier) fest.
 \item Die Option \option{shade=boolean} "andert das prinzipielle Layout der \texttt{definition}-Umgebung, kann die Werte \texttt{true} oder \texttt{false} einnehmen, Default ist \texttt{shade=false}.
 \item Die Option \option{shadecolor=farbe} "andert die Hintergrundfarbe der \texttt{definition}-Umgebung in \texttt{farbe}, wenn die Option `\texttt{shade}' ausgew"ahlt ist.
 \item Die Option \option{numcolor=farbe} "andert die Farbe der Nummern, mit denen die Aufgaben und L"osungen durchnummeriert werden, in \texttt{farbe}.
\end{itemize}

\section{Neue Befehle}
\subsection{Mathematik}
Ich habe einige kleine Befehle definiert, die ich immer wieder brauchte:
\begin{itemize}
 \item \verb=\vek{}= Pfeilschreibweise f\"ur Vektoren: \verb=\vek{a}, \vek{A}= ergibt \vek{a}, \vek{A}.
 \item \verb=$\abs{}$= Betrag: \verb=\abs{\vek{a}}, \abs{-\frac{i}{2}}= ergibt \abs{\vek{a}}, \abs{-\frac{i}{2}}.
\end{itemize}
Beide Befehle funktionieren sowohl in normalem Text als auch in der Ma\-the\-ma\-tik-Um\-ge\-bung.

\subsection{Chemie}
F\"ur die Chemie habe ich die folgenden Befehle immer wieder als sehr n\"utzlich empfunden.
\subsubsection{Teilchen und Ladungen}
\begin{itemize}
 \item \verb=\el= Elektron: \el
 \item \verb=\prt= Proton: \prt
 \item \verb=\ntr= Neutron: \ntr
 \item \verb=\Hpl= Proton: \Hpl
 \item \verb=\Hyd= Hydroxid: \Hyd
 \item \verb=\ox{}{}= Oxidationszahlen\\
       \verb=Ca\ox{-1}{F}$_2$=  Ca\ox{-1}{F}$_2$;\\
       das erste Argument ist die Oxidationszahl, das zweite das Element.
 \item \verb=\om= und \verb=\op= Ladungen \om\ und \op.\\
       Beide Befehle haben ein optionales Argument f"ur die Anzahl der Ladungen:\\\verb=Ca\op[2]=  Ca\op[2], \verb=\phosphat\om[3]=  \phosphat\om[3].
\end{itemize}

\subsubsection{Stereodeskriptoren}\label{sssec:stereo}
Einige Deskriptoren zur Erleichterung der Nomenklatur.
\begin{itemize}
 \item \verb=\Rcip= und \verb=\Scip=, rectus und sinister: \Rcip\ \Scip
 \item \verb=\Dfi= und \verb=\Lfi=, dexter und laevus: \Dfi\ \Lfi
 \item \verb=\E= und \verb=\Z=, entgegen und zusammen: \E\ \Z
 \item \verb=\rconf= und \verb=\sconf= R/S-Konfiguration: \rconf\ und \sconf. Beide Befehle haben ein optionales Argument, mit dem der Buchstabe ge"andert werden kann: \verb=\rconf[]= \rconf[]
\end{itemize}
\subsubsection{Anionen}
Ebenfalls definiert sind folgende S\"aurereste:
\begin{itemize}
 \item \verb=\nitrat= : \nitrat
 \item \verb=\nitrit= : \nitrit
 \item \verb=\sulfat= : \sulfat
 \item \verb=\sulfit= : \sulfit
 \item \verb=\phosphat= : \phosphat
 \item \verb=\phosphit= : \phosphit
 \item \verb=\carbonat= : \carbonat
\end{itemize}
Alle Chemie-Befehle sind sowohl im Text- als auch im Mathematik-Modus einsetzbar.

\subsubsection{Kompatibilit"at mit \texttt{mhchem.sty}}
Die Chemie-Befehle sind auch in den Formelsatz-Befehlen (\verb=\ce{}= u."a.) des `mhchem'-Pakets von Martin Hensel\footnote{\url{http://www.ctan.org/tex-archive/macros/latex/contrib/mhchem/}} einsetzbar. Tats"achlich l"adt \CEx `mhchem' (in der Version 3) automatisch, falls es vorhanden ist. Es muss also nur dann geladen werden, wenn ihm Optionen mitgegeben werden sollen.
\begin{lstlisting}[numbers=none,basicstyle=\normalsize\ttfamily,showspaces=true]
 \ce{2| |\ox{0}{Ca} +| |\ox{0}{O}_2 ->T[{~~~REDOX~~~}] 2Ca| |\op[2] + 2O| |\om[2]}
\end{lstlisting}
\ce{ 2 \ox{0}{Ca} + \ox{0}{O}_2 ->T[{~~~REDOX~~~}] 2Ca \op[2] + 2O \om[2] }\\
Beachten Sie bitte, dass Sie die L"ucke vor \verb=\om=, \verb=\op=, \verb=\ox{}{}= lassen sollten, sonst kann das zu Fehlermeldungen oder falscher Darstellung f"uhren:\\
\verb=\ce{Ca\op[2]}=  \ce{ Ca\op[2] }

\noindent Weitere Beispiele:
\begin{lstlisting}[numbers=none,basicstyle=\normalsize\ttfamily]
 \begin{align*}
  \cee{Na                  &->T[ox] Na\op|{}| + \el}\\
  \cee{HCl_{aq}            &<=>> H\op_{aq} + Cl\om_{aq}}\\
  \cee{H2O                 &<<=> \Hpl + \Hyd}\\
  \cee{CaCl2 + H2\sulfat|{}| &-> Ca\sulfat|{}| v + 2 HCl}
 \end{align*}
\end{lstlisting}
\begin{align*}
 \cee{Na                  &->T[ox] Na\op{} + \el}\\
 \cee{HCl_{aq}            &<=>> H\op_{aq} + Cl\om_{aq}}\\
 \cee{H2O                 &<<=> \Hpl + \Hyd}\\
 \cee{CaCl2 + H2\sulfat{} &-> Ca\sulfat{} v + 2 HCl}
\end{align*}
\CEx sollte \emph{nach} `mhchem' eingebunden werden, wenn Sie das `mhchem' Paket Laden, um ihm Optionen mitzugeben.

\subsubsection{Befehle f"ur `mhchem'}
\CEx stellt einige Befehle f"ur das Erstellen von Reaktionen mit `mhchem' zur Verf"ugung:
\begin{lstlisting}
 nummerierte Reaktion:
 \reaction{2 H2 + O2 -> 2 H2O}%
 unnummerierte Reaktion:
 \reaction*{2 CO + O2 -> 2 CO2}
 mehrere ausgerichtete Reaktionen:
 \reactions{Cl_2 ||&||-> 2 Cl. ||\\|| Cl. + CH4 ||&||-> HCl + {}.CH3}
\end{lstlisting}
nummerierte Reaktion:
\reaction{2 H2 + O2 -> 2 H2O}%
unnummerierte Reaktion:
\reaction*{2 CO + O2 -> 2 CO2}
mehrere ausgerichtete Reaktionen:
\reactions{Cl2 &-> 2 Cl. \\ Cl. + CH4 &-> HCl + {}.CH3}

\section{Neue Umgebungen}
\subsection{Die {\ttfamily beispiel}-Umgebung}\label{subsec:beispiele}
F\"ur \"Ubungsbl\"atter und \"ahnliches ben\"otigte ich immer wieder eine Umgebung, die Beispiele hervorhebt und durchnummeriert:
\begin{lstlisting}
 |\begin{beispiel}|
  Ein Beispiel.
 |\end{beispiel}|
\end{lstlisting}
Das ergibt folgenden Output:
\begin{beispiel}
 Ein Beispiel.
\end{beispiel}
\noindent Wenn man ein zweites Beispiel im gleichen Rahmen bringen m\"ochte, kann man den Befehl \verb=\bsp=
verwenden:
\begin{lstlisting}
 \begin{beispiel}
  Ein erstes Beispiel.
  |\bsp|
  Ein zweites.
 \end{beispiel}
\end{lstlisting}
%\setcounter{beispiel}{0}
\begin{beispiel}
 Ein erstes Beispiel.
 \bsp
 Ein zweites.
\end{beispiel}

\subsubsection{Die Optionen \texttt{color}, \texttt{linecolor} \&\ \texttt{english}}
Die Paket-Option \option{linecolor=farbe} erm"oglicht, die Default-Farbe der umschlie\ss enden Linien zu "andern. Zum Beispiel ergibt
\begin{lstlisting}
 % Pr"aambel:
 \usepackage|[linecolor={rgb:red,4;green,6}]|{chemexec}
 % im Dokument:
 \begin{beispiel}
  Gr"une Linien per Paketoption.
 \end{beispiel}
\end{lstlisting}
folgenden Output:
\begin{beispiel}[linecolor={rgb:red,4;green,6}]
 Gr"une Linien per Paketoption.
\end{beispiel}
\noindent Auch die Paket-Option \option{color=farbe} "andert die Farben der Linien, wirkt sich aber noch auf weitere Befehle wie die \texttt{definition}-Umgebung aus.\\
Mit der Befehls-Option \option{linecolor=farbe} kann man auch die Farbe eines konkreten Beispiels "andern. So ergibt
\begin{lstlisting}
 \begin{beispiel}|[linecolor=purple]|
  Die purpurne Einzelversion.
 \end{beispiel}
\end{lstlisting}
folgenden Output:
\begin{beispiel}[linecolor=purple]
 Die purpurne Einzelversion.
\end{beispiel}
\noindent Die Paket-Option \option{english} erzeugt die englische "Uberschrift `Example'.
\subsubsection{Unnummerierte Beispiele}
Wenn Sie gerne unnummerierte Beispiele m"ogen oder die Beispiele mit Buchstaben durch\-z"ah\-len wollen, k"onnen Sie das wie "ublich mit der Neudefinition der Z"ahlerausgabe realisieren.
\begin{lstlisting}
 \renewcommand{\thebeispiel}{}
 \begin{beispiel}
  Jetzt ohne Z"ahler!
 \end{beispiel}
\end{lstlisting}
\renewcommand{\thebeispiel}{}
\begin{beispiel}
 Jetzt ohne Z"ahler!
\end{beispiel}
\begin{lstlisting}
 \renewcommand{\thebeispiel}{\alph{beispiel})}
 \begin{beispiel}
  Oder alphabetisch \ldots
 \end{beispiel}
\end{lstlisting}
\renewcommand{\thebeispiel}{\alph{beispiel})}
\begin{beispiel}
 Oder alphabetisch \ldots
\end{beispiel}

\subsection{Die {\ttfamily definition}-Umgebung}\label{sec:definition}
Die \texttt{definition}-Umgebung erstellt einen Kasten mit farbiger \"Uberschrift:
\begin{lstlisting}
 |\begin{definition}|
  Der Betrag eines Vektors betr\"agt
  \begin{equation}
   |\abs{\vek{|a|}}|=\sqrt{a_x^2+a_y^2+a_z^2}
  \end{equation}
 |\end{definition}|
\end{lstlisting}
\begin{definition}
 Der Betrag eines Vektors betr\"agt:
 \begin{equation}
  \abs{\vek{a}}=\sqrt{a_x^2+a_y^2+a_z^2}
 \end{equation}
\end{definition}

\subsubsection{Die Optionen \texttt{shade}, \texttt{shadecolor} \& \texttt{color}}
Mit der Umgebungs-Option \option{shade=boolean} "andert sich das Layout:
\begin{lstlisting}
 \begin{definition}|[shade=true]|
  Der Betrag eines Vektors betr\"agt
  \begin{equation}
   \abs{\vek{a}}=\sqrt{a_x^2+a_y^2+a_z^2}
  \end{equation}
 \end{definition}
\end{lstlisting}
%\setcounter{equation}{0}
\begin{definition}[shade=true]
 Der Betrag eines Vektors betr\"agt:
 \begin{equation}
  \abs{\vek{a}}=\sqrt{a_x^2+a_y^2+a_z^2}
 \end{equation}
\end{definition}
Mit den Optionen \option{shadecolor=farbe} und \option{color=farbe} l"asst sich das Layout noch weiter beeinflussen:
\begin{lstlisting}
 \begin{definition}|[shade=true,shadecolor=green!15,color=black]|
  Der Betrag eines Vektors betr\"agt
  \begin{equation}
   \abs{\vek{a}}=\sqrt{a_x^2+a_y^2+a_z^2}
  \end{equation}
 \end{definition}
\end{lstlisting}
\begin{definition}[shade=true,shadecolor=green!15,color=black]
 Der Betrag eines Vektors betr\"agt:
 \begin{equation}
  \abs{\vek{a}}=\sqrt{a_x^2+a_y^2+a_z^2}
 \end{equation}
\end{definition}
Die Optionen \option{shade=boolean} und \option{shadecolor=farbe} sind auch als Paketoptionen einsetzbar. Damit l"asst sich das grunds"atzliche Aussehen der K"asten einstellen. Die Option \option{color=farbe} ist ebenfalls als Paket-Option einsetzbar, wirkt sich dann aber nicht nur auf die \texttt{definition}-Umgebung aus.
\begin{lstlisting}
 % Pr"aambel:
 \usepackage|[shade=true,shadecolor=yellow!15]|{chemexec}
 % im Dokument:
 \begin{definition}
  Der Betrag ...
 \end{definition}
\end{lstlisting}
\begin{definition}[shade=true,shadecolor=yellow!15]
 Der Betrag eines Vektors betr\"agt:
 \begin{equation}
  \abs{\vek{a}}=\sqrt{a_x^2+a_y^2+a_z^2}
 \end{equation}
\end{definition}

\subsubsection{Die \texttt{defformel}-Umgebung}
Zus"atzlich gibt es die \texttt{defformel}-Umgebung, die lediglich einen wei\ss{}en Hintergrund erzeugt und ein optionales Argument f"ur die Breite des wei\ss en Kastens besitzt\footnote{Die chemische Struktur wurde mit Hilfe des stree\TeX-Pakets von Igor Strokov erstellt.}.
\begin{lstlisting}
 \begin{definition}[shade=true]
  Ein Kohlenstoffatom mit vier verschiedenen Substituenten nennt man \textbf{chiral}. Chiralit"atszentren werden oft mit einem \textasteriskcentered\ markiert.
  |\begin{defformel}[.5\textwidth]|
   \chemfig{R_1-[:30](-[2]R_2)(-[6]R_3)(-[:30,.15,,,white]{\text{\textasteriskcentered}})-[:-30]R_4}
  |\end{defformel}|
 \end{definition}
\end{lstlisting}

\begin{definition}[shade=true]
  Ein Kohlenstoffatom mit vier verschiedenen Substituenten nennt man \textbf{chiral}. Chiralit"atszentren werden oft mit einem \stec\ markiert.
  \begin{defformel}[.5\textwidth]
   \chemfig{R_1-[:30](-[2]R_2)(-[6]R_3)(-[:30,.15,,,white]{\text{\stec}})-[:-30]R_4}
  \end{defformel}
 \end{definition}

\subsection{Die \texttt{exkurs}-Umgebung}
Die \texttt{exkurs}-Umgebung ist dazu gedacht, in B"uchern oder l"angeren Texten einen Exkurs "uber ein Thema optisch hervorzuheben und einen Eintrag ins Inhaltsverzeichnis hinzuzuf"ugen.
\begin{lstlisting}[numbers=none,basicstyle=\normalsize\ttfamily]
 \begin{exkurs}[options]{titel}
  ...
 \end{exkurs}
\end{lstlisting}
Es gibt zwei Optionen: \option{toc=toclevel} mit der Default-Einstellung \verb=section= und\linebreak\option{color=farbe} mit der Default-Einstellung \verb=dunkelblau=.
\begin{lstlisting}
 |\begin{exkurs}[color=-yellow]{Lorem ipsum}|
  Lorem ipsum dolor sit amet, consectetuer adipiscing elit, sed diam nonummy nibh euismod tincidunt ut laoreet dolore magna aliquam erat volutpat. ...
 |\end{exkurs}|
\end{lstlisting}
\begin{exkurs}[color=-yellow,toc=paragraph]{Lorem ipsum}
 Lorem ipsum dolor sit amet, consectetuer adipiscing elit, sed diam nonummy nibh euismod tincidunt ut laoreet dolore magna aliquam erat volutpat. Ut wisi enim ad minim veniam, quis nostrud exerci tation ullamcorper suscipit lobortis nisl ut aliquip ex ea commodo consequat. Duis autem vel eum iriure dolor in hendrerit in vulputate velit esse molestie consequat, vel illum dolore eu feugiat nulla facilisis at vero eros et accumsan et iusto odio dignissim qui blandit praesent luptatum zuril delenit augue duis dolore te feugait nulla facilisi.
\end{exkurs}
Die Paket-Option \option{color=farbe} wirkt sich ebenfalls auf die Farbe der Einsch"ube aus, mit der Paket-Option \option{english} wird die "Uberschrift in `excursus' ge"andert.

\section[Aufgaben/L\"osungen]{Die Option \texttt{exercise}: Nummerierte Aufgaben/L\"osungen}\label{sec:aufgaben}
Als eigentlicher Kern des Pakets ist ein Z\"ahler/eine \"Uberschrift f\"ur Aufgaben definiert, die man mit der Option \option{exercise} aktivieren kann. Die Aufgaben erhalten als De\-fault-\"U\-ber\-schrift `Aufgabe', k\"onnen aber eine beliebige andere als Argument bekommen. Die Nummern sind farbig. Der Befehl lautet:
\begin{lstlisting}[numbers=none,basicstyle=\normalsize\ttfamily]
 \aufgabe{aufgabentitel}
\end{lstlisting}
Da ich gerne auch die M\"oglichkeit habe, die L\"osungen anzugeben, habe ich zudem die Befehle
\begin{lstlisting}[numbers=none,basicstyle=\normalsize\ttfamily]
 \loesung[aufgabentitel]{Loesung} % Loesung eingeben
 \doloesung % Loesung kapitelweise ausgeben
 \makeloesung % Loesungen auf einmal ausgeben
\end{lstlisting}
definiert. In den ersten Befehl \verb=\loesung{}= gibt man die L\"osung der Aufgabe ein, eventuell mit dem Aufgabentitel als optionalem Argument. Er sollte immer direkt nach der zugeh\"origen Aufgabe eingesetzt werden. Der zweite Befehl \verb=\doloesung= erzeugt die Ausgabe der L\"osungen, die in der aktuellen \verb=\section= gesammelt wurden und der dritte Befehl \verb=\makeloesung= erzeugt alle gesammelten L\"osungen auf einmal. Beachten Sie, dass \verb=\doloesung= und \verb=\makeloesung= einander ausschlie\ss{}en. Sie m\"ussen Sich f\"ur eine von beiden Varianten entscheiden.\\
F\"ur die Ausgabe der L\"osungen ist es unerheblich, ob man jeder Aufgabe eine L\"osung zugewiesen hat. \verb=\makeloesung= sollte sinnvollerweise erst \emph{nach allen Aufgaben} gesetzt werden und \emph{\textbf{kann nur einmal aufgerufen werden}}.

\subsection{Optionen}
Die Option\option{exersize=groesse} erm"oglicht die Einstellung der Schriftgr"osse der "Uberschriften. Erlaubt sind die bekannten Varianten: \texttt{tiny}, \texttt{scriptsize}, \texttt{footnotesize}, \texttt{small}, \texttt{normalsize}, \texttt{large}, \texttt{Large}, \texttt{LARGE}, \texttt{huge} und \texttt{Huge}. Die Default-Einstellung ist \texttt{normalsize}.\\
Mit der Paket-Option \option{numcolor=farbe} kann man die Farbe der Nummern in \texttt{farbe} "andern. Die Paket-Option \option{english} erzeugt die englischen "Uberschriften `Exercise' bzw. `Solution'.\\
In der Default-Einstellung werden die Nummern der Aufgaben mit jeder neuen \texttt{section} zur"uckgesetzt. Die Paket-Option \option{chapter} "andert die Einstellung, so dass der Z"ahler mit einem neuen \texttt{chapter} zur"uckgesetzt wird.
\subsection{Die {\ttfamily alphlist}-Umgebung}
Mit der \texttt{alphlist}-Umgebung steht eine Liste zur Verf"ugung, die z.B. Aufgaben automatisch mit a), b) etc. durchz"ahlt.
\begin{lstlisting}
 \begin{alphlist}
  \item erster Punkt
  \item zweiter Punkt
 \end{alphlist}
\end{lstlisting}
\begin{alphlist}
 \item erster Punkt
 \item zweiter Punkt
\end{alphlist}

\subsection{Beispiel}
In folgendem Listing k"onnen Sie die Aufgaben und den Befehl \texttt{\textbackslash doloesung} einmal im Einsatz sehen. Das Ergebnis sehen Sie direkt im Anschluss.
\begin{lstlisting}
 % Pr"aambel:
 \usepackage|[exercise,exersize=large]|{chemexec}
 % Im Dokument:
 \par{\Large\bfseries\noindent Aufgaben}
 |\aufgabe{}|
  Geben Sie die Protolysereaktionen von Phosphors"aure an.|\loesung{|\ce{H3PO4 <=> \Hpl{} + H2PO4\om{} <=> 2\Hpl{} + HPO4 \om[2]{} <=> 3\Hpl{} + PO4 \om[3]}|}|
 |\aufgabe{|Oxidationszahlen|}|
  Welche Oxidationsstufe hat der Stickstoff in den folgenden Verbindungen: Ammoniak, Stickstoffmonoxid, Stickstoffdioxid, Salpeters"aure?|\loesung[|Oxidationszahlen|]{|
  \ce{ \ox{-3}{N} H3}, \ce{ \ox{+2}{N} O}, \ce{ \ox{+4}{N} O2}, \ce{H \ox{+5}{N} O3}|}|
 |\aufgabe{|Nomenklatur|}|
  Benennen Sie folgende Molek"ule:\\\setatomsep{1.4em}
  \begin{inparaenum}[a)]
   \item\chemfig{-[::30](=[::60]O)-[::-60]OH}
   \item\chemfig{-[::30](=[::60]O)-[::-60]O-[::60]-[::-60]}
   \item\chemfig{HO-[::-30](=[::-60]O)-[::60]-[::-60]-[::60](=[::60]O)-[::-60]OH}
  \end{inparaenum}
 |\loesung[|Nomenklatur|]{|
  \begin{inparaenum}[a)]
   \item Ethans"aure
   \item Ethans"aureethylester
   \item Butandis"aure
  \end{inparaenum}
 |}|
 |\aufgabe{}|
  Zeichnen Sie die Strukturformel von Glycerin. Geben Sie den systematischen Namen nach IUPAC an.
 |\loesung{\chemname{\chemfig{HO-[::-30]-[::60](-[::60]OH)-[::-60]-[::60]OH}}{1,2,3-Propantriol}}|
 \vspace{\baselineskip}
 \par{\Large\bfseries\noindent L\"osungen}
 |\doloesung|
\end{lstlisting}
\par{\Large\bfseries\noindent Aufgaben}
\aufgabe{}
 Geben Sie die Protolysereaktionen von Phosphors"aure an.\loesung{\ce{H3PO4 <=> \Hpl{} + H2PO4\om{} <=> 2\Hpl{} + HPO4 \om[2]{} <=> 3\Hpl{} + PO4 \om[3]}}
\aufgabe{Oxidationszahlen}
 Welche Oxidationsstufe hat der Stickstoff in den folgenden Verbindungen: Ammoniak, Stickstoffmonoxid, Stickstoffdioxid, Salpeters"aure?\loesung[Oxidationszahlen]{\ce{ \ox{-3}{N} H3}, \ce{ \ox{+2}{N} O}, \ce{ \ox{+4}{N} O2}, \ce{H \ox{+5}{N} O3}}
\aufgabe{Nomenklatur}
 Benennen Sie folgende Molek"ule:\\\setatomsep{1.4em}
 \begin{inparaenum}[a)]
  \item\chemfig{-[::30](=[::60]O)-[::-60]OH} \item \chemfig{-[::30](=[::60]O)-[::-60]O-[::60]-[::-60]} \item \chemfig{HO-[::-30](=[::-60]O)-[::60]-[::-60]-[::60](=[::60]O)-[::-60]OH}
 \end{inparaenum}
 \loesung[Nomenklatur]{
 \begin{inparaenum}[a)]
 \item Ethans"aure
 \item Ethans"aureethylester
 \item Butandis"aure
 \end{inparaenum}
}
\aufgabe{}
 Zeichnen Sie die Skelettformel von Glycerin. Geben Sie den systematischen Namen nach IUPAC an.
\loesung{\chemname{\chemfig{HO-[::-30]-[::60](-[::60]OH)-[::-60]-[::60]OH}}{1,2,3-Propantriol}}
\newpage
\par{\Large\bfseries\noindent L\"osungen}
\doloesung

\section{Ersatz f"ur \texttt{echem.sty}}
Das Paket \verb=echem.sty= geh"ort zum OCHEM-Programm von Ingo Kl"ockl\footnote{\url{http://www.2k-software.de/ingo/ochem.html}}. Es erm"oglicht die Darstellung von Elektronen als Punkten und Elektronenpaaren als Strichen an Atome. Die Definition der `lw'-Befehle orientiert sich an den Definitionen des `echem'-Pakets. Zus"atzlich stellt \verb=echem.sty= die beiden Makros \verb=\sbond= und \verb=\dbond= zur Verf"ugung, die es erlauben, im normalen Text eine Einzel- bzw. Doppelbindung darzustellen. Diese beiden Befehle werden im `mhchem'-Paket von Martin Hensel ebenfalls definiert.
\begin{itemize}
 \item `mhchem'-Version:
 \begin{itemize}
  \item\texttt{\textbackslash ce\{F\textbackslash sbond F\}}: \ce{F\sbond F}
  \item\texttt{\textbackslash ce\{O\textbackslash dbond O\}}: \ce{O\dbond O}
 \end{itemize}
 \item\newcommand{\sibond}{\,\ensuremath{\cdot}\,}\newcommand{\dobond}{\,=\,} `echem'-Version:
 \begin{itemize}
  \item\texttt{F\textbackslash sbond F}: F\sibond F
  \item\texttt{O\textbackslash dbond O}: O\dobond O
 \end{itemize}
\end{itemize}
Um keine Konflikte mit `mhchem' zu provozieren, liefert \CEx abgesehen von diesen beiden Befehlen und den Makros \verb=\2= und \verb=\3= die gleiche Funktionalit"at wie `echem' und kann an dessen Stelle in der \texttt{chemspecial}-Umgebung von OCHEM verwendet werden.
\begin{lstlisting}
 % Pr"aambel
 \usepackage{ochem,chemexec}
 % im Dokument:
 \begin{chemspecial}
  |package("chemexec")|
 \end{chemspecial}
 \begin{chemistry}
  formula(L,R){
    bond(30;-30;30)
    branch { bond(90,=C)
             atom("|\vdd{|O|}|O\vdd{O}");
           }
    bond(-30;30;-30)
    atom("|\hdl[\echhbar]{|O|}\hdu|[\echhbar]|{|O|}|O",L,R)
    bond(30;-30)
  }
 \end{chemistry}
\end{lstlisting}
\includegraphics{./formel.jpg}%\\% formel.jpg: 131x44 pixel, 100dpi, 3.33x1.12 cm, bb=0 0 94 32

\section{Nachwort}
Auch wenn ich mich bemüht habe, sinnvolle chemische Reaktionen einzusetzen, habe ich nicht extra überprüft, ob jedes Beispiel chemisch sinnvoll ist. Vertrauen Sie den Beispielen diesbezüglich nicht, sondern sehen Sie in einem Lehrbuch der Chemie nach
\end{document}