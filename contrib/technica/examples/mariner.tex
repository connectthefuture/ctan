\documentclass{book}
\usepackage[pagestyles,outermarks]{titlesec}[2005/01/22 v2.6]
\usepackage[repeat]{poetry}
\usepackage{drama}
\usepackage{example}

\TextWidth  {4in}
\TextHeight {6.75in}

\newlength {\GlossWidth}
\setlength {\GlossWidth}{20mm}
\newlength {\GlossSep}
\setlength {\GlossSep}{4mm}


\Facies \(    {}
\SpatiumSupra {2ex plus .25ex minus .1ex \penalty -100}
\SpatiumInfra {2ex plus .25ex minus .1ex \penalty 100}

\Locus  \numerus {\textrightmargin}
\Facies          {\RelSize{-1}\oldstylenums{#1}}

\Novus  \titulus \Titulus
\Facies          {\RelSize{+1}#1}
\SpatiumSupra    {2\leading}
\SpatiumInfra    {1\leading\penalty 10000}

\Novus \titulus \voice
\Facies         {\RelSize{-1}\textit{#1 Voice}.}
\Forma          {\rightskip 1,5cm}
\SpatiumSupra   {.25\leading plus .125\leading minus .125\leading\penalty -100}
\SpatiumInfra   {\penalty 10000}


\Novus \numerus \nPart

\Novus \titulus \Part
\Facies         {\RelSize{-1}\nPart{+1}\ordinal{\value{nPart}}%
                 \textsc{part the \theordinal}}
\SpatiumSupra   {2\leading plus 1\leading minus 1\leading
                 \penalty [.2]}
\SpatiumInfra   {.75\leading plus .25\leading minus .25\leading
                 \penalty 10000}

\SpatiumInfra \stropham  {.75\leading plus .25\leading minus .25\leading
                          \penalty -100}

\Forma \strophae {0{\penalty 10000}00{\penalty 10000}0}

\newcommand{\strophaV}{%
  \Forma \strophae {0{\penalty 10000}000{\penalty 10000}0}}

\newcommand{\strophaVI}{%
  \Forma \strophae {0{\penalty 10000}0000{\penalty 10000}0}}

\newcommand{\strophaIX}{%
  \Forma \strophae {0{\penalty 10000}0000000{\penalty 10000}0}}

\Novus \textus \gloss
\Locus         {\textrightmargin + \GlossSep \\ \leftmargin}
\Facies        {\FullHyphenation \RelSize[.9]{-2}#1}
\Modus         {\rangedleft{\GlossWidth}}

\Novus \textus \sgloss
\Locus         {\area [top, wrap] 
                \textrightmargin + \GlossSep \\ \leftmargin}
\Facies        {\FullHyphenation \RelSize[.9]{-2}#1}
\Modus         {\aligned {top} \rangedleft{\GlossWidth}}


\newpagestyle{maintext}{

  \sethead []
           [SAMUEL TAYLOR COLERIDGE]
           []
           {}
           {THE ANCIENT MARINER}
           {}

  \setfoot [][\arabic{page}][]
           {}{\arabic{page}}{}

}

\begin{document}

\ExampleTitle {Samuel Taylor Coleridge}
              {The Rime of the Ancient Mariner}
              {The English Parnassus\\
               Oxford University Press, 1909}
\cleardoublepage

\titulus{THE RIME OF THE ANCIENT MARINER\\[1\leading]
 \RelSize{-3}IN SEVEN PARTS}

\prosa
{\Facies \( {\RelSize[.85]{-2} \parindent 1em}
\(Facile credo, plures esse Naturas invisibiles quam visibiles in rerum
universitate. Sed horum omnium familiam quis nobis enarrabit? et gradus
et cognationes et discrimina et singulorum munera? Quid agunt? quae loca
habitant? Harum rerum notitiam semper ambivit ingenium humanum, nunquam
attigit. Juvat, interea, non diffiteor, quandoque in animo, tanquam in
tabula, majoris et melioris mundi imaginem contemplari: ne mens
assuefacta hodiernae vitae minutiis se contrahat nimis, et tota subsidat
in pusillas cogitationes. Sed veri\-tati interea invigilandum est,
modusque servandus, ut certa ab incertis, diem a nocte,
distinguamus---\textsc{T.Burnet}, \textit{Archaeol. Phil} p. 68.\)
}

\titulus {\RelSize{-2}ARGUMENT}

{\Facies \( {\RelSize[.85]{-1} \parindent 1em}
\(How a Ship having passed the Line was driven by storms to the cold
Country towards the South Pole; and how from thence she made her
course to the tropical Latitude of the Great Pacific Ocean; and of
the strange things that befell; and in what manner the Ancyent
Marinere came back to his own Country. [\oldstylenums{1798}]\)
}
\endprosa
\thispagestyle{empty}
\pagestyle{maintext}


\Locus \textus {\leftmargin \\ \leftmargin + \GlossWidth + \GlossSep}
\Modus         {\measure {\linewidth - (\GlossWidth + \GlossSep)}}

\Versus


\numerus{1}

\Part

\gloss {An ancient Mariner meet\-eth three Gallants bidden to
a wedding-feast, and detaineth one.}
  It is an ancient Mariner,
  And he stoppeth one of three.
  ``By thy long grey beard and glittering eye,
  Now wherefore stopp'st thou me?

  The bridegroom's doors are opened wide,                              
  And I am next of kin;
  The guests are met, the feast is set:
  May'st hear the merry din.''

  He holds him with his skinny hand,
  ``There was a ship,'' quoth he.                                       
  ``Hold off! unhand me, grey-beard loon!''
  Eftsoons his hand dropt he.

\gloss {The Wedding-Guest is spell\-bound by the eye of the old
seafaring man,}%
  He holds him with his glittering eye--
  The Wedding-Guest stood still,
  And listens like a three years' child:                              
  The Mariner hath his will.

\gloss {and constrained to hear his tale.}%
  The Wedding-Guest sat on a stone:
  He cannot choose but hear;
  And thus spake on that ancient man,
  The bright-eyed Mariner.                                            

  ``The ship was cheered, the harbor cleared,
  Merrily did we drop
  Below the kirk, below the hill,
  Below the lighthouse top.

\gloss {The Mariner tells how the ship sailed southward with a good
wind and fair weather, till it reached the Line.}%
  The sun came up upon the left,                                      
  Out of the sea came he!
  And he shone bright, and on the right
  Went down into the sea.

  Higher and higher every day,
  Till over the mast at noon--''                                       
\gloss {The Wedding-Guest heareth\\ the bridal\\ music; but\\ the
Mariner\\ continueth his tale.}%
  The Wedding-Guest here beat his breast,
  For he heard the loud bassoon.

  The bride hath paced into the hall,
  Red as a rose is she;
  Nodding their heads before her goes                                 
  The merry minstrelsy.

  The Wedding-Guest he beat his breast,
  Yet he cannot choose but hear;
  And thus spake on that ancient man,
  The bright-eyed Mariner.                                            


\gloss {The ship driven by a storm toward the south pole.}%
  ``And now the Storm-blast came, and he
  Was tyrannous and strong:
  He struck with his o'ertaking wings,
  And chased us south along.
{\strophaVI
  With sloping masts and dipping prow,                                
  As who pursued with yell and blow
  Still treads the shadow of his foe,
  And forward bends his head,
  The ship drove fast, loud roared the blast,
  And southward aye we fled.                                          
}
  And now there came both mist and snow,
  And it grew wondrous cold:
\gloss {The land of ice, and of fearful sounds where no living thing
was to be seen.}%
  And ice, mast-high, came floating by,
  As green as emerald.

  And through the drifts the snowy clifts                             
  Did send a dismal sheen:
  Nor shapes of men nor beasts we ken--
  The ice was all between.

  The ice was here, the ice was there,
  The ice was all around:                                             
  It cracked and growled, and roared and howled,
  Like noises in a swound!


\gloss {Till a great sea-bird, called the Albatross, came through the
snow-fog, and was received with great joy and hospitality.}%
  At length did cross an Albatross,
  Thorough the fog it came;
  As if it had been a Christian soul,                                 
  We hailed it in God's name.

  It ate the food it ne'er had eat,
  And round and round it flew.
  The ice did split with a thunder-fit;
  The helmsman steered us through!                                    

\gloss {And lo!\\ the Albatross proveth a bird of good omen, and
followeth the ship as it returned northward through fog and floating
ice.}%

  And a good south wind sprung up behind;
  The Albatross did follow,
  And every day, for food or play,
  Came to the mariners' hollo!

  In mist or cloud, on mast or shroud,                                
  It perched for vespers nine;
  Whiles all the night, through fog-smoke white,
  Glimmered the white moon-shine.''

\gloss {The ancient Mariner inhospitably killeth the pious bird of
good omen.}%
  ``God save thee, ancient Mariner!
  From the fiends, that plague thee thus!--                           
  Why look'st thou so?''--``With my cross-bow
  I shot the Albatross.


\Part

  The Sun now rose upon the right:
  Out of the sea came he,
  Still hid in mist, and on the left                                  
  Went down into the sea.

  And the good south wind still blew behind,
  But no sweet bird did follow,
  Nor any day for food or play
  Came to the mariners' hollo!                                        

\gloss {His shipmates cry out against the ancient Mariner, for
killing the bird of good luck.}%
{\strophaVI
   And I had done a hellish thing,
  And it would work 'em woe:
  For all averred, I had killed the bird
  That made the breeze to blow.
  Ah wretch! said they, the bird to slay,                             
  That made the breeze to blow!

\gloss {But when the fog cleared off, they justify the same, and thus
make themselves accomplices in the crime.}%
  Nor dim nor red, like God's own head,
  The glorious Sun uprist:
  Then all averred, I had killed the bird
  That brought the fog and mist.                                     
  'T was right, said they, such birds to slay,
  That bring the fog and mist.
}
\gloss {The fair breeze continues; the ship enters the Pacific Ocean,
and sails northward, even till it reaches the Line.}%
  The fair breeze blew, the white foam flew,
  The furrow followed free;
  We were the first that ever burst                                  
  Into that silent sea.

  Down dropt the breeze, the sails dropt down,
  'T was sad as sad could be;
\gloss {The ship hath been suddenly becalmed.}%
  And we did speak only to break
  The silence of the sea!                                            

  All in a hot and copper sky,
  The bloody Sun, at noon,
  Right up above the mast did stand,
  No bigger than the Moon.

  Day after day, day after day,                                      
  We stuck, nor breath nor motion;
  As idle as a painted ship
  Upon a painted ocean.

\gloss {And the Albatross begins to be avenged.}%
  Water, water, every where,
  And all the boards did shrink;                                     
  Water, water, every where
  Nor any drop to drink.

  The very deep did rot: O Christ!
  That ever this should be!
  Yea, slimy things did crawl with legs                              
  Upon the slimy sea.

  About, about, in reel and rout
  The death-fires danced at night;
  The water, like a witch's oils,
  Burnt green, and blue and white. 

\sgloss {A Spirit had followed them: one of the invisible inhabitants
         of this planet, neither departed souls nor angels, concerning
         whom the learned Jew, Josephus, and the Platonic Constantinopolitan,
         Michael Psellus, may be consulted. They are very numerous, and
         there is no climate or element without one or more.}
\area 
  And some in dreams assured were
  Of the Spirit that plagued us so;
  Nine fathom deep he had followed us
  From the land of mist and snow.
\endarea
  And every tongue, through utter drought,                           
  Was withered at the root;
  We could not speak, no more than if
  We had been choked with soot.

  Ah! well-a-day! what evil looks
\sgloss {The shipmates, in their sore distress, would fain throw the
whole guilt on the ancient Mariner: in sign whereof
they hang the dead sea-bird round his neck.}
\area
  Had I from old and young!                                          
  Instead of the cross, the Albatross
  About my neck was hung.
\endarea

\Part 

\gloss {The ancient Mariner beholdeth a sign in the element afar off.}%
{\strophaVI
  There passed a weary time. Each throat
  Was parched, and glazed each eye.
  A weary time! a weary time!                                        
  How glazed each weary eye,
  When looking westward, I beheld
  A something in the sky.
}
  At first it seemed a little speck,
  And then it seemed a mist;                                         
  It moved and moved, and took at last
  A certain shape, I wist.

  A speck, a mist, a shape, I wist!
  And still it neared and neared:
  As if it dodged a water-sprite,                                    
  It plunged and tacked and veered.


\gloss {At its nearer approach, it seemeth him to be a ship; and at a
dear ransom he freeth his speech from the bonds of thirst.}%
{\strophaV
  With throats unslaked, with black lips baked,
  We could nor laugh nor wail;
  Through utter drought all dumb we stood!
  I bit my arm, I sucked the blood,                                  
  And cried, A sail! a sail!

With throats unslaked, with black lips baked,
Agape they heard me call:
  Gramercy! they for joy did grin,
\gloss {A flash of joy;}%
  And all at once their breath drew in,                              
  As they were drinking all.
}

\gloss {And horror follows. For can it be a ship that comes onward
without wind or tide?}%
  See! see! (I cried) she tacks no more!
  Hither to work us weal;
  Without a breeze, without a tide,
  She steadies with upright keel!                                    

{\strophaVI
  The western wave was all a-flame.
  The day was well nigh done!
  Almost upon the western wave
  Rested the broad bright Sun;
  When that strange shape drove suddenly                             
  Betwixt us and the Sun;
}

\gloss {It seemeth him but the skeleton of a ship.}%
  And straight the Sun was flecked with bars,
  (Heaven's Mother send us grace!)
  As if through a dungeon-grate he peered
  With broad and burning face.                                       
  Alas (thought I, and my heart beat loud)
\gloss {And its ribs are seen as bars on the face of the setting Sun.}
  How fast she nears and nears!
  Are those her sails that glance in the Sun,
  Like restless gossameres?

{\strophaV
\gloss {The Spectre-Woman and her Deathmate, and no other on board the
skeleton-ship.}
  Are those her ribs through which the Sun                           
  Did peer, as through a grate?
  And is that Woman all her crew?
  Is that a Death? and are there two?
  Is Death that woman's mate?

\gloss {Like vessel, like crew!}%
  Her lips were red, her looks were free,                            
  Her locks were yellow as gold:
\gloss {Death and Life-in-Death have diced for the ship's crew, and
she (the latter) winneth the ancient Mariner.}%
  Her skin was as white as leprosy,
  The Night-mare Life-in-Death was she,
  Who thicks man's blood with cold.
}
  The naked hulk alongside came,                                     
  And the twain were casting dice;
  `The game is done! I've won! I've won!'
  Quoth she, and whistles thrice.

\gloss {No twilight within the courts of the Sun.}%
  The Sun's rim dips; the stars rush out;
  At one stride comes the dark;                                      
  With far-heard whisper, o'er the sea,
  Off shot the spectre-bark.

{\strophaIX
\gloss {At the rising of the moon.}%
  We listened and looked sideways up!
  Fear at my heart, as at a cup,
  My life-blood seemed to sip!                                       
  The stars were dim, and thick the night,
  The steersman's face by his lamp gleamed white;
  From the sails the dew did drip--
  Till clomb above the eastern bar
  The horned Moon, with one bright star                              
  Within the nether tip.
}

\gloss {One after another,}%
  One after one, by the star-dogged Moon,
  Too quick for groan or sigh,
  Each turned his face with a ghastly pang,
  And cursed me with his eye.                                        

\gloss {His shipmates drop down dead.}%
  Four times fifty living men,
  (And I heard nor sigh nor groan)
  With heavy thump, a lifeless lump,
  They dropped down one by one.

\gloss {But Life-in-Death begins her work on the ancient Mariner.}%
  The souls did from their bodies fly,--                             
  They fled to bliss or woe!
  And every soul, it passed me by,
  Like the whizz of my cross-bow!''



\Part

\gloss {The Wedding-Guest feareth that a Spirit is talking to him;}%
  ``I Fear thee, ancient Mariner!
  I fear thy skinny hand!                                            
  And thou art long, and lank, and brown,
  As is the ribbed sea-sand.

  I fear thee and thy glittering eye,
  And thy skinny hand, so brown.''--
\gloss {But the ancient Mariner assureth him of his bodily life, and
proceedeth to relate his horrible penance.}%
  ``Fear me not, fear not, thou wedding-guest!                        
  This body dropt not down.

  Alone, alone, all, all alone,
  Alone on the wide, wide sea!
  And never a saint took pity on
  My soul in agony.                                                  

\gloss {He despiseth the creatures of the calm.}%
  The many men, so beautiful!
  And they all dead did lie:
  And a thousand thousand slimy things
  Lived on; and so did I.

\gloss {And envieth that they should live, and so many lie dead.}%
  I looked upon the rotting sea,                                     
  And drew my eyes away;
  I looked upon the rotting deck,
  And there the dead men lay.

  I looked to heaven, and tried to pray;
  But or ever a prayer had gusht,                                    
  A wicked whisper came, and made
  My heart as dry as dust.

{\strophaV
  I closed my lids, and kept them close,
  And the balls like pulses beat;
  For the sky and the sea, and the sea and the sky                   
  Lay like a load on my weary eye,
  And the dead were at my feet.
}
\gloss {But the curse liveth for him in the eye of the dead men.}%
  The cold sweat melted from their limbs,
  Nor rot nor reek did they:
  The look with which they looked on me
  Had never passed away.
{\strophaVI
  An orphan's curse would drag to hell
  A spirit from on high;
  But oh! more horrible than that
  Is a curse in a dead man's eye!
  Seven days, seven nights, I saw that curse,
  And yet I could not die.
}
\sgloss {In his loneliness and fixedness he yearneth towards the
         journeying Moon, and the stars that still sojourn, yet still
         move onward; and every where the blue sky belongs to them,
         and is their appointed rest, and their native country and
         their own natural homes, which they enter unannounced, as
         lords that are certainly expected and yet there is a silent
         joy at their arrival.}
\area
  The moving Moon went up the sky,
  And nowhere did abide:
  Softly she was going up,
  And a star or two beside--
\endarea
 {\strophaV 
  Her beams bemocked the sultry main,
  Like April hoar-frost spread;
  But where the ship's huge shadow lay,
  The charmed water burnt alway
  A still and awful red.
 }

 {\strophaV
  \gloss {By the light of the Moon he beholdeth God's creatures of the
          great calm.}%
  Beyond the shadow of the ship,
  I watched the water-snakes:
  They moved in tracks of shining white,
  And when they reared, the elfish light
  Fell off in hoary flakes.

  Within the shadow of the ship
  I watched their rich attire:
  Blue, glossy green, and velvet black,
  They coiled and swam; and every track                              
  Was a flash of golden fire.
}
{\strophaVI

\gloss {Their beauty and their happiness.}%
  O happy living things! no tongue
  Their beauty might declare:
  A spring of love gushed from my heart,
\gloss {He blesseth them in his heart.}
  And I blessed them unaware:                                        
  Sure my kind saint took pity on me,
  And I blessed them unaware.
}

\gloss {The spell begins to break.}%
  The selfsame moment I could pray;
  And from my neck so free
  The Albatross fell off, and sank                                   
  Like lead into the sea.



\Part
{\strophaV
  Oh sleep! it is a gentle thing,
  Beloved from pole to pole!
  To Mary Queen the praise be given!
  She sent the gentle sleep from Heaven,                             
  That slid into my soul.
}
\gloss {By grace of the holy Mother, the ancient Mariner is refreshed
with rain.}%
  The silly buckets on the deck,
  That had so long remained,
  I dreamt that they were filled with dew;
  And when I awoke, it rained.                                       

  My lips were wet, my throat was cold,
  My garments all were dank;
  Sure I had drunken in my dreams,
  And still my body drank.

  I moved, and could not feel my limbs:                              
  I was so light--almost
  I thought that I had died in sleep,
  And was a blessed ghost.

\gloss {He heareth sounds and seeth strange sights and commotions in
the sky and the element.}%
  And soon I heard a roaring wind:
  It did not come anear;                                             
  But with its sound it shook the sails,
  That were so thin and sere.

{\strophaV
  The upper air burst into life!
  And a hundred fire-flags sheen,
  To and fro they were hurried about!                                
  And to and fro, and in and out,
  The wan stars danced between.
}
  And the coming wind did roar more loud,
  And the sails did sigh like sedge;
  And the rain poured down from one black cloud;                     
  The Moon was at its edge.

{\strophaV
  The thick black cloud was cleft, and still
  The Moon was at its side.
  Like waters shot from some high crag,
  The lightning fell with never a jag,                               
  A river steep and wide.
}
\gloss {The bodies of the ship's crew are inspired, and the ship
moves on;}%
  The loud wind never reached the ship,
  Yet now the ship moved on!
  Beneath the lightning and the Moon
  The dead men gave a groan.                                         

  They groaned, they stirred, they all uprose,
  Nor spake, nor moved their eyes;
  It had been strange, even in a dream,
  To have seen those dead men rise.

{\strophaVI
  The helmsman steered, the ship moved on;                           
  Yet never a breeze up blew;
  The mariners all 'gan work the ropes,
  Where they were wont to do;
  They raised their limbs like lifeless tools--
  We were a ghastly crew.                                            
}
  The body of my brother's son
  Stood by me, knee to knee:
  The body and I pulled at one rope,
  But he said nought to me.''

\gloss {But not by the souls of the men, nor by daemons of earth or
middle air, but by a blessed troop of angelic spirits, sent down by the
invocation of the guardian saint.}%
{\strophaV
  ``I fear thee, ancient Mariner!''                                    
  ``Be calm, thou Wedding-Guest!
  'T was not those souls that fled in pain,
  Which to their corses came again,
  But a troop of spirits blest:
}
  For when it dawned--they dropped their arms,
  And clustered round the mast;                                      
  Sweet sounds rose slowly through their mouths,
  And from their bodies passed.

  Around, around, flew each sweet sound,
  Then darted to the Sun;                                            
  Slowly the sounds came back again,
  Now mixed, now one by one.

{\strophaV
  Sometimes a-dropping from the sky
  I heard the sky-lark sing;
  Sometimes all little birds that are,                               
  How they seemed to fill the sea and air
  With their sweet jargoning!
}
  And now 't was like all instruments,
  Now like a lonely flute;
  And now it is an angel's song,                                     
  That makes the heavens be mute.

{\strophaVI
  It ceased; yet still the sails made on
  A pleasant noise till noon,
  A noise like of a hidden brook
  In the leafy month of June,                                        
  That to the sleeping woods all night
  Singeth a quiet tune.
}
  Till noon we quietly sailed on,
  Yet never a breeze did breathe:
  Slowly and smoothly went the ship,                                 
  Moved onward from beneath.
\newpage
{\strophaVI
\gloss {The lonesome Spirit from the south-pole carries on the ship
as far as the Line, in obedience to the angelic troop, but still
requireth vengeance.}%
  Under' the keel nine fathom deep,
  From the land of mist and snow,
  The spirit slid: and it was he
  That made the ship to go.                                          
  The sails at noon left off their tune,
  And the ship stood still also.

  The Sun, right up above the mast,
  Had fixed her to the ocean:
  But in a minute she 'gan stir,                                     
  With a short uneasy motion--
  Backwards and forwards half her length
  With a short uneasy motion.
}
  Then like a pawing horse let go,
  She made a sudden bound:                                           
  It flung the blood into my head,
  And I fell down in a swound.


{\strophaV
\gloss {The Polar Spirit's fellow-daemons, the invisible inhabitants
of the element, take part in his wrong; and two of them relate, one to
the other, that penance long and heavy for the ancient Mariner hath been
accorded to the Polar Spirit, who returneth southward.}%
  How long in that same fit I lay,
  I have not to declare;
  But ere my living life returned,                                   
  I heard and in my soul discerned
  Two voices in the air.
}
  `Is it he?' quoth one, `Is this the man?
  By him who died on cross,
  With his cruel bow he laid full low                                
  The harmless Albatross.

  The spirit who bideth by himself
  In the land of mist and snow,
  He loved the bird that loved the man
  Who shot him with his bow?'                                        

  The other was a softer voice,
  As soft as honey-dew:
  Quoth he, `The man hath penance done,
  And penance more will do.'



\Part

\voice{First}

  `But tell me, tell me! speak again, 
  Thy soft response renewing--
  What makes that ship drive on so fast?
  What is the ocean doing?'

\voice{Second}

  `Still as a slave before his lord,
  The ocean hath no blast; 
  His great bright eye most silently
  Up to the Moon is cast--

  If he may know which way to go;
  For she guides him smooth or grim.
  See, brother, see! how graciously 
  She looketh down on him.'

\voice{First}

\gloss {The Mariner hath been cast into a trance; for the angelic
power causeth the vessel to drive northward faster than human life could
endure.}%
  `But why drives on that ship so fast?
  Without or wave or wind?'

\voice{Second}

  `The air is cut away before,
  And closes from behind. 
  Fly, brother, fly! more high, more high!
  Or we shall be belated:
  For slow and slow that ship will go,
  When the Mariner's trance is abated.'

\gloss {The supernatural motion is retarded; the Mariner awakes, and
his penance begins anew.}%
  I woke, and we were sailing on                                     
  As in a gentle weather:
  'T was night, calm night, the moon was high,
  The dead men stood together.

  All stood together on the deck,
  For a charnel-dungeon fitter:                                      
  All fixed on me their stony eyes,
  That in the Moon did glitter.

  The pang, the curse, with which they died,
  Had never passed away:
  I could not draw my eyes from theirs,                              
  Nor turn them up to pray.

\gloss {The curse is finally expiated.}%
  And now this spell was snapt: once more
  I viewed the ocean green,
  And looked far forth, yet little saw
  Of what had else been seen--                                       

{\strophaVI
  Like one, that on a lonesome road
  Doth walk in fear and dread,
  And having once turned round walks on,
  And turns no more his head;
  Because he knows, a frightful fiend                                
  Doth close behind him tread.
}
  But soon there breathed a wind on me,
  Nor sound nor motion made:
  Its path was not upon the sea,
  In ripple or in shade.                                             

  It raised my hair, it fanned my cheek
  Like a meadow-gale of spring--
  It mingled strangely with my fears,
  Yet it felt like a welcoming.

  Swiftly, swiftly flew the ship,                                    
  Yet she sailed softly too:
  Sweetly, sweetly blew the breeze--
  On me alone it blew.

\gloss {And the ancient Mariner beholdeth his native country.}%
  Oh! dream of joy! is this indeed
  The light-house top I see?                                         
  Is this the hill? is this the kirk?
  Is this mine own countree?

  We drifted o'er the harbor-bar,
  And I with sobs did pray--
  O let me be awake, my God!                                         
  Or let me sleep alway.

  The harbor-bay was clear as glass,
  So smoothly it was strewn!
  And on the bay the moonlight lay,
  And the shadow of the Moon.                                        

  The rock shone bright, the kirk no less,
  That stands above the rock:
  The moonlight steeped in silentness
  The steady weathercock.

  And the bay was white with silent light                            
 \gloss {The angelic spirits leave the dead bodies,}%
  Till rising from the same,
  Full many shapes, that shadows were,
  In crimson colors came.

\gloss {And appear in their own forms of light.}%
  A little distance from the prow
  Those crimson shadows were:                                        
  I turned my eyes upon the deck--
  Oh, Christ! what saw I there!

  Each corse lay flat, lifeless and flat,
  And, by the holy rood!
  A man all light, a seraph-man,                                     
  On every corse there stood.

  This seraph-band, each waved his hand:
  It was a heavenly sight!
  They stood as signals to the land,
  Each one a lovely light;                                           

  This seraph-band, each waved his hand,
  No voice did they impart--
  No voice; but oh! the silence sank
  Like music on my heart.

  But soon I heard the dash of oars,                                 
  I heard the Pilot's cheer;
  My head was turned perforce away,
  And I saw a boat appear.

  The Pilot and the Pilot's boy,
  I heard them coming fast:                                          
  Dear Lord in Heaven! it was a joy
  The dead men could not blast.

{\strophaVI
  I saw a third--I heard his voice:
  It is the Hermit good!
  He singeth loud his godly hymns                                    
  That he makes in the wood.
  He'll shrieve my soul, he'll wash away
  The Albatross's blood.
}


\Part

{\strophaV
\gloss {The Hermit of the Wood,}%
  This Hermit good lives in that wood
  Which slopes down to the sea.                                      
  How loudly his sweet voice he rears!
  He loves to talk with marineres
  That come from a far countree.
}
  He kneels at morn, and noon, and eve--
  He hath a cushion plump:                                           
  It is the moss that wholly hides
  The rotted old oak-stump.

  The skiff-boat neared: I heard them talk,
  `Why, this is strange, I trow!
  Where are those lights, so many and fair,                          
  That signal made but now?'

{\strophaVI
\gloss {Approacheth the ship with wonder.}%
  `Strange, by my faith!' the Hermit said--
  `And they answered not our cheer!
  The planks looked warped! and see those sails,
  How thin they are and sere!                                        
  I never saw aught like to them,
  Unless perchance it were
}
{\strophaV
  Brown skeletons of leaves that lag
  My forest-brook along;
  When the ivy-tod is heavy with snow,                               
  And the owlet whoops to the wolf below,
  That eats the she-wolf's young.'
}
  `Dear Lord! it hath a fiendish look--
  (The Pilot made reply)
  I am a-feared'--`Push on, push on!'                                
  Said the Hermit cheerily.

  The boat came closer to the ship,
  But I nor spake nor stirred;
  The boat came close beneath the ship,
  And straight a sound was heard.                                    

\gloss {The ship suddenly sinketh.}%
  Under the water it rumbled on,
  Still louder and more dread:
  It reached the ship, it split the bay;
  The ship went down like lead.

{\strophaVI
\gloss {The ancient Mariner is saved in the Pilot's boat.}%
  Stunned by that loud and dreadful sound,                           
  Which sky and ocean smote,
  Like one that hath been seven days drowned
  My body lay afloat;
  But swift as dreams, myself I found
  Within the Pilot's boat.                                           
}
  Upon the whirl, where sank the ship,
  The boat spun round and round;
  And all was still, save that the hill
  Was telling of the sound.

  I moved my lips--the Pilot shrieked                                
  And fell down in a fit;
  The holy Hermit raised his eyes,
  And prayed where he did sit.

{\strophaVI
  I took the oars: the Pilot's boy,
  Who now doth crazy go,                                             
  Laughed loud and long, and all the while
  His eyes went to and fro.
  `Ha! ha!' quoth he, `full plain I see,
  The Devil knows how to row.'
}
  And now, all in my own countree,                                   
  I stood on the firm land!
  The Hermit stepped forth from the boat,
  And scarcely he could stand.

\newpage
\gloss {The ancient Mariner earnestly entreateth the Hermit to
shrieve him; and the penance of life falls on him.}%
  `O shrieve me, shrieve me, holy man!'
  The Hermit crossed his brow.                                       
  `Say quick,' quoth he, `I bid thee say--
  What manner of man art thou?'

  Forthwith this frame of mine was wrenched
  With a woful agony,
  Which forced me to begin my tale;                                  
  And then it left me free.

\gloss {And ever and anon throughout his future life an agony
constraineth him to travel from land to land,}%
  Since then, at an uncertain hour,
  That agony returns:
  And till my ghastly tale is told,
  This heart within me burns.                                        

{\strophaV
  I pass, like night, from land to land;
  I have strange power of speech;
  That moment that his face I see,
  I know the man that must hear me:
  To him my tale I teach.                                            
}
{\strophaVI
  What loud uproar bursts from that door!
  The wedding-guests are there:
  But in the garden-bower the bride
  And bride-maids singing are:
  And hark the little vesper bell,                                   
  Which biddeth me to prayer!
}

  O Wedding-Guest! this soul hath been
  Alone on a wide, wide sea:
  So lonely 't was, that God himself
  Scarce seemed there to be.                                         

  O sweeter than the marriage-feast,
  'T is sweeter far to me,
  To walk together to the kirk
  With a goodly company!--

{\strophaV
  To walk together to the kirk,                                      
  And all together pray,
  While each to his great Father bends,
  Old men, and babes, and loving friends
  And youths and maidens gay!
}
\gloss {And to teach, by his own example, love and reverence to all
things that God made and loveth.}%
  Farewell, farewell! but this I tell                                
  To thee, thou Wedding-Guest!
  He prayeth well, who loveth well
  Both man and bird and beast.

  He prayeth best, who loveth best
  All things both great and small;                                   
  For the dear God who loveth us,
  He made and loveth all.''

  The Mariner, whose eye is bright,
  Whose beard with age is hoar,
  Is gone: and now the Wedding-Guest                                 
  Turned from the bridegroom's door.

  He went like one that hath been stunned,
  And is of sense forlorn:
  A sadder and a wiser man,
  He rose the morrow morn.                                           

\endVersus
\end{document}
