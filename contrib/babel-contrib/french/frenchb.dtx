%\CheckSum{3486}
%
%\iffalse
%    Tell the \LaTeX\ system who we are and write an entry on the
%    transcript.
%<*dtx>
\ProvidesFile{frenchb.dtx}
%</dtx>
% \changes{v2.1d}{2008/05/04}{Argument of \cs{ProvidesLanguage} changed
%    from `french’ to `frenchb’, otherwise \cs{listfiles} prints
%    no date/version information.  The bug with \cs{listfiles}
%    (introduced in v.~1.5!), was pointed out by Ulrike Fischer.}
%<ldf>\ProvidesLanguage{frenchb}
%\ProvidesFile{frenchb.dtx}
%<*!lua>
         [2017/04/30 v3.3a French support from the babel system]
%</!lua>
%<*internal>
\iffalse
%</internal>
%<*lua>
--
--  File `frenchb.lua’ generated from frenchb.dtx [2017/04/30 v3.3a]
--
--  Copyright © 2014-2017 Daniel Flipo
--  <daniel (dot) flipo (at) free (dot) fr>
--  License LPPL: see frenchb.ldf.
--
%</lua>
%<*internal>
\fi
%</internal>
%<*dtx>
%%
%%  File `frenchb.dtx’
%</dtx>
%<*ldf>
%%
%%  File `frenchb.ldf’
%</ldf>
%<*dtx|ldf>
%%  Babel package for LaTeX version 2e
%%  Copyright © 1996 - 2017
%%            by Daniel Flipo <daniel (dot) flipo (at) free (dot) fr>
%%
%</dtx|ldf>
%<*dtx>
%%  This file is part of the babel system, it provides the source
%%  code for the French language definition file.
%%
%</dtx>
%<*filedriver>
\documentclass[a4paper]{ltxdoc}
\usepackage[left=42mm,right=42mm,top=40mm,bottom=40mm]{geometry}
\usepackage[utf8]{inputenc}
\usepackage{bera,textcomp}
\renewcommand{\rmdefault}{fvs}
\usepackage{array}
\usepackage[dvipsnames]{xcolor}
\def\PrintMacroName#1{%
  \strut\MacroFont\color{Sepia}\normalsize\string#1}
\usepackage[expansion=true,protrusion=true]{microtype}
\def\ColorArg{\color{PineGreen}}
\usepackage{url}
\usepackage{hypdoc}
\hypersetup{colorlinks,urlcolor=blue,unicode}
\newcommand*\babel{\textsf{babel}}
\newcommand*\frenchb{\textsf{babel-french}}
\newcommand*\Lopt[1]{\texttt{#1}}
\newcommand*\opt[1]{\texttt{#1}}
\newcommand*\file[1]{\texttt{#1}}
\newcommand*\cls[1]{\texttt{#1}}
\newcommand*\pkg[1]{\texttt{#1}}
\newcommand*\env[1]{\texttt{#1}}
\newcommand*\fbo[1]{\texttt{\ColorArg #1}}
\newcommand*\fbsetup[1]{\cs{frenchsetup\{\fbo{#1}\}}}
\renewcommand*\descriptionlabel[1]{%
   \hspace\labelsep \normalfont \fbo{#1}}
\providecommand*\eTeX{\leavevmode\hbox{$\varepsilon$}-\TeX}
\let\up\textsuperscript
\newcommand*{\FRguillspace}{\penalty 10000
                            \hskip.8\fontdimen2\font
                              plus.3\fontdimen3\font
                             minus.8\fontdimen4\font}
\hyphenation{Lua-TeX Lua-LaTeX Xe-LaTeX}
\lefthyphenmin=2 \righthyphenmin=3
\RecordChanges
\AtEndDocument{%
   \clearpage
   \section{Change History}%
   \GlossaryPrologue{}%
   Changes are listed in reverse order (latest first) and limited to
   \frenchb{}~v3.
   \PrintChanges
}
\begin{document}
\setlength{\parindent}{0pt}
\GetFileInfo{frenchb.dtx}
\begin{center}
  \textbf{\Large A Babel language definition file for French}\\[2mm]^^A\]
  \textbf{\large frenchb.dtx \fileversion, \filedate}\\[3mm]^^A\]
  Daniel Flipo\\
  \texttt{daniel.flipo@free.fr}
\end{center}
\vspace{2\baselineskip}
\thispagestyle{empty}
\tableofcontents
\clearpage
\DocInput{frenchb.dtx}
\end{document}
%</filedriver>
%\fi
%
%  \section{The French language}
%
%    The file \file{\filename}\footnote{The file described in this
%    section has version number \fileversion\ and was last revised on
%    \filedate.}, defines all the language definition macros for the
%    French language.
%
%    Customisation for the French language is achieved following the
%    book ``Lexique des règles typographiques en usage à
%    l’Imprimerie Nationale’’ troisième édition (1994),
%    ISBN-2-11-081075-0.
%
%    First version released: 1.1 (May 1996) as part of
%    \babel-3.6beta. Version~2.0a was released in February 2007 and
%    version~3.0a in February 2014.
%
%    \frenchb{} has been improved using helpful suggestions from many
%    people, mainly from Jacques André, Michel Bovani, Thierry Bouche,
%    Vincent Jalby and Denis Bitouzé.  Thanks to all of them!
%
%    \LaTeX-2.09 is no longer supported.
%    This new version (3.x) has been designed to be used only with
%    \LaTeXe{} and Plain formats based on TeX, pdfTeX, LuaTeX or XeTeX
%    engines.
%
%    Changes between version 3.0 and \fileversion{} are listed in
%    subsection~\ref{ssec-changes} p.~\pageref{ssec-changes}.
%
%    An extensive documentation is available in French here:\\
%    \url{http://daniel.flipo.free.fr/frenchb}
%
%  \subsection{Basic interface}
%
%    In a multilingual document, some typographic rules are language
%    dependent, i.e.\ spaces before `high punctuation’ (|:| |;| |!|
%    |?|) in French, others modify the general layout (i.e.\ layout of
%    lists, footnotes, indentation of first paragraphs of sections) and
%    should apply to the whole document.
%
%    \frenchb{} takes account of \babel’s \emph{main language} defined
%    as the \emph{last} option at \babel’s loading.  When French is
%    not \babel’s main language, \frenchb{} does not alter the general
%    layout of the document (even in parts where French is the current
%    language): the layout of lists, footnotes, indentation of first
%    paragraphs of sections are not customised by \frenchb.
%
%    When French is loaded as the last option of \babel, \frenchb{}
%    makes the following changes to the global layout, \emph{both in
%    French and in all other languages}\footnote{%
%       For each item, hooks are provided to reset standard
%       \LaTeX{} settings or to emulate the behavior of former versions
%       of \frenchb{} (see command \fbsetup{},
%       section~\ref{ssec-custom} p.~\pageref{ssec-custom}).}:
%    \begin{enumerate}
%    \item the first paragraph of each section is indented
%          (\LaTeX{} only);
%    \item the default items in itemize environment are set to `---’
%          instead of `\textbullet’, and all vertical spacing and glue
%          is deleted; it is possible to change `---’ to something else
%          (`--’ for instance) using \fbsetup{} (see
%          section~\ref{ssec-custom} p.~\pageref{ssec-custom});
%    \item vertical spacing in general \LaTeX{} lists is
%          shortened;
%    \item footnotes are displayed ``à la française’’.
%    \item the separator following the table or figure number in
%          captions is printed as `~--~’ instead of `:~’; for changing
%          this see~\ref{sssec-captions} p.~\pageref{sssec-captions}.
%    \end{enumerate}
%
%    Regarding local typography, the command |\selectlanguage{french}|
%    switches to the French language\footnote{%
%      \cs{selectlanguage\{francais\}} and
%      \cs{selectlanguage\{frenchb\}} are no longer supported.},
%    with the following effects:
%    \begin{enumerate}
%    \item French hyphenation patterns are made active;
%    \item `high punctuation’ characters (|:| |;| |!| |?|)
%           automatically add correct spacing in French; this is
%           achieved using callbacks in Lua(La)TeX or `XeTeXinterchar’
%           mechanism in Xe(La)TeX; with TeX’82 and pdf(La)TeX these
%           four characters are made active in the whole document;
%    \item |\today| prints the date in French;
%    \item the caption names are translated into French
%          (\LaTeX{} only).  For customisation of caption names see
%          section~\ref{sssec-captions} p.~\pageref{sssec-captions}.
%    \item the space after |\dots| is removed in French.
%    \end{enumerate}
%
%    Some commands are provided by \frenchb{} to make typesetting
%    easier:
%    \begin{enumerate}
%    \item French quotation marks can be entered using the commands
%          |\og| and |\fg| which work in \LaTeXe and Plain\TeX,
%          their appearance depending on what is available to draw
%          them; even if you use \LaTeXe{} \emph{and} |T1|-encoding,
%          you should refrain from entering them as
%          |<<~French quotation~>>|: |\og| and |\fg| provide
%          better horizontal spacing (controlled by |\FBguillspace|).
%          If French quote characters are available on your keyboard,
%          you can use them, to get proper spacing in \LaTeXe{} see
%          option \fbo{og=«, fg=»} p.~\pageref{option-og-fg}.
%
%          |\og| and |\fg| can be used outside French, they typeset
%          then English quotes `` and ’’.
%
%          A new command |\frquote{}| has been added in version~3.1 to
%          enter French quotations.  |\frquote{|\textit{texte}|}| is
%          equivalent to |\og| \textit{texte} |\fg{}| for short
%          quotations.  For quotations spreading over more than one
%          paragraph, |\frquote| will add at the beginning of every
%          paragraph of the quotation either an opening French
%          guillemet («), or a closing one (») or nothing depending
%          on option \fbo{EveryParGuill=open} or \fbo{=close} or
%          \fbo{=none}, see p.~\pageref{everyparguill}.\label{frquote}
%
%          |\frquote| is recommended to enter embedded quotations ``à la
%          française’’, several variants are provided through options.
%          \begin{itemize}
%          \item with all engines: the inner quotation
%            is surrounded by double quotes (``\textit{texte}’’)
%            unless option \fbo{InnerGuillSingle=true}, then
%            a) the inner quotation is printed as
%            \guilsinglleft\FRguillspace\textit{texte}\FRguillspace
%            \guilsinglright{} and b) if the inner quotation
%            spreads over more than one paragraph, every paragraph
%            included in the inner quotation starts with
%            a \guilsinglleft{} or a \guilsinglright{} or nothing,
%            depending on option \fbo{EveryParGuill=open} (default)
%            or \fbo{=close} or \fbo{=none}.
%          \item with LuaTeX based engines, it is possible to add a
%            French opening or closing guillemet (« or ») at the
%            beginning of every line of the inner quotation using
%            option \fbo{EveryLineGuill=open} or \fbo{=close}; note
%            that with any of these options, the inner quotation is
%            surrounded by French guillemets (« and ») regardless
%            option \fbo{InnerGuillSingle};
%            the default is \fbo{EveryLineGuill=none} so that
%            |\frquote{}| behaves as with non-LuaTeX engines.
%          \end{itemize}
%          A starred variant |\frquote*| is meant for inner quotations
%          which end together with the outer one: using |\frquote*| for
%          the inner quotation will print only one closing quote
%          character (the outer one) as recommended by the French
%          `Imprimerie Nationale’.
%
%    \item A command |\up| is provided to typeset superscripts like
%          |M\up{me}| (abbreviation for ``Madame’’), |1\up{er}| (for
%          ``premier’’).  Other commands are also provided for
%          ordinals: |\ier|, |\iere|, |\iers|, |\ieres|, |\ieme|,
%          |\iemes| (|3\iemes| prints 3\up{es}).
%          All these commands take advantage of real superscript
%          letters when they are available in the current font.
%
%    \item Family names should be typeset in small capitals and never
%          be hyphenated, the macro |\bsc| (boxed small caps) does
%          this, e.g., |L.~\bsc{Lamport}| will print the same as
%          |L.~\mbox{\textsc{Lamport}}|.  Note that composed names
%          (such as Dupont-Durant) may now be hyphenated on explicit
%          hyphens, this differs from \frenchb~v.~1.x.
%
%    \item Commands |\primo|, |\secundo|, |\tertio| and |\quarto|
%          print 1\up{o}, 2\up{o}, 3\up{o}, 4\up{o}.
%          |\FrenchEnumerate{6}| prints 6\up{o}.
%    \item Abbreviations for ``Numéro(s)’’ and ``numéro(s)’’
%          (N\up{o} N\up{os} n\up{o} and n\up{os}~) are obtained
%          via the commands |\No|, |\Nos|, |\no|, |\nos|.
%
%    \item Two commands are provided to typeset the symbol for
%          ``degré’’: |\degre| prints the raw character and
%          |\degres| should be used to typeset temperatures (e.g.,
%          ``|20~\degres C|’’ with an nobreak space), or for
%          alcohols’’ strengths (e.g., ``|45\degres|’’ with \emph{no}
%          space in French).
%
%    \item In math mode the comma has to be surrounded with
%          braces to avoid a spurious space being inserted after it,
%          in decimal numbers for instance (see the \TeX{}book p.~134).
%          The command |\DecimalMathComma| makes the comma behave as an
%          ordinary character \emph{when the current language is French}
%          (no space added);
%          as a counterpart, if |\DecimalMathComma| is active, an
%          explicit space has to be added in lists and intervals:
%          |$[0,\ 1]$|, |$(x,\ y)$|.  |\StandardMathComma| switches
%          back to the standard behaviour of the comma in French.
%
%          The \pkg{icomma} package is an alternative workaround.
%
%    \item A command |\nombre| was provided in 1.x versions to easily
%          format numbers in slices of three digits separated either
%          by a comma in English or with a space in French; |\nombre|
%          is now mapped to |\numprint| from \file{numprint.sty}, see
%          \file{numprint.pdf} for more information.
%
%    \item \frenchb{} has been designed to take advantage of the
%          \pkg{xspace} package if present: adding
%          |\usepackage{xspace}| in the preamble will force macros like
%          |\fg|, |\ier|, |\ieme|, |\dots|, \dots, to respect the
%          spaces you type after them, for instance typing
%          `|1\ier juin|’ will print `1\up{er} juin’
%           (no need for a forced space after |1\ier|).
%    \end{enumerate}
%
%  \subsection{Customisation}
%  \label{ssec-custom}
%
%    Customisation of \frenchb{} relies on command \fbsetup{}
%    (formerly called |\frenchbsetup{}|, the latter name will be kept
%    for ever to ensure backwards compatibility), options are entered
%    using the \file{keyval} syntax.
%    The command \fbsetup{} is to appear in the preamble only
%    (after loading \babel).
%
%     \subsubsection[\textbackslash frenchsetup{}]
%     {\fbsetup{options}}
%  \label{sssec-frenchbsetup}
%
%    \vspace{.5\baselineskip}
%    |\frenchsetup{}| and |\frenchbsetup{}| are synonymous;
%    the latter should be preferred as the language name for
%    French in \babel{} is no longer |frenchb| but |french|.
%
%    \fbsetup{ShowOptions} prints all available options to
%    the \file{.log} file, it is just meant as a remainder of the
%    list of offered options.  As usual with \file{keyval} syntax,
%    boolean options (as \fbo{ShowOptions}) can be entered as
%    \fbo{ShowOptions=true} or just \fbo{ShowOptions}, the
%    \fbo{=true} part can be omitted.
%
%    \vspace{.5\baselineskip}
%    The other options are listed below.  Their default value is
%    shown between braces, sometimes followed be a `\texttt{*}’.
%    The `\texttt{*}’ means that the default shown applies when
%    \frenchb{} is loaded as the \emph{last} option of \babel{}
%    ---\babel’s \emph{main language}---, and is toggled otherwise.
%    \begin{description}
%    \item [StandardLayout=true (false*)] forces \frenchb{} not to
%      interfere with the layout: no action on any kind of lists,
%      first paragraphs of sections are not indented (as in English),
%      no action on footnotes.  This option can be used to avoid
%      conflicts with classes or packages which customise lists or
%      footnotes.\par
%      When French is not the main language, \fbo{StandardLayout=false}
%      can be misused to ensure French typography (in French only).
%      This is a \emph{bad practice}: the document layout should not
%      be altered by language switches.
%    \item [GlobalLayoutFrench=false (true*)] should no longer be
%      used; it was intended to emulate, when French is the main
%      language, what prior versions of \frenchb{} (pre-2.2) did:
%      lists, and first paragraphs of sections would be displayed
%      the standard way in other languages than French, and ``à la
%      française’’ in French.  Note that the layout of footnotes is
%      language independent anyway (see below \fbo{FrenchFootnotes}
%      and \fbo{AutoSpaceFootnotes}).
%    \item [ReduceListSpacing=false (true*)]; \frenchb{} reduces the
%      values of the vertical spaces used in the \emph{all} list
%      environments in French (this includes itemize, enumerate,
%      description, but also abstract, quote, quotation and verse and
%      possibly others).  Setting this option to \fbo{false}
%      reverts to the standard settings of the \env{list}
%      environment.
%    \item [ListOldLayout=true (false)]; starting with version~2.6a,
%      the layout of lists has changed regarding leftmargins’ sizes
%      and default itemize label (`---’ instead of `--’ up to~2.5k).
%      This option, provided for backward compatibility, displays
%      lists as they were up to version~2.5k.
%    \item [CompactItemize=false (true*)]; should no longer be used
%      (kept only for backward compatibility), it is replaced by the
%      next two options.
%    \item [StandardItemizeEnv=true (false*)]; \frenchb{} redefines
%      the \env{itemize} environment to suppress any vertical space
%      between items of \env{itemize} lists in French and customises
%      left margins.  Setting this option to \fbo{false} reverts to
%      the standard definition of \env{itemize}.
%    \item [StandardEnumerateEnv=true (false*)]; starting with
%      version~2.6 \frenchb{} redefines the \env{enumerate} and
%      \env{description} environments to make left margins match those
%      of the French version of \env{itemize} lists.  Setting this
%      option to \fbo{false} reverts to the standard definition of
%      \env{enumerate} and \env{description}.
%    \item [StandardItemLabels=true (false*)] when set to \fbo{true}
%      this option prevents \frenchb{} from changing the labels in
%      \env{itemize} lists in French.
%    \item [ItemLabels=\cs{textbullet}, \cs{textendash},
%           \cs{ding\{43\}},...(\cs{textemdash}*)];\\
%      when \fbo{StandardItemLabels=false} (the default), this option
%      enables to choose the label used in French \env{itemize} lists
%      for all levels.  The next four options do the same but each
%      one for a specific level only.  Note that the example
%      |\ding{43}| requires |\usepackage{pifont}|.
%    \item [ItemLabeli=\cs{textbullet}, \cs{textendash},
%           \cs{ding\{43\}},...(\cs{textemdash}*)]
%    \item [ItemLabelii=\cs{textbullet}, \cs{textendash},
%           \cs{ding\{43\}},...(\cs{textemdash}*)]
%    \item [ItemLabeliii=\cs{textbullet}, \cs{textendash},
%           \cs{ding\{43\}},..(\cs{textemdash}*)]
%    \item [ItemLabeliv=\cs{textbullet}, \cs{textendash},
%           \cs{ding\{43\}},...(\cs{textemdash}*)]
%    \item [StandardLists=true (false*)] forbids \frenchb{} to
%      customise any kind of list.  Try the option \fbo{StandardLists}
%      in case of conflicts with classes or packages that customise
%      lists too.  This option is just a shorthand setting all four
%      options \fbo{ReduceListSpacing=false},
%      \fbo{StandardItemizeEnv=true}, \fbo{StandardEnumerateEnv=true}
%      and \fbo{StandardItemLabels=true}.
%    \item [IndentFirst=false (true*)]; set this option to \fbo{false}
%      if you do not want \frenchb{} to  force indentation of the
%      first paragraph of sections.  When French is the main language,
%      this option applies to all languages.
%    \item [FrenchFootnotes=false (true*)] reverts to the standard
%      layout of footnotes.  By default \frenchb{} typesets leading
%      numbers as `1.\hspace{.5em}’ instead of `\up{1}’, but
%      has no effect on footnotes numbered with symbols (as in the
%      |\thanks| command).  Two commands |\StandardFootnotes|
%      and |\FrenchFootnotes| are available to change the layout of
%      footnotes locally; |\StandardFootnotes| can help when some
%      footnotes are numbered with letters (inside minipages for
%      instance).
%    \item [AutoSpaceFootnotes=false (true*)]; by default \frenchb{}
%      adds a thin space in the running text before the number or
%      symbol calling the footnote.  Making this option \fbo{false}
%      reverts to the standard setting (no space added).
%    \item [FrenchSuperscripts=false (true)]; then
%      |\up=\textsuperscript|. % (option added in version 2.1).
%      Should only be made \fbo{false} to recompile documents written
%      before 2008 without changes: by default |\up| now relies on
%      |\fup| designed to produce better looking superscripts.
%    \item [AutoSpacePunctuation=false (true)]; in French, the user
%      \emph{should} input a space before the four characters `|:;!?|’
%      but as many people forget about it (even among native French
%      writers!), the default behaviour of \frenchb{} is to
%      automatically typeset nobreak spaces the width of which is
%      either |\FBthinspace| (defauts to a thin space) before `|;|’
%      `|!|’ `|?|’ or |\FBcolonspace| (defauts to |\space|) before
%      `|:|’; the defaults follow the French `Imprimerie Nationale’s
%      recommendations.  This is convenient in most cases but can lead
%      to addition of spurious spaces in URLs, in MS-DOS paths or in
%      timetables (10:55), except if they are typed in |\texttt| or
%      verbatim mode.  When the current font is a monospaced
%      (typewriter) font, no spurious space is added in that case%
%      \footnote{Unless option \fbo{OriginalTypewriter} is set,
%        \cs{ttfamily} is redefined in French to switch off space
%        tuning, see below.},
%      so the default behaviour of of \frenchb{} in that area should
%      be fine in most circumstances.
%
%      Choosing \fbo{AutoSpacePunctuation=false} will ensure that
%      a proper space is added before `|:;!?|’ \emph{if and only if}
%      a (normal) space has been typed in.  Those who are unsure
%      about their typing in this area should stick to the default
%      option and use the provided |\NoAutoSpacing| command
%      inside a group in case an unwanted space is added by \frenchb{}
%      (i.e.\ |{\NoAutoSpacing 10:55}|).
%    \item [ThinColonSpace=true (false)] changes the inter-word
%      unbreakable space added before the colon `:’ to a thin space,
%      so that the same amount of space is added before any of the
%      four `high punctuation’ characters.  The default setting is
%      supported by the French `Imprimerie Nationale’.
%    \item [OriginalTypewriter=true (false)] prevents any
%      customisation of |\ttfamily| and |\texttt{}| in French.
%    \item [LowercaseSuperscripts=false (true)]; by default \frenchb{}
%      inhibits the uppercasing of superscripts (for instance when
%      they are moved to page headers).  Making this option
%      \fbo{false} will disable this behaviour (not recommended).
%    \item [PartNameFull=false (true)]; when true, \frenchb{} numbers
%      the title of |\part{}| commands as ``Première partie’’,
%      ``Deuxième partie’’ and so on.  With some classes which
%      change the |\part{}| command (AMS classes do so), you could
%      get ``Première partie~1’’, ``Deuxième partie~2’’ in the toc;
%      when this occurs, this option should be set to
%      \fbo{false}, part titles will then be printed as ``Partie I’’,
%      ``Partie II’’.
%    \item [CustomiseFigTabCaptions=false (true*)]; when \fbo{false}
%      the default separator (colon) is used instead of
%      |\CaptionSeparator|.  Anyway, \frenchb{} makes sure that the
%      colon will be typeset with proper preceeding space in French.
%    \item [OldFigTabCaptions=true (false)] is to be used when
%      figures’ and tables’ captions must be typeset as with pre~3.0
%      versions of \frenchb{} (with |\CaptionSeparator| in French and
%      colon otherwise). Intended for standard \LaTeX{} classes only.
%    \item [SmallCapsFigTabCaptions=false (true*)]; when set to
%      \fbo{false}, |\figurename| and |\tablename| will be printed in
%      French captions as ``Figure’’ and ``Table’’ instead of being
%      printed in small caps (the default).
%    \item [SuppressWarning=true (false)]; can be turned to \fbo{true}
%      if you are bored with \frenchb’s warnings.
%    \item [INGuillSpace=true (false)] resets the dimensions of spaces
%      after opening French quotes and before closing French quotes to
%      the French `Imprimerie Nationale’ standards (inter-word space).
%      \frenchb’s default setting produces slightly narrower spaces
%      with lesser stretchability.
%    \item [EveryParGuill=open, close, none (open)]; sets whether
%      an opening quote (\texttt{«}) or a closing one (\texttt{»}) or
%      nothing should be printed by |\frquote{}| at the beginning of
%      every parapraph included in a level 1 (outer) quotation.
%      This option is also considered for level 2 (inner) quotations
%      to decide between \texttt{\guilsinglleft} and
%      \texttt{\guilsinglright} when \fbo{InnerGuillSingle=true}
%      (see below). \label{everyparguill}
%    \item [EveryLineGuill=open, close, none (none)]; with LuaTeX
%      based engines \emph{only}, it is possible to set this option
%      to \fbo{open} [resp. \fbo{close}]; this ensures that a `«’
%      [resp. `»’] followed by a proper space will be inserted at the
%      beginning of every line of embedded (inner) quotations
%      spreading over more than one line (provided that both outer
%      and inner quotations are entered with |\frquote{}|). When
%      \fbo{EveryLineGuill=open} or \fbo{=close} the inner quotation is
%      always surrounded by « and », the next option is ineffective.
%    \item [InnerGuillSingle=true (false)];
%      if \fbo{InnerGuillSingle=false} (default), inner quotations
%      entered with |\frquote{}| start with~\texttt{``} and end
%      with~\texttt{’’}.
%      If \fbo{InnerGuillSingle=true}, \texttt{\guilsinglleft} and
%      \texttt{\guilsinglright} are used instead of British double
%      quotes; moreover if option \fbo{EveryParGuill=open} (or
%      \fbo{close}) is set, a \texttt{\guilsinglleft} (or
%      \texttt{\guilsinglright}) is added at the beginning of every
%      parapraph included in the inner quotation.
%    \item [og=«, fg=»]; when guillemets characters are available on
%      the keyboard (through a compose key for instance), it is nice
%      to use them instead of typing |\og| and |\fg|.  This option
%      tells \frenchb{} which characters are opening and closing
%      French guillemets (they depend on the input encoding), then
%      you can type either \label{option-og-fg}
%      \texttt{« guillemets »} or \texttt{«guillemets»} (with or
%      without spaces) to get properly typeset French quotes.
%      This option works with LuaLaTeX and XeLaTeX; with pdfLaTeX it
%      requires \file{inputenc} to be loaded with a proper encoding:
%      8-bits encoding (latin1, latin9, ansinew, applemac,\dots) or
%      multi-byte encoding (utf8, utf8x).
%    \end{description}
%
%    \textbf{Options’ order} -- Please remember that options are read
%    in the order they appear in the \fbsetup{} command.
%    Someone wishing that \frenchb{} leaves the layout of lists
%    and footnotes untouched but caring for indentation of first
%    paragraph of sections should choose\\
%    \fbsetup{StandardLayout,IndentFirst} to get the expected
%    layout.  The reverse order
%    \fbsetup{IndentFirst,StandardLayout} would lead to option
%    \fbo{IndentFirst} being overwritten by \fbo{StandardLayout}.
%
%  \subsubsection{Captions}
%  \label{sssec-captions}
%
%    Caption names can be customised in French using the simplified
%    syntax introduced by \babel~3.9, for instance:
%    |\def\frenchproofname{Preuve}|.
%    The older syntax |\addto\captionsfrench{\def\proofname{Preuve}}|
%    still works. Keep in mind that \emph{only} |french| can be used to
%    redefine captions, even if \babel{}’s option was entered as
%    |francais| or |frenchb|.
%
%    When French is the main language, by default (see below)
%    \frenchb{} changes the separator (colon) used in figures’ and
%    tables’ captions \emph{for all languages} to |\CaptionSeparator|
%    which defaults to ` -- ’ and can be redefined in the preamble with
%    |\renewcommand*{\CaptionSeparator}{...}|.
%
%    When French is not the main language, the colon is preserved for
%    all languages but \frenchb{} makes sure that a proper space is
%    typeset before it.
%
%    Three new options are provided: if \fbo{CustomiseFigTabCaptions} is
%    set to \fbo{false} the colon will be used as separator in all
%    languages, with a proper space before the colon in French.
%    The second option, \fbo{OldFigTabCaptions}, can be set to \fbo{true}
%    to print figures’ and tables’ captions as they were with versions
%    pre~3.0 of \frenchb{} (using |\CaptionSeparator| in French and
%    colon in other languages); this option only makes sense with the
%    standard \LaTeX{} classes \cls{article}, \cls{report} and
%    \cls{book}.
%    The last option, \fbo{SmallCapsFigTabCaptions}, can be set to
%    \fbo{false} to typeset |\figurename| and |\tablename| in French as
%    ``Figure’’ and ``Table’’ rather than in small caps (the default).
%
%  \subsection{Hyphenation checks}
%  \label{ssec-hyphen}
%
%    Once you have built your format, a good precaution would be to
%    perform some basic tests about hyphenation in French.
%    For \LaTeXe{} I suggest this:
%    \begin{itemize}
%    \item run |pdfLaTeX| on the following file, with the encoding
%      suitable for your machine (\textit{my-encoding} will be |latin1|
%      for Unix machines, |ansinew| for PCs running~Windows,
%      |applemac| or |latin1| for Macintoshs, or |utf8|\dots\\[3mm]^^A\]
%      |%%% Test file for French hyphenation.|\\
%      |\documentclass{article}|\\
%      |\usepackage[|\textit{my-encoding}|]{inputenc}|\\
%      |\usepackage[T1]{fontenc} % Use LM fonts|\\
%      |\usepackage{lmodern}     % for French|\\
%      |\usepackage[frenchb]{babel}|\\
%      |\begin{document}|\\
%      |\showhyphens{signal container \'ev\'enement alg\`ebre}|\\
%      |\showhyphens{signal container événement algèbre}|\\
%      |\end{document}|
%    \item check the hyphenations proposed by \TeX{} in your log-file;
%      in French you should get with both 7-bit and 8-bit encodings\\
%      \texttt{si-gnal contai-ner évé-ne-ment al-gèbre}.\\
%      Do not care about how accented characters are displayed in the
%      log-file, what matters is the position of the `|-|’ hyphen
%      signs \emph{only}.
%    \end{itemize}
%    If they are all correct, your installation (probably) works fine,
%    if one (or more) is (are) wrong, ask a local wizard to see what’s
%    going wrong and perform the test again (or e-mail me about what
%    happens).\\
%    Frequent mismatches:
%    \begin{itemize}
%    \item you get |sig-nal con-tainer|, this probably means that the
%    hyphenation patterns you are using are for US-English, not for
%    French;
%    \item you get no hyphen at all in \texttt{évé-ne-ment}, this
%    probably means that you are using CM fonts and the macro
%    |\accent| to produce accented characters.
%    Using 8-bits fonts with built-in accented characters avoids
%    this kind of mismatch.
%    \end{itemize}
%
%  \subsection{Changes}
%  \label{ssec-changes}
%
%  \subsubsection*{What’s new in version 3.3?}
%
%    Release 3.3a is compatible with LuaTeX v.~0.95 (TL2016) and up.
%    Former skips |\FBcolonskip|, |\FBthinskip| and |\FBguillskip|
%    controlling punctuation spacings in LuaTeX have been removed;
%    all three engines now rely on the same commands |\FBcolonspace|,
%    |\FBthinspace| and |\FBguillspace|.
%
%    An alias |\frenchsetup{}| for |\frenchbsetup{}| has been added
%    in version 3.3a, it might appear more relevant in the future as
%    the language name |frenchb| should vanish.
%
%    Further customisation of the |\part{}| command is provided via
%    three new commands |\frenchpartfirst|, |\frenchpartsecond| and
%    |\frenchpartnameord|.
%
%  \subsubsection*{What’s new in version 3.2?}
%
%    Version 3.2g changes the default behaviour of |\frquote{}|
%    with LuaTeX based engines, the output is now the same with
%    all engines; to recover the former behaviour, add option
%    \fbo{EveryLineGuill=open}.
%
%    The handling of footnotes has been redesigned for the
%    \cls{beamer}, \cls{memoir} and koma-script classes.
%    The layout of footnotes ``à la française’’ should be unchanged
%    but footnotes’ customisations offered by these classes
%    (i.e.\ font or color changes) are now available even when option
%    \fbo{FrenchFootnotes} is \fbo{true}.
%
%    A long standing bug regarding the \pkg{xspace} package has been
%    fixed: |\xspace| has been moved up from the internal command
%    |\FB@fg| to |\fg|; |\frquote{}| now works properly when the
%      \pkg{xspace} package is loaded.
%
%    Version 3.2b is the first one designed to work with LuaTeX
%    v.~0.95 as included in TeXLive 2016 (LuaTeX’s new glue node
%    structure is not compatible with previous versions).
%
%    \textbf{Warning to Lua(La)TeX users:} starting with version~3.2b
%    the lua code included in \file{frenchb.lua} will \emph{not work}
%    on older installations (TL2015 f.i.), so \frenchb{} reverts to
%    active characters while handling high punctuation with LuaTeX
%    engines older than 0.95!
%    The best way to go is to upgrade to TL2016 or equivalent asap.
%    Xe(La)TeX and pdf(La)TeX users can safely use \frenchb{} v.~3.2b
%    and later on older installations too.
%
%    Tne internals of commands |\NoAutoSpacing|, |\ttfamilyFB|,
%    |\rmfamilyFB| and |\sffamilyFB| have been completely redesigned
%    in version~3.2c, they behave now consistently with all engines.
%
%  \subsubsection*{What’s new in version 3.1?}
%
%    New command |\frquote{}| meant to enter French quotations,
%    especially long ones (spreading over several paragraphs) and/or
%    embedded ones. see p.~\pageref{frquote} for details.
%
%  \subsubsection*{What’s new in version 3.0?}
%
%    Many deep changes lead me to step \frenchb’s version number
%    to~3.0a:
%    \begin{itemize}
%    \item \babel{}~3.9 is required now to process
%      \file{frenchb.ldf}, this change allows for cleaner definitions
%      of dates and captions for the Unicode engines LuaTeX and
%      XeTeX and also provides a simpler syntax for end-users, see
%      section~\ref{sssec-captions} p.\pageref{sssec-captions}.
%    \item \fbsetup{} options management has been completely
%      reworked; two new options added.
%    \item Canadian French didn’t work as a normal \babel’s dialect,
%      it should now; btw. the French language should now be loaded
%      as \Lopt{french}, \emph{not as} \Lopt{frenchb} or
%      \Lopt{francais} and preferably as a \emph{global} option of
%      |\documentclass|. Some tolerance still exists in v3.0, but
%      do not rely on it.
%    \item \frenchb{} no longer loads \file{frenchb.cfg}:
%      customisation should definitely be done using \fbsetup{}
%      options.
%    \item Description lists labels are now indented; try setting
%      |\descindentFB=0pt| (or |\listindentFB=0pt| for all lists)
%      in the preamble if you don’t like it.
%    \item The last but not least change affects the (recent)
%      LuaTeX-based engines, (this means version 0.76 as included in
%      TL2013 and up): active characters are no longer used in French
%      for `high punctuation’%
%      \footnote{The current \frenchb{} version requires LuaTeX
%      v.~0.95 as included in TL2016, see above.}.
%      Functionalities and user interface are unchanged.
%
%      Many thanks to Paul Isambert who provided the basis for the
%      lua code (see his presentation at GUT’2010) and kindly
%      reviewed my first drafts suggesting significant improvements.
%
%      Please note that this code, still experimental, is likely to
%      change until LuaTeX itself has reached version~1.0.
%    \end{itemize}
%
%    Starting with version~3.0c, \frenchb{} no longer customises
%    lists with the \cls{beamer} class and offers a new option
%    (\fbo{INGuillSpace}) to follow French `Imprimerie Nationale’
%    recommendations regarding quotes’ spacing.
%
% \StopEventually{}
%
% \changes{v2.6c}{2013/05/20}{Dummy file frenchb.cfg is no longer
%    generated from frenchb.dtx.}
%
%  \clearpage
%  \newgeometry{left=53mm,right=25mm}
%
%  \section{The code}
%
%  \subsection{Initial setup}
%
% \changes{v2.1d}{2008/05/02}{Argument of \cs{ProvidesLanguage} changed
%    above from `french’ to `frenchb’ (otherwise \cs{listfiles} prints
%    no date/version information).  The real name of current language
%    (french) as to be corrected before calling \cs{LdfInit}.}
%
% \iffalse
%<*ldf>
% \fi
%
% \changes{v3.0a}{2014/02/28}{Support for options frenchb, francais,
%    canadien, acadian changed.}
%
%    If \file{frenchb.ldf} was loaded with \babel{}’s options
%    \Lopt{francais} or \Lopt{frenchb}, we make it behave as if
%    \Lopt{french} was specified.
%    In Plain formats, \texttt{@} catcode is not `letter’.
%    \begin{macrocode}
\chardef\atcatcode=\catcode`\@
\catcode`\@=11\relax
\def\bbl@tempa{francais}
\ifx\CurrentOption\bbl@tempa
  \let\l@francais\l@french
  \def\captionsfrancais{\captionsfrench}
  \def\datefrancais{\datefrench}
  \def\extrasfrancais{\extrasfrench}
  \def\noextrasfrancais{\extrasfrench}
  \def\CurrentOption{french}
\fi
\def\bbl@tempa{frenchb}
\ifx\CurrentOption\bbl@tempa
  \let\l@frenchb\l@french
  \def\captionsfrenchb{\captionsfrench}
  \def\datefrenchb{\datefrench}
  \def\extrasfrenchb{\extrasfrench}
  \def\noextrasfrenchb{\extrasfrench}
  \def\CurrentOption{french}
\fi
\catcode`\@=\atcatcode \let\atcatcode\relax
%    \end{macrocode}
%
% \changes{v3.0a}{2014/02/19}{\cs{LdfInit} checks \cs{datefrench}
%    instead of \cs{captionsfrench} to avoid a conflict with
%    papertex.cls which loads datetime.sty.}
%
%    The macro |\LdfInit| takes care of preventing that this file is
%    loaded more than once, checking the category code of the
%    \texttt{@} sign, etc.
%
%    \begin{macrocode}
\LdfInit\CurrentOption\captionsfrench
%    \end{macrocode}
%
% \changes{v3.0c}{2014/03/30}{No need to define \cs{l@french} as
%    \cs{lang@french}, babel.def (3.9j) takes care for this.}
%
%    Make sure that |\l@french| is defined (possibly as 0).
%    \file{babel.def} now (3.9i) defines |\l@<languagename>| also for
%    eTeX, LuaTeX and XeTeX formats which set |\lang@<languagename>|.
%    \begin{macrocode}
\def\FB@nopatterns{%
   \ifx\l@nohyphenation\@undefined
      \edef\bbl@nulllanguage{\string\language=0}%
      \adddialect\l@french0
   \else
      \adddialect\l@french\l@nohyphenation
      \edef\bbl@nulllanguage{\string\language=nohyphenation}%
   \fi
   \@nopatterns{French}}
\ifx\l@french\@undefined
   \FB@nopatterns
\fi
%    \end{macrocode}
%
% \changes{v2.1d}{2008/05/04}{Avoid warning `\cs{end} occurred
%    when \cs{ifx} ... incomplete’ with LaTeX-2.09.}
%
%  \begin{macro}{\ifLaTeXe}
%    No support is provided for late \LaTeX-2.09: issue a warning
%    and exit if \LaTeX-2.09 is in use.  Plain is still supported.
%    \begin{macrocode}
\newif\ifLaTeXe
\let\bbl@tempa\relax
\ifx\magnification\@undefined
   \ifx\@compatibilitytrue\@undefined
     \PackageError{frenchb.ldf}
        {LaTeX-2.09 format is no longer supported.\MessageBreak
         Aborting here}
        {Please upgrade to LaTeX2e!}
     \let\bbl@tempa\endinput
   \else
     \LaTeXetrue
   \fi
\fi
\bbl@tempa
%    \end{macrocode}
%  \end{macro}
%
% \changes{v3.0a}{2014/02/15}{In Plain, provide a substitute for
%    \cs{PackageWarning} and \cs{PackageInfo}.}
%
%    Let’s provide a substitute for |\PackageError|, |\PackageWarning|
%    and |\PackageInfo| not defined in Plain:
%    \begin{macrocode}
\def\fb@error#1#2{%
    \begingroup
      \newlinechar=`\^^J
      \def\\{^^J(frenchb.ldf) }%
      \errhelp{#2}\errmessage{\\#1^^J}%
    \endgroup}
\def\fb@warning#1{%
   \begingroup
      \newlinechar=`\^^J
      \def\\{^^J(frenchb.ldf) }%
      \message{\\#1^^J}%
   \endgroup}
\def\fb@info#1{%
   \begingroup
      \newlinechar=`\^^J
      \def\\{^^J}%
      \wlog{#1}%
   \endgroup}
%    \end{macrocode}
%
% \changes{v3.0c}{2014/03/30}{frenchb requires babel-3.9i.}
%
%    Quit if \babel’s version is less than 3.9i.
%    \begin{macrocode}
\let\bbl@tempa\relax
\ifx\babeltags\@undefined
   \let\bbl@tempa\endinput
   \ifLaTeXe
      \PackageError{frenchb.ldf}
         {frenchb requires babel v.3.9i.\MessageBreak
          Aborting here}
         {Please upgrade Babel!}
   \else
      \fb@error{frenchb requires babel v.3.9i.\\
                Aborting here}
               {Please upgrade Babel!}
   \fi
\fi
\bbl@tempa
%    \end{macrocode}
%
%    \file{frenchb.ldf} can be loaded with options \Lopt{canadien} or
%    \Lopt{acadian}, which both stand for Canadian French.  Internally,
%    \Lopt{acadian} will be the name of the corresponding \babel’s
%    dialect, so we set |\CurrentOption| to \Lopt{acadian} in both
%    cases.
%    If no specific hyphenation patterns are available, Canadian French
%    will use the French ones.
%
%    TODO: Canadian French hyphenation doesn’t work with LuaTeX.
%    \begin{macrocode}
\ifx\l@acadian\@undefined
   \ifx\l@canadien\@undefined
      \adddialect\l@acadian\l@french
      \adddialect\l@canadien\l@french
   \else
      \adddialect\l@acadian\l@canadien
   \fi
\else
  \adddialect\l@canadien\l@acadian
\fi
\def\bbl@tempa{canadien}
\ifx\CurrentOption\bbl@tempa
  \def\captionscanadien{\captionsacadian}
  \def\datecanadien{\dateacadian}
  \def\extrascanadien{\extrasacadian}
  \def\noextrascanadien{\extrasacadian}
  \def\CurrentOption{acadian}
\fi
%    \end{macrocode}
%
%    French uses the standard values of |\lefthyphenmin| (2)
%    and |\righthyphenmin| (3); let’s provide their values though,
%    as required by \babel.
%
%    \begin{macrocode}
\expandafter\providehyphenmins\expandafter{\CurrentOption}{\tw@\thr@@}
%    \end{macrocode}
%
%  \begin{macro}{\ifFBunicode}
%  \begin{macro}{\ifFBLuaTeX}
%  \begin{macro}{\ifFBXeTeX}
%    French hyphenation patterns are now coded in Unicode, see file
%    \file{hyph-fr.tex}.  XeTeX and LuaTeX engines require some extra
%    code to deal with the French ``apostrophe’’.
%    Let’s define three new `if’: |\ifFBLuaTeX|, |\ifFBXeTeX| and
%    |\ifFBunicode| which will be true for XeTeX and LuaTeX engines
%    and false for 8-bits engines.
%
% \changes{v2.4a}{2009/11/23}{Added a new `if’ \cs{FBunicode} and
%    some \cs{lccode} definitions to \cs{extrasfrench} and
%    \cs{noextrasfrench}.}
%
% \changes{v2.5d}{2011/01/17}{Added two new `if’ \cs{FBXeTeX} and
%    \cs{FBLuaTeX} as XeTeX and behave differently regarding the status
%     of the French `apostrophe’.}
%
% \changes{v2.6g}{2013/12/15}{lccode values for the French
%    `apostrophe’ are now the same for XeTeX and LuaTeX.}
%
%    We cannot rely on \eTeX’s |\ifdefined| at this stage, as it is not
%    defined in Plain \TeX{} format.
%    \begin{macrocode}
\newif\ifFBunicode
\newif\ifFBLuaTeX
\newif\ifFBXeTeX
\begingroup\expandafter\expandafter\expandafter\endgroup
\expandafter\ifx\csname luatexversion\endcsname\relax
\else
  \FBunicodetrue \FBLuaTeXtrue
\fi
\begingroup\expandafter\expandafter\expandafter\endgroup
\expandafter\ifx\csname XeTeXrevision\endcsname\relax
\else
  \FBunicodetrue \FBXeTeXtrue
\fi
%    \end{macrocode}
% \end{macro}
% \end{macro}
% \end{macro}
%
% \begin{macro}{\extrasfrench}
% \changes{v3.0a}{2014/02/18}{Take advantage of babel’s
%    \cs{babel@savevariable} to handle apostrophe’s \cs{lccode}.}
%
% \begin{macro}{\noextrasfrench}
%    The macro |\extrasfrench| will perform all the extra
%    definitions needed for the French language.
%    The macro |\noextrasfrench| is used to cancel the actions of
%    |\extrasfrench|.
%
%    In French, character ``apostrophe’’ is a letter in expressions
%    like |l’ambulance| (French hyphenation patterns provide entries
%    for this kind of words).  This means that the |\lccode| of
%    ``apostrophe’’ has to be non null in French for proper hyphenation
%    of those expressions, and has to be reset to null when exiting
%    French.
%
%    The following code ensures correct hyphenation of words like
%    |d’aventure|, |l’utopie|, with all TeX engines (XeTeX, LuaTeX,
%    pdfTeX) using \file{hyph-fr.tex} patterns.
%
%    \begin{macrocode}
\@namedef{extras\CurrentOption}{%
     \babel@savevariable{\lccode`\'}%
     \ifFBunicode
        \babel@savevariable{\lccode"2019}%
        \lccode`\'="2019\lccode"2019="2019
     \else
        \lccode`\'=`\'
     \fi
}
\@namedef{noextras\CurrentOption}{}
%    \end{macrocode}
%
%    Let’s define a handy command for adding stuff to
%    |\extras\CurrentOption|,\linebreak[0] |\noextras\CurrentOption| or
%    |\captions\CurrentOption| but first let’s save the value of
%    |\CurrentOption| for later use in \fbsetup{} (`AfterEndOfPackage’,
%    |\CurrentOption| will be lost).
%    \begin{macrocode}
\let\FB@CurOpt\CurrentOption
\newcommand*{\FB@addto}[2]{%
    \expandafter\addto\csname #1\FB@CurOpt\endcsname{#2}}
%    \end{macrocode}
%
%    One more thing |\extrasfrench| needs to do is to make sure that
%    ``Frenchspacing’’ is in effect.  |\noextrasfrench| will switch
%    ``Frenchspacing’’ off again if necessary.
%    \begin{macrocode}
\FB@addto{extras}{\bbl@frenchspacing}
\FB@addto{noextras}{\bbl@nonfrenchspacing}
%    \end{macrocode}
% \end{macro}
% \end{macro}
%
%  \subsection{Punctuation}
%  \label{ssec-punct}
%
% \changes{v2.5a}{2010/08/10}{Punctation is no longer made active
%    with XeTeX-based engines.}
%
%    As long as no better solution is available, the `high
%    punctuation’ characters (|;| |!| |?| and |:|) have to be made
%    |\active| for an automatic control of the amount of space to be
%    inserted before them.  Both XeTeX and LuaTeX provide an
%    alternative to active characters (`XeTeXinterchar’ mechanism and
%    LuaTeX’s callbacks).
%
% \changes{v3.0a}{2014/01/10}{New flag \cs{ifFB@luatex@punct} for
%    `high punctuation’ management with LuaTeX engines.}
%
%  \begin{macro}{\ifFB@active@punct}
%    \begin{macrocode}
\newif\ifFB@active@punct \FB@active@puncttrue
%    \end{macrocode}
%    \end{macro}
%
%  \begin{macro}{\ifFB@luatex@punct}
%    Three internal flags are needed for the three different techniques
%    used for `high punctuation’ management.
%
%    With LuaTeX, starting with version 0.95, callbacks are used to
%    get rid of active punctuation.  With previous versions, `high
%    punctuation’ characters remain active (see below).
%    \begin{macrocode}
\newif\ifFB@luatex@punct
\ifFBLuaTeX
  \ifnum\luatexversion<95
    \ifx\PackageWarning\@undefined
      \fb@warning{Please upgrade LuaTeX to version 0.95 or above!\\%
         frenchb will make high punctuation characters (;:!?) active\\%
         with LuaTeX < 0.95.}%
    \else
      \PackageWarning{frenchb.ldf}{Please upgrade LuaTeX
         to version 0.95 or above!\MessageBreak
         frenchb will make high punctuation characters\MessageBreak
         (;:!?) active with LuaTeX < 0.95;\MessageBreak reported}%
    \fi
  \else
    \FB@luatex@puncttrue\FB@active@punctfalse
  \fi
\fi
%    \end{macrocode}
%    \end{macro}
%
%  \begin{macro}{\ifFB@xetex@punct}
%    For XeTeX, the availability of |\XeTeXinterchartokenstate|
%    decides whether the `high punctuation’ characters (|;| |!| |?|
%    and |:|) have to be made |\active| or not.
%
% \changes{v3.2b}{2016/04/18}{New counter \cs{FB@nonchar} needed for
%    non characters: it’s value will be 4095 for new engines and 255
%    for older ones.}
%
%    The number of available character classes has been increased
%    from 256 to 4096 in XeTeX v.~0.99994, the class for
%    non-characters is now 4095 instead of 255.
%    \begin{macrocode}
\newcount\FB@nonchar
\newif\ifFB@xetex@punct
\begingroup\expandafter\expandafter\expandafter\endgroup
\expandafter\ifx\csname XeTeXinterchartokenstate\endcsname\relax
\else
  \FB@xetex@puncttrue\FB@active@punctfalse
  \ifdim\the\XeTeXversion\XeTeXrevision pt<0.99994pt
    \FB@nonchar=255 \relax
  \else
    \FB@nonchar=4095 \relax
  \fi
\fi
%    \end{macrocode}
%  \end{macro}
%
%  \begin{macro}{\FBcolonspace}
%  \begin{macro}{\FBthinspace}
% \changes{v2.5a}{2010/08/10}{Define \cs{FBthinspace} for those who want
%    to customise the width of the space before ; and co.}
%
% \changes{v2.6d}{2013/06/19}{Rename \cs{Fthinspace} to
%    \cs{FBthinspace} and \cs{Fcolonspace} to \cs{FBcolonspace} to avoid
%    a conflict with fournier.sty.}
%
% \changes{v3.1f}{2015/05/31}{\cs{FBthinspace} is no longer a kern but
%    a skip (frenchb adds a nobreak penalty before it).}
%
% \changes{v3.2c}{2016/05/14}{Change .16667em to
%    .5\cs{fontdimen2}\cs{font} to get in XeTeX and pdfTeX
%    the same spacing as in LuaTeX.}
%
% \changes{v3.3a}{2016/04/30}{Skips \cs{FBcolonskip} and
%    \cs{FBthinskip} replaced by toks \cs{FBcolonsp} and
%    \cs{FBthinsp}.}
%
%    According to the I.N.\ specifications, the `:’ requires an
%    inter-word space before it, the other three require just a thin
%    space.  We define |\FBcolonspace| as |\space| (inter-word space)
%    and |\FBthinspace| as an half inter-word space with no shrink nor
%    stretch, both are user customisable in the preamble.
%    \begin{macrocode}
\newcommand*{\FBcolonspace}{\space}
\newcommand*{\FBthinspace}{\hskip.5\fontdimen2\font \relax}
%    \end{macrocode}
%    These commands will be converted into toks `AtBeginDocument’
%    for LuaTeX.
%    \begin{macrocode}
\newtoks\FBcolonsp
\newtoks\FBthinsp
%    \end{macrocode}
%  \end{macro}
%  \end{macro}
%
% \changes{v3.2c}{2016/05/14}{Newif \cs{ifFB@spacing} and new commands
%    \cs{FB@spacingon}, \cs{FB@spacingoff} to control space tuning in
%    French.}
%
%    With LuaTeX and XeTeX engines, \frenchb{} handles French quotes
%    together with `high punctuation’; the conditional |\ifFB@spacing|
%    will be used by PdfTeX and XeTeX engines to switch on or off space
%    tuning before high punctuation and inside French quotes.
%    A matching attribute will be defined later for LuaTeX.
%    \begin{macrocode}
\newif\ifFB@spacing \FB@spacingtrue
%    \end{macrocode}
%
% \begin{macro}{\FB@spacing@off}
% \begin{macro}{\FB@spacing@on}
%    Two internal commands to switch on and off all space tuning
%    for all six characters `;:!?«»’.
%    They will be triggered by user command |\NoAutoSpacing| and by
%    font family switching commands |\ttfamilyFB| |\rmfamilyFB| and
%    |\sffamilyFB|. These four commands will now behave the same with
%    any engine (up to version 3.2b, results were engine dependent).
%    \begin{macrocode}
\newcommand*{\FB@spacing@on}{%
  \ifFB@luatex@punct
    \FB@spacing=1 \relax
  \else
    \FB@spacingtrue
  \fi}
\newcommand*{\FB@spacing@off}{%
  \ifFB@luatex@punct
    \FB@spacing=0 \relax
  \else
    \FB@spacingfalse
  \fi}
%    \end{macrocode}
% \end{macro}
% \end{macro}
%
%  \subsubsection{Punctuation with LuaTeX}
%  \label{sssec-punct-luatex}
%
% \changes{v3.0a}{2014/01/10}{New handling of `high punctuation’
%    through callbacks with LuaTeX engines.}
%
%    The following part holds specific code for punctuation with modern
%    LuaTeX engines, i.e. version 0.95 (included in TL2016) or newer.
%
% \changes{v3.0b}{2014/02/13}{Require luatexbase with LaTeXe in case
%    fontspec has not been loaded before babel.}
%
% \changes{v3.0c}{2014/03/26}{Just load luatexbase.sty instead of
%    luaotfload.sty with plain formats.}
%
% \changes{v3.1a}{2014/05/30}{Misplaced \cs{fi} for plain formats.}
%
% \changes{v3.1j}{2015/10/16}{Loading luatexbase.sty is no longer
%    needed with LaTeX release 2015/10/01 or later.}
%
% \changes{v3.2b}{2016/04/18}{Load ltluatex.tex for plain LuaTeX to
%    ensure \cs{newattribute} is defined.}
%
% \changes{v3.2c}{2016/05/14}{New LuaTeX attribute \cs{FB@spacing}.}
%
% \changes{v3.3a}{2017/04/30}{Compatibility code for pre
%    2015/10/01 LaTeX release removed, see ltnews23.tex.}
%
%    We define three LuaTeX attributes to control spacing in French
%    for `high punctuation’ and quotes, making sure that
%    |\newattribute| is defined.
%    \begin{macrocode}
\ifFB@luatex@punct
  \begingroup\expandafter\expandafter\expandafter\endgroup
  \expandafter\ifx\csname newluafunction\endcsname\relax
%    \end{macrocode}
%    This code is for Plain: load\file{ltluatex.tex} if it hasn’t been
%    loaded before \babel.
%    \begin{macrocode}
    \input ltluatex.tex
  \fi
%    \end{macrocode}
%    |\FB@spacing=0| switches off any space tuning both before high
%    punctuation characters and inside French quotes (i.e.\ function
%    |french_punctuation| doesn’t alter the node list at all).
%    |\FB@addDPspace=0| switches off automatic insertion of spaces
%    before high punctuation characters (but typed spaces are still
%    turned into nobreak thin- or word-spaces).
%    |\FB@addGUILspace| will be set to~1 by option \fbo{og=«, fg=»},
%    thus enabling automatic insertion of proper spaces after `«’ and
%    before `»’.
%    \begin{macrocode}
  \newattribute\FB@spacing      \FB@spacing=1 \relax
  \newattribute\FB@addDPspace   \FB@addDPspace=1 \relax
  \newattribute\FB@addGUILspace \FB@addGUILspace=0 \relax
  \ifLaTeXe
    \PackageInfo{frenchb.ldf}{No need for active punctuation
                 characters\MessageBreak with this version
                 of LuaTeX!\MessageBreak reported}
  \else
    \fb@info{No need for active punctuation characters\\
             with this version of LuaTeX!}
  \fi
\fi
%    \end{macrocode}
%
% \iffalse
%</ldf>
%<*lua>
% \fi
%    This is \file{frenchb.lua}. It holds Lua code to deal with  `high
%    punctuation’ and quotes.  This code is based on suggestions from
%    Paul Isambert.
%
%    \begin{macro}{frenchb.lua}
%    First we define two flags to control spacing before French `high
%    punctuation’ (thin space or inter-word space).
%
% \changes{v3.1a}{2014/06/16}{Codes 0x13 and 0x14 added for French
%    quotes in T1-encoding.}
%
%    \begin{macrocode}
local FB_punct_thin =
  {[string.byte("!")] = true,
   [string.byte("?")] = true,
   [string.byte(";")] = true}
local FB_punct_thick =
  {[string.byte(":")] = true}
%    \end{macrocode}
%    Managing spacing after `«’ (U+00AB) and before `»’ (U+00BB) can be
%    done by the way; we define two flags, |FB_punct_left| for
%    characters requiring some space before them and |FB_punct_right|
%    for `\guillemotleft’ which must be followed by some space.
%    In case LuaTeX is used to output T1-encoded fonts instead of
%    OpenType fonts, codes \texttt{0x13} and \texttt{0x14}  have to be
%    added for `«’ and `»’.
%    \begin{macrocode}
local FB_punct_left =
  {[string.byte("!")] = true,
   [string.byte("?")] = true,
   [string.byte(";")] = true,
   [string.byte(":")] = true,
   [0x14]             = true,
   [0xBB]             = true}
local FB_punct_right =
  {[0x13]             = true,
   [0xAB]             = true}
%    \end{macrocode}
%    Two more flags will be needed to avoid spurious spaces in
%    strings like !! ?? or (?)
%    \begin{macrocode}
local FB_punct_null =
  {[string.byte("!")] = true,
   [string.byte("?")] = true,
   [string.byte("[")] = true,
   [string.byte("(")] = true,
%    \end{macrocode}
%    or if the user has typed a nobreak space U+00A0 or a nobreak thin
%    space U+202F before a `high punctuation’ character: no space
%    should be added by \frenchb.  Same is true inside French quotes.
%    \begin{macrocode}
   [0xA0]             = true,
   [0x202F]           = true}
local FB_guil_null =
  {[0xA0]             = true,
   [0x202F]           = true}
%    \end{macrocode}
% \changes{v3.1g}{2015/06/17}{Node HLIST added; node TEMP added
%    for the first node of \cs{hbox}es.}
%
%    Local definitions for nodes:
%    \begin{macrocode}
local new_node     = node.new
local copy_node    = node.copy
local node_id      = node.id
local HLIST        = node_id("hlist")
local TEMP         = node_id("temp")
local KERN         = node_id("kern")
local GLUE         = node_id("glue")
local GLYPH        = node_id("glyph")
local PENALTY      = node_id("penalty")
local nobreak      = new_node(PENALTY)
nobreak.penalty    = 10000
local insert_node_before = node.insert_before
local insert_node_after  = node.insert_after
local remove_node        = node.remove
%    \end{macrocode}
%    Commands |\FBthinspace|, |\FBcolonspace| and |\FBguillspace| are
%    converted `AtBeginDocument’ into toks |\FBthinsp|, |\FBcolonsp|
%    and |\FBguillsp|;
%    the latter are processed by the next function |get_glue| which
%    returns a table of three values which are fractions of
%    |\fontdimen2|, |\fontdimen3| and |\fontdimen4|.
%    \label{lua-scaling}
%    \begin{macrocode}
local function get_glue(toks)
  local t = nil
  local f = string.match(toks, "\092hskip%s*([%d%.]*)%s*\092fontdimen")
  if f == "" then f = 1 end
  if f then
     t = {f, 0, 0}
     f = string.match(toks, "plus%s*([%d%.]*)%s*\092fontdimen")
     if f == "" then f = 1 end
     if f then
        t[2] = f
        f = string.match(toks, "minus%s*([%d%.]*)%s*\092fontdimen")
        if f == "" then f = 1 end
        if f then
           t[3] = f
        end
     end
  elseif string.match(toks, "\092F?B?thinspace") then
     t = {0.5, 0, 0}
  elseif string.match(toks, "\092space") then
     t = {1, 1, 1}
  end
  return t
end
local colngl = get_glue(tex.toks['FBcolonsp']) or {1, 1, 1}
local thingl = get_glue(tex.toks['FBthinsp'])  or {.5, 0, 0}
local guilgl = get_glue(tex.toks['FBguillsp']) or {.8, .3, .8}
%    \end{macrocode}
%    The next function converts glue sizes returned in fontdimens
%    by function |get_glue| into sp for the current font;
%    beware of null values for fid, see |\nullfont| in TikZ, and of
%    special fonts like lcircle1.pfb for which |font.getfont(fid)|
%    does not return a proper font table, in such cases the function
%    returns |nil|.
%
% \changes{v3.1l}{2016/02/13}{font.getfont(fid) possibly
%    returns nil even for a positive fid (i.e. AMS lcircle1.pfb).
%    Reported by François Legendre.}
%
% \changes{v3.1m}{2016/02/16}{new\_glue\_scaled returns
%    nil in case of invalid font table (i.e. lcircle1.pfb).  In such
%    cases frenchb leaves the node list unchanged.}
%
% \changes{v3.2b}{2016/04/18}{glue\_spec removed;
%    starting with LuaTeX 0.95, glue specifications fit in glue.}
%
%    \begin{macrocode}
local font_table = {}
local function new_glue_scaled (fid,table)
  if fid > 0 then
     local fp = font_table[fid]
     if not fp then
        local ft = font.getfont(fid)
        if ft then
           font_table[fid] = ft.parameters
           fp = font_table[fid]
        end
     end
     local gl = new_node(GLUE,0)
     if fp then
        node.setglue(gl, table[1]*fp.space,
                         table[2]*fp.space_stretch,
                         table[3]*fp.space_shrink)
        return gl
     else
        return nil
     end
  else
     return nil
  end
end
%    \end{macrocode}
%    Let’s catch LuaTeX attributes |\FB@spacing|, |\FB@addDPspace|
%    and |\FB@addGUILspace|.
%    \begin{macrocode}
local FBspacing    = luatexbase.attributes['FB@spacing']
local addDPspace   = luatexbase.attributes['FB@addDPspace']
local addGUILspace = luatexbase.attributes['FB@addGUILspace']
local has_attribute = node.has_attribute
%    \end{macrocode}
%    The following function will be added to |kerning| callback.
%    It catches all nodes of type |GLYPH| in the list starting at
%    |head| and checks the language attributes of the current glyph:
%    nothing is done if the current language is not French and only
%    specific punctuation characters (those for which |FB_punct_left|
%    or |FB_punct_right| is true) need a special treatment.
%    In French, local variables are defined to hold the properties of
%    the current glyph (|item|) and of the previous one (|prev|) or the
%    next one (|next|).  Constant |FR=lang.id(french)| is defined by
%    command |\activate@luatexpunct|.
%
% \changes{v3.1b}{2014/09/16}{Add a check for null fid
%    in french\_punctuation (Tikz \cs{nullfont}).
%    Bug pointed out by Paul Gaborit.}
%
%    \begin{macrocode}
local function french_punctuation (head)
  for item in node.traverse_id(GLYPH, head) do
    local lang = item.lang
    local char = item.char
    local fid  = item.font
    local FRspacing = has_attribute(item, FBspacing)
    FRspacing = FRspacing and FRspacing > 0
    local SIG  = has_attribute(item, addGUILspace)
    SIG = SIG and SIG >0
    if lang == FR and FRspacing and
                      FB_punct_left[char] and fid > 0 then
       local prev = item.prev
       local prev_id, prev_subtype, prev_char
       if prev then
          prev_id = prev.id
          prev_subtype = prev.subtype
          if prev_id == GLYPH then
             prev_char = prev.char
          end
       end
%    \end{macrocode}
%
% \changes{v3.0c}{2014/04/18}{Null glues should not trigger space
%    insertion before high ponctuation.  Bug pointed out by Benoit
%    Rivet for the `lstlisting’ environment of the listings package.}
%
% \changes{v3.1c}{2014/10/25}{Previous bug fix for null glues
%    (v3.0c) did not work properly.  Fixed now (I hope!).
%    Pointed out by Jacques André.}
%
%    If the previous item is a glue, check its natural width, only
%    positive glues (actually glues > 1 sp, for tabular `l’
%    columns) are to be replaced by a nobreakspace.
%    \begin{macrocode}
       local is_glue = prev_id == GLUE
       local glue_wd
       if is_glue then
          glue_wd = prev.width
       end
       local realglue = is_glue and glue_wd > 1
%    \end{macrocode}
%    For characters for which |FB_punct_thin| or |FB_punct_thick| is
%    \emph{true}, the amount of spacing to be typeset before them
%    is controlled by commands |\FBthinspace| and |\FBcolonspace|
%    respectively.
%    Two options: if a space has been typed in before (turned into
%    \emph{glue} in the node list), we remove the \emph{glue} and add a
%    nobreak penalty and the required \emph{glue}.
%    Otherwise (|auto| option), the penalty and the required
%    \emph{glue} are inserted if attribute |\FB@addDPspace| is set,
%    unless one of these three conditions is met:
%    a) the previous character is part of type |FB_punct_null| (this
%       avoids spurious spaces in strings like |(!)| or |??|),
%    b) a null glue (actually glues <= 1 sp for tabulars) preceeds the
%       punctuation character,
%    c) the punctuation character starts a paragraph or an |\hbox{}|.
%
%    \begin{macrocode}
       if FB_punct_thin[char] or FB_punct_thick[char] then
          local SBDP = has_attribute(item, addDPspace)
          local auto = SBDP and SBDP > 0
          if auto then
             if (prev_char and FB_punct_null[prev_char]) or
                (is_glue and glue_wd <= 1) or
                (prev_id == HLIST and prev_subtype == 3) or
                (prev_id == TEMP) then
                auto = false
             end
          end
          local fbglue
          if FB_punct_thick[char] then
             fbglue = new_glue_scaled(fid,colngl)
          else
             fbglue = new_glue_scaled(fid,thingl)
          end
%    \end{macrocode}
%    In case |new_glue_scaled| fails (returns |nil|) the node list
%    remains unchanged.
%    \begin{macrocode}
          if (realglue or auto) and fbglue then
             if realglue then
                head = remove_node(head,prev,true)
             end
             insert_node_before(head, item, copy_node(nobreak))
             insert_node_before(head, item, copy_node(fbglue))
          end
%    \end{macrocode}
%    Let’s consider `»’ now (the only remaining glyph of
%    |FB_punct_left| class): we just have to remove any \emph{glue}
%    possibly preceeding `»’, then to insert the nobreak penalty and
%    the proper \emph{glue} (controlled by |\FBguillspace|).
%    This is done only if French quotes have been `activated’ by
%    options \fbo{og=«, fg=»} in \fbsetup{} and can be denied locally
%    with |\NoAutoSpacing| (this is controlled by the |SIG| flag).
%    If either a) the preceding glyph is member of |FB_guil_null|, or
%    b) `»’ is the first glyph of an |\hbox{}| or a paragraph, nothing
%    is done, this is controlled by the |addgl| flag.
%
% \changes{v3.1a}{2014/06/18}{Added flag addgl which must also be true
%    when prev or next is not a char (i.e. \cs{kern}0 in
%    «\cs{texttt}\{a\}»).}
%
% \changes{v3.1g}{2015/06/26}{flag addgl set to false for `»’
%    at the beginning of an \cs{hbox} or a paragraph or a tabular
%    `l’ and `c’ columns.}
%
%    \begin{macrocode}
       elseif SIG then
          local addgl = (prev_char and not FB_guil_null[prev_char]) or
                        (not prev_char and
                         prev_id ~= TEMP and
                         not (prev_id == HLIST and prev_subtype == 3)
                        )
%    \end{macrocode}
%    Correction for tabular `c’ (glue 0 plus 1 fil) and `l’ (glue 1sp)
%    columns:
%    \begin{macrocode}
          if is_glue and glue_wd <= 1 then
             addgl = false
          end
          local fbglue = new_glue_scaled(fid,guilgl)
          if addgl and fbglue then
             if is_glue then
                head = remove_node(head,prev,true)
             end
             insert_node_before(head, item, copy_node(nobreak))
             insert_node_before(head, item, copy_node(fbglue))
          end
       end
    end
%    \end{macrocode}
%    Similarly, for `«’ (unique member of the |FB_punct_right| class):
%    unless either a) the next glyph is member of |FB_guil_null|, or
%    b) `«’ is the last glyph of an |\hbox{}| or a paragraph (then the
%    |addgl| flag is false, nothing is done), we remove any \emph{glue}
%    possibly following it and insert first the proper \emph{glue} then
%    a nobreak penalty so that finally the penalty preceeds the
%    \emph{glue}.
%
% \changes{v3.1a}{2014/06/18}{Look ahead when next is a kern
%    (i.e. in « \cs{texttt}\{a\} »).}
%
% \changes{v3.1g}{2015/06/26}{Flag addgl set to false for
%   `«’ at the end of an \cs{hbox} or a paragraph or when followed by
%   a null glue (i.e. springs).}
%
%    \begin{macrocode}
    if lang == FR and FRspacing and FB_punct_right[char]
                                and fid > 0 and SIG then
       local next = item.next
       local next_id, next_subtype, next_char, nextnext, kern_wd
       if next then
          next_id = next.id
          next_subtype = next.subtype
          if next_id == GLYPH then
             next_char = next.char
%    \end{macrocode}
%    A |kern0| might hide a |glue|, so look ahead if |next| is a kern
%    (this occurs with |« \texttt{a} »|):
%    \begin{macrocode}
          elseif next_id == KERN then
             kern_wd = next.kern
             if kern_wd == 0 then
                nextnext = next.next
                if nextnext then
                   next = nextnext
                   next_id = nextnext.id
                   next_subtype = nextnext.subtype
                   if next_id == GLYPH then
                      next_char = nextnext.char
                   end
                end
             end
          end
       end
       local is_glue = next_id == GLUE
       if is_glue then
          glue_wd = next.width
       end
       local addgl = (next_char and not FB_guil_null[next_char]) or
                     (next and not next_char)
%    \end{macrocode}
%    Correction for tabular `c’ columns.  For `r’ columns, a final
%    `«’ character needs to be coded as |\mbox{«}| for proper spacing
%    (|\NoAutoSpacing| is another option).
%    \begin{macrocode}
       if is_glue and glue_wd == 0 then
          addgl = false
       end
       local fid = item.font
       local fbglue = new_glue_scaled(fid,guilgl)
       if addgl and fbglue then
          if is_glue then
             head = remove_node(head,next,true)
          end
          insert_node_after(head, item, copy_node(fbglue))
          insert_node_after(head, item, copy_node(nobreak))
       end
    end
  end
  return head
end
return french_punctuation
%    \end{macrocode}
%    \end{macro}
% \iffalse
%</lua>
%<*ldf>
% \fi
%
%  \begin{macro}{\FB@luatex@punct@french}
%    As a language tag is part of glyph nodes in LuaTeX, nothing needs
%    to be added to |\extrasfrench| and |\noextrasfrench|; we will just
%    redefine |\shorthandoff| and |\shorthandon| in French to issue a
%    warning reminding the user that active characters are no longer
%    used in French with recent LuaTeX engines.
%
% \changes{v3.1l}{2016/02/07}{Use \cs{babel@save} to save and restore
%    \cs{shorthandon} and \cs{shorthandoff}.}
%
%    \begin{macrocode}
\ifFB@luatex@punct
  \newcommand*{\FB@luatex@punct@french}{%
     \babel@save{\shorthandon}%
     \babel@save{\shorthandoff}%
     \def\shorthandoff##1{%
        \ifx\PackageWarning\@undefined
          \fb@warning{\noexpand\shorthandoff{;:!?} is helpless with
           LuaTeX,\\ use \noexpand\NoAutoSpacing
           *inside a group* instead.}%
        \else
          \PackageWarning{frenchb.ldf}{\protect\shorthandoff{;:!?} is
           helpless with LuaTeX,\MessageBreak use \protect\NoAutoSpacing
           \space *inside a group* instead;\MessageBreak reported}%
        \fi}%
     \def\shorthandon##1{}%
  }
  \FB@addto{extras}{\FB@luatex@punct@french}
%    \end{macrocode}
%  \end{macro}
%
% \changes{v3.0b}{2014/02/13}{frenchb.lua was not found by
%    Lua function dofile (not kpathsea aware).
%    Call function kpse.find\_file first, as suggested by Paul Gaborit.}
%
% \changes{v3.1g}{2015/06/17}{Lua function french\_punctuation is now
%    inserted at the end of the `kerning’ callback (no priority)
%    instead of `hpack\_filter’ and `pre\_linebreak\_filter’.}
%
%    In \LaTeXe, file \file{frenchb.lua} will be loaded
%    `AtBeginDocument’ \emph{after} processing options
%    (\fbo{ThinColonSpace} needs to be taken into account).  The next
%    definition will be used to activate Lua punctuation: it sets the
%    language number for French, loads \file{frenchb.lua} and adds
%    function |french_punctuation| at the end of the |kerning| callback
%    (no priority).
%    \begin{macrocode}
  \def\activate@luatexpunct{%
    \directlua{%
      FR = \the\l@french
      local path = kpse.find_file("frenchb.lua", "lua")
      if path then
         local f = dofile(path)
         luatexbase.add_to_callback("kerning",
                    f, "frenchb.french_punctuation")
      else
         texio.write_nl('')
         texio.write_nl('*****************************')
         texio.write_nl('Error: frenchb.lua not found.')
         texio.write_nl('*****************************')
         texio.write_nl('')
      end
     }%
  }
\fi
%    \end{macrocode}
%    End of specific code for punctuation with LuaTeX engines.
%
%  \subsubsection{Punctuation with XeTeX}
%  \label{sssec-punct-xetex}
%
%    If |\XeTeXinterchartokenstate| is available, we use the
%    ``inter char’’ mechanism to provide correct spacing in French
%    before the four characters |;| |!| |?| and |:|.  The basis of the
%    following code was borrowed from the \pkg{polyglossia} package,
%    see \file{gloss-french.ldf}.  We use the same mechanism for French
%    quotes (\texttt{«} and \texttt{»}), when automatic spacing for
%    quotes is required by options \fbo{og=«} and \fbo{fg=»} in
%    \fbsetup{} (see section~\ref{ssec-keyval}).
%
%    The default value for |\XeTeXcharclass| is 0 for characters tokens
%    and |\FB@nonchar| for all other tokens (glues, kerns, math and box
%    boundaries, etc.).
%    These defaults should not be changed otherwise the spacing before
%    the `high punctuation’ characters and inside quotes might not be
%    correct.
%
%    We switch |\XeTeXinterchartokenstate| to 1 and change the
%    |\XeTeXcharclass| values of |;| |!| |?| |:| |(| |]|
%    \texttt{«} and \texttt{»} when
%    entering French.  Special care is taken to restore them to their
%    inital values when leaving French.
%
% \changes{v2.5d}{2011/01/19}{Moved the \cs{newcount} command outside
%    \cs{ifFB@xetex@punct} ... \cs{fi} (it broke Plain formats).}
%
% \changes{v3.1g}{2015/06/14}{Use Babel defined loops \cs{bbl@for}
%    instead of \cs{@for} borrowed from file ltcntrl.dtx
%    (\cs{@for} is undefined in Plain).}
%
%    The following part holds specific code for punctuation with XeTeX
%    engines.
%
%    \begin{macrocode}
\ifFB@xetex@punct
   \ifLaTeXe
    \PackageInfo{frenchb.ldf}{No need for active punctuation characters%
                    \MessageBreak with this version of XeTeX!%
                    \MessageBreak reported}
   \else
    \fb@info{No need for active punctuation characters\\
             with this version of XeTeX!}
   \fi
%    \end{macrocode}
%
% \changes{v2.5i}{2012/04/20}{Temporary fix: as long as
%    \file{xeCJK.sty} will not use \cs{newXeTeXintercharclass} to
%    allocate its classes, we will have to define 3 fake classes.}
%
% \changes{v2.5j}{2012/05/08}{Previous fix removed: bug fixed in
%    \file{xeCJK.sty} version 3.0.4 (06-May-2012).}
%
% \changes{v2.6g}{2013/12/09}{U+00A0 (Unicode nobreakspace) and U+202F
%    (Unicode nobreakthinspace) added to class \cs{FB@punctnul}
%    to prevent frenchb from adding it’s own space before `high
%    punctuation’ characters.}
%
% \changes{v2.6h}{2014/01/03}{New class \cs{FB@guilnul} for characters
%    U+00A0 (Unicode nobreakspace) and U+202F (Unicode
%    nobreakthinspace), to prevent frenchb from adding spurious spaces
%    inside quotes.}
%
%    Six new character classes are defined for \frenchb.
%    \begin{macrocode}
   \newXeTeXintercharclass\FB@punctthick
   \newXeTeXintercharclass\FB@punctthin
   \newXeTeXintercharclass\FB@punctnul
   \newXeTeXintercharclass\FB@guilo
   \newXeTeXintercharclass\FB@guilf
   \newXeTeXintercharclass\FB@guilnul
%    \end{macrocode}
%
% \changes{v3.1l}{2016/02/07}{Add a variant of \cs{babel@savevariable}
%    to save \cs{XeTeXcharclass}(es) in a loop.}
%
%    As |\babel@savevariable| doesn’t work inside a |\bbl@for| loop,
%    we define a variant to save the |\XeTeXcharclass| values which
%    will be modified in French.
%    \begin{macrocode}
   \def\FBsavevariable@loop#1#2{\begingroup
     \toks@\expandafter{\originalTeX #1}%
     \edef\x{\endgroup
       \def\noexpand\originalTeX{\the\toks@ #2=\the#1#2\relax}}%
     \x}
%    \end{macrocode}
%    |\FB@charlist| holds the all list of characters which have their
%    |\XeTeXcharclass| value modified in French: the first set includes
%    high punctuation, French quotes, opening delimiters and no-break
%    spaces \\[5pt]
%    \begin{tabular}{*{10}{|c}|}\hline
%      "21 &"3A &"3B &"3F &"AB &"BB &"28 &"5B &"A0 &"202F
%      \\ \hline
%      \char"21 &\char"3A &\char"3B &\char"3F & « & »
%        &\char"28 &\char"5B  &  &  \\ \hline
%    \end{tabular}\\[5pt]
%    the second one holds those which need resetting in French when
%    \file{xeCJK.sty} is in use\\[5pt]
%    \begin{tabular}{*{12}{|c}|}\hline
%      "29 &"5D &"7B &"7D &"2C &"2D &"2E &"22 &"25 &"27 &"60 &"2019
%      \\ \hline
%      \char"29 &\char"5D &\char"7B &\char"7D &\char"2C &\char"2D
%      &\char"2E &\char"22 &\char"25 &\textquotesingle &\char"60
%      &\char"27 \\ \hline
%    \end{tabular}
%
%    \begin{macrocode}
   \def\FB@charlist{"21,"3A,"3B,"3F,"AB,"BB,"28,"5B,"A0,"202F,%
                    "29,"5D,"7B,"7D,"2C,"2D,"2E,"22,"25,"27,"60,"2019}
%    \end{macrocode}
%
%  \begin{macro}{\FB@xetex@punct@french}
%    The following command will be executed when entering French, it
%    first saves the values to be modified, then fits them to our
%    needs.  It also redefines |\shorthandoff| and |\shorthandon|
%    (locally) to avoid error messages with XeTeX-based engines.
%
% \changes{v2.5g}{2011/12/31}{XeTeXcharclass(es) for French quotes will
%    be set to \cs{FB@guilo} and \cs{FB@guilf} by options
%    `og’ and `fg’ in \cs{frenchbsetup}.  French quotes should behave
%    as normal characters by default in XeLaTeX as in LaTeX.}
%
% \changes{v2.5i}{2012/04/20}{\file{xeCJK.sty} changes the
%    \cs{XeTeXcharclass} of ASCII chars `-’ `,’ `.’ `)’ `]’ `\}’
%    `\{’ `\%’ opening and closing single and double quotes.
%    We set their class to 0 in French and reset their class
%    to their original value when leaving French.  See
%    \cs{FB@xetex@punct@nonfrench} below.}
%
% \changes{v3.1k}{2015/12/05}{Thin glues (less than 1sp) should not
%    trigger space insertion before high ponctuation. Add a check on
%    \cs{lastkip}.}
%
% \changes{v3.1l}{2016/02/07}{Save and restore
%    \cs{XeTeXinterchartokenstate}, \cs{shorthandon}, \cs{shorthandoff}
%    using \cs{babel@savevariable} and \cs{babel@save},
%    \cs{XeTeXcharclass}(es) using \cs{FB@savevariable@loop}.}
%
% \changes{v3.2c}{2016/05/14}{Switch \cs{ifFB@spacing} added to all
%    \cs{XeTeXinterchartoks} commands.}
%
%    \begin{macrocode}
   \newcommand*{\FB@xetex@punct@french}{%
     \babel@savevariable{\XeTeXinterchartokenstate}%
     \babel@save{\shorthandon}%
     \babel@save{\shorthandoff}%
     \bbl@for\FB@char\FB@charlist
          {\FBsavevariable@loop{\XeTeXcharclass}{\FB@char}}%
     \def\shorthandoff##1{%
       \ifx\PackageWarning\@undefined
         \fb@warning{\noexpand\shorthandoff{;:!?} is helpless with
           XeTeX,\\ use \noexpand\NoAutoSpacing
           *inside a group* instead.}%
       \else
         \PackageWarning{frenchb.ldf}{\protect\shorthandoff{;:!?} is
           helpless with XeTeX,\MessageBreak use \protect\NoAutoSpacing
           \space *inside a group* instead;\MessageBreak reported}%
       \fi}%
      \def\shorthandon##1{}%
%    \end{macrocode}
%    Let’s now set the classes and interactions between classes.
%    When false, the flag |\ifFB@spacing| switches off any interaction
%    between classes (this flag is controlled by user-level command
%    |\NoAutoSpacing|; this flag is also set to false when the current
%    font is a typewriter font).
%    \begin{macrocode}
     \XeTeXinterchartokenstate=1
     \XeTeXcharclass `\: = \FB@punctthick
     \XeTeXinterchartoks \z@ \FB@punctthick = {%
           \ifFB@spacing\ifhmode\FDP@colonspace\fi\fi}%
     \XeTeXinterchartoks \FB@guilf \FB@punctthick = {%
           \ifFB@spacing\FDP@colonspace\fi}%
%    \end{macrocode}
%    Small glues such as ``glue 1sp’’ in tabular `l’ columns or ``glue
%    0 plus 1 fil’’ in tabular `c’ columns or \env{lstlisting}
%    environment should not trigger any extra space; they will still do
%    when \fbo{AutoSpacePunctuation} is true: unfortunately
%    |\XeTeXcharclass=\FB@nonchar| isn’t specific to glue tokens (this
%    class includes box and math boundaries f.i.), so the |\else| part
%    cannot be omitted.
%    \begin{macrocode}
     \XeTeXinterchartoks \FB@nonchar \FB@punctthick = {%
           \ifFB@spacing
             \ifhmode
               \ifdim\lastskip>1sp
                 \unskip\penalty\@M\FBcolonspace
               \else
                 \FDP@colonspace
               \fi
             \fi
           \fi}%
     \bbl@for\FB@char
             {`\;,`\!,`\?}%
             {\XeTeXcharclass\FB@char=\FB@punctthin}%
     \XeTeXinterchartoks \z@ \FB@punctthin = {%
           \ifFB@spacing\ifhmode\FDP@thinspace\fi\fi}%
     \XeTeXinterchartoks \FB@guilf \FB@punctthin = {%
           \ifFB@spacing\FDP@thinspace\fi}%
     \XeTeXinterchartoks \FB@nonchar \FB@punctthin = {%
           \ifFB@spacing
             \ifhmode
               \ifdim\lastskip>1sp
                 \unskip\penalty\@M\FBthinspace
               \else
                 \FDP@thinspace
               \fi
             \fi
           \fi}%
     \XeTeXinterchartoks \FB@guilo \z@ = {%
           \ifFB@spacing\FB@guillspace\fi}%
     \XeTeXinterchartoks \FB@guilo \FB@nonchar = {%
           \ifFB@spacing\FB@guillspace\ignorespaces\fi}%
     \XeTeXinterchartoks \z@ \FB@guilf = {%
           \ifFB@spacing\FB@guillspace\fi}%
     \XeTeXinterchartoks \FB@punctthin \FB@guilf = {%
           \ifFB@spacing\FB@guillspace\fi}%
     \XeTeXinterchartoks \FB@nonchar \FB@guilf = {%
           \ifFB@spacing\unskip\FB@guillspace\fi}%
%    \end{macrocode}
%    This will avoid spurious spaces in (!), [?] and with Unicode
%    nobreakspaces (U+00A0, U+202F):
%    \begin{macrocode}
     \bbl@for\FB@char
             {`\[,`\(,"A0,"202F}%
             {\XeTeXcharclass\FB@char=\FB@punctnul}%
%    \end{macrocode}
%    These characters have their class changed by \file{xeCJK.sty},
%    let’s reset them to 0 in French.
%    \begin{macrocode}
     \bbl@for\FB@char
             {`\{,`\,,`\.,`\-,`\),`\],`\},`\%,"22,"27,"60,"2019}%
             {\XeTeXcharclass\FB@char=\z@}%
   }
   \FB@addto{extras}{\FB@xetex@punct@french}
%    \end{macrocode}
%  \end{macro}
%
%    End of specific code for punctuation with modern XeTeX engines.
%    \begin{macrocode}
\fi
%    \end{macrocode}
%
%  \subsubsection{Punctuation with standard (pdf)TeX}
%  \label{sssec-punct-tex}
%
%    In standard (pdf)TeX we need to make the four characters |;| |!| |?|
%    and |:| `active’ and provide their definitions.
%    \begin{macrocode}
\ifFB@active@punct
  \initiate@active@char{:}%
  \initiate@active@char{;}%
  \initiate@active@char{!}%
  \initiate@active@char{?}%
%    \end{macrocode}
%    We first tune the amount of space before \texttt{;}
%    \texttt{!}  \texttt{?} and \texttt{:}.  This should only happen
%    in horizontal mode, hence the test |\ifhmode|.
%
%    In horizontal mode, if a space has been typed before `;’ we
%    remove it and put an unbreakable |\FBthinspace| instead.  If no
%    space has been typed, we add |\FDP@thinspace| which will be
%    defined, up to the user’s wishes, as |\FBthinspace|, or as
%    |\@empty|.
%
% \changes{v3.1k}{2015/12/05}{(pdfTeX shorthands) test on \cs{lastskip}
%    changed from 0pt to 1sp for active punctuation for consistency
%    with XeTeX and LuaTeX.}
%
% \changes{v3.2c}{2016/05/14}{Switch \cs{ifFB@spacing} added to the
%    four French shorthands.}
%
%    \begin{macrocode}
  \declare@shorthand{french}{;}{%
    \ifFB@spacing
      \ifhmode
        \ifdim\lastskip>1sp
          \unskip\penalty\@M\FBthinspace
        \else
          \FDP@thinspace
        \fi
      \fi
    \fi
%    \end{macrocode}
%    Now we can insert a |;| character.
%    \begin{macrocode}
    \string;}
%    \end{macrocode}
%    The next three definitions are very similar.
%    \begin{macrocode}
  \declare@shorthand{french}{!}{%
    \ifFB@spacing
      \ifhmode
        \ifdim\lastskip>1sp
          \unskip\penalty\@M\FBthinspace
        \else
          \FDP@thinspace
        \fi
      \fi
    \fi
    \string!}
  \declare@shorthand{french}{?}{%
    \ifFB@spacing
      \ifhmode
        \ifdim\lastskip>1sp
          \unskip\penalty\@M\FBthinspace
        \else
          \FDP@thinspace
        \fi
      \fi
    \fi
    \string?}
  \declare@shorthand{french}{:}{%
    \ifFB@spacing
      \ifhmode
        \ifdim\lastskip>1sp
          \unskip\penalty\@M\FBcolonspace
        \else
          \FDP@colonspace
        \fi
      \fi
    \fi
    \string:}
%    \end{macrocode}
%    When the active characters appear in an environment where their
%    French behaviour is not wanted they should give an `expected’
%    result.  Therefore we define shorthands at system level as well.
%    \begin{macrocode}
  \declare@shorthand{system}{:}{\string:}
  \declare@shorthand{system}{!}{\string!}
  \declare@shorthand{system}{?}{\string?}
  \declare@shorthand{system}{;}{\string;}
%}
%    \end{macrocode}
%    We specify that the French group of shorthands should be used when
%    switching to French.
%    \begin{macrocode}
  \FB@addto{extras}{\languageshorthands{french}%
%    \end{macrocode}
%    These characters are `turned on’ once, later their definition may
%    vary.  Don’t misunderstand the following code: they keep being
%    active all along the document, even when leaving French.
%    \begin{macrocode}
    \bbl@activate{:}\bbl@activate{;}%
    \bbl@activate{!}\bbl@activate{?}%
  }
  \FB@addto{noextras}{%
    \bbl@deactivate{:}\bbl@deactivate{;}%
    \bbl@deactivate{!}\bbl@deactivate{?}%
  }
\fi
%    \end{macrocode}
%
%  \subsubsection{Punctuation switches common to all engines}
%  \label{sssec-punct-switches}
%
%    A new `if’ |\ifFBAutoSpacePunctuation| needs to be defined now to
%    control the two possible ways of dealing with `high punctuation’.
%    it’s default value is true, but it can be set to false by
%    \fbsetup{AutoSpacePunctuation=false} for finer control.
%    \begin{macrocode}
\newif\ifFBAutoSpacePunctuation  \FBAutoSpacePunctuationtrue
%    \end{macrocode}
%
%  \begin{macro}{\AutoSpaceBeforeFDP}
%  \begin{macro}{\NoAutoSpaceBeforeFDP}
%
% \changes{v2.3a}{2008/10/10}{\cs{NoAutoSpaceBeforeFDP} and
%    \cs{AutoSpaceBeforeFDP} now set the flag
%    \cs{ifFBAutoSpacePunctuation} accordingly (LaTeX only).}
%
% \changes{v2.3e}{2009/10/10}{Execute \cs{AutoSpaceBeforeFDP}
%    also in LaTeX to define \cs{FDP@colonspace}: needed for
%    tex4ht, pointed out by MPG.}
%
%    |\autospace@beforeFDP| and |\noautospace@beforeFDP| are internal
%    commands.  |\autospace@beforeFDP| defines |\FDP@thinspace| and
%    |\FDP@colonspace| as unbreakable spaces and sets LuaTeX attribute
%    |\FB@addDPspace| to 1 (true), while |\noautospace@beforeFDP|
%    lets these spaces empty and  sets flag |\FB@addDPspace|
%    to 0 (false).  User commands |\AutoSpaceBeforeFDP| and
%    |\NoAutoSpaceBeforeFDP| do the same and take care of the flag
%    |\ifFBAutoSpacePunctuation| in \LaTeX{}.
%    Set the default now for Plain (done later for \LaTeX).
%    \begin{macrocode}
\def\autospace@beforeFDP{%
         \ifFB@luatex@punct\FB@addDPspace=1 \fi
         \def\FDP@thinspace{\penalty\@M\FBthinspace}%
         \def\FDP@colonspace{\penalty\@M\FBcolonspace}}
\def\noautospace@beforeFDP{%
         \ifFB@luatex@punct\FB@addDPspace=0 \fi
         \let\FDP@thinspace\@empty
         \let\FDP@colonspace\@empty}
\ifLaTeXe
    \def\AutoSpaceBeforeFDP{\autospace@beforeFDP
                            \FBAutoSpacePunctuationtrue}
    \def\NoAutoSpaceBeforeFDP{\noautospace@beforeFDP
                              \FBAutoSpacePunctuationfalse}
    \AtEndOfPackage{\AutoSpaceBeforeFDP}
\else
    \let\AutoSpaceBeforeFDP\autospace@beforeFDP
    \let\NoAutoSpaceBeforeFDP\noautospace@beforeFDP
    \AutoSpaceBeforeFDP
\fi
%    \end{macrocode}
% \end{macro}
% \end{macro}
%
% \changes{v2.3a}{2008/10/10}{In LaTeX, frenchb no longer adds spaces
%    before `high punctuation’ characters in computer code.
%    Suggested by Yannis Haralambous.}
%
% \begin{macro}{\rmfamilyFB}
% \begin{macro}{\sffamilyFB}
% \begin{macro}{\ttfamilyFB}
%
% \changes{v2.3c}{2009/02/07}{Commands \cs{ttfamily}, \cs{rmfamily}
%    and \cs{sffamily} have to be robust.  Bug introduced in 2.3a,
%    pointed out by Manuel Pégourié-Gonnard.}
%
%    In \LaTeXe{} |\ttfamily| (and hence |\texttt|) will be redefined
%    `AtBeginDocument’ as |\ttfamilyFB| so that no space
%    is added before the four |; : ! ?| characters, even if
%    \fbo{AutoSpacePunctuation} is \fbo{true}.  When
%    \fbo{AutoSpacePunctuation} is \fbo{false}, the eventually typed
%    spaces are left unchanged (not turned into thin spaces, no penalty
%    added).  |\rmfamily| and |\sffamily| need to be redefined also
%    (|\ttfamily| is not always used inside a group, its effect can be
%    cancelled by |\rmfamily| or |\sffamily|).
%
%    These redefinitions can be canceled if necessary, for instance to
%    recompile older documents, see option \fbo{OriginalTypewriter}
%    below.
%
% \changes{v2.4c}{2010/05/23}{In \cs{ttfamilyFB}, also cancel
%    automatic spaces inside French guillemets entered as characters
%    (see \cs{frenchbsetup}).}
%
% \changes{v3.2c}{2016/05/14}{New definitions of \cs{ttfamilyFB} and
%    co, common to all engines, based on \cs{FB@spacing@off}
%    and\cs{FB@spacing@on}.}
%
%    To be consistent with what is done for the |; : ! ?|
%    characters, |\ttfamilyFB| also switches off insertion of spaces
%    inside French guillemets \emph{when they are typed in as
%    characters} with the `og’/`fg’ options in \fbsetup{}.
%    This is also a workaround for the weird behaviour of these
%    characters in verbatim mode.
%
%    \begin{macrocode}
\ifLaTeXe
  \DeclareRobustCommand\ttfamilyFB{\FB@spacing@off \ttfamilyORI}
  \DeclareRobustCommand\rmfamilyFB{\FB@spacing@on  \rmfamilyORI}
  \DeclareRobustCommand\sffamilyFB{\FB@spacing@on  \sffamilyORI}
\fi
%    \end{macrocode}
% \end{macro}
% \end{macro}
% \end{macro}
%
% \changes{v2.5a}{2010/08/14}{New command \cs{NoAutoSpacing},
%    suggested by MPG.}
%
%  \begin{macro}{\NoAutoSpacing}
%    The following command disables automatic spacing for high
%    punctuation and French quote characters; it also switches off
%    active punctuation characters (if any).
%    It is engine independent (works for TeX, LuaTeX and
%    XeTeX based engines) and is meant to be used inside a group.
%
% \changes{v3.2b}{2016/04/18}{\cs{NoAutoSpacing} made robust.}
%
% \changes{v3.2c}{2016/05/14}{New definition based on
%    \cs{FB@spacing@off} common to all engines.}
%
%    \begin{macrocode}
\DeclareRobustCommand*{\NoAutoSpacing}{%
  \FB@spacing@off
  \ifFB@active@punct\shorthandoff{;:!?}\fi
}
%    \end{macrocode}
%  \end{macro}
%
%  \subsection{Commands for French quotation marks}
%  \label{ssec-quotes}
%
%  \begin{macro}{\guillemotleft}
%  \begin{macro}{\guillemotright}
%  \begin{macro}{\textquoteddblleft}
%  \begin{macro}{\textquoteddblright}
%    With pdfLaTeX \LaTeX{} users are supposed to use 8-bit output
%    encodings (T1, LY1,\dots) to typeset French, those who still
%    stick to OT1 should load \pkg{aeguill} or a similar package.
%    In both cases the commands |\guillemotleft| and |\guillemotright|
%    will print the French opening and closing quote characters from
%    the output font.
%    For XeLaTeX and LuaLaTeX, |\guillemotleft| and |\guillemotright|
%    are defined by package \pkg{fontspec} (v.~2.5d and up).
%
% \changes{v2.5a}{2010/08/20}{Change \cs{guillemotleft} and
%    \cs{guillemotright} definitions for Unicode anf provide
%    definitions for \cs{textquotedblleft} and
%    \cs{textquotedbright}.  Insures correct printing of quotes
%    by \cs{og} and \cs{fg} in French and outside.}
%
%    We provide the following definitions for non-LaTeX users only as
%    fall-back, they are welcome to change them for anything better.
%    \begin{macrocode}
\ifLaTeXe
\else
  \ifFBunicode
    \def\guillemotleft{{\char"00AB}}
    \def\guillemotright{{\char"00BB}}
    \def\textquotedblleft{{\char"201C}}
    \def\textquotedblright{{\char"201D}}
  \else
    \def\guillemotleft{\leavevmode\raise0.25ex
                       \hbox{$\scriptscriptstyle\ll$}}
    \def\guillemotright{\raise0.25ex
                        \hbox{$\scriptscriptstyle\gg$}}
    \def\textquotedblleft{``}
    \def\textquotedblright{''}
  \fi
  \let\xspace\relax
\fi
%    \end{macrocode}
%  \end{macro}
%  \end{macro}
%  \end{macro}
%  \end{macro}
%
%  \begin{macro}{\FB@og}
%  \begin{macro}{\FB@fg}
%  \begin{macro}{\FBguillspace}
%    The next step is to provide correct spacing after `«’ and before
%    `»’; no line break is allowed neither \emph{after} the opening
%    one, nor \emph{before} the closing one.
%    |\FBguillspace| which does the spacing, has been fine tuned by
%    Thierry Bouche to 80\% of an inter-word space with reduced
%    stretchability.  French quotes (including spacing) are printed by
%    |\FB@og| and |\FB@fg|, the expansion of the top level commands
%    |\og| and |\og| is different in and outside French.
%
% \changes{v3.0a}{2014/01/10}{Definitions of \cs{FB@og} and \cs{FB@fg}
%    now depend on punctuation handling (LuaTeX / XeTeX / active).}
%
% \changes{v3.3a}{2016/04/30}{Skip \cs{FBguillskip} for LuaTeX
%    replaced by toks \cs{FBguillsp}.}
%
%    LuaTeX requires toks; |\FBguillsp| will be computed from
%    |\FBguillspace| `AtBeginDocument’, its dimensions will
%    be scaled by \file{frenchb.lua} for the current font and used
%    after `«’ and before `»’.
%    \begin{macrocode}
\newcommand*{\FBguillspace}{\hskip.8\fontdimen2\font
                              plus.3\fontdimen3\font
                             minus.8\fontdimen4\font \relax}
\newcommand*{\FB@guillspace}{\penalty\@M\FBguillspace}
\newtoks\FBguillsp
%    \end{macrocode}
%    The definitions of |\FB@og| and |\FB@fg| need some engine-dependent
%    tuning: for LuaTeX, |\FB@spacing| is set to~0 locally to prevent
%    the quotes characters from adding space when option
%    \fbo{og=«, fg=»} is set.
%    \begin{macrocode}
\ifFB@luatex@punct
  \DeclareRobustCommand*{\FB@og}{\leavevmode
          \bgroup\FB@spacing=0 \guillemotleft\egroup
          \FB@guillspace}
  \DeclareRobustCommand*{\FB@fg}{\ifdim\lastskip>\z@\unskip\fi
          \FB@guillspace
          \bgroup\FB@spacing=0 \guillemotright\egroup}
\fi
%    \end{macrocode}
%    With XeTeX, |\ifFB@spacing| is set to |false| locally for the
%    same reason.
%    \begin{macrocode}
\ifFB@xetex@punct
  \DeclareRobustCommand*{\FB@og}{\leavevmode
        \bgroup\FB@spacingfalse\guillemotleft\egroup
        \FB@guillspace}
  \DeclareRobustCommand*{\FB@fg}{\ifdim\lastskip>\z@\unskip\fi
        \FB@guillspace
        \bgroup\FB@spacingfalse\guillemotright\egroup}
\fi
\ifFB@active@punct
  \DeclareRobustCommand*{\FB@og}{\leavevmode
        \guillemotleft
        \FB@guillspace}
  \DeclareRobustCommand*{\FB@fg}{\ifdim\lastskip>\z@\unskip\fi
        \FB@guillspace
        \guillemotright}
\fi
%    \end{macrocode}
%  \end{macro}
%  \end{macro}
%  \end{macro}
%
%  \begin{macro}{\og}
%  \begin{macro}{\fg}
%
%    The user level macros for quotation marks are named |\og|
%    (``\underline{o}uvrez \underline{g}uillemets’’) and |\fg|
%    (``\underline{f}ermez \underline{g}uillemets’’).
%    Another option for typesetting quotes in French is to use the
%    command |\frquote| (see below).
%    Dummy definition of |\og| and |\fg| just to ensure that this
%    commands are not yet defined.
%    \begin{macrocode}
\newcommand*{\og}{\@empty}
\newcommand*{\fg}{\@empty}
%    \end{macrocode}
%
%    The definitions of |\og| and |\fg| for quotation marks are
%    switched on and off through the |\extrasfrench| |\noextrasfrench|
%    mechanism.  Outside French, |\og| and |\fg| will typeset standard
%    English opening and closing double quotes.
%    We’ll try to be smart to users of David~Carlisle’s \pkg{xspace}
%    package: if this package is loaded there will be no need for |{}|
%    or |\ | to get a space after |\fg|, otherwise |\xspace| will be
%    defined as |\relax| (done at the end of this file).
%
% \changes{v2.5a}{2010/08/20}{\cs{og} and \cs{fg} do not print
%    correctly in English when using XeTeX or LuaTeX, fixed by using
%    \cs{textquotedblleft} and \cs{textquotedblright} defined above.}
%
% \changes{v3.2a}{2016/03/18}{\cs{xspace} moved from \cs{FB@fg} to
%    \cs{fg}: \cs{xspace} messes up \cs{frquote}, pointed out by Sonia
%    Labetoulle.  As a side effect \cs{xspace} is now active in \cs{fg}
%    in and outside French.}
%
%    \begin{macrocode}
\ifLaTeXe
  \def\bbl@frenchguillemets{\renewcommand*{\og}{\FB@og}%
                            \renewcommand*{\fg}{\FB@fg\xspace}}
  \renewcommand*{\og}{\textquotedblleft}
  \renewcommand*{\fg}{\ifdim\lastskip>\z@\unskip\fi
                      \textquotedblright\xspace}
\else
  \def\bbl@frenchguillemets{\let\og\FB@og
                            \let\fg\FB@fg}
  \def\og{\textquotedblleft}
  \def\fg{\ifdim\lastskip>\z@\unskip\fi\textquotedblright}
\fi
%    \end{macrocode}
%  \end{macro}
%  \end{macro}
%
% \changes{v3.0a}{2014/01/10}{\cs{bbl@nonfrenchguillemets} deleted,
%    use \cs{babel@save} instead.}
%
%    \begin{macrocode}
\FB@addto{extras}{\babel@save\og \babel@save\fg \bbl@frenchguillemets}
%    \end{macrocode}
%
% \changes{v3.1a}{2014/05/20}{New command \cs{frquote} for imbedded or
%    long French quotations.}
%
%    \begin{macro}{\frquote}
%    Another way of entering French quotes relies on |\frquote{}|
%    with supports up to two levels of quotes.
%    Let’s define the default quote characters to be used for level one
%    or two of quotes\dots{}
%    \begin{macrocode}
\newcommand*{\ogi}{\FB@og}
\newcommand*{\fgi}{\FB@fg}
\newcommand*{\ogii}{\textquotedblleft}
\newcommand*{\fgii}{\textquotedblright}
%    \end{macrocode}
%    and the needed technical stuff to handle options:
%
% \changes{v3.2g}{2017/01/30}{Default options of \cs{frquote} are no
%    longer engine-dependent.}
%
%    \begin{macrocode}
\newcount\FBguill@level
\newtoks\FB@everypar
\newif\ifFBcloseguill  \FBcloseguilltrue
\newif\ifFBInnerGuillSingle
\def\FBguillopen{\bgroup\NoAutoSpacing\guillemotleft\egroup}
\def\FBguillclose{\bgroup\NoAutoSpacing\guillemotright\egroup}
\let\FBguillnone\empty
\let\FBeveryparguill\FBguillopen
\let\FBeverylineguill\FBguillnone
%    \end{macrocode}
%    The main command |\frquote| accepts (in \LaTeXe{} only) a starred
%    version which suppresses the closing quote; it is meant to be used
%    for inner quotations which end together with the outer one, then
%    only one closing guillemet (the outer one) should be printed.
%    \begin{macrocode}
\ifLaTeXe
  \DeclareRobustCommand\frquote{%
       \@ifstar{\FBcloseguillfalse\fr@quote}%
               {\FBcloseguilltrue\fr@quote}}
\else
  \newcommand\frquote[1]{\fr@quote{#1}}
\fi
%    \end{macrocode}
%
% \changes{v3.1j}{2015/11/25}{\cs{fr@quote} completely rewritten:
%    \cs{leavevmode} added and explicitly save/retore \cs{everypar} and
%    \cs{localleftbox} instead of using a group in order to ensure
%    compatibility with package wrapfig.}
%
% \changes{v3.1j}{2015/11/21}{\cs{PackageWarning} is undefined in
%    Plain, use \cs{fb@warning} instead.}
%
% \changes{v3.3a}{2017/04/30}{\cs{FB@quotespace} (kern), changed
%    into \cs{FB@guillspace}.}
%
%    The internal command |\fr@quote| takes one (long) argument: the
%    quotation text.
%    \begin{macrocode}
\newcommand{\fr@quote}[1]{%
  \leavevmode
  \advance\FBguill@level by \@ne
  \ifcase\FBguill@level
  \or
%    \end{macrocode}
%    This for level 1 (outer) quotations: save |\everypar| before
%    customising it, set |\FBeverypar@quote| for level 1 quotations
%    and add it to |\everypar|, then print the quotation:
%    \begin{macrocode}
    \FB@everypar=\everypar
    \ifx\FBeveryparguill\FBguillnone
    \else
      \def\FBeverypar@quote{\FBeveryparguill\FB@guillspace}%
      \everypar=\expandafter{\the\everypar \FBeverypar@quote}%
    \fi
    \ogi #1\fgi
  \or
%    \end{macrocode}
%    This for level 2 (inner) quotations: Omega’s command
%    |\localleftbox| included in LuaTeX, is convenient for repeating
%    guillemets at the beginning of every line.\label{cs:localleftbox}
%
% \changes{v3.1i}{2015/10/04}{\cs{luatexlocalleftbox} changed to
%    \cs{localleftbox} by new LaTeX release 2015/10/01.}
%
%    \begin{macrocode}
    \ifx\FBeverylineguill\FBguillopen
      \localleftbox{\guillemotleft\FB@guillspace}%
      \let\FBeverypar@quote\relax
      \ogi #1\ifFBcloseguill\fgi\fi
    \else
      \ifx\FBeverylineguill\FBguillclose
        \localleftbox{\guillemotright\FB@guillspace}%
        \let\FBeverypar@quote\relax
        \ogi #1\ifFBcloseguill\fgi\fi
      \else
%    \end{macrocode}
%    otherwise we need to redefine |\FBeverypar@quote| (and eventually
%    |\ogii|, |\fgii|) for level 2 quotations:
%    \begin{macrocode}
        \let\FBeverypar@quote\relax
        \ifFBInnerGuillSingle
          \def\ogii{\leavevmode
                    \guilsinglleft\FB@guillspace}%
          \def\fgii{\ifdim\lastskip>\z@\unskip\fi
                    \FB@guillspace\guilsinglright}%
          \ifx\FBeveryparguill\FBguillopen
            \def\FBeverypar@quote{\guilsinglleft\FB@guillspace}%
          \fi
          \ifx\FBeveryparguill\FBguillclose
            \def\FBeverypar@quote{\guilsinglright\FB@guillspace}%
          \fi
        \fi
        \ogii #1\ifFBcloseguill \fgii \fi
      \fi
    \fi
  \else
%    \end{macrocode}
%    Warn if |\FBguill@level| $\ge 3$:
%    \begin{macrocode}
    \ifx\PackageWarning\@undefined
      \fb@warning{\noexpand\frquote\space handles up to
                  two levels.\\ Quotation not printed.}%
    \else
      \PackageWarning{frenchb.ldf}{%
         \protect\frquote\space handles up to two levels.
         \MessageBreak Quotation not printed.  Reported}
    \fi
  \fi
%    \end{macrocode}
%    Clean on exit: adjust |\FBguill@level| and restore
%    |\localleftbox| and |\everypar|.
%    \begin{macrocode}
  \advance\FBguill@level by \m@ne
  \ifx\FBeverylineguill\FBguillnone\else\localleftbox{}\fi
  \ifx\FBeveryparguill\FBguillnone\else\everypar=\FB@everypar\fi
}
%    \end{macrocode}
%    \end{macro}
%
%  \subsection{Date in French}
% \label{sssec-date}
%
% \begin{macro}{\datefrench}
%    The macro |\datefrench| redefines the command |\today| to
%    produce French dates. This new implementation requires \babel~3.9i
%    or newer but, as of 3.9k, doesn’t work with Plain based formats,
%    so |\date\CurrentOption| is defined the old way for these formats.
%
% \changes{v2.0}{2006/11/06}{2 `\cs{relax}’ added in
%    \cs{today}’s definition.}
%
% \changes{v2.1a}{2008/03/25}{\cs{today} changed (correction in 2.0
%    was wrong: \cs{today} was printed without spaces in toc).}
%
% \changes{v3.0a}{2014/02/18}{Take advantage of babel’s \cs{SetString}
%    commands for \cs{datefrench}.  Doesn’t work with Plain (yet?).}
%
% \changes{v3.0c}{2014/03/26}{\cs{SetString} still does not work for
%    Plain with babel 3.9k.  Need to define \cs{datefrench}.}
%
%    \begin{macrocode}
\ifLaTeXe
  \def\BabelLanguages{french,acadian}
  \StartBabelCommands*{\BabelLanguages}{date}
      [unicode, fontenc=EU1 EU2, charset=utf8]
    \SetString\monthiiname{février}
    \SetString\monthviiiname{août}
    \SetString\monthxiiname{décembre}
  \StartBabelCommands*{\BabelLanguages}{date}
    \SetStringLoop{month#1name}{%
        janvier,f\'evrier,mars,avril,mai,juin,juillet,%
        ao\^ut,septembre,octobre,novembre,d\'ecembre}
    \SetString\today{{\number\day}\ifnum1=\day {\ier}\fi\space
        \csname month\romannumeral\month name\endcsname \space
        \number\year
       }
  \EndBabelCommands
\else
  \ifFBunicode
    \@namedef{date\CurrentOption}{%
      \def\today{{\number\day}\ifnum1=\day {\ier}\fi \space
          \ifcase\month
            \or janvier\or février\or mars\or avril\or mai\or
            juin\or juillet\or août\or septembre\or
            octobre\or novembre\or décembre\fi
          \space \number\year}}
  \else
    \@namedef{date\CurrentOption}{%
      \def\today{{\number\day}\ifnum1=\day {\ier}\fi \space
        \ifcase\month
          \or janvier\or f\'evrier\or mars\or avril\or mai\or
          juin\or juillet\or ao\^ut\or septembre\or
          octobre\or novembre\or d\'ecembre\fi
        \space \number\year}}
  \fi
\fi
%    \end{macrocode}
% \end{macro}
%
%  \subsection{Extra utilities}
%
%    Let’s provide the French user with some extra utilities.
%
% \changes{v2.1a}{2008/03/24}{Command \cs{fup} added to produce
%    better superscripts than \cs{textsuperscript}.}
%
%  \begin{macro}{\up}
%
% \changes{v2.1c}{2008/04/29}{Provide a temporary definition
%    (hyperref safe) of \cs{up} in case it has to be expanded in
%    the preamble (by beamer’s \cs{title} command for instance).}
%
% \changes{v2.4d}{2010/07/28}{Command \cs{up} defined with
%    \cs{providecommand} instead of \cs{newcommand} as \cs{up} may be
%    defined elsewhere (catalan.ldf).
%    Bug pointed out by Felip Manyé i Ballester.}
%
%  \begin{macro}{\fup}
%
% \changes{v2.1b}{2008/04/02}{Command \cs{fup} changed to use
%    real superscripts from fourier v.~1.6.}
%
% \changes{v2.2a}{2008/05/08}{\cs{newif} and \cs{newdimen} moved
%    before \cs{ifLaTeXe} to avoid an error with plainTeX.}
%
% \changes{v2.3a}{2008/09/30}{\cs{lowercase} changed to
%    \cs{MakeLowercase} as the former doesn’t work for non ASCII
%    characters in encodings like applemac, utf-8,\dots}
%
%    |\up| eases the typesetting of superscripts like
%    `1\up{er}’.  Up to version 2.0 of \frenchb{} |\up| was just a
%    shortcut for |\textsuperscript| in \LaTeXe, but several users
%    complained that |\textsuperscript| typesets superscripts
%    too high and too big, so we now define |\fup| as an attempt to
%    produce better looking superscripts.  |\up| is defined as |\fup|
%    but \fbsetup{FrenchSuperscripts=false} redefines |\up|
%    as |\textsuperscript| for compatibility with previous versions.
%
%    When a font has built-in superscripts, the best thing to do is
%    to just use them, otherwise |\fup| has to simulate superscripts
%    by scaling and raising ordinary letters.  Scaling is done using
%    package \pkg{scalefnt} which will be loaded at the end of
%    \babel’s loading (\frenchb{} being an option of \babel, it cannot
%    load a package while being read).
%
%    \begin{macrocode}
\newif\ifFB@poorman
\newdimen\FB@Mht
\ifLaTeXe
  \AtEndOfPackage{\RequirePackage{scalefnt}}
%    \end{macrocode}
%    |\FB@up@fake| holds the definition of fake superscripts.
%    The scaling ratio is 0.65, raising is computed to put the top of
%    lower case letters (like `m’) just under the top  of upper case
%    letters (like `M’), precisely 12\% down.  The chosen settings
%    look correct for most fonts, but can be tuned by the end-user
%    if necessary by changing |\FBsupR| and |\FBsupS| commands.
%
%    |\FB@lc| is defined as |\MakeLowercase| to inhibit the uppercasing
%    of superscripts (this may happen in page headers with the standard
%    classes but is wrong); |\FB@lc| can be redefined to do nothing
%    by option \fbo{LowercaseSuperscripts=false} of \fbsetup{}.
%    \begin{macrocode}
  \newcommand*{\FBsupR}{-0.12}
  \newcommand*{\FBsupS}{0.65}
  \newcommand*{\FB@lc}[1]{\MakeLowercase{#1}}
  \DeclareRobustCommand*{\FB@up@fake}[1]{%
    \settoheight{\FB@Mht}{M}%
    \addtolength{\FB@Mht}{\FBsupR \FB@Mht}%
    \addtolength{\FB@Mht}{-\FBsupS ex}%
    \raisebox{\FB@Mht}{\scalefont{\FBsupS}{\FB@lc{#1}}}%
    }
%    \end{macrocode}
%    The only packages I currently know to take advantage of real
%    superscripts are a) \pkg{realscripts} used in conjunction with
%    XeLaTeX or LuaLaTeX and OpenType fonts having the font feature
%    `VerticalPosition=Superior’ and b) \pkg{fourier}
%    (from version 1.6) when Expert Utopia fonts are available.
%
%    |\FB@up| checks whether the current font is a Type1 `Expert’
%    (or `Pro’) font with real superscripts or not (the code works
%    currently only with \file{fourier-1.6} but could work with any
%    Expert Type1 font with built-in superscripts, see below), and
%    decides to use real or fake superscripts.
%    It works as follows: the content of |\f@family| (family name of
%    the current font) is split by |\FB@split| into two pieces, the
%    first three characters (`|fut|’ for Fourier, `|ppl|’ for Adobe’s
%    Palatino, \dots) stored in |\FB@firstthree| and the rest stored
%    in |\FB@suffix| which is expected to be `|x|’ or `|j|’ for expert
%    fonts.
%    \begin{macrocode}
  \def\FB@split#1#2#3#4\@nil{\def\FB@firstthree{#1#2#3}%
                             \def\FB@suffix{#4}}
  \def\FB@x{x}
  \def\FB@j{j}
  \DeclareRobustCommand*{\FB@up}[1]{%
    \bgroup \FB@poormantrue
      \expandafter\FB@split\f@family\@nil
%    \end{macrocode}
%    Then |\FB@up| looks for a \file{.fd} file named \file{t1fut-sup.fd}
%    (Fourier) or \file{t1ppl-sup.fd} (Palatino), etc.\ supposed to
%    define the subfamily (|fut-sup| or |ppl-sup|, etc.) giving access
%    to the built-in superscripts.  If the \file{.fd} file is not found
%    by |\IfFileExists|, |\FB@up| falls back on fake superscripts,
%    otherwise |\FB@suffix| is checked to decide whether to use fake or
%    real superscripts.
%    \begin{macrocode}
      \edef\reserved@a{\lowercase{%
         \noexpand\IfFileExists{\f@encoding\FB@firstthree -sup.fd}}}%
      \reserved@a
        {\ifx\FB@suffix\FB@x \FB@poormanfalse\fi
         \ifx\FB@suffix\FB@j \FB@poormanfalse\fi
         \ifFB@poorman \FB@up@fake{#1}%
         \else         \FB@up@real{#1}%
         \fi}%
        {\FB@up@fake{#1}}%
    \egroup}
%    \end{macrocode}
%    |\FB@up@real| just picks up the superscripts from the subfamily
%    (and forces lowercase).
%    \begin{macrocode}
  \newcommand*{\FB@up@real}[1]{\bgroup
       \fontfamily{\FB@firstthree -sup}\selectfont \FB@lc{#1}\egroup}
%    \end{macrocode}
%    |\fup| is defined as |\FB@up| unless |\realsuperscript| is
%    defined by \file{realscripts.sty}.
%    \begin{macrocode}
  \DeclareRobustCommand*{\fup}[1]{%
    \ifx\realsuperscript\@undefined
      \FB@up{#1}%
    \else
      \bgroup\let\fakesuperscript\FB@up@fake
             \realsuperscript{\FB@lc{#1}}\egroup
    \fi}
%    \end{macrocode}
%    Let’s provide a temporary definition for |\up| (redefined
%    `AtBeginDocument’ as |\fup| or |\textsuperscript| according to
%    \fbsetup{} options).
%    \begin{macrocode}
  \providecommand*{\up}{\relax}
%    \end{macrocode}
%    Poor man’s definition of |\up| for Plain.
%    \begin{macrocode}
\else
  \providecommand*{\up}[1]{\leavevmode\raise1ex\hbox{\sevenrm #1}}
\fi
%    \end{macrocode}
%  \end{macro}
%  \end{macro}
%
%  \begin{macro}{\ieme}
%  \begin{macro}{\ier}
%  \begin{macro}{\iere}
%  \begin{macro}{\iemes}
%  \begin{macro}{\iers}
%  \begin{macro}{\ieres}
% \changes{v3.1b}{2014/06/23}{Removed \cs{lowercase} from definitions
%    of \cs{ieme} and co: \cs{up} already does the conversion.}
%
%    Some handy macros for those who don’t know how to abbreviate
%    ordinals:
%    \begin{macrocode}
\def\ieme{\up{e}\xspace}
\def\iemes{\up{es}\xspace}
\def\ier{\up{er}\xspace}
\def\iers{\up{ers}\xspace}
\def\iere{\up{re}\xspace}
\def\ieres{\up{res}\xspace}
%    \end{macrocode}
%  \end{macro}
%  \end{macro}
%  \end{macro}
%  \end{macro}
%  \end{macro}
%  \end{macro}
%
% \changes{v2.1c}{2008/04/29}{Added commands \cs{Nos} and \cs{nos}.}
%
%  \begin{macro}{\No}
%  \begin{macro}{\no}
%  \begin{macro}{\Nos}
%  \begin{macro}{\nos}
%  \begin{macro}{\primo}
%  \begin{macro}{\fprimo)}
% \changes{v3.1b}{2014/06/23}{Removed \cs{lowercase} from definitions
%    of \cs{FrenchEnumerate}, \dots{} \cs{No} and co: \cs{up} already
%    does the conversion.}
%
%    And some more macros relying on |\up| for numbering,
%    first two support macros.
%    \begin{macrocode}
\newcommand*{\FrenchEnumerate}[1]{%
                       #1\up{o}\kern+.3em}
\newcommand*{\FrenchPopularEnumerate}[1]{%
                       #1\up{o})\kern+.3em}
%    \end{macrocode}
%
%    Typing |\primo| should result in `\up{o}\kern+.3em’,
%    \begin{macrocode}
\def\primo{\FrenchEnumerate1}
\def\secundo{\FrenchEnumerate2}
\def\tertio{\FrenchEnumerate3}
\def\quarto{\FrenchEnumerate4}
%    \end{macrocode}
%    while typing |\fprimo)| gives `\up{o})\kern+.3em.
%    \begin{macrocode}
\def\fprimo){\FrenchPopularEnumerate1}
\def\fsecundo){\FrenchPopularEnumerate2}
\def\ftertio){\FrenchPopularEnumerate3}
\def\fquarto){\FrenchPopularEnumerate4}
%    \end{macrocode}
%
%    Let’s provide four macros for the common abbreviations
%    of ``Numéro’’.
%    \begin{macrocode}
\DeclareRobustCommand*{\No}{N\up{o}\kern+.2em}
\DeclareRobustCommand*{\no}{n\up{o}\kern+.2em}
\DeclareRobustCommand*{\Nos}{N\up{os}\kern+.2em}
\DeclareRobustCommand*{\nos}{n\up{os}\kern+.2em}
%    \end{macrocode}
%  \end{macro}
%  \end{macro}
%  \end{macro}
%  \end{macro}
%  \end{macro}
%  \end{macro}
%
%  \begin{macro}{\bsc}
%    As family names should be written in small capitals and never be
%    hyphenated, we provide a command (its name comes from Boxed Small
%    Caps) to input them easily.  Note that this command has changed
%    with version~2 of \frenchb: a |\kern0pt| is used instead of |\hbox|
%    because |\hbox| would break microtype’s font expansion; as a
%    (positive?) side effect, composed names (such as Dupont-Durand)
%    can now be hyphenated on explicit hyphens.
%    Usage: |Jean~\bsc{Duchemin}|.
%
% \changes{v2.0}{2006/11/06}{\cs{hbox} dropped, replaced by
%    \cs{kern0pt}.}
%
%    \begin{macrocode}
\DeclareRobustCommand*{\bsc}[1]{\leavevmode\begingroup\kern0pt
                                           \scshape #1\endgroup}
\ifLaTeXe\else\let\scshape\relax\fi
%    \end{macrocode}
%  \end{macro}
%
%    Some definitions for special characters.  We won’t define |\tilde|
%    as a Text Symbol not to conflict with the macro |\tilde| for math
%    mode and use the name |\tild| instead.  Note that |\boi| may
%    \emph{not} be used in math mode, its name in math mode is
%    |\backslash|.  |\degre| can be accessed by the command |\r{}|
%    for ring accent.
%
% \changes{v2.5f}{2011/07/18}{Changed definitions of \cs{at},
%    \cs{circonflexe}, \cs{tild}, \cs{boi} and \cs{degre} for
%    Unicode based engines.}
%
% \changes{v3.2g}{2017/01/24}{Changed Unicode definition of \cs{boi}.}
%
%    \begin{macrocode}
\ifFBunicode
  \newcommand*{\at}{{\char"0040}}
  \newcommand*{\circonflexe}{{\char"005E}}
  \newcommand*{\tild}{{\char"007E}}
  \newcommand*{\boi}{{\char"005C}}
  \newcommand*{\degre}{{\char"00B0}}
\else
  \ifLaTeXe
    \DeclareTextSymbol{\at}{T1}{64}
    \DeclareTextSymbol{\circonflexe}{T1}{94}
    \DeclareTextSymbol{\tild}{T1}{126}
    \DeclareTextSymbolDefault{\at}{T1}
    \DeclareTextSymbolDefault{\circonflexe}{T1}
    \DeclareTextSymbolDefault{\tild}{T1}
    \DeclareRobustCommand*{\boi}{\textbackslash}
    \DeclareRobustCommand*{\degre}{\r{}}
  \else
    \def\T@one{T1}
    \ifx\f@encoding\T@one
      \newcommand*{\degre}{{\char6}}
    \else
      \newcommand*{\degre}{{\char23}}
    \fi
    \newcommand*{\at}{{\char64}}
    \newcommand*{\circonflexe}{{\char94}}
    \newcommand*{\tild}{{\char126}}
    \newcommand*{\boi}{$\backslash$}
  \fi
\fi
%    \end{macrocode}
%
%  \begin{macro}{\degres}
%    We now define a macro |\degres| for typesetting the abbreviation
%    for `degrees’ (as in `degrees Celsius’).  As the bounding box of
%    the character `degree’ has \emph{very} different widths in CM/EC
%    and PostScript fonts, we fix the width of the bounding box of
%    |\degres| to 0.3\,em, this lets the symbol `degree’ stick to the
%    preceding (e.g., |45\degres|) or following character
%    (e.g., |20~\degres C|).
%
% \changes{v2.6e}{2013/07/06}{Refrain from redefining \cs{textdegree}
%    from latin1.def, applemac.def, etc. as \cs{degres} because it
%    loops in hyperref’s bookmarks.  Pointed out by Eddy Flas on fctt.}
%
%    If \TeX{} Companion fonts are available (\file{textcomp.sty}),
%    we pick up |\textdegree| from them instead of emulating `degrees’
%    from the |\r{}| accent.  Otherwise we advise the user (once only)
%    to use TS1-encoding.
%
% \changes{v2.1c}{2008/04/29}{Provide a temporary definition (hyperref
%    safe) of \cs{degres} in case it has to be expanded in the preamble
%    (by beamer’s \cs{title} command for instance).}
%
% \changes{v2.5h}{2012/03/21}{textcomp.sty has changed.  The test
%    about \cs{M@TS1} is no longer relevant, let’s change it.}
%
%    \begin{macrocode}
\ifLaTeXe
  \newcommand*{\degres}{\degre}
  \ifFBunicode
    \DeclareRobustCommand*{\degres}{\degre}
  \else
    \def\Warning@degree@TSone{\FBWarning
             {Degrees would look better in TS1-encoding:%
              \MessageBreak add \protect
              \usepackage{textcomp} to the preamble.%
              \MessageBreak Degrees used}}
    \AtBeginDocument{\ifx\DeclareEncodingSubset\@undefined
                       \DeclareRobustCommand*{\degres}{%
                          \leavevmode\hbox to 0.3em{\hss\degre\hss}%
                       \Warning@degree@TSone
                       \global\let\Warning@degree@TSone\relax}%
                     \else
                       \DeclareRobustCommand*{\degres}{%
                          \hbox{\UseTextSymbol{TS1}{\textdegree}}}%
                     \fi
                     }
  \fi
\else
  \newcommand*{\degres}{%
    \leavevmode\hbox to 0.3em{\hss\degre\hss}}
\fi
%    \end{macrocode}
%  \end{macro}
%
%  \subsection{Formatting numbers}
%  \label{ssec-numbers}
%
%  \begin{macro}{\StandardMathComma}
%  \begin{macro}{\DecimalMathComma}
%    As mentioned in the \TeX{}book p.~134, the comma is of type
%    |\mathpunct| in math mode: it is automatically followed by a
%    thin space.  This is convenient in lists and intervals but
%    unpleasant when the comma is used as a decimal separator
%    in French: it has to be entered as |{,}|.
%    |\DecimalMathComma| makes the comma be an ordinary character
%    (of type |\mathord|) in French \emph{only} (no space added);
%    |\StandardMathComma| switches back to the standard behaviour
%    of the comma.
%
% \changes{v3.2e}{2017/01/09}{\cs{DecimalMathComma} didn’t work
%    with LuaTeX.  Fixed now.}
%
% \changes{v3.2f}{2017/01/15}{Fixed conflict with the icomma
%    package.}
%
%    Unfortunately, |\newcount| inside |\if| breaks Plain formats.
%    \begin{macrocode}
\newif\ifFB@icomma
\newcount\mc@charclass
\newcount\mc@charfam
\newcount\mc@charslot
\newcount\std@mcc
\newcount\dec@mcc
\ifFBLuaTeX
  \mc@charclass=\Umathcharclass`\,
  \newcommand*{\dec@math@comma}{%
    \mc@charfam=\Umathcharfam`\,
    \mc@charslot=\Umathcharslot`\,
    \Umathcode`\,= 0 \mc@charfam \mc@charslot
  }
  \newcommand*{\std@math@comma}{%
    \mc@charfam=\Umathcharfam`\,
    \mc@charslot=\Umathcharslot`\,
    \Umathcode`\,= \mc@charclass \mc@charfam \mc@charslot
  }
\else
  \std@mcc=\mathcode`\,
  \dec@mcc=\std@mcc
  \@tempcnta=\std@mcc
  \divide\@tempcnta by "1000
  \multiply\@tempcnta by "1000
  \advance\dec@mcc by -\@tempcnta
  \newcommand*{\dec@math@comma}{\mathcode`\,=\dec@mcc}
  \newcommand*{\std@math@comma}{\mathcode`\,=\std@mcc}
\fi
\newcommand*{\DecimalMathComma}{%
  \iflanguage{french}{\dec@math@comma}{}%
  \ifFB@icomma\else\FB@addto{extras}{\dec@math@comma}\fi
}
\newcommand*{\StandardMathComma}{%
  \std@math@comma
  \ifFB@icomma\else\FB@addto{extras}{\std@math@comma}\fi
}
\ifLaTeXe
  \AtBeginDocument{\@ifpackageloaded{icomma}%
                      {\FB@icommatrue}%
                      {\FB@addto{noextras}{\std@math@comma}}%
  }
\else
  \FB@addto{noextras}{\std@math@comma}
\fi
%    \end{macrocode}
%  \end{macro}
%  \end{macro}
%
%  \begin{macro}{\nombre}
%
% \changes{v2.0}{2006/11/06}{\cs{nombre} now requires numprint.sty.}
%
%    The command |\nombre| is now borrowed from \file{numprint.sty} for
%    \LaTeXe.  There is no point to maintain the former tricky code
%    when a package is dedicated to do the same job and more.
%    For Plain based formats, |\nombre| no longer formats numbers,
%    it prints them as is and issues a warning about the change.
%
%    Fake command |\nombre| for Plain based formats, warning users of
%    \frenchb{} v.~1.x.\ about the change:
%    \begin{macrocode}
\newcommand*{\nombre}[1]{{#1}\fb@warning{*** \noexpand\nombre
                                no longer formats numbers\string! ***}}
%    \end{macrocode}
%  \end{macro}
%
%    The next definitions only make sense for \LaTeXe.
%    For Plain based formats, let’s activate LuaTeX punctuation if
%    necessary, then cleanup and exit. Temporary fix: |\l@french| is
%    not properly set by \babel~3.9h with Plain LuaTeX format.
%
%    \begin{macrocode}
\let\FBstop@here\relax
\def\FBclean@on@exit{\let\ifLaTeXe\undefined
                     \let\LaTeXetrue\undefined
                     \let\LaTeXefalse\undefined}
\ifx\magnification\@undefined
\else
  \def\FBstop@here{\ifFB@luatex@punct
                     \activate@luatexpunct
                   \fi
                   \FBclean@on@exit
                   \ldf@quit\CurrentOption\endinput}
\fi
\FBstop@here
%    \end{macrocode}
%
%    What follows is for \LaTeXe{} \emph{only}; as all \LaTeXe{}
%    based formats include \eTeX, we can use |\ifdefined| now.
%    We redefine |\nombre| for \LaTeXe.  A warning is issued
%    at the first call of |\nombre| if |\numprint| is not
%    defined, suggesting what to do.  The package \pkg{numprint}
%    is \emph{not} loaded automatically by \frenchb{} because of
%    possible options conflict.
%
% \changes{v3.1i}{2015/10/03}{\cs{nombre} command changed when
%    numprint.sty is not loaded: only one warning, no error.}
%
%    \begin{macrocode}
\renewcommand*{\nombre}[1]{\Warning@nombre{#1}}
\newcommand*{\Warning@nombre}[1]{%
   \ifdefined\numprint
     \numprint{#1}%
   \else
     \PackageWarning{frenchb.ldf}{%
        \protect\nombre\space now relies on package numprint.sty,%
        \MessageBreak add \protect
        \usepackage[autolanguage]{numprint},\MessageBreak
        see file numprint.pdf for more options.\MessageBreak
        \protect\nombre\space called}%
     \global\let\Warning@nombre\relax
     {#1}%
   \fi
}
%    \end{macrocode}
%
% \changes{v2.0c}{2007/06/25}{There is no need to define here
%    numprint’s command \cs{npstylefrench}, it will be redefined
%    `AtBeginDocument’ by \cs{FBprocess@options}.}
%
%  \subsection{Caption names}
%  \label{ssec-captions}
%
%    The next step consists in defining the French equivalents for
%    the \LaTeX{} caption names.
%
% \begin{macro}{\captionsfrench}
%    Let’s first define  |\captionsfrench| which sets all strings used
%    in the four standard document classes provided with \LaTeX.
%
% \changes{v2.0}{2006/11/06}{`Fig.’ changed to `Figure’ and
%    `Tab.’ to `Table’.}
%
% \changes{v2.5a}{2010/08/16}{\cs{emph} deleted in \cs{seename}
%    and \cs{alsoname} to match what is done for the other languages.
%    Suggested by Marc Baudoin.}
%
% \changes{v3.0a}{2014/02/18}{Take advantage of babel’s \cs{SetString}
%    commands for captionnames.}
%
% \changes{v3.1b}{2014/09/15}{Change \cs{scshape} to customisable
%    \cs{FBfigtabshape} for \cs{figurename} and \cs{tablename}.}
%
% \changes{v3.1g}{2015/05/10}{\cs{partname}’s definition depends now on
%    flag PartNameFull.  No need to redefine it in \cs{frenchbsetup}.}
%
% \changes{v3.3a}{2017/04/30}{Commands \cs{frenchpartfirst},
%    \cs{frenchpartsecond} and \cs{frenchpartnameord} added.}
%
%   Let’s give a chance to a class or a package read before frenchb to
%   define |\FBfigtabshape| as |\relax|, otherwise |\FBfigtabshape|
%   will be defined as |\scshape| (can be changed with
%    \fbsetup{SmallCapsFigTabCaptions=false}).
%    \begin{macrocode}
\ifx\FBfigtabshape\@undefined \let\FBfigtabshape\scshape \fi
%    \end{macrocode}
%
%   New implementation for caption names (requires \babel’s~3.9 or up).
%    \begin{macrocode}
\StartBabelCommands*{\BabelLanguages}{captions}
      [unicode, fontenc=EU1 EU2 TU, charset=utf8]
   \SetString{\refname}{Références}
   \SetString{\abstractname}{Résumé}
   \SetString{\prefacename}{Préface}
   \SetString{\contentsname}{Table des matières}
   \SetString{\ccname}{Copie à }
   \SetString{\proofname}{Démonstration}
   \SetString{\partfirst}{Première}
   \SetString{\partsecond}{Deuxième}
   \SetStringLoop{ordinal#1}{%
     \frenchpartfirst,\frenchpartsecond,Troisième,Quatrième,%
     Cinquième,Sixième,Septième,Huitième,Neuvième,Dixième,Onzième,%
     Douzième,Treizième,Quatorzième,Quinzième,Seizième,%
     Dix-septième,Dix-huitième,Dix-neuvième,Vingtième}
\StartBabelCommands*{\BabelLanguages}{captions}
   \SetString{\refname}{R\'ef\'erences}
   \SetString{\abstractname}{R\'esum\'e}
   \SetString{\bibname}{Bibliographie}
   \SetString{\prefacename}{Pr\'eface}
   \SetString{\chaptername}{Chapitre}
   \SetString{\appendixname}{Annexe}
   \SetString{\contentsname}{Table des mati\`eres}
   \SetString{\listfigurename}{Table des figures}
   \SetString{\listtablename}{Liste des tableaux}
   \SetString{\indexname}{Index}
   \SetString{\figurename}{{\FBfigtabshape Figure}}
   \SetString{\tablename}{{\FBfigtabshape Table}}
   \SetString{\pagename}{page}
   \SetString{\seename}{voir}
   \SetString{\alsoname}{voir aussi}
   \SetString{\enclname}{P.~J. }
   \SetString{\ccname}{Copie \`a }
   \SetString{\headtoname}{}
   \SetString{\proofname}{D\'emonstration}
   \SetString{\glossaryname}{Glossaire}
%    \end{macrocode}
%   When \fbo{PartNameFull=true} (default), |\part{}| is printed in
%   French as ``Première partie’’ instead of ``Partie I’’.
%   As logic is prohibited inside |\SetString|, let’s hide the test
%   about \fbo{PartNameFull} in |\FB@partname|.
%    \begin{macrocode}
   \SetString{\partfirst}{Premi\`ere}
   \SetString{\partsecond}{Deuxi\`eme}
   \SetString{\partnameord}{partie}
   \SetStringLoop{ordinal#1}{%
     \frenchpartfirst,\frenchpartsecond,Troisi\`eme,Quatri\`eme,%
     Cinqui\`eme,Sixi\`eme,Septi\`eme,Huiti\`eme,Neuvi\`eme,Dixi\`eme,%
     Onzi\`eme,Douzi\`eme,Treizi\`eme,Quatorzi\`eme,Quinzi\`eme,%
     Seizi\`eme,Dix-septi\`eme,Dix-huiti\`eme,Dix-neuvi\`eme,%
     Vingti\`eme}
   \AfterBabelCommands{%
     \DeclareRobustCommand*{\FB@emptypart}{\def\thepart{}}%
     \DeclareRobustCommand*{\FB@partname}{%
        \ifFBPartNameFull
          \csname ordinal\romannumeral\value{part}\endcsname\space
          \frenchpartnameord\FB@emptypart
        \else
          Partie%
        \fi}%
    }
   \SetString{\partname}{\FB@partname}
\EndBabelCommands
%    \end{macrocode}
%
% \changes{v3.1g}{2015/05/10}{Bug fix for koma-scripts classes:
%    a spurious dot was added by the \cs{partformat} command.}
%
%    The following patch is for koma-script classes: |\partformat|
%    needs to be redefined in French as this command, defined as
%    |\partname~\thepart\autodot| is incompatible with our redefinition
%    of |\partname|.  The code is postponed to the end of package
%    because |\ifFB@koma| will be defined and set later on
%    (see p.~\pageref{ifFB@koma}).
%    \begin{macrocode}
\AtEndOfPackage{%
   \ifFB@koma
     \ifdefined\partformat
       \FB@addto{captions}{%
           \ifFBPartNameFull
             \babel@save\partformat
             \renewcommand*{\partformat}{\partname}%
           \fi}%
     \fi
   \fi
}
%    \end{macrocode}
% \end{macro}
%
% \changes{v3.0a}{2014/02/28}{Merging of \cs{captionsfrenchb},
%    \cs{captionsfrancais} with \cs{captionsfrench} deleted in favor
%    of new babel 3.9 syntax.}
%
%    Up to v2.6h \frenchb{} used to merge |\captionsfrenchb| and
%    |\captionsfrancais| into |\captionsfrench| at |\begin{document}|.
%    This is deprecated in favor of the new (much simpler!) syntax
%    introduced in \babel~3.9.  No need to define |\captionscanadien|
%    and |\captionsacadian| either.
%
% \begin{macro}{\CaptionSeparator}
%    Let’s consider now captions in figures and tables.
%    In French, captions in figures and tables should never be printed
%    as `Figure 1:’ which is the default in standard \LaTeXe{} classes;
%    the `:’ is made active too late, no space is added before it.
%    With LuaLaTeX and XeLaTeX, this glitch doesn’t occur, you get
%    `Figure 1~:’ which is correct in French.  With pdfLaTeX \frenchb{}
%    provides the following workaround.
%
%    The standard definition of |\@makecaption| (e.g., the one provided
%    in article.cls, report.cls, book.cls which is frozen for \LaTeXe{}
%    according to Frank Mittelbach), is saved in |\STD@makecaption|.
%    `AtBeginDocument’ we compare it to its current definition (some
%    classes like \cls{memoir}, koma-script classes, AMS classes,
%    ua-thesis.cls\dots change it).
%    If they are identical, \frenchb{} just adds a hook called
%    |\FBCaption@Separator| to |\@makecaption|; |\FBCaption@Separator|
%    defaults to `: ’ as in the standard |\@makecaption| and will be
%    changed to ` : ’ in French `AtBeginDocument’; it can be also set to
%    |\CaptionSeparator| (` -- ’) using \fbo{CustomiseFigTabCaptions}.
%
% \changes{v2.4a}{2009/11/23}{\cs{PackageWarning} changed to
%    \cs{FBWarning} (in case \cs{@makecaption} has been customised).
%    \cs{FBWarning} is defined as \cs{PackageWarning} by default but
%    can be made silent using \cs{frenchbsetup}, (suggested by MPG).}
%
% \changes{v2.6c}{2013/05/18}{Former \cs{CaptionSeparator}
%    has been renamed as \cs{FBCaption@Separator};
%    Newif \cs{if@FBwarning@capsep} added.}
%
% \changes{v2.6h}{2014/01/03}{No active catcodes in
%    \cs{STD@makecaption}’s definition.}
%
% \changes{v3.0a}{2014/02/28}{Remove \cs{CaptionSeparatorORI}, use
%    \cs{babel@save} instead.}
%
%    While saving the standard definition of |\@makecaption| we have to
%    make sure that characters `:’ and `|>|’ have |\catcode| 12
%    (\frenchb{} makes `:’ active and \file{spanish.ldf} makes `|>|’
%    active).
%    \begin{macrocode}
\bgroup
  \catcode`:=12 \catcode`>=12 \relax
  \long\gdef\STD@makecaption#1#2{%
    \vskip\abovecaptionskip
    \sbox\@tempboxa{#1: #2}%
    \ifdim \wd\@tempboxa >\hsize
      #1: #2\par
    \else
      \global \@minipagefalse
      \hb@xt@\hsize{\hfil\box\@tempboxa\hfil}%
    \fi
    \vskip\belowcaptionskip}
\egroup
%    \end{macrocode}
%
%    No warning is issued for SMF and AMS classes as their layout of
%    captions is compatible with French typographic standards.
%
%    With \cls{memoir} and koma-script classes, \frenchb{} customises
%    |\captiondelim| or |\captionformat| in French (unless option
%    \fbo{CustomiseFigTabCaptions} is set to \fbo{false}) and issues
%    no warning.
%
%    When |\@makecaption| has been changed by another class
%    or package, a warning is printed in the .log file.
%
%    \begin{macrocode}
\newif\if@FBwarning@capsep
\@FBwarning@capseptrue
\newcommand{\FBWarning}[1]{\PackageWarning{frenchb.ldf}{#1}}
\newcommand*{\CaptionSeparator}{\space\textendash\space}
\def\FBCaption@Separator{: }
\long\def\FB@makecaption#1#2{%
  \vskip\abovecaptionskip
  \sbox\@tempboxa{#1\FBCaption@Separator #2}%
  \ifdim \wd\@tempboxa >\hsize
    #1\FBCaption@Separator #2\par
  \else
    \global \@minipagefalse
    \hb@xt@\hsize{\hfil\box\@tempboxa\hfil}%
  \fi
  \vskip\belowcaptionskip}
%    \end{macrocode}
% \end{macro}
%
% \changes{v2.6c}{2013/05/18}{No warning about \cs{@makecaption} for
%    AMS classes.}
%
% \changes{v3.0a}{2014/02/20}{No warning about \cs{@makecaption} for
%    SMF classes. No warning either with LuaTeX or XeTeX engines.}
%
%    Disable the standard warning with AMS and SMF classes.
%    \begin{macrocode}
\@ifclassloaded{amsart}{\@FBwarning@capsepfalse}{}
\@ifclassloaded{amsbook}{\@FBwarning@capsepfalse}{}
\@ifclassloaded{amsdtx}{\@FBwarning@capsepfalse}{}
\@ifclassloaded{amsldoc}{\@FBwarning@capsepfalse}{}
\@ifclassloaded{amproc}{\@FBwarning@capsepfalse}{}
\@ifclassloaded{smfart}{\@FBwarning@capsepfalse}{}
\@ifclassloaded{smfbook}{\@FBwarning@capsepfalse}{}
%    \end{macrocode}
%
%    Disable the standard warning unless high punctuation is active.
%    \begin{macrocode}
\ifFB@active@punct\else\@FBwarning@capsepfalse\fi
%    \end{macrocode}
%
% \changes{v2.6c}{2013/05/18}{No warning about \cs{@makecaption} for
%    koma-script classes. \cs{captionformat} customised in French.}
%
% \changes{v2.6h}{2014/01/05}{If \cs{@makecaption} is undefined, no
%    warning.}
%
%    No warning with \cls{memoir} or koma-script classes: they change
%    |\@makecaption| but we will manage to customise them in French
%    later on (see below after executing |\FBprocess@options|)
%    \label{ifFB@koma}.
%    \begin{macrocode}
\newif\ifFB@koma
\@ifclassloaded{memoir}{\@FBwarning@capsepfalse}{}
\@ifclassloaded{scrartcl}{\@FBwarning@capsepfalse\FB@komatrue}{}
\@ifclassloaded{scrbook}{\@FBwarning@capsepfalse\FB@komatrue}{}
\@ifclassloaded{scrreprt}{\@FBwarning@capsepfalse\FB@komatrue}{}
%    \end{macrocode}
%    No warning with the \cls{beamer} class which defines
%    |\beamer@makecaption| (customised below) instead of
%    |\@makecaption|.  No warning either if |\@makecaption| is
%    undefined (i.e.\ \cls{letter}).
%    \begin{macrocode}
\@ifclassloaded{beamer}{\@FBwarning@capsepfalse}{}
\ifdefined\@makecaption\else\@FBwarning@capsepfalse\fi
%    \end{macrocode}
%
% \changes{v2.6c}{2013/05/18}{Warning added when the caption
%    or floatrow package is loaded before babel/frenchb.}
%
%    The \pkg{caption}, \pkg{subcaption} and \pkg{floatrow} packages are
%    compatible with \frenchb{} if they are loaded after \babel{}.
%
%    Check if package \pkg{caption} is loaded now (before \frenchb),
%    then issue a warning advising to load it after \frenchb{} and
%    disable the standard warning.
%    \begin{macrocode}
\@ifpackageloaded{caption}
   {\FBWarning{Please load the "caption" package\MessageBreak
               AFTER babel/frenchb; reported}%
    \@FBwarning@capsepfalse}%
   {}
%    \end{macrocode}
%
% \changes{v3.2b}{2016/03/30}{Warning added when the subcaption
%    package is loaded before babel/frenchb.}
%
%    Same for package \pkg{subcaption}.
%    \begin{macrocode}
\@ifpackageloaded{subcaption}
   {\FBWarning{Please load the "subcaption" package\MessageBreak
               AFTER babel/frenchb; reported}%
    \@FBwarning@capsepfalse}%
   {}
%    \end{macrocode}
%
%    Same for package \pkg{floatrow}.
%    \begin{macrocode}
\@ifpackageloaded{floatrow}
   {\FBWarning{Please load the "floatrow" package\MessageBreak
               AFTER babel/frenchb; reported}%
    \@FBwarning@capsepfalse}%
   {}
%    \end{macrocode}
%
% \changes{v3.0a}{2014/02/20}{More informative, less TeXnical warning
%    about \cs{@makecaption}.}
%
%    First check the definition of |\@makecaption|, change it or issue
%    a warning in case it has been changed by a class or package not
%    (yet) compatible with \frenchb; then change the definition of
%    |\FBCaption@Separator|, taking care that the colon is typeset
%    correctly in French (\emph{not} `Figure~1: légende’).
%
% \changes{v3.1f}{2015/05/31}{\cs{FBCaption@Separator} changed when
%    option CustomiseFigTabCaptions is set to false.}
%
%    \begin{macrocode}
\AtBeginDocument{%
  \ifx\@makecaption\STD@makecaption
    \global\let\@makecaption\FB@makecaption
%    \end{macrocode}
%    Do not overwrite |\FBCaption@Separator| if already saved
%    as `: ’  for other languages and set to |\CaptionSeparator| by
%    |\extrasfrench| when French is the main language.
%    \begin{macrocode}
    \ifFBOldFigTabCaptions
    \else
      \def\FBCaption@Separator{{\autospace@beforeFDP : }}%
    \fi
    \ifFBCustomiseFigTabCaptions
      \ifx\bbl@main@language\FB@french
        \def\FBCaption@Separator{\CaptionSeparator}%
      \fi
    \fi
    \@FBwarning@capsepfalse
  \fi
  \if@FBwarning@capsep
    \FBWarning
       {Figures' and tables' captions might look like\MessageBreak
        `Figure 1:' which is wrong in French.\MessageBreak
        Check your class or packages to change this;\MessageBreak
        reported}%
  \fi
  \let\FB@makecaption\relax
  \let\STD@makecaption\relax
}
%    \end{macrocode}
%
%  \subsection{Dots\dots}
%  \label{ssec-dots}
%
%  \begin{macro}{\FBtextellipsis}
%    \LaTeXe’s standard definition of |\dots| in text-mode is
%    |\textellipsis| which includes a |\kern| at the end;
%    this space is not wanted in some cases (before a closing brace
%    for instance) and |\kern| breaks hyphenation of the next word.
%    We define |\FBtextellipsis| for French (in \LaTeXe{} only).
%
%    The |\if| construction in the \LaTeXe{} definition of |\dots|
%    doesn’t allow the use of |xspace| (|xspace| is always followed
%    by a |\fi|), so we use the AMS-\LaTeX{} construction of |\dots|;
%    this has to be done `AtBeginDocument’ not to be overwritten
%    when \file{amsmath.sty} is loaded after \babel.
%
% \changes{v2.0}{2006/11/06}{Added special case for LY1 encoding,
%    see  bug report from Bruno Voisin (2004/05/18).}
%
% \changes{v2.5f}{2011/06/18}{Unicode fonts also provide a ready made
%    character for \cs{textellipsis}, let’s just use it!
%    (reported by Maxime Chupin, 2011/06/04).}
%
%    LY1 has a ready made character for |\textellipsis|, it should be
%    used in French too.  The same is true for Unicode fonts in use
%    with XeTeX and LuaTeX.
%
%    \begin{macrocode}
\ifFBunicode
  \let\FBtextellipsis\textellipsis
\else
  \DeclareTextSymbol{\FBtextellipsis}{LY1}{133}
  \DeclareTextCommandDefault{\FBtextellipsis}{%
    .\kern\fontdimen3\font.\kern\fontdimen3\font.\xspace}
\fi
%    \end{macrocode}
%    |\Mdots@| and |\Tdots@| hold the definitions of |\dots| in
%    Math and Text mode.  They default to those of amsmath-2.0, and
%    will revert to standard \LaTeX{} definitions `AtBeginDocument’,
%    if amsmath has not been loaded.  |\Mdots@| doesn’t change when
%    switching from/to French, while |\Tdots@| is redefined as
%    |\FBtextellipsis| in French.
%    \begin{macrocode}
\newcommand*{\Tdots@}{\@xp\textellipsis}
\newcommand*{\Mdots@}{\@xp\mdots@}
\AtBeginDocument{\DeclareRobustCommand*{\dots}{\relax
                 \csname\ifmmode M\else T\fi dots@\endcsname}%
                 \ifdefined\@xp\else\let\@xp\relax\fi
                 \ifdefined\mdots@\else\let\Mdots@\mathellipsis\fi
                }
\def\bbl@frenchdots{\babel@save\Tdots@ \let\Tdots@\FBtextellipsis}
\FB@addto{extras}{\bbl@frenchdots}
%    \end{macrocode}
%  \end{macro}
%
%  \subsection{More checks about packages’ loading order}
%  \label{ssec-loading-order}
%
% \changes{v3.1d}{2015/04/06}{New section: issue warnings if packages
%    listings, numprint and natbib are loaded too early or too late
%    vs babel.}
%
% \changes{v3.1i}{2015/10/03}{Remove restriction about loading
%    numprint.sty after babel.}
%
%    Like packages \file{captions} and \file{floatrow} (see
%    section~\ref{ssec-captions}), package \file{listings} should
%    be loaded after \frenchb{} due to active characters issues
%    (pdfLaTeX only).
%
%    \begin{macrocode}
\ifFB@active@punct
   \@ifpackageloaded{listings}
      {\FBWarning{Please load the "listings" package\MessageBreak
                  AFTER babel/frenchb; reported}%
      }{}
\fi
%    \end{macrocode}
%
%    Package \file{natbib} should be loaded before \frenchb{}
%    due to active characters issues (pdfLaTeX only).
%
%    \begin{macrocode}
\newif\if@FBwarning@natbib
\ifFB@active@punct
  \@ifpackageloaded{natbib}{}{\@FBwarning@natbibtrue}
\fi
\AtBeginDocument{%
   \if@FBwarning@natbib
     \@ifpackageloaded{natbib}{}{\@FBwarning@natbibfalse}%
   \fi
   \if@FBwarning@natbib
     \FBWarning{Please load the "natbib" package\MessageBreak
                BEFORE babel/frenchb; reported}%
   \fi
}
%    \end{macrocode}
%
% \changes{v3.2g}{2017/01/30}{Issue a warning if beamerarticle.sty
%    is loaded after babel.}
%
%    Package \pkg{beamerarticle} should be loaded before \frenchb{}
%    to avoid list’s conflicts, see p.~\pageref{beamerarticle}.
%
%    \begin{macrocode}
\newif\if@FBwarning@beamerarticle
\@ifpackageloaded{beamerarticle}{}{\@FBwarning@beamerarticletrue}
\AtBeginDocument{%
   \if@FBwarning@beamerarticle
     \@ifpackageloaded{beamerarticle}{}%
                                     {\@FBwarning@beamerarticlefalse}%
   \fi
   \if@FBwarning@beamerarticle
     \FBWarning{Please load the "beamerarticle" package\MessageBreak
                BEFORE babel/frenchb; reported}%
   \fi
}
%    \end{macrocode}
%
%  \subsection{Setup options: keyval stuff}
%  \label{ssec-keyval}
%
% \changes{v2.0}{2006/11/06}{New command \cs{frenchbsetup} added
%     for global customisation.}
%
% \changes{v3.0a}{2014/02/24}{Options processing completely
%    reorganised, now \cs{babel@save} and\cs{babel@savevariable} are
%    usable for French.}
%
%    All setup options are handled by command \fbsetup{} using the
%    keyval syntax.  A list of flags is defined and set to a default
%    value which will possibly be changed `AtEndOfPackage’ if French is
%    the main language.  After this, \fbsetup{} eventually modifies the
%    preset values of these flags.
%
%    Option processing can occur either in \fbsetup{}, but \emph{only
%    for options explicitly set} by \fbsetup{}, or `AtBeginDocument’;
%    any option affecting |\extrasfrench{}| \emph{must} be processed by
%    \fbsetup{}: when French is the main language, |\extrasfrench{}| is
%    executed by \babel{} when it switches the main language and this
%    occurs \emph{before} reading the stuff postponed by \frenchb{}
%    `AtBeginDocument’.  Reexecuting |\extrasfrench{}| is an option
%    which was used up to v2.6h, it has been dropped in v3.0a because
%    of its side-effects (f.i.\ |\babel@save| and |\babel@savevariable|
%    did not work for French).
%
%  \begin{macro}{\frenchsetup}
%    Let’s now define this command which reads and sets the options
%    to be processed either immediately (i.e.\ just after setting the
%    key) or later (at |\begin{document}|) by |\FBprocess@options|.
%    \fbsetup{} can only be called in the preamble.
%
% \changes{v2.0c}{2007/06/25}{Option ThinSpaceInFrenchNumbers added.}
%
% \changes{v2.0d}{2007/07/15}{Options og and fg changed: limit
%    the definition to French so that quote characters can be used
%    in German.}
%
% \changes{v2.0e}{2007/10/05}{New option: StandardLists.}
%
% \changes{v2.0f}{2008/03/23}{Two typos corrected in
%    option StandardLists: [false] $\to$ [true] and
%    StandardLayout $\to$ StandardLists.}
%
% \changes{v2.0f}{2008/03/23}{StandardLayout option had no
%    effect on lists.  Test moved to \cs{FBprocess@options}.}
%
% \changes{v2.0g}{2008/03/23}{Revert previous change to
%    StandardLayout.  This option must set the three flags
%    \cs{FBReduceListSpacingfalse}, \cs{FBCompactItemizefalse},
%    and \cs{FBStandardItemLabeltrue} instead of
%    \cs{FBStandardListstrue}, so that later options can still
%    change their value before executing \cs{FBprocess@options}.
%    Same thing for option StandardLists.}
%
% \changes{v2.1a}{2008/03/24}{New option: FrenchSuperscripts
%    to define \cs{up} as \cs{fup} or as \cs{textsuperscript}.}
%
% \changes{v2.1a}{2008/03/30}{New option: LowercaseSuperscripts.}
%
% \changes{v2.2a}{2008/05/08}{The global layout of the document is
%    no longer changed when frenchb is not the last option of babel
%    (\cs{bbl@main@language}).  Suggested by Ulrike Fischer.}
%
% \changes{v2.2a}{2008/05/08}{Default values of flags changed:
%    default now means `StandardLayout’, they will be changed to
%    `FrenchLayout’ AtEndOfPackage only if french is
%    \cs{bbl@main@language}.}
%
% \changes{v2.2a}{2008/05/08}{When frenchb is babel’s last option,
%    French becomes the document’s main language, so
%    GlobalLayoutFrench applies.}
%
% \changes{v2.4a}{2009/11/23}{New option SuppressWarning.}
%
% \changes{v2.6a}{2013/01/30}{New options ListOldLayout,
%    StandardItemizeEnv and StandardEnumerateEnv (CompactItemize is
%    deprecated).}
%
% \changes{v3.0a}{2014/02/28}{New options OldFigTabCaptions and
%    CustomiseFigTabCaptions.}
%
% \changes{v3.0c}{2014/04/18}{New option INGuillSpace.}
%
% \changes{v3.1a}{2014/05/20}{New options InnerGuillSingle,
%    EveryParGuill and EveryLineGuill to control \cs{frquote}.}
%
% \changes{v3.1b}{2014/09/15}{New option SmallCapsFigTabCaptions.}
%
% \changes{v3.1e}{2015/04/10}{Corrected typo: SmallCapsFigTabcaptions
%    instead of SmallCapsFigTabCaptions.  Pointed out by Céline
%    Chevalier.}
%
% \changes{v3.1g}{2015/05/10}{PartNameFull now just sets the flag,
%    nothing to add to \cs{captionsfrench} when false.}
%
% \changes{v3.3a}{2017/04/30}{\cs{frenchbsetup} is now an alias for
%    \cs{frenchsetup}.}
%
% \changes{v3.3a}{2017/04/30}{Options INGuillSpace, ThinColonSPace no
%    longer delayed AtBeginDocument.}
%
%    \begin{macrocode}
\newcommand*{\frenchsetup}[1]{%
  \setkeys{FB}{#1}%
}%
\@onlypreamble\frenchsetup
%    \end{macrocode}
%    Keep the former name |\frenchbsetup| working for compatibility.
%    \begin{macrocode}
\let\frenchbsetup\frenchsetup
\@onlypreamble\frenchbsetup
%    \end{macrocode}
%
%    We define a collection of conditionals with their defaults
%    (true or false).
%
%    \begin{macrocode}
\newif\ifFBShowOptions              \FBShowOptionsfalse
\newif\ifFBStandardLayout           \FBStandardLayouttrue
\newif\ifFBGlobalLayoutFrench       \FBGlobalLayoutFrenchtrue
\newif\ifFBReduceListSpacing        \FBReduceListSpacingfalse
\newif\ifFBListOldLayout            \FBListOldLayoutfalse
\newif\ifFBCompactItemize           \FBCompactItemizefalse
\newif\ifFBStandardItemizeEnv       \FBStandardItemizeEnvtrue
\newif\ifFBStandardEnumerateEnv     \FBStandardEnumerateEnvtrue
\newif\ifFBStandardItemLabels       \FBStandardItemLabelstrue
\newif\ifFBStandardLists            \FBStandardListstrue
\newif\ifFBIndentFirst              \FBIndentFirstfalse
\newif\ifFBFrenchFootnotes          \FBFrenchFootnotesfalse
\newif\ifFBAutoSpaceFootnotes       \FBAutoSpaceFootnotesfalse
\newif\ifFBOriginalTypewriter       \FBOriginalTypewriterfalse
\newif\ifFBThinColonSpace           \FBThinColonSpacefalse
\newif\ifFBThinSpaceInFrenchNumbers \FBThinSpaceInFrenchNumbersfalse
\newif\ifFBFrenchSuperscripts       \FBFrenchSuperscriptstrue
\newif\ifFBLowercaseSuperscripts    \FBLowercaseSuperscriptstrue
\newif\ifFBPartNameFull             \FBPartNameFulltrue
\newif\ifFBCustomiseFigTabCaptions  \FBCustomiseFigTabCaptionsfalse
\newif\ifFBOldFigTabCaptions        \FBOldFigTabCaptionsfalse
\newif\ifFBSmallCapsFigTabCaptions  \FBSmallCapsFigTabCaptionstrue
\newif\ifFBSuppressWarning          \FBSuppressWarningfalse
\newif\ifFBINGuillSpace             \FBINGuillSpacefalse
%    \end{macrocode}
%
%    The defaults values of these flags have been choosen so that
%    \frenchb{} does not change anything regarding the global layout.
%    |\bbl@main@language|, set by the last option of \babel, controls
%    the global layout of the document.  `AtEndOfPackage’ we check the
%    main language in |\bbl@main@language|; if it is French,
%    the values of some flags have to be changed to ensure a French
%    looking layout for the whole document (even in parts written in
%    languages other than French); the end-user will then be able to
%    customise the values of all these flags with \fbsetup{}.
%
% \changes{v2.5b}{2010/10/30}{Do not use the
%    test \cs{iflanguage}\{french\} to check whether French is the
%    main language or not, as it might be be erroneously positive
%    when English is the main language and no hyphenation patterns
%    are available for French.
%    In this case \cs{l@french} and \cs{l@english} are 0.
%    Pointed out by Günter Milde.}
%
% \changes{v3.0c}{2014/03/30}{No list customisation when beamer
%    class is loaded.}
%
% \changes{v3.2g}{2017/01/30}{Minimal list customisation when
%    beamerarticle.sty is loaded.}
%
% \changes{v3.2h}{2017/04/11}{Option GlobalLayoutFrench skipped
%    when French is not the main language.}
%
%    Our list customisation conflicts with the \cls{beamer} class and
%    with the \pkg{beamerarticle} package.  The patch provided in
%    \pkg{beamerbasecompatibility} solves the conflict except in case
%    of language changes, so we provide our own patch.
%    When the \cls{beamer} is loaded, lists are not customised at all
%    to ensure compatibility.\label{beamerarticle}
%    The \pkg{beamerarticle} package needs to be loaded \emph{before}
%    \babel, a warning is issued otherwise, see
%    section~\ref{ssec-loading-order}; a light customisation is
%    compatible with the \pkg{beamerarticle} package.
%
%    \begin{macrocode}
\edef\FB@french{\CurrentOption}
\AtEndOfPackage{%
  \ifx\bbl@main@language\FB@french
    \FBGlobalLayoutFrenchtrue
    \@ifclassloaded{beamer}%
      {\PackageInfo{frenchb.ldf}{%
          No list customisation for the beamer class,%
          \MessageBreak reported}}%
      {\@ifpackageloaded{beamerarticle}%
         {\FBStandardItemLabelsfalse
          \FBReduceListSpacingtrue
          \PackageInfo{frenchb.ldf}{%
             Minimal list customisation for the beamerarticle%
             \MessageBreak package; reported}}%
%    \end{macrocode}
%    Otherwise customise lists ``à la française’’:
%    \begin{macrocode}
         {\FBReduceListSpacingtrue
          \FBStandardItemizeEnvfalse
          \FBStandardEnumerateEnvfalse
          \FBStandardItemLabelsfalse}%
      }
    \FBIndentFirsttrue
    \FBFrenchFootnotestrue
    \FBAutoSpaceFootnotestrue
    \FBCustomiseFigTabCaptionstrue
  \else
    \FBGlobalLayoutFrenchfalse
  \fi
%    \end{macrocode}
%    \frenchb{} being an option of \babel, it cannot load a package
%    (\pkg{keyval}) while \file{frenchb.ldf} is read, so we defer the
%    loading of \file{keyval} and the options setup at the end of
%    \babel’s loading.
%    \begin{macrocode}
  \RequirePackage{keyval}%
  \define@key{FB}{ShowOptions}[true]%
          {\csname FBShowOptions#1\endcsname}%
  \define@key{FB}{StandardLayout}[true]%
          {\csname FBStandardLayout#1\endcsname
           \ifFBStandardLayout
             \FBReduceListSpacingfalse
             \FBStandardItemizeEnvtrue
             \FBStandardItemLabelstrue
             \FBStandardEnumerateEnvtrue
             \FBIndentFirstfalse
             \FBFrenchFootnotesfalse
             \FBAutoSpaceFootnotesfalse
             \FBGlobalLayoutFrenchfalse
           \else
             \FBReduceListSpacingtrue
             \FBStandardItemizeEnvfalse
             \FBStandardItemLabelsfalse
             \FBStandardEnumerateEnvfalse
             \FBIndentFirsttrue
             \FBFrenchFootnotestrue
             \FBAutoSpaceFootnotestrue
           \fi}%
  \define@key{FB}{GlobalLayoutFrench}[true]%
          {\csname FBGlobalLayoutFrench#1\endcsname
%    \end{macrocode}
%    If this key is set to \fbo{true} when French is the main language,
%    nothing to do: all flags keep their default value.
%    If this key is set to \fbo{false}, nothing to do either:
%    |\babel@save| will do the job.  Warn and reset in case this key
%    is set to true while the main language is \emph{not} French.
%    \begin{macrocode}
           \ifFBGlobalLayoutFrench
             \ifx\bbl@main@language\FB@french
             \else
               \FBGlobalLayoutFrenchfalse
               \PackageWarning{frenchb.ldf}%
                  {Option `GlobalLayoutFrench' skipped:\MessageBreak
                   French is *not* babel's last option.\MessageBreak
                   Reported}%
             \fi
           \fi}%
  \define@key{FB}{ReduceListSpacing}[true]%
          {\csname FBReduceListSpacing#1\endcsname}%
  \define@key{FB}{ListOldLayout}[true]%
          {\csname FBListOldLayout#1\endcsname
           \ifFBListOldLayout
             \FBStandardEnumerateEnvtrue
             \renewcommand*{\FrenchLabelItem}{\textendash}%
           \fi}%
  \define@key{FB}{CompactItemize}[true]%
          {\csname FBCompactItemize#1\endcsname
           \ifFBCompactItemize
             \FBStandardItemizeEnvfalse
             \FBStandardEnumerateEnvfalse
           \else
             \FBStandardItemizeEnvtrue
             \FBStandardEnumerateEnvtrue
           \fi}%
  \define@key{FB}{StandardItemizeEnv}[true]%
          {\csname FBStandardItemizeEnv#1\endcsname}%
  \define@key{FB}{StandardEnumerateEnv}[true]%
          {\csname FBStandardEnumerateEnv#1\endcsname}%
  \define@key{FB}{StandardItemLabels}[true]%
          {\csname FBStandardItemLabels#1\endcsname}%
  \define@key{FB}{ItemLabels}%
          {\renewcommand*{\FrenchLabelItem}{#1}}%
  \define@key{FB}{ItemLabeli}%
          {\renewcommand*{\Frlabelitemi}{#1}}%
  \define@key{FB}{ItemLabelii}%
          {\renewcommand*{\Frlabelitemii}{#1}}%
  \define@key{FB}{ItemLabeliii}%
          {\renewcommand*{\Frlabelitemiii}{#1}}%
  \define@key{FB}{ItemLabeliv}%
          {\renewcommand*{\Frlabelitemiv}{#1}}%
  \define@key{FB}{StandardLists}[true]%
          {\csname FBStandardLists#1\endcsname
           \ifFBStandardLists
             \FBReduceListSpacingfalse
             \FBCompactItemizefalse
             \FBStandardItemizeEnvtrue
             \FBStandardEnumerateEnvtrue
             \FBStandardItemLabelstrue
           \else
             \FBReduceListSpacingtrue
             \FBCompactItemizetrue
             \FBStandardItemizeEnvfalse
             \FBStandardEnumerateEnvfalse
             \FBStandardItemLabelsfalse
           \fi}%
  \define@key{FB}{IndentFirst}[true]%
          {\csname FBIndentFirst#1\endcsname}%
  \define@key{FB}{FrenchFootnotes}[true]%
          {\csname FBFrenchFootnotes#1\endcsname}%
  \define@key{FB}{AutoSpaceFootnotes}[true]%
          {\csname FBAutoSpaceFootnotes#1\endcsname}%
  \define@key{FB}{AutoSpacePunctuation}[true]%
          {\csname FBAutoSpacePunctuation#1\endcsname}%
  \define@key{FB}{OriginalTypewriter}[true]%
          {\csname FBOriginalTypewriter#1\endcsname}%
  \define@key{FB}{ThinColonSpace}[true]%
          {\csname FBThinColonSpace#1\endcsname
           \ifFBThinColonSpace
             \renewcommand*{\FBcolonspace}{\FBthinspace}%
          \fi}%
  \define@key{FB}{ThinSpaceInFrenchNumbers}[true]%
          {\csname FBThinSpaceInFrenchNumbers#1\endcsname}%
  \define@key{FB}{FrenchSuperscripts}[true]%
          {\csname FBFrenchSuperscripts#1\endcsname}
  \define@key{FB}{LowercaseSuperscripts}[true]%
          {\csname FBLowercaseSuperscripts#1\endcsname}
  \define@key{FB}{PartNameFull}[true]%
          {\csname FBPartNameFull#1\endcsname}%
  \define@key{FB}{CustomiseFigTabCaptions}[true]%
          {\csname FBCustomiseFigTabCaptions#1\endcsname}%
  \define@key{FB}{OldFigTabCaptions}[true]%
          {\csname FBOldFigTabCaptions#1\endcsname
%    \end{macrocode}
%    |\CurrentOption| no longer defined.  It’s value has been saved
%    in |\FB@CurOpt| while reading \file{frenchb.ldf}.
%    \begin{macrocode}
           \ifFBOldFigTabCaptions
             \FB@addto{extras}{\babel@save\FBCaption@Separator
                      \def\FBCaption@Separator{\CaptionSeparator}}%
           \fi}%
  \define@key{FB}{SmallCapsFigTabCaptions}[true]%
          {\csname FBSmallCapsFigTabCaptions#1\endcsname
           \ifFBSmallCapsFigTabCaptions
             \let\FBfigtabshape\scshape
           \else
             \let\FBfigtabshape\relax
           \fi}%
  \define@key{FB}{SuppressWarning}[true]%
          {\csname FBSuppressWarning#1\endcsname
           \ifFBSuppressWarning
             \renewcommand{\FBWarning}[1]{}%
           \fi}%
%    \end{macrocode}
%    Here are the options controlling French guillemets spacing
%    and the output of |\frquote{}|.
%
% \changes{v3.2g}{2017/01/30}{Warn when wrong values are provided to
%    options EveryParGuill or EveryLineGuill.}
%
%    \begin{macrocode}
  \define@key{FB}{INGuillSpace}[true]%
          {\csname FBINGuillSpace#1\endcsname
           \ifFBINGuillSpace
             \renewcommand*{\FBguillspace}{\space}%
           \fi}%
  \define@key{FB}{InnerGuillSingle}[true]%
          {\csname FBInnerGuillSingle#1\endcsname}%
  \define@key{FB}{EveryParGuill}[open]%
          {\expandafter\let\expandafter
             \FBeveryparguill\csname FBguill#1\endcsname
           \ifx\FBeveryparguill\FBguillopen
           \else\ifx\FBeveryparguill\FBguillclose
                \else\ifx\FBeveryparguill\FBguillnone
                     \else
                       \let\FBeveryparguill\FBguillopen
                       \PackageWarning{frenchb.ldf}%
                          {Wrong value for `EveryParGuill':
                           try `open',\MessageBreak
                           `close' or `none'. Reported}%
                     \fi
                \fi
           \fi}%
  \define@key{FB}{EveryLineGuill}[open]%
          {\ifFB@luatex@punct
             \expandafter\let\expandafter
               \FBeverylineguill\csname FBguill#1\endcsname
             \ifx\FBeverylineguill\FBguillopen
             \else\ifx\FBeverylineguill\FBguillclose
                  \else\ifx\FBeverylineguill\FBguillnone
                       \else
                         \let\FBeverylineguill\FBguillnone
                         \FBWarning{Wrong value for `EveryLineGuill':
                                    try `open',\MessageBreak
                                    `close' or `none'. Reported}%
                       \fi
                  \fi
             \fi
           \else
             \FBWarning{Option `EveryLineGuill' skipped:%
                        \MessageBreak this option is for
                        LuaTeX *only*.\MessageBreak Reported}%
           \fi}%
%    \end{macrocode}
%    Inputing French quotes as \emph{single characters} when they are
%    available on the keyboard (through a compose key for instance)
%    is more comfortable than typing |\og| and |\fg|.
%
%    With pdfTeX (or old LuaTeX and XeTeX engines), quote characters
%    are made active and expand to |\og\ignorespaces| and |{\fg}|
%    respectively if the current language is French, and to
%    |\guillemotleft| and |\guillemotright| otherwise (think of German
%    quotes), this is done by |\FB@@og| and |\FB@@fg|; thus correct
%    unbreakable spaces will be added automatically to French quotes.
%    The quote characters typed in depend on the input encoding,
%    it can be single-byte (latin1, latin9, applemac,\dots) or
%    multi-bytes (utf-8, utf8x); the \pkg{inputenc} package has to
%    be loaded before the |\begin{document}| with the proper coding
%    option, so we check if |\DeclareInputText| is defined.
%
%    Life is much simpler here with modern LuaTeX or XeTeX engines: we
%    just have to activate the |\FB@addGUILspace| attribute for LuaTeX
%    or set |\XeTeXcharclass| of quotes to the proper value for XeTeX.
%
% \changes{v2.4c}{2010/05/23}{In \cs{ttfamilyFB}, also cancel
%    automatic spaces inside French guillemets coded as characters
%    (see \cs{frenchbsetup}).}
%
% \changes{v2.5g}{2011/12/31}{When \cs{ifFB@xetex@punct} is true,
%   `og’ and `fg’ options now set XeTeXcharclasses of these characters
%   to \cs{FB@guilo} and \cs{FB@guilf}.  Otherwise French
%   quotes behave as normal characters (their XeTeXcharclass is 0).}
%
% \changes{v2.6h}{2014/01/03}{\cs{FG@og}  and \cs{FG@fg} changed:
%    former clumsy code removed.}
%
% \changes{v3.1a}{2014/06/16}{Codes 0x13 and 0x14 added for French
%    quotes in T1-encoding.  Support for older versions of LuaTeX and
%    XeTeX dropped.}
%
% \changes{v3.2c}{2016/05/03}{Add a warning about options og/fg for old
%    XeTeX or LuaTeX engines requiring active characters.}
%
%    \begin{macrocode}
  \define@key{FB}{og}%
          {\ifFBunicode
%    \end{macrocode}
%    LuaTeX or XeTeX in use, first try modern LuaTeX: we just need
%    to set LuaTeX’s attribute |\FB@addGUILspace| to 1,
%    \begin{macrocode}
             \ifFB@luatex@punct
               \FB@addGUILspace=1 \relax
             \fi
%    \end{macrocode}
%    then with XeTeX it is a bit more tricky:
%    \begin{macrocode}
             \ifFB@xetex@punct
%    \end{macrocode}
%    |\XeTeXinterchartokenstate| is defined, we just need to set
%    |\XeTeXcharclass| to |\FB@guilo| for the French opening quote
%    in T1 and Unicode encoding (see subsection~\ref{ssec-punct}).
%    \begin{macrocode}
               \XeTeXcharclass"13   = \FB@guilo
               \XeTeXcharclass"AB   = \FB@guilo
               \XeTeXcharclass"A0   = \FB@guilnul
               \XeTeXcharclass"202F = \FB@guilnul
             \fi
%    \end{macrocode}
%    Issue a warning with older Unicode engines requiring active
%    characters.
%    \begin{macrocode}
             \ifFB@active@punct
               \PackageWarning{frenchb.ldf}%
                  {Option og=« not supported with this version
                   of\MessageBreak LuaTeX/XeTeX; reported}%
             \fi
           \else
%    \end{macrocode}
%    This is for conventional TeX engines:
%    \begin{macrocode}
             \newcommand*{\FB@@og}{%
                \iflanguage{french}%
                  {\ifFB@spacing\FB@og\ignorespaces
                   \else\guillemotleft
                   \fi}%
                  {\guillemotleft}}%
             \AtBeginDocument{%
                \ifdefined\DeclareInputText
                  \ifdefined\uc@dclc
%    \end{macrocode}
%    Package \pkg{inputenc} with utf8x encoding loaded, use |\uc@dclc|,
%    \begin{macrocode}
                    \uc@dclc{171}{default}{\FB@@og}%
                  \else
%    \end{macrocode}
%    if encoding is not utf8x, try utf8\dots
%    \begin{macrocode}
                    \ifdefined\DeclareUnicodeCharacter
%    \end{macrocode}
%    utf8 loaded, use |\DeclareUnicodeCharacter|,
%    \begin{macrocode}
                      \DeclareUnicodeCharacter{00AB}{\FB@@og}%
                    \else
%    \end{macrocode}
%    if utf8 is not loaded either, we assume 8-bit character input
%    encoding.  Package \pkg{MULEenc} (from CJK) defines |\mule@def|
%    to map characters to control sequences.
%    \begin{macrocode}
                      \@tempcnta`#1\relax
                      \ifdefined\mule@def
                        \mule@def{11}{\FB@@og}%
                      \else
                        \DeclareInputText{\the\@tempcnta}{\FB@@og}%
                      \fi
                    \fi
                  \fi
                \else
%    \end{macrocode}
%    Package \pkg{inputenc} not loaded, no way\dots{}
%    \begin{macrocode}
                  \PackageWarning{frenchb.ldf}%
                     {Option `og' requires package inputenc;%
                      \MessageBreak reported}%
                \fi
              }%
           \fi
          }%
%    \end{macrocode}
%
%    Same code for the closing quote.
%    \begin{macrocode}
  \define@key{FB}{fg}%
          {\ifFBunicode
             \ifFB@luatex@punct
               \FB@addGUILspace=1 \relax
             \fi
             \ifFB@xetex@punct
               \XeTeXcharclass"14   = \FB@guilf
               \XeTeXcharclass"BB   = \FB@guilf
               \XeTeXcharclass"A0   = \FB@guilnul
               \XeTeXcharclass"202F = \FB@guilnul
             \fi
             \ifFB@active@punct
               \PackageWarning{frenchb.ldf}%
                  {Option fg=» not supported with this version
                   of\MessageBreak LuaTeX/XeTeX; reported}%
             \fi
           \else
             \newcommand*{\FB@@fg}{%
                \iflanguage{french}%
                  {\ifFB@spacing\FB@fg
                   \else\guillemotright
                   \fi}%
                  {\guillemotright}}%
             \AtBeginDocument{%
               \ifdefined\DeclareInputText
                 \ifdefined\uc@dclc
                   \uc@dclc{187}{default}{\FB@@fg}%
                 \else
                   \ifdefined\DeclareUnicodeCharacter
                     \DeclareUnicodeCharacter{00BB}{\FB@@fg}%
                   \else
                     \@tempcnta`#1\relax
                     \ifdefined\mule@def
                       \mule@def{27}{{\FB@@fg}}%
                     \else
                       \DeclareInputText{\the\@tempcnta}{\FB@@fg}%
                     \fi
                   \fi
                 \fi
               \else
                 \PackageWarning{frenchb.ldf}%
                    {Option `fg' requires package inputenc;%
                     \MessageBreak reported}%
               \fi
              }%
           \fi
          }%
}
%    \end{macrocode}
%  \end{macro}
%
% \begin{macro}{\FBprocess@options}
%    |\FBprocess@options| will be executed at |\begin{document}|:
%    it first checks about packages loaded in the preamble (possibly
%    after \babel) which customise lists: currently \pkg{enumitem},
%    \pkg{paralist} and \pkg{enumerate}; then it processes the options
%    as set by \fbsetup{} or forced for compatibility with packages
%    loaded in the preamble.
%
%    When French is the main language, |\extrasfrench| and
%    |\captionsfrench| \emph{have already been processed} by \babel{}
%    at |\begin{document}| \emph{before} |\FBprocess@options|.
%    \begin{macrocode}
\newcommand*{\FBprocess@options}{%
%    \end{macrocode}
%    Update flags if a package customising lists has been loaded,
%    currently: \pkg{enumitem}, \pkg{paralist}, \pkg{enumerate}.
%    \begin{macrocode}
  \@ifpackageloaded{enumitem}{%
     \ifFBStandardItemizeEnv
     \else
       \FBStandardItemizeEnvtrue
       \PackageInfo{frenchb.ldf}%
          {Setting StandardItemizeEnv=true for\MessageBreak
           compatibility with enumitem package,\MessageBreak
           reported}%
     \fi
     \ifFBStandardEnumerateEnv
     \else
       \FBStandardEnumerateEnvtrue
       \PackageInfo{frenchb.ldf}%
          {Setting StandardEnumerateEnv=true for\MessageBreak
           compatibility with enumitem package,\MessageBreak
           reported}%
     \fi}{}%
  \@ifpackageloaded{paralist}{%
     \ifFBStandardItemizeEnv
     \else
       \FBStandardItemizeEnvtrue
       \PackageInfo{frenchb.ldf}%
          {Setting StandardItemizeEnv=true for\MessageBreak
           compatibility with paralist package,\MessageBreak
           reported}%
     \fi
     \ifFBStandardEnumerateEnv
     \else
       \FBStandardEnumerateEnvtrue
       \PackageInfo{frenchb.ldf}%
          {Setting StandardEnumerateEnv=true for\MessageBreak
           compatibility with paralist package,\MessageBreak
           reported}%
     \fi}{}%
  \@ifpackageloaded{enumerate}{%
     \ifFBStandardEnumerateEnv
     \else
       \FBStandardEnumerateEnvtrue
       \PackageInfo{frenchb.ldf}%
          {Setting StandardEnumerateEnv=true for\MessageBreak
           compatibility with enumerate package,\MessageBreak
           reported}%
     \fi}{}%
%    \end{macrocode}
%    Reset |\FB@ufl|’s normal meaning and update lists’ settings
%    now in case French is the main language: \label{FB@ufl}
%    \begin{macrocode}
  \def\FB@ufl{\update@frenchlists}
  \ifx\bbl@main@language\FB@french
    \update@frenchlists
  \fi
%    \end{macrocode}
%    The layout of footnotes is handled at the |\begin{document}|
%    depending on the values of flags \fbo{FrenchFootnotes}
%    and \fbo{AutoSpaceFootnotes} (see section~\ref{ssec-footnotes}),
%    nothing has to be done here for footnotes.
%
%    \fbo{AutoSpacePunctuation} adds an unbreakable space (in French
%    only) before the four active characters (:;!?) even if none has
%    been typed before them.
%    \begin{macrocode}
  \ifFBAutoSpacePunctuation
    \autospace@beforeFDP
  \else
    \noautospace@beforeFDP
  \fi
%    \end{macrocode}
%    When \fbo{OriginalTypewriter} is set to \fbo{false} (the default),
%    |\ttfamily|, |\rmfamily| and |\sffamily| are redefined as
%    |\ttfamilyFB|, |\rmfamilyFB| and |\sffamilyFB| respectively
%    to prevent addition of automatic spaces before the four active
%    characters in computer code.
%    \begin{macrocode}
  \ifFBOriginalTypewriter
  \else
    \let\ttfamilyORI\ttfamily
    \let\rmfamilyORI\rmfamily
    \let\sffamilyORI\sffamily
    \let\ttfamily\ttfamilyFB
    \let\rmfamily\rmfamilyFB
    \let\sffamily\sffamilyFB
  \fi
%    \end{macrocode}
%
%    When package \pkg{numprint} is loaded with option
%    \opt{autolanguage}, \pkg{numprint}’s command |\npstylefrench| has
%    to be redefined differently according to the value of flag
%    \fbo{ThinSpaceInFrenchNumbers}.
%    As |\npstylefrench| was undefined in old versions of
%    \pkg{numprint}, we have to provide this command.
%    \begin{macrocode}
  \@ifpackageloaded{numprint}%
  {\ifnprt@autolanguage
     \providecommand*{\npstylefrench}{}%
     \ifFBThinSpaceInFrenchNumbers
       \renewcommand*\npstylefrench{%
          \npthousandsep{\,}%
          \npdecimalsign{,}%
          \npproductsign{\cdot}%
          \npunitseparator{\,}%
          \npdegreeseparator{}%
          \nppercentseparator{\nprt@unitsep}%
          }%
     \else
       \renewcommand*\npstylefrench{%
          \npthousandsep{~}%
          \npdecimalsign{,}%
          \npproductsign{\cdot}%
          \npunitseparator{\,}%
          \npdegreeseparator{}%
          \nppercentseparator{\nprt@unitsep}%
          }%
     \fi
     \npaddtolanguage{french}{french}%
   \fi}{}%
%    \end{macrocode}
%    \fbo{FrenchSuperscripts}: if \fbo{true} |\up=\fup|, else
%    |\up=\textsuperscript|.  Anyway |\up*=\FB@up@fake|.  The star-form
%    |\up*{}| is provided for fonts that lack some superior letters:
%    Adobe Jenson Pro and Utopia Expert have no ``g superior’’ for
%    instance.
%    \begin{macrocode}
  \ifFBFrenchSuperscripts
    \DeclareRobustCommand*{\up}{\@ifstar{\FB@up@fake}{\fup}}%
  \else
    \DeclareRobustCommand*{\up}{\@ifstar{\FB@up@fake}%
                                        {\textsuperscript}}%
  \fi
%    \end{macrocode}
%    \fbo{LowercaseSuperscripts}: if \fbo{false} |\FB@lc| is redefined
%    to do nothing.
%    \begin{macrocode}
  \ifFBLowercaseSuperscripts
  \else
    \renewcommand*{\FB@lc}[1]{##1}%
  \fi
%    \end{macrocode}
%
% \changes{v3.0a}{2014/02/20}{With koma-script and memoir class,
%    customise \cs{captionformat} and \cs{captiondelim}.}
%
% \changes{v3.1f}{2015/05/31}{Definition of \cs{captionformat} and
%    \cs{captiondelim} changed when option CustomiseFigTabCaptions
%    is set to false.}
%
% \changes{v3.1f}{2015/05/31}{Bug fix for the beamer class:
%    figure and table captions are now consistent with frenchb’s
%    documentation.  Pointed out by Denis Bitouzé.}
%
%    Unless \fbo{CustomiseFigTabCaptions} has been set to \fbo{false},
%    use |\CaptionSeparator| for koma-script, \cls{memoir} and
%    \cls{beamer} classes.
%    \begin{macrocode}
  \ifFBCustomiseFigTabCaptions
    \ifFB@koma
      \renewcommand*{\captionformat}{\CaptionSeparator}%
    \fi
    \@ifclassloaded{memoir}%
       {\captiondelim{\CaptionSeparator}}{}%
    \@ifclassloaded{beamer}%
       {\defbeamertemplate{caption label separator}{FBcustom}{%
            \CaptionSeparator}%
        \setbeamertemplate{caption label separator}[FBcustom]}{}%
  \else
%    \end{macrocode}
%    When \fbo{CustomiseFigTabCaptions} is \fbo{false}, have the colon
%    behave properly in French: locally force |\autospace@beforeFDP| in
%    case of \fbo{AutoSpacePunctuation=false}.
%    \begin{macrocode}
    \ifFB@koma
      \renewcommand*{\captionformat}{{\autospace@beforeFDP : }}%
    \fi
    \@ifclassloaded{memoir}%
       {\captiondelim{{\autospace@beforeFDP : }}%
       }{}%
    \@ifclassloaded{beamer}%
       {\defbeamertemplate{caption label separator}{FBcolon}{%
             {\autospace@beforeFDP : }}%
        \setbeamertemplate{caption label separator}[FBcolon]%
       }{}%
  \fi
%    \end{macrocode}
%    \fbo{ShowOptions}: if \fbo{true}, print the list of all options
%    to the \file{.log} file.
%    \begin{macrocode}
  \ifFBShowOptions
    \GenericWarning{* }{%
     * **** List of possible options for frenchb ****\MessageBreak
     [Default values between brackets when frenchb is loaded *LAST*]%
     \MessageBreak
     ShowOptions=true [false]\MessageBreak
     StandardLayout=true [false]\MessageBreak
     GlobalLayoutFrench=false [true]\MessageBreak
     StandardLists=true [false]\MessageBreak
     IndentFirst=false [true]\MessageBreak
     ReduceListSpacing=false [true]\MessageBreak
     ListOldLayout=true [false]\MessageBreak
     StandardItemizeEnv=true [false]\MessageBreak
     StandardEnumerateEnv=true [false]\MessageBreak
     StandardItemLabels=true [false]\MessageBreak
     ItemLabels=\textemdash, \textbullet,
        \protect\ding{43},... [\textendash]\MessageBreak
     ItemLabeli=\textemdash, \textbullet,
        \protect\ding{43},... [\textendash]\MessageBreak
     ItemLabelii=\textemdash, \textbullet,
        \protect\ding{43},... [\textendash]\MessageBreak
     ItemLabeliii=\textemdash, \textbullet,
        \protect\ding{43},... [\textendash]\MessageBreak
     ItemLabeliv=\textemdash, \textbullet,
        \protect\ding{43},... [\textendash]\MessageBreak
     FrenchFootnotes=false [true]\MessageBreak
     AutoSpaceFootnotes=false [true]\MessageBreak
     AutoSpacePunctuation=false [true]\MessageBreak
     OriginalTypewriter=true [false]\MessageBreak
     ThinColonSpace=true [false]\MessageBreak
     ThinSpaceInFrenchNumbers=true [false]\MessageBreak
     FrenchSuperscripts=false [true]\MessageBreak
     LowercaseSuperscripts=false [true]\MessageBreak
     PartNameFull=false [true]\MessageBreak
     SuppressWarning=true [false]\MessageBreak
     CustomiseFigTabCaptions=false [true]\MessageBreak
     OldFigTabCaptions=true [false]\MessageBreak
     SmallCapsFigTabCaptions=false [true]\MessageBreak
     INGuillSpace=true [false]\MessageBreak
     InnerGuillSingle=true [false]\MessageBreak
     EveryParGuill=open, close, none [open]\MessageBreak
     EveryLineGuill=open, close, none
                   [open in LuaTeX, none otherwise]\MessageBreak
     og= <left quote character>, fg= <right quote character>%
     \MessageBreak
     *********************************************%
     \MessageBreak\protect\frenchsetup{ShowOptions}}
  \fi
}
%    \end{macrocode}
%  \end{macro}
%
% \changes{v2.0}{2006/12/06}{Added warning for OT1 encoding.}
%
% \changes{v2.1b}{2008/04/07}{Disable some commands in bookmarks.}
%
% \changes{v2.5e}{2011/04/03}{\cs{pdfstringdefDisableCommands} should
%    redefine \cs{FB@og} and \cs{FB@fg} instead of \cs{og} and \cs{fg}
%    so that it works also when quotes are entered as characters.
%    Reported by Sébastien Gouezel.}
%
%    At |\begin{document}|, we have to provide an |\xspace| command
%    in case the \pkg{xspace} package is not loaded, do some setup for
%    \file{hyperref}’s bookmarks, execute |\FBprocess@options|, switch
%    LuaTeX punctuation on and issue some warnings if necessary.
%    \begin{macrocode}
\AtBeginDocument{%
   \providecommand*{\xspace}{\relax}%
%    \end{macrocode}
%
% \changes{v2.5g}{2011/11/13}{Redefine \cs{degre}, \cs{degres} \cs{at}
%    \cs{circonflexe} and \cs{tild} for bookmarks.  Add \cs{fup} also.}
%
% \changes{v3.2g}{2017/01/24}{Add \cs{boi} to redefinitions for
%    bookmarks.}
%
%    Let’s redefine some commands in \file{hyperref}’s bookmarks.
%    \begin{macrocode}
   \ifdefined\pdfstringdefDisableCommands
     \pdfstringdefDisableCommands{%
        \let\up\relax
        \let\fup\relax
        \let\degre\textdegree
        \let\degres\textdegree
        \def\ieme{e\xspace}%
        \def\iemes{es\xspace}%
        \def\ier{er\xspace}%
        \def\iers{ers\xspace}%
        \def\iere{re\xspace}%
        \def\ieres{res\xspace}%
        \def\FrenchEnumerate#1{#1\degre\space}%
        \def\FrenchPopularEnumerate#1{#1\degre)\space}%
        \def\No{N\degre\space}%
        \def\no{n\degre\space}%
        \def\Nos{N\degre\space}%
        \def\nos{n\degre\space}%
        \def\FB@og{\guillemotleft\space}%
        \def\FB@fg{\space\guillemotright}%
        \def\at{@}%
        \def\circonflexe{\string^}%
        \def\tild{\string~}%
        \def\boi{\textbackslash}%
        \let\bsc\textsc
      }%
   \fi
%    \end{macrocode}
%    Let’s now process the remaining options, either not explicitly
%    set by \fbsetup{} or possibly modified by packages loaded after
%    \frenchb.
%    \begin{macrocode}
   \FBprocess@options
%    \end{macrocode}
%    The final definitions of commands ruling spacing in French
%    been known, let’s reset the corresponding toks for LuaTeX
%    and load file \file{frenchb.lua} (LuaTeX only).
%    \begin{macrocode}
   \ifFB@luatex@punct
     \FBcolonsp=\expandafter{\meaning\FBcolonspace}
     \FBthinsp= \expandafter{\meaning\FBthinspace}
     \FBguillsp=\expandafter{\meaning\FBguillspace}
     \activate@luatexpunct
   \fi
%    \end{macrocode}
%
% \changes{v3.0a}{2014/01/28}{Test \cs{ifXeTeX} changed to
%    \cs{ifFBunicode} and `xltxtra’ changed to `fontspec’.}
%
% \changes{v3.1a}{2014/06/14}{fontspec is not required for T1 fonts
%    used with the luainputenc.sty package.}
%
% \changes{v3.2g}{2017/01/24}{fontspec defines TU encoding now and
%    no longer loads xunicode.sty. Test changed.}
%
%    Some warnings are issued when output font encodings are not
%    properly set.  With XeLaTeX or LuaLaTeX, \file{fontspec.sty}
%    should be loaded unless T1 encoded fonts are used through
%    \file{luainputenc}, in the latter case |\FB@og| and |\FB@fg|
%    have to be redefined; with (pdf)\LaTeX, a warning is issued
%    when OT1 encoding is in use at the |\begin{document}|.
%    Mind that |\encodingdefault| is defined as `long’, defining
%    |\FBOTone| with |\newcommand*| would fail!
%    \begin{macrocode}
   \ifFBunicode
     \@ifpackageloaded{fontspec}{}%
        {\@ifpackageloaded{luainputenc}{}%
            {\PackageWarning{frenchb.ldf}%
                {Add \protect\usepackage{fontspec} to the\MessageBreak
                 preamble of your document, reported}%
            }%
        }
   \else
     \begingroup \newcommand{\FBOTone}{OT1}%
     \ifx\encodingdefault\FBOTone
       \PackageWarning{frenchb.ldf}%
          {OT1 encoding should not be used for French.%
           \MessageBreak
           Add \protect\usepackage[T1]{fontenc} to the
           preamble\MessageBreak of your document; reported}%
     \fi
     \endgroup
   \fi
}
%    \end{macrocode}
%
%  \subsection{French lists}
%  \label{ssec-lists}
%
% \changes{v2.6a}{2013/01/30}{Bug correction: changing \cs{leftmargin}
%    cannot be done only for itemize-lists: it messes up
%    embedded enumerate lists.  Pointed out by Denis Bitouzé.
%    Lists have been completely redesigned in frenchb v.~2.6a.
%    An option for backward compatibility is provided.}
%
%  \begin{macro}{\listFB}
%  \begin{macro}{\listORI}
%  \begin{macro}{\FB@listVsettings}
%    Vertical spacing in lists should be shorter in French texts than
%    the defaults provided by \LaTeX.
%    Note that the easy way, just changing values of vertical spacing
%    parameters when entering French and restoring them to their
%    defaults on exit would not work; so we define the command
%    |\FB@listVsettings| to hold the settings to be used by
%    the French variant |\listFB| of |\list|.  Note that switching
%    to |\listFB| reduces vertical spacing in \emph{all} environments
%    built on |\list|: |itemize|, |enumerate|, |description|, but also
%    |abstract|, |quotation|, |quote| and |verse|\dots{}
%
%    The amount of vertical space before and after a list is given by
%    |\topsep| + |\parskip| (+ |\partopsep| if the list starts a new
%    paragraph).  IMHO, |\parskip| should be added \emph{only} when
%    the list starts a new paragraph, so I subtract |\parskip| from
%    |\topsep| and add it back to |\partopsep|; this will normally
%    make no difference because |\parskip|’s default value is 0pt, but
%    will be noticeable when |\parskip| is \emph{not} null.
%
%    \begin{macrocode}
\let\listORI\list
\let\endlistORI\endlist
\def\FB@listVsettings{%
      \setlength{\itemsep}{0.4ex plus 0.2ex minus 0.2ex}%
      \setlength{\parsep}{0.4ex plus 0.2ex minus 0.2ex}%
      \setlength{\topsep}{0.8ex plus 0.4ex minus 0.4ex}%
      \setlength{\partopsep}{0.4ex plus 0.2ex minus 0.2ex}%
%    \end{macrocode}
%    |\parskip| is of type `skip’, its mean value only (\emph{not
%    the glue}) should be subtracted from |\topsep| and added to
%    |\partopsep|, so convert |\parskip| to a `dimen’ using
%    |\@tempdima|.
%    \begin{macrocode}
      \@tempdima=\parskip
      \addtolength{\topsep}{-\@tempdima}%
      \addtolength{\partopsep}{\@tempdima}%
}
\def\listFB#1#2{\listORI{#1}{\FB@listVsettings #2}}
\let\endlistFB\endlist
%    \end{macrocode}
%  \end{macro}
%  \end{macro}
%  \end{macro}
%
%    Let’s now consider French itemize-lists.  They differ from those
%    provided by the standard \LaTeXe{} classes:
%    \begin{itemize}
%      \item The `\textbullet’ is never used in French itemize-lists,
%          an emdash `---’ or an endash  `--’ is preferred for all
%          levels.  The item label to be used in French is stored in
%          |\FrenchLabelItem}|, it defaults to `---’ and can be changed
%          using \fbsetup{} (see section~\ref{ssec-keyval}).
%      \item Vertical spacing between items, before and after
%         the list, should be \emph{null} with \emph{no glue} added;
%      \item In French the labels of itemize-lists are vertically
%         aligned as follows:\\
%          \fbox{\parbox[t]{.5\textwidth}{%
%            \parindent=1.5em
%            \def\FBlabel{\textemdash}%
%            \settowidth{\labelwidth}{\FBlabel}%
%            Text starting at `parindent’ \\
%            \noindent$\Leftarrow$ Leftmargin\\
%            \hspace*{\parindent}\textemdash{} first item\dots\\
%            \hspace*{\parindent}\hspace*{\labelwidth}%
%                 \textemdash{} first second level item\\
%            \hspace*{\parindent}\hspace*{\labelwidth}%
%                 \textemdash{} next one\dots{}\\
%            \hspace*{\parindent}\textemdash{} second item\dots
%          }}
%    \end{itemize}
%
%  \begin{macro}{\FrenchLabelItem}
% \changes{v2.6a}{2013/01/30}{default changed from \cs{textendash} to
%    \cs{textemdash}.}
%  \begin{macro}{\Frlabelitemi}
%  \begin{macro}{\Frlabelitemii}
%  \begin{macro}{\Frlabelitemiii}
%  \begin{macro}{\Frlabelitemiv}
%    Default labels for French itemize-lists (same label for all levels):
%    \begin{macrocode}
\newcommand*{\FrenchLabelItem}{\textemdash}
\newcommand*{\Frlabelitemi}{\FrenchLabelItem}
\newcommand*{\Frlabelitemii}{\FrenchLabelItem}
\newcommand*{\Frlabelitemiii}{\FrenchLabelItem}
\newcommand*{\Frlabelitemiv}{\FrenchLabelItem}
%    \end{macrocode}
%  \end{macro}
%  \end{macro}
%  \end{macro}
%  \end{macro}
%  \end{macro}
%
%  \begin{macro}{\listindentFB}
%  \begin{macro}{\descindentFB}
%  \begin{macro}{\labelwidthFB}
%    Let’s define three lengths |\listindentFB|, |\descindentFB| and
%    |\labelwidthFB| to customise lists’ horizontal indentations.
%    They are given silly values here ($-1$\,pt) in order to eventually
%    enable their customisation in the preamble.  They will get
%    reasonnable defaults later when entering French (see
%    |\bbl@frenchlabelitems|) unless they have been customised.
%    \begin{macrocode}
\newlength\listindentFB
\setlength{\listindentFB}{-1pt}
\newlength\descindentFB
\setlength{\descindentFB}{-1pt}
\newlength\labelwidthFB
\setlength{\labelwidthFB}{-1pt}
%    \end{macrocode}
%  \end{macro}
%  \end{macro}
%  \end{macro}
%
% \changes{v3.2e}{2016/10/28}{Add missing redefinitions for
%    \cs{leftmarginv}, \cs{leftmarginvi}. Suggested by J.F. Burnol.}
%
%  \begin{macro}{\FB@listHsettings}
%  \begin{macro}{\leftmarginFB}
%    |\FB@listHsettings| holds the new horizontal settings chosen for
%    French lists itemize and enumerate starting with version~2.6a.
%    They are based on the look resquested in French for
%    itemize-lists.
%    \begin{macrocode}
\newlength\leftmarginFB
\def\FB@listHsettings{%
   \leftmarginFB\labelwidthFB
   \advance\leftmarginFB \labelsep
   \bbl@for\FB@dp {1, 2, 3, 4, 5, 6}%
     {\csname leftmargin\romannumeral\FB@dp\endcsname \leftmarginFB}%
   \advance\leftmargini \listindentFB
   \leftmargin\csname leftmargin\ifnum\@listdepth=\@ne i\else
                                                      ii\fi\endcsname
}
%    \end{macrocode}
%  \end{macro}
%  \end{macro}
%
%  \begin{macro}{\itemizeFB}
%  \begin{macro}{\FB@itemizesettings}
%
% \changes{v2.6f}{2013/11/25}{\cs{labelwidth} must be reset, f.i. when
%    an itemize list occurs inside environments based on trivlist which
%    set \cs{labelwidth} to 0 (see proof environment in amsthm.sty).
%    Bug pointed out by Julien Hauseux.}
%
% \changes{v2.6g}{2013/12/30}{Suppress all vertical spaces only if
%    ReduceListSpacing is true.  Pointed out by Pierre Willaime.}
%
%    New environment for French itemize-lists.\\ |\FB@itemizesettings|
%    does two things: first suppress all vertical spaces including glue
%    when option \fbo{ReduceListSpacing} is set, then set horizontal
%    indentations according to |\FB@listHsettings| unless option
%    \fbo{ListOldLayout} is \fbo{true} (compatibility with lists up to
%    v.~2.5k).
%    \begin{macrocode}
\def\FB@itemizesettings{%
    \ifFBReduceListSpacing
      \setlength{\itemsep}{\z@}%
      \setlength{\parsep}{\z@}%
      \setlength{\topsep}{\z@}%
      \setlength{\partopsep}{\z@}%
      \@tempdima=\parskip
      \addtolength{\topsep}{-\@tempdima}%
      \addtolength{\partopsep}{\@tempdima}%
    \fi
    \settowidth{\labelwidth}{\csname\@itemitem\endcsname}%
    \ifFBListOldLayout
      \setlength{\leftmargin}{\labelwidth}%
      \addtolength{\leftmargin}{\labelsep}%
      \addtolength{\leftmargin}{\parindent}%
    \else
      \FB@listHsettings
    \fi
}
%    \end{macrocode}
%    The definition of |\itemizeFB| follows the one of |\itemize|
%    in standard \LaTeXe{} classes (see \file{ltlists.dtx}),
%    spaces are customised by |\FB@itemizesettings|.
%
%    \begin{macrocode}
\def\itemizeFB{%
    \ifnum \@itemdepth >\thr@@\@toodeep\else
      \advance\@itemdepth\@ne
      \edef\@itemitem{labelitem\romannumeral\the\@itemdepth}%
      \expandafter
      \listORI
      \csname\@itemitem\endcsname
      \FB@itemizesettings
    \fi
}
\let\enditemizeFB\endlistORI
%    \end{macrocode}
%    \end{macro}
%    \end{macro}
%
%    \begin{macrocode}
\def\labelitemsFB{%
    \let\labelitemi\Frlabelitemi
    \let\labelitemii\Frlabelitemii
    \let\labelitemiii\Frlabelitemiii
    \let\labelitemiv\Frlabelitemiv
    \ifdim\labelwidthFB<\z@
      \settowidth{\labelwidthFB}{\FrenchLabelItem}%
    \fi
    \ifdim\listindentFB<\z@
      \ifdim\parindent=\z@
        \setlength{\listindentFB}{1.5em}%
      \else
        \setlength{\listindentFB}{\parindent}%
      \fi
    \fi
    \ifdim\descindentFB<\z@
      \setlength{\descindentFB}{\listindentFB}%
    \fi
}
%    \end{macrocode}
%
%  \begin{macro}{\enumerateFB}
%    The definition of |\enumerateFB|, new to version~2.6a, follows
%    the one of |\enumerate| in standard \LaTeXe{} classes (see
%    \file{ltlists.dtx}), vertical spaces are customised (or not)
%    via |\list| (=|\listFB| or |\listORI|) and horizontal spaces
%    (leftmargins) are borrowed from |itemize| lists via
%    |\FB@listHsettings|.
%
%    \begin{macrocode}
\def\enumerateFB{%
  \ifnum \@enumdepth >\thr@@\@toodeep\else
    \advance\@enumdepth\@ne
    \edef\@enumctr{enum\romannumeral\the\@enumdepth}%
    \expandafter
    \list
      \csname label\@enumctr\endcsname
      {\FB@listHsettings
       \usecounter\@enumctr\def\makelabel##1{\hss\llap{##1}}}%
  \fi
}
\let\endenumerateFB\endlistORI
%    \end{macrocode}
%    \end{macro}
%
%  \begin{macro}{\descriptionFB}
% \changes{v2.6b}{2013/02/10}{Settings of \cs{FB@listHsettings} should
%    apply to description lists too.}
%
% \changes{v3.0a}{2014/02/28}{Added \cs{listindentFB} to
%    \cs{itemindent}.  Suggested by Denis Bitouzé.}
%
% \changes{v3.2d}{2014/07/01}{Changed \cs{listindentFB} to
%    \cs{descindentFB} which defaults to \cs{listindentFB}.
%    \cs{leftmargini} reduced when \cs{descindentFB} is null.}
%
%    Same tuning for the \env{description} environment (see
%    \file{classes.dtx} for the original definition).
%    Customisable length |\descindentFB|, which defaults to
%    |\listindentFB|, is added to |\itemindent| (first level only).
%    When |\descindentFB=0pt| (1rst level labels start at the left
%    margin), |\leftmargini| is reduced to |\listindentFB| instead
%    of |\listindentFB| + |\leftmarginFB|.
%
%    \begin{macrocode}
\def\descriptionFB{%
      \list{}{\FB@listHsettings
              \labelwidth\z@
              \itemindent-\leftmargin
              \ifnum\@listdepth=1
                \ifdim\descindentFB=\z@
                  \ifdim\listindentFB>\z@
                    \leftmargini\listindentFB
                    \leftmargin\leftmargini
                    \itemindent-\leftmargin
                  \fi
                \else
                  \advance\itemindent by \descindentFB
                \fi
              \fi
              \let\makelabel\descriptionlabel}%
}
\let\enddescriptionFB\endlistORI
%    \end{macrocode}
%    \end{macro}
%
%    \begin{macro}{\update@frenchlists}
%    \begin{macro}{\bbl@frenchlistlayout}
%    |\update@frenchlists| will set up lists according to the final
%    options (default or part of \fbsetup{} eventually overruled in
%    |\FBprocess@options|).
%    \begin{macrocode}
\def\update@frenchlists{%
  \ifFBReduceListSpacing \let\list\listFB \fi
  \ifFBStandardItemizeEnv
  \else \let\itemize\itemizeFB \fi
  \ifFBStandardItemLabels
  \else \labelitemsFB \fi
  \ifFBStandardEnumerateEnv
  \else \let\enumerate\enumerateFB \let\description\descriptionFB \fi
}
%    \end{macrocode}
%
% \changes{v3.2h}{2017/02/03}{Execute \cs{update@frenchlists} only
%    if GlobalLayoutFrench is false.
%    Delete stuff for lists in \cs{noextrasfrench}.}
%
%    If \fbo{GlobalLayoutFrench=true}, nothing has to be done at
%    language’s switches regarding lists.  Otherwise, |\extrasfrench|
%    saves the standard settings for lists and then executes
%    |\update@frenchlists|. In both cases, there is nothing to do for
%    lists in |\noextrasfrench|.
%
%    In order to ensure compatibility with packages customising lists,
%    the command |\update@frenchlists| should not be included in the
%    first call to |\extrasfrench| which occurs \emph{before} the
%    relevant flags are finally set, so we define |\FB@ufl| as
%    |\relax|, it will be redefined later `AtBeginDocument’ by
%    |\FBprocess@options| as |\update@frenchlists|,
%    see p.~\pageref{FB@ufl}.
%    \begin{macrocode}
\def\FB@ufl{\relax}
\def\bbl@frenchlistlayout{%
  \ifFBGlobalLayoutFrench
  \else
    \babel@save\list         \babel@save\itemize
    \babel@save\enumerate    \babel@save\description
    \babel@save\labelitemi   \babel@save\labelitemii
    \babel@save\labelitemiii \babel@save\labelitemiv
    \FB@ufl
  \fi
}
\FB@addto{extras}{\bbl@frenchlistlayout}
%    \end{macrocode}
%    \end{macro}
%    \end{macro}
%
%  \subsection{French indentation of sections}
%  \label{ssec-indent}
%
%  \begin{macro}{\bbl@frenchindent}
%  \begin{macro}{\bbl@nonfrenchindent}
%    In French the first paragraph of each section should be indented,
%    this is another difference with US-English.  This is controlled
%    by the flag |\if@afterindent|.
%
% \changes{v2.3d}{2009/03/16}{Bug correction: previous versions of
%    frenchb set the flag \cs{if@afterindent} to false outside
%    French which is correct for English but wrong for some languages
%    like Spanish.  Pointed out by Juan José Torrens.}
%
%    We will need to save the value of the flag |\if@afterindent|
%    `AtBeginDocument’ before eventually changing its value.
%
%    \begin{macrocode}
\def\bbl@frenchindent{%
  \ifFBGlobalLayoutFrench
  \else
    \babel@save\@afterindentfalse
  \fi
  \ifFBIndentFirst
    \let\@afterindentfalse\@afterindenttrue
    \@afterindenttrue
  \fi}
\def\bbl@nonfrenchindent{%
  \ifFBGlobalLayoutFrench
    \ifFBIndentFirst
      \@afterindenttrue
    \fi
  \fi}
\FB@addto{extras}{\bbl@frenchindent}
\FB@addto{noextras}{\bbl@nonfrenchindent}
%    \end{macrocode}
%  \end{macro}
%  \end{macro}
%
%  \subsection{Formatting footnotes}
%  \label{ssec-footnotes}
%
% \changes{v2.0}{2006/11/06}{Footnotes are now printed
%     by default `à la française’ for the whole document.}
%
% \changes{v2.0b}{2007/04/18}{Footnotes: Just do nothing
%    (except warning) when the bigfoot package is loaded.}
%
%    The \pkg{bigfoot} package deeply changes the way footnotes are
%    handled.  When \file{bigfoot} is loaded, we just warn the user
%    that \frenchb{} will drop the customisation of footnotes.
%
%    The layout of footnotes is controlled by two flags
%    |\ifFBAutoSpaceFootnotes| and |\ifFBFrenchFootnotes| which are
%    set by options of \fbsetup{} (see section~\ref{ssec-keyval}).
%    The layout of footnotes \emph{does not depend} on the current
%    language (just think of two footnotes on the same page looking
%    different because one was called in a French part, the other one
%    in English!).
%
%    We save the original definition of |\@footnotemark| at the
%    |\begin{document}| in order to include any customisation that
%    packages might have done; we define a variant |\@footnotemarkFB|
%    which just adds a thin space before the number or symbol calling
%    a footnote (any space typed in is removed first). The choice
%    between the two definitions (valid for the whole document) is
%    controlled by flag |\ifFBAutoSpaceFootnotes|.
%
% \changes{v2.4a}{2009/11/23}{\cs{PackageWarning} changed to
%    \cs{PackageInfo} (when bigfoot package in use).}
%
%    \begin{macrocode}
\AtBeginDocument{\@ifpackageloaded{bigfoot}%
                   {\PackageInfo{frenchb.ldf}%
                     {bigfoot package in use.\MessageBreak
                      frenchb will NOT customise footnotes;%
                      \MessageBreak reported}}%
                   {\let\@footnotemarkORI\@footnotemark
                    \def\@footnotemarkFB{\leavevmode\unskip\unkern
                                         \,\@footnotemarkORI}%
                    \ifFBAutoSpaceFootnotes
                      \let\@footnotemark\@footnotemarkFB
                    \fi}%
                }
%    \end{macrocode}
%
%  \begin{macro}{\@makefntextFB}
%    We then define |\@makefntextFB|, a variant of |\@makefntext|
%    which is responsible for the layout of footnotes, to match the
%    specifications of the French `Imprimerie Nationale’:  footnotes
%    will be indented by |\parindentFFN|, numbers (if any) typeset on
%    the baseline (instead of superscripts), right aligned on
%    |\parindentFFN| and followed by a dot and an half quad kern.
%    Whenever symbols are used to number footnotes (as in |\thanks|
%    for instance), we switch back to the standard layout (the French
%    layout of footnotes is meant for footnotes numbered by arabic or
%    roman digits).
%
% \changes{v2.0}{2006/11/06}{\cs{parindentFFN} not changed if
%    already defined (required by JA for cah-gut.cls).}
%
% \changes{v2.3b}{2008/12/06}{New commands \cs{dotFFN} and
%    \cs{kernFFN} for more flexibility (suggested by JA).}
%
% \changes{v3.2a}{2016/03/14}{beamer.cls requires a specific
%    definition of \cs{@makefntextFB} (pointed out by DB).
%    The same is true for memoir and koma-script classes (done).}
%
%    The value of |\parindentFFN| will be redefined at the
%    |\begin{document}|, as the maximum of |\parindent| and 1.5em
%    \emph{unless} it has been set in the preamble (the weird value
%    10in is just for testing whether |\parindentFFN| has been set
%    or not).
%    \begin{macrocode}
\newdimen\parindentFFN
\parindentFFN=10in
%    \end{macrocode}
%    |\FBfnindent| will be set `AtBeginDocument’ to the width of
%    the box holding the footnote mark, |\dotFFN| and |\kernFFN|
%    (flushed right).  It is used by \cls{memoir} and koma-script
%    classes.
%    \begin{macrocode}
\newcommand*{\dotFFN}{.}
\newcommand*{\kernFFN}{\kern .5em}
\newlength\FBfnindent
%    \end{macrocode}
%    |\@makefntextFB|’s definition is now tuned according to the
%    document’s class for better compatibility.
%
%    Koma-script classes provide |\deffootnote|, a handy command
%    to customise the footnotes’ layout (see English manual
%    \file{scrguien.pdf}); it redefines |\@makefntext| and
%    |\@@makefnmark|.  First, save the original definitions.
%    \begin{macrocode}
\ifFB@koma
  \let\@makefntextORI\@makefntext
  \let\@@makefnmarkORI\@@makefnmark
%    \end{macrocode}
%    |\@makefntextFB| and |\@@makefnmarkFB| will be used when option
%    \fbo{FrenchFootnotes} is \fbo{true}.
%    \begin{macrocode}
  \deffootnote[\FBfnindent]{0pt}{\parindentFFN}%
              {\thefootnotemark\dotFFN\kernFFN}
  \let\@makefntextFB\@makefntext
  \let\@@makefnmarkFB\@@makefnmark
%    \end{macrocode}
%    |\@makefntextTH| and |\@@makefnmarkTH| are meant for the
%    |\thanks| command used by |\maketitle| when \fbo{FrenchFootnotes}
%    is \fbo{true}.
%    \begin{macrocode}
  \deffootnote[\parindentFFN]{0pt}{\parindentFFN}%
              {\textsuperscript{\thefootnotemark}}
  \let\@makefntextTH\@makefntext
  \let\@@makefnmarkTH\@@makefnmark
%    \end{macrocode}
%    Restore the original definitions.
%    \begin{macrocode}
  \let\@makefntext\@makefntextORI
  \let\@@makefnmark\@@makefnmarkORI
\fi
%    \end{macrocode}
%    Definitions for the \cls{memoir} class:
%    \begin{macrocode}
\@ifclassloaded{memoir}
%    \end{macrocode}
%    (see original definition in \file{memman.pdf})
%    \begin{macrocode}
   {\newcommand{\@makefntextFB}[1]{%
      \def\footscript##1{##1\dotFFN\kernFFN}%
      \setlength{\footmarkwidth}{\FBfnindent}%
      \setlength{\footmarksep}{-\footmarkwidth}%
      \setlength{\footparindent}{\parindentFFN}%
      \makefootmark #1}%
   }{}
%    \end{macrocode}
%    Definitions for the \cls{beamer} class:
%    \begin{macrocode}
\@ifclassloaded{beamer}
%    \end{macrocode}
%    (see original definition in \file{beamerbaseframecomponents.sty}),
%    note that for the \cls{beamer} class footnotes are LR-boxes, not
%    paragraphs, so |\parindentFFN| is irrelevant.
%    class.
%
% \changes{v3.2h}{2016/04/09}{With beamer.cls, add \cs{llap} to
%    \cs{@thefnmark} for notes numbered over 99.}
%
%    \begin{macrocode}
   {\def\@makefntextFB#1{%
      \def\insertfootnotetext{#1}%
      \def\insertfootnotemark{\insertfootnotemarkFB}%
      \usebeamertemplate***{footnote}}%
    \def\insertfootnotemarkFB{%
      \usebeamercolor[fg]{footnote mark}%
      \usebeamerfont*{footnote mark}%
      \llap{\@thefnmark}\dotFFN\kernFFN}%
   }{}
%    \end{macrocode}
%    Now the default definition of |\@makefntextFB| for standard LaTeX
%    and AMS classes.
%
%    The next command prints the footnote mark according to the
%    specifications of the French `Imprimerie Nationale’.
%    Keep in mind that |\@thefnmark| might be empty (i.e.\ in
%    AMS classes’ titles)!
%    \begin{macrocode}
\providecommand*{\insertfootnotemarkFB}{%
  \parindent=\parindentFFN
  \rule\z@\footnotesep
  \setbox\@tempboxa\hbox{\@thefnmark}%
  \ifdim\wd\@tempboxa>\z@
    \llap{\@thefnmark}\dotFFN\kernFFN
  \fi}
\providecommand\@makefntextFB[1]{\insertfootnotemarkFB #1}
%    \end{macrocode}
%  \end{macro}
%
%    The rest of |\@makefntext|’s customisation is done at the
%    |\begin{document}|.  We save the original definition of
%    |\@makefntext|, and then redefine |\@makefntext| according
%    to the value of flag |\ifFBFrenchFootnotes| (true or false).
%    Koma-script classes require a special treatment.
%
%    \begin{macrocode}
\AtBeginDocument{%
   \@ifpackageloaded{bigfoot}{}%
      {\ifdim\parindentFFN<10in
       \else
         \parindentFFN=\parindent
         \ifdim\parindentFFN<1.5em \parindentFFN=1.5em \fi
       \fi
       \settowidth{\FBfnindent}{\dotFFN\kernFFN}%
       \addtolength{\FBfnindent}{\parindentFFN}%
       \let\@makefntextORI\@makefntext
       \ifFB@koma
%    \end{macrocode}
%    Definition of |\@makefntext| for koma-script classes:
%    \begin{macrocode}
         \let\@@makefnmarkORI\@@makefnmark
         \long\def\@makefntext#1{%
                \ifFBFrenchFootnotes
                  \ifx\footnote\thanks
                    \let\@@makefnmark\@@makefnmarkTH
                    \@makefntextTH{#1}%
                  \else
                    \let\@@makefnmark\@@makefnmarkFB
                    \@makefntextFB{#1}%
                  \fi
                \else
                  \let\@@makefnmark\@@makefnmarkORI
                  \@makefntextORI{#1}%
                \fi}%
       \else
%    \end{macrocode}
%    Special add-on for the \cls{memoir} class: |\maketitle| redefines
%    |\@makefntext| as |\makethanksmark| which is customised as follows
%    to match the other notes’ vertical alignment.
%    \begin{macrocode}
         \@ifclassloaded{memoir}%
            {\ifFBFrenchFootnotes
               \setlength{\thanksmarkwidth}{\parindentFFN}%
               \setlength{\thanksmarksep}{-\thanksmarkwidth}%
             \fi
            }{}%
%    \end{macrocode}
%    Special add-on for the \cls{beamer} class: issue a warning
%    in case |\parindentFFN| has been changed.
%    \begin{macrocode}
         \@ifclassloaded{beamer}%
            {\ifFBFrenchFootnotes
               \ifdim\parindentFFN=1.5em\else
                 \FBWarning{%
                    \protect\parindentFFN\space is ineffective%
                    \MessageBreak within the beamer class.%
                    \MessageBreak Reported}%
               \fi
             \fi
            }{}%
%    \end{macrocode}
%    Definition of |\@makefntext| for all classes other than
%    koma-script:
%    \begin{macrocode}
         \long\def\@makefntext#1{%
                 \ifFBFrenchFootnotes
                   \@makefntextFB{#1}%
                 \else
                   \@makefntextORI{#1}%
                 \fi}%
       \fi
    }%
}
%    \end{macrocode}
%
%    For compatibility reasons, we provide definitions for the commands
%    dealing with the layout of footnotes in \frenchb{} version~1.6.
%    \fbsetup{} (see in section \ref{ssec-keyval}) should be
%    preferred for setting these options.  |\StandardFootnotes| may
%    still be used locally (in minipages for instance), that’s why the
%    test |\ifFBFrenchFootnotes| is done inside |\@makefntext|.
%    \begin{macrocode}
\newcommand*{\AddThinSpaceBeforeFootnotes}{\FBAutoSpaceFootnotestrue}
\newcommand*{\FrenchFootnotes}{\FBFrenchFootnotestrue}
\newcommand*{\StandardFootnotes}{\FBFrenchFootnotesfalse}
%    \end{macrocode}
%
%  \subsection{Clean up and exit}
%
% \changes{v3.0a}{2014/02/20}{french.cfg will be loaded (if found)
%    instead of frenchb.cfg.  NO NEED for .cfg files in French anyway.}
%
% \changes{v3.1h}{2015/08/19}{french.cfg from e-french conflicts with
%    frenchb. Do NOT load it (no need for .cfg files with frenchb
%    anyway).}
%
%    Final cleaning.
%    The macro |\ldf@finish| takes care for setting the main language
%    to be switched on at |\begin{document}| and resetting the
%    category code of \texttt{@} to its original value. |\loadlocalcfg|
%    is redefined locally in order not to load any \file{.cfg} file for
%    French.
%    \begin{macrocode}
\FBclean@on@exit
\let\FB@llc\loadlocalcfg
\let\loadlocalcfg\@gobble
\ldf@finish\CurrentOption
\let\loadlocalcfg\FB@llc
%    \end{macrocode}
% \iffalse
%</ldf>
%<*dtx>
% \fi
%%
%% \CharacterTable
%%  {Upper-case    \A\B\C\D\E\F\G\H\I\J\K\L\M\N\O\P\Q\R\S\T\U\V\W\X\Y\Z
%%   Lower-case    \a\b\c\d\e\f\g\h\i\j\k\l\m\n\o\p\q\r\s\t\u\v\w\x\y\z
%%   Digits        \0\1\2\3\4\5\6\7\8\9
%%   Exclamation   \!     Double quote  \"     Hash (number) \#
%%   Dollar        \$     Percent       \%     Ampersand     \&
%%   Acute accent  \'     Left paren    \(     Right paren   \)
%%   Asterisk      \*     Plus          \+     Comma         \,
%%   Minus         \-     Point         \.     Solidus       \/
%%   Colon         \:     Semicolon     \;     Less than     \<
%%   Equals        \=     Greater than  \>     Question mark \?
%%   Commercial at \@     Left bracket  \[     Backslash     \\
%%   Right bracket \]     Circumflex    \^     Underscore    \_
%%   Grave accent  \`     Left brace    \{     Vertical bar  \|
%%   Right brace   \}     Tilde         \~}
%%
% \iffalse
%</dtx>
% \fi
%
% \Finale
\endinput
