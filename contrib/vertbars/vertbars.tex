
%%%%%%%%%1%%%%%%%%%2%%%%%%%%%3%%%%%%%%%4%%%%%%%%%5
% Bundled source file for the VERTBARS package 
%--------1---------2---------3---------4---------5
% Please see the accompanying README for author,
% license, documentation and installation information
%%%%%%%%%1%%%%%%%%%2%%%%%%%%%3%%%%%%%%%4%%%%%%%%%5

\RequirePackage{filecontents}
\begin{filecontents}{vertbars.sty}
\NeedsTeXFormat{LaTeX2e}
\ProvidesPackage{vertbars}[2010/11/27 v1.0b vertical bars in the margin]

\newcommand{\LNenv}{runninglinenumbers}
\DeclareOption{switch}{%
  \renewcommand{\LNenv}{runningpagewiselinenumbers}
  \PassOptionsToPackage{\CurrentOption}{lineno}
}

\DeclareOption{switch*}{%
  \renewcommand{\LNenv}{runningpagewiselinenumbers}
  \PassOptionsToPackage{\CurrentOption}{lineno}
}

\DeclareOption*{\PassOptionsToPackage{\CurrentOption}{lineno}}
\ProcessOptions\relax
\RequirePackage{lineno}

% Code to add stuff at start and end of a pre-existing zero argument macro:
\newcommand{\addtodef}[3]{\begingroup
  \@temptokena{#2}%
  \toks@\expandafter{#1#3}%
  \edef\x{\endgroup
    \def\noexpand#1{\the\@temptokena \the\toks@}}%
  \x
}

% It's useful to preserve \cmd\baselineskip:
\newlength{\pwvbbl}
\setlength{\pwvbbl}{\baselineskip}

% Width of bars:
\newlength{\barwidth}
\setlength{\barwidth}{0.4pt}

% Horizontal space between bars:
\newlength{\barspace}
\setlength{\barspace}{0.5\linenumbersep}

\newcommand{\addtomakeLNL}{{\rule[-0.25\pwvbbl]{\barwidth}{1.1\pwvbbl}\hskip\barspace\relax}}
\newcommand{\pwvbLNL}{}

\newenvironment{vertbar}{%
  \pagewiselinenumbers % <= added v1.0b
  \begin{\LNenv}%
    \addtodef{\pwvbLNL}{}{\addtomakeLNL}%
    \let\LineNumber\pwvbLNL
}{%
  \end{\LNenv}%
}

\end{filecontents}
%%%%%%%%%1%%%%%%%%%2%%%%%%%%%3%%%%%%%%%4%%%%%%%%%5


% Conditionally compile the documentation & generate the .ins file:
\providecommand\documentationCompile{Y}
\makeatletter
\if\documentationCompile N
  \expandafter\@@end
\fi


\begin{filecontents*}{vertbars.ins}
%&latex
\def\documentationCompile{N}
\input vertbars.tex
\csname@@end\endcsname
\end{filecontents*}




\makeatletter
\documentclass{article}

\usepackage[it,medium]{titlesec}

\usepackage{array,bigfoot,vertbars}
\usepackage[svgnames]{xcolor}
\usepackage[colorlinks,linktocpage]{hyperref}

\usepackage{geometry}
\geometry{b5paper}

\usepackage{gmdoc}
\usepackage{gmverb}
\dekclubs
\stanzaskip=\bigskipamount 
\CodeSpacesGrey

\usepackage{tocloft,varwidth}
\setcounter{tocdepth}{1}
\def\tocwidthA{0.55}
\def\tocwidthB{0.44}
\def\cftpartfont{\scshape}
\def\cftsecfont{\small}
\cftbeforesecskip=0pt
\def\cftpartleader{}
\def\cftpartafterpnum{\cftparfillskip}
\def\cftsecleader{}
\def\cftsecafterpnum{\cftparfillskip}

\DeclareRobustCommand\pkg{\textsf}
\def\pkgopt#1{\texttt{[#1]}}
\newcommand\chng[1]{\marginpar{\footnotesize\raggedright\textsf{#1}}}

\def\PDF{\textsc{pdf}}
\def\PS{\textsc{ps}}
\def\DVI{\textsc{dvi}}
\def\EPS{\textsc{eps}}

\usepackage{amsmath,listings}
\lstset{basicstyle=\ttfamily,columns=fullflexible}

\usepackage{changepage}
\usepackage[T1]{fontenc}
\usepackage{microtype}
\usepackage{lmodern}
\usepackage[sc,osf]{mathpazo}
\linespread{1.1}
\frenchspacing

\GetFileInfo{vertbars.sty}
\begin{document}
{\addtocontents{toc}{\protect\begin{varwidth}[t]{\tocwidthA\linewidth}}}

\title{The \pkg{vertbars} package}
\author{%
 Author: Peter Wilson, Herries Press\\
 Maintainer: Will Robertson\\
 \texttt{will dot robertson at latex-project dot org}%
}
\date{\fileversion \qquad \filedate}

\maketitle

\section{Documentation}

Because this package uses \textsf{lineno} as a driver,
please read the documentation of that package first to
understand the mechanisms used here.

The \textsf{vertbars} package takes the same options as the \textsf{lineno} package.
In particular, the switch and switch* options control which side of the page any bars will be printed.
The package automatically calls the lineno package, so you just need to write:
\begin{verbatim}
    \usepackage[...]{vertbars}
\end{verbatim}

The package provides the \texttt{vertbar} environment, which is equivalent to the linenumbers environment except that a vertical bar replaces the line number.
Text paragraphs within a vertbar environment will be marked with a vertical bar in the margin. 
Nested environments generate multiple marginal bars.

The width of the bars is controlled by the value of \cmd\barwidth, which is initialised to \verb|0.4pt|, and can be changed via \cmd\setlength.

The horizontal seperation between adjacent bars is controlled by the value of \cmd\barspace, which is initialised to \verb|0.5\linenumbersep|, and can be changed via \cmd\setlength.
(\cmd\linenumbersep is a command from the lineno package that controls the spacing between line numbers and the text body).

Note: Bars can only be applied to complete paragraphs.
For bars between arbitrary points, use the \textsf{changebar} package.
Any limitations of the \textsf{lineno} package apply equally to the \textsf{vertbars} package.

\section*{Change History}

\begin{itemize}
\item[v1.0b] 2010/11/27 Fix behaviour with \texttt{switch} option
\item[v1.0a] 2009/09/04 New maintainer (Will Robertson)
\end{itemize}



\section*{Licence and copyright}
  
This work may be modified and/or distributed under the terms and
conditions of the \LaTeX\ Project Public License\footnote{\url{http://www.latex-project.org/lppl.txt}}, version~1.3c or later (your choice).
The current maintainer of this work is Will Robertson.

\bigskip
  \noindent
  Copyright Peter Wilson, 2000 \\
  Copyright Will Robertson, 2010 \\

{\addtocontents{toc}{\protect\end{varwidth}\protect\hfill}}
{\addtocontents{toc}{\protect\begin{varwidth}[t]{\protect\tocwidthB\protect\linewidth}}}
\clearpage
\parindent=0pt

\section{Implementation}
\DocInput{vertbars.sty}

{\addtocontents{toc}{\protect\end{varwidth}}}

\end{document}
