\documentclass{article}
\usepackage{graphicxsp}
\usepackage[tight,designiv,usetemplates]{web}
\usepackage{aeb_tilebg}
\usepackage{graphicxbox}

\title{The \textsf{GraphicxBox} Package\texorpdfstring{\\\textsf{GraphicxSP}, Transparency, Tiling}{: GraphicxSP, Transparency, Tiling}}
\author{D. P. Story}
\subject{Test file for the AcroTeX eDucation Bundle}
\keywords{LaTeX, Web package, tiled backgrounds, Adobe Acrobat}
\university{Acro\negthinspace\TeX.Net}
\email{dpstory@acrotex.net}
\def\webversion{\textcolor{webbrown}{www.acrotex.net}}
\revisionLabel{Prepared:}
\versionLabel{}

\newcommand{\cs}[1]{\texttt{\char`\\#1}}

\embedEPS[transparencyGroup]{cle_ind_back}{graphics/bg_cle_tile}
\embedEPS[transparencyGroup]{indianblanket}{graphics/indianblanket}
%
% Create a 100 by 100 rectangle of while for used as a transparent
% background to the \graphicxbox and \fgraphicbox
%
\begin{createImage}[transparencyGroup]{0 0 100 100}{whiteBG}
1 1 1 setrgbcolor 0 0 100 100 rectfill
\end{createImage}

\begin{createImage}[transparencyGroup]{0 0 100 100}{blueBG}
0 0 1 setrgbcolor 0 0 100 100 rectfill
\end{createImage}

\parindent0pt\parskip\medskipamount

\begin{document}

\maketitle

\vspace*{0.25in}

\begin{center}\sffamily\Large\bfseries\color{blue}
    Introduction
\end{center}
\begin{quote}\parskip6pt
This is the original application that I had envisioned for the
\textsf{GraphicxBox} package; using a graphical background behind a
\cs{parbox} with an interesting dark (and tiled) background for the
page. I wished to write on top of the graphical background, yet
have a degree of transparency for seeing through to the background.

We'll begin the tiling on the next page so you can see what I mean,
shall we.
\end{quote}

\newpage
\setTileBgGraphic[hiresbb,scale=.4,name=cle_ind_back]{\null}

\null\vskip-\baselineskip\vfil

\begin{center}
\graphicxbox[name=whiteBG,transparency={/ca .7 /BM/Normal}]{} %
{%
    \begin{minipage}{0.67\linewidth}\parskip6pt\bfseries
        This document introduces a new command, \cs{graphicxbox}. This
        command is quite similar to \cs{colorbox}, except
        \cs{graphicxbox} places a graphic in the background instead
        of a color. The graphic, in this case, is a simple white rectangle
        that has been given a an opacity of 0.7.

        As with \cs{colorbox}, the box is increased by \cs{fboxsep} on all sides.

        We use the \textsf{graphicxsp} package to get the transparency, and the
        \textsf{aeb\_tilebg} package to tile the background.
    \end{minipage}
}
\end{center}

\newpage

\null\vskip-\baselineskip
\vfil

\begin{center}
\setlength{\fboxrule}{2bp}\setlength{\fboxsep}{10bp}%
\fgraphicxbox{blue}[name=whiteBG,transparency={/ca .7 /BM/Normal}]{}
{%
    \begin{minipage}{0.67\linewidth}\parskip6pt\bfseries
        This display panel demos \cs{fgraphicxbox}. This command
        is similar to \cs{fcolorbox}, it does draw a boundary rule, but
        inserts a graphic image instead of a flat background. The
        graphic, in this case, is a simple white rectangle that has been
        given a an opacity of 0.7.

        As with \cs{fcolorbox}, the box is increased by \cs{fboxsep}
        on all sides, and the rule width is set by \cs{fboxrule}.
    \end{minipage}
}
\end{center}

\newpage

\null\vskip-\baselineskip\vfil

\begin{center}
\setlength{\fboxsep}{10bp}%
\graphicxbox{graphics/indianblanket}
{%
    \parbox{0.67\linewidth}{\parskip6pt\bfseries
        The `Indian Blanket' background graphic is inserted with the
        \textsf{graphicx} package, not by \textsf{graphicxsp}.  We have no
        transparency, of course, but it still looks pretty swave!
    }%
}
\end{center}

\newpage

\null\vskip-\baselineskip\vfil

\begin{center}
\setlength{\fboxsep}{10bp}%
\graphicxbox[name=indianblanket,transparency={/ca .7 /BM/Normal}]{}
{%
    \parbox{0.67\linewidth}{\parskip6pt\bfseries
        Same `Indian Blanket' graphic as the previous page, but
        using \textsf{graphicxsp}, with transparency!
        Cool

        Go Indians!
    }%
}
\end{center}


\newpage

\null\vskip-\baselineskip
\vfil

\begin{center}
\setlength{\fboxrule}{0bp}\setlength{\fboxsep}{4bp}%
\graphicxbox[name=blueBG,transparency={/ca .5 /BM/Normal}]{}
{%
    \parbox[c]{0.67\linewidth}{%
    \setlength{\fboxrule}{0bp}\setlength{\fboxsep}{10bp}%
        \graphicxbox[name=whiteBG,transparency={/ca .7 /BM/Normal}]{}
        {%
            \begin{minipage}{\linewidth-2\fboxsep}\parskip6pt\bfseries
            Someone asked me if the border can be made to be transparent. On first blush,
            I said ``No! Not at this time.'' The latter phrase I throw in to cover
            myself in case the answer is ``Yes!''
            \end{minipage}
        }
    }
}
\end{center}


\end{document}
