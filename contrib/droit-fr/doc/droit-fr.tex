\documentclass{ltxdockit}
\usepackage{btxdockit}
\usepackage[utf8]{inputenc}
\usepackage[french]{babel}
\usepackage[babel]{csquotes}
\usepackage{tabularx}
\usepackage{longtable}
\usepackage{booktabs}
\usepackage{shortvrb}
\usepackage{pifont}

\MakeOuterQuote{"}
\MakeAutoQuote*{<}{>}
\MakeShortVerb{\|}

\rcsid{$Id: droit-fr.tex,v 1.1 2016/20/02 16:00:00 ydsp stable $}

\newcommand*{\biber}{Biber\xspace}
\newcommand*{\biblatex}{Biblatex\xspace}

\newcommand*{\classname}{droit-fr\xspace}
\newcommand*{\bibstylename}{droit-fr\xspace}

\makeatletter
\renewenvironment*{typelist}
  {\list{}{%
     \setlength{\labelwidth}{\marglistwidth}%
     \setlength{\labelsep}{\marglistsep}%
     \setlength{\leftmargin}{0pt}%
     \renewcommand*{\makelabel}[1]{\hss\marglistfont##1}}%
   \def\typeitem##1{%
     \item[{##1}]%
     \ltd@pdfbookmark{##1}{##1}}%
   \def\reqitem{%
     \par\begingroup\itemsep0.5\itemsep\item
     Requis: \btd@fieldloop}
   \def\optitem{%
     \par\begingroup\itemsep0.5\itemsep\item
     Optionnel: \btd@fieldloop}}
  {\endlist}

\renewenvironment*{fieldlist}
  {\list{}{%
     \setlength{\labelwidth}{\marglistwidth}%
     \setlength{\labelsep}{\marglistsep}%
     \setlength{\leftmargin}{0pt}%
     \renewcommand*{\makelabel}[1]{\hss\marglistfont##1}}%
   \def\fielditem##1##2{%
     \item[##1]%
     \ltd@pdfbookmark{##1}{##1}%
     champ (##2)\par\nobreak
     \vspace{\itemsep}}%
   \def\listitem##1##2{%
     \item[##1]%
     \ltd@pdfbookmark{##1}{##1}%
     liste (##2)\par\nobreak
     \vspace{\itemsep}}}
  {\endlist}
\makeatother

\titlepage{%
  title={\LaTeX\ appliqué au droit français},
  subtitle={Des outils adaptés à la rédaction de thèses en droit},
  author={Yves de Saint-Pern},
  email={yves.desaintpern@gmail.com},
  revision={\rcsrevision},
  date={\rcstoday}}

\hypersetup{%
  pdftitle={\latex appliqué au droit français},
%  pdfsubject={},
  pdfauthor={Yves de Saint-Pern}}

% tables

\newcolumntype{H}{>{\sffamily\bfseries\spotcolor}l}
\newcolumntype{L}{>{\raggedright\let\\=\tabularnewline}p}
\newcolumntype{R}{>{\raggedleft\let\\=\tabularnewline}p}
\newcolumntype{C}{>{\centering\let\\=\tabularnewline}p}
\newcolumntype{V}{>{\raggedright\let\\=\tabularnewline\ttfamily}p}

\newcommand*{\sorttablesetup}{%
  \tablesetup
  \ttfamily
  \def\new{\makebox[1.25em][r]{\ensuremath\rightarrow}\,}%
  \def\alt{\par\makebox[1.25em][r]{\ensuremath\hookrightarrow}\,}%
  \def\note##1{\textrm{##1}}}

\newcommand{\tickmarkyes}{\Pisymbol{psy}{183}}
\newcommand{\tickmarkno}{\textendash}
\providecommand*{\textln}[1]{#1}
\providecommand*{\lnstyle}{}

% markup and misc

\setcounter{secnumdepth}{4}

\makeatletter

\newenvironment{nameparts}
  {\trivlist\item
   \tabular{@{}ll@{}}}
  {\endtabular\endtrivlist}

\newenvironment{namedelims}
  {\trivlist\item
   \tabularx{\textwidth}{@{}c@{=}l>{\raggedright\let\\=\tabularnewline}X@{}}}
  {\endtabularx\endtrivlist}

\newenvironment{namesample}
  {\def\delim##1##2{\@delim{##1}{\normalfont\tiny\bfseries##2}}%
   \def\@delim##1##2{{%
     \setbox\@tempboxa\hbox{##1}%
     \@tempdima=\wd\@tempboxa
     \wd\@tempboxa=\z@
     \box\@tempboxa
     \begingroup\spotcolor
     \setbox\@tempboxa\hb@xt@\@tempdima{\hss##2\hss}%
     \vrule\lower1.25ex\box\@tempboxa
     \endgroup}}%
   \ttfamily\trivlist
   \setlength\itemsep{0.5\baselineskip}}
  {\endtrivlist}

\makeatother

\begin{document}

\printtitlepage
\tableofcontents
\listoftables

\section{Introduction}
\label{int}

\subsection{Des limites de l'informatique en droit français}

Beaucoup d'auteurs ont déjà utilisé l'un des deux logiciels de traitement de texte les plus courants, Microsoft Word ou OpenOffice Writer. Relativement faciles à prendre en main, ces logiciels généralistes répondent à un grand nombre de besoins, mais ne répondent pas aux exigences d'ouvrages complexes et de grande taille.

Le mémoire de thèse en est un exemple. Dans ce type d'ouvrage de plusieurs centaines de pages, la bibliographie tient une place très importante et exige une grande rigueur de présentation. Face à ce défi, les traitements de texte de base précités ne sont pas d'un grand secours, si bien que de nombreux rédacteurs perdent un temps précieux à recopier manuellement des lignes de références bibliographiques pour obtenir un résultat souvent médiocre. 

Il existe des logiciels qui facilitent ce processus de manière significative. Toutefois, les codes juridiques de présentation d'une bibliographie s'avèrent exotiques: jurisprudence, lois, mélanges et autres ouvrages juridiques doivent être cités selon des règles que la plupart des logiciels en bibliographie ne prennent pas en charge. Les doctorants en droit n'ont donc d'autre choix que de consacrer une part très substantielle de leur temps à la mise en forme de leur document, et ce au détriment du fond.

\subsection{Qu'est-ce que \latex ?}

Les traitements de texte grand public proposent une interface intuitive: l’utilisateur voit s'afficher en temps réel le rendu final de son document\footnote{On parle dans ce cas d'interface WYSIWYG, acronyme de "What you see is what you get".}. 
Le traitement de texte \latex propose une approche radicalement différente: séparer le fond et la forme. La rédaction s'effectue alors en deux étapes:
\begin{enumerate}

\item l'auteur travaille d'abord sur un document "source" (d'extension \file{.tex}), dans lequel il utilise des balises (rappelant le HTML) afin de structurer son document, formater le texte, etc.

\item le rendu final est obtenu par un processus dit de "compilation" au cours duquel le contenu du fichier source est analysé puis transformé en fichier PDF (ou autre).

\end{enumerate}

Il ne faut pas se leurrer: \latex demande un temps d'adaptation. L'utilisation des commandes de mise en forme n'est pas très complexe mais déroutante pour les personnes habituées à travailler sur des logiciels comme Word, où le rendu d'un document leur est présenté immédiatement; sans compter que des erreurs peuvent être détectées à la compilation.

Toutefois, les atouts de \latex sont nombreux:

\begin{enumerate}

\item Le principe fondamental est la séparation du fond et de la forme. Pendant la rédaction, l'auteur ne se préocupe \emph{jamais} du rendu final, mais reste focalisé sur son propos ce qui permet de lui faire gagner beaucoup de temps. \latex est donc particulièrement avantageux lorsque le document a rédiger est très gros, car dans ce cas, le temps d'apprentissage du langage devient négligeable par rapport à la durée totale de rédaction.

\item la gestion automatisée de la bibliographie: citer un ouvrage est une opération très simple dans un document \latex. Les règles complexes de présentation des citations sont gérées atomatiquement au moment de la compilation.

\item d'autres outils automatiques sont proposés dans \latex: la création d'index, de glossaire, de table des matières active, de liens Internet actifs.

\item \latex est gratuit, et compatible avec Windows, Mac et Linux.

\end{enumerate}

\subsection{\latex en droit français, est-ce possible ?}

\latex a déjà fait ses preuves dans le monde académique. Il est aujourd'hui le traitement de texte de prédiléction des doctorants en sciences mais aussi de nombreuses revues scientifiques de renom. Toutefois, \latex peine à s'imposer en sciences humaines, et ce pour plusieurs raisons: la caractère non-intuitif de cet outil est probablement rebutant pour des personnes peu portées sur la technique et l'informatique en général; d'autre part, il n'existe pas de règle universelle de présentation en droit français, seulement des usages.

Certaines universités en droit autorisent maintenant la rédaction de thèse sous \latex. Toutefois, aucune feuille de style n'est proposée et les étudiants aventureux se retrouvent généralement perdus face aux contraintes techniques et au manque de formation.

Ce document a pour objectif de répondre à leur besoin, au moyen de deux extensions spécifiquement prévues pour le droit français vont être présentées dans ce document:
\begin{enumerate}
\item une classe \latex de thèse en droit (\secref{classe});
\item un style bibliographique pour les ouvrages juridiques (\secref{bibliographie}).
\end{enumerate}
Un exemple pratique utilisant ces deux extensions sera exposé dans \secref{template}. Enfin, des recommandations, astuces et notes diverses feront l'objet d'une dernière partie (\secref{astuces}).

\subsection{Comment utiliser cette documentation}

Avant d'aller plus loin, des connaissances de base en \latex sont nécessaires. La présente documentation n'a en effet pas pour but de revenir sur les fondamentaux de \latex, qui font l'objet de nombreux tutoriaux, site et spécifications abondantes sur Internet. En voici quelques-uns:
\begin{itemize}
\item Le site de l'ENS propose un tutoriel\fnurl{http://www.tuteurs.ens.fr/logiciels/latex} un peu austère mais complet et pédagogue (en français)
\item un guide officiel présenté sous la forme d'un wiki\fnurl{http://fr.wikibooks.org/wiki/LaTeX}, très bien fait. Une section \href{http://fr.wikibooks.org/wiki/LaTeX/Vade_mecum#Structuration_du_document}{vade mecum} peut servir de mémo des commandes les plus courantes.
\end{itemize}

Mais l'ouvrage introductif le plus pertinent pour les étudiants de droit est certainement \enquote{Latex appliqué aux sciences humaines}, écrit par Maieul Rouquette\fnurl{http://geekographie.maieul.net/Xe-LaTeX-applique-aux-sciences}. Cet ouvrage présente plusieurs avantages: il est particulièrement adapté, comme son titre l'indique, à des personnes qui ne viennent pas du monde des sciences dures; il est par ailleurs lisible, et très complet.

\section{\sty{\classname}, une classe \latex pour les thèses en droit}
\label{classe}

La classe \sty{\classname} est adaptée à des contraintes inhérentes aux mémoires de thèses en droit français. 
En voici quelques caractéristiques générales:
\begin{itemize}
\item Page de titre adaptable
%\item Double table des matières (sommaire et version longue)
\item En-têtes et pieds-de-pages automatiques
\item 10 niveaux hiérarchiques disponibles
\item Paragraphes numérotés (versets)
\item Index par numéro de versets
\end{itemize}

\subsection{Installation}

La classe \sty{\classname} est définie dans le fichier \file{\classname.cls}. Cette classe doit être placée dans le répertoire de travail ou bien dans un répertoire connu de la distribution \latex installée.

Les packages \latex suivants doivent également avoir été installés:
\begin{itemize}
\item \sty{memoir}: classe de laquelle \sty{\classname} est dérivée.
\item \sty{csquotes}: pour la gestion des guillemets français
\item \sty{hyperref}: support des hyperliens PDF
\item \sty{bookmark}: support des signets PDF
\item \sty{engrec}: support des caractères grecs
\end{itemize}

\subsection{Options}

Cette section liste les options spécifiques à la classe \sty{\classname}.

\begin{optionlist}

\boolitem[false]{footnotereset}

Cette option contrôle le mode de numérotation des notes de bas de page. Lorsque cette option est passée avec la valeur \texttt{false}, les notes de bas de pages sont numérotées de manière continue tout le long du document. La valeur \texttt{true} permet de renuméroter les notes à partir de 1 pour chaque nouvelle page.

\boolitem[true]{versetitle}

Cette option contrôle l'affichage des titres de versets (voir \secref{versets}). Lorsque cette option est passée avec la valeur \texttt{false}, les titres passés en paramètre de la commande \cmd{verset} sont ignorés à la compilation; la commande \cmd{verset} se borne alors à numéroter un paragraphe. La valeur \texttt{true} affiche normalement les titres, comme le font les commandes listées en section \secref{sections}.


\end{optionlist}

\subsection{Page de titre}

\begin{ltxsyntax}

\cmditem{university}{université}
\cmditem{school}{école}
\cmditem{speciality}{spécialité}
\cmditem{approvaldate}{date}

Ces commandes renseignent sur l'université, l'école doctorante, la spécialité de la thèse et la date de soutenance.

\cmditem{director}{civilité}{prénom}{nom}{poste}
\cmditem{reportera}{civilité}{prénom}{nom}{poste}
\cmditem{reporterb}{civilité}{prénom}{nom}{poste}
\cmditem{membera}{civilité}{prénom}{nom}{poste}
\cmditem{memberb}{civilité}{prénom}{nom}{poste}

Ces commandes renseignent les cinq membres du jury: le directeur de thèse, les deux rapporteurs et les deux examinateurs. 4 arguments sont à remplir, la civilité, le prénom, le nom de famille et enfin l'intitulé complet de poste.

\cmditem{maketitlepage}

Imprime la page de titre à la position courante.

\end{ltxsyntax}

\subsection{Niveaux hiérarchiques}
\label{sections}

En droit français, tout niveau hiérarchique est subdivisé de manière binaire: une partie est divisée en deux titres, chacun d'entre eux en deux chapitres et ainsi de suite. Pour cette raison, le nombre de niveaux hiérarchiques proposés par défaut dans les classes de base de \latex s'avère insuffisant pour les besoins d'une thèse en droit.

La classe \sty{memoir} (de laquelle dérive la classe \sty{\classname}) offre non seulement une nombre de niveaux  plus élevé mais aussi des outils permettant d'en créer de nouveaux, allant en-deçà du \cmd{subsubparapgraph}\footnote{Pour plus de précisions, voir la documentation de la classe memoir et plus précisément la commande \cmd{newlist}.}. Toutes les commandes de cette classe sont donc utilisables, mais leur nom peut prêter à confusion. Aussi, sont également proposés des alias francisés, plus intuitifs dans un contexte de droit français.

\begin{table}
\tablesetup
\begin{tabularx}{\textwidth}{@{}p{100pt}@{}p{100pt}@{}>{\ttfamily}X@{}}
\toprule
\multicolumn{1}{@{}H}{Commande francisée} &
\multicolumn{1}{@{}H}{Commande d'origine} &
\multicolumn{1}{@{}H}{Niveau} \\
\cmidrule(r){1-1}\cmidrule(r){2-2}\cmidrule{3-3}
\cmd{partie} & \cmd{book} & -2 \\
\cmd{titre}  & \cmd{part} & -1 \\
\cmd{chapitre}  & \cmd{chapter} & 0 \\
\cmd{section} &  \cmd{section} & 1 \\
\cmd{paragraphe}  & \cmd{subsection} & 2 \\
\cmd{souspara} &  \cmd{subsubsection} & 3 \\
\cmd{alinea} &  \cmd{paragraph} & 4 \\
\cmd{sousalinea}  & \cmd{subparagraph} & 5 \\
\cmd{point} &  \cmd{subsubparagraph} & 6 \\
\cmd{souspoint}  & -- & 7 \\
\cmd{verset}  & -- & 8 \\
\bottomrule
\end{tabularx}
\caption{Niveaux de hiérarchie disponibles}
\label{sectionlevels}
\end{table}

La convention de nommage des niveaux hiérarchiques choisie ici correspond aux usages les plus courants en droit français. Toutefois, les niveaux intermédiaires suivants ne sont pas pris en charge:
\begin{itemize}
\item les "sous-titres", entre les titres et les chapitres
\item les "sous-sections", entre les sections et les paragraphes.
\end{itemize}
Le rendu visuel de chaque niveau a été choisi de manière arbitraire.

\subsection{Versets}
\label{versets}

Il est d'usage en droit français de numéroter des blocs de paragraphe, notamment dans les thèses. Ces paragraphes numérotés seront dans la suite appelés "versets".

\begin{ltxsyntax}

\cmditem{verset}{titre}

La commande \cmd{verset} peut être considérée comme une commande de section de faible niveau, mais présente des particularités: son compteur n'est jamais remis à zéro et peut être utilisée à n'importe quel niveau hiérarchique; de plus il est possible d'en laisser le titre vide.

Un point final est automatiquement ajouté à la fin du titre du verset, si l'argument passé à la commande n'est pas déjà terminé par l'un des caractère de ponctuation suivants: point (.), point d'exclamation (!), point d'interrogation (?) ou trois petits points (\ldots). Ce mécanisme dispense donc le rédacteur de systématiquement terminer chaque libellé de verset par un tel point.

\cmditem{vref}{label}

Dans la mesure où la commande \cmd{verset} est une commande de section conventionnelle\footnote{C'est-à-dire qu'elle se fonde sur la commande interne \cmd{\\@startsection}.}, les mécanismes habituels de références croisées sont applicables aux versets \footnote{Typiquement, la commande \cmd{label} pour créer une référence croisée, commande \cmd{ref} pour créer un lien vers une référence croisée.}.

La commande \cmd{vref} propose un rendu spécial pour les versets.
\begin{itemize}
\item Lorsque le verset désigné par \texttt{label} se trouve à une page antérieure à la page courante, est imprimé le terme "supra" en italiques, suivi du numéro de verset, suivi du numéro de page auquel débute le verset.

\item Lorsque le verset désigné par \texttt{label} se trouve à une page antérieure à la page courante, est imprimé le terme "infra" en italiques, suivi du numéro de verset, suivi du numéro de page auquel débute le verset.

\item Lorsque le verset désigné par \texttt{label} se trouve à une page antérieure à la page courante, est imprimé le terme "supra", suivi du numéro de verset, suivi du numéro de page correspondant.
\end{itemize}

Supposons par exemple que l'on invoque \cmd{vref} depuis la page 100, que le paramètre de la commande désigne une référence croisée déclarée au moyen de \cmd{label} à la page 50 et à l'intérieur d'un verset ayant pour numéro 27. 
Le rendu sera alors:

\begin{quote}
\emph{supra}, n\textsuperscript{o}27, p. 50
\end{quote}

\cmditem{fvref}{label}

Une variante de la commande \cmd{vref}, qui renvoie à une note de page de page.

\end{ltxsyntax}

\subsection{En-têtes et pied-de-pages}
\label{entete}

L'en-tête des pages impaires contient le titre de la division hiérarchique courante (partie, titre, chapitre, section ou paragraphe). Depuis une subdivision de niveau inférieur (sous-paragraphe, etc), le paragraphe courant est affiché dans l'en-tête.

L'en-tête des pages paires contient le titre de l'ouvrage (passé à la commande \cmd{title}).

Le pied de page contient le numéro de la page courante centré.

À noter que l'en-tête et le pied-de-page sont désactivés dans le cas d'une page blanche.

\subsection{Index}
\label{index}

L'index de base proposé par \latex référence des termes à l'aide de numéros de page. Deux index supplémentaires, utilisant des numéros de verset, sont proposés par la classe \sty{droit-fr}.

\subsubsection{Index de base par versets}

\begin{ltxsyntax}

\cmditem{makeindexv}

Génère un fichier d'index par versets. À invoquer dans le préambule du document, similairement à la commande \cmd{makeindex} fournie par \latex.

\cmditem{indexv}{mot}

Ajoute un mot-clé au fichier d'index par versets. Cette commande est identique à la commande de base \cmd{index} fournie par \latex à ceci près que le référencement est effectué par versets et non par pages. Toutes les paramètres optionnels de la commande \cmd{index} sont applicables.

\cmditem{printindexv}

Imprime l'index complet par versets à la position courante. Bien entendu, la commande \cmd{makeindexv} doit avoir été invoquée dans le préambule du document.

\end{ltxsyntax}

\subsubsection{Index d'auteurs par versets}
\label{indexauteurs}

L'index d'auteurs est un index spécifiquement dédié, comme son nom l'indique, aux auteurs cités dans le document. Il référence des numéros de versets, comme l'index de base.

\begin{ltxsyntax}

\cmditem{makeindexa}

Génère un fichier d'index d'auteurs par versets. À invoquer dans le préambule du document, similairement à la commande \cmd{makeindex} fournie par \latex.

\cmditem{indexa}{mot}

Ajoute un mot-clé au fichier d'index d'auteurs par versets. Cette commande est identique à la commande de base \cmd{index} fournie par \latex à ceci près que le référencement est effectué par versets et non par pages. Toutes les paramètres optionnels de la commande \cmd{index} sont applicables.

Cette commande peut être utilisée manuellement dans le document. Toutefois, il est préférable de déléguer ce travail d'indexation au package biblatex moyennant l'activation de l'option \texttt{indexing=cite}, et de l'ajout d'une instruction spéciale dans le préambule du document permettant à biblatex d'utiliser en interne cette commande \cmd{indexa} (voir à ce propos le préambule du template de la section \ref{template}). L'index par auteur est ainsi entièrement constitué de tous les auteurs cités via la commande \cmd{cite}.

\cmditem{printindexa}

Imprime l'index d'auteurs à la position courante. La commande \cmd{makeindexa} doit avoir été invoquée dans le préambule du document.

\end{ltxsyntax}



%%%%%%%%%%%%%%%%%%%%%%%%%%%%%%%%%%%%%%%%%%%%%%%%%%%%%%%%%%%%%%%%%%%%%

\section{\sty{\bibstylename}, un style de bibliographie juridique}
\label{bibliographie}

\subsection{Installation}

Le style \sty{\bibstylename} est composé de deux fichiers, \file{droit-fr.bbx} et \file{droit-fr.cbx} contenant respectivement les réglages relatifs à la bibliographie et aux citations. Ces deux fichiers doivent être placés dans un répertoire reconnu par la distribution \latex installée ou bien dans le répertoire de travail courant.

Le package \biblatex est nécessaire. Il doit être invoqué dans le préambule du document de travail à l'aide de la commande \cmd{usepackage} (voir \secref{template} pour un exemple concret).

Le programme \biber doit également être installé. Il faut l'utiliser en lieu et place du vénérable \bibtex au moment de la compilation (voir \secref{compilation} pour plus détails concernant la compilation).

\textbf{Attention, il est nécessaire d'installer une version suffisamment récente de \biblatex (3.0 minimum), sous peine d'erreurs à la compilation}. De plus, la version de \biber doit être compatible avec la version de \biblatex installée . Une solution simple pour résoudre ces deux problèmes d'un coup est probablement d'installer les dernières versions proposées par le gestionnaire de packages de la distribution \latex utilisée.

\subsection{Options}
\label{biboptions}

Les options suivantes sont utilisables lors du chargement du package \biblatex au moyen de la commande \cmd{usepackage}, dans le préambule du document.

\begin{optionlist}

\optitem[permit]{citepages}{\opt{permit}, \opt{suppress}, \opt{omit}, \opt{separate}}

Cette option contrôle le traitement du champ \bibfield{pages} dans les citations. Lorsqu'une entrée avec un champ \bibfield{pages} renseigné est citée pour la première fois, et que la postnote est un numéro ou intervalle de pages, la citation peut comporter plusieurs informations de pages concurrentes.

L'option \texttt{citepages} propose différentes stratégies de présentation, déjà proposées dans d'autres styles\footnote{Les styles \texttt{verbose-ibid}, \texttt{verbose-trad} en particulier.}. Les rendus respectifs de ces stratégies vont être exposés au regard de la séquence d'exemple suivante:

\begin{ltxexample}
\cite{cle}
\cite[une note]{cle}
\cite[125]{cle}
\end{ltxexample}

Dans cet exemple, une référence bibliographique de clé \bibtex \prm{cle} est cité trois fois; 125 est la postnote et 100-150 la valeur passée au champ \bibfield{pages}.

\begin{valuelist}
\item[permit] Autorise l'affichage simultané des champs \texttt{pages}\slash \texttt{pagetotal} et de la \texttt{postnote}:

\begin{quote}
\textsc{Auteur}, \emph{titre}, p. 100-150. 

\textsc{Auteur}, \emph{titre}, p. 100-150, une note.

\textsc{Auteur}, \emph{titre}, p. 100-150, p. 125.
\end{quote}
%
\item[suppress] Supprime inconditionnellement les champs \texttt{pages}\slash \texttt{pagetotal} dans les citations, quelque soit la \texttt{postnote}.

\begin{quote}
\textsc{Auteur}, \emph{titre}. 

\textsc{Auteur}, \emph{titre}, une note.

\textsc{Auteur}, \emph{titre}, p. 125.
\end{quote}
%
\item[omit] Supprime les champs \texttt{pages}\slash
\texttt{pagetotal} seulement dans le troisième cas. Ils sont malgré tout imprimés si la \texttt{postnote} est vide ou n'est pas un chiffre ni un intervalle (voir \secref{use_cav_pag} à ce sujet):

\begin{quote}
\textsc{Auteur}, \emph{titre}, p. 100-150. 

\textsc{Auteur}, \emph{titre}, p. 100-150, une note.

\textsc{Auteur}, \emph{titre}, p. 125.
\end{quote}
%
\item[separate] Sépare les champs \texttt{pages}\slash
\texttt{pagetotal} de la \texttt{postnote} dans le troisième cas:

\begin{quote}
\textsc{Auteur}, \emph{titre}, p. 100-150. 

\textsc{Auteur}, \emph{titre}, p. 100-150, une note.

\textsc{Auteur}, \emph{titre}, p. 100-150, spéc. p. 125.
\end{quote}
%
On notera que le mot séparateur \enquote{spéc.} est spécifié dans la macro \texttt{thiscite}, qui peut être modifiée.
\end{valuelist}

\optitem[parens]{namestyle}{\opt{parens}, \opt{parensprefix}, \opt{plain}}

Cette option contrôle le rendu des prénoms et des particules de nom de famille dans la bibliographie.

\begin{valuelist}
\item[parens] Valeur par défaut. Le prénom et la particule sont groupés ensemble et mis entre parenthèses après le nom de famille, comme suit:
\begin{quote}
\textsc{Saint-Pern} (Yves de) 
\end{quote}

\item[parensprefix] La particule est positionnée juste après le nom de famille, entre parenthèses; le prénom est positionné après la particule, sans parenthèses, comme suit:
\begin{quote}
\textsc{Saint-Pern} (de), Yves 
\end{quote}

\item[plain] Le prénom et la particule sont groupés ensemble et séparés du nom de famille par une virgule, comme suit:
\begin{quote}
\textsc{Saint-Pern}, Yves de 
\end{quote}
\end{valuelist}

\boolitem[false]{reverseauthors}

Cette option contrôle le rendu du champ \bibfield{author} dans la bibliographie. Lorsque l'option vaut \texttt{true}, alors tous les auteurs listés dans ce champ sont rendus de la même manière, avec le nom de famille en premier suivi du prénom entres parenthèses, comme suit:
\begin{quote}
\textsc{Malaurie} (Philippe), \textsc{Aynès} (Laurent) et \textsc{Stoffel-Munck} (Philippe) 
\end{quote}

Lorsque cette option est positionnée à \texttt{false}, qui est la valeur par défaut, seul le premier auteur est imprimé avec son nom de famille suivi du prénom entre parenthèses; tout auteur supplémentaire est ensuite imprimé avec le prénom suivi du nom de famille, comme suit:
 
\begin{quote}
\textsc{Malaurie} (Philippe), Laurent \textsc{Aynès} et Philippe \textsc{Stoffel-Munck} 
\end{quote}

\boolitem[true]{smartinitials}
Cette option contrôle le mécanisme d'abréviation des prénoms dans les citations. Sont affectés les prénoms contenus dans les champs \bibfield{author} ou \bibfield{editor}. 

Lorsque cette option vaut \texttt{false}, tout prénom est abrégé en son initiale.

Lorsque cette option vaut \texttt{true}, le nombre de lettres du prénom abrégé dépend des premières lettres de ce prénom, conformément à l'usage en France:
\begin{itemize}
\item tout prénom commençant par un digramme (c'est-à-dire deux consonnes formant un seul son) est abrégé en ce digramme. Par exemple, \enquote{Charles} est abrégé en \enquote{Ch.}.
\item tout prénom commençant par un consonne ou un digramme, puis par l ou r, est abrégé en ces premières lettres. Par exemple, \enquote{Claire} est abrégé en \enquote{Cl.}; \enquote{Christian} est abrégé en \enquote{Chr.}.
\item tout autre prénom est abrégé en son initiale.
\end{itemize}

\boolitem[true]{indexnameformat}

Contrôle le rendu des noms de famille dans un index d'auteurs. Lorsque cette option vaut \texttt{true}, chaque nom de famille est imprimé en petites majuscules dans l'index; lorsqu'elle vaut \texttt{false}, chaque nom de famille est imprimé tel quel.

\end{optionlist}

\subsection{Base de données bibliographiques}

On abordera dans cette section les nouveautés apportées par le style \sty{\classname} dans de la base de données bibliographiques: 
\begin{itemize}
\item les types de références (\secref{bib_typ}).
\item les champs affectables à chaque type (\secref{bib_fld}).
\item les chaînes de caractères (\secref{bib_str}).
\item les schémas de tri de réferences (\secref{tri}).
\end{itemize}

\subsubsection{Types de références}
\label{bib_typ}

Les types de références bibliographiques suivants sont décrits dans cette section:
\begin{enumerate}
\item les types \enquote{classiques}, initialement pris en charge par le package \biblatex, mais modifiés par le style bibliographique \sty{\classname} dans le but de répondre à certains usages particuliers en droit, et
\item les types juridiques créés spécifiquement par le style \sty{\classname}.
\end{enumerate}

Tous les autres types de sources pris en charge par le package \biblatex et non modifiés par le style \sty{\classname} ne seront pas abordés ici, étant abondamment décrits dans la documentation de ce package.

Le tableau \ref{bibtypes} récapitule les types à utiliser, en fonction de la référénce.

\begin{table}
\tablesetup
\begin{tabularx}{\textwidth}{@{}@{}p{200pt}>{\ttfamily}X@{}}
\toprule
\multicolumn{1}{@{}H}{Référence} &
\multicolumn{1}{@{}H}{Type \biblatex à utiliser} \\
\cmidrule(r){1-1}\cmidrule(r){2-2}
Ouvrages généraux, monographies & \bibtype{book} \\
Ouvrages spéciaux & \bibtype{book} \\
Articles de périodique & \bibtype{article} \\
Mélanges & \bibtype{collection} \\
Contributions à un ouvrage collectif (mélange) & \bibtype{incollection} \\
Articles de colloque & \bibtype{inproceedings} \\
Conférences, colloques & \bibtype{proceedings} \\
Thèses & \bibtype{thesis} \\
Rapports officiels & \bibtype{report} \\
Lois, traités & \bibtype{legislation} \\
Commentaires d'arrêt/de décision & \bibtype{commentary} \\
Arrêts, décisions contenant des éventuels commentaires & \bibtype{jurisdiction} \\

\bottomrule
\end{tabularx}
\caption{Détermination des types à utiliser}
\label{bibtypes}
\end{table}


\paragraph{Types classiques modifiés}
\label{typesclassiques}

\begin{typelist}

\typeitem{article}

Un article de journal, quotidien, ou tout autre périodique. Le nom du périodique est identifié par le champ \bibfield{journaltitle}. La seule modification apportée à ce type est la prise en charge des champs \bibfield{part} et \bibfield{chapter} pouvant être uilisés pour désigner des sections thématiques de périodiques tels que le Recueil Dalloz ou le JCP.

\reqitem{author, title, journaltitle, year/date}
\optitem{translator, annotator, commentator, subtitle, titleaddon, editor, editora, editorb, editorc, journalsubtitle, issuetitle, issuesubtitle, language, origlanguage, series, volume, part, chapter, number, eid, issue, month, pages, version, note, issn, addendum, pubstate, doi, eprint, eprintclass, eprinttype, url, urldate}

\typeitem{report}

Un rapport officiel publié par une institution.

\reqitem{author, title, type, institution, date}
\optitem{subtitle, language, editor, number, version, note, location, month, isrn, chapter, pages, pagetotal, addendum, pubstate, doi, url, urldate}

\typeitem{thesis}

Un mémoire de thèse écrit au sein d'une université. Le champ \bibfield{editor} renseigne le nom du directeur de thèse, \bibfield{institution} celui de l'université.

\reqitem{author, title, type, institution, date}
\optitem{subtitle, editor, note, location, month, isbn, chapter, pages, pagetotal, pubstate, doi, url}

\end{typelist}

\paragraph{Types juridiques}
\label{typesjuridiques}

\begin{typelist}

\typeitem{jurisdiction}

Un arrêt de jurisprudence rendu par une institution. La date de décision est renseignée par le champ \bibfield{eventdate}, celle de publication dans un support officiel dans \bibfield{date}. Le champ \bibfield{number} désigne le numéro d'affaire.
\reqitem{institution, date}
\optitem{title, subtitle, eventdate, number, journaltitle, volume, chapter, pages}

\typeitem{commentary}

Un commentaire d'arrêt de jurisprudence publié dans un journal. Cette entrée ne contient pas de champ \bibfield{author}; le rédacteur du commentaire est à consigner dans le champ \bibfield{editor} et son rôle dans \bibfield{editortype}. Voir section \secref{roles} pour plus de détails sur les rôles disponibles.
\reqitem{editor, journaltitle, date}
\optitem{title, subtitle, editora, editorb, editorc, journalsubtitle, series, volume, number, eid, month, pages, issn, doi, url}

\typeitem{legislation}

Une loi, un traité ou un code juridique. Ce type d'entrée de comporte pas de champ \bibfield{author}, mais éventuellement un \bibfield{editor}. On veillera à entrer le nom complet de la loi dans le champ \bibfield{titre}. Attention, le champ \bibfield{number} se rapporte à \bibfield{series}, il ne doit pas être utilisé pour consigner le numéro de la loi.
\reqitem{title, date}
\optitem{subtitle, editor, editora, editorb, editorc, journaltitle, journalsubtitle, series, volume, number, eid, month, pages, issn, doi, url}

\end{typelist}

\subsubsection{Champs}
\label{bib_fld}

Ne sont listés ici les champs modifiés ou ajoutés par le style \sty{\classname}. Pour obtenir des informations sur l'ensemble des champs proposés par \biblatex, voir la documentation de ce package.

\begin{fieldlist}
\listitem{commentator}{nom}

Le ou les auteur(s) d'un commentaire d'un ouvrage. Voir aussi \bibfield{annotator}.

\fielditem{editortype}{clé}

Le rôle rédactionnel de l'\bibfield{editor}. Lorsque ce champ est laissé vide, le rôle "\texttt{editor}" est appliqué par défaut. Ce champ n'a pas été modifié en tant que tel par le style \bibstylename, mais de nouveaux rôles ont été ajoutés. Voir \secref{roles} pour plus de détails.

\fielditem{eventdate}{date}

La date d'une conférence, colloque ou autre événement pour les types \bibtype{proceedings} et \bibtype{inproceedings}; la date de décision rendue par une juridiction pour le type \bibtype{jurisdiction}. Voir \secref{dates} pour plus de détails sur les formats autorisés.

\listitem{institution}{littéral}

Le nom d'une institution: université pour le type \bibtype{thesis}, juridiction pour le type \bibtype{jurisdiction}.

\fielditem{number}{littéral}

Le numéro d'un journal ou le numéro de volume d'un livre faisant partie d'une collection (\bibfield{series}). Pour le type \bibtype{jurisdiction}, ce champ désigne le numéro de l'affaire. Ce champ ne doit pas être confondu avec une information de pagination, ni être utilisé pour désigner une section logique d'un ouvrage (voir à ce propos les champs \bibfield{part} et \bibfield{chapter}).

\fielditem{related}{clé1, clé2, ...}

Ce champ est spécial. Il est particulièrement destiné à être utilisé au sein d'une entrée de type \bibtype{jurisdiction}, et à contenir une clé ou plusieurs clés d'entrées de type \bibtype{commentary}. Tous les commentaires seront alors imprimés à la suite de l'arrêt. Voir \ref{templatejurisprudence} pour un exemple d'utilisation.

\end{fieldlist}

\subsubsection{Chaînes de caractères}
\label{bib_str}

Les chaînes de caractères suivantes (appelées "localization strings" dans la documentation \biblatex) sont définies par le style \sty{\classname}.

\paragraph{Rôles rédactionnels}
\label{stringsroles}

Les rôles suivants sont destinés aux champs \bibfield{editortype}, \bibfield{editoratype}, \bibfield{editorbtype}, etc.
\begin{keymarglist}
\item[editor] "sous la direction de", abrégé en "sous la dir. de". Notamment utilisé dans les types \bibtype{incollection}, \bibtype{inproceedings} et \bibtype{thesis}.
\item[annotator] "note"; cette chaîne est tout particulièrement dédiée au type \bibtype{commentary}.
\item[observator] "obs.": cette chaîne est tout particulièrement dédiée au type \bibtype{commentary}.
\item[commentator] "comm."; Cette chaîne est tout particulièrement dédiée au type \bibtype{commentary}.
\item[chronicler] "chron."; Cette chaîne est tout particulièrement dédiée au type \bibtype{commentary}.
\item[redactor] "rapp."; Cette chaîne est tout particulièrement dédiée au type \bibtype{commentary}.
\end{keymarglist}

\paragraph{Autres chaînes}
\label{stringsother}

\begin{keymarglist}
\item[in] "in". Précède le titre d'un ouvrage collectif dans les types \bibtype{inproceedings} et \bibtype{incollection}.
\item[thiscite] "spéc.". Voir l'option de package \texttt{citepages} décrite en section \secref{biboptions}.
\end{keymarglist}

\subsubsection{Schémas de tri de références}
\label{tri}

Le schéma de tri des références bibliographiques actif par défaut dans le style bibliographique \sty{\bibstylename} est \texttt{nyt}: les références sont triées par nom d'auteur, puis par année, enfin par titre.

Toutefois, des schémas de tri complémentaires sont fournis avec ce style bibliographique:
\begin{keymarglist}
\item[iymd] Tri par \bibfield{institution}, puis par \bibfield{eventdate}\footnote{Idem.}. Ce schéma est à utiliser pour le classement de la plupart des références jurisprudentielles, dans lesquelles un nom d'affaire n'est pas systématique, et ne doit donc pas intervenir dans le tri. Ce schéma est particulièrement adapté pour les références de jurisprudence française.
\item[tymdi] Tri par \bibfield{title}, puis par \bibfield{eventdate}\footnote{Les dates sont triées par années, puis mois, puis jour.}, puis par \bibfield{institution}. Ce schéma peut êtr utilisé pour les références jurisprudentielles pour lesquelles un nom d'affaire est systématiquement fourni (par exemple, les arrêts de la CEDH).

\end{keymarglist}

\textbf{Attention, ces nouveaux schémas de tri ne doivent pas être sélectionnés de façon globale}, au moyen de l'option \texttt{sorting} du package \biblatex. 

Ils sont destinés à contrôler le tri d'entrées dans des sous-bibliographies ne comprenant que les types juridiques listés en section \secref{typesjuridiques}. Ces schémas de tris doivent par conséquent être utilisés comme paramètre de la commande \cmd{printbibliography}, par exemple de la façon suivante:

\begin{ltxexample}
\printbibliography[..., type=<<jurisdiction>>, sorting=<<iymd>>]
\end{ltxexample}
%

\subsection{Rendu}

Il n'existe aucun standard officiel pour la présentation des bibliographies en droit français, seulement des recommandations.

Le style bibliographique \texttt{\bibstylename} présenté dans cette section ne prétend pas faire office de norme, mais s'efforce de reproduire des usages rencontrés fréquemment en droit français.

\subsubsection{Bibliographie}

\begin{ltxsyntax}

\cmditem{printbibliography}

Cette commande génère la bibliographie.

Les noms de familles sont en petites majuscules, les prénoms en minuscules. Le premier auteur d'une référence bibliographique est affiché selon un ordre inversé (nom de famille d'abord, puis prénom entre parenthèses), afin de mettre en évidence un classement par nom de famille. Si la référence des auteurs supplémentaires, ceux-ci sont affichés selon un ordre normal (prénom puis nom de famille).

Par défaut, les réference sont classées par nom d'auteur (voir \secref{tri} pour plus de détails).
Si plusieurs ouvrages d'un même auteur sont présents dans la bibliographie, le nom de l'auteur est remplacé par un tiret long (---) à partir de la deuxième référence. Une exception à cette règle est lorsque la référence est la première sur une nouvelle double page; dans ce cas, le nom d'auteur est tout de même affiché pour plus de clarté, même si au moins une référence précédente est de même auteur.

Les titres sont:
\begin{itemize}
\item au format normal pour les textes juridiques de référence (lois, codes)
\item en italique pour les autres types de références.
\end{itemize}

\end{ltxsyntax}

\subsubsection{Citations}

Les citations effectuées dans le corps du texte au moyen de la commande \cmd{cite} et de ses variantes, sont renvoyées en bas de page comme le veut l'usage. 

\begin{ltxsyntax}

\cmditem{cite}[prenote][postnote]{clé bibtex}

Voir la documentation du package \biblatex pour plus de précisions sur les arguments de la commande.

Lorsque la commande est invoquée depuis le corps du texte, celle-ci crée une citation dans une note de bas de page. Un point est automatiquement ajouté à la fin de la note de bas de page générée.

La commande peut également être invoquée depuis une note de bas de page (c'est-à-dire depuis l'intérieur d'une commande \cmd{footnote}). Dans ce cas, la commande \cmd{cite} ne fait que générer les données relatives au document cité, sans ajout de point, qu'il faut ajouter manuellement si nécessaire.

Les noms de famille sont en petites majuscules.

Les prénoms sont abrégés\footnote{À cet égard, l'option \texttt{smartinitials} permet de choisir le mécanisme d'abréviation des prénoms. Voir section \ref{biboptions}.}.

Les titres sont:
\begin{itemize}
\item au format normal pour les textes juridiques de référence (lois, codes),
\item entre guillemets pour les ouvrages \enquote{contributifs} (articles de journal, de mélange ou de colloque),
\item en italiques pour les autres types de références.
\end{itemize}

Le style bibliographique \sty{\bibstylename} remplace automatiquement des citations répétées par l'abréviation \textit{ibid.} à moins que la citation soit la première sur la page courante (ou double page courante si le document est en recto-verso).
\footnote{
Certains ouvrages de thèse utilisent d'autres locutions latines telles que \textit{opere citato} ou \textit{locare citato} lorsque qu'un ouvrage est cité plusieurs fois de suite, probablement dans le but d'alléger les citations en note de bas de page. Toutefois, les locutions \textit{opere citato} et \textit{locare citato} n'ont pas toujours le même sens et s'appliquent mal aux références sans auteur. Il a donc été choisi de ne pas les prendre en charge dans le le présent style.
}.

\cmditem{cites}[prenote][postnote]{clé 1}[prenote][postnote]{clé 2}\ldots

Voir la documentation du package \biblatex pour plus de précisions sur les arguments de la commande.

Lorsque la commande est invoquée depuis le corps du texte, celle-ci crée un groupe de citations dans une note de bas de page. Les différentes citations de ce groupe sont séparées par un point-virgule. Un point est automatiquement ajouté à la fin de la note de bas de page générée.

La commande peut également être invoquée depuis une note de bas de page (c'est-à-dire depuis l'intérieur d'une commande \cmd{footnote}). Dans ce cas, la commande \cmd{cites} ne fait que générer les données relatives aux documents cités, sans ajout de point, qu'il faut ajouter manuellement si nécessaire.

\end{ltxsyntax}

%%%%%%%%%%%%%%%%%%%%%%%%%%%%%%%%%%%%%%%%%%%%

\section{Exemple de thèse en droit français}
\label{template}

Cette partie présente une mise en application directe de la classe \sty{\classname} et du style bibliographique \sty{\bibstylename}, détaillés respectivement en sections \ref{classe} et \ref{bibliographie}, dans un document \latex pouvant servir de point de départ pour la rédaction d'une thèse en droit. Le contenu des fichiers d'exemple sera détaillé, puis les étapes de modification seront exposées.

\subsection{Description des fichiers}

\subsubsection{\file{main.tex}}

Il s'agit du fichier "ma\^itre" du projet de thèse. Tous les autres fichiers de l'exemple sont des sous-fichiers inclus dans ce document à l'aide des commandes \cmd{include} et \cmd{input}.

\paragraph{Préambule}

\begin{enumerate}

\item La classe \sty{\classname} décrite \secref{classe} est invoquée avec diverses options: format A4, taille de police 12, recto-verso. L'option \texttt{footnotereset} est spécifique à cette classe.
\begin{ltxexample}
\documentclass[a4paper,12pt,twoside,footnotereset=true]{<<droit-fr>>}
\end{ltxexample}

\item Le package \biblatex est évidemment chargé avec le style \sty{\classname} décrit \secref{bibliographie}.
% TODO: autres options ?
\begin{ltxexample}
\usepackage[%
  style=<<droit-fr>>, % nom du fichier de style
  backend=<<biber>> % utilisation de Biber a la place de BibTeX
  indexing=<<cite>> % activation de l'indexation des auteurs
]{biblatex}
\end{ltxexample}

\item La commande \cmd{hypersetup} est destinée à la configuration du package \sty{hyperref}, indispensable à la création de liens actifs lors de la compilation au format PDF. De la même manière, la commande \cmd{bookmarksetup} configure le package \sty{bookmark}, un complément du précédent pour l'affichage optimal des signets PDF.
\begin{ltxexample}
<<\hypersetup>>{%
  pdftitle={Le titre de these},
  pdfauthor={Prenom Nom}
}

<<\bookmarksetup>>{numbered=true, depth=5}
\end{ltxexample}

\item Les commandes \cmd{makeindexa}, \cmd{makeindexv} et \cmd{makeglossary} sont également présentes pour activer la génération des tables correspondantes.
\begin{ltxexample}
<<\makeglossary>> % fichier genere: .gls
<<\makeindexv>> % fichier genere: .idx
<<\makeindexa>> % fichier genere: .idx
\end{ltxexample}

La commande suivante est très importante pour permettre la formation d'un index d'auteurs séparé de l'index de base. Elle indique au paquet biblatex d'utiliser la commande \cmd{indexa} pour indexer les auteurs (voir notamment la section \ref{indexauteurs} pour plus d'informations sur l'index par auteurs).
\begin{ltxexample}
\DeclareIndexNameFormat{default}{%
  \usebibmacro{index:name}{\indexa}{#1}{#3}{#5}{#7}}
\end{ltxexample}

\item Enfin, les fichiers contenant les références bibliographiques sont chargés à l'aide de la commande \cmd{bibliography}.
\begin{ltxexample}
<<\addbibresource>>{journaux.bib}
<<\addbibresource>>{bibliographie.bib}
\end{ltxexample}

\end{enumerate}

\paragraph{Corps du document}

Le document est divisé en trois blocs\footnote{Les trois commandes \cmd{frontmatter}, \cmd{mainmatter} et \cmd{backmatter} proviennent de la classe \sty{memoir} de laquelle dérive \sty{\classname}.} dans chacun desquels la numérotation des pages et des niveaux hiérarchique suit une logique spécifique.
\begin{enumerate}
\item Le premier bloc, initié par la commande \cmd{frontmatter}, numérote les pages en chiffres romains et désactive la numérotation des sections.
\begin{ltxexample}
<<\frontmatter>> % pages en chiffres romains, sections non numerotees
\pagestyle{plain} % en-tetes vides
\author{Prénom}{Nom}
\title{Mon titre de thèse}

\university{Nom de l'université}
\school{Droit international, droit européen, relations internationales et droit comparé}
\speciality{Droit privé}
\approvaldate{2 Janvier 2012}

\director{Monsieur}{Amstram}{Gram}{Professeur à l'Université Rébenthine}
\reportera{Monsieur}{Bidule}{Trucmuche}{Professeur à l'Université Lévision}
\reporterb{Monsieur}{Blabla}{Chose}{Professeur à l'Université Tanisé}
\membera{Madame}{Machin}{Chouette}{Professeur à l'Université Lorisation}
\memberb{Monsieur}{Pif}{Pouf}{Professeur à l'Université Alamenthe}

\maketitlepage

 % page de titre
\section{Avertissement}

La Faculté n’entend donner aucune approbation ni improbation aux opinions émises dans cette thèse; ces opinions doivent être considérées comme propres à leur auteur.

\cleardoublepage

\section{Remerciements}


\cleardoublepage

 % avertissement, remerciements, resume
% sommaire
\renewcommand*{\contentsname}{Sommaire}
\let\oldchangetocdepth\changetocdepth
\renewcommand*{\changetocdepth}[1]{}
\let\oldcftchapterfillnum\cftchapterfillnum
\setcounter{tocdepth}{0}% Parties / Titres / Chapitres seulement

%\cftpagenumbersoff{book}
%\cftpagenumbersoff{part}
%\cftpagenumbersoff{chapter}

\renewcommand{\tocheadstart}{}

\renewcommand{\cftbeforebookskip}{1em}
%\renewcommand{\cftbookfont}{}
\renewcommand{\cftbookindent}{0em}
%\renewcommand{\cftbooknumwidth}{}
\renewcommand{\cftbookpagefont}{\normalfont\bfseries\large}
%\renewcommand{\cftafterbookskip}{}
%\renewcommand{\cftbookleader}{\cftdotfill{\cftdotsep}}

\renewcommand{\cftbeforepartskip}{0.5em}
\renewcommand{\cftpartfont}{\normalfont}
\renewcommand{\cftpartindent}{0.5em}
%\renewcommand{\cftpartnumwidth}{}
\renewcommand{\cftpartpagefont}{\normalfont\scshape}
%\renewcommand{\cftpartleader}{\cftdotfill{\cftdotsep}}

\renewcommand{\cftbeforechapterskip}{0em}
\renewcommand{\cftchapterfont}{\normalfont}
\renewcommand{\cftchaptername}{Chapitre\space}
\renewcommand{\cftchapterindent}{1em}
%\renewcommand{\cftchapternumwidth}{}
\renewcommand{\cftchapterpagefont}{\normalfont}
%\renewcommand{\cftchapterleader}{\cftdotfill{\cftdotsep}}

\clearpage
\tableofcontents

\newacronym{ibid}{ibid.}{ibidem}
\newacronym{dalloz}{D.}{Recueil Dalloz}
\newacronym{cedh}{CEDH}{Cour Européenne des Droits de l'Homme}
\newacronym{rtdciv}{RTD civ.}{Revue Trimestrielle de droit civil}
\newacronym{jcpg}{JCP G.}{Jurisclasseur Périodique, édition Générale}
\newacronym{defrenois}{Defrénois}{Répertoire Defrénois}
\newacronym{alinea}{al.}{alinéa}
\newacronym{assplen}{Ass. plén.}{Assemblée plénière}
\newacronym{bullciv}{Bull. civ.}{Bulletin civil des arrêts de la Cour de cassation}
\newacronym{appel}{CA}{Cour d'appel}
\newacronym{cassass}{Cass.}{Cour de cassation}
\newacronym{cciv}{C. civ.}{Code civil}
\newacronym{chreunies}{Ch. réun.}{Chambres réunies}
\newacronym{dretpatr}{Dr. et patr.}{Droit et Patrimoine}
\newacronym{fasc}{Fasc.}{Fascicule}
\newacronym{gazpal}{Gaz. Pal.}{Gazette du Palais}
\newacronym{jcpn}{JCP N.}{Jurisclasseur Périodique, édition Notariale}
\newacronym{jo}{JO}{Journal Officiel}
\newacronym{prec}{préc.}{précité}
\newacronym{puf}{PUF}{Presse Universitaire de France}
\newacronym{ridc}{RIDC}{Revue internationale de droit comparé}
\newacronym{rjpf}{RJPF}{Revue juridique personnes et famille}
\newacronym{supra}{supra}{ci-dessus}
\newacronym{sq}{sq.}{et suivants}
\newacronym{t}{t.}{tome}
\newacronym{infra}{infra}{ci-dessous}
\newacronym{IR}{IR}{Informations Rapides du recueil Dalloz}
\newacronym{ajfam}{AJ Famille}{Actualité Juridique Famille}
\newacronym{an}{AN}{Assemblée Nationale}
\newacronym{bullcrim}{Bull. crim.}{Bulletin des arrêts de la Cour de cassation (chambre criminelle)}
\newacronym{cah}{Cah.}{Cahiers}
\newacronym{ce}{CE}{Conseil d'État}
\newacronym{chap}{chap.}{chapitre}
\newacronym{v}{v.}{voir}
\newacronym{circmin}{Circ. Min.}{Circulaire Ministérielle}
\newacronym{constit}{Cons. Const.}{Conseil Constitutionnel}
\newacronym{etal}{et al.}{et autres}
\newacronym{gajciv}{GAJ civ.}{Grands Arrêts Jurisprudence civile}
\newacronym{jaf}{JAF}{Juge aux affaires familiales}
\newacronym{loi}{L.}{Loi}
\newacronym{lpa}{LPA}{Les Petites Affiches}
\newacronym{cpc}{CPC}{Code de procédure civile}
\newacronym{obs}{obs.}{observations}
\newacronym{comm}{comm.}{commentaire}
\newacronym{opcit}{op. cit.}{opere citato}
\newacronym{prop}{prop.}{proposition}
\newacronym{rapp}{rapp.}{rapporteur}
\newacronym{somm}{somm.}{sommaire}
\newacronym{vol}{vol.}{volume}
\newacronym{concl}{concl.}{conclusion}

\glsaddall
\printglossary[style=mcolindex,title=Glossaire]


\end{ltxexample}
Dans ce bloc figurent le titre, des informations diverses issues du sous-fichier \file{misc}, le sommaire et le glossaire. Les en-têtes sont laissés vides.

\item La commande \cmd{mainmatter} permet de basculer vers le mode de numérotation principal (pages en chiffres arabes, sections numérotées normalement).
\begin{ltxexample}
<<\mainmatter>> % pages en chiffres arabes, sections numerotees
\pagestyle{corpus} % en-tetes/pied-de-pages en style "corpus"
% ======================================================================
% introduction.tex
% Copyright (c) Markus Kohm, 2001-2017
%
% This file is part of the LaTeX2e KOMA-Script bundle.
%
% This work may be distributed and/or modified under the conditions of
% the LaTeX Project Public License, version 1.3c of the license.
% The latest version of this license is in
%   http://www.latex-project.org/lppl.txt
% and version 1.3c or later is part of all distributions of LaTeX 
% version 2005/12/01 or later and of this work.
%
% This work has the LPPL maintenance status "author-maintained".
%
% The Current Maintainer and author of this work is Markus Kohm.
%
% This work consists of all files listed in manifest.txt.
% ----------------------------------------------------------------------
% introduction.tex
% Copyright (c) Markus Kohm, 2001-2017
%
% Dieses Werk darf nach den Bedingungen der LaTeX Project Public Lizenz,
% Version 1.3c, verteilt und/oder veraendert werden.
% Die neuste Version dieser Lizenz ist
%   http://www.latex-project.org/lppl.txt
% und Version 1.3c ist Teil aller Verteilungen von LaTeX
% Version 2005/12/01 oder spaeter und dieses Werks.
%
% Dieses Werk hat den LPPL-Verwaltungs-Status "author-maintained"
% (allein durch den Autor verwaltet).
%
% Der Aktuelle Verwalter und Autor dieses Werkes ist Markus Kohm.
% 
% Dieses Werk besteht aus den in manifest.txt aufgefuehrten Dateien.
% ======================================================================
%
% Introduction of the KOMA-Script guide
% Maintained by Markus Kohm
%
% ----------------------------------------------------------------------
%
% Einleitung der KOMA-Script-Anleitung
% Verwaltet von Markus Kohm
%
% ======================================================================

\KOMAProvidesFile{introduction.tex}
                 [$Date: 2017-01-02 13:30:07 +0100 (Mon, 02 Jan 2017) $
                  KOMA-Script guide introduction]

\chapter{Einleitung}
\labelbase{introduction}

Dieses Kapitel enth�lt \iffree{unter anderem }{}wichtige Informationen �ber
den Aufbau \iffree{der Anleitung}{des Buches} und die Geschichte von
\KOMAScript, die Jahre vor der ersten Version beginnt. Dar�ber hinaus finden
Sie Informationen f�r den Fall, dass Sie %
\iffalse % Umbruchkorreturauslassung
 \KOMAScript{} noch nicht installiert haben, oder %
\fi%
auf Fehler sto�en.

\iffree{%
\section{Vorwort}
\seclabel{preface}

{\KOMAScript} ist ein sehr komplexes Paket (engl. \emph{bundle}). Dies
ist schon allein darin begr�ndet, dass es nicht nur aus einer einzigen
Klasse (engl. \emph{class}) oder einem einzigen Paket (engl.
\emph{package}), sondern einer Vielzahl derer besteht. Zwar sind die
Klassen als Gegenst�cke zu den Standardklassen konzipiert (siehe
\autoref{cha:maincls}), das hei�t jedoch insbesondere nicht,
dass sie nur �ber die Befehle, Umgebungen und Einstellm�glichkeiten
der Standardklassen verf�gen oder deren Aussehen als
Standardeinstellung �bernehmen.
Die F�higkeiten von {\KOMAScript} reichen
teilweise weit �ber die F�higkeiten der Standardklassen hinaus.
Manche davon sind auch als Erg�nzung zu den Grundf�higkeiten des
\LaTeX-Kerns zu betrachten.

Allein aus dem Vorgenannten ergibt sich schon zwangsl�ufig, dass die
Dokumentation zu {\KOMAScript} sehr umfangreich ausf�llt. Hinzu kommt,
dass {\KOMAScript} in der Regel nicht gelehrt wird. Das hei�t, es gibt
keinen Lehrer, der seine Sch�ler kennt und damit den Unterricht und
das Unterrichtsmaterial entsprechend w�hlen und anpassen kann. Es w�re
ein Leichtes, die Dokumentation f�r irgendeine Zielgruppe zu
verfassen.  Die Schwierigkeit, der sich der Autor gegen�ber sieht,
besteht jedoch darin, dass eine Anleitung f�r alle m�glichen
Zielgruppen ben�tigt wird. Ich habe mich bem�ht, eine Anleitung zu
erstellen, die f�r den Informatiker gleicherma�en geeignet ist wie f�r
die Sekret�rin des Fischh�ndlers. Ich habe mich bem�ht, obwohl es sich
dabei eigentlich um ein unm�gliches Unterfangen handelt. Ergebnis sind
zahlreiche Kompromisse. Ich bitte jedoch, die Problematik bei
eventuellen Beschwerden zu ber�cksichtigen und bei der Verbesserung
der derzeitigen L�sung zu helfen.

Trotz des Umfangs der Anleitung bitte ich au�erdem darum, im Falle von
Problemen zun�chst die Dokumentation zu konsultieren. Als erste Anlaufstelle
sei auf den mehrteiligen Index am Ende des \iffree{Dokuments}{Buches}
hingewiesen. \iffree{Zur Dokumentation geh�ren neben dieser Anleitung auch
  alle Text-Dokumente, die Bestandteil des Pakets sind. Sie sind in
  \File{manifest.txt} vollst�ndig aufgef�hrt}{}
}{}

\section{Dokumentaufbau}
\seclabel{structure}

Diese Anleitung ist in mehrere Teile untergliedert. Es gibt einen Teil f�r
Anwender, einen f�r fortgeschrittene Anwender und Experten und einen Anhang
mit weiterf�hrenden Informationen und Beispielen f�r diejenigen, die es ganz
genau wissen wollen.

\autoref{part:forAuthors} richtet sich dabei an alle \KOMAScript-Anwender. Das
bedeutet, dass hier auch einige Informationen f�r \LaTeX-Neulinge zu finden
sind. Insbesondere ist dieser Teil mit vielen Beispielen angereichert, die dem
reinen Anwender zur Verdeutlichung der Erkl�rungen dienen sollen. Scheuen Sie
sich nicht, diese Beispiele selbst auszuprobieren und durch Abwandlung
herauszufinden, wie \KOMAScript{} funktioniert. Trotz allem ist diese
Anleitung jedoch keine Einf�hrung in \LaTeX. \LaTeX-Neulingen seien daher
Dokumente wie \cite{l2kurz} nahe gelegt. Wiedereinsteiger aus der Zeit von
\LaTeX~2.09 sei zumindest \cite{latex:usrguide} empfohlen. Auch das Studium des
einen oder anderen Buches zu \LaTeX{} wird empfohlen.  Literaturempfehlungen
finden sich beispielsweise in \cite{DANTE:FAQ}. Der Umfang von
\cite{DANTE:FAQ} ist ebenfalls erheblich. Dennoch wird darum gebeten, das
Dokument nicht nur irgendwo vorliegen zu haben, sondern es mindestens einmal
zu lesen und bei Problemen zu konsultieren.

\autoref{part:forExperts} richtet sich an fortgeschrittene
\KOMAScript-Anwender. Das sind all diejenigen, die sich bereits mit \LaTeX{}
auskennen oder schon einige Zeit mit \KOMAScript{} gearbeitet haben und jetzt
etwas besser verstehen wollen, wie \KOMAScript{} funktioniert, wie es mit
anderen Paketen interagiert und wie man speziellere Aufgaben mit \KOMAScript{}
l�sen kann. Dazu werden die Klassenbeschreibungen aus
\autoref{part:forAuthors} in einigen Aspekten nochmals aufgegriffen und n�her
erl�utert. Dazu kommt die Dokumentation von Anweisungen, die speziell f�r
fortgeschrittene Anwender und Experten vorgesehen sind. Erg�nzt wird dies
durch die Dokumentation von Paketen, die f�r den Anwender normalerweise insofern
verborgen sind, als sie unter der Oberfl�che der Klassen und Anwenderpakete
ihre Arbeit verrichten. Diese Pakete sind ausdr�cklich auch f�r die Verwendung
durch andere Klassen- und Paketautoren vorgesehen.

Der Anhang\iffree{, der nur in der Buchfassung zu finden ist,}{} richtet sich
an diejenigen, denen all diese Informationen nicht gen�gen. Es gibt dort zum
einen Hintergrundwissen zu Fragen der Typografie, mit denen dem
fortgeschrittenen Anwender eine Grundlage f�r fundierte eigene Entscheidungen
vermittelt werden soll. Dar�ber hinaus sind dort Beispiele f�r angehende
Paketautoren zu finden. Diese Beispiele sind weniger dazu gedacht, einfach
�bernommen zu werden. Vielmehr sollen sie Wissen um Planung und Durchf�hrung
von \LaTeX-Projekten sowie einiger grundlegender \LaTeX-Anweisungen f�r
Paketautoren vermitteln.

Die Kapitel-Einteilung der Anleitung soll ebenfalls dabei
helfen, nur die Teile lesen zu m�ssen, die tats�chlich von Interesse sind. Um
dies zu erreichen, sind die Informationen zu den einzelnen Klassen und Paketen
nicht �ber das gesamte Dokument verteilt, sondern jeweils in einem Kapitel
konzentriert. Querverweise in ein anderes Kapitel sind damit in der Regel auch
Verweise auf einen anderen Teil des Gesamtpakets. Da die drei Hauptklassen in
weiten Teilen �bereinstimmen, sind sie in einem gemeinsamen Kapitel
zusammengefasst. Die Unterschiede werden deutlich hervorgehoben, soweit
sinnvoll auch durch eine entsprechende Randnotiz. Dies geschieht
beispielsweise wie hier, wenn etwas nur die Klasse
\Class{scrartcl}\OnlyAt{\Class{scrartcl}} betrifft. Nachteil dieses Vorgehens
ist, dass diejenigen, die \KOMAScript{} insgesamt kennenlernen wollen, in
einigen Kapiteln auf bereits Bekanntes sto�en werden. Vielleicht nutzen Sie
die Gelegenheit, um Ihr Wissen zu vertiefen.

Unterschiedliche Schriftarten werden auch zur Hervorhebung unterschiedlicher
Dinge verwendet. So werden die Namen von \Package{Paketen} und \Class{Klassen}
anders angezeigt als \File{Dateinamen}. \Option{Optionen}, \Macro{Anweisungen},
\Environment{Umgebungen}, \Variable{Variablen} und
\PLength{Pseudol�ngen} werden einheitlich hervorgehoben. Der
\PName{Platzhalter} f�r einen Parameter wird jedoch anders dargestellt als ein
konkreter \PValue{Wert} eines Parameters. So zeigt etwa
\Macro{begin}\Parameter{Umgebung}, wie eine Umgebung ganz allgemein
eingeleitet wird, wohingegen \Macro{begin}\PParameter{document} angibt, wie die
konkrete Umgebung \Environment{document} beginnt. Dabei ist dann
\PValue{document} ein konkreter Wert f�r den Parameter \PName{Umgebung}
der Anweisung \Macro{begin}.

\iffalse% Umbruchkorrekturtext
Damit sollten Sie nun alles wissen, um diese Anleitung lesen und verstehen zu
k�nnen. Trotzdem k�nnte es sich lohnen, den Rest dieses Kapitels gelegentlich
auch zu lesen.
\fi


\section{Die Geschichte von \KOMAScript}
\seclabel{history}

%\begin{Explain}
  Anfang der 90er~Jahre wurde Frank Neukam damit beauf"|tragt, ein
  Vorlesungsskript zu setzen. Damals war noch \LaTeX~2.09 aktuell und
  es gab keine Unterscheidung nach Klassen und Paketen, sondern alles
  waren Stile (engl. \emph{styles}). Die Standarddokumentstile
  erschienen ihm f�r ein Vorlesungsskript nicht optimal und boten auch
  nicht alle Befehle und Umgebungen, die er ben�tigte.
  
  Zur selben Zeit besch�ftigte sich Frank auch mit Fragen der
  Typografie, insbesondere mit \cite{JTsch87}. Damit stand f�r ihn
  fest, nicht nur irgendeinen Dokumentstil f�r Skripten zu erstellen,
  sondern allgemein eine Stilfamilie, die den Regeln der europ�ischen
  Typografie folgt. {\Script} war geboren.
  
  Der \KOMAScript-Autor traf auf {\Script} ungef�hr zum Jahreswechsel
  1992/""1993. Im Gegensatz zu Frank Neukam hatte er h�ufig mit Dokumenten im
  A5-Format zu tun. Zu jenem Zeitpunkt wurde A5 weder von den Standardstilen
  noch von {\Script} unterst�tzt. Daher dauerte es nicht lange, bis er erste
  Ver�nderungen an {\Script} vornahm. Diese fanden sich auch in {\ScriptII}
  wieder, das im Dezember~1993 von Frank ver�ffentlicht wurde.
  
  Mitte 1994 erschien dann \LaTeXe. Die damit einhergehenden �nderungen waren
  tiefgreifend. Daher blieb dem Anwender von {\ScriptII} nur die Entscheidung,
  sich entweder auf den Kompatibilit�tsmodus von \LaTeX{} zu beschr�nken, oder
  auf {\Script} zu verzichten. Wie viele andere wollte ich beides nicht. Also
  machte der \KOMAScript-Autor sich daran, einen \Script-Nachfolger f�r
  {\LaTeXe} zu entwickeln, der am 7.~Juli~1994 unter dem Namen {\KOMAScript}
  erschienen ist. Ich will hier nicht n�her auf die Wirren eingehen, die es um
  die offizielle Nachfolge von {\Script} gab und warum dieser neue Name
  gew�hlt wurde. Tatsache ist, dass auch aus Franks Sicht {\KOMAScript} der
  Nachfolger von {\ScriptII} ist. Zu erw�hnen ist noch, dass {\KOMAScript}
  urspr�nglich ohne Briefklasse erschienen war. Diese wurde im Dezember~1994
  von Axel Kielhorn beigesteuert. Noch etwas sp�ter erstellte Axel Sommerfeldt
  den ersten richtigen scrguide zu {\KOMAScript}.
  
  Seither ist einige Zeit vergangen. {\LaTeX} hat sich kaum ver�ndert, die
  \LaTeX-Landschaft erheblich. {\KOMAScript} wurde weiterentwickelt. Es findet
  nicht mehr allein im deutschsprachigen Raum Anwender, sondern in ganz
  Europa, Nordamerika, Australien und Asien.  Diese Anwender suchen bei
  {\KOMAScript} nicht allein nach einem typografisch ansprechenden
  Ergebnis. Zu beobachten ist vielmehr, dass bei {\KOMAScript} ein neuer
  Schwerpunkt entstanden ist: Flexibilisierung durch Variabilisierung. Unter
  diesem Schlagwort verstehe ich die M�glichkeit, in das Erscheinungsbild an
  vielen Stellen eingreifen zu k�nnen. Dies f�hrte zu vielen neuen Makros, die
  mehr schlecht als recht in die existierende Dokumentation integriert wurden.
  Irgendwann wurde es damit auch Zeit f�r eine komplett �berarbeitete
  Anleitung.
%\end{Explain}


\iffree{%
\section{Danksagung}
\seclabel{thanks}

Mein pers�nlicher Dank gilt Frank Neukam, ohne dessen \Script-Familie es
vermutlich {\KOMAScript} nie gegeben h�tte. Mein Dank gilt denjenigen, die an
der Entstehung von {\KOMAScript} und den Anleitungen mitgewirkt
haben. Dieses Mal danke ich Elke, Jana, Ben und Edoardo stellvertretend f�r
Beta-Test und Kritik. Ich hoffe, ihr macht damit noch weiter.

Ganz besonderen Dank bin ich den Gr�ndern und den Mitgliedern von DANTE,
Deutschsprachige Anwendervereinigung \TeX~e.V\kern-.18em., schuldig, durch die
letztlich die Verbreitung von \TeX{} und \LaTeX{} und allen Paketen
einschlie�lich {\KOMAScript} an einer zentralen Stelle �berhaupt erm�glicht
wird. In gleicher Weise bedanke ich mich bei den aktiven Helfern auf der
Mailingliste \texttt{\TeX-D-L} (siehe \cite{DANTE:FAQ})m in der Usenet-Gruppe
\texttt{de.comp.text.tex} und den vielen \LaTeX-Foren im Internet, die mir so
manche Antwort auf Fragen zu {\KOMAScript} abnehmen.

Mein Dank gilt aber auch allen, die mich immer wieder aufgemuntert haben,
weiter zu machen und dieses oder jenes noch besser, weniger fehlerhaft oder
schlicht zus�tzlich zu implementieren. Ganz besonders bedanke ich mich noch
einmal bei dem �u�erst gro�z�gigen Spender, der mich mit dem bisher und
vermutlich f�r alle Zeiten gr��ten Einzelbetrag f�r die bisher geleistete
Arbeit an \KOMAScript{} bedacht hat.


\section{Rechtliches}
\seclabel{legal}

{\KOMAScript} steht unter der {\LaTeX} Project Public Licence. Eine nicht
offizielle deutsche �bersetzung ist Bestandteil des \KOMAScript-Pakets. In
allen Zweifelsf�llen gilt im deutschsprachigen Raum der Text
\File{lppl-de.txt}, w�hrend in allen anderen L�ndern der Text \File{lppl.txt}
anzuwenden ist.

\iffree{F�r die Korrektheit der Anleitung, Teile der Anleitung oder einer
  anderen in diesem Paket enthaltenen Dokumentation wird keine Gew�hr
  �bernommen.}%
{Diese gedruckte Ausgabe der Anleitung ist davon und von den in den Dateien
  \File{lppl.txt} und \File{lppl-de.txt} des \KOMAScript-Pakets
  festgeschriebenden rechtlichen Bedingungen ausdr�cklich ausgenommen.}
}{}

\section{Installation}
\seclabel{installation}

Die drei wichtigsten \TeX-Distributionen, Mac\TeX, MiK\TeX{} und \TeX~Live,
stellen \KOMAScript{} �ber Ihre jeweiligen Paketverwaltungen bereit. Es wird
empfohlen, die Installation und Aktualisierung von \KOMAScript{} dar�ber
vorzunehmen. Die manuelle Installation von {\KOMAScript} ohne Verwendung der
jeweiligen Paketverwaltung wird in der Datei \File{INSTALLD.txt}, die
Bestandteil jeder \KOMAScript-Verteilung ist, beschrieben.  Beachten Sie dazu
auch die jeweilige Dokumentation zur installierten \TeX-Distribution.

Daneben gibt es auf \cite{homepage} seit einiger Zeit Installationspakete von
Zwischenversionen von \KOMAScript{} f�r die wichtigsten Distributionen. F�r
deren Installation ist die dortige Anleitung zu beachten.


\section{Fehlermeldungen, Fragen, Probleme}
\seclabel{errors}

\iffree{%
  Sollten Sie der Meinung sein, dass Sie einen Fehler in der Anleitung, einer
  der \KOMAScript-Klassen, einem der \KOMAScript-Pakete oder einem anderen
  Bestandteil von \KOMAScript{} gefunden haben, so sollten Sie wie folgt
  vorgehen. Pr�fen Sie zun�chst, ob inzwischen eine neue Version von
  \KOMAScript{} erschienen ist. Installieren Sie diese neue Version und
  kontrollieren Sie, ob der Fehler oder das Problem auch dann noch vorhanden
  ist.

  Wenn es sich nicht um einen Fehler in der Dokumentation handelt und der
  Fehler oder das Problem nach einem Update noch immer auf"|tritt, erstellen
  Sie bitte wie in \cite{DANTE:FAQ} angegeben ein minimales Beispiel. Gehen
  Sie dazu wie unten beschrieben vor.
  % Ein solches Beispiel sollte nur einen minimalen Text und nur die Pakete
  % und Definitionen enthalten, die f�r die Verdeutlichung des Fehlers
  % unbedingt notwendig sind. Auf exotische Pakete sollte m�glichst ganz
  % verzichtet werden.
  Oft l�sst sich ein Problem durch ein minimales Beispiel so weit eingrenzen,
  dass bereits vom Anwender selbst festgestellt werden kann, ob es sich um
  einen Anwendungsfehler handelt oder nicht. Auch ist so sehr h�ufig zu
  erkennen, welche Pakete oder Klassen konkret das Problem verursachen und ob
  es sich �berhaupt um ein \KOMAScript-Problem handelt. Dies k�nnen Sie
  gegebenenfalls zus�tzlich �berpr�fen, indem Sie statt einer
  \KOMAScript-Klasse einen Test mit der entsprechenden Standardklasse
  vornehmen. Danach ist dann auch klar, ob der Fehlerbericht an den Autor von
  \KOMAScript{} oder an den Autor eines anderen Pakets zu richten ist. Sie
  sollten sp�testens jetzt noch einmal gr�ndlich die Anleitungen der
  entsprechenden Paket, Klassen und \KOMAScript-Bestandteile studieren sowie
  \cite{DANTE:FAQ} konsultieren. M�glicherweise existiert ja bereits eine
  L�sung f�r Ihr Problem, so dass sich eine Fehlermeldung er�brigt.

  Wenn Sie denken, dass Sie einen noch unbekannten Fehler gefunden haben, oder
  es aus anderem Grund f�r sinnvoll oder notwendig erachten, den
  \KOMAScript-Autor zu kontaktieren, so sollten Sie dabei folgende Angaben
  keinesfalls vergessen:
  \begin{itemize}
  \item Tritt das Problem auch auf, wenn statt einer \KOMAScript-Klasse eine
    Standardklasse verwendet wird? In dem Fall liegt der Fehler h�chst
    wahrscheinlich nicht bei \KOMAScript. Es ist dann sinnvoller, die Frage in
    einem �ffentlichen Forum, einer Mailingliste oder im Usenet zu stellen.
  \item Welche \KOMAScript-Version verwenden Sie? Entsprechende Informationen
    finden Sie in der \File{log}-Datei des \LaTeX-Laufs jedes Dokuments, das
    eine \KOMAScript-Klasse verwendet.
  \item Welches Betriebssystem und welche \TeX-Distribution wird verwendet?
    Diese Angaben erscheinen bei einem bestriebssystemunabh�ngigen Paket wie
    \KOMAScript{} oder \LaTeX{} eher �berfl�ssig. Es zeigt sich aber immer
    wieder, dass sie durchaus eine Rolle spielen k�nnen.
  \item Was genau ist das Problem oder der Fehler? Beschreiben Sie das Problem
    oder den Fehler lieber zu ausf�hrlich als zu knapp. Oftmals ist es
    sinnvoll auch die Hintergr�nde zu erl�utern.
  \item Wie sieht ein vollst�ndiges Minimalbeispiel aus? Ein solches
    vollst�ndiges Minimalbeispiel kann jeder leicht selbst erstellen, indem
    Schritt f�r Schritt Inhalte und Pakete aus dem Problemdokument
    auskommentiert werden. Am Ende bleibt ein Dokument, das nur die Pakete
    l�dt und nur die Teile enth�lt, die f�r das Problem notwendig
    sind. Au�erdem sollten alle geladenen Abbildungen durch
    \Macro{rule}-Anweisungen entsprechender Gr��e ersetzt werden. Vor dem
    Verschicken entfernt man nun die auskommentierten Teile, f�gt als erste
    Zeile die Anweisung \Macro{listfiles} ein und f�hrt einen weiteren
    \LaTeX-Lauf durch. Man erh�lt dann am Ende der \File{log}-Datei eine
    �bersicht �ber die verwendeten Pakete. Das vollst�ndige Minimalbeispiel
    und die \File{log}-Datei f�gen Sie ihrer Beschreibung hinzu.
  \end{itemize}
  Schicken Sie keine Pakete, PDF- oder PS- oder DVI-Dateien mit.  Falls die
  gesamte Problem- oder Fehlerbeschreibung einschlie�lich Minimalbeispiel und
  \File{log}-Datei gr��er als ein paar Dutzend KByte ist, haben Sie mit
  gr��ter Wahrscheinlichkeit etwas falsch gemacht. Anderenfalls schicken Sie
  Ihre Mitteilung an \href{mailto:komascript@gmx.info}{komascript@gmx.info}.

  H�ufig werden Sie eine Frage zu \KOMAScript{} oder im Zusammenhang mit
  \KOMAScript{} lieber �ffentlich, beispielsweise in \texttt{de.comp.text.tex}
  oder dem Forum auf \cite{homepage}, stellen wollen. Auch in diesem Fall
  sollten Sie obige Punkte beachten, in der Regel jedoch auf die
  \File{log}-Datei verzichten. F�gen Sie stattdessen nur die Liste der Pakete
  und Paketversionen aus der \File{log}-Datei an. Im Falle einer Fehlermeldung
  zitieren Sie diese ebenfalls aus der \File{log}-Datei.%
}{%
  Der Autor hat sich gro�e M�he gegeben, Fehler in diesem Buch zu
  vermeiden. Die Beispiele, die in diesem Buch abgedruckt sind, wurden
  gr��tenteils w�hrend ihrer Entstehung getestet. Leider sind trotzdem weder
  orthografische noch inhaltliche Fehler komplett auszuschlie�en. Sollten Sie
  Fehler in diesem Buch finden, so melden Sie diese bitte �ber die
  \KOMAScript-Support-Adresse, \mbox{komascript@gmx.info}, oder �ber
  \cite{homepage} an den Autor.

  Bei Fehlern an \KOMAScript{} selbst beachten Sie bitte die Prozedur, die in
  der Einleitung der freien \KOMAScript-Anleitung erkl�rt ist. Nur so ist
  sichergestellt, dass das Problem auch reproduziert werden kann. Dies ist
  f�r die Beseitigung eventueller Fehler eine Grundvoraussetzung.%
}

Bitte beachten Sie, dass typografisch nicht optimale Voreinstellungen keine
Fehler darstellen. Aus Gr�nden der Kompatibilit�t werden Voreinstellungen nach
M�glichkeit auch in neuen \KOMAScript-Versionen beibehalten. Dar�ber hinaus
ist Typografie auch eine Frage der Sprache und Kultur. Die Voreinstellungen
von \KOMAScript{} stellen daher zwangsl�ufig einen Kompromiss dar.

\iffree{%
\section{Weitere Informationen}
\seclabel{moreinfos}

Sobald Sie im Umgang mit \KOMAScript{} ge�bt sind, werden Sie sich
m�glicherweise Beispiele zu schwierigeren Aufgaben w�nschen. Solche Beispiele
gehen �ber die Vermittlung von Grundwissen hinaus und sind daher\iffree{}{
  au�er im Angang} nicht Bestandteil dieser Anleitung. Auf den Internetseiten
des \KOMAScript{} Documentation Projects \cite{homepage} finden Sie jedoch
weiterf�hrende Beispiele. Diese sind f�r fortgeschrittene \LaTeX-Anwender
konzipiert. F�r Anf�nger sind sie wenig oder nicht geeignet.
}{}

\endinput
%%% Local Variables: 
%%% mode: latex
%%% mode: flyspell
%%% coding: iso-latin-1
%%% ispell-local-dictionary: "de_DE"
%%% TeX-master: "../guide.tex"
%%% End: 

%  LocalWords:  Installationspakete Zwischenversionen

\partie{Un nom de partie}
\label{panom}

\verset{}
\lipsum % texte de remplissage
\footnote{Note de remplissage.}

\titre{Un nom de titre}
\label{tinom}

\verset{}
\lipsum % texte de remplissage
\footnote{Note de remplissage.}

\chapitre{Un nom de chapitre}
\label{chnom}

\verset{}
\lipsum % texte de remplissage
\footnote{Note de remplissage.}

\section{Un nom de section}
\label{secnom}

\verset{}
\lipsum % texte de remplissage
\footnote{Note de remplissage.}

\paragraphe{Un \emph{nom} de paragraphe}
\label{pgnom}

\verset{}
\lipsum % texte de remplissage
\footnote{Note de remplissage.}

\souspara{Un nom de sous-paragraphe}
\label{spgnom}

\verset{}
\lipsum % texte de remplissage
\footnote{Note de remplissage.}

\alinea{Un nom de alinéa}
\label{alnom}

\verset{}
\lipsum % texte de remplissage
\footnote{Note de remplissage.}

\sousalinea{Un nom de sous-alinéa}
\label{salnom}

\verset{}
\lipsum % texte de remplissage
\footnote{Note de remplissage.}

\point{Un nom de point}
\label{ptnom}

\verset{}
\lipsum % texte de remplissage
\footnote{Note de remplissage.}

\souspoint{Un nom de sous-point}
\label{sptnom}

\verset{}
\lipsum % texte de remplissage
\footnote{Note de remplissage.}
 % premiere partie
\partie{Un autre nom de partie}

\verset{}
\lipsum % texte de remplissage

\titre{Un autre nom de titre}

\verset{}
\lipsum % texte de remplissage

\chapitre{Un autre nom de chapitre}

\verset{}
\lipsum % texte de remplissage

\section{Un autre nom de section}

\verset{}
\lipsum % texte de remplissage
 % deuxieme partie
\end{ltxexample}
Ce bloc est constitué du "corpus" principal de la thèse: introduction, première et seconde partie. On remarquera que la conclusion générale ne fait pas partie de ce bloc.

\item Enfin, \cmd{backmatter} initialise le dernier bloc du document. La numérotation des sections y est à nouveau désactivée mais les numéros de pages restent inchangés. Les en-têtes sont activées selon la logique exposée \secref{entete} gr\^ace au style de page \texttt{corpus} fourni avec la classe \sty{\classname}.
\begin{ltxexample}
<<\backmatter>> % pages en chiffres arabes, sections non numerotees
\bookmarksetup{startatroot} % RAZ du niveau des signets PDF
\section{Conclusion and future work}
\label{sec:conclusion}

Current rule-based query optimizers do not provide a very intuitive and
conceptually streamlined framework to define rules and actions.  Our
experiences with the Volcano optimizer generator suggest that its model
of rules and the expression of these rules is much more complicated and
too low-level than it needs to be.  As a consequence, rule sets in
Volcano are fragile, hard to write, and debug.  Similar problems may
exist in other contemporary rule-based query optimizers.

We believe that rule-based query optimizers will be standard tools
of future database systems.  The pragmatic difficulties of using
existing rule-based optimizers led us to develop Prairie, an
extensible and structured algebraic framework for specifying rules.
Prairie is similar to existing optimizers in that it supports both
transformation rules and implementation rules.  However, Prairie
makes several improvements:
\begin{enumerate}
\item it offers a conceptually more streamlined model for rule specification;
\item rules are encapsulated, there are no ``hidden'' operators or
      ``hidden'' algorithms;
\item implementation hints (\eg enforcers) are deduced automatically;
\item and it has efficient implementations.
\end{enumerate}

We have explained how the first three points are important for
simplifying rule specifications and making rule sets less brittle for
extensibility.  A consequence is that Prairie rules are simpler and
more robust than rules of existing optimizers (\eg Volcano).  We
addressed the fourth point by building a P2V pre-processor which uses
sophisticated algorithms to compose and compact a Prairie rule set into
a Volcano rule set.  To demonstrate the scalability of our approach, we
rewrote the TI Open OODB rule set as a Prairie rule set, generated its
Volcano counterpart, and showed that the performance of the synthesized
Volcano rule set closely matches the hand-crafted Volcano rule set.

Our future work will concentrate on developing higher-level
abstractions using Prairie, including automatically generating Prairie
rule sets, and combining multiple Prairie rule sets to automatically
generate efficient optimizers.

\section*{Acknowledgments}
\label{sec:acknowledgments}

We wish to thank Texas Instruments, Inc.\ for making the Open OODB
source code available to us.  Comments by Jos\'e Blakeley, Anne Ngu,
Vivek Singhal, Thomas Woo and the anonymous referees greatly improved
the quality of the paper.

\chapitre{Annexes}

\verset{}
\lipsum % texte de remplissage


%%% paramètres de la bibliographie %%%

% forcer l'impression de toutes les références non citées dans le texte
\nocite{*}

% titres des sections de la bibliographie
\defbibheading{france}{\section{Droit français}}
\defbibheading{europe}{\section{Droit européen}}

% titres de types de sources
\defbibheading{lois}{\paragraphe{Lois}}
\defbibheading{rapports}{\paragraphe{Rapports officiels}}
\defbibheading{jurisprudence}{\paragraphe{Jurisprudence}}
\defbibheading{generaux}{\paragraphe{Ouvrages généraux}}
\defbibheading{speciaux}{\paragraphe{Ouvrages speciaux}}
\defbibheading{theses}{\paragraphe{Thèses}}
\defbibheading{speciaux}{\paragraphe{Ouvrages spéciaux}}
\defbibheading{collectifs}{\paragraphe{Ouvrages collectifs}}
\defbibheading{articles}{\paragraphe{Articles}}

% titres pour la jurisprudence
\defbibheading{juris_ccel}{\souspara{Conseil constitutionnel}}
\defbibheading{juris_ce}{\souspara{Conseil d'État}}
\defbibheading{juris_cass}{\souspara{Cour de cassation}}
	\defbibheading{juris_cass_ass}{\alinea{Assemblée plénière}}
	\defbibheading{juris_cass_1civ}{\alinea{1\iere{} chambre civile}}
	\defbibheading{juris_cass_2civ}{\alinea{2\ieme{} chambre civile}}
	\defbibheading{juris_cass_3civ}{\alinea{3\ieme{} chambre civile}}
	\defbibheading{juris_cass_com}{\alinea{Chambre commerciale}}
	\defbibheading{juris_cass_soc}{\alinea{Chambre sociale}}
	\defbibheading{juris_cass_crim}{\alinea{Chambre criminelle}}
\defbibheading{juris_ca}{\souspara{Cours d'appel}}
\defbibheading{juris_tgi}{\souspara{Tribunaux de grande instance}}
\defbibheading{juris_ti}{\souspara{Tribunaux d'instance}}

% filtres de sélection
\defbibfilter{gen}{\(\type{book} \or \type{inbook}\) \and \not \keyword{special}}
\defbibfilter{spec}{\(\type{book} \or \type{inbook}\) \and \keyword{special}}
\defbibfilter{col}{\type{collection} \or \type{proceedings}}
\defbibfilter{art}{\type{incollection} \or \type{inproceedings} \or \type{article}}

%%% impression des différentes bibliographies %%%

\chapitre{Bibliographie}

\printbibheading[heading=france]

\printbibliography[heading=lois,type=legislation,keyword=french]
\printbibliography[heading=rapports,type=report,keyword=french]
\printbibheading[heading=jurisprudence]
	\newrefcontext[sorting=iymd]
	\printbibliography[heading=juris_ccel, type=jurisdiction, keyword=ccel]
	\printbibliography[heading=juris_ce, type=jurisdiction, keyword=ce]
	\printbibheading[heading=juris_cass]
		\printbibliography[heading=juris_cass_ass, type=jurisdiction, keyword=cassass]
		\printbibliography[heading=juris_cass_1civ, type=jurisdiction, keyword=cass1civ]
		\printbibliography[heading=juris_cass_2civ, type=jurisdiction, keyword=cass2civ]
		\printbibliography[heading=juris_cass_3civ, type=jurisdiction, keyword=cass3civ]
		\printbibliography[heading=juris_cass_com, type=jurisdiction, keyword=casscom]
		\printbibliography[heading=juris_cass_soc, type=jurisdiction, keyword=casssoc]
		\printbibliography[heading=juris_cass_crim, type=jurisdiction, keyword=casscrim]
	\printbibliography[heading=juris_ca, type=jurisdiction, keyword=ca]
	\printbibliography[heading=juris_tgi, type=jurisdiction, keyword=tgi]
	\printbibliography[heading=juris_ti, type=jurisdiction, keyword=ti]
	\endrefcontext % sorting=iymd
\printbibliography[heading=generaux, filter=gen, keyword=french]
\printbibliography[heading=theses, type=thesis, keyword=french]
\printbibliography[heading=speciaux, filter=spec, keyword=french]
\printbibliography[heading=collectifs, type=collection, keyword=french]
\printbibliography[heading=articles, filter=art, keyword=french]

\printbibheading[heading=europe]

\printbibliography[heading=lois, type=legislation, keyword=ue]
\printbibliography[heading=rapports, type=report, keyword=ue]
\newrefcontext[sorting=tymdi]
\printbibliography[heading=jurisprudence, type=jurisdiction, keyword=ue]
\endrefcontext % sorting=tymdi
\printbibliography[heading=generaux, filter=gen, keyword=ue]
\printbibliography[heading=speciaux, filter=spec, keyword=ue]
\printbibliography[heading=collectifs, filter=col, keyword=ue]
\printbibliography[heading=theses, type=thesis, keyword=ue]
\printbibliography[heading=articles, filter=art, keyword=ue]


\pagestyle{plain} % en-tetes vides
% \iffalse
% ====================================================================
%  @LaTeX-style-file{
%     filename        = "index.dtx",
%     version         = "4.02beta",
%     date            = "20 January 2004",
%     time            = "21:15:11 EST",
%     author          = "David M. Jones",
%     address         = "MIT Laboratory for Computer Science
%                        Room NE43-316
%                        545 Technology Square
%                        Cambridge, MA 02139
%                        USA",
%     telephone       = "(617) 253-5936",
%     FAX             = "(617) 253-3480",
%     checksum        = "39539 1398 6018 50774",
%     email           = "dmjones@theory.lcs.mit.edu",
%     codetable       = "ISO/ASCII",
%     keywords        = "LaTeX, index",
%     supported       = "yes",
%     docstring       = "This is a reimplementation of LaTeX's
%                        indexing macros to provide better support
%                        for indexing in LaTeX.  For example, it
%                        supports multiple indexes in a single
%                        document and provides a more robust \index
%                        command.
%
%                        The checksum field above contains a CRC-16
%                        checksum as the first value, followed by
%                        the equivalent of the standard UNIX wc
%                        (word count) utility output of lines,
%                        words, and characters.  This is produced
%                        by Robert Solovay's checksum utility.",
%
%  }
% ====================================================================
%
% This file may be distributed and/or modified under the
% conditions of the LaTeX Project Public License, either version 1.2
% of this license or (at your option) any later version.
% The latest version of this license is in
%    http://www.latex-project.org/lppl.txt
% and version 1.2 or later is part of all distributions of LaTeX
% version 1999/12/01 or later.
%
%    CAUTION: Use only as directed.  Do not take internally.  May cause
%    rash if applied directly to skin.  Federal law prohibits distributing
%    without a proscription.
%
% \fi
%
%% \CheckSum{755}
%% \CharacterTable
%%  {Upper-case    \A\B\C\D\E\F\G\H\I\J\K\L\M\N\O\P\Q\R\S\T\U\V\W\X\Y\Z
%%   Lower-case    \a\b\c\d\e\f\g\h\i\j\k\l\m\n\o\p\q\r\s\t\u\v\w\x\y\z
%%   Digits        \0\1\2\3\4\5\6\7\8\9
%%   Exclamation   \!     Double quote  \"     Hash (number) \#
%%   Dollar        \$     Percent       \%     Ampersand     \&
%%   Acute accent  \'     Left paren    \(     Right paren   \)
%%   Asterisk      \*     Plus          \+     Comma         \,
%%   Minus         \-     Point         \.     Solidus       \/
%%   Colon         \:     Semicolon     \;     Less than     \<
%%   Equals        \=     Greater than  \>     Question mark \?
%%   Commercial at \@     Left bracket  \[     Backslash     \\
%%   Right bracket \]     Circumflex    \^     Underscore    \_
%%   Grave accent  \`     Left brace    \{     Vertical bar  \|
%%   Right brace   \}     Tilde         \~}
%
%    \iffalse
%    \begin{macrocode}
%<*driver>
\ProvidesFile{index.dtx}[1995/09/28 v4.1beta Improved index support (dmj)]
%</driver>
%    \end{macrocode}
%
%    \begin{macrocode}
%<*driver>
\documentclass{ltxdoc}
\def\docdate {7 March 1994}
%    \end{macrocode}
%    We don't want everything to appear in the index
%    \begin{macrocode}
\DoNotIndex{\!,\/,\?,\@,\^,\_}
\DoNotIndex{\@@par,\@M,\@auxout,\@bsphack,\@esphack,\@depth,\@ehc}
\DoNotIndex{\@for,\@flushglue,\@gobble,\@gobbletwo,\@height,\@idxitem}
\DoNotIndex{\@ifnextchar,\@ifstar,\@ifundefined,\@input,\@latexerr}
\DoNotIndex{\@makeschapterhead,\@namedef,\@nameuse,\@nil}
\DoNotIndex{\@nobreakfalse,\@restonecolfalse,\@restonecoltrue}
\DoNotIndex{\@tempa,\@tempf,\@temptokena,\@themark,\@width}
\DoNotIndex{\active,\aindex,\baselineskip,\begin,\begingroup,\box}
\DoNotIndex{\c@page,\catcode,\chapter,\char,\chardef,\closeout}
\DoNotIndex{\CodelineIndex,\sp,\sb,\label,\leavevmode,\mark}
\DoNotIndex{\mark,\newinsert,\newwrite,\newtoks,\xdef}
\DoNotIndex{\columnsep,\columnseprule,\columnwidth,\csname,\def}
\DoNotIndex{\dimen,\do,\DocInput,\documentstyle,\edef,\em}
\DoNotIndex{\EnableCrossrefs,\end,\endcsname,\endgroup,\endinput}
\DoNotIndex{\everypar,\expandafter,\filedate,\fileversion}
\DoNotIndex{\footnotesize,\gdef,\global,\glossary,\hangindent}
\DoNotIndex{\if@filesw,\else,\fi}
\DoNotIndex{\if@nobreak,\if@twocolumn,\if@twoside,\fi,\fi,\fi}
\DoNotIndex{\hsize,\hskip}
\DoNotIndex{\ifhmode,\ifmmode,\ifodd,\ifvmode,\ifx,\fi,\fi,\fi,\fi,\fi}
\DoNotIndex{\immediate,\insert,\item,\jobname,\long}
\DoNotIndex{\let,\lineskip,\marginparsep,\marginparwidth,\maxdimen}
\DoNotIndex{\makeatletter,\noexpand,\openout,\protect,\rlap}
\DoNotIndex{\min,\newpage,\nobreak,\normalbaselineskip}
\DoNotIndex{\normallineskip,\p@,\par,\parfillskip,\parindent,\parskip}
\DoNotIndex{\penalty,\relax,\section,\sin,\sloppy,\space,\string}
\DoNotIndex{\tableofcontents,\the,\thepage,\thispagestyle,\toks,\tt}
\DoNotIndex{\twocolumn,\uppercase,\vbox,\vrule,\vskip,\vss}
\DoNotIndex{\write,\z@,\z@skip}
%    \end{macrocode}
%  Some useful macros and parameter settings:
%    \begin{macrocode}

\setcounter{StandardModuleDepth}{1}

\GetFileInfo{index.dtx}

\newcommand*{\email}[1]{$\langle$\texttt{#1}$\rangle$}
\newcommand*{\vdate}[1]{$\langle$#1$\rangle$}
\newcommand*{\bundle}[1]{\texttt{#1}}
\newcommand*{\program}[1]{\textsf{#1}}
\newcommand*{\Ltag}[1]{\texttt{\bslash#1}}
\newcommand*{\Lopt}[1]{\textsf {#1}}
\newcommand*{\Lenv}[1]{\texttt {#1}}
\newcommand*{\cls}[1]{\texttt {#1}}
\newcommand*{\pck}[1]{\texttt {#1}}
\newcommand*{\file}[1]{\texttt {#1}}
\CodelineIndex
%    \end{macrocode}
%    And the document itself:
%    \begin{macrocode}
\begin{document}
\DocInput{index.dtx}
\PrintIndex
% ^^A\PrintChanges
\end{document}
%</driver>
%    \end{macrocode}
%
%    \fi
%
%    \title{A new implementation of \LaTeX's indexing
%        commands\thanks{This file has version number \fileversion,
%        last revised \filedate, documentation dated \docdate.  The
%        definitive version of this file is at
%        \file{ftp://theory.lcs.mit.edu/pub/tex/index/}.}}
%
%    \author{David M. Jones}
%
%    \date{\filedate}
%
%    \maketitle
%
%    \section{Introduction}
%
%    This style file reimplements \LaTeX's indexing macros to provide
%    better and more robust support for indexes.  In particular, it
%    provides the following features:\footnote{Earlier versions of
%    this package provided a ``shortindexing'' feature (see below for
%    description).  This feature is now deprecated and will be removed
%    in a future release of this package.}
%    \begin{enumerate}
%
%    \item
%    Support for multiple indexes.
%
%    \item
%    Indexing of items by counters other than the page number.
%
%    \item
%    A $*$-variant of the \cs{index} command that, in addition to
%    putting it's argument in the index, also typesets it in the
%    running text.
%
%    \item
%    The \bundle{showidx} style option has been merged into this file.
%    The command \cs{proofmodetrue} can be used to enable the printing
%    of index entries in the margin of pages.  The size and style of
%    font can be controlled with the \cs{indexproofstyle} command.
%
%    \item
%    A two-stage process, similar to that used to create tables of
%    contents, for creating the raw index files.  This means that when
%    processing a portion of a document using the \cs{includeonly}
%    command, the index entries from the rest of the document are not
%    lost.
%
%    \item
%    A more robust \cs{index} command.  In particular, it no longer
%    depends on \cs{catcode} changes to work properly, so the new
%    \cs{index} command can be used in places that the original
%    couldn't, such as inside the arguments of other macros.
%
%    \end{enumerate}
%
%
%    \section{Creating an index with \LaTeX}
%
%    Conceptually, there are four stages to creating an index.  First,
%    \LaTeX\ must be informed of your intention to include an index in
%    your document.  Second, you must add appropriate markup commands
%    to your document to tell \LaTeX\ what to put in the index.
%    Third, after \LaTeX\ has been run on your document, the raw index
%    information must be processed and turned into a form that \LaTeX\
%    can process to typeset the index.  Finally, the finished index
%    must be inserted at the appropriate point in your document.
%
%    In \LaTeX, these steps are accomplished with the commands
%    \cs{makeindex}, \cs{index}, \cs{printindex}, and (typically) with
%    the auxiliary program \program{MakeIndex}.  For example, assuming
%    that your main file is called \file{foo.tex}, \cs{makeindex}
%    opens the file \file{foo.idx} and initializes it for holding the
%    raw index entries, and \cs{index} is used to add raw index
%    entries into \file{foo.idx}.  Then the raw index file is
%    processed by \program{MakeIndex}, which puts the finished index
%    in \file{foo.ind}.  Finally, the \cs{printindex} command is used
%    in your \LaTeX\ document to indicate where the file
%    \file{foo.idx} should be inserted, i.e., where the index should
%    appear in your document.
%
%    The \bundle{index} package modifies the \cs{makeindex},
%    \cs{index}, and \cs{printindex} commands, as described below.
%
%
%    \section{The user interface}
%
%    There are four pieces of information associated with each index:
%    \begin{enumerate}
%
%    \item
%    A short, unique tag that identifies the index.
%
%    \item
%    The extension of the output file where the raw index information
%    will be put by \LaTeX.
%
%    \item
%    The extension of the input file where the processed information
%    created by \program{MakeIndex} will be stored to be read in later
%    by \LaTeX.
%
%    \item
%    The title of the index.
%
%    \end{enumerate}
%
%    \DescribeMacro{\newindex}
%    Correspondingly, the \cs{newindex} command has four required
%    arguments.  For example, to declare an author index, you might
%    use the following:
%    \begin{verbatim}
%    \newindex{aut}{adx}{and}{Name Index}\end{verbatim}
%    Here, \texttt{aut} is the tag used to identify the author index,
%    and ``Name Index'' is the title of the index.  If the name of
%    your main file is \file{root.tex}, then \LaTeX\ will write the
%    raw index entries to the file \file{root.adx}, and you will
%    execute the following \program{MakeIndex} command to process the
%    author index:
%    \begin{verbatim}
%    makeindex -o root.and root.adx\end{verbatim}
%
%    By default, the \cs{index} tags its argument with the page number
%    (i.e., the value of \cs{thepage}), but occasionaly you may want
%    to index items according to a different counter.  For example,
%    you may want an index that contains figure numbers instead of
%    page numbers.  To accomodate, this, the \cs{newindex} command
%    takes an optional argument, which is the name of the command that
%    generates the number that should be included in the index.  For
%    instance, to include the number of a figure, you might say
%    \begin{verbatim}
%    \newindex[thefigure]{fig}{fdx}{fnd}{Figures}\end{verbatim}
%
%    However, this introduces a new technicality: When creating an
%    index with page numbers, the choice of which page number is to be
%    written to the \texttt{aux} file should be deferred until the
%    page containing the entry is shipped out to the \texttt{dvi}
%    file, otherwise the wrong number will sometimes be chosen.
%    However, when using counters other than the page counter, one
%    normally wants the opposite behaviour: the number written to the
%    \texttt{aux} file should be chosen immediately, otherwise every
%    item on a given page will be tagged with the number of the last
%    item on that page.  So, when a counter is specified using the
%    optional argument of \cs{newindex}, it is assumed that the
%    counter should be evaluated immediately.  If for some reason you
%    need the choice to be deferred until the page is written to the
%    \texttt{dvi} file, you can force this behaviour by putting a $*$
%    {\em after\/} the optional argument:
%    \begin{verbatim}
%    \newindex[thefigure]*{fig}{fdx}{fnd}{Figures}\end{verbatim}
%    (One consequence of this scheme is that if, for some reason, you
%    need the choice of page number to be made immediately instead of
%    being deferred until a page is shipped out to the \texttt{dvi}
%    file, you can acomplish this by beginning your index declaration
%    with
%    \begin{verbatim}
%    \newindex[thepage]*\end{verbatim}
%
%
%    \DescribeMacro{\renewindex}
%    The \cs{renewindex} command takes the same arguments as the
%    \cs{newindex} command and can be used to redefine indexes that
%    have been previously declared.
%
%
%    \DescribeMacro{\makeindex}
%    For backwards compatibility, the \cs{makeindex} command is
%    redefined to use \cs{newindex}.  It is essentially equivalent to
%    \begin{verbatim}
%    \newindex{default}{idx}{ind}{Index}\end{verbatim}
%    The index labeled \texttt{default} is special: it is the one that
%    will be used by \cs{index} and \cs{printindex} unless another
%    index is specified (see below).
%
%
%    \DescribeMacro{\printindex}
%    The \cs{printindex} command is modified by the addition of an
%    optional argument, which is the tag of the index that should be
%    printed.
%
%
%    \DescribeMacro{\index}
%    The \cs{index} command is modified in two ways.  First, there is
%    a $*$-variant of the command that, in addition to putting its
%    argument into an index, also typesets it on the page.  Second,
%    \cs{index} now takes an optional argument to indicate which index
%    the new entry should be added to.  If given, the optional
%    argument should be the identifying tag of a previously-defined
%    index.  If no such tag is supplied, the \texttt{default} index
%    (such as that opened by \cs{makeindex} above) is used.
%
%
%    \DescribeMacro{\shortindexingon}
%    \DescribeMacro{\shortindexingoff}
%    Perhaps the most dubious feature of \bundle{index.sty} is that it
%    allows you to define the characters |^| and |_| to be
%    abbreviations for \cs{index*} and \cs{index} outside of math
%    mode.  These abbreviations are enabled by the
%    \cs{shortindexingon} command and disabled by the
%    \cs{shortindexingoff} command.  The scope of both of these latter
%    commands is local to the current group.  (This might be useful,
%    for example, if you wanted the abbreviations turned on throughout
%    most of the documentation, but turned off in one particular
%    environment.)  In addition,
%    \DescribeEnv{shortindexingon}\Lenv{shortindexingon} can be used
%    as an environment if that seems appropriate.  \textbf{Warning:
%    This feature is deprecated and will disappear in a future release
%    of this package.}
%
%
%    \DescribeMacro{\proofmodetrue}
%    \DescribeMacro{\proofmodefalse}
%    \DescribeMacro{\indexproofstyle}
%    As mentioned above, the \bundle{showidx} document-style option
%    has been merged into \bundle{index.sty}.  It can be turned on
%    with \cs{proofmodetrue} and turned off with \cs{proofmodefalse}.
%    When it is turned on, all index entries\footnote{Well, most, at
%    least.  There are some circumstances under which the index
%    entries won't show up in the proofs, although they will show up
%    in the index.} will be put in the margin of the page where they
%    appear.  By default, they appear in the typewriter font at
%    \cs{footnotesize}, but the user can override this with the
%    \cs{indexproofstyle} command; for example,
%    \begin{verbatim}
%    \indexproofstyle{\footnotesize\it}\end{verbatim}
%    will cause them to be put in italics instead.
%
%
%    \DescribeMacro{\disableindex}
%    There are some circumstances where it might be helpful to
%    suppress the writing of a particular index.  The
%    \cs{disableindex} command is provided for this purpose.  It takes
%    one argument, a comma-separated list of tags of the indexes that
%    should be disabled.  This command should come {\em before\/} the
%    declarations for the indexes that are being
%    disabled\footnote{This limits its usefulness somewhat, but since
%    the output file for an index is opened when the index is
%    declared, the damage has already been done.  We could close the
%    file, but we can't prevent a new output stream from being
%    allocated and we can't keep the old file from being truncated.}.
%    One situation where the \cs{disableindex} command might be useful
%    is if there are so many indexes that you are exhausting \TeX's
%    supply of output streams\footnote{\TeX\ only has 16 output
%    streams, which are allocated with the {\tt\string\newwrite}
%    command.  The standard \LaTeX\ styles use from 3 to 7 of these,
%    which should leave room for up to 9 indexes.  Of course, if you
%    have extra output files, then there will be fewer output streams
%    left for indexes.}.  For example, suppose you have 10 indexes,
%    but only 5 output streams available for indexes.  Then you could
%    add a \cs{disableindex} command to the top of your file to
%    suppress the writing of all but 5 of the indexes.  (Note that the
%    index entries would still get written to the \texttt{aux} file;
%    they just wouldn't get copied to the individual raw index files
%    at the end of the run.)  At the end of the run, you could then
%    re-run your main file a couple of times with different indexes
%    disabled until you had created all of the raw index files.  This
%    is somewhat clumsy, but safer than any alternative I've come up
%    with\footnote{A less clumsy (for the user, at least) solution
%    would be to read the \texttt{aux} file multiple times at the end
%    of the run, each time writing just one of the raw index files.
%    The main disadvantage of this scheme at present is that it would
%    require a modification of {\tt\string\enddocument}.}.
%
%
%    \section{Caveats}
%
%    In order to implement this style file, it's been necessary to
%    modify a number of \LaTeX\ commands seemingly unrelated to
%    indexing, namely, \cs{@starttoc}, \cs{raggedbottom},
%    \cs{flushbottom}, \cs{addcontents}, \cs{markboth}, and
%    \cs{markright}.  Naturally, this could cause incompatibilities
%    between \bundle{index.sty} and any style files that either
%    redefine these same commands or make specific assumptions about
%    how they operate.  See Section~\ref{sec:thecode} for explanations
%    of why these various commands needed modification.
%
%    The redefinition of \cs{@starttoc} is particularly bad, since it
%    introduces an incompatibility with the AMS document classes.
%    This will be addressed soon.
%
%    Unfortunately, it's also been necessary to modify the
%    \Lenv{theindex} environment, so if you don't like the default
%    \LaTeX\ definition, you'll need copy the definition of
%    \Lenv{theindex} from this file and modify it appropriately.
%
%    In the current implementation, \bundle{index.sty} uses one output
%    stream for each index.  Since there are a limited number of
%    output indexes, this means that there is a limit on the number of
%    indexes you can have in a document.  See the description of
%    \cs{disableindex} for a fuller discussion of this problem and one
%    way around it.
%
%
%    \section{To do's}
%
%    It might be nice if the \cs{index*} command parsed its argument
%    so that, for example, instead of writing
%    `|\index{sin@$\sin$}$\sin$|', one could write
%    `|index*{sin@$\sin$}|'.  However, this is fraught with numerous
%    dangers, and I'm both too lazy and too cowardly to undertake it
%    now.
%
%    It would be reasonable to add support for \cs{makeglossary} and
%    similar things, if they were well-defined enough to decide what
%    the general syntax for defining them should be.
%
%    The documentation should be carefully read, edited, and finished,
%    especially since it's still based on the 2.09 version, even
%    though a few substantial changes have been made for the \LaTeXe\
%    version.
%
%    For some truly outlandish ideas, see the file \file{TODO} in the
%    distribution.
%
%    \StopEventually{}
%
%    \section{The code}
%    \label{sec:thecode}
%
%    As is customary, identify this as a \LaTeXe\ package.
%    \begin{macrocode}
%<*style>
\NeedsTeXFormat{LaTeX2e}[1995/06/01]

\ProvidesPackage{index}[2004/01/20 v4.2beta Improved index support (dmj)]
%    \end{macrocode}
%
%    \begin{macro}{\disableindex}
%    The \cs{disableindex} should come before the declarations of the
%    indexes it refers to.  (Question: If an index has been disabled,
%    should it show up in index proofs?  Maybe there should be a
%    separate command to disable index proofs on and index-by-index
%    basis.)
%    \begin{macrocode}
\def\disableindex#1{%
    \@for\@tempa:=#1\do{%
        \@namedef{disable@\@tempa}{}%
        \@ifundefined{tf@\@tempa}{}{%
            \PackageWarningNoLine{index}{It's too late to disable
                the `\@tempa' index;\MessageBreak
                \jobname.\@tempa\space has already
                been opened for output. You \MessageBreak
                should put the \string\disableindex\space command
                before\MessageBreak
                the declaration of the `\@tempa' index}%
        }%
    }%
}
%    \end{macrocode}
%    \end{macro}
%
%    \begin{macro}{\if@newindex}
%    \begin{macro}{\newindex}
%    \begin{macro}{\renewindex}
%    The \cs{newindex} and \cs{renewindex} commands are defined on
%    analogy with the \cs{[re]newcommand} macros.  Each index is
%    identified by a unique tag, which is specified in the first
%    required argument of \cs{newindex}.  Much of the information
%    about the index labeled \meta{tag} is kept in the macro
%    \cs{idx@}\meta{tag}, so we can check to see if a particular index
%    has already been defined by checking whether \cs{idx@}\meta{tag}
%    is defined.  \cs{newindex} and \cs{renewindex} both check to see
%    if their first argument is already associated with an index and
%    then either issue an appropriate error message or call
%    \cs{def@index}.
%
%    The \cs{if@newindex} flag will be used to keep \cs{renewindex}
%    from re-allocating \cs{write} and \cs{toks} registers later.  The
%    \cs{if@tempswa} switch will be used to determine whether the
%    \cs{write}s for this index should be done \cs{immediate}ly or
%    not.
%    \begin{macrocode}
\newif\if@newindex

\def\newindex{%
    \@tempswafalse
    \@ifnextchar[{\@tempswatrue\x@newindex}{\x@newindex[thepage]}%
}

\def\x@newindex[#1]{%
    \@ifstar {\@tempswafalse\y@newindex{#1}}
             {\y@newindex{#1}}%
}

\def\y@newindex#1#2{%
    \@ifundefined{idx@#2}%
        {\@newindextrue\def@index{#1}{#2}}%
        {%
            \@latexerr{Index type `\string#2' already defined}\@ehc
            \expandafter\@gobble\@gobbletwo
        }%
}

\def\renewindex{%
    \@tempswafalse
    \@ifnextchar[{\@tempswatrue\x@renewindex}{\x@renewindex[thepage]}%
}

\def\x@renewindex[#1]{%
    \@ifstar {\@tempswafalse\y@renewindex{#1}}
             {\y@renewindex{#1}}%
}

\def\y@renewindex#1#2{%
    \@ifundefined{idx@#2}%
        {%
            \@newindextrue
            \@latexerr{Index type `\string#2' not defined}\@ehc
        }%
        {\@newindexfalse}%
    \def@index{#1}{#2}%
}
%    \end{macrocode}
%    \end{macro}
%    \end{macro}
%    \end{macro}
%
%    \begin{macro}{\@preamblecmds}
%    Neither \cs{newindex}, \cs{renewindex}, nor \cs{disableindex}
%    should be used anywhere except inside style files or in the
%    preamble of a document, so we add them to the \cs{@preamblecmds}
%    list.
%    \begin{macrocode}
\@onlypreamble\newindex
\@onlypreamble\renewindex
\@onlypreamble\disableindex
%    \end{macrocode}
%    \end{macro}
%
%    \begin{macro}{\def@index}
%    \cs{def@index} does most of the work.  First, it picks up the
%    first three arguments of the \cs{[re]newindex} command and stores
%    the second two in an appropriate \cs{idx@} macro.  The title of
%    the index is treated differently, however, since it is
%    potentially fragile in a particularly odd way.  To prevent
%    mishaps, it is stored in a token register.  In addition to
%    stashing away the information about the index, \cs{def@index}
%    also opens an appropriate output file if we are writing auxiliary
%    files (i.e., unless \cs{nofiles} is in effect).
%
%    \begin{macrocode}
\def\def@index#1#2#3#4{%
    \@namedef{idx@#2}{#3:#4:#1}%
    \expandafter\let\csname if@immediate@#2\endcsname\if@tempswa
    \if@filesw
        \if@newindex
            \expandafter\newtoks\csname idxtitle@#2\endcsname
        \fi
        \@ifundefined{disable@#2}{%
            \if@newindex
                \expandafter\newwrite\csname tf@#2\endcsname
            \else
                \immediate\closeout\@nameuse{tf@#2}%
            \fi
            \immediate\openout\@nameuse{tf@#2}\jobname.#3 %
            \PackageInfo{index}{Writing index file \jobname.#3}%
        }
        {\PackageInfo{index}{Index `#2' disabled -- not opening
                      \jobname.#3}}%
    \fi
    \expandafter\csname idxtitle@#2\endcsname
}
%    \end{macrocode}
%    \end{macro}
%
%    \begin{macro}{\@second}
%    \begin{macro}{\@third}
%    These are useful macros for retrieving the second and third field
%    of an index specification.
%    \begin{macrocode}
\def\@second#1:#2:#3\@nil{#2}

\def\@third#1:#2:#3\@nil{#3}
%    \end{macrocode}
%    \end{macro}
%    \end{macro}
%
%    \begin{macro}{\@nearverbatim}
%    |\@nearverbatim\foo| is much like |\meaning\foo|,
%    except that it suppresses the ``\texttt{macro ->}'' string
%    produced when \cs{meaning} expands a macro.  It is used by
%    \cs{@wrindex} to produce an ``almost verbatim'' copy of their
%    arguments.  This method replaces the use of \cs{@sanitize} from
%    latex.tex and allows indexing macros to be used in places (such
%    as inside macro arguments) where the original \cs{index} command
%    could not.  Thanks to Donald Arseneau
%    \email{asnd@erich.triumf.ca} for pointing out this trick to me.
%    (For more information on this trick, see Dirty Trick \#3 of the
%    \TeX book, page 382).
%
%    As defined, \@nearverbatim only works on macros.  It would be
%    nice if it could work with other tokens, but it's more important
%    that it work only by expansion, which means we can't put in tests
%    to see what the next token is.
%    \begin{macrocode}
\def\@nearverbatim{\expandafter\strip@prefix\meaning}
%    \end{macrocode}
%    \end{macro}
%
%    Now we define the \cs{index} macro itself.  The following
%    definitions are adapted from \bundle{latex.tex} v2.09 \vdate{25
%    March 1992}.
%
%    \begin{macro}{\makeindex}
%    First we redefine \cs{makeindex} to define the default index
%    using \cs{newindex}.  We use \cs{edef} to make sure that
%    \cs{indexname} gets expanded here.  Otherwise we'll get into an
%    infinite loop later on when we try to redefine \cs{indexname}
%    inside the \cs{theindex} environment.
%
%    Unfortunately, this means that if the user changes \cs{indexname}
%    in the preamble, the index will come out with the wrong heading.
%    \begin{macrocode}
\edef\makeindex{%
    \noexpand\newindex{default}{idx}{ind}{\indexname}%
}
%    \end{macrocode}
%    \end{macro}
%
%    \begin{macro}{\if@silentindex}
%    \begin{macro}{\if@addtoindex}
%    \begin{macro}{\if@proofmode}
%    We need three new flags.  The first, \cs{if@silentindex},
%    indicates whether the entry should be typeset in running text, as
%    well as written out to the index; this is used to implement the
%    \cs{index*} command.  The second, \cs{if@addtoindex}, indicates
%    whether entries should be written to the index; this is used to
%    disable the \cs{index} command inside of page headings and tables
%    of contents.  The third, \cs{ifproofmode}, indicates whether
%    index entries should be put in the margin of the page for
%    proofing purposes.
%    \begin{macrocode}
\newif\if@silentindex\@silentindextrue

\newif\if@addtoindex\@addtoindextrue

\newif\ifproofmode\proofmodefalse
%    \end{macrocode}
%    \end{macro}
%    \end{macro}
%    \end{macro}
%
%    \begin{macro}{\index}
%    \begin{macro}{\p@index}
%    \begin{macro}{\x@index}
%    \cs{index} will be made self-protecting (a la \cs{em}, etc.) so
%    it can be used inside, for example, sectioning commands.
%    Unfortunately, to really make \cs{index} robust, we have to
%    redefine some of \LaTeX's commands for dealing with tables of
%    contents and page headings.  (See below.) $*$sigh$*$
%    \begin{macrocode}
\def\index{\protect\p@index}

\def\p@index{%
    \if@silentindex\@bsphack\fi
    \@ifstar{\@silentindexfalse\@xindex}{\@silentindextrue\@xindex}%
}

\def\@xindex{\@ifnextchar[{\@index}{\@index[default]}}
%    \end{macrocode}
%    \end{macro}
%    \end{macro}
%    \end{macro}
%
%    \begin{macro}{\@index}
%    \begin{macro}{\@@index}
%    \begin{macro}{\@wrindex}
%    The following is much more complicated than it should have to be.
%    First, note the check to see if \cs{index} is equal to
%    \cs{@gobble}.  This is so I don't have to redefine
%    \cs{@outputpage}, which temporarily disables \cs{label},
%    \cs{index}, and \cs{glossary} by \cs{let}'ing them equal to
%    \cs{@gobble}.  (For this reason, we have to be very careful to
%    make sure that \cs{index} has expanded to \cs{p@index} before it
%    gets to \cs{@outputpage}.)  Second, note that if
%    \cs{if@addtoindex} is false, we don't complain about undefined
%    index types.  This is because if your page headings, for example,
%    are being typeset in all uppercase, you might end up with
%    something like \cs{index[AUT]{...}} instead of
%    \cs{index[aut]{...}}.
%    \begin{macrocode}
\def\@index[#1]{%
    \ifx\index\@gobble
        \@addtoindexfalse
    \fi
    \def\@tempf{%
        \begingroup
            \@sanitize
            \@@index{#1}%
    }%
    \if@addtoindex
        \@ifundefined{idx@#1}%
            {%
              \def\@tempf{%
                  \@latexerr{Index type `\string#1' undefined}%
                  \@ehc
                  \@silentindextrue
                  \@gobble
              }%
            }%
            {}%
    \fi
    \@tempf
}

\def\@@index#1#2{%
    \endgroup
    \if@addtoindex
        \if@filesw\@wrindex{#1}{#2}\fi
        \ifproofmode\@showidx{#2}\fi
    \fi
    \if@silentindex
        \expandafter\@esphack
    \else
        \@silentindextrue#2%
    \fi
}

\def\@wrindex#1#2{%
    \begingroup
        \def\@tempa{#2}%
        \edef\@tempb{\@nameuse{idx@#1}}%
        \edef\@tempb{\expandafter\@third\@tempb\@nil}%
        \csname if@immediate@#1\endcsname \else
            \expandafter\let\csname\@tempb\endcsname\relax
        \fi
        \edef\@tempa{%
           \write\@auxout{%
              \string\@writefile{#1}{%
                  \string\indexentry{\@nearverbatim\@tempa}%
                                    {\@nameuse{\@tempb}}%
              }%
           }%
        }%
    \expandafter\endgroup\@tempa
    \if@nobreak\ifvmode\nobreak\fi\fi
}
%    \end{macrocode}
%    \end{macro}
%    \end{macro}
%    \end{macro}
%
%    \begin{macro}{\seename}
%    \begin{macro}{\see}
%    \begin{macro}{\printindex}
%    \begin{macro}{\@printindex}
%    The following are adapted from \bundle{makeidx.sty}, v2.09
%    \vdate{21 Oct 91}.  \cs{index@prologue} adapted from
%    \bundle{doc.dtx}.  \Lenv{theindex} based on version from
%    \bundle{classes.dtx}, v1.3g, 26 June 1995.
%    \begin{macrocode}
\providecommand{\seename}{see}

\providecommand*{\see}[2]{\emph{\seename} #1}

\@ifclassloaded{article}{%

    \renewenvironment{theindex}{%
        \edef\indexname{\the\@nameuse{idxtitle@\@indextype}}%
        \if@twocolumn
            \@restonecolfalse
        \else
            \@restonecoltrue
        \fi
        \columnseprule \z@
        \columnsep 35\p@
        \twocolumn[%
            \section*{\indexname}%
            \ifx\index@prologue\@empty\else
                \index@prologue
                \bigskip
            \fi
        ]%
        \@mkboth{\MakeUppercase\indexname}%
                {\MakeUppercase\indexname}%
        \thispagestyle{plain}%
        \parindent\z@
        \parskip\z@ \@plus .3\p@\relax
        \let\item\@idxitem
    }{%
        \if@restonecol
            \onecolumn
        \else
            \clearpage
        \fi
    }
}{%
    \renewenvironment{theindex}{%
        \edef\indexname{\the\@nameuse{idxtitle@\@indextype}}%
        \if@twocolumn
            \@restonecolfalse
        \else
            \@restonecoltrue
        \fi
        \columnseprule \z@
        \columnsep 35\p@
        \twocolumn[%
            \@makeschapterhead{\indexname}%
            \ifx\index@prologue\@empty\else
                \index@prologue
                \bigskip
            \fi
        ]%
        \@mkboth{\MakeUppercase\indexname}%
                {\MakeUppercase\indexname}%
        \thispagestyle{plain}%
        \parindent\z@
        \parskip\z@ \@plus .3\p@\relax
        \let\item\@idxitem
    }{%
        \if@restonecol
            \onecolumn
        \else
            \clearpage
        \fi
    }
}

\def\printindex{\@ifnextchar[{\@printindex}{\@printindex[default]}}

\def\@printindex[#1]{%
    \@ifnextchar[{\@print@index[#1]}{\@print@index[#1][]}%
}

\long\def\@print@index[#1][#2]{%
    \def\@indextype{#1}%
    \long\def\index@prologue{#2}%
    \@ifundefined{idx@#1}%
        {\@latexerr{Index type `\string#1' undefined}\@ehc}%
        {%
            \edef\@tempa{\@nameuse{idx@#1}}%
            \edef\@tempa{%
                \noexpand\@input@{\jobname.\expandafter\@second\@tempa\@nil}%
            }%
            \@tempa
        }%
}
%    \end{macrocode}
%    \end{macro}
%    \end{macro}
%    \end{macro}
%    \end{macro}
%
%    \begin{macro}{\@indexstar@}
%    Now we set things up for
%    \cs{shortindexing}.\footnote{\textbf{Warning:} This feature is
%    deprecated and will be removed entirely in a future release of
%    this package.} First, we define a one-token shorthand for
%    \cs{index*}.  This will be needed in the definition of
%    \cs{idx@activehat}.
%    \begin{macrocode}
\def\@indexstar@{\index*}
%    \end{macrocode}
%    \end{macro}
%    \begin{macro}{\idx@activehat}
%    \begin{macro}{\idx@activebar}
%    Next, we define the values that |^| and |_| will have when
%    shortindexing is turned on.
%    \begin{macrocode}
\def\idx@activehat{%
    \relax
    \ifmmode\expandafter\sp\else\expandafter\@indexstar@\fi
}

\def\idx@activebar{%
    \relax
    \ifmmode\expandafter\sb\else\expandafter\index\fi
}
%    \end{macrocode}
%    \end{macro}
%    \end{macro}
%
%    \begin{macro}{\shortindexingon}
%    \begin{macro}{\shortindexingoff}
%    Now we define the \cs{shortindexingon} and \cs{shortindexinoff}
%    commands to turn shortindexing on and off (surprise!).
%    \cs{shortindexingon} saves the old definitions and \cs{catcode}'s
%    of |^| and |_| so they can later be restored by
%    \cs{shortindexingoff}.  Both of these make their changes local to
%    any enclosing group, so they can be used as declarations to
%    disable or enable shortindexing temporarily.  In addition,
%    \Lenv{shortindexingon} can also be used as an environment.
%
%    This is potentially very confusing.  My basic rationale (if it
%    can be described as such) was that under normal circumstances,
%    one would put \cs{shortindexingon} in the preamble of one's
%    document, and never want to turn it off.  \cs{shortindexingoff}
%    is an attempt to make allowance for the contingency that someone
%    might want to turn shortindexing off, either permanently or
%    temporarily.
%    \begin{macrocode}
\newif\if@shortindexing

\begingroup

    \catcode`\^\active
    \catcode`\_\active

    \gdef\shortindexingon{%
        \@shortindexingtrue
        \chardef\old@idxhatcode\catcode`\^\relax
        \chardef\old@idxbarcode\catcode`\_\relax
        \catcode`\^\active
        \catcode`\_\active
        \let\old@idxhat ^%
        \let\old@idxbar _%
        \let^\idx@activehat
        \let_\idx@activebar
    }

    \gdef\shortindexingoff{%
        \if@shortindexing
            \@shortindexingfalse
            \let^\old@idxhat
            \let_\old@idxbar
            \catcode`\^\old@idxhatcode
            \catcode`\_\old@idxbarcode
        \fi
    }

\endgroup
%    \end{macrocode}
%    \end{macro}
%    \end{macro}
%
%    Now we take some code from \bundle{showidx.sty} and merge it into
%    our new system.  There are four reasons for redefining the
%    commands here rather than just inputting \bundle{showidx.sty} (or
%    requiring the user to do so).  First, \bundle{showidx.sty} ends
%    with a call to \cs{flushbottom}, which I want to avoid.  Second,
%    the instructions for successfully using \bundle{showidx.sty}
%    along with \bundle{index.sty} would be somewhat tricky.  This
%    way, I can just tell users not to use \bundle{showidx.sty} at
%    all.  Third, I need to make some alterations to \cs{@showidx}
%    anyway.  In particular, (a) I need to add the \cs{@sanitizeat}
%    command so this works correctly with AMS-\LaTeX\ and (b) I want
%    to add the \cs{indexproofstyle} command so the user can customize
%    the size and font used for the index proofs.  Finally,
%    \bundle{showidx.sty} has at least two annoying bugs in it.  See
%    the edit-history for version 2.01 for a description.
%
%    \begin{macro}{\@indexbox}
%    This code is adapted from \bundle{showidx.sty}, v2.09 \vdate{16
%    Jun 1991}.
%    \begin{macrocode}
\newinsert\@indexbox

\dimen\@indexbox\maxdimen
%    \end{macrocode}
%    \end{macro}
%
%    \begin{macro}{\@sanitizeat}
%    The definition of \cs{@sanitizeat} is slightly tricky, since we
%    need |@| to be active when this macro is defined, but we also
%    need it to be part of the control sequence name.
%    \begin{macrocode}
\begingroup
    \catcode`\@\active
    \expandafter\gdef\csname\string @sanitizeat\endcsname
        {\def @{\char`\@}}
\endgroup
%    \end{macrocode}
%    \end{macro}
%
%    \begin{macro}{\indexproofstyle}
%    \begin{macro}{\@showidx}
%    \begin{macro}{\@leftidx}
%    \begin{macro}{\@rightidx}
%    \begin{macro}{\@mkidx}
%    \begin{macro}{\raggedbottom}
%    \begin{macro}{\flushbottom}
%    \begin{macro}{\@texttop}
%    \begin{macrocode}
\newtoks\indexproofstyle

\indexproofstyle{\footnotesize\reset@font\ttfamily}

\def\@showidx#1{%
    \insert\@indexbox{%
        \@sanitizeat
        \the\indexproofstyle
        \hsize\marginparwidth
        \hangindent\marginparsep \parindent\z@
        \everypar{}\let\par\@@par \parfillskip\@flushglue
        \lineskip\normallineskip
        \baselineskip .8\normalbaselineskip\sloppy
        \raggedright \leavevmode
        \vrule \@height .7\normalbaselineskip \@width \z@\relax#1\relax
        \vrule \@height\z@ \@depth.3\normalbaselineskip \@width\z@\relax
    }%
    \ifhmode\penalty\@M \hskip\z@skip\fi
}

\def\@leftidx{\hskip-\marginparsep \hskip-\marginparwidth}

\def\@rightidx{\hskip\columnwidth \hskip\marginparsep}

\def\@mkidx{%
    \vbox to \z@{%
        \rlap{%
            \if@twocolumn
                \if@firstcolumn \@leftidx \else \@rightidx \fi
            \else
                \if@twoside
                    \ifodd\c@page \@rightidx \else \@leftidx \fi
                \else
                    \@rightidx
                \fi
            \fi
            \box\@indexbox
        }%
        \vss
    }%
}

\def\raggedbottom{%
    \def\@textbottom{\vskip\z@ plus.0001fil}%
    \let\@texttop\@mkidx
}

\def\flushbottom{\let\@textbottom\relax \let\@texttop\@mkidx}

\let\@texttop\@mkidx
%    \end{macrocode}
%    \end{macro}
%    \end{macro}
%    \end{macro}
%    \end{macro}
%    \end{macro}
%    \end{macro}
%    \end{macro}
%    \end{macro}
%
%    Now, this next bit really gets up my nose.  The only way to make
%    sure that the \cs{index} command gets handled correctly when used
%    inside of sectioning commands is to redefine a bunch of \LaTeX's
%    table of contents and running-heads macros. $*$blech$*$ Fragility
%    rears its ugly head again.
%
%    These are based on \bundle{latex.tex} 2.09 \vdate{25 March 1992}.
%
%    \begin{macro}{\addcontentsline}
%    We need to redefine \cs{addcontentsline} to keep it from
%    expanding \cs{index} commands too far.  In particular, we have
%    removed \cs{index} from the list of macros that are set equal to
%    \cs{@gobble} and we substitute \cs{@vwritefile} for
%    \cs{@writefile}.  This latter change also means that we can
%    simplify the definition of \cs{protect} somewhat.
%    \begin{macrocode}
\CheckCommand\addtocontents[2]{%
  \protected@write\@auxout
      {\let\label\@gobble \let\index\@gobble \let\glossary\@gobble}%
      {\string\@writefile{#1}{#2}}%
}

\renewcommand{\addtocontents}[2]{%
    \protected@write\@auxout
      {\let\label\@gobble \let\glossary\@gobble}%
      {\string\@writefile{#1}{#2}}%
}
%    \end{macrocode}
%    \end{macro}
%
%    \begin{macro}{\@starttoc}
%    We need to redefine \cs{@starttoc} to \cs{@addtoindexfalse} so
%    that items don't get written to the index from within tables of
%    contents.  The only change here is the addition of
%    \cs{@addtoindexfalse}.
%
%    Unfortunately, this will break pretty badly with the AMS document
%    classes, since they redefine \cs{@starttoc} to take two arguments
%    rather than one.  This must be addressed.
%
%    \begin{macrocode}
\let\old@starttoc\@starttoc

\renewcommand{\@starttoc}[1]{%
    \begingroup
        \@addtoindexfalse
        \old@starttoc{#1}%
    \endgroup
}
%    \end{macrocode}
%    \end{macro}
%
%    \begin{macro}{\markboth}
%    \begin{macro}{\markright}
%    Finally, we have to redefine \cs{markboth} and \cs{markright} to
%    keep them from disabling the expansion of \cs{index} while
%    putting section heads into the \cs{mark}.  Otherwise, we'd end up
%    with ``\cs{index}'' in the mark, which would cause problems when
%    \cs{@outputpage} redefines \cs{index} to be equal to
%    \cs{@gobble}.  Instead, we want \cs{index} to expand to
%    \cs{p@index} in the \cs{mark}, so we retain control over what
%    happens in \cs{@outputpage}.
%
%    This time, the only change is to remove \cs{index} from the list
%    of macros that are \cs{let} equal to \cs{relax}.
%    \begin{macrocode}
\CheckCommand*{\markboth}[2]{%
  \begingroup
    \let\label\relax \let\index\relax \let\glossary\relax
    \unrestored@protected@xdef\@themark {{#1}{#2}}%
    \@temptokena \expandafter{\@themark}%
    \mark{\the\@temptokena}%
  \endgroup
  \if@nobreak\ifvmode\nobreak\fi\fi}
\CheckCommand*{\markright}[1]{%
  \begingroup
    \let\label\relax \let\index\relax \let\glossary\relax
    \expandafter\@markright\@themark {#1}%
    \@temptokena \expandafter{\@themark}%
    \mark{\the\@temptokena}%
  \endgroup
  \if@nobreak\ifvmode\nobreak\fi\fi}

\renewcommand{\markboth}[2]{%
  \begingroup
    \let\label\relax \let\glossary\relax
    \unrestored@protected@xdef\@themark {{#1}{#2}}%
    \@temptokena \expandafter{\@themark}%
    \mark{\the\@temptokena}%
  \endgroup
  \if@nobreak\ifvmode\nobreak\fi\fi}

\renewcommand{\markright}[1]{%
  \begingroup
    \let\label\relax \let\glossary\relax
    \expandafter\@markright\@themark {#1}%
    \@temptokena \expandafter{\@themark}%
    \mark{\the\@temptokena}%
  \endgroup
  \if@nobreak\ifvmode\nobreak\fi\fi}
%</style>
%    \end{macrocode}
%    \end{macro}
%    \end{macro}
%
%    \section{Edit history}
%
%    \begin{description}
%
%    \item[v1.00 (4 Mar 1993)]
%    initial version, posted to comp.text.tex.
%
%    \item[v1.01 (4 Mar 1993)]
%    added \cs{renewindex} command and checking to make sure index is
%    (or is not) defined in \cs{newindex}, \cs{index} and
%    \cs{printindex}.  Also tightened up the code in various places
%    and added check to make sure file is only loaded once.
%
%    \item[v2.00 (24 Mar 1993)]
%    added support for \cs{index*}, proofmode, \cs{shortindexingon}
%    and \cs{shortindexingoff}.
%
%    \item[v2.01 (24 Jun 1993)]
%    Fixed 3 bugs.  (1) If proofmode was turned on, then something
%    like ``\cs{index{WORD}WORD}'' would suppress the hyphenation of
%    WORD.  This was fixed by adding ``|\penalty\@M\hskip\z@skip|'' to
%    the end of \cs{@showidx}.  (This is just the definition of
%    \cs{allowhyphens} borrowed from \bundle{german.sty}, v2 \vdate{4
%    Nov 1988}).  (2) The \cs{hbox} in \cs{@mkidx} was being set at
%    its natural width, which had a tendency to interfere with the
%    width of the page.  The \cs{hbox} is now replaced by \cs{rlap}.
%    (3) If the title of an index (i.e., the fourth argument of
%    \cs{newindex}) contained a particularly fragile command
%    like~\cs{d}, havoc would ensue when \cs{theindex} tried to
%    extract the title.  Titles are now kept in token registers to
%    prevent such unpleasantness.  Bugs (2) and (3) were reported by
%    Dominik Wujastyk \email{D.Wujastyk@ucl.ac.uk} on 24 June 1993.
%    Note that bugs (1) and (2) are actually bugs in showidx.sty,
%    v2.09 \vdate{16 Jun 1991}.
%
%    \item[v2.02 (25 Jun 1993)]
%    Rewrote the code that implements the short indexing commands (|^|
%    and |_|) to make index.sty compatible with other style files that
%    need to make |^| and |^| active in some contexts.  See the code
%    for more details.
%
%    \item[v2.03 (30 Jun 1993)]
%    Once again rewrote the code that implements the short indexing
%    commands.  Dumped the shortindexing environment and rewrote the
%    \cs{shortindexingon} and \cs{shortindxingoff} commands to save
%    and restore the \cs{catcode}'s and meanings of |^| and |^| in the
%    safest possible (I hope) order.  Also added the
%    \cs{if@shortindexing} flag to keep \cs{shortindexingoff} from
%    doing anything if it is called outside of the scope of a
%    \cs{shortindexingon} command.  (Question: Should
%    \cs{shortindexingon} check that flag before doing anything?)
%
%    \item[v2.04 (beta) (14 Jul 1993)]
%    Added \cs{disableindex} command.  Added \cs{newindex} and
%    \cs{renewindex} to \cs{@preamblecmds}.  Added \cs{if@newindex}
%    flag to \cs{@newindex} to prevent \cs{renewindex} from
%    re-allocating new \cs{write} and \cs{toks} registers.  Rewrote
%    using \bundle{doc.sty} and \program{DocStrip}.  Also cleaned up
%    the code somewhat.
%
%    \item[v3.00 (15 Jul 1993)]
%    Made further minor tweaks to code and internal documentation.
%    Booted version number up to 3.00 and released on the world.
%
%    \item[v3.01 (19 Jul 1993)]
%    Fixed \program{DocStrip} CheckSum.
%
%    \item[v3.02 (15 Sep 1993)]
%    Corrected spelling of \cs{@shortindexingfalse} in definition of
%    \cs{shortindexingoff}.  Thanks to Hendrik G. Seliger
%    \email{hank@Blimp.automat.uni-essen.de} for this bug report.
%    Also added redefinitions of \cs{@leftmark} and \cs{@rightmark} to
%    fix a bug reported by Dominik Wujastyk
%    \email{D.Wujastyk@ucl.ac.uk}.
%
%    \item[v3.03 (beta) (20 Feb 1994)]
%    Added \cs{long} to the definition of \cs{@ifundefined} to cover
%    the unlikely contingency that someone wanted to use, for example,
%    |\string\par| in the middle of a control sequence name.  Added an
%    optional argument to \cs{newindex} to specify which counter to
%    use in place of \cs{thepage}.  The first change was suggested by
%    Martin Schr\"oder \email{l15d@zfn.uni-bremen.de}; the second was
%    suggested independently by Schr\"oder and Stefan Heinrich
%    H\"oning \email{hoening@pool.informatik.rwth-aachen.de}.  The
%    \cs{@newindex} command was renamed \cs{def@index}.  Also fixed
%    the \cs{disableindex} command.
%
%    \item[v3.04 (7 Mar 1994)]
%    Rewrote the user documentation (Sections 1--5) and released on
%    the world.  Also deleted some extraneous spaces that had crept
%    into some macros.
%
%    \item[v4.00beta, (20 Feb 1995)]
%    Preliminary conversion to a native \LaTeXe\ package.  Fixed
%    \cs{@printindex} to work under \LaTeXe\ (bug reported by Carsten
%    Folkertsma \email{cai@butler.fee.uva.nl}).  Removed much code
%    that had been put in to work around various ancient versions of
%    \LaTeX~2.09.  Added \cs{index@prologue} support (modelled on
%    \bundle{doc.sty}) at suggestion of Nick Higham
%    \email{higham@ma.man.ac.uk}.
%
%    \item[v4.01beta (28 Sep 1995)]
%    Rewrote as a \LaTeXe\ package (finally!).  Changes too numerous
%    to list, but in general deleted some now-superfluous code,
%    replaced some tricks by tricks from the \LaTeXe\ kernel, and
%    added some bullet-proofing.  Much still remains to be done, but
%    this should be good enough for testing.
%
%    Changed definition of \cs{protect} in \cs{markright} and
%    \cs{markboth} to fix bug reported by Dominik Wujastyk.
%
%    \item[??? (5 Jan 2004)]
%
%
% \end{description}
%
% \DisableCrossrefs
%
% \section{The sample file}
%
%    \begin{macrocode}
%<*sample>
%% latex sample.tex
%% makeindex sample
%% makeindex -o sample.and sample.adx
%% makeindex -o sample.nnd sample.ndx
%% makeindex -o sample.lnd sample.ldx
%% latex sample.tex

\documentclass{book}
\usepackage{index}

\listfiles

\makeindex
\newindex{aut}{adx}{and}{Name Index}
\newindex{not}{ndx}{nnd}{List of Notation}

\newindex[theenumi]{list}{ldx}{lnd}{Items}

\shortindexingon

\proofmodetrue

\def\aindex{\index*[aut]}

\begin{document}

\tableofcontents

\newpage

\chapter{Here is a ^[aut]{chapter} title}

\section{Section header\index[aut]{section}}

Here is some text.\index{subject}

Here is \index[not]{notation}some more \index[not]{sin@$\sin$}
text.

\newpage

Here is some ^{more} _[not]{notation} text.

Here is yet more \aindex{text}.

\section{Another Section header _[aut]{section2}}

And here is some math: $x^1_b$.

Here is an ^[aut]{index} entry \fbox{inside an
\index[not]{min@$\min$}fbox}

\fbox{Here is an ^[aut]{entry} in a box.}

\section{An indexed list environment}

\begin{enumerate}

\item
First item

\item
Second item\index[list]{second item}

\item
Third item

\newpage

\item
Fourth item

\item
Fifth item\index[list]{fifth item}

\item
Sixth item

\end{enumerate}

\printindex[not]

\printindex[aut][Here is a prologue for the author index.
Note that it is set in a single column at the top of the
first page of the index.]

\printindex[list]

\printindex

\end{document}
%</sample>
%    \end{macrocode}
%
% \Finale
%
\endinput

% paramètres de la table des matières
\renewcommand*{\contentsname}{Table des matières}
\let\changetocdepth\oldchangetocdepth
\setcounter{tocdepth}{7}

% \cftpagenumberson{book}
% \cftpagenumberson{part}
% \cftpagenumberson{chapter}

\renewcommand{\tocheadstart}{\chapterheadstart}
\makeatletter
\renewcommand{\cftbookbreak}{\addpenalty{-\@highpenalty}\clearpage}
\makeatother
\renewcommand{\cftbeforebookskip}{2em}
\renewcommand{\cftbookfont}{\bfseries\huge}
\renewcommand{\cftbookindent}{0em}
\renewcommand{\cftbooknumwidth}{2em}
%\renewcommand{\cftafterbookskip}{}
\renewcommand{\cftbookleader}{\cftdotfill{\cftdotsep}}

\renewcommand{\cftbeforepartskip}{1.5em}
\renewcommand{\cftpartfont}{\huge}
\renewcommand{\cftpartindent}{0em}
%\renewcommand{\cftpartnumwidth}{}
\renewcommand{\cftpartleader}{\cftdotfill{\cftdotsep}}

\renewcommand{\cftbeforechapterskip}{1em}
\renewcommand{\cftchapterfont}{\LARGE}
\renewcommand{\cftchaptername}{Chapitre\space}
\renewcommand{\cftchapterindent}{0em}
%\renewcommand{\cftchapternumwidth}{}
\renewcommand{\cftchapterleader}{\cftdotfill{\cftdotsep}}

\renewcommand{\cftbeforesectionskip}{0.5em}
\renewcommand{\cftsectionfont}{\Large}
%\renewcommand{\cftsectionname}{Section\space}
\renewcommand{\cftsectionindent}{0em}
\renewcommand{\cftsectionnumwidth}{5.5em}
%\renewcommand{\cftsectionleader}{\cftdotfill{\cftdotsep}}

\renewcommand{\cftbeforesubsectionskip}{0.5em}
\renewcommand{\cftsubsectionfont}{\large}
%\renewcommand{\cftsubsectionname}{}
\renewcommand{\cftsubsectionindent}{0em}
\renewcommand{\cftsubsectionnumwidth}{2em}
%\renewcommand{\cftsubsectionleader}{\cftdotfill{\cftdotsep}}

%\renewcommand{\cftbeforesubsubsectionskip}{}
\renewcommand{\cftsubsubsectionfont}{\normalsize}
%\renewcommand{\cftsubsubsectionname}{}
\renewcommand{\cftsubsubsectionindent}{1em}
\renewcommand{\cftsubsubsectionnumwidth}{2em}
%\renewcommand{\cftsubsubsectionleader}{\cftdotfill{\cftdotsep}}

%\renewcommand{\cftbeforeparagraphskip}{}
\renewcommand{\cftparagraphfont}{\normalsize}
%\renewcommand{\cftparagraphname}{}
\renewcommand{\cftparagraphindent}{2em}
\renewcommand{\cftparagraphnumwidth}{2em}
%\renewcommand{\cftparagraphleader}{\cftdotfill{\cftdotsep}}

%\renewcommand{\cftbeforesubparagraphskip}{}
\renewcommand{\cftsubparagraphfont}{\normalsize}
%\renewcommand{\cftsubparagraphname}{}
\renewcommand{\cftsubparagraphindent}{3em}
\renewcommand{\cftsubparagraphnumwidth}{2em}
%\renewcommand{\cftsubparagraphleader}{\cftdotfill{\cftdotsep}}

\renewcommand{\cftsubsubparagraphfont}{\small}

\renewcommand{\cftsubsubsubparagraphfont}{\footnotesize}


\clearpage
\tableofcontents
 % table des matieres
\end{ltxexample}
Font partie de ce bloc: la conclusion générale, les annexes, la bibliographie, l'index et enfin la table des matières. Les en-têtes sont activées selon la logique exposée \secref{entete}.
\end{enumerate}

\subsubsection{\file{misc.tex}}

Ce fichier contient des informations diverses présentées en début de document: un avertissement, les remerciements de l'auteur et le résumé en français et anglais.

\subsubsection{\file{sommaire.tex}}

Ce fichier imprime le sommaire. Le sommaire est ici configuré pour n'afficher que les parties, titres et chapitres.

Ce fichier contient également des réglages de rendu du sommaire, qui peuvent être modifiés.

\subsubsection{\file{glossaire.tex}}

Le glossaire est géré dans le fichier \file{glossaire.tex}. Il contient la liste en vrac des termes à référencer, suivie de l'appel à la commande \cmd{printglossary}. L'option style=mcolindex permet d'imprimer le glossaire en deux colonnes. Toutes les commandes utilisées sont fournies par le package \sty{glossaries}.
\begin{ltxexample}
\<<newacronym>>{ibid.}{ibidem}
\<<newacronym>>{D.}{Recueil Dalloz}
...
\<<glsaddall>>
\<<printglossary>>[style=mcolindex]
\end{ltxexample}

\subsubsection{\file{journaux.bib}}

Ce fichier contient des chaînes de caractères \bibtex contenant des noms de revues juridiques en abrégé. Ces chaînes peuvent être utilisées dans d'autre fichiers contenant des références bibliographiques.

\subsubsection{\file{bibliographie.bib}}

Ce fichier contient une ensemble des références bibliographiques proposées à titre d'exemple. Il est chargé après le fichier \file{journaux.bib}.

\paragraph{Exemple de jurisprudence}
\label{templatejurisprudence}

Il est d'usage en droit français de grouper un arrêt (type \bibtype{jurisdiction}) avec tous les commentaires s'y référant (type \bibtype{commentary}). 

Pour lier tous ces éléments, le champ spécial \bibfield{related} dans l'entrée de type \bibtype{jurisdiction} décrivant l'arrêt:
\begin{lstlisting}[style=bibtex]{}	
@COMMENTARY{<<cass:ass:19910531:thouvenin>>,
  ...
}

@COMMENTARY{<<cass:ass:19910531:gobert>>,
  ...
}

@JURISDICTION{<<cass:ass:19910531>>,
  institution = {Cass. Ass. Pl\'en.},
  date = {1991-05-31},
  related = {<<cass:ass:19910531:thouvenin>>, <<cass:ass:19910531:gobert>>},
  keywords = {cassass}
}
\end{lstlisting}
%
Pour citer d'un seul tenant la jurisprudence couverte par ces 3 entrées, il suffit d'appeler l'entrée de type \bibtype{jurisdiction} via commande suivante depuis le corps du texte:
\begin{ltxexample}
\cite{<<cass:ass:19910531>>}
\end{ltxexample}
%

Toutes les entrées de type \bibtype{commentary} seront automatiquement accolées à la suite de l'arrêt\footnote{Attention, ce mécanisme ne fonctionne que pour des entrées de type \bibtype{commentary}.}.

On notera que l'ordre dans lequel les différentes entrées de type \bibfield{commentary} sont écrites dans le champ \bibfield{related} correspond à l'ordre d'affichage dans le document final.

On notera par ailleurs la présence d'un mot clé (ici, \enquote{cassass}) dans le champ \bibfield{keywords} du type \bibtype{jurisdiction}, pour marquer que ce groupe correspond à un arrêt rendus par la cour de cassation en assemblée plénière.

\paragraph{Exemple de code juridique}

Est ici pris en exemple le code civil. L'entrée correspondante est la suivante:
\begin{lstlisting}[style=bibtex]{}
@LEGISLATION{<<cciv>>,
  title = {Code civil},
  pagination = {section}
}
\end{lstlisting}
On notera l'utilisation de la pagination \texttt{section}, utilisée pour générer le mot-clé abrégé "art." dans la citation crée en note de bas de page.

Pour citer l'article 1642-1 du code civil, il suffit d'appeler la commande suivante depuis le corps du texte:
\begin{ltxexample}
\cite[1642-1]{<<cciv>>}
\end{ltxexample}

\paragraph{Exemple d'ouvrage dit "spécial"}

Dans beaucoup de bibliographies en droit, certains ouvrages sont qualifiés de "spéciaux", par opposition à des ouvrages dits "généraux". Cependant, cette distinction porte généralement sur le fond, et est difficilement gérable à l'aide des différents types de références proposés dans \biblatex.

Aussi, cette distinction quant au fond peut être réalisée grâce à l'utilisation d'un mot clé, par exemple "special", dans le champ \bibfield{keywords}:

\begin{ltxexample}
@BOOK{alland:dicoculturejur,
  author = {Alland, Denis and Rials, Stephane},
  title = {Dictionnaire de la culture juridique},
  date = {2003},
  publisher = {Lamy-PUF},
  keywords = {french, <<special>>}
}
\end{ltxexample}

\subsubsection{\file{bibliographie.tex}}

L'organisation de la bibliographie est gérée dans le fichier \file{bibliographie.tex}. La classification choisie est ici par pays (France et Europe) puis en 8 groupes de références: lois, rapports officiels, jurisprudence, ouvrages généraux, ouvrages spéciaux, thèses, ouvrages collectifs et enfin articles.

Le fichier est divisé en deux parties: la première définit les réglages d'organisation et la seconde imprime la bibliographie proprement dite.

\paragraph{Réglages}

Les lignes suivantes définissent des libellés de sous-bibliographies:
\begin{ltxexample}
\defbibheading{france}{\section{Droit francais}}
\defbibheading{europe}{\section{Droit europeen}}

\defbibheading{lois}{\paragraphe{Lois}}
\defbibheading{rapports}{\paragraphe{Rapports officiels}}
\defbibheading{jurisprudence}{\paragraphe{Jurisprudence}}
\defbibheading{generaux}{\paragraphe{Ouvrages generaux}}
\defbibheading{speciaux}{\paragraphe{Ouvrages speciaux}}
\defbibheading{theses}{\paragraphe{Theses}}
\defbibheading{collectifs}{\paragraphe{Ouvrages collectifs}}
\defbibheading{articles}{\paragraphe{Articles}}
\end{ltxexample}
%
On notera plus spécifiquement les libellés sous-bibliographies ci-dessous, relatives à la jurisprudence française:

\begin{ltxexample}
\defbibheading{juris:ccel}{\souspara{Conseil constitutionnel}}
\defbibheading{juris:ce}{\souspara{Conseil d'Etat}}
\defbibheading{juris:cass}{\souspara{Cour de cassation}}
\defbibheading{juris:cass:ass}{\alinea{Assemblee pleniere}}
\defbibheading{juris:cass:1civ}{\alinea{1\iere{} chambre civile}}
\defbibheading{juris:cass:2civ}{\alinea{2\ieme{} chambre civile}}
\defbibheading{juris:cass:3civ}{\alinea{3\ieme{} chambre civile}}
\defbibheading{juris:cass:com}{\alinea{Chambre commerciale}}
\defbibheading{juris:cass:soc}{\alinea{Chambre sociale}}
\defbibheading{juris:cass:crim}{\alinea{Chambre criminelle}}
\defbibheading{juris:ca}{\souspara{Cours d'appel}}
\defbibheading{juris:tgi}{\souspara{Tribunaux de grande instance}}
\defbibheading{juris:ti}{\souspara{Tribunaux d'instance}}
\end{ltxexample}
%

Chacune des ces sous-bibliographies est destinée à être peuplée d'entrées issues du fichier \file{bibliographie.bib}.

Les sous-bibliographies suivantes contiennent un unique type de référence:
\begin{itemize}
\item les lois: \bibtype{legislation}
\item la jurisprudence: \bibtype{jurisdiction} (pouvant contenant implicitement un ou plusieurs \bibtype{commentary} via le champ \bibfield{related}, voir \ref{templatejurisprudence}). Le champ \bibfield{keywords} est utilisé pour filtrer ces entrées par juridiction.
\item les thèses: \bibtype{thesis}
\end{itemize}

Les catégories suivantes comprennent plusieurs types de références:
\begin{itemize}
\item Les ouvrage dits "collectifs" comprennent les \bibtype{collection} et \bibtype{proceedings}
\item Les articles comprennent les \bibtype{article} et les contributions à des ouvrages collectifs (\bibtype{incollection} et \bibtype{inproceedings}).
\item Les ouvrage généraux et spéciaux comprennent les types \bibtype{book} et \bibtype{inbook}. La distinction entre les deux est effectuée par examen du champ \bibfield{keywords}: si la clé \texttt{special} est présente, la référence sera affichée comme ouvrage spécial, sinon comme général.
\end{itemize}

Ce choix de groupement est réalisé à l'aide des filtres ci-dessous:
\begin{ltxexample}
\defbibfilter{gen}{%
  \(\type{<<book>>} \or \type{<<inbook>>}\) \and \not \keyword{<<special>>}}
\defbibfilter{spec}{%
  \(\type{<<book>>} \or \type{<<inbook>>}\) \and \keyword{<<special>>}}
\defbibfilter{col}{\type{<<collection>>} \or \type{<<proceedings>>}}
\defbibfilter{art}{%
  \type{<<incollection>>} \or \type{<<inproceedings>>} \or \type{<<article>>}}
\end{ltxexample}
%

Pour chaque pays, le même système de groupement est utilisé. Il est combiné avec une autre clé du champ \bibfield{keywords} utilisée pour filtrer les références selon le pays.

\paragraph{Structure}

La bibliographie est imprimée grâce au jeu de commandes ci-dessous:

\begin{ltxexample}
\chapitre{Bibliographie}

\printbibheading[heading=france]

\printbibliography[heading=<<lois>>,type=<<legislation>>,keyword=<<french>>]
\printbibliography[heading=<<rapports>>,type=<<report>>,keyword=<<french>>]
\printbibheading[heading=jurisprudence]
\printbibliography[heading=<<juris:ccel>>, type=<<juridsiction>>, keyword=<<ccel>>]
\printbibliography[heading=<<juris:ce>>, type=<<juridsiction>>, keyword=<<ce>>]
\printbibheading[heading=<<juris:cass>>]
\printbibliography[heading=<<juris:cass:ass>>, type=<<juridsiction>>, keyword=<<cassass>>]
\printbibliography[heading=<<juris:cass:1civ>>, type=<<juridsiction>>, keyword=<<cass1civ>>]
\printbibliography[heading=<<juris:cass:2civ>>, type=<<juridsiction>>, keyword=<<cass2civ>>]
\printbibliography[heading=<<juris:cass:3civ>>, type=<<juridsiction>>, keyword=<<cass3civ>>]
\printbibliography[heading=<<juris:cass:com>>, type=<<juridsiction>>, keyword=<<casscom>>]
\printbibliography[heading=<<juris:cass:soc>>, type=<<juridsiction>>, keyword=<<casssoc>>]
\printbibliography[heading=<<juris:cass:crim>>, type=<<juridsiction>>, keyword=<<casscrim>>]
\printbibliography[heading=<<juris:ca>>, type=<<juridsiction>>, keyword=<<ca>>]
\printbibliography[heading=<<juris:tgi>>, type=<<juridsiction>>, keyword=<<tgi>>]
\printbibliography[heading=<<juris:ti>>, type=<<juridsiction>>, keyword=<<ti>>]
\printbibliography[heading=<<generaux>>,filter=<<gen>>,keyword=<<french>>]
\printbibliography[heading=<<speciaux>>,filter=<<spec>>,keyword=<<french>>]
\printbibliography[heading=<<collectifs>>,filter=<<col>>,keyword=<<french>>]
\printbibliography[heading=<<theses>>,type=<<thesis>>,keyword=<<french>>]
\printbibliography[heading=<<articles>>,filter=<<art>>,keyword=<<french>>]

\printbibheading[heading=europe]

\printbibliography[heading=<<lois>>,type=<<legislation>>,keyword=<<ue>>]
\printbibliography[heading=<<rapports>>,type=<<report>>,keyword=<<ue>>]
\printbibliography[heading=<<jurisprudence>>,type=<<juridsiction>>,keyword=<<ue>>]
\printbibliography[heading=<<generaux>>,filter=<<gen>>,keyword=<<ue>>]
\printbibliography[heading=<<speciaux>>,filter=<<spec>>,keyword=<<ue>>]
\printbibliography[heading=<<collectifs>>,filter=<<col>>,keyword=<<ue>>]
\printbibliography[heading=<<theses>>,type=<<thesis>>,keyword=<<ue>>]
\printbibliography[heading=<<articles>>,filter=<<art>>,keyword=<<ue>>]
\end{ltxexample}
%

Deux commandes fournies par le package \biblatex sont utilisées pour imprimer la bibliographie: \cmd{printbibliography} et \cmd{printbibheading}.

La commande \cmd{printbibheading} est relativement simple: elle sert uniquement à imprimer un titre de bibliographie, sans contenu.

la commande \cmd{printbibliography} imprime un ensemble de références bibliographiques issues des fichiers qui ont été référencés en préambule du document au moyen de la commande \cmd{addbibresource} (voir le fichier maître \file{main.tex}). La commande \cmd{printbibliography} sélectionne des références sur la base de critères de sélection passés en arguments. Comme le montre les lignes ci-dessus, les options \texttt{heading}, \texttt{filter}, \texttt{type} et \texttt{keyword} sont utilisées comme critères de sélection.

On notera en particulier que le champ \bibfield{keywords} est utilisé de façon à filter les références de type \bibtype{jurisdiction} par juridictions. Le tableau \ref{clesjuridictions} propose des exemples de mots-clés pour la plupart des juridictions françaises. Bien entendu, ces mots-clés sont à la discrétion du rédacteur.

\begin{table}
\tablesetup
\begin{tabularx}{\textwidth}{@{}@{}p{200pt}>{\ttfamily}X@{}}
\toprule
\multicolumn{1}{@{}H}{Juridiction} &
\multicolumn{1}{@{}H}{mot-clé à inclure au champ \bibfield{keywords}} \\
\cmidrule(r){1-1}\cmidrule(r){2-2}
Conseil constitutionnel & cc \\
Conseil d'État & ce \\
Cour de cassation (générique) & cass \\
Cour de cassation en assemblée plénière & cassass \\
1\iere{} chambre civile & cass1civ \\
2\ieme{} chambre civile & cass2civ \\
3\ieme{} chambre civile & cass3civ \\
Chambre criminelle & casscrim \\
Chambre sociale & casssoc \\
Chambre commerciale & casscom \\
Tribunal de grande instance & tgi \\
Tribunal d'instance & ti \\

\bottomrule
\end{tabularx}
\caption{Mots-clés de filtrage par juridiction française}
\label{clesjuridictions}
\end{table}

\subsubsection{\file{index.tex}}

Ce fichier imprime l'index par versets de base et l'index d'auteurs par verset, au moyen des commandes suivantes.
\begin{ltxexample}
\printindexv
\printindexa
\end{ltxexample}

\subsubsection{\file{toc.tex}}

Ce fichier imprime la table des matières. Contrairement au sommaire, la table des matière affiche tous les niveaux de subdivisions fournis par la classe \sty{\classname}, sauf les versets.

Ce fichier contient également des réglages de rendu de la table des matières, qui peuvent être modifiés.

\subsubsection{\file{main.pdf}}

Ce fichier est le résultat de la compilation du projet complet. Tous les liens actifs dans le document sont encadrés de rouge. Ces cadres disparaissent lors d'une impression papier.

Voir \secref{compilation} pour plus de détails concernant l'étape de compilation.

\subsection{Compilation}
\label{compilation}

\subsubsection{En ligne de commande}

Voici la séquence de compilation minimale à effectuer sur le fichier principal \file{main.tex} pour produire le fichier \file{main.pdf}:

\begin{lstlisting}[style=plain]{}
pdflatex main.tex
makeindex -s index.ist -o index.idx index.ind
makeindex -s auteurs.ist -o auteurs.idx auteurs.ind
makeglossaries main.glo
biber main.bcf
pdflatex main.tex
pdflatex main.tex
\end{lstlisting}
%
Plusieurs passes sur le programme \texttt{pdflatex} sont nécessaires pour résoudre les problèmes de références croisées.

Le programme \texttt{makeindex} est à appeler deux fois, pour l'index de base et l'index d'auteurs. Les fichiers d'extension \file{.ist} sont des fichiers de style passés en paramètre de \texttt{makeindex}; ils sont créés à la première passe de pdflatex si absents.

Le programme \texttt{makeglossaries} est un script Perl qui génère le glossaire. Il nécessite l'installation préalable d'une distribution Perl.

Le programme \texttt{\biber} participe à la génération de la bibliographie. Il doit être appelé une fois en lieu et place de \bibtex.

\subsubsection{Avec Latexmk}

Latexmk\fnurl{http://www.phys.psu.edu/~collins/software/latexmk-jcc/} est un programme qui gère de manière interne les étapes intermédiaires de compilation décrites précédemment. Dans un unique appel à latexmk, latex, makeindex et biber sont appelés autant de fois que nécessaire afin de produire un fichier PDF:

\begin{lstlisting}[style=plain]{}
latexmk main.tex
\end{lstlisting}
%

Pour effacer tous les fichiers générés au cours de la compilation sauf le PDF final, il suffit d'appeler le programme avec l'option -c.

\begin{lstlisting}[style=plain]{}
latexmk -c
\end{lstlisting}
%

Le fichier \file{latexmkrc} contient des paramètres permettant d'exécuter automatiquement la séquence manuelle définie dans la section précédente; sa présence dans le répertoire de travail est donc \emph{indispensable}.

%\subsection{Adaptation de l'exemple}
%TODO: finir cette section
%UTF8

\section{Recommandations et astuces}
\label{astuces}

Cette partie a pour objectif de recenser un certain nombre d'astuces utiles pour le rédacteur.

\subsection{Le choix des logiciels}

\subsubsection{L'éditeur \latex}

Beaucoup d'éditeurs \latex gratuits existent, avec des approches différentes.

\texttt{Texmaker} offre un bon compromis entre simplicité et puissance:
\begin{itemize}
\item il est disponible sous Linux, Mac et Windows
\item l'interface graphique n'est pas surchargées de boutons inutiles
\item on peut utiliser LatexMk pour la compilation (très fortement recommandé!)
\item prévisualisation simple et efficace du PDF généré, avec changements récents apparaissant brièvement en rouge à l'ouverture du PDF
\item bascule très simple depuis un endroit du PDF vers le code source \latex correspondant, et vice-versa. Ceci  est particulièrement efficace pour corriger des coquilles à la lecture du PDF\fnurl{http://itexmac.sourceforge.net/SyncTeX.html}.
\end{itemize}

\subsubsection{Le gestionnaire de bibiographie}

Éditer à la main son ou ses fichiers \file{.bib} est possible, mais s'avère fastidieux et sensible aux erreurs. Il est donc conseillé de passer par un gestionnaire de bibliographie, qui présentera le contenu du fichier \file{.bib} sous la forme d'une base de donnée.

Attention, \biblatex est une extension relativement récente de \bibtex, que beaucoup de gestionnaires de bibliographies ne prennent pas en charge.

Le logiciel Jabref est très fortement recommandé\fnurl{http://jabref.sourceforge.net}. Il présente beaucoup d'avantages:
\begin{itemize}
\item il est disponible sous Linux, Mac et Windows
\item il dispose d'un mode "bibatex" (à activer)
\item il est possible d'ajouter facilement des types d'entrées non standards (comme \bibtype{legislation}, \bibtype{jurisdiction} ou \bibtype{commentary}) 
\end{itemize}

\subsection{Conventions de nommage}

Il est très fortement recommandé de suivre une certaine discipline pour le nommage de certaines clés.

\subsubsection{Clés \bibtex}

Chaque référence bibliographique comporte une clé \bibtex devant être absolument unique. Cette clé est passée en paramètre de la commande \cmd{cite} dans le corps du texte, pour créer une citation de cette entrée en bas de page.

Pour les références classiques, une clé formée par des abbréviations de nom d'auteur, de titre et l'année de publication est efficace. Pour éviter les incertitudes, tout est mis en minuscules et |:| est utilisé comme caractère séparateur.

Exemple: 
\begin{lstlisting}[style=bibtex]{}
@THESIS{<<baillon:famillemort:2006>>,
  author = {Baillon-Wirtz, Nathalie},
  title = {La famille et la mort},
  date = {2006},
}
\end{lstlisting}

Pour les références jurisprudentielles, on peut opter pour une forme différente:
\begin{itemize}
\item Pour les \bibtype{jurisdiction}, on peut utiliser le nom de l'institution suivi de la date précise de l'arrêt au format YYYYMMDD;
\item Pour les \bibtype{commentary}, même chose, en indiquant en plus le nom de l'auteur principal du commentaire d'arrêt;
\end{itemize}

Exemple:
\begin{lstlisting}[style=bibtex]{}
@COMMENTARY{<<cass1civ:20100622:dupond>>,
  editor = {Dupond, Albert},
  ...
}

@JURISDICTION{<<cass1civ:20100622>>,
  institution = {Cass 1\iere{} civ.},
  eventdate = {2010-06-22},
  related = {cass1civ:20100622:dupond}
}

\end{lstlisting}

\subsubsection{Clés du champ \texttt{keywords}}

Ce champ est utilisé pour la sélection d'entrées particulières lors de l'invocation de la commande \cmd{printbibliography}.

Des clés courtes et explicites sont à préférer.

\subsubsection{Références croisées}

Pour les références croisées, ne pas hésiter à entrer des clés suffisamment précises pour ne pas créer de doublons. 

\subsection{Usage des guillemets}

\begin{ltxsyntax}

\cmditem{enquote}{texte}
\cmditem{enquote*}{texte}
\label{guillemets}

Les règles de typographie française sont très strictes. Elles imposent l'utilisation de caractères spécifiques pour les guillemets: \enquote{} dans un contexte habituel, et \enquote*{ } à l'intérieur d'autres guillemets. La commande \cmd{enquote} du package \sty{csquotes} permet de gérer ces deux niveaux en fonction du contexte. La variante étoilée permet de forcer l'affichages des guillemets de second niveau \enquote*{ }.

Voici une exemple où les deux niveaux sont utilisés, et le rendu obtenu après compilation:
\begin{ltxexample}
\enquote{Il m'a dit: \enquote{tu...}}
\end{ltxexample}
\enquote{Il m'a dit: \enquote{tu...}}\\

L'utilisation du guillemet anglais |"| est à proscrire. La commande \cmd{enquote} est la seule capable de gérer les deux niveaux de guillemets français; les commandes \cmd{og} et \cmd{fg} proposées par le package \sty{babel} ne gèrent pas les guillemets de second niveau et sont donc insuffisants.

\end{ltxsyntax}

\subsection{Usages particuliers de \biblatex}

Les fonctionnalités du package \biblatex sont pour la plupart faciles à prendre en main, pour peu que l'on s'attaque à sa documentation assez titanesque. Toutefois, certains d'entre eux doivent être utilisés de manière spéciale en français.

Sont donc ici répertoriés certaines options de ce package particulièrement intéressantes, et certaines finesses d'utilisation des champs de références bibliographiques.

\subsubsection{Usages particuliers de champs}

\paragraph{Noms complexes à particule}

Pour rappel, les champs relatifs à des noms d'auteur, tels que le champ \bibfield{author}, peuvent contenir une liste d'éléments distincts délimités à l'aide du délimiteur \texttt{and}. Chaque élément est décomposé à la compilation en quatre parties: prénom, particule (de, von, of), nom de famille, et suffixe (junior, senior, \dots).

Or, certains noms de famille complexes posent des problèmes de rendu car ils ne sont pas facilement décomposables en ces 4 parties.

Pour prévenir ce genre de problème, une paire d'accolades supplémentaires et/ou des espaces insécables peuvent être insérées dans la définition des champs \bibfield{author} et \bibfield{editor}. En voici un exemple:

\begin{lstlisting}[style=bibtex]{}
author = {<<{Delaisi de Parseval}>>, Genevieve and Depadt-Sebag, Valerie},
\end{lstlisting}

\paragraph{Groupes d'auteurs/rédacteurs}

Les champs \bibfield{author} or \bibfield{editor} peuvent contenir des noms de groupes lorsque cela est nécessaire\footnote{Les noms de groupes sont particulièrement fréquents dans les rapports officiels.}. Pour éviter que ces groupes soient traités comme des noms de personnes (prénom, nom, etc), une paire supplémentaire d'accolades est nécessaire. Les champs optionnels \bibfield{shortauthor} et \bibfield{shorteditor} sont destinés à contenir des noms abrégés pour usage dans les citations.

\begin{lstlisting}[style=bibtex]{}
editor       = {<<{Comit\'e Consultatif National d'\'Ethique}>>
                and Dupont, Marcel},
shorteditor  = {CCNE and Dupont, Marcel},
\end{lstlisting}

\paragraph{Titres d'ouvrages}

Les exemples suivants montrent comment gérer les titres dans différents contextes.
Voici un premier exemple avec un ouvrage de 5 volumes:

\begin{lstlisting}[style=bibtex]{}
@Book{works,
  author     = {Shakespeare, William},
  title      = {Collected Works},
  volumes    = {5},
  ...
\end{lstlisting}
%
Chaque volume d'un ouvrage en plusieurs volumes ont généralement un titre spécifique. Supposons que le quatrième volume de l'ouvrage \emph{Collected Works} inclut les sonnets de Shakespeare et que l'on souhaite faire référence à ce volume seul:

\begin{lstlisting}[style=bibtex]{}
@Book{works4,
  author     = {Shakespeare, William},
  maintitle  = {Collected Works},
  title      = {Sonnets},
  volume     = {4},
  ...
\end{lstlisting}
%
Si les différents volumes sont sans titre, utiliser simplement le champ \bibfield{title} et indiquer le numéro de volume:

\begin{lstlisting}[style=bibtex]{}
@Book{works4,
  author     = {Shakespeare, William},
  title      = {Collected Works},
  volume     = {4},
  ...
\end{lstlisting}
%
Dans l'exemple suivant, on souhaite faire référence à la partie titrée d'une volume ayant également son titre propre:

\begin{lstlisting}[style=bibtex]{}
@InBook{lear,
  author     = {Shakespeare, William},
  bookauthor = {Shakespeare, William},
  maintitle  = {Collected Works},
  booktitle  = {Tragedies},
  title      = {King Lear},
  volume     = {1},
  pages      = {53-159},
  ...
\end{lstlisting}

\paragraph{Rôles rédactionnels}
\label{roles}

Le rôle joué par un rédacteur dans les champs dédiés (\ie \bibfield{editor}, \bibfield{editora}, \bibfield{editorb}, \bibfield{editorc}) peut être spécifié dans les champs \bibfield{editor...type} correspondants. 

Les rôles suivantes sont supportés par le style \bibstylename:

\begin{marglist}
\setlength{\itemsep}{0pt}
\item[editor] la personne ayant dirigé la rédaction de l'ouvrage (nom précédé de "sous la dir. de"). Il s'agit du rôle le plus répandu et est la valeur par défaut.
\item[commentator] Auteur d'une commentaire (nom précédé de "comm.").
\item[annotator] Auteur d'une note (nom précédé de "note.").
\item[observator] Auteur d'une observation (nom précédé de "obs.")
\item[chronicler] Auteur d'une chronique (nom précédé de "chron.")
\item[author] Auteur (nom précédé de "par ").
\end{marglist}
%
Voici l'exemple d'un commentaire d'arrêt la personne indiquée en \bibfield{editor} occupe le rôle d'observateur:

\begin{lstlisting}[style=bibtex]{}
@COMMENTARY{...,
  editor      = {Dupond, Marcel},
  editortype  = {<<observator>>},
  ...
\end{lstlisting}
%
L'exemple ci-dessous présente un commentaire d'arrêt dans lequel plusieurs rédacteurs on joué un rôle spécifique:

\begin{lstlisting}[style=bibtex]{}
@COMMENTARY{...
  editor = {Bernard, J},
  editortype = {<<commentator>>},
  editora = {Terr\'e, F.},
  editoratype = {<<annotator>>},
  afterword = {Dontewille}
  ...
}
\end{lstlisting}
%
On notera au passage que le champ \bibfield{afterword} (tout comme \bibfield{foreword}) suit une logique différente.

\paragraph{Dates}
\label{dates}

\begin{table}
\tablesetup
\begin{tabularx}{\textwidth}{@{}>{\ttfamily}llX@{}}
\toprule
\multicolumn{1}{H}{Valeur du champ date} &
\multicolumn{1}{H}{Rendu obtenu} \\
\cmidrule{1-1}\cmidrule(l){2-2}
1850			& 1850 \\
1997/			& 1997-- \\
1967-02			& fév. 1967 \\
2009-01-31		& 31 jan. 2009 \\
1988/1992		& 1988--1992 \\
2002-01/2002-02		& jan. 2002--fév. 2002 \\
1995-03-30/1995-04-05	& 30 Mar. 1995--5 avr. 1995 \\
\bottomrule
\end{tabularx}
\caption{Formats de dates}
\label{bib_use_tab1}
\end{table}

Les champs \bibfield{date} et \bibfield{eventdate} requièrent un format \texttt{yyyy-mm-dd}. Les intervalles de dates requièrement le format \texttt{yyyy-mm-dd\slash yyyy-mm-dd}. Une date de commencement peut être donnée en omettant la date de fin après le slash de séparation (\texttt{yyyy/}). Des exemples sont données \tabref{bib_use_tab1}.

\paragraph{Pagination}
\label{bib_use_pag}

Lorsque un numéro de page ou un intervalle de pages est indiqué dans le champ \bibfield{pages} (ou dans l'argument \prm{postnote} d'une commande de citation), le préfixe "p." est automatiquement ajouté en amont de ce numéro ou intervalle.

Or, certains ouvrages optent pour un système de pagination différent: numéro de paragraphe, de ligne, etc. Les champs \bibfield{pagination} et \bibfield{bookpagination} permet de modifier le comportement par défaut. Pour exemple, considérons l'entrée suivante:

\begin{lstlisting}[style=bibtex]{}
@InBook{..,
  title          = {...},
  pagination     = {verse},
  booktitle      = {...},
  bookpagination = {page},
  pages          = {53--65},
  ...
\end{lstlisting}
%
Le champ \bibfield{bookpagination} affecte le rendu des champs \bibfield{pages} et \bibfield{pagetotal} dans la liste des références bibliographiques. Le champ \bibfield{pagination}, lui, affecte le rendu des informations de pages passée en argument \prm{postnote} des commandes de citation. Lorsque l'un ou l'autre de ces deux champs est laissé vide, la pagination \texttt{page} est appliquée par défaut.

Dans l'exemple ci-dessus, le contenu du champ \bibfield{pages} apparaîtra sous la forme "p.~53--65" (le champ \bibfield{bookpagination} aurait pu être laissé vide, la pagination \texttt{page} étant celle par défaut). Dans une citation du style |\cite[17]{key}|, la postnote sera sous la forme "v.~17".

Les valeurs autorisées pour les champs \bibfield{pagination} and \bibfield{bookpagination} sont: \texttt{page}, \texttt{column}, \texttt{line}, \texttt{verse}, \texttt{section}, \texttt{paragraph} et \texttt{article}. La valeur "\texttt{none}" désactive tout préfixe devant l'information de page.

\subsubsection{Présentation en sous-bibliographies}

En droit, il est d'usage de subdiviser une bibliographie selon certains critères. etc. Ceci est réalisable grace aux arguments passés à la commande \cmd{printbibliography}, décrite précédemment.

Dans l'exemple ci-dessous, on souhaite divider un chapitre bibliographie en deux sections, l'une dédiée au droit français, l'autre au droit européen. Chaque section sera elle-même décomposée en paragraphes, contenant chacun un type de référence.

On commence par définier des titres de paragraphe pour chaque type d'entrée:

\begin{lstlisting}[style=latex]{}
\defbibheading{lois}{\paragraphe{Lois}}
\defbibheading{rapports}{\paragraphe{Rapports officiels}}
...
\end{lstlisting}
%

L'attribution des références dans les deux sections majeures est effectuée à l'aide du champ \bibfield{keywords}: toutes les entrées contenant le mot clé \texttt{french} seront imprimées dans la section française, celles contenant \texttt{eu} dans la section européenne:

\begin{lstlisting}[style=latex]{}
\section{Droit interne}
\printbibliography[<<heading=lois>>,type=legislation,<<keyword=french>>]
\printbibliography[<<heading=rapports>>,type=report,<<keyword=french>>]
...

\section{Droit de l'Europe}
\printbibliography[<<heading=lois>>,type=legislation,<<keyword=eu>>]
\printbibliography[<<heading=rapports>>,type=report,<<keyword=eu>>]
...
\end{lstlisting}
%

\subsubsection{Numéros de page dans les citations}
\label{use_cav_pag}

Si l'argument \prm{postnote} d'une commande de citation est un numéro de page ou un intervalle de pages, le préfixe de pagination est généré (voir \secref{bib_use_pag} sur la pagination). Voici quelques exemples de \prm{postnote} reconnus comme des informations de pages:

\begin{lstlisting}[style=latex]{}
\cite[<<25>>]{key}
\cite[<<vii>>]{key}
\cite[<<XIV>>]{key}
\cite[<<34--38>>]{key}
\cite[<<iv--x>>]{key}
\cite[<<185/86>>]{key}
\cite[<<XI \& XV>>]{key}
\cite[<<3, 5, 7>>]{key}
\cite[<<vii--x; 5, 7>>]{key}
\end{lstlisting}
%

Si la postnote ne contient pas exactement une information purement numérique, elle est imprimée telle quelle. Aussi, il est possible d'écrire le préfixe de pagination manuellement:

\begin{lstlisting}[style=latex]{}
\cite[<<p.~5>>]{key}
\end{lstlisting}
%
Il est possible de supprimer le préfixe de pagination référence par référence dans les citations, en affectant au champ \bibfield{pagination} la valeur "\texttt{none}"; voir \secref{bib_use_pag} pour plsus de détails.

\subsubsection{Tri manuel des références bibliographiques}
\label{use_srt}

Le classement des références bibliographiques actif par défaut dans le style bibliographique \sty{\bibstylename} est le classement \texttt{nyt} se fait d'abord par nom d'auteur, puis par date, enfin par titre.

Cependant, il s'avère que ce système de classement n'est pas le plus adapté dans certains cas.

Il est possible d'influer sur le classement automatique auteur/année/titre grâce au champ spécial \bibfield{presort} dont la format attendu est une lettre alphabétique. Les entrées affublées de la valeur "\texttt{a}" seront imprimées en premier, celles ayant la valeur "\texttt{z}" en dernier. Ce tri peut être affiné en utilisant un deuxième caractère: la valeur "\texttt{aa}" est imprimée en premier, suivi de "\texttt{ab}", "\texttt{ac}", etc. Les entrées dont le champ \bibfield{presort} est laissé vide sont implicitement classées à la valeur "\texttt{mm}".

Prenons un exemple concret de 3 entrées \texttt{ref1}, \texttt{ref1} et\texttt{ref3}, que l'on souhaite afficher dans cet ordre dans la bibliographie, quels que sont leur auteur, titre et année. Le champ presort peut être utilisé comme suit:

\begin{lstlisting}[style=bibtex]{}
@Book{ref1,
  presort = {<<a>>},
  ...
}

@Book{ref2,
  presort = {<<b>>},
  ...
}

@Book{ref3,
  % champ presort indefini => valeur implicite <<"m">>
  ...
}
\end{lstlisting}
%

Le tri manuel des références est particulièrement adapté aux références jurisprudentielles, que l'on classe généralement par importance de juridiction. On définira alors une valeur explicite au champ \bibfield{presort} des entrées de type \bibtype{jurisdiction}.



% TODO
%Erreurs de compilation fréquentes


%\section{Guide du programmeur}
%
%Cette partie s'addresse aux personnes souhaitant comprendre plus en détail le fonctionnement de la classe \sty{\classname} et du style bibliographique afin de l'adapter à des besoins spécifiques. De bonnes connaissances de \latex sont souhaitables avant d'aller plus loin.
%
%\subsection{Style des niveaux hiérarchiques}
%
%classe \sty{\classname}.
%
%\subsection{Style de la bibliographie}
%
%Cette partie s'addresse aux personnes souhaitant modifier le code du style bibliographique. 
%la lecture de la documentation complète du package Biblatex absolument indispensable.
%
%Deux fichiers de styles sont disponibles:
%bbx: style de la bibliographie présentée en fin de document
%cbx: styles des citations (en note de bas de page).
%
%Le fonctionnement général est le suivant: 
%
%pour chaque type d'entrée bibliographique, le fichier bbx définit un "driver". Ce driver expose les champs supportés par le type d'entrée, leur ordre et leur format (entre guillements, en italique...) dans la bibliographie.
%
%La présentation des références dans une citation en bas de page présente des variantes.
%
%l'ordre et le style de pésentatun "driver" est défini


\end{document}

