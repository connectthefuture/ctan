\ProvidesFile{blog.tex}[2013/01/04 documenting blog.sty]
\title{\textsf{blog.sty}\\---\\%
       Generating \HTML\ Quickly with \TeX\thanks{This 
       document describes version 
       \textcolor{blue}{\UseVersionOf{\jobname.sty}} 
       of \textsf{\jobname.sty} as of \UseDateOf{\jobname.sty}.}}
% \listfiles 
{ \RequirePackage{makedoc} 
  \ProcessLineMessage{} 
  %% three section levels 2012/08/07:
  \renewcommand\mdSectionLevelOne  {\string\subsection}
  \renewcommand\mdSectionLevelTwo  {\string\subsubsection}
  \renewcommand\mdSectionLevelThree{\string\paragraph}
  \MainDocParser{\SectionLevelThreeParseInput}
  \HeaderLines{16}              %% 2012/10/03
  \MakeSingleDoc{blog.sty}
  \HeaderLines{18}              %% 2012/11/28
  \MakeSingleDoc{blogligs.sty}
  \HeaderLines{18}              %% 2012/11/28
  \MakeSingleDoc{markblog.sty}
  \HeaderLines{18}              %% 2012/10/05
  \MakeSingleDoc{lnavicol.sty}
  \HeaderLines{17}              %% 2012/10/05
  \MakeSingleDoc{blogdot.sty}
}
\documentclass[fleqn]{article}%% TODO paper dimensions!?
\sfcode`-=1001      %% 2011/10/15
\ProvidesFile{makedoc.cfg}[2011/06/27 documentation settings] 

\author{Uwe L\"uck\thanks{\url{http://contact-ednotes.sty.de.vu}}}
% \author{Uwe L\"uck---{\tt http://contact-ednotes.sty.de.vu}}

%% hyperref:
\RequirePackage{ifpdf}
\usepackage[%
  \ifpdf
%     bookmarks=false,          %% 2010/12/22
%     bookmarksnumbered,
    bookmarksopen,              %% 2011/01/24!?
    bookmarksopenlevel=2,       %% 2011/01/23
%     pdfpagemode=UseNone,
%     pdfstartpage=10,
%     pdfstartview=FitH,
    citebordercolor={ .6 1    .6},
    filebordercolor={1    .6 1},
    linkbordercolor={1    .9  .7},
     urlbordercolor={ .7 1   1},   %% playing 2011/01/24
  \else
    draft
  \fi
]{hyperref}

\RequirePackage{niceverb}[2011/01/24] 
\RequirePackage{readprov}               %% 2010/12/08
\RequirePackage{hypertoc}               %% 2011/01/23
\RequirePackage{texlinks}               %% 2011/01/24
\makeatletter
  \@ifundefined{strong} 
               {\let\strong\textbf}     %% 2011/01/24
               {} 
  \@ifundefined{file} 
               {\let\file\texttt}       %% 2011/05/23
               {} 
\makeatother

\errorcontextlines=4
\pagestyle{headings}

\endinput

 %% shared formatting settings
\usepackage{filesdo} \MDfinaldatechecks             %% 2012/12/20
\ReadPackageInfos{blog}
  %% \tagcode seems to be a quite recent pdfTeX primitive, 
  %% cf. microtype.pdf ... %% 2010/11/06
\newcommand*{\xmltagcode}[1]{\texttt{<#1>}}
\newcommand*{\HTML}{\acro{HTML}}            %% 2011/09/08
\newcommand*{\CSS}{\acro{CSS}}              %% 2011/11/09
\newcommand*{\secref}[1]{Sec.~\ref{sec:#1}} %% 2011/11/23
\providecommand*{\LuaTeX}{Lua\TeX}          %% 2012/12/20
\sloppy
\begin{document}
\maketitle
\begin{abstract}\noindent
'blog.sty' provides \TeX\ macros for generating web pages, 
based on processing text files using the 'fifinddo' package. 
Some \LaTeX\ 
commands %%% command names 
are redefined to access their \HTML\ 
equivalents, other new macro names ``quote" the names of \HTML\ elements. 
The package has evolved in several little steps 
each aiming at getting pretty-looking ``hypertext" \textbf{notes}
with little effort, 
where ``little effort" also has meant avoiding studying 
documentation of similar packages already existing. 
[\textcolor{blue}{TODO:} list them!]
% Version v0.3 is the remainder of v0.2 after moving some stuff 
% to 'fifinddo.sty' (especially `\CopyFile'); 
% moreover, the new `\BlogCopyFile' replaces empty source lines 
% by \HTML's \xmltagcode{p} (starting a new paragraph).---Real 
% \emph{typesetting} from the same `.tex' source 
% (pretty printable output) has not been tried yet.
%% <- 2011/01/24 ->
The package %%% rather 
\emph{``misuses"} \TeX's macro language 
for generating \HTML\ code and entirely \emph{ignores} 
\TeX's typesetting capabilities.%%%---What about 
% such a ``small" \TeX\ with macros only and 
% \emph{no} typesetting capabilities ...!?
---'lnavicol.sty' adds a more \strong{professional} look 
(towards CMS?), and 'blogdot.sty' uses 'blog.sty' 
for \HTML\ \strong{beamer} presentations.
\end{abstract}
\tableofcontents

\section{Installing and Usage}
The file 'blog.sty' is provided ready, 
\strong{installation} only requires 
putting it somewhere where \TeX\ finds it 
(which may need updating the filename data 
 base).\urlfoot{ukfaqref}{inst-wlcf}

\strong{User commands} are described near their implementation below.

However, we must present an \strong{outline} of the procedure 
for generating \HTML\ files: 

At least one \strong{driver} file and one \strong{source} file are 
needed.

The \strong{driver} file's name is stored in `\jobname'. 
It loads 'blog.sty' by 
\begin{verbatim}
  \RequirePackage{blog}
\end{verbatim}
and uses file handling commands from 'blog.sty' and 
\CtanPkgRef{nicetext}{fifinddo}
(cf. `mdoccheat.pdf' from the \ctanpkgref{nicetext} bundle).\urlfoot{CtanPkgRef}{nicetext} 
%% <- \urlfoot 2012/11/30 
It chooses \strong{source} files and the name(s) for the resulting 
\HTML\ file(s). It may also need to load local settings, such as 
%% 2012/11/29: 
`\uselangcode' with the \ctanpkgdref{langcode} package  %% dref 2012/11/30
and settings for converting the editor's text encoding 
into the encoding that the head of the resulting \HTML\ file 
advertises---or into \HTML\ named entities 
(for me, `atari_ht.fdf' has done this).

The driver file could be run a terminal dialogue in order to choose source 
and target files and settings. So far, I rather have programmed a 
dialogue just for converting UTF-8 into an encoding that my 
Atari editor \textsc{xEDIT} can deal with. 
I do not present this now because it was conceptually mistaken, 
I must set up this conversion from scratch some time.
% [TODO: present in 'nicetext'].    %% 2011/01/24

The \strong{source} file(s) should contain user commands defined below 
to generate the necessary \xmltagcode{head} section and the 
\xmltagcode{body} tags. 

\section{Examples}
\subsection{Hello World!}
This is the \strong{source} code for a ``Hello World" example, 
in `hellowor.tex':
\MDsamplecodeinput{hellowor}
The \HTML\ file `hellowor.htm' is generated from `hellowor.tex'
by the following \strong{driver} file `mkhellow.tex':
\pagebreak[2]
\MDsamplecodeinput{mkhellow}

  \iffalse                                  %% 2012/11/29
\subsection{A Very Plain Style}
My ``\TeX-generated 
pages"{\foothttpurlref{www.webdesign-bu.de/%
                       uwe\string_lueck/texmap.htm}}
use a \strong{driver} file `makehtml.tex'. 
To choose a page to generate, I ``uncomment"ed just one 
of several lines that set the ``current conversion job" 
from a list (for some time). 
I choose the example of a simple ``site map:" 
`texmap.htm' is generated from \strong{source} file 
`texmap.tex'.---More recently however, I have started to 
read the job name and perhaps extra settings from a file 
`jobname.tex' that is created by a Bash script.

In order to make it easier for the reader to see what is essential, 
I~have moved many `.cfg'-like extra definitions into a file 
`texblog.fdf'. Some of these definitions may later move into 
`blog.sty'. You should find `makehtml.tex', `texmap.tex', and 
`texblog.fdf' in a directory `demo/texblog' 
(or `texblog.fdf' may be together with the `.sty' files), 
perhaps you can use them as templates.

\begingroup
  \MakeOther\|
%   \MakeOther\`\MakeOther\'  %% disables \tt! 2011/09/08
  \MakeOther\<
  \MakeActive\� \def�{\"o}                  %% 2011/10/10
  \MakeActive\� \def�{\"u}
  \hfuzz=\textwidth \advance \hfuzz by 28pt
\subsubsection{Driver File `makehtml.tex'}
 %% <- TODO \file needs protection for PDF 2011/09/08
 \enlargethispage{1\baselineskip}
  \listinginput[5]{1}{CTAN/morehype/demo/texblog/makehtml.tex}
\subsubsection{Source File \texttt{texmap.tex}}
  \listinginput[5]{1}{CTAN/morehype/demo/texblog/texmap.tex}
\endgroup

  \fi
\subsection{A Style with a Navigation Column}
\label{sec:example-lnavicol}
A style of web pages looking more professional 
% than                                %% rm. 2012/11/29
% `texmap.htm'                        %% was `texhax.hmt' 2011/09/02
(while perhaps becoming outdated) has a small navigation column 
on the left, side by side with a column for the main content. 
Both columns are spanned by a header section above and a footer 
section below. The package 'lnavicol.sty' provides commands 
`\PAGEHEAD', `\PAGENAVI', `\PAGEMAIN', `\PAGEFOOT', `\PAGEEND' 
(and some more) for structuring the source so that the code 
following `\PAGEHEAD' generates the header, the code following 
`\PAGENAVI' forms the content of the navigation column, etc.
Its code is presented in Sec.~\ref{sec:lnavicol}.
For real professionality, somebody must add some fine \acro{CSS}, 
and the macros mentioned may need to be redefined to use the `@class' 
attribute. Also, I am not sure about the table macros in 'blog.sty', 
so much may change later.

With things like these, can 'blog.sty' become a part of a 
``\Wikienref{content management system}" for \TeX\ addicts? 
This idea rather is based on the 
\wikideref{Content Management System}{\meta{German}} 
Wikipedia article.

As an example, I present parts of the source for my 
``home page"{\foothttpurlref{www.webdesign-bu.de/%
                             uwe\string_lueck/schreibt.html}}.
As the footer is the same on all pages of this style, 
it is added in the driver file `makehtml.tex'. 
`schreibt.tex' is the source file for generating `schreibt.html'.
You should find \emph{this} `makehtml.tex', a cut down version of 
`schreibt.tex', and `writings.fdf' with my extra macros for these pages 
in a directory 
`blogdemo/writings',                        %% blog 2012/11/30
hopefully useful as templates.

\begingroup
  \MakeActive\� \def�{\"a}                  %% 2011/10/05
  \MakeActive\� \def�{\"u}
  \hfuzz=\textwidth \advance \hfuzz by 10pt
  %% 2012/11/29 CTAN/morehype/demo -> blogdemo:
\subsubsection{Driver File `makehtml.tex'}
  \listinginput[5]{1}{blogdemo/writings/makehtml.tex}
\subsubsection{Source File `schreibt.tex'}
  \listinginput[5]{1}{blogdemo/writings/schreibt.tex}
\endgroup

\pagebreak                                  %% 2013/01/04
\section{The File \file{blog.sty}}          %% 2011/11/09
% \section{The File \texttt{blog.sty}}
% \section{The File {\tt blog.sty}} 
%% <- strange 2011/11/08 ->
% \section{The File `blog.sty'}             %% 2011/10/04 allow other files
\subsection{Preliminaries}                  %% 2012/10/03
\subsubsection{Package File Header (Legalese)} %% ize -> ese, subsub 2012/10/03
\ResetCodeLineNumbers
\ProvidesPackage{blog}[2013/01/21 v0.81a simple fast HTML (UL)] 
%% copyright (C) 2010 2011 2012 2013 Uwe Lueck, 
%% http://www.contact-ednotes.sty.de.vu 
%% -- author-maintained in the sense of LPPL below.
%%
%% This file can be redistributed and/or modified under 
%% the terms of the LaTeX Project Public License; either 
%% version 1.3c of the License, or any later version.
%% The latest version of this license is in
%%     http://www.latex-project.org/lppl.txt
%% We did our best to help you, but there is NO WARRANTY. 
%%
%% Please report bugs, problems, and suggestions via 
%% 
%%   http://www.contact-ednotes.sty.de.vu 
%%
%% === \cs{newlet} ===                                  %% 2012/10/03
%% |\newlet<cmd><cnd>| is also useful in surrounding files:
\newcommand*{\newlet}[2]{\@ifdefinable#1{\let#1#2}}
%%
%% == Processing ==
%% === Requirement ===
%% We are building on the 'fifinddo' package
%% (using `\protected@edef' for \secref{ligs}):
\RequirePackage{fifinddo}[2011/11/21]
%%
%% === Output File Names ===
%% |\htmakeext| is the extension of the generated file. 
%% Typically it should be `.html', as set here, 
%% but my Atari emulator needs `.htm' 
%% (see `texblog.fdf'):
\newcommand*{\htmakeext}{.html}
%%
%% === General Insertions ===
%% |\CLBrk| is a \emph{code line break} 
%% (also saving subsequent comment mark in macro definitions):
\newcommand*{\CLBrk}{^^J}
%% |\ | is turned into an alias for `\space', 
%% so it inserts a blank space. It even works at line ends, 
%% thanks to the choice of `\endlinechar' in \secref{catcodes}.
\let\ \space
%% %% v0.42:
%% |\ProvidesFile{<file-name>.tex}[<file-info>]| \ is supported 
%% for use with the \CtanPkgRef{morehype}{myfilist} 
%% package  %% TODO \urlpkgfoot!? 2011/02/22
%% to get a list of source file infos. 
%% In generating the \HTML\ file, the file infos are transformed 
%% into an \HTML\ comment. Actually it is 
%% |\BlogProvidesFile| (for the time being, 2011/02/22):
\@ifdefinable\BlogProvidesFile{%
    \def\BlogProvidesFile#1[#2]{%
        <!DOCTYPE html>\CLBrk               %% TODO more!? 2012/09/06
        \comment{ generated from\CLBrk\CLBrk
                  \ \ \ \ \ \ \ \ \ #1, #2,\CLBrk\CLBrk
                  \ \ \ \ \ with blog.sty,
                  \isotoday\ }}} 
\edef\isotoday{%% texblog 2011/11/02, here 2011/11/20
   \the\year-\two@digits{\the\month}-\two@digits{\the\day}}
%% %% changes 2011/02/24:
%% (TODO: customizable style.)---Due to the limitations 
%% of the approach reading the source file
%% line by line, the ``optional argument" `[<file-info>]' 
%% of `\ProvidesFile' must appear in the same line as 
%% the closing brace of its mandatory argument.
%% The feature may require inserting 
%% \[`\let\ProvidesFile\BlogProvidesFile'\] 
%% somewhere, e.g., in `\BlogProcessFile'. 
%%
%% === Category Codes etc. ===
%% \label{sec:catcodes}
%% For a while, line endings swallowed inter-word spaces, 
%% until I found the setting of `\endlinechar' 
%% ('fifinddo''s default is `-1') 
%% in |\BlogCodes|:
\newcommand*{\BlogCodes}{%                              %% 2010/09/07
    \endlinechar`\ %
%% <- Comment character to get space rather than `^^M'!%%% 2011/11/08
%% ---The tilde |~| is active as in Plain~TeX too, it is so natural to 
%% use it for abbreviating \HTML's `&nbsp;'! %% moved up 2011/11/08
%     \catcode`\~\active
    \MakeActiveDef\~{&nbsp;}%%          for \FDpseudoTilde 2012/01/07
%% \qtd{&'} for \HTML\ convenience (cf. \secref{quotes}):
    \MakeActiveLet\'\rq                        %% actcodes 2012/08/28
    \BasicNormalCatCodes}
% \MakeOther\< \MakeOther\>                         %% rm. 2011/11/20
%%
%% === The Processing Loop ===
%% \[|\BlogProcessFile[<changes>]{<source-file>}|\]
%% %% <- display 2011/11/06
%% ``copies" the \TeX\ source file <source-file> 
%% into the file specified by `\ResultFile'. 
\newcommand*{\BlogProcessFile}[2][]{%                   %% 2011/11/05
    \ProcessFileWith[\BlogCodes 
                     \let\ProvidesFile\BlogProvidesFile %% 2011/02/24 
                     \let\protect\@empty                %% 2011/03/24
                     \let\@typeset@protect\@empty       %% 2012/03/17
                     #1]{#2}{%
        \IfFDinputEmpty
            {\IfFDpreviousInputEmpty
                \relax
                {\WriteResult{\ifBlogAutoPars<p>\fi}}}% 
            \BlogProcessLine                            %% 2011/11/05
    }%
}
%% 'fifinddo' v0.5 allows the following                 %% 2011/11/20
%% \[|\BlogProcessFinalFile[<changes>]{<source-file>}|\]
%% working just like `\BlogProcessFile' except that the final 
%% `\CloseResultFile' is issued automatically, no more need 
%% having it in the driver file. 
\newcommand*{\BlogProcessFinalFile}{%
    \FinalInputFiletrue\BlogProcessFile}
%% TODO:                                                %% 2011/11/20
%% optionally include `.css' code with \xmltagcode{style}.
%% === \emph{Executing} Source File Code Optionally ===
%% For v0.7,                                            %% 2011/11/05 
%% `\BlogCopyFile' is renamed     `\BlogProcessFile'; and in its code,
%% `\CopyLine'     is replaced by `\BlogProcessLine'. The purpose 
%% of this is supporting 'blogexec.sty' that allows intercepting 
%% certain commands in the line. We provide initial versions of 
%% 'blogexec''s switching commands that allow invoking 'blogexec'
%% ``on the fly": 
\newcommand*{\ProvideBlogExec}{\RequirePackage{blogexec}}
%% \CtanPkgRef{dowith}{dowith.sty} is used in the       %% 2012/01/06
%% present package to reduce package code and documentation space:
\RequirePackage{dowith}
\setdo{\providecommand*#1{\ProvideBlogExec#1}}
\DoDoWithAllOf{\BlogInterceptExecute \BlogInterceptEnvironments
               \BlogInterceptExtra   \BlogInterceptHash        }
%% %% <- Hash 2011/11/08
%% |\BlogCopyLines| switches to the ``copy only" 
%% (``compressing" empty lines) functionality of the original 
%% `\BlogCopyFile':
\newcommand*{\BlogCopyLines}{%
%     \let\BlogProcessLine\CopyLine}
    \def\BlogProcessLine{%          %% 2011/11/21, corr. 2012/03/14:
        \WriteResult{\ProcessInputWith\BlogOutputJob}}}
%% <- This is a preliminary support for ``ligatures"---see 
%% \secref{ligs}. %%% --, not working inside braces. 
%% |\NoBlogLigs| sets the default to mere copying:
\newcommand*{\NoBlogLigs}{\def\BlogOutputJob{LEAVE}}
\NoBlogLigs
%% TODO more from `texblog.fdf' here, problems with `writings.fdf', 
%% see its `makehtml.tex'
%% %% 2011/11/06:
%% % ... and this will be the setting with pure 'blog.sty':
%%
%% `\BlogCopyLines' %% 2011/11/21
%% will be the setting with pure 'blog.sty':
\BlogCopyLines
%% OK, let's not remove |\BlogCopyFile| altogether, rebirth:
\newcommand*{\BlogCopyFile}{\BlogCopyLines\BlogProcessFile}
%% 
%% === ``Ligatures", Package Options ===
%% \label{sec:ligs}
%% With v0.7, we introduce a preliminary method to use the 
%% ``ligatures" `--' and `---' with pure expansion. 
%% At this occasion, we also can support the notation 
%% \code{\dots} for `\dots', as well as arrows 
%% (as in `mdoccorr.cfg'). Note that this is somewhat 
%% \strong{dangerous}, especially the source must not 
%% contain ``explicit" \HTML\ comment, comments must use 
%% 'blog.sty''s `\comment' or the `{commentlines}'
%% environment. Therefore these ``ligatures" must be activated 
%% explicitly by |\UseBlogLigs|: 
\newcommand*{\UseBlogLigs}{\def\BlogOutputJob{BlogLIGs}}
%% In order to work inside braces, the source file better should 
%% be preprocessed in ``plain text mode."
%% (TODO: Use `\ifBlogLigs', and in a group use 
%%        `\ResultFile' for an intermediate `\htmljob.lig'.
%% And TODO: Use `\let\BlogOutputJob'.)
%% On the other hand, the present approach allows switching 
%% while processing with `\EXECUTE'! Also, intercepted commands 
%% could apply the replacements on their arguments---using 
%% |\ParseLigs{<arg>}|:                                 %% 2011/11/27
\newcommand*{\ParseLigs}[1]{\ProcessStringWith{#1}{BlogLIGs}}
%% (`\ProcessStringWith' is from 'fifinddo'.)---The package 
%% 'blogligs.sty' described in \secref{moreligs} does these 
%% things in a more powerful way. You can load it by calling 
%% 'blog.sty''s package option |[ligs]| (v0.8):
\DeclareOption{ligs}{\AtEndOfPackage{\RequirePackage{blogligs}}}
%% 
%% The replacement chain follows (TODO move to `.cfg').
%% As opposed to the file `mdoccorr.cfg' for 'makedoc.sty', 
%% we are dealing with ``normal \TeX" code 
%% (regarding category codes, 'fifinddo.sty' 
%%  as of 2011/11/21 is needed for `\protect').
%% Moreover, space tokens after patterns are already there 
%% and need not be inserted after control sequences.
\FDpseudoTilde
\StartPrependingChain
\PrependExpandableAllReplacer{blog...}{...}{\protect\dots}
\PrependExpandableAllReplacer{blog--}{--}{\protect\endash}
\PrependExpandableAllReplacer{blog---}{---}{\protect\emdash}
%% <- Cf. thin surrounding spaces with 
%% `\enpardash' ('texblog', maybe \meta{hair space} U+200A 
%% instead of thin space), difficult at code line beginnings 
%% or endings and when a paragraph starts with an emdash.
%% I.e., perhaps better don't use it if you want to have 
%% such spaces.---\qtd{`---'} must be replaced before \qtd{`--'}!
\PrependExpandableAllReplacer{blog->}{->}{\protect\to}
\PrependExpandableAllReplacer{blog<-}{<-}{\protect\gets}
%% You also could set `\BlogOutputJob' to a later part of the chain, 
%% or more globally change the following:
\CopyFDconditionFromTo{blog<-}{BlogLIGs}
%% % TODO: Seems to slow down processing by a major part of a second.
%% % Reduce replacements? Maybe do all of this within source file!?
%% The package 'markblog.sty' described in \secref{mark} extends this 
%% to some markup resembling \Wikiref{wiki} editing.
%% This package may be loaded by 'blog.sty''s package option |[mark]| 
%% (v0.8):
\DeclareOption{mark}{\AtEndOfPackage{\RequirePackage{markblog}}}
%% 
%% === \xmltagcode{p} from Empty Line, Package Option ===
%% \label{sec:autopars}
%% As in \TeX\ an empty line starts a new paragraph, 
%% we might ``interpret" an empty source line as 
%% \HTML\ tag \xmltagcode{p} for starting a new paragraph. 
%% Empty source lines following some first empty source 
%% line immediately are ignored 
%% (``compression" of empty lines). 
%% However, this sometimes has unwanted effects 
%% (comment lines TODO),                        %% 2011/11/20
%% so it must be required explicitly by |\BlogAutoPars|, 
%% or by calling the package with option |[autopars]|.
%% In the latter case, it can be turned off by 
%% |\noBlogAutoPars|
\newif\ifBlogAutoPars
\newcommand*{\BlogAutoPars}{\BlogAutoParstrue}
\newcommand*{\noBlogAutoPars}{\BlogAutoParsfalse}
%% `\BlogAutoPars' is issued by package option          %% 2011/11/20
%% |[autopars]|:
\DeclareOption{autopars}{\BlogAutoPars}
\ProcessOptions
%% See \secref{p-br} for other ways of breaking paragraphs.
%%
%% == General \HTML\ Matters ==
%% The following stuff is required for any web page 
%% (or hardly evitable).
%% === General Tagging ===
%% %% $$ -> \[ ... 2011/10/10 el-name -> elt-name 2012/09/14:
%% \[|\TagSurr{<elt-name>}{<attr>}{<content>}|\]
%% (I hoped this way code would be more readable 
%%  than with `\TagSurround' ...)
%% and \[|\SimpleTagSurr{<elt-name>}{<content>}|\]
%% are used to avoid repeating element names <elt-name> in 
%% definitions of \TeX\ macros that refer to ``entire" 
%% elements---as opposed to elements whose content often
%% spans lines (as readable \HTML\ code). 
%% We will handle the latter kind of elements 
%% using \LaTeX's idea of ``environments."
%% `\TagSurr' also inserts specifications of element
%% \textbf{attributes}, 
%% [TODO: 'wiki.sty' syntax would be so nice here]
%% while `\SimpleTagSurr' is for elements used without 
%% specifying attributes. |\STS| is an abbreviation for 
%% `\SimpleTagSurr' that is useful as the `\SimpleTagSurr'
%% function occurs so frequently: 
\newcommand*{\SimpleTagSurr}[2]{<#1>#2</#1>}
\newlet\STS\SimpleTagSurr                           %% 2010/05/23
%% %% 2012/09/08:
%% With the space in `\declareHTMLattrib' as of 2012/08/28, 
%% we remove the space between #1 and #2. 
%% (Doing this by an option may be better TODO; 
%%  any separate attribute definitions must take care of this.)
% \newcommand*{\TagSurr}[3]{<#1#2>#3</#1>}
%% ... undone 2012/11/16, bad with ``direct" use of #2 
%% (with attributes not declared):
\newcommand*{\TagSurr}[3]{<#1 #2>#3</#1>}
%%
%% === Attributes === %%% (General) %%% 
%% Inspired by the common way to use `@' for referring 
%% to element attributes---i.e., `@<attr>' refers to attribute
%% `<attr>'---in \HTML/\acro{XML} documentation, we often use 
%% \[`\@<attr>{<value>}' \qquad 
%%   \mbox{to ``abbreviate"}\qquad  `<attr>="<value>"'\] 
%% within the starting tag of an \HTML\ element.
%% This does not really 
%% make typing easier or improve readability, 
%% it rather saves \TeX's memory by using a single token 
%% for referring to an attribute.
%% This ``abbreviation" is declared by 
%% |\declareHTMLattrib{<attr>}|, even with a check 
%% whether `\@<attr>' has been defined before: 
\newcommand*{\declareHTMLattrib}[1]{%
  \def\reserved@a{@#1}%
  \@ifundefined\reserved@a                   %% \res... 2012/09/06
               {\@namedef{@#1}##1{ #1="##1"}}%% space   2012/08/28
                \@notdefinable}
%% So after `\declareHTMLattrib{<attr>}', `\@<attr>' is a 
%% \TeX\ macro expecting one parameter for the specification. 
%%
%% % |\declareHTMLattribs{{<attr>}<list>}| essentially issues 
%% % \[`\declareHTMLattrib{<attr>}\declareHTMLattribs{<list>}'\]
%% % while `\declareHTMLattribs{}' essentially does nothing---great, 
%% % this is an explanation by recursion!
%% % % \newcommand*{\declareHTMLattribs}{\DoWithAllOf\declareHTMLattrib}
%% A few frequent attributes are declared this way here. 
%% %% 2011/09/27:
%% |@class|, |@id|, |@style|, |@title|, |@lang|, and |@dir|
%% are the ones named on 
%% \wikienref{HTML\#Attributes}{\meta{Wikipedia}}:
\let\@class\relax     %% for tab/arr in latex.ltx
\let\@title\relax     %% for \title in latex.ltx, %% 2011/04/26
\DoWithAllOf\declareHTMLattrib{{class}{id}{style}{title}{lang}{dir}}
%% |@type| is quite frequent too:               %% doc. 2011/09/27
\declareHTMLattrib{type}
%% |@href| is most important for that ``hyper-text:"
\declareHTMLattrib{href} 
%% ... and |@name| (among other uses) is needed for 
%% hyper-text anchors: 
\declareHTMLattrib{name}                        %% 2010/11/06
%% |@content| appears with `\MetaTag' below:    %% 2011/11/05
\declareHTMLattrib{content} 
%% |@bgcolor| is used in tables as well as 
%% for the appearance of the entire page:
\declareHTMLattrib{bgcolor} 
%% 
%% %%% Attributes (Tables) %%%
%% Of course, conflicts may occur, as the form 
%% `\@<ASCII-chars>' of macro names is used for internal 
%% (La)\TeX\ macros. 
%% Indeed, `\@width' that we want to have for the |@width|
%% attribute already ``abbreviates" 
%% \TeX's ``keyword" (\TeX book p.~61) %% 2010/11/27
%% `width' in \LaTeX\ 
%% (for specifying the width of a `\hrule' or `\vrule' from \TeX; 
%%  again just saving \TeX\ tokens rather than for readibility).
\PackageWarning{blog}{Redefining \protect\@width}
\let\@width\relax
\declareHTMLattrib{width} 
%% Same with |@height|:
\PackageWarning{blog}{Redefining \protect\@height}
\let\@height\relax
\declareHTMLattrib{height}              %% 2010/07/24
%% \pagebreak[3]                                        %% 2012/09/08
%% We can enumerate the specifications allowed for 
%% |@align|:                            %% reimpl. 2011/04/24
\newcommand*{\@align@c}{\@align{center}} 
\newcommand*{\@align@l}{\@align{left}} 
\newcommand*{\@align@r}{\@align{right}} 
% \newcommand*{\@align}[1]{ align="#1"}
\declareHTMLattrib{align}               %% 2012/09/08
%% |@valign@t|:
% \newcommand*{\@valign@t}{v\@align{top}} %% 2011/04/24
\newcommand*{\@valign@t}{ valign="top"} %% 2012/09/08
%% Some other uses of `\declareHTMLattrib' essential for 
%% \emph{tables:}                       %% emph 2011/04/24
\declareHTMLattrib{border}              %% 2011/04/24
\declareHTMLattrib{cellpadding}         %% 2010/07/18
\declareHTMLattrib{cellspacing}         %% 2010/07/18
\declareHTMLattrib{colspan}             %% 2010/07/17
\declareHTMLattrib{frame}               %% 2010/07/24
%% \textbf{Another problem} with this namespace idea is that 
%% \emph{either} this reference to attributes cannot be used in 
%% ``author" source files for generating HTML---\emph{or} `@' 
%% cannot be used for ``private" (internal) macros. 
%% % Cf. |\ContentAtt| for \xmltagcode{meta} tags ...
%% % well, not so bad, as the main purpose of this namespace 
%% % is saving tokens \emph{in macros.}
%%
%% === Hash Mark ===
%% |#| is needed for numerical specifications in \HTML, 
%% especially colours and Unicode symbols,
%% while it plays a different (essential) role in our 
%% definitions of \TeX\ macros here. 
%% We redefine \LaTeX's |\#| 
%% for a kind of ``quoting" `#' 
%% (in macro definitions)                     %% (...) 2011/11/08 
%% in order to refer to their \HTML\ meaning.%%%---I wonder
%% %% <- rm. 2011/11/08 ->
%% % what I had in mind with the |&| things here. 
%% % I cannot find any use of |\AmpMark| in my code 
%% % (including my web pages). There is no real problem 
%% % with calling special \HTML\ symbols, `&' is simply 
%% % made `other' already here for macros calling those symbols 
%% % (below), and in processing source files, it is as well 
%% % `other' by default. The symbols section, however, redefines |\&| 
%% % for calling \HTML's ampersand symbol. 
{ \MakeOther\# \gdef\#{#}                       %% \M... 2011/11/08
% \catcode`\&=12 \gdef\AmpMark{&}               %%   rm. 2011/11/08
}
%% ... `\CompWordMark' etc.?
%%
%% === ``Escaping" \HTML\ Code for ``Verbatim" ===
%% \label{sec:html-verb}                                %% 2011/11/23
%% |\xmltagcode{<chars>}| yields \qtd{\xmltagcode{<chars>}}:
\newcommand*{\xmltagcode}[1]{\code{\lt#1\gt}} 
%% %% elements 2012/09/14:
%% |\xmleltcode{<name>}{<content>}| displays the code for an entire 
%% <name> element containing <content> without attributes:
\newcommand*{\xmleltcode}[2]{\code{\lt#1\gt#2\lt/#1\gt}} 
%% |\xmleltcode{<name>}{<attrs>}{<content>}| displays the code for an entire 
%% <name> element \emph{with} attribute text \qtd{<attrs>} 
%% containing <content>:
\newcommand*{\xmleltattrcode}[3]{\code{\lt#1 #2\gt#3\lt/#1\gt}} 
%% |\xmlentitycode{<name>}| yields the code \qtd{`&&<name>;'} for an 
%% entity with name <name>:
\newcommand*{\xmlentitycode}[1]{\code{\&#1;}}
%%
%% === Head ===
%% |\head| produces the first two tags that an \HTML\ file must start: 
\newcommand*{\head}{<html><head>}                 %% ^^J rm 2010/10/10
%% |\MetaTag{<inside>}| creates a \xmltagcode{meta} tag:
\newcommand*{\MetaTag}[1]{\indenti<meta #1>} 
%% |\charset{<code-page>}|                  %% Content-T -> content-t 2012/09/06
\newcommand*{\charset}[1]{%
  \MetaTag{ http-equiv="content-type"\@content{text/html; #1}}}
  %% <- space 2012/09/08
%% \pagebreak[3]                                                %% 2012/09/07
%% |\metanamecontent{<name>}{<content>}| obviously:             %% 2012/09/06
\newcommand*{\metanamecontent}[2]{%
    \MetaTag{\@name{#1}\@content{#2}}}
%% %% 2012/09/07:
%% |\author{<name>}| and |\date{<date>}| set according metadata, 
%% somewhat opposing \LaTeX\ (TODO!?):
\renewcommand*{\author}{\metanamecontent{author}}
\renewcommand*{\date}{\metanamecontent{date}}
%% The name of |\metadescription{<text>}| allows using 
%% `\begin{description}' (cf. secref{env}):
\newcommand*{\metadescription}{\metanamecontent{description}}
%% |\keywords{<text>}|:
\newcommand*{\keywords}{\metanamecontent{keywords}}
%% |\robots{<instructions>}|:                       %% using above 2012/09/06
\newcommand*{\robots}{\metanamecontent{robots}}
    %% #2 juergenf: index, follow, noarchive
%% |\norobots| for privacy                                      %% 2011/03/16
%% (cf. \urlhttpref{noarchive.net/meta}                         %% 2012/09/07
%%  and \wikienref{Robots meta tag\#The_robots_attribute}{%
%%  \meta{Wikipedia}}:
\newcommand*{\norobots}{\robots{noarchive,nofollow,noindex}}
%% |\metanamelangcontent{<name>}{<lang>}{<content>}|,\\         %% 2012/09/06
%% in addition to the above, uses language code <lang>:         %% 2012/09/07
\newcommand*{\metanamelangcontent}[3]{%
    \MetaTag{\@name{#1}\@lang{#2}\@content{#3}}}
%% So there can be language-dependent descriptions and keywords:\\
%% |\langdescription{<text>}| and |\langkeywords{<>}|
\newcommand*{\langdescription}{\metanamelangcontent{description}}
\newcommand*{\langkeywords}   {\metanamelangcontent{keywords}}
%% |\stylesheet{<media>}{<css>}| uses <css>`.css' for 
%% `media="<media>"':
\newcommand*{\stylesheet}[2]{%
  \space\space                          %% 2010/09/10
  <link rel="stylesheet" media="#1"%
        \@type{text/css}%               %% \@type 2011/10/05
        \@href{#2.css}>}
%% %% 2011/10/05:
%% Alternatively, style declarations may occur in the 
%% \xmltagcode{style} element. It can be accessed by the 
%% |{style}| environment (cf. \secref{env}):
\newenvironment*{style}[1]
                {<style\@type{text/css} media="#1">}
                {</style>}
%% With |\title{<text>}|, <text> heads %% grammar? TODO 2010/04/08
%% the browser window: 
%% % \renewcommand*{\title}[1]{\space\space<title>#1</title>} 
%% %% <- 2011/12/12 ->
\renewcommand*{\title}{\space\space\SimpleTagSurr{title}} 
%%
%% === Body === 
%% |\body| separates the `head' element from the `body' element 
%% of the page.
\newcommand*{\body}{</head><body>}
%% |\topofpage| generates an anchor `top-of-page':
\newcommand*{\topofpage}{\hanc{top-of-page}{}}
%% |\finish| finishes the page, closing the `body' and `html'
%% elements.
\newcommand*{\finish}{</body></html>}
%%
%% === Comments ===
%% |\comment{<comment>}| produces a one-line \HTML\ comment. 
%% By contrast, there is an environment 
%% |{commentlines}{<comment>}|
%% for multi-line comments.           %% was mult-l... 2011/09/08
%% It is convenient for ``commenting out" 
%% code (unless the latter contains other \HTML\ comments ...)
%% where <comment> is a \emph{comment} for explaining what is 
%% commented out.                                   %% 2010/11/14
\newcommand*{\comment}[1]{<!--#1-->}
% \newcommand{\commentlines}[1]{\comment{^^J#1^^J}} %% 2010/05/07
%   %% <- TODO bzw. \endlinechar=`\^^J 2010/05/09 back 2010/05/10
\newenvironment{commentlines}[1]                    %% 2010/05/17
  {<!--#1}
  {-->}
%% === \CSS ===       %% 2012/07/30 spanstyle -> stylespan 2012/10/28
%% |\stylespan{<css-style>}{<text>}| applies the \CSS\ styling 
%% <css-style> to <text>: 
\newcommand*{\stylespan}[1]{\TagSurr{span}{\@style{#1}}}
%% Not sure about \xmltagcode{div} yet ... TODO
%%
%% == Paragraphs and Line Breaks ==
%% \label{sec:p-br}
%% 2010/04/28:
%% \xmltagcode{br} for manual line breaking 
%% can be generated either by |\newline| or by |\\|: 
\renewcommand*{\newline}{<br>}
\let\\\newline
%% Automatical insertion of \xmltagcode{p} tags 
%% for starting new paragraphs according to 
%% \secref{autopars} has been difficult, 
%% especially comment lines so far insert unwanted 
%% paragraph breaks (TODO 2011/11/20).
%% So here are some ways to use \LaTeX/ Plain \TeX\ 
%% commands---or ...:                                   %% 2011/11/23
% \def\par{<p>} %% + empty lines !? 2010/04/26
%% <- difficult with `\stop'; 2010/09/10:
%% |\endgraf| produces \xmltagcode{p}---TODO!?     %% todo 2011/11/23
\renewcommand*{\endgraf}{<p>}                  %% was </p> 2012/11/19
%% %% 2011/11/23:
%% However, I rather have decided for inserting a literal 
%% \qtd{\xmltagcode{p}} using an editor (keyboard) shortcut.
%% 
%% |\rightpar{<text>}| places <text> flush right. 
%% I have used this for \qtd{Last revised \dots} and for placing 
%% navigation marks. 
\newcommand*{\rightpar}{\TagSurr p\@align@r}            %% 2010/06/17
%% %% 2011/04/29:
%% Often I use `\rightpar' with \textit{italics}, 
%% now there is |\rightitpar{<text>}| for this purpose:
\newcommand*{\rightitpar}[1]{\rightpar{\textit{#1}}}
%%
%% == Physical Markup (Inline) ==                       %% mod. 2012/09/13
%% \label{sec:phys}
%% We ``re-use" some \LaTeX\ commands for specifying font attributes, 
%% rather than (re)defining macros `\i', `\b', `\tt', ...
%% \[|\textit{<text>}| \qquad 
%%   \mbox{just expands to}\qquad 
%%   \xmltagcode{i}<text>\xmltagcode{/i}\]
\renewcommand*{\textit}{\SimpleTagSurr i}
%% etc. for |\textbf|, |\texttt| ...:
\renewcommand*{\textbf}{\SimpleTagSurr b}
\renewcommand*{\texttt}{\SimpleTagSurr{tt}}             %% 2010/06/07
%% |\textsf{<text>}| chooses some sans-serif:           %% 2011/10/20 2012/07/30
\renewcommand*{\textsf}{\stylespan{font-family:sans-serif}}
%% |\textup{<text>}| may undo surrounding slanting or ...:  %% 2012/08/01
\renewcommand*\textup{\stylespan{font-style:normal}}
%% |\textcolor{<color>}{<text>}|                            %% args 2012/08/01
%% is from \LaTeX's 'color' package 
%% that we won't load for generating \HTML, 
%% so it is ``new" here, it is just natural to use it 
%% for coloured text.                                       %% (2010/05/15):
%% \xmltagcode{font} is deprecated, use                     %% 2011/10/20
%% \xmltagcode{span} instead:                               %% \span... 2012/07/30:
\newcommand*{\textcolor}[1]{\stylespan{color:#1}}
%% %% moves here from "Misc." 2012/09/13:
%% \TeX/\LaTeX's |\underbar{<text>}| is redirected to the \xmltagcode{u}
%% element:                                             %% 2012/06/07
\renewcommand*{\underbar}{\SimpleTagSurr u}
%%
%% == Logical Markup == 
%% %% \heading moves here from former section 2012/09/13:
%% |\heading{<level>}{<text>}| prints <text> with size dependent 
%% on <level>. The latter may be one out of 
%% `1', `2', `3', `4', `5', `6'.                    %% 2011/05/12
\newcommand*{\heading}[1]{\SimpleTagSurr{h#1}}
%% ... I might use `\section' etc. one day, I made `\heading'
%% when I could not control the sizes of the section titles 
%% properly and decided first to experiment with the level numbers.
%% %% <- TODO re-order alltogether, cf. SelfHTML       2011/05/12
%%
%% |\code{<text>}| marks <text> as ``code," 
%% just accessing te \xmltagcode{code} element, 
%% while standard \LaTeX\ does not provide a `\code' command:
\newcommand*{\code}{\SimpleTagSurr{code}}   %% 2010/04/27
%% |\emph{<text>}| is \LaTeX's command again, but somewhat abused, 
%% expanding to \lq\xmltagcode{em}<text>\xmltagcode{/em}\rq:
\renewcommand*{\emph}  {\SimpleTagSurr{em}}
%% ... Note that \LaTeX's `\emph' feature of switching to `up' 
%% when `\emph' appears in an italic context doesn't work here ...
%% 
%% |\strong{<text>}| again just calls an \HTML\ element. 
%% It may behave like `\textbf{<text>}', or ... I don't know ...
\newcommand*{\strong}{\SimpleTagSurr{strong}}
%% %% 2011/03/09:
%% |\var{<symbol(s)>}| accesses the \xmltagcode{var} element:
\newcommand*{\var}{\SimpleTagSurr{var}}
%% %% 2011/09/21, 2012/01/06:
%% For tagging acronyms, 
%% \HTML\ offers the \xmltagcode{acronym} element, 
%% and the \httpref{ctan.org/pkg/tugboat}{\acro{TUG}boat macros} 
%% provide |\acro{<LETTERS>}|. I have used the latter for some 
%% time in my package documentations anyway. For v0.7, %% 2012/01/06
%% I add the latter here as an alias for |\acronym{<LETTERS>}|
%% (supporting both naming policies mentioned in \secref{name-policy}):
\newcommand*{\acronym}{\SimpleTagSurr{acronym}}
\newlet\acro\acronym
%% %% 2012/02/04:
%% |\newacronym{<LETTERS>}| saves you from doubling the <LETTERS>
%% when you want to create the shorthand macro `\<LETTERS>': 
\newcommand*{\newacronym}[1]{%
    \expandafter\newcommand\expandafter*\csname#1\endcsname{%
        \acronym{#1}}}
%% However, \xmltagcode{acronym} is 
%% \wikienref{HTML elements#Phrase_elements}{deprecated}. 
%% You may use |\abbr{<LETTERS>}| and |\newabbr{<LETTERS>}| instead: 
\newcommand*{\abbr}{\SimpleTagSurr{abbr}}               %% 2012/09/13
\newcommand*{\newabbr}[1]{%
    \expandafter\newcommand\expandafter*\csname#1\endcsname{%
        \abbr{#1}}}
%%
%% == Environments ==
%% \label{sec:env}
%% We reduce \LaTeX's |\begin| and |\end| to their most primitive 
%% core.
%% \begin{description}
%% \cmdboxitem|\begin{<command>}| just executes the macro `\<command>', 
%% and 
%% \cmdboxitem|\end{<command>}| just executes the macro `\end<command>'.
%% \end{description}
%%
%% They don't constitute a group with local settings. 
%% Indeed, the present (2010/11/07) version of 'blog.sty'
%% does not allow any assignments while ``copying" 
%% the \TeX\ source into the `.htm'.
%% There even is no check for proper nesting.
%% `\begin' and `\end' just represent \HTML\ elements 
%% (their starting/ending tags)
%% that typically have ``long" content. 
%% (We might ``intercept" `\begin' and `\end' before 
%%  copying for executing some assignments in a future version.)
\let\begin\@nameuse
\def\end#1{\csname end#1\endcsname}
%% ... moving |{english}| to `xmlprint.cfg' 
%% 2010/05/22 ...%% doc 2010/11/06
%%
%% As formerly with 
%% \hyperref[sec:phys]{physical markup},      %% \hyperref 2012/01/06
%% we have \emph{two} policies for
%% \label{sec:name-policy}                   %% 2012/01/06 2012/11/28
%% \textbf{choosing macro names}: 
%% (i)~using an \emph{existing} \HTML\ element name, 
%% (ii)~using a \LaTeX\ command name for accessing a somewhat 
%% similar \HTML\ element having a \emph{different} name. 
%% [\,2011/10/05: so what? TODO\,]
%%
%% New 2011/10/05:
%% With |\useHTMLelement{<ltx-env>}{<html-el>}|, 
%% you can access the \xmltagcode{<html-el>} element 
%% by the `<ltx-env>' environment. 
%% The ``starred" form is for ``list" environments 
%% where I observed around 2011/10/01 that certain links 
%% (with Mozilla Firefox) need \xmltagcode{/li}: 
\newcommand*{\useHTMLelement}{%
    \@ifstar{\@useHTMLelement[</li>]}{\@useHTMLelement}}
\newcommand*{\@useHTMLelement}[3][]{%
    \@namedef{#2}{<#3>}%
    \@namedef{end#2}{#1\CLBrk</#3>}}        %% \CLBrk 2012/04/03
%% Applications: 
%%
%% \acro{CARE:} |{small}| is an environment here, 
%% it is not in \LaTeX:
\useHTMLelement{small}{small}
%% |{center}|: %% TODO von 2010/07/18, weiter oben
% \renewenvironment*{center}{<p align="center">}{</p>}
% \renewenvironment*{center}{<p \@align@c>}{</p>}
\useHTMLelement{center}{center}
%% The next definitions for |{enumerate}|, |{itemize}|, 
%% `{verbatim}' follow policy~(ii):
\useHTMLelement*{enumerate}{ol}
\useHTMLelement*{itemize}  {ul}
%% |\begin{enumtype}{<type>}| starts an enumeration 
%% environment with enumeration type <type>
%% which can be one out of `1', `a', `A', `i', `I' 
%% (somewhat resembling the functionality of the 
%%  \ctanpkgref{enumerate} package):
\newenvironment{enumtype}[1]{<ol \@type{#1}}{</ol>} 
%% %% 2010/11/16:
%% With 'blog.sty', |{verbatim}| really doesn't work much 
%% like its original \LaTeX\ variant. \TeX\ macros inside 
%% still are expanded, and you must care yourself for 
%% wanted \dqtd{quoting}:
\useHTMLelement{verbatim} {pre}
%% |{quote}|:
\useHTMLelement{quote}{blockquote}
%% For list |\item|s, I tried to get readable \HTML\ code
%% using `\indenti'. This fails with nested lists. 
%% The indent could be increased for nested lists 
%% if we supported assignments with `\begin' and `\end'.
%% 2011/10/04 including \xmltagcode{</li>}, 
%% repairs more links in \acro{DANTE} talk
%% (missing again 2011/10/11!?):
\renewcommand*{\item}{%
    \indenti</li>\CLBrk                             %% 2011/10/11
    \indenti<li>}
%% \LaTeX's |{description}| environment redefines the 
%% label format for the optional argument of `\item'.
%% Again, \emph{we} cannot do this here 
%% (we even cannot use optional arguments, 
%%  at least not easily). 
%% Instead we define a different |\ditem{<term>}| 
%% having a \emph{mandatory} argument 
%% (TODO star?). %% 2011/10/05
\useHTMLelement{description}{dl} 
\newcommand*{\ditem}[1]{\indenti<dt>\strong{#1}<dd>}
%%
%% == Links ==
%% === Basic Link Macros ===
%% %% doc. fixes + hyperref 2011/10/16:
%% |\hanc{<name>}{<text>}| makes <text> an anchor with 
%% \HTML\ label <name> like \ctanpkgref{hyperref}'s 
%% |\hypertarget{<name>}{<text>}| 
%% (that we actually provide as well, 
%% towards printing from the same source): 
\newcommand*{\hanc}[1]{\TagSurr a{\@name{#1}}}
\newlet\hypertarget\hanc
%% |\hancref{<name>}{<target>}{<text>}| makes <text> an anchor with 
%% \HTML\ label <name> and at the same time a link to <target>: 
\newcommand*{\hancref}[2]{\TagSurr a{\@name{#1} \@href{#2}}}
%% |\href{<name>}{<text>}| makes <text> a link to <name> 
%% (as with 'hyperref'): 
\newcommand*{\href}[1]{\TagSurr a{\@href{#1}}}
%%
%% === Special cases of Basic Link Macros ===
%% |\autanc{<text>}| creates an anchor where <text> is 
%% the text and the internal label at the same time:
\newcommand*{\autanc}[1]{\hanc{#1}{#1}}             %% 2010/07/04
%% |\ancref{<name>}{<text>}| makes <text> a link to 
%% an anchor <name> on the same web page.
%% This is especially useful for a ``table of contents"---a 
%% list of links to sections of the page. 
%% It is just like 'hyperref''s |\hyperlink{<name>}{<text>}|:
\newcommand*{\ancref}[1]{\href{\##1}}
\newlet\hyperlink\ancref
%% |\autref{<text>}| makes <text> a link to an anchor 
%% named <text> itself: 
\newcommand*{\autref}[1]{\ancref{#1}{#1}}           %% 2010/07/04
%%
%% === Italic Variants ===
%% Some of the link macros get ``emphasized" or ``italic" 
%% variants. Originally I used ``emphasized," later I decided 
%% to replace it by ``italic," as I found that I had used 
%% italics for another reason than emphasizing. 
%% E.g., <text> may be \qtd{bug,} and I am not referring to 
%% some bug, but to the Wikipedia article \textit{Bug.}
%% This has been inspired by some Wikipedia typography convention 
%% about referring to titles of books or movies. 
%% (The `em'\,->\,`it' replacement has not been completed yet.)
% \newcommand*{\emhref}[2]{\href{#1}{\emph{#2}}}
\newcommand*{\ithref}[2]{\href{#1}{\textit{#2}}}
\newcommand*{\itancref}[2]{\ancref{#1}{\textit{#2}}}%% 2010/05/30
\newcommand*{\emancref}[2]{\ancref{#1}{\emph{#2}}}
%%
%% === Built Macros for Links to Local Files ===
%% Originally, I wanted to refer to my web pages only, 
%% using \[|\fileref{<filename-base>}|.\] I have used 
%% extension `.htm' to avoid disturbing my Atari 
%% editor 'xEDIT' or the the Atari emulator (Hatari). 
%% %% rm. 2011/08/18:
%% % I could switch to `.html' some time using symbolic links. 
%% The extension I actually use is stored as macro 
%% |\htext| in a more local file (e.g., `.cfg').---Later 
%% I realized that I may want to refer to local files 
%% other than web pages, and therefore I introduced a more
%% %% mod. 2011/08/18:
%% general `\FileRef{<filename>}|', overlooking 
%% that it was the same as |\href|.
% \newcommand*{\FileRef}[1]{\TagSurr a{\@href{#1}}}
\newcommand*{\htext}{.htm}                              %% 2011/10/05
\newcommand*{\fileref}[1]{\href{#1\htext}}
% \newcommand*{\emfileref}[2]{\fileref{#1}{\emph{#2}}}
\newcommand*{\itfileref}[2]{\fileref{#1}{\textit{#2}}}
%% |\fileancref{<file>}{<anchor>}{<text>}| links to 
%% %% <- `text' added 2011/03/19
%% anchor <anchor> on web page <file>:
\newcommand*{\fileancref}[2]{%
  \TagSurr a{\@href{#1\htext\##2}}}
% \newcommand*{\emfileancref}[3]{\fileancref{#1}{#2}{\emph{#3}}}
%% <- 2010/05/31 ->
\newcommand*{\itfileancref}[3]{\fileancref{#1}{#2}{\textit{#3}}}
%%
%% === Built Macros for Links to Remote Files ===
%% 'blog.sty' currently (even 2011/01/24) implements my style 
%% \emph{not} to open a new browser window or tab for \emph{local} 
%% files but to open a new one for \emph{remote} files, i.e., 
%% when a file is addressed by a full URL. 
%% This may change (as with 'blogdot.sty', 2011/10/12, 
%% or more generally with local non-\HTML\ files),
%% so let us have a backbone 
%% |\hnewref{<prot>}{<host-path[#frag]>}{<text>}| 
%% that                                             %% 2011/10/21
%% makes <text> a link to <prot><host-path[#frag]>:
\newcommand*{\hnewref}[2]{%
    \TagSurr a{\@href{#1#2" target="_blank}}}
%% So \[|\httpref{<host-path[#frag]>}{<text>}|\] 
%% makes <text> a link to \urlfmt{http://}<host-path[#frag]>:
\newcommand*{\httpref}{\hnewref{http://}}
%% With v0.4, macros based on `\httpref' are moved to 'texlinks.sty': 
\RequirePackage[blog]{texlinks}[2011/02/10]
%% Former |\urlref| appears as |\urlhttpref| there ...
\newlet\urlref\urlhttpref
%% ... and `\ctanref' has changed its meaning there as of 2011/10/21.
%% %% 2011/02/07:
%% 'texlinks' sometimes uses a ``permanent alias" 
%% |\NormalHTTPref| of `\httpref':
\newlet\NormalHTTPref\httpref
%% |\httpsref| is the analogue of `\httpref' for `https://':
\newcommand*{\httpsref}{\hnewref{https://}}
%%
%% % \pagebreak                          %% 2012/07/30 rm. 2012/09/13
%% == Characters/Symbols ==             %% `Symbols' until 2011/09/25 
%% === Basic Preliminaries    ===
%% |&| is made `other' for using it to call \HTML's 
%% ``character entities."   %% dot inside 2011/02/25 
\MakeOther\&
%% Again we have the two policies about choosing macro names
%% and respectively two new definition commands. 
%% |\declareHTMLsymbol{<name>}| defines a macro `\<name>'
%% expanding to `&<name>;'. Checking for prior definedness 
%% hasn't been implemented yet. 
%% (TODO; but sometimes redefining ...)         %% 2011/02/26
\newcommand*{\declareHTMLsymbol}[1]{\@namedef{#1}{&#1;}}
%% %% TODO replace \@namedef with \defcsname
%% |\declareHTMLsymbols{<name>}{<list>}| essentially issues %% corr. br. 2012/08/02
%% \[`\declareHTMLsymbol{<attr>}\declareHTMLsymbols{<list>}'\]
%% while `\declareHTMLsymbols{}' essentially does nothing---great, 
%% this is an explanation by recursion!
\newcommand*{\declareHTMLsymbols}{\DoWithAllOf\declareHTMLsymbol}
%% |\renderHTMLsymbol{<macro>}{<name>}| \emph{redefines} 
%% macro <macro> to expand to `&<name>;':
\newcommand*{\renderHTMLsymbol} [2]{\renewcommand*{#1}{&#2;}}
%% Redefinitions of |\&| and |\%| 
%% (well, `\PercentChar' is 'fifinddo''s %% add. apostroph 2011/07/22
%%  version of \LaTeX's `\@percentchar'):
\renderHTMLsymbol{\&}{amp}
\let\%\PercentChar 
%%
%% === Diacritics             ===                       %% changed 2011/03/04
%% For the difference between \dqtd{diacritic} and 
%% \dqtd{accent,} see \wikienref{Diacritic}{\meta{Wikipedia.}}
%% 
%% %% replacing \ccedil 2011/11/23:
%% %%|\ccedil|: %%%, |\eacute|, |\ocirc| (``r\^ole") 
%% % \declareHTMLsymbol{ccedil} 
%% %% 2011/03/04, added example results 2011/11/24:
%% \HTML\ entities |&eacute;| (\'e), 
%% |&ccedil| (\c{c}),                               %% 2011/11/23
%% |&ocirc;| (\^o) etc. can be accessed 
%% by \TeX's accent commands |\'|, |\c|, |\^|, |\`|, |\"|:
% \declareHTMLsymbol{eacute}
% \declareHTMLsymbol{ocirc}
\renewcommand*{\'}[1]{&#1acute;}
\renewcommand*{\c}[1]{&#1cedil;}
\renewcommand*{\^}[1]{&#1circ;}
\renewcommand*{\`}[1]{&#1grave;}
\renewcommand*{\"}[1]{&#1uml;}
%% %% 2012/05/13:
%% ... former |\uml{<char>}| is obsolete, use |\"<char>| 
%% (or `\"<char>') instead.
%% % may have been overestimated:
%% % (useful in macro definitions):                 %% 2010/11/13
%% % \newcommand*     {\uml}[1]  {&#1uml;}             %% 2010/08/24
%%
%% %% 2013/01/01:
%% |\v{<char>}| just works with $<char>=`s'$ and $<char>=`S'$ 
%% for \v{s} and \v{S}:
\renewcommand*{\v}[1]{#1caron;}
%%
%% === Ligatures and the Like ===                       %% 2013/01/01
%% |\lig{<char1><char2>}| forms a ligature from <char1> and <char2>: 
\newcommand*{\lig}[1]{&#1lig;}
%% With v0.81, we use this to reimplement
%% |\ss| from Plain~\TeX\ and \LaTeX\ for the 
%% putative                                             %% 2013/01/01
%% ``s-z ligature", the German ``\Wikienref{sharp s}" 
%% (``\ss"):                                            %% 2013/01/01
% \renderHTMLsymbol{\ss}{szlig}
\renewcommand*{\ss}{\lig{sz}}
%% |\AE|, |\ae|, |\OE|, |\oe| (``\AE", ``\ae", ``\OE", ``\oe") 
%% are reimplemented likewise:
\renewcommand*{\AE}{\lig{AE}}
\renewcommand*{\ae}{\lig{ae}}
\renewcommand*{\OE}{\lig{OE}}
\renewcommand*{\oe}{\lig{oe}}
%% 
%% === Greek                  ===                       %% 2011/02/26
\declareHTMLsymbols{{Alpha}{alpha}                      %% 2012/01/06
    {Beta}{beta}{Gamma}{gamma}{Delta}{delta}{Epsilon}{epsilon} 
    {Zeta}{zeta}{Eta}{eta}{Theta}{theta}{Iota}{iota}{Kappa}{kappa} 
    {Lambda}{lambda}{My}{my}{Ny}{ny}{Xi}{xi}{Omikron}{omikron} 
    {Pi}{pi}{Rho}{rho}{Sigma}{sigma}{sigmaf}{Tau}{tau} 
    {Upsilon}{upsilon}{Phi}{phi}{Chi}{chi}{Psi}{psi}
    {Omega}{omega}                    %% render -> declare 2011/02/26
    {thetasym}{upsih}{piv} }
%% 
%% === Arrows                 ===
%% ---somewhat completed 2012/07/25.
%% 
%% |\downarrow|, |\leftarrow|, |\leftrightarrow|, |\rightarrow|, |\uparrow|: 
\renderHTMLsymbol {\downarrow}     {darr}   %% 2010/09/15
\renderHTMLsymbol {\leftarrow}     {larr} 
\renderHTMLsymbol {\leftrightarrow}{harr} 
\renderHTMLsymbol {\rightarrow}    {rarr} 
\renderHTMLsymbol {\uparrow}       {uarr}   %% 2010/09/15
%% Aliases |\gets| and |\to| were implemented first as stand-alones, 
%% now are treated by `\let':
\let \gets \leftarrow
\let \to   \rightarrow
%% |\Downarrow|, |\Leftarrow|, |\Leftrightarrow|, 
%% |\Rightarrow|, |\Uparrow| (i.e., double variants): 
\renderHTMLsymbol {\Downarrow}     {dArr}
\renderHTMLsymbol {\Leftarrow}     {lArr}
\renderHTMLsymbol {\Leftrightarrow}{hArr}
\renderHTMLsymbol {\Rightarrow}    {rArr} 
\renderHTMLsymbol {\Uparrow}       {uArr} 
%% |\crarrow| accesses \HTML's `crarr' entity (symbol for return key), 
%% named ``downwards arrow with tip leftwards" in Unicode (U+21b2):
\newcommand*{\crarrow}{&crarr;}             %% 2012/09/13
%% 
%% === Dashes                 ===
%% The ligatures `--' and `---' for en~dash and em~dash 
%% don't work in our expanding mode. Now, \HTML's policy 
%% for choosing names often prefers shorter names than 
%% are recommended for (La)\TeX, so here I adopt a \emph{third}
%% policy                                               %% was "police" 2012/07/25
%% besides (i) and (ii) earlier; cf. \LaTeX's
%% `\textemdash' and `\textendash'.---`\newcommand' 
%% does not accept macros whose names start with `end', so:
%% |\endash|, |\emdash| ...
\def         \endash  {&ndash;}         %% \end... illegal
\newcommand*{\emdash} {&mdash;}
%% 
%% === Spaces                 ===
%% ``Math" (not only!)\ spaces |\,|, |\enspace|, |\quad|, |\qquad|: 
%% %% <- 2011/04/26
\renderHTMLsymbol{\enspace}{ensp}
\renderHTMLsymbol{\quad}   {emsp} 
\renewcommand*   {\qquad}  {\quad\quad} 
%% 2011/07/22: `&thinsp;' allows line breaks, 
%% so we introduce |\thinsp| to access `&thinsp;', 
%% while |\thinspace| and |\,| use Unicode ``Narrow No-Break Space" 
%% (`U+202F', see \meta{Wikipedia \Wikienref{Space (punctuation)}}; 
%%  browser support?):
% \renderHTMLsymbol{\thinspace}{thinsp}
% \renderHTMLsymbol{\,}        {thinsp}
\declareHTMLsymbol{thinsp}
\renderHTMLsymbol{\thinspace}{\#8239}
\renderHTMLsymbol{\,}        {\#8239}
%% |\figurespace|                                   %% 2011/09/25
%% (U+2007, cf.~\wikienref{Figure space}{\meta{Wikipedia}}): 
\newcommand*{\figurespace}{&\#8199;}
%%
%% === Quotes, Apostrophe     ===
%% \label{sec:quotes}
%% |\lq|, |\rq| 
\renderHTMLsymbol{\lq}    {lsquo}
\renderHTMLsymbol{\rq}    {rsquo}
%% In order to use the right single quote for the \HTML\ apostrophe, 
%% we must save other uses before. 
%% %% rm. \screenqtd 2011/11/08:
%% % |\screentqtd{<text>}| is used for screen messages, and 
%% |\urlapostr| is the version of the right single quote 
%% for \acro{URL}s of Wikipedia articles:           %% 2011/08/31
% \newcommand*{\screenqtd}[1]{`#1'}                 %% rm. 2011/11/08
\newcommand*{\urlapostr}   {'}                      %% 2010/09/10
%% The actual change of |'| is in `\BlogCodes' 
%% (\secref{catcodes}).
%%
%% %% mod. 2012/10/24f.:
%% |\bdquo| (bottom), |\ldquo|, |\rdquo|, |\sbquo| (single bottom):
\declareHTMLsymbol{bdquo}                           %% 2011/09/23
\declareHTMLsymbols{{ldquo}{rdquo}}
\declareHTMLsymbol{sbquo}                           %% 2010/07/01 
\declareHTMLsymbols{{laquo}{raquo}}
%% Angled quotes |\laquo| and |\raquo| as well as their 
%% ``single" versions |\lsaquo| and |\rsaquo|:
\declareHTMLsymbols{{laquo}{lsaquo}{raquo}{rsaquo}} %% 2012/10/25
%% As of 2012/09/17, |\asciidq| and |\asciidqtd{<no-dqs>}| 
%% (e.g., for attributes after `\catchdqs' 
%%  or typesetting code) 
%% move to package 'catchdq.sty' in the 'catcodes' bundle.
%%
%% |\quot| accesses the same symbol in \HTML's terms
%% (e.g., for displaying code):
\declareHTMLsymbol{quot}                            %% 2012/01/21
%% |\endqtd{<text>}| quotes in the English style using double quote 
%% marks, |\enqtd{<text>}| uses single quote marks instead, 
%% |\dedqtd{<text>}| quotes in German style,        %% 2011/12/21:
%% |\quoted{<text>}| uses straight double quotation marks. 
%% %% 2012/10/24:
%% Settings from the \ctanpkgref{langcode} package may need to 
%% be overridden. (A warning might be nice then TODO)
\def\endqtd#1{\ldquo#1\rdquo} 
\def\enqtd #1{\lq#1\rq}                             %% 2010/09/08
\def\dedqtd#1{\bdquo#1\ldquo}
\def\deqtd #1{\sbquo#1\lq}                    %% corr. 2012/10/25
\newcommand*{\quoted}   [1]{\quot#1\quot}           %% 2012/01/21
%% |\squoted{<text>}| surrounds <text> with ``straight"
%% single quotation marks, useful for other kinds 
%% of quoting in computer code:
\newcommand*{\squoted}[1]{\urlapostr#1\urlapostr}   %% 2012/01/21
%%
%% === (Sub- and) Superscript Digits/Letters ===        %% 2012/10/25
%% As Plain \TeX\ and \LaTeX\ provides an alias `\sp' for 
%% `^', I use |\spone|, |\sptwo|, |\spthree|, |\spa|, and 
%% |\spo| for superscript 1, 2, 3, \qtd{a}, and \qtd{o}:
\newcommand*{\spone}{&sup1;}
\newcommand*{\sptwo}{&sup2;}
\newcommand*{\spthree}{&sup3;}
\newcommand*{\spa}{&ordf;}
\newcommand*{\spo}{&ordm;}
%% For slanted fractions, I think of \ctanpkgref{xfrac}'s 
%% `\sfrac{<numerator>}{<denominator>}'. 
%% |\sfrac{1}{2}|, |\sfrac{1}{4}|, and |\sfrac{3}{4}| work so far:
\newcommand*{\sfrac}[2]{&frac#1#2;}
%%
%% === Math                   ===
%% %% divided/reordered section 2012/08/07
%% ==== Symbols           ====
%% (\TeX\ math type ``Ord")---|\aleph|:
\renderHTMLsymbol{\aleph}{alefsym}
%% I provide |\degrees| for the degree symbol. 
%% \LaTeX\ already has `\deg' as an operator, 
%% therefore I do not want to use `\declareHTMLsymbol'
%% here.
\newcommand*{\degrees}{&deg;}
%% We stick to \TeX's |\emptyset|                       %% 2011/05/08
\renderHTMLsymbol{\emptyset}{empty}                     %% 2011/04/14
%% |\exists| and |\forall|:                             %% 2012/10/05
\renderHTMLsymbol{\exists}{exist} 
\declareHTMLsymbol{forall} 
%% |\prime| can be used for minutes, |\Prime| for seconds:
\renderHTMLsymbol{\prime}{prime} \declareHTMLsymbol{Prime}
%%
%% ==== Relations         ====
%% %% doc. extended 2011/05/08:
%% Because \verb+<+ and \verb+>+ are used for \HTML's
%% element notation, we provide aliases |\gt|, |\lt| 
%% for mathematical $\lt$ and $\gt$---and for reference 
%% to \HTML\ (or just \acro{XML}) code                  %% 2011/11/23 
%% (see \secref{html-verb}):
\declareHTMLsymbols{{gt}{lt}}
%% |\ge|, |\le|, and |\ne| for $\geq$, $\leq$, and $\neq$ resp.:
\declareHTMLsymbols{{ge}{le}{ne}}
%% We also provide their \TeX\ aliases 
%% |\geq|, |\leq|, |\neq|:                              %% 2011/05/11
\let\geq\ge     \let\leq\le     \let\neq\ne
%% Besides \TeX's |\subset| and |\subseteq|, 
%% we provide short versions |\sub| and |\sube|     %% 2011/05/08
%% inspired by \HTML:
\declareHTMLsymbol{sub}                             %% 2011/04/04
\let\subset\sub                                     %% 2011/05/08
\declareHTMLsymbol{sube}                            %% 2011/03/29
\let\subseteq\sube                                  %% 2011/05/08
%%
%% ==== Delimiters        ====
%% Angle braces |\langle| and |\rangle|:
\renderHTMLsymbol{\langle}{lang}
\renderHTMLsymbol{\rangle}{rang}
%% The one-argument macro |\angled{<angled>}| 
%% allows better readable code                  %% TODO:
%% (\textcolor{red}{should be in a more general package}): 
\newcommand*{\angled}[1]{\langle#1\rangle}
%% Curly braces |\{| and |\}| \textcolor{red}{\dots}: 
\begingroup
    \Delimiters\[\] \gdef\{[{] \gdef\}[}]
\endgroup
%% 
%% ==== Binary Operations ====
%% \TeX's |\ast| corresponds to the ``lower" version 
%% of the asterisk:                                %% here 2012/10/05
\renderHTMLsymbol{\ast}{lowast}                         %% 2011/03/29
%% |\pm| renders the plus-minus symbol:                 %% 2012/08/07
\renderHTMLsymbol{\pm}{plusmn}
%% \TeX\ and \HTML\ agree on |\cap|, |\cup|, 
%% and |\times|: 
%% 2011/05/08
%% 2011/04/04
\declareHTMLsymbols{{cap}{cup}{times}}              %% 2012/01/06
%% We need |\minus| since math mode switching is 
%% not supported by 'blog':                         %% 2011/05/08
\declareHTMLsymbol{minus}                           %% 2011/03/31
%%
%% We override \HTML's \lq`&circ;'\rq\       %% \lq\rq 2011/09/08 
%% to get \TeX's \cs{circ} 
%% (i.e., $\circ$;                                  %% 2011/05/08:
%%  \textcolor{red}{but I cannot see it 
%%                  on my own pages!?}):
\renderHTMLsymbol{\circ}{\#x2218}                   %% 2011/04/28
\renderHTMLsymbol{\cdot}{middot}                    %% 2011/05/07
%% |\sdot| generates `&sdot,', a variant of of `&middot;'
%% reserved for the \Wikienref{dot product} according to 
%% the \wikideref{Malzeichen\#Skalarprodukt}
%%               {German \meta{Wikipedia}}          %% 2011/09/08!
\declareHTMLsymbol{sdot}                            %% 2011/05/08
%%
%% ==== Operators         ====                          %% 2012/08/07
%% |\prod|, |sum|:
\renderHTMLsymbol{\prod}{product}
\declareHTMLsymbol{sum}
%%
%% 
%% === Currencies             ===                       %% 2012/08/06
%% |\cent|, |\currency|, |\euro|, |\pound|, |\yen|:     %% ...symbols 2012/08/07:
\declareHTMLsymbols{{cent}{currency}{euro}{pound}{yen}}
%% You get the \$ symbol simply by |$|.
%% 
%% === Other                  ===
%% The tilde |~| is used for its wonderful purpose, 
%% by analogy to \TeX (TODO overridden by `\FDpseudoTilde'):
\renderHTMLsymbol{~}{nbsp} 
%% But now we need a replacement |\tilde| for URLs involving 
%% home directories of institution members 
%% (should better be |\tildechar| or `\TildeChar', 
%%  cf.~'fifinddo'):                    %% 2011/04/29
{ \MakeOther\~ \gdef\tilde{~} \gdef\tildechar{~}} 
%% % <- more elegant macro                          %% 2011/05/18
%% %    \newcommand*{\StoreOtherCharAs}[2]{%
%% %                 \edef#2{\expandafter\@gobble\string#1}}
%% Horizontal ellipsis: |\dots| ...
\renderHTMLsymbol {\dots} {hellip}
%% Plain~\TeX's and \LaTeX's |\-| becomes a soft hyphen:
\renderHTMLsymbol{\-}{shy}
%% |\copyright|:
\renderHTMLsymbol{\copyright}{copy} 
%% |\bullet|                                    %% 2011/03/29
\renderHTMLsymbol{\bullet}{bull}
%% \LaTeX's |\S| prints the \dqtd{\Wikienref{section sign}}
%% \qtd{\S}. In \HTML, the latter accessed by `&sect;', 
%% we \dqtd{redirect} `\S' to this:         %% 2011/05/18
\renderHTMLsymbol{\S}{sect}
%% |\dagger|, |\ddagger|:                       %% 2011/10/01
\renderHTMLsymbol{\dagger}{dagger}
\renderHTMLsymbol{\ddagger}{Dagger}
%% |\P| renders the paragraph sign or pilcrow:  %% 2012/08/07
\renderHTMLsymbol{\P}{para}
%% Sometimes (due to certain local settings) the notations 
%% \dqtd{`&&<characters>;'} or \dqtd{`&&&#<number>;'} 
%% (for \Wikienref{Unicode}) may not be available. 
%% We provide 
%% \[|\htmlentity{<characters>}|\]
%% as well as 
%% \[|\unicodeentity{<decimal>}|\]
%% and
%% \[|\unicodehexentity{<hexadecimal>}|\]
%% for such situations:
\newcommand*{\htmlentity}[1]{&#1;}
\newcommand*{\unicodeentity}[1]{&\##1;}
\newcommand*{\unicodehexentity}[1]{&\#x#1;}
%% 
%% == \TeX-related ==                                   %% 2010/08/24
%% Somebody actually using 'blog.sty' must have a need to put down 
%% notes about \TeX\ for her own private purposes at least---I expect.
%% === Logos ===
%% ``Program" names might be typeset in a special font, 
%% I once thought, and started tagging program names with 
%% |\prg|. It could be `\texttt' or `\textsf' like in 
%% documentations of \LaTeX\ packages. However, 
%% sans-serif is of doubtable usefulness on web pages, 
%% and typewriter imitations usually look terrible on 
%% web pages. So I am waiting for a better idea and 
%% let `\prg' just remove the braces. 
\newlet\prg\@firstofone 
\newcommand*{\BibTeX}{\prg{BibTeX}} %% 2010/09/13
\renewcommand*{\TeX}{\prg{TeX}}
\renewcommand*{\LaTeX}{\prg{LaTeX}}
\newcommand*{\allTeX}{\prg{(La)TeX}}%% 2010/10/05
\newcommand*{\LuaTeX}{\prg{LuaTeX}}
\newcommand*{\pdfTeX}{\prg{pdfTeX}}
\newcommand*{\XeTeX}{\prg{XeTeX}}   %% 2010/10/09
\newcommand*{\TeXbook}{TeXbook}     %% 2010/09/13
%%
%% === Describing Macros ===
%% With v0.4, \TeX-related \emph{links} are moved to 'texlinks.sty'.
%%
%% \leavevmode % \errorcontextlines=15       %% TODO 2011/11/05
%% |\texcs{\<tex-cmd-name>}| or `\texcs\<tex-cmd-name>'
%% (care for spacing yourself):
\newcommand*{\texcs}[1]{\code{\string#1}}           %% 2010/11/13
%% Good old |\cs{<tex-cmd-name>}| may be preferable: 
\def\cs#1{\code{\BackslashChar#1}}                  %% 2011/03/06
%% |\metavar{<name>}|:                              %% 2011/09/22
\newcommand*{\metavar}[1]{\angled{\meta{#1}}}
%%
%% == Tables ==
%% %% 2011/04/24: rm. remark on doubtful stability, ordered
%% I am not so sure about this section ...
%% === Indenting === 
%% There are three levels of indenting:
%% \[|\indenti|, \quad 
%%   |\indentii|,\quad \mbox{and}\quad 
%%   |\indentiii|.\] 
%% The intention for these was to get readable \HTML\ code. Not sure ...
{\catcode`\ =12%% 2010/05/19
\gdef\indenti{  }\gdef\indentii{    }\gdef\indentiii{      }}
%%
%% === Starting/Ending Tables ===
%% |\startTable{<attributes>}| and |\endTable| have been made 
%% for appearing in different macros, such as in the two parts 
%% of a `\newenvironment':
%% %% 2010/07/17:
\newcommand*{\startTable}[1]{<table #1>}
\def\endTable{</table>}
%% |\@frame@box| among the `\startTable' <attributes>
%% draws a frame around the table, |\@frame@groups|             %% about -> around
%% separates ``groups" by rules: 
\newcommand*{\@frame@box}{\@frame{box}}
\newcommand*{\@frame@groups}{\@frame{groups}}
%% |\begin{allrulestable}{<cell-padding>}{<width>}| 
%% starts a table environment with all possible rules
%% and some code cosmetic. <width> may be empty ...
\newenvironment{allrulestable}[2]
  {\startTable{\@cellpadding{#1} \@width{#2} 
               \@frame@box\ rules="all"}\CLBrk  %% \ 2011/10/12
   \ \tbody} %% <- tbody 2011/10/13, `\ ' 2011/11/09 ->
  {\ \endtbody\CLBrk\endTable}
%% \xmltagcode{tbody}...\xmltagcode{/tbody} seemed to be 
%% better with `\HVspace' for 'blogdot.sty',    %% 2011/10/13
%% so it gets an environment |{tbody}| 
%% (i.e., macros |\tbody| and |\endtbody|):
\useHTMLelement{tbody}{tbody}
%% 
%% === Rows ===
%% I first thought it would be good for readability if 
%% some \HTML\ comments explain nesting or briefly describe 
%% the content of some column, row, or cell. 
%% But this is troublesome when you want to comment out 
%% an entire table ...
%% 
%% |\begin{TableRow}{<comment>}{<attributes>}|\\ starts an environment 
%% producing an \HTML\ comment <comment> and a table row
%% with attributes <attributes>, including code cosmetic. 
\newenvironment*{TableRow}[2]{%% lesser indentation 2011/04/25
  \ \comment{ #1 }\CLBrk
  \indenti<tr #2>%
  }{%
  \indenti\endtr}                           %% \endtr 2011/11/08
%% |\begin{tablecoloredrow}{<comment>}{<background-color>}|\\
%% is a special case of `{TableRow}' where `@bgcolor' is the only 
%% attribute:
\newenvironment{tablecoloredrow}[2]
  {\TableRow{#1}{\@bgcolor{#2}}}
  {\endTableRow}
%% |\begin{tablecoloredboldrow}{<comment>}{<background-color>}|\\
%% is like `{tablecoloredrow}' except that content text is 
%% rendered in boldface (TODO horizontal centering?):
\newenvironment{tablecoloredboldrow}[2]         %% 2011/11/03/08
  {\TableRow{#1}{\@bgcolor{#2} 
                 \@style{font-weight:bold}}}
  {\endTableRow}
%% |\begin{tablerow}{<comment>}|    %% rm. {<b...>} 2011/12/19
%% is a special case of `{TableRow}' where the only attribute 
%% yields ``top" vertical alignment (TODO strange):
%% %% ``top" 2010/05/18:
\newenvironment{tablerow}[1]{\TableRow{#1}{\@valign@t}}
                            {\endTableRow}
%% |\starttr| and |\endtr| delimit a row; these commands again 
%% have been made for appearing in different macros. 
%% There is \emph{no} code indenting, probably for 
%% heavy table nesting where indenting was rather useless 
%% (? TODO only in 'texblog.fdf'? there indents would have 
%%         been useful). 
%% %% 2010/07/18:
\newcommand*{\starttr}{<tr>} 
\def\endtr{</tr>} 
%%
%% === Cells === 
%% |simplecell{<content>}| produces the most \emph{simple} kind of 
%% an \HTML\ table cell: 
\newcommand*{\simplecell}{\SimpleTagSurr{td}}   %% 2010/07/18
%% |\TableCell{<attributes>}{<content>}| produces the most
%% \emph{general} kind of a cell, together with a code indent:
%% % \newcommand*{\TableCell}[2]{\indentiii<td #1>#2</td>}
%% % \newcommand*{\TableCell}[2]{\indentiii\TagSurr{td}{#1}{#2}}
%% %% <- 2010/07/18 ->
\newcommand*{\TableCell}[2]{\indentiii\startTd{#1}#2\endTd}
%% |\colorwidthcell{<color>}{<width>}{<content>}| 
%% uses just the `@bgcolor' and the `@width' attribute: 
%% %% 2010/06/15:
\newcommand*{\colorwidthcell}[2]{\TableCell{\@bgcolor{#1}\@width{#2}}}
%% |\tablewidthcell{<color>}{<width>}{<content>}| 
%% uses just the `@bgcolor' and the `@width' attribute: 
\newcommand*{\tablewidthcell}[1]{\TableCell{\@width{#1}}}
%% %% |\tablewidthcell{<color>}{<width>}{<content>}|   %% 2011/11/09
%% %% uses just the `@bgcolor' and the `@width' attribute: 
%% %\newcommand*{\tablecolorcell}[1]{\TableCell{\@bgcolor{#1}}}
%% |\tablecell{<content>}| is like `\simplecell{<content>}', 
%% except that it has a code indent:
\newcommand*{\tablecell}{\TableCell{}}
%% |\tableCell{<content>}| is like `\tablecell{<content>}', 
%% except that the content <content> is horizontically centered. 
%% The capital `C' in the name may be considered indicating 
%% ``\strong{c}entered":
\newcommand*{\tableCell}{\TableCell\@align@c}
%% Idea: use closing star for environment variants!?
%%
%% |\begin{bigtablecell}{<comment>}| 
%% starts an \emph{environment} yielding a table cell element
%% without attributes, preceded by a \HTML\ comment
%% <comment> unless <comment> is empty. At least the \HTML\ tags 
%% are indented:
%% %% 2010/05/19:
\newenvironment{bigtablecell}[1]{\BigTableCell{#1}{}}
                                {\endBigTableCell}
%                {\ifx\\#1\\%             %% 2010/05/30
%                   \indentii\ \comment{#1}\CLBrk
%                 \fi
%                 \indentiii<td>}
%                {\indentii</td>}         %% !? 2010/05/23
%% |\begin{BigTableCell}{<comment>}{<attributes>}|\\ 
%% is like `\begin{bigtablecell{<comment>}}' except that it uses 
%% attributes <attributes>:
%% %% 2010/06/05:
\newenvironment{BigTableCell}[2]
    {\ifx\\#1\\\indentii\ \comment{#1}\CLBrk\fi
     \indentiii\startTd{#2}}
    {\indentii\endTd}           %% TODO indent? 2010/07/18
%% |\startTd{<attributes>}| and |\endTd| delimit a cell element 
%% and may appear in separate macros, e.g., in an environment 
%% definition. There is no code cosmetic. And finally there is 
%% |\StartTd| that yields less confusing code without attributes:
%% %% 2010/07/18:
\newcommand*{\startTd}[1]{<td #1>} 
\newcommand*{\StartTd}{<td>}                    %% 2011/11/09
\def\endTd{</td>} 
%% |\emptycell| uses \xmltagcode{td /} instead of 
%% \xmltagcode{td}\xmltagcode{/td} for an empty cell:
\newcommand*{\emptycell}{<td />}                %% 2011/10/07
%% 
%% === ``Implicit" Attributes and a ``\TeX-like" Interface ===
%% %% <- 2011/11/08 ->
%% After some more experience, much musing, and trying new tricks, 
%% I arrive at the following macros (v0.7). \ 
%% (i)\enspace When a page or a site has many tables that use the same 
%% attribute values, these should not be repeated for the single 
%% tables, rather the values should be invoked by shorthand macros, 
%% and the values should be determined at a single separate place. 
%% We will have |\stdcellpadding|, %%% |\stdtableheadrow|, 
%% |\stdtableheadcolor| and 
%% |\stdtableheadstyle|. \
%% (ii)\enspace As with \TeX, `\cr' should suffice to \emph{close} 
%% a \emph{cell} and a \emph{row}, and then to \emph{open} another 
%% \emph{row} and its first \emph{cell}. And there should be a 
%% single command to close a cell within a row and open a next one.
%%
%% We use `\providecommand' so the user can determine the values 
%% in a file for 'blog' where 'blogexec' is loaded later. 
%% |\stdcellpadding| should correspond to the \CSS\ settings, 
%% the value of `6' you find here is just what I used recently.
\providecommand*{\stdcellpadding}{6}
%% For |\stdtableheadcolor|, I provide a 
%% % ``web-safe"                            %% very wrong 2012/05/13
%% gray, `#EEEEEE', that the German Wikipedia uses for articles about 
%% \Wikienref{networking protocol}s
%% (unfortunately, it doesn't have a 
%% \wikienref{Web colors\#X11_color_names}{%
%% \CSS-3}\wikienref{X11 color names}{X11} color name):
\providecommand*{\stdtableheadcolor}{\#EEEEEE}
%% |\stdtableheadstyle| demands a boldface font. 
%% In general, it is used for the `@style' attribute:
\providecommand*{\stdtableheadstyle}{font-weight:bold}
%% |\begin{stdallrulestable}| starts an `{allrulestable}'
%% environment with ``standard" cell padding and empty width 
%% attribute, then opens a ``standard" row element 
%% with a ``standard" comment as well as a cell:
\newenvironment{stdallrulestable}{%
    \allrulestable{\stdcellpadding}{}\CLBrk
      \TableRow{standard all-rules table}%
               {\@bgcolor{\stdtableheadcolor}
                \@style{\stdtableheadstyle}}\CLBrk
        \indentii\StartTd
%% `\end{stdallrulestable}' will provide closing of a cell
%% and a row, including a code cosmetic:
    }{\indenti\endTd\CLBrk\endTableRow\CLBrk
    \endallrulestable}
%% |\endcell| closes a cell and opens a new one. The idea behind 
%% this is that an active character will invoke it. The name is 
%% inspired by `\endgraf' and `\endline' from Plain \TeX\ and 
%% \LaTeX\ (`\newcommand' does not work with `\end'\code{\dots}):
\def\endcell{\endTd\StartTd}
%% Plain \TeX's and \LaTeX's |\cr| and |\endline| are redefined 
%% for closing and opening rows and cells, including code cosmetic:
\renewcommand*{\cr}{\indentii\endTd\CLBrk\indenti\endtr\CLBrk
                    \indenti\startTR\CLBrk\indentii\StartTd}
\let\endline\cr
%% |\startTR| is a hook defaulting to `\starttr':               %% 2012/08/23
\newlet\startTR\starttr
%% 
%% === Filling a Row with Dummy Cells ===
%% These macros were made, e.g., for imitating a program window 
%% with a title bar (spanning someting more complex below), 
%% perhaps also for a Gantt chart. 
%% |\FillRow{<span>}{<attributes>}| produces a cell without text, 
%% spanning <span> columns, with additional attributes <attributes>. 
%% %% Generalization 2010/06/28:
%% % \newcommand*{\FillRow}[2]{%             %% broke line 2011/01/24
%% %                   \indentiii\TagSurr{td}{\@colspan{#1} #2}{}} 
%% %% <- 2010/07/18 ->
\newcommand*{\FillRow}[2]{\indentiii\startTd{\@colspan{#1} #2}\endTd}
%% |\fillrow{<span>}| instead only uses the `@colspan' attribute:
\newcommand*{\fillrow}[1]{\FillRow{#1}{}}
%% |\fillrowcolor{<span>}{<color>}| just uses the `@colspan'
%% and the `@bgcolor' attributes:
\newcommand*{\fillrowcolor}[2]{\FillRow{#1}{\@bgcolor{#2}}}
%% === Skipping Tricks ===
%% %% 2011/10/13
%% |\HVspace{<text>}{<width>}{<height>}| may change, 
%% needed for 'blogdot.sty' but also for |\vspace{<height>}|
%% with 'texblog'. It is now here so I will be careful 
%% when I want to change something. \xmltagcode{tbody}
%% improved the function of `\HVspace' constructions 
%% as link text with 'blogdot.sty'.
\newcommand*{\HVspace}[3]{% 
    \CLBrk
    \startTable{\@width{#2} \@height{#3}
                \@border{0} 
                \@cellpadding{0} \@cellspacing{0}}%
      \tbody
       \CLBrk                                       %% 2011/10/14
        \tablerow{HVspace}%                         %% 2011/10/13
%% <- inserting text at top for 'blogdot' attempts---that 
%%    finally did not help anything (2011/10/15) ->
          \simplecell{#1}%
        \endtablerow                                %% 2011/10/13
       \CLBrk                                       %% 2011/10/14
      \endtbody
    \endTable
    \CLBrk}
%% |\hvspace{<width>}{<height>}| ...:
\newcommand*{\hvspace}{\HVspace{}}
%% |\vspace{<height>}| ... (TODO: `{0}'!?):
\renewcommand*{\vspace}[1]{\hvspace{}{#1}}
%%
%% == Misc ==
%% \TeX's |\hrule| (rather deprecated in \LaTeX) is redefined 
%% to produce an \HTML\ horizontal line:
\renewcommand*{\hrule}{<hr>}
%% For references, there were 
% \catcode`\^=\active
% \def^#1{\SimpleTagSurr{sup}{#1}}
%% and 
% \newcommand*{\src}[1]{\SimpleTagSurr{sup}{[#1]}}
%% as of 2010/05/01, inspired by the \xmltagcode{ref} element 
%% of MediaWiki; moved to `xmlprint.tex' 2010/06/02. 
%%
%% == Leaving and HISTORY ==
\endinput
%%
        VERSION HISTORY
v0.1    2010/08/20  final version for DFG
v0.2    2010/11/08  final documentation version before
                    moving some functionality to 'fifinddo' 
v0.3    2010/11/10  removed ^^J from \head
        2010/11/11  moving stuff to fifinddo.sty; \BlogCopyFile
        2010/11/12  date updated; broke too long code lines etc.;
                    \CatCode replaced (implemented in niceverb only); 
                    \ifBlogAutoPars etc. 
        2010/11/13  doc: \uml useful in ...; \texcs
        2010/11/14  doc: argument for {commentlines}, 
                         referring to environments with curly braces, 
                         more on \ditem
        2010/11/15  TODO: usage, templates
        2010/11/16  note on {verbatim}
        2010/11/23  doc. corr. on \CtanPkgRef
        2010/11/27  "keyword"; \CopyLine without `fd'
        2010/12/03  \emhttpref -> \ithttpref
        2010/12/23  `%' added to \texhaxpref
        2011/01/23  more in \Provides...
        2011/01/24  updated copyright; resolving `td' ("today")
        JUST STORED as final version before texlinks.sty
v0.4    2011/01/24  moving links to texlinks.sty
v0.41   2011/02/07  \NormalHTTPref
        2011/02/10  refined call of `texlinks'
part of MOREHYPE RELEASE r0.3
v0.5    2011/02/22  \BlogProvidesFile
        2011/02/24  ... in \BlogCopyFile
        2011/02/25  ordering symbols
        2011/02/26  subsection Greek; note on \declareHTMLsymbol
        2011/03/04  diacritics
        2011/03/06  \cs
        2011/03/09  \var
        2011/03/16  \robots
        2011/03/19  doc. \fileancref arg.s corr.
        2011/03/29  \Sigma, ...
        2011/03/31  \minus
        2011/04/04  \times, \sub, \delta
        2011/04/11  Greek completed
        2011/04/14  \emptyset
        2011/04/22  \deqtd
        2011/04/24  doc.: folding, \stylesheet, ordered "tables"; 
                    @border, @align, @valign
        2011/04/25  lesser indentation with TableRow
        2011/04/26  \,, \thinspace, \@title; doc. \@name
        2011/04/28  [\circ] PROBLEM still 
        2011/04/29  \rightitpar
        2011/05/07  \cdot
        2011/05/08  extended doc. on math symbols; \sdot; 
                    \ast replaces \lowast; \subset, \subseteq; 
                    \angled
        2011/05/09  \euro
        2011/05/11  |\geq| etc.; new section "logical markup"
        2011/05/12  corr. doc. \heading
        2011/05/14  right mark of \deqtd was rsquo instead of lsquo!
        2011/05/18  \S and note on \StoreOtherCharAs
        2011/06/27  \httpsref; doc: \acro
        2011/07/22  \thinspace vs. \thinsp; 'fifinddo''s
        2011/07/25  "todo" on \description
        2011/08/18f.removing \FileRef, 0.42-> 0.5
        2011/08/31  clarified use of \urlapostr
part of MOREHYPE RELEASE r0.4
v0.6    2011/09/08  doc. uses \HTML, \lq/\rq with &circ;, 
                    doc. fix `mult-'; \degrees
        2011/09/21  \acronym
        2011/09/22  \metavar; TODO \glqq...
        2011/09/23  \bdquo 
        2011/09/25  doc. `Characters/Symbols'; \figurespace
        2011/09/27  "universal" attributes completed, reworked doc.
        2011/09/30  end lists with </li>
        2011/10/01  \dagger, \ddagger
        2011/10/04  \item includes </li> [2011/10/11: ???]
        2011/10/05  {style}; doc. \acronym -> \acro, \pagebreak, 
                    rm. \description; {center} accesses <center>, 
                    \useHTMLenvironment replaces \declareHTMLelement
                    and \renderHTMLelement, message "generating"
        2011/10/07  \emptycell
        2011/10/10  doc.: page breaks, $$->\[/\]
part of MOREHYPE RELEASE r0.5
v0.61   2011/10/11  </li> in \item again, \Provides... v wrong
        2011/10/12  \hnewref, `\ ' in allrulestable
        2011/10/14  \CLBrk's
        2011/10/15  doc. note on \HVspace/blogdot
part of MOREHYPE RELEASE r0.51
v0.62   2011/10/16  \hyperlink, \hypertarget; doc. fixes there
        2011/10/20  \textcolor by <span>, \textsf 
        2011/10/21  \ctanref now in texlinks.sty; 
                    doc.: grammar with `that'
        2011/10/22  \BlogCopyFile message removed
part of MOREHYPE RELEASE r0.52
v0.7    2011/11/03  {tablecoloredboldrow}
        2011/11/05  \ContentAtt -> \@content, 
                    \BlogCopyFile -> \BlogProcessFile (blogexec),
                    doc. different \pagebreak's
        2011/11/06  run \BlogCopyLines, doc. \[...\]
        2011/11/07  \ProvideBlogExec
        2011/11/08  \endtr in \endTableRow, using \MakeOther,
                    right quote change moves to \BlogCodes,
                    \BlogInterceptHash; rm. \AmpMark & doc. about it, 
                    mod. on #; doc. for tables; start doc. "implicit" 
                    table attributes and "TeX-like" interface
        2011/11/09  \tablecolorcell(?); cont. "implicit" etc.; 
                    \StartTd
        2011/11/20  \isotoday, \BlogProcessFinalFile, 
                    catcodes of `<' `>' untouched; restructured, 
                    structured processing, misc -> ordinary
        2011/11/21  BlogLIGs
        2011/11/23  \xmltagcode, \xmlentitycode, \c; 
                    doc: <p>, \secref, \pagebreak
        2011/11/24  doc: example results for diacritics
        2011/11/27  \ParseLigs; doc. rm. \pagebreak
        2011/12/12  \title uses \SimpleTagSurr
        2011/12/19  doc. fix {tablerow}
        2011/12/21  \asciidq, \asciidqtd
        2012/01/06  \acro; using dowith.sty (\declareHTMLsymbols); 
                    doc.: cross-referring for naming policies
        2012/01/07  \MakeActiveDef\~ for \FDpseudoTilde
        2012/01/11  (C)
        2012/01/21  \quot, \quoted. \squoted
        2012/02/04  \newacronym
        2012/03/14  removed hidden and another comment with 
                    \BlogCopyLines, fixed latter, TODO on \NoBlogLigs
        2012/03/17  tweaked \@typeset@protect for \EXECUTE
        2012/03/30  space in stdallrules... after @bgcolor
        2012/04/03  \CLBrk in \@useHTMLelement
        2012/04/09  \htmlentity, \unicodeentity
        2012/05/13  \ss; better comment on \uml; 
                    #EEEEEE not "web-safe"
        2012/05/15  xEDIT folding in tables section
part of MOREHYPE RELEASE r0.6
v0.8    2012/06/07  \underbar
        2012/07/25  arrows completed [no: 2012/09/13]; 
                    doc. "police" -> "policy"
        2012/07/30  \spanstyle, applied; doc. \pagebreak
        2012/08/01  \textup
        2012/08/02  doc. corr. braces for \DeclareHTMLsymbols
        2012/08/06  sec. currencies
        2012/08/07  divided math section, using \declareHTMLsymbols, 
                    various additional symbols
        2012/08/23  \startTR
        2012/08/28  \MakeActiveLet\'\rq with `actcodes.sty', 
                    attributes start with space
        2012/09/02  about -> around 
        2012/09/06  Content-T -> content-t - bugfix?, 
                    \BlogProvidesFile with DOCTYPE, some attribute 
                    lists rely on space from \declareHTMLattrib, 
                    there another \reserved@a; 
                    "Head": \metanamecontent, \metanamelangcontent
        2012/09/07  "Head": \author, \date, \metadescription, 
                    \keywords; lang variants
        2012/09/08  \TagSurr and \MetaTag without space, 
                    \declareHTMLattrib{align}, \@valign@t adjusted; 
                    \pagebreak[3]
        2012/09/13  \crarrow, "Fonts" -> "Physical markup" etc.,
                    \abbr, \newabbr
        2012/09/14  \xmleltcode, \xmleltattrcode; el-name -> elt-name
        2012/09/17  \asciidq + \asciidqtd move to `catchdq.sty'
        2012/10/03  \newlet; 
                    doc.: label process -> catcodes, using \secref
        2012/10/05  moved \ast; \exists, \forall
        2012/10/24  quotes: completed, override `langcode.sty'
        2012/10/25  using \DeclareHTMLsymbols for quotes, corr. there, 
                    \spone etc., \sfrac
        2012/10/28  spanstyle -> stylespan
        2012/11/16  \TagSurr and \MetaTag with space again
        2012/11/19  \endgraf -> <p>
        2012/11/29  `blogligs.sty', `markblog.sty' ([ligs], [mark])
part of MOREHYPE RELEASE r0.7
v0.81   2012/12/20  \-, {enumtype}
        2013/01/02  caron, "Ligatures ..." (&aelig; etc.)
        2013/01/04  updating copyright
part of MOREHYPE RELEASE r0.81 
v0.81a  2013/01/21  \newlet in subsubsection


\section{``Pervasive Ligatures" with 'blogligs.sty'} %% 2012/11/29
\label{sec:moreligs}
% \AddQuotes
This is the code and documentation of the package mentioned in 
Sec.~\ref{sec:ligs}, loadable by option |[ligs]|.
See below for what is offered. 
\ResetCodeLineNumbers
\NeedsTeXFormat{LaTeX2e}[1994/12/01] %% \newcommand* etc. 
\ProvidesPackage{blogligs}[2012/11/29 v0.2 
                           pervasive blog ligatures (UL)]
%% copyright (C) 2012 Uwe Lueck, 
%% http://www.contact-ednotes.sty.de.vu 
%% -- author-maintained in the sense of LPPL below.
%%
%% This file can be redistributed and/or modified under 
%% the terms of the LaTeX Project Public License; either 
%% version 1.3c of the License, or any later version.
%% The latest version of this license is in
%%     http://www.latex-project.org/lppl.txt
%% We did our best to help you, but there is NO WARRANTY. 
%%
%% Please report bugs, problems, and suggestions via 
%% 
%%   http://www.contact-ednotes.sty.de.vu 
%%
%% == 'blog' Required ==
%% 'blogdot' is an extension of 'blog', and must be loaded \emph{later}
%% (but what about options? TODO):
\RequirePackage{blog}
%% == Task and Idea   ==
%% |\UseBlogLigs| as offered by 'blog.sty' does not work 
%% inside macro arguments. You can use |\ParseLigs{<text>}|
%% at such locations to enable ``ligatures" again. 
%% 'blogligs.sty' saves you from this manual trick. 
%% Many macros have one ``text" argument only, 
%% others additionally have ``attribute" arguments. 
%% Most macros `<elt-cmd>{<text>}' of the first kind are defined 
%% to expand to `\SimpleTagSurr{<elt>}{<text>}'
%% or to `\TagSurr{<elt>}{<attrs>}{<text>}' for some 
%% \HTML\ element <elt> and some attribute assignments <attrs>. 
%% When a macro in addition to a ``text" element has 
%% ``attribute" parameters, `\TagSurr' is used as well.
%% %% 2012/01/08, eigentlich schon 2012/01/04, verloren ...:
% \let\blogtextcolor\textcolor
% \renewcommand*{\textcolor}[2]{\blogtextcolor{#1}{\ParseLigs{#2}}}
%%
%% \pagebreak[2]
%% == Quotation Marks ==
%% ``Inline quote" macros `<qtd>{<text>}' to surround <text>
%% by quotation marks do not follow this rule. We are just 
%% dealing with English and German double quotes 
%% that I have mostly treated by `catchdq.sty'. 
%% `"<text>"' then (eventually) expands to either 
%% `\deqtd{<text>}' or `\endqtd{<text>}', so we redefine these:
%% %% 2012/01/10:
\let\blogdedqtd\dedqtd 
\renewcommand*{\dedqtd}[1]{\blogdedqtd{\ParseLigs{#1}}}
%% %% 2012/08/20:
\let\blogendqtd\endqtd
\renewcommand*{\endqtd}[1]{\blogendqtd{\ParseLigs{#1}}}
%%
%% == \HTML\ Elements ==
%% When the above rule holds:
%% %% 2012/01/19:
\let\BlogTagSurr\TagSurr 
\renewcommand*{\TagSurr}[3]{%
    \BlogTagSurr{#1}{#2}{\ParseLigs{#3}}}
\let\BlogSimpleTagSurr\SimpleTagSurr 
\renewcommand*{\SimpleTagSurr}[2]{%
    \BlogSimpleTagSurr{#1}{\ParseLigs{#2}}}
%%
%% == Avoiding ``Ligatures" though ==
%% |\noligs{<text>}| saves <text> from ``ligature" replacements 
%% (except in arguments of macros inside <text> where 
%%  'blogligs' enables ligatures):
\newcommand*{\noligs}{}     \let\noligs\@firstofone     %% !!!
%% I have found it useful to disable replacements within
%% |\code{<text>}|: 
\renewcommand*{\code}[1]{\STS{code}{\noligs{#1}}}
%% TODO: kind of mistake, `\STS' has not been affected anyway so far, 
%% then defining `\code' as `\STS{code}' should suffice.
%%
%% |\NoBlogLigs| has been meant to disable ``ligatures" altogether again. 
%% I am not sure about everything ...
%% %% 2012/03/14, not optimal TODO:
\renewcommand*{\NoBlogLigs}{%
    \def\BlogOutputJob{LEAVE}%
%     \let\deqtd\blogdeqtd                       %% rm. 2012/06/03
    \let\TagSurr\BlogTagSurr
    \let\SimpleTagSurr\BlogSimpleTagSurr
    \FDnormalTilde 
    \MakeActiveDef\~{&nbsp;}%                    %% TODO new blog cmd
}
%% TODO: |\UseBlogLigs| might be redefined likewise 
%% (\textcolor{red}{in fact 'blogligs' activates ligatures 
%%                  inside text arguments unconditionally at present}, 
%%  I keep this for now since I have used it this way with `texblog.fdf' 
%%  over months, and changing it may be dangerous 
%%  where I have used tricky workarounds to overcome the 
%%  `texblog.fdf' mistake). 
%% But with \[`\BlogInteceptEnvironments'\] this is not needed 
%% when you use `\NoBlogLigs' for the contents of some \LaTeX\ 
%% environment.
%% 
%% == The End and \acro{HISTORY} ==
\endinput
%% VERSION HISTORY
v0.1    2012/01/08ff. developed in `texblog.fdf'
v0.2    2012/11/29    own file


% \DontAddQuotes

\section{Wiki Markup by 'markblog.sty'}     %% 2012/11/29
\label{sec:mark}
\subsection{Introduction}                   %% 2012/12/20
\AddQuotes
This is the code and documentation of the package mentioned in 
Sec.~\ref{sec:ligs}, loadable by option |[mark]|.
See below for what is offered. You should also find a file 
`markblog.htm' that sketches it. Moreover, `texlinks.pdf'
describes in detail to what extent Wikipedia's 
``\wikiref{Help:Links#Piped_links}{piped links}"
with `[[<wikipedia-link>]]' is supported.       %% <...> 2012/12/20

\subsection{Similar Packages}               %% 2012/12/20
'wiki.sty' from the \ctanpkgdref{nicetext} bundle has offered 
some Wikipedia-like markup as a front-end for ordinary 
typesetting with \LaTeX\ (for \acro{DVI}/\acro{PDF}), 
implemented in a way very different from what is going on here, 
rather converting markup sequences \emph{during} typesetting.

More similar to the present approach is the way how 
Wikipedia section titles in package documentation 
is implemented by 'makedoc' from the 'nicetext' bundle, 
based on \strong{preprocessing} by 'fifinddo'.

In general, John MacFarlane's 
\httpref{johnmacfarlane.net/pandoc}{\pkg{pandoc}}
(cf.~\wikideref{pandoc}{German Wikipedia})
converts between wiki-like (simplified) markup and 
\LaTeX\ markup. (It deals with rather fixed 
markup rules, while we here process markup sequences 
independently of an entire markup \emph{language}.)

Another straightforward and well-documented way to 
\emph{preprocess} source files for converting simplified 
markup into \TeX\ markup is \ctanpkgauref{isambert}{Paul Isambert}'s 
\ctanpkgref{interpreter}. It relies on \wikiref{LuaTeX}{\LuaTeX}
where Lua does the preprocessing.

\subsection{Package File Header}            %% 2012/12/20
\ResetCodeLineNumbers
\NeedsTeXFormat{LaTeX2e}[1994/12/01] %% \newcommand* etc. 
\ProvidesPackage{markblog}[2012/11/29 v0.2 
                           wiki markup with blog.sty (UL)]
%% copyright (C) 2012 Uwe Lueck, 
%% http://www.contact-ednotes.sty.de.vu 
%% -- author-maintained in the sense of LPPL below.
%%
%% This file can be redistributed and/or modified under 
%% the terms of the LaTeX Project Public License; either 
%% version 1.3c of the License, or any later version.
%% The latest version of this license is in
%%     http://www.latex-project.org/lppl.txt
%% We did our best to help you, but there is NO WARRANTY. 
%%
%% Please report bugs, problems, and suggestions via 
%% 
%%   http://www.contact-ednotes.sty.de.vu 
%%
%% == 'blog' Required ==
%% 'blogdot' is an extension of 'blog' and must be loaded \emph{later}
%% (but what about options? TODO):
\RequirePackage{blog}
%% == Replacement Rules ==
%% 2012/01/06f.:
\FDpseudoTilde
%% |[[<wikipedia-link>]]|: a 'fifinddo' job is defined that 
%% passes to the ``ligature" job for arrows in 'blog.sty':
\MakeExpandableAllReplacer{blog[[}{[[}{\protect\catchdbrkt}{blog<-}
\def\catchdbrkt#1]]{\Wikiref{#1}}                 %% + t 2012/01/09
%% The stars are inspired by \Wikiref{Markdown} 
%% (thanks to Uwe Ziegenhagen October 2011), 
%% while I have own ideas about them.
\MakeExpandableAllReplacer{blog**}{**}
                                  {\protect\doublestar:}{blog[[}
\MakeExpandableAllReplacer{blog***}{***}
                                  {\protect\triplestar:}{blog**}
% \CopyFDconditionFromTo{blog***}{BlogLIGs}
%% Apostrophes: %% 2012/01/11:
\MakeActiveDef\'{\noexpand'}
\MakeExpandableAllReplacer{blog\string'\string'}{''}
                   {\protect\doubleapostr:}{blog***}
\MakeExpandableAllReplacer{blog\string'\string'\string'}{'''}
               {\protect\tripleapostr:}{blog\string'\string'}
\MakeOther\'
%% Replacing three apostrophes by `\tripleapostr'
%% becomes the first job called with `\UseBlogLigs':
\CopyFDconditionFromTo{blog'''}{BlogLIGs}
%%
%% == Connecting to \LaTeX\ commands ==
%% |\MakePairLaTeXcmd#1#2| replaces `#1<text>#1' by `#2{<text>}':
\newcommand*{\MakePairLaTeXcmd}[2]{%
    \@ifdefinable#1{\def#1:##1#1:{#2{##1}}}} 
    %% ":" for "..." 2012/01/30
%% %% 2012/01/15: 
%% |**<text>**| is turned into `\mystrong{<text>}', 
%% and |***<text>***| is turned into `\myalert{<text>}'. 
%% I have used two shades of red for them:
\MakePairLaTeXcmd\doublestar\mystrong
\MakePairLaTeXcmd\triplestar\myalert
%% As in editing Wikipedia, |''<text>''| renders <text>
%% in italics (or \emph{slanted}), 
%% and |'''<text>'''| renders <text> bold.
\MakePairLaTeXcmd\doubleapostr\textit
\MakePairLaTeXcmd\tripleapostr\textbf
%% 
%% == The End and \acro{HISTORY} ==
\endinput
%% VERSION HISTORY
v0.1    2012/01/06ff. developed in `texblog.fdf'
v0.2    2012/11/29    own file

\DontAddQuotes

%% rm. \pagebreak 2013/01/04
\section{Real Web Pages with 'lnavicol.sty'}
\label{sec:lnavicol}
This is the code and documentation of the package mentioned in 
Sec.~\ref{sec:example-lnavicol}.
\ResetCodeLineNumbers
\ProvidesPackage{lnavicol}[2011/10/13
                           left navigation column with blog.sty]
%%
%% Copyright (C) 2011 Uwe Lueck, 
%% http://www.contact-ednotes.sty.de.vu 
%% -- author-maintained in the sense of LPPL below -- 
%%
%% This file can be redistributed and/or modified under 
%% the terms of the LaTeX Project Public License; either 
%% version 1.3c of the License, or any later version.
%% The latest version of this license is in
%%     http://www.latex-project.org/lppl.txt
%% We did our best to help you, but there is NO WARRANTY. 
%%
%% Please report bugs, problems, and suggestions via 
%% 
%%   http://www.contact-ednotes.sty.de.vu 
%%
%% == 'blog.sty' Required ==
%% ---but what about options (TODO)?    %% 2011/10/09
\RequirePackage{blog} 
%%
%% == Switches ==
%% %% introduced 2011/04/29, it seems
%% There is a ``standard" page width and a ``tight one" 
%% (the latter for contact forms)---|\iftight|:
\newif\iftight 
%% In order to move an anchor to the \emph{top} of the screen 
%% when the anchor is near the page end, the page must get 
%% some extra length by adding empty space at its 
%% bottom---|\ifdeep|:
\newif\ifdeep 
%% 
%% == Page Style Settings (to be set locally) ==
% \newcommand*{\pagebgcolor}{\#f5f5f5}  %% CSS whitesmoke
% \newcommand*{\pagespacing}{\@cellpadding{4} \@cellspacing{7}} 
% \newcommand*{\pagenavicolwidth}{125}
% \newcommand*{\pagemaincolwidth}{584}
% \newcommand*{\pagewholewidth}  {792}
%% == Possible Additions to 'blog.sty' ==
%% === Tables ===
%% |\begin{spancolscell}{<number>}{<style>}|
%% opens an environment that contains a row and a single cell 
%% that will span <number> table cells and have style <style>:
\newenvironment{spancolscell}[2]{%
    \starttr\startTd{\@colspan{#1} #2 % 
                     \@width{100\%}}% %% TODO works? 
    }{\endTd\endtr}
%% The |{hiddencells}| einvironment contains cells that do not align 
%% with other cells in the surrounding table. The purpose is using
%% cells for horizontal spacing.
\newenvironment{hiddencells}
    {\startTable{}\starttr}
    {\endtr\endTable}
%% |{pagehiddencells}| is like `{hiddencells}' except that 
%% the \HTML\ code is indented:
\newenvironment{pagehiddencells}
    {\indentii\hiddencells}
    {\indentii\endhiddencells}
%% |\begin{FixedWidthCell}{<width>}{<style>}| \ opens the 
%% `{FixedWidthCell}' environment. The content will form a cell 
%% of width <width>. <style> are additional formatting parameters:
\newenvironment{FixedWidthCell}[2]
    {\startTd{#2}\startTable{\@width{#1}}%
     \starttr\startTd{}}
    {\endTd\endtr\endTable\endTd}
%% |\tablehspace{<width>}| is a variant of \LaTeX's `\hspace{<glue>}'. 
%% It may appear in a table row: 
\newcommand*{\tablehspace}[1]{\startTd{\@width{#1} /}}
%%
%% === Graphics ===
%% The command names in this section are inspired by the names 
%% in the standard \LaTeX\ \ctanpkgref{graphics} package.
%% (They may need some re-organization TODO.)
%% 
%% |\simpleinclgrf{<file>}| embeds a graphic file <file> without 
%% the tricks of the remaining commands.
\newcommand*{\simpleinclgrf}[1]{\IncludeGrf{alt="" \@border{0}}%
                                           {#1}}
%% |\IncludeGrf{<style>}{<file>}| embeds a graphic file <file> 
%% with style settings <style>:
\newcommand*{\IncludeGrf}[2]{<img #1 src="#2">}
%% |\includegraphic{<width>}{<height>}{<file>}{<border>}{<alt>}{<tooltip>}| 
%% ...:                             %% fine with mdoccorr 2011/10/13
\newcommand*{\includegraphic}[6]{% 
    \IncludeGrf{%
        \@width{#1} \@height{#2} %% data; presentation:
        \@border{#4} 
        alt="#5" \@title{#6}}%
        {#3}}
%% |\insertgraphic{<wd>}{<ht>}{<f>}{<b>}{<align>}{<hsp>}{<vsp>}{<alt>}{<t>}|
%% \\adds <hsp> for the `@hspace' and <vsp> for the `@vspace' 
%% attribute:
\newcommand*{\insertgraphic}[9]{%
    \IncludeGrf{%
        \@width{#1} \@height{#2} %% data; presentation:
        \@border{#4} 
        align="#5" hspace="#6" vspace="#8"
        alt="#8" \@title{#9}}%
        {#3}}
%% |\includegraphic{<wd>}{<ht>}{<file>}{<anchor>}{<border>}{<alt>}{<tooltip>}| 
%% \\uses an image with `\includegraphic' parameters as a link to 
%% <anchor>:
\newcommand*{\inclgrfref}[7]{%
    \fileref{#4}{\includegraphic{#1}{#2}{#3}%
                                {#5}{#6}{#7}}}
%%
%% === \acro{HTTP}/Wikipedia tooltips ===
%% |\httptipref{<tip>}{<www>}{<text>}| \ works like \
%% `\httpref{<www>}{<text>}' except that <tip> appears as ``tooltip":
\newcommand*{\httptipref}[2]{%
  \TagSurr a{\@title{#1}\@href{http://#2}\@target@blank}}
%% |\@target@blank| abbreviates the `@target' setting for 
%% opening the target in a new window or tab:
\newcommand*{\@target@blank}{target="_blank"}
%% % |\wikitipref{<language-code>}{<lemma>}{<text>}| \ 
%% |\wikitipref{<lc>}{<lem>}{<text>}| 
%% works like
%% % \\
%% % `\wikiref{<language-code>}{<lemma>}{<text>}' 
%% `\wikiref{<lc>}{<lem>}{<text>}' 
%% except that 
%% ``Wikipedia" appears as ``tooltip". 
%% |\wikideref| and |\wikienref| are redefined to use it:
\newcommand*{\wikitipref}[2]{%
    \httptipref{Wikipedia}{#1.wikipedia.org/wiki/#2}}
\renewcommand*{\wikideref}{\wikitipref{de}}
\renewcommand*{\wikienref}{\wikitipref{en}}
%%
%% == Page Structure ==
%% The body of the page is a table of three rows and two columns. 
%% === Page Head Row ===
%% |\PAGEHEAD| opens the head row and a single cell that will span 
%% the two columns of the second row.
\newcommand*{\PAGEHEAD}{%
  \startTable{%
    \@align@c\ 
    \@bgcolor{\pagebgcolor}%
    \@border{0}%%                       %% TODO local 
    \pagespacing
    \iftight \else \@width\pagewholewidth \fi 
  }\CLBrk
  %% omitting <tbody>
  \ \comment{ HEAD ROW }\CLBrk
  \indenti\spancolscell{2}{}%
}
% \newcommand*{\headgrf}  [1]{%                     %% rm. 2011/10/09
%     \indentiii\simplecell{\simpleinclgrf{#1}}}
% \newcommand*{\headgrfskiptitle}[3]{%
%   \pagehiddencells
%     \headgrf{#1}\CLBrk
%     \headskip{#2}\CLBrk
%     \headtitle1{#3}\CLBrk
%   \endpagehiddencells}
%% |\headuseskiptitle{<grf>}{<skip>}{<title>}|
%% first places <grf>, then skips horizontally by <skip>, 
%% and then prints the page title as \xmltagcode{h1}:
\newcommand*{\headuseskiptitle}[3]{%
  \pagehiddencells\CLBrk
    \indentiii\simplecell{#1}\CLBrk
    \headskip{#2}\CLBrk
    \headtitle1{#3}\CLBrk
  \endpagehiddencells}
%% |\headskip{<skip>}| is like `\tablehspace{<skip>}'
%% except that the \HTML\ code gets an indent.
\newcommand*{\headskip}    {\indentiii\tablehspace}
%% Similarly, |\headtitle{<digit>}{<text>}| is like 
%% `\heading<digit>{<text>}' apart from an indent and 
%% being put into a cell:
\newcommand*{\headtitle}[2]{\indentiii\simplecell{\heading#1{#2}}}
%%
%% === Navigation and Main Row ===
%% |\PAGENAVI| closes the head row and opens the ``navigation" 
%% column, actually including an `{itemize}' environment.
%% Accordingly, `writings.fdf' has a command `\fileitem'. 
%% But it seems that I have not been sure ...
\newcommand*{\PAGENAVI}{%
    \indenti\endspancolscell\CLBrk
    \indenti\starttr\CLBrk 
    \ \comment{NAVIGATION COL}\CLBrk 
    \indentii\FixedWidthCell\pagenavicolwidth
                           {\@class{paper} 
%% <- using `@class'=`paper' here is my brother's idea, 
%% not sure about it ...
                           \@valign@t}
    %% omitting `\@height{100\%}' 
    \itemize}
%% |\PAGEMAINvar{<width>}| closes the navigation column 
%% and opens the ``main content" column. The latter gets 
%% width <width>:
\newcommand*{\PAGEMAINvar}[1]{%
    \indentii\enditemize\ \endFixedWidthCell\CLBrk
    \ \comment{ MAIN COL }\CLBrk
    \indentii\FixedWidthCell{#1}{}} 
%% ... The width may be specified as |\pagemaincolwidth|, 
%% then |\PAGEMAIN| works like `\PAGEMAINvar{\pagemaincolwidth}':
\newcommand*{\PAGEMAIN}{\PAGEMAINvar\pagemaincolwidth}
%%
%% === Footer Row ===
%% |\PAGEFOOT| closes the ``main content" column as well as 
%% the second row, and opens the footer row:
\newcommand*{\PAGEFOOT}{%
    \indentii\endFixedWidthCell\CLBrk
%     \indentii\tablehspace{96}\CLBrk %% vs. \pagemaincolwidth
  %% <- TODO margin right of foot
    \indenti\endtr\CLBrk
    \ \comment{ FOOT ROW / }\CLBrk
    \indenti\spancolscell{2}{\@class{paper} \@align@c}%
%% <- again class ``paper"!?
}
%% |\PAGEEND| closes the footer row and provides all the rest 
%% ... needed?
\newcommand*{\PAGEEND}{\indenti\endspancolscell\endTable}
%%
%% == The End and HISTORY ==
\endinput

HISTORY 

2011/04/29   started (? \if...)
2011/09/01   to CTAN as `twocolpg.sty'
2011/09/02   renamed
2011/10/09f. documentation more serious 
2011/10/13   `...:' OK


\section{Beamer Presentations with 'blogdot.sty'}
\subsection{Overview}
'blogdot.sty' extends 'blog.sty' in order to construct ``\HTML\ 
slides." One ``slide" is a 3$\times$3 table such that 
\begin{enumerate}
  \item it \strong{fills} the computer \strong{screen}, 
  \item the center cell is the \strong{``type area,"}
  \item the ``margin cell" below the center cell 
        is a \strong{link} to the \strong{next} ``slide,"
  \item the lower right-hand cell is a \strong{``restart"} link.
\end{enumerate}
Six \strong{size parameters} listed in Sec.~\ref{sec:dot-size} 
must be adjusted to the screen in `blogdot.cfg' 
(or in a file with project-specific definitions).

We deliver a file |blogdot.css| containing \strong{\acro{CSS}} font size 
declarations that have been used so far; you may find better ones 
or ones that work better with your screen size, or you may need to add 
style declarations for additional \HTML\ elements.

Another parameter that the user may want to modify is the 
\strong{``restart" anchor name} |\BlogDotRestart| 
(see Sec.~\ref{sec:dot-fin}). 
Its default value is |START| for the ``slide" opened by the command 
|\titlescreenpage| that is defined in Sec.~\ref{sec:dot-start}.

That slide is meant to be the ``\strong{title} slide" 
of the presentation. In order to \strong{display} it, 
I recommend to make and use a \strong{link} to |START| somewhere 
(such as with 'blog.sty''s `\ancref' command). 
The \emph{content} of the title slide is \emph{centered} horizontically, 
so certain commands mentioned \emph{below} 
(centering on other slides) may be useful.

\emph{After} `\titlescreenpage', the next main \strong{user commands} 
are
\begin{description}
  \cmdboxitem|\nextnormalscreenpage{<anchor-name>}| \
    starts a slide whose content is aligned flush left,
  \cmdboxitem|\nextcenterscreenpage{<anchor-name>}| \
    starts a slide whose content is centered horizontally.
\end{description}
---cf.~Sec.~\ref{sec:dot-next}. Right after these commands, 
as well as right after `\title'\-`screen`\-'page', code is used to 
generate the content of the \strong{type area} of the corresponding 
slide. Another `\next...' command closes that content and opens 
another slide. The presentation (the content of the very last slide) 
may be finished using |\screenbottom{<final>}| where <final> may be 
arbitrary, or `START' may be a fine choice for <final>.

Finally, there are user commands for \strong{centering} slide content 
horizontically (cf.~Sec.~\ref{sec:dot-type}): 
\begin{description}
  \cmdboxitem|\cheading{<digit>}{<title>}| \
    ``printing" a heading centered horizontically---even on slides 
    whose remaining content is aligned \emph{flush left} 
    (I have only used <digit>=2 so far), 
  \cmdboxitem|\begin{textblock}{<width>}| \     %% not metavar 2012/07/19
    ``printing" the content of a `{textblock}' environment with 
    maximum line width <width> flush left, 
    while that ``block" as a whole may be centered 
    horizontically on the slide due to choosing 
    `\nextcenterscreenpage'---especially for \strong{list} 
    environments with entry lines that are shorter than the 
    type area width and thus would not look centered 
    (below a centered heading from `\cheading'). 
\end{description}

The so far single \strong{example} of a presentation prepared using 'blogdot' 
is \ctanfileref{info/fifinddo-info}{dantev45.htm}
%% <- 2011/10/21 ->
(\ctanpkgref{fifinddo-info} bundle), 
a sketch of applying 'fifinddo' to package 
documentation and \HTML\ generation. A ``driver" file is needed 
for generating the \HTML\ code for the presentation from a `.tex' 
source by analogy to generating any \HTML\ file using 'blog.sty'. 
For the latter purpose, I have named my driver files `makehtml.tex'. 
For `dantev45.htm', I have called that file |makedot.tex|, 
the main difference to `makehtml.tex' is loading `blogdot.sty' 
in place of `blog.sty'.

This example also uses a file `dantev45.fdf' that defines some 
commands that may be more appropriate as user-level commands 
than the ones presented here (which may appear to be still too 
low-level-like): 
\begin{description}
  \cmdboxitem|\teilpage{<number>}{<title>}|
    making a ``cover slide" for announcing a new ``part" 
    of the presentation in German, 
  \cmdboxitem|\labelsection{<label>}{<title>}|
    starting a slide with heading <title> 
    and with anchor <label> 
    (that is displayed on clicking a \emph{link} to 
     <label>)---using 
     \[`\nextnormalscreenpage{<label>}'\mbox{ and } 
     `\cheading2{<title>}',\] 
  \cmdboxitem|\labelcentersection{<label>}{<title>}|
    like the previous command except that the slide content will be 
    \emph{centered} horizontally, using 
    \[`\nextcenterscreenpage{<title>}'.\]
\end{description}

%% 2011/10/10:
\strong{Reasons} to make \HTML\ presentations may be:\ \
(i)~As opposed to office software, this is a transparent light-weight 
approach.\ \
Considering \emph{typesetting} slides with \TeX,\ \
(ii)~\TeX's advanced typesetting abilities such as automatical 
page breaking are not very relevant for slides;\ \
(iii)~a typesetting run needs a second or a few seconds, 
while generating \HTML\ with 'blog.sty' needs a fraction of a second;\ \ 
(iv)~adjusting formatting parameters such as sizes and colours 
needed for slides is somewhat more straightforward with \HTML\ 
than with \TeX.

%% 2011/10/11, 2011/10/15:
\strong{Limitations:} \
First I was happy about how it worked on my netbook, 
but then I realized how difficult it is to present the ``slides" ``online."
Screen sizes (centering) are one problem. 
(Without the ``restart" idea, this might be much easier.)
Another problem is that the ``hidden links" don't work with 
\Wikienref{Internet Explorer} as they work with 
\Wikienref{Firefox}, \Wikienref{Google Chrome}, and 
\Wikiendisambref{Opera}{web browser}.
% I am now working at an easy choice of ``recompiling options." 
And finally, in internet shops some 
\HTML\ entities/symbols were not supported. 
In any case I (again) became aware of the fact 
that \HTML\ is not as \strong{``portable"} as \acro{PDF}.

Some \strong{workarounds} are described in Sec.~\ref{sec:cfgs}. 
|\FillBlogDotTypeArea| has two effects: \ (i)~providing an additional 
link to the \emph{next} slide for MSIE, \ (ii)~\emph{widening} 
and centering the \emph{type area} on larger screens 
than the one which the presentation originally was made for. \ 
An optional argument of |\TryBlogDotCFG| is offered for a `.cfg' file 
overriding the original settings for the presentation. 
Using it, I learnt that for ``portability," some manual line breaks 
(`\\', \xmltagcode{br}) should be replaced by ``ties" between the 
words \emph{after} the intended line break 
(when the line break is too ugly in a wider type area). 
For keeping the original type area width on wider screens 
(for certain ``slides", perhaps when line breaks really are wanted 
 to be preserved), the |{textblock}| environment may be used. 
Better \HTML\ and \acro{CSS} expertise may eventually 
lead to better solutions. 

The \strong{name} \qtd{blogdot} is a ``pun" on the name of the 
\ctanpkgref{powerdot} package (which in turn refers to 
``\Wikienref{PowerPoint}").

\subsection{File Header}
\ResetCodeLineNumbers
\NeedsTeXFormat{LaTeX2e}[1994/12/01] %% \newcommand* etc. 
\ProvidesPackage{blogdot}[2013/01/22 v0.41b HTML presentations (UL)]
%% copyright (C) 2011 Uwe Lueck, 
%% http://www.contact-ednotes.sty.de.vu 
%% -- author-maintained in the sense of LPPL below.
%%
%% This file can be redistributed and/or modified under 
%% the terms of the LaTeX Project Public License; either 
%% version 1.3c of the License, or any later version.
%% The latest version of this license is in
%%     http://www.latex-project.org/lppl.txt
%% We did our best to help you, but there is NO WARRANTY. 
%%
%% Please report bugs, problems, and suggestions via 
%% 
%%   http://www.contact-ednotes.sty.de.vu 
%%
%% == 'blog' Required ==
%% 'blogdot' is an extension of 'blog' 
%% (but what about options? TODO):
\RequirePackage{blog}
%% == Size Parameters ==
%% \label{sec:dot-size}
%% I assume that it is clear what the following
%% six page dimension parameters 
%% \begin{quote}
%% |\leftpagemargin|, |\rightpagemargin|, 
%% |\upperpagemargin|,\\|\lowerpagemargin|, 
%% |\typeareawidth|, |\typeareaheight|
%% \end{quote}
%% mean. 
%% The choices are what I thought should work best 
%% on my 1024$\times$600 screen (in fullscreen mode); 
%% but I had to optimize the left and right margins experimentally
%% (with Mozilla Firefox~3.6.22 for Ubuntu canonical~-~1.0).
%% It seems to be best when the horizontal parameters 
%% together with what the brouswer adds 
%% (scroll bar, probably 32px with me) 
%% sum up to the screen width.
\newcommand*{\leftpagemargin}{176}
\newcommand*{\rightpagemargin}{\leftpagemargin}
%% So |\rightpagemargin| ultimately is the same as 
%% |\leftpagemargin| as long as you don't redefine it, 
%% and it suffices to `\renewcommand' `\leftpagemargin'
%% in order to get a horizontically centered type area 
%% with user-defined margin widths.---Something analogous
%% applies to |\upperpagemargin| and |\lowerpagemargin|:
\newcommand*{\upperpagemargin}{80}
\newcommand*{\lowerpagemargin}{\upperpagemargin}
%% A difference to the ``horizontal" parameters is 
%% (I expect) that the position of the type area on the 
%% screen is affected by |\upperpagemargin| only, 
%% and you may choose |\lowerpagemargin| just large enough 
%% that the next slide won't be visible on any computer screen 
%% you can think of.
\newcommand*{\typeareawidth}{640}
\newcommand*{\typeareaheight}{440}
%% Centering with respect to web page body may work better on 
%% different screens (2011/10/03), but it doesn't work here
%% (2011/10/04).
% \renewcommand*{\body}{%
%     </head>\CLBrk
%     <body \@bgcolor{\bodybgcolor} \@align@c>}
%% |\CommentBlogDotWholeWidth| procuces no \HTML\ code ...
\global\let\BlogDotWholeWidth\@empty
%% ... unless calculated with |\SumBlogDotWidth|: 
\newcommand*{\SumBlogDotWidth}{%
    \relax{%                        %% \relax 2011/10/22 magic ...
    \count@\typeareawidth
    \advance\count@ \leftpagemargin
    \advance\count@\rightpagemargin
    \typeout{ * blogdot slide width = \the\count@\space*}%
    \xdef\CommentBlogDotWholeWidth{%
        \comment{ slide width = \the\count@\ }}}}
%%
%% == (Backbone for) Starting a ``Slide" ==
%% \label{sec:dot-start}
%% |\startscreenpage{<style>}{<anchor-name>}|
\newcommand*{\startscreenpage}[2]{%% 0 2011/09/25!?:
    \\\CLBrk                                %% 2012/11/19
%% <- `\\' suddenly necessary, likewise in `texblog.fdf'
%%    with `\NextView' and `\nextruleview'. 
%%    Due to recent `firefox'?              %% 2012/11/21
    \startTable{%
        \@cellpadding{0} \@cellspacing{0}%
        \maybe@blogdot@borders              %% 2011/10/12
        \maybe@blogdot@frame                %% 2011/10/14
    }%
    \CLBrk                                  %% 2011/10/03
    \starttr
%% First cell determines both
%% height of upper page margin |\upperpagemargin|
%% and
%% width of left page margin |\leftpagemargin|:
      \startTd{\@width {\leftpagemargin }%
               \@height{\upperpagemargin}}%
%         \textcolor{\bodybgcolor}{XYZ}%
      \endTd
%% Using |\typeareawidth|:
%       \startTd{\@width{\typeareawidth}}\endTd
      \simplecell{%
        \CLBrk
        \hanc{#2}{\hvspace{\typeareawidth}% 
                          {\upperpagemargin}}%
        \CLBrk
      }%
%% Final cell of first row determines right margin width:
      \startTd{\@width{\leftpagemargin}}\endTd
    \endtr
    \starttr
    \emptycell\startTd{\@height{\typeareaheight}#1}%
}
%% |\titlescreenpage| \ (`\STARTscreenpage' TODO?) \ %% 2011/10/03 \ 2012/11/19
%% opens the title page (I thought). To get it to your screen, 
%% (make and) click a link like 
%% \[`\ancref{START}{start presentation}':\]
\newcommand*{\titlescreenpage}{%
    \startscreenpage{\@align@c}{START}}
%% 
%% == Finishing a ``Slide" and ``Restart" (Backbone) ==
%% \label{sec:dot-fin}
%% |\screenbottom{<next-anchor>}| finishes the current slide 
%% and links to the <next-anchor>, the anchor of a slide opened by 
%% \[`\startscreenpage{<style>}{<next-anchor>}'.\] 
%% More precisely, the margin below the type area is that link.
%% The corner at its right is a link to the anchor to whose name 
%% |\BlogDotRestart| expands. 
\newcommand*{\screenbottom}[1]{%
    \ifFillBlogDotTypeArea 
      <p>\ancref{#1}{\BlogDotFillText}%    %% not </p> 2011/10/22
    \fi
    \endTd\emptycell
    \endtr
    \CLBrk
    \tablerow{bottom margin}%                       %% 2011/10/13
      \emptycell
      \CLBrk
      \startTd{\@align@c}%
        \ancref{#1}{\HVspace{\BlogDotBottomFill}%
%% <- seems to be useless now (2011/10/15).
                            {\typeareawidth}%
                            {\lowerpagemargin}}%
      \endTd
      \CLBrk
      \simplecell{\ancref{\BlogDotRestart}% 
                         {\hvspace{\rightpagemargin}% 
                                  {\lowerpagemargin}}}%
    \endtablerow
    \CLBrk
    \endTable
}
%% The default for |\BlogDotRestart| is |START|---the title page. 
%% You can `\renew'\-`command' it so you get to a slide 
%% containing an overview of the presentation.
\newcommand*{\BlogDotRestart}{START}
%% 
%% == Moving to Next ``Slide" (User Level) == 
%% \label{sec:dot-next}
%% |\nextscreenpage{<style>}{<anchor-name>}|
%% puts closing the previous slide and opening the next 
%% one---having anchor name `<anchor-name>'---together.
%% <style> is for style settings for the next page, 
%% made here for choosing between centering the page/slide content 
%% and aligning it flush left.
\newcommand*{\nextscreenpage}[2]{%
    \screenbottom{#2}\CLBrk
    \hrule           \CLBrk 
    \startscreenpage{#1}{#2}}
%% |\nextcenterscreenpage{<anchor-name>}| chooses centering 
%% the slide content:
\newcommand*{\nextcenterscreenpage}{\nextscreenpage{\@align@c}}
%% |\nextnormalscreenpage{<anchor-name>}| chooses flush left
%% on the type area determined by |\typeareawidth|:
\newcommand*{\nextnormalscreenpage}{\nextscreenpage{}}
%% 
%% == Constructs for Type Area ==
%% \label{sec:dot-type}
%% If you want to get centered titles with \xmltagcode{h2} etc., 
%% you should declare this in `.css' files. But you may consider 
%% this way too difficult, and you may prefer to declare this 
%% right in the \HTML\ code. That's what I do! I use 
%% |\cheading{<digit>}{<text>}| for this purpose. 
\newcommand*{\cheading}[1]{\CLBrk\TagSurr{h#1}{\@align@c}}
%% |\begin{textblock}{<width>}| opens a |{textblock}| 
%% environment. The latter will contain text that will be flush left
%% in a narrower text area---of width <width>---than the one 
%% determined by |\typeareawidth|. It may be used on 
%% "centered" slides. It is made for lists whose entries are 
%% so short that the page would look unbalanced under a 
%% centered title with the list adjusted to the left 
%% of the entire type area. (Thinking of standard \LaTeX, 
%% it is almost the `{minipage}' environment, however lacking 
%% the footnote feature, in that respect it is rather 
%% similar to `\parbox' which however is not an environment.)
\newenvironment*{textblock}[1]
    {\startTable{\@width{#1}}\starttr\startTd{}}
    {\endTd\endtr\endTable}
%%
%% == Debugging and `.cfg's ==
%% \label{sec:cfgs}
%% |\ShowBlogDotBorders| shows borders of the page margins 
%% and may be undone by |\DontShowBlogDotBorders|:
\newcommand*{\ShowBlogDotBorders}{%
    \def\maybe@blogdot@borders{rules="all"}}
\newcommand*{\DontShowBlogDotBorders}{%
    \let\maybe@blogdot@borders\@empty}
\DontShowBlogDotBorders
%% %% 2011/10/14:
%% |\ShowBlogDotFrame| shows borders of the page margins 
%% and may be undone by |\DontShowBlogDotFrame|:
\newcommand*{\ShowBlogDotFrame}{%
    \def\maybe@blogdot@frame{\@frame@box}}
\newcommand*{\DontShowBlogDotFrame}{%
    \let\maybe@blogdot@frame\@empty}
\DontShowBlogDotFrame
%% However, the rules seem to affect horizontal positions ...
%%
%% |\BlogDotFillText| is a dirty trick ... seems to widen 
%% %% doc. extended 2011/10/13
%% the type area and this way centers the text on wider screens 
%% than the one used originally. Of course, this can corrupt 
%% intended line breaks. 
\newcommand*{\BlogDotFillText}{%            %% 2011/10/11
    \center
        \BlogDotFillTextColor{%             %% 2011/10/12
%                 X\\X                      %% insufficient
                 X X X X X X X X X X X X X X X X X X X X 
                 X X X X X X X X X X X X X X X X X X X X 
                 X X X X X X X X X X 
                 X X X X X X X X X X 
%                  X X X X X X X X X X X X X X X X X X X X 
        }
    \endcenter
}
%% |\FillBlogDotTypeArea| fills `\BlogDotFillText' into the 
%% type area, also as a link to the next slide. This may widen
%% the type area so that the text is centered on wider screens 
%% than the one the \HTML\ page was made for. The link may serve 
%% as an alternative to the bottom margin link 
%% (which sometimes fails). 
%% `\FillBlogDotTypeArea'                   %% 2011/10/22
%% can be undone 
%% by |\DontFillBlogDotTypeArea|:
\newcommand*{\FillBlogDotTypeArea}{%
    \let\ifFillBlogDotTypeArea\iftrue 
    \typeout{ * blogdot filling type area *}}       %% 2011/10/13
\newcommand*{\DontFillBlogDotTypeArea}{%
    \let\ifFillBlogDotTypeArea\iffalse}
\DontFillBlogDotTypeArea
%% |\FillBlogDotBottom| fills `\BlogDotFillText' into the 
%% center bottom cell. I tried it before `\FillBlogDotTypeArea'
%% and I am not sure ... 
%% It can be undone by 
%% |\DontFillBlogDotBottom|:
\newcommand*{\FillBlogDotBottom}{%
    \let\BlogDotBottomFill\BlogDotFillText}
%% ... actually, it doesn't seem to make a difference! 
%% (2011/10/13)
\newcommand*{\DontFillBlogDotBottom}{\let\BlogDotBottomFill\@empty} 
\DontFillBlogDotBottom
%% |\DontShowBlogDotFillText| makes `\BlogDotFillText' invisible,\\ 
%% |\ShowBlogDotFillText| makes it visible. 
%% Until 2011/10/22, `\textcolor' ('blog.sty') used the 
%% \xmltagcode{font} element that is deprecated. 
%% I still use it here because it seems to suppress the 
%% `hover' \acro{CSS} indication for the link. 
%% (I might offer a choice---TODO)
\newcommand*{\DontShowBlogDotFillText}{%
%     \def\BlogDotFillTextColor{\textcolor{\bodybgcolor}}}
    \def\BlogDotFillTextColor{%
        \TagSurr{font}{color="\bodybgcolor"}}}
\newcommand*{\ShowBlogDotFillText}{%
    \def\BlogDotFillTextColor{\textcolor{red}}}
\DontShowBlogDotFillText
%% As of 2013/01/22, 'texlinks.sty' provides    %% adjusted 2013/01/22
%% `\ctanfileref{<path>}{<file-name>}' that uses an online 
%% \TeX\ archive randomly chosen or determined by the user. 
%% This is preferable for an online version of the presentation. 
%% In `dantev45.htm', this is used for example files.
%% When, on the other hand, internet access during the presentation is 
%% bad, such example files may instead be loaded from the 
%% ``current directory." \ |\usecurrdirctan| \ modifies `\ctanfileref' 
%% for this purpose (i.e., it will ignore <path>):
\newcommand*{\usecurrdirctan}{%
    \renewcommand*{\ctanfileref}[2]{%
        \hnewref{}{##2}{\filenamefmt{##2}}}}
%% (Using a local \acro{TDS} tree would be funny, but I don't 
%%  have good idea for this right now. )
%%
%% |\TryBlogDotCFG| looks for `blogdot.cfg', 
%% \[|\TryBlogDotCFG[<file-name-base>]|\]       %% \[...\] 2011/10/21 
%% looks for `<file-name-base>.cfg' 
%% (for recompiling a certain file):
\newcommand*{\TryBlogDotCFG}[1][blogdot]{%
    \InputIfFileExists{#1.cfg}{%
        \typeout{
            * Using local settings from \string`#1.cfg\string' *}%
    }{}%
}
\TryBlogDotCFG
%%
%% %% rm. \pagebreak 2013/01/04
%% == The End and HISTORY ==
\endinput
%% VERSION HISTORY
v0.1    2011/09/21f.  started
        2011/09/25    spacing/padding off
        2011/09/27    \CLBrk
        2011/09/30    \BlogDotRestart
        used for DANTE meeting
v0.2    2011/10/03    four possibly independent page margin 
                      parameters; \hvspace moves to texblog.fdf
        2011/10/04    renewed \body commented out
        2011/10/07    documentation
        2011/10/08    added some labels
        2011/10/10    v etc. in \ProvidesPackage
        part of morehype RELEASE r0.5 
v0.3    2011/10/11    \HVspace, \BlogDotFillText
        2011/10/12    commands for \BlogDotFillText
        2011/10/13    more doc. on "debugging"; 
                      \ifFillBlogDotTypeArea, \tablerow, messages
        2011/10/14    \maybe@blogdot@frame
        2011/10/15    doc. note: \HVspace useless
        part of morehype RELEASE r0.51 
v0.4    2011/10/21    \usecurrdirctan
        2011/10/22    FillText with <p> instead of </p>, its color 
                      uses <font>; some more reworking of doc.
        part of morehype RELEASE r0.6 
v0.41   2012/11/19    \startscreenpage with \\; doc. \ 
        2012/11/21    updating version infos, doc. \pagebreak
v0.41a  2013/01/04    rm. \pagebreak 
        part of morehype RELEASE r0.81
v0.41b  2013/01/22    adjusted doc. on `texlinks'


\end{document}

HISTORY

2010/11/05   for v0.2
2010/11/11   for v0.3
2011/01/23   using readprov and color
2011/01/27   using \urlfoot
2011/09/01   using new makedoc.cfg incl. \acro and \foothttp...; 
             extension for twocolpg.sty
with morehype RELEASE r0.4
2011/09/02   twocolpg.sty renamed into lnavicol.sty, typo fixes
2011/09/08   \HTML
2011/09/23   TODO in abstract blue again
2011/10/05   umlaut-a in schreibt.tex
2011/10/07f. blogdot
2011/10/09   lnavicol
2011/10/10   tuning; reasons for blogdot
2011/10/11   limitations of blogdot, corrected makedoc code
2011/10/15   more on limitations of blogdot; 
             abstract on lnavicol and blogdot
2011/10/21   links to fifinddo-info/dantev45.htm
2011/11/05   using \MakeSingleDoc from makedoc.sty v0.42
2011/11/08   playing with alternatives to defective `blog.sty' 
             in page head
2011/11/09   so use \file with new makedoc.cfg for hyperref; \CSS 
2011/11/23   \secref
2012/07/19   `textblock' not metavar
2012/08/07   three section levels
2012/10/03   ize -> ese, some restructuring, corr. \HeaderLines
2012/10/05   adjusted \HeaderLines (differ!)
2012/11/29   adding `blogligs' and `markblog'
2012/11/30   hello world, texblog sample removed, url foots
2012/12/20   filedate checks, doc. more about `markblog'
2013/01/04   \pagebreak +/-
