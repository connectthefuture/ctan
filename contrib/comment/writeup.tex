\documentclass[parskip=false, DIV=8, headings=normal, pagesize=auto]{scrartcl}

\usepackage{fixltx2e}
\usepackage{etex}
\usepackage{xspace}
\usepackage{lmodern}
\usepackage[T1]{fontenc}
\usepackage{textcomp}
\usepackage{microtype}
\usepackage[unicode=true]{hyperref}

\newenvironment*{noverb}{%
  \flushleft
  \vskip\parskip
  \parskip=0pt\relax
}{%
  \endflushleft
}

\newcommand*{\mail}[1]{\href{mailto:#1}{\texttt{#1}}}
\newcommand*{\pkg}[1]{\textsf{#1}}
\newcommand*{\cs}[1]{\texttt{\textbackslash#1}}
\makeatletter
\newcommand*{\cmd}[1]{\cs{\expandafter\@gobble\string#1}}
\makeatother
\newcommand*{\env}[1]{\texttt{#1}}
\newcommand*{\meta}[1]{\textlangle\textsl{#1}\textrangle}
\newcommand*{\marg}[1]{\texttt{\{}\meta{#1}\texttt{\}}}

\addtokomafont{title}{\rmfamily}

\title{The \pkg{comment} package\thanks{This manual corresponds to
    \pkg{comment}~v3.8 of July 2016.}}
%
\author{Victor Eijkhout\\\mail{victor@eijkhout.net}} \date{August 2016}


\begin{document}

\maketitle

\section{Purpose:}

Selectively in/exclude pieces of text: the user can define new
comment versions, and each is controlled separately.
Special comments can be defined where the user specifies the
action that is to be taken with each comment line.

Plain \TeX\ support has been phased out.

As of 3.8 the package will increasingly use e\TeX\ features, for
instance to solve Unicode support issues.

\section{Usage:}

All text included between
%
\begin{verbatim}
\begin{comment}
 ... 
\end{comment}
\end{verbatim}
%
is discarded. 

The opening and closing commands should appear on a line
of their own. No starting spaces, nothing after it.
This environment should work with arbitrary amounts
of comment, and the comment can be arbitrary text.

Other `\env{comment}' environments are defined by
and are selected/deselected with
%
\begin{verbatim}
\includecomment{versiona}
\excludecoment{versionb}
\end{verbatim}
%
These environments are used as
%
\begin{verbatim}
\versiona ... \endversiona
\end{verbatim}
%
or
%
\begin{verbatim}
\begin{versiona} ... \end{versiona}
\end{verbatim}
%
with the opening and closing commands again on a line of 
their own.

\pagebreak[1]

\LaTeX\ users note: for an included comment, the
\cmd{\begin} and \cmd{\end} lines act as if they don't exist.
In particular, they don't imply grouping, so assignments 
\&c are not local.

\pagebreak[2]

Special comments are defined as
%
\begin{noverb}
\cmd{\specialcomment}\marg{name}\marg{before commands}\marg{after commands}
\end{noverb}
%
where the second and third arguments are executed before
and after each comment block. You can use this for global
formatting commands.
To keep definitions \&c local, you can include \cmd{\begingroup}
in the `\meta{before commands}' and \cmd{\endgroup} in the `\meta{after commands}'.
ex:
%
\begin{verbatim}
\specialcomment{smalltt}
    {\begingroup\ttfamily\footnotesize}{\endgroup}
\end{verbatim}
%
You do \emph{not} have to do an additional
%
\begin{verbatim}
\includecomment{smalltt}
\end{verbatim}
%
To remove `\env{smalltt}' blocks, give \verb+\excludecomment{smalltt}+
after the definition.

Processing comments can apply processing to each line.
%
\begin{noverb}
\cmd{\processcomment}\marg{name}\marg{each-line commands}\marg{before commands}\marg{after commands}
\end{noverb}
%
By defining a control sequence
%
\begin{verbatim}
\def\Thiscomment##1{...}
\end{verbatim}
%
in the before commands the user can
specify what is to be done with each comment line.
BUG this does not work quite yet BUG

Trick for short in/exclude macros (such as \verb+\maybe{this snippet}+):
%
\begin{verbatim}
\includecomment{cond}
\newcommand{\maybe}[1]{}
\begin{cond}
\renewcommand{\maybe}[1]{#1}
\end{cond}
\end{verbatim}


\section{Basic approach of the implementation:}

to comment something out, scoop up  every line in verbatim mode
as macro argument, then throw it away.
For inclusions, in \LaTeX\ the block is written out to
a file \cmd{\CommentCutFile} (default ``\texttt{comment.cut}''), which is
then included.
In plain \TeX\ (and other formats) both the opening and
closing comands are defined as noop.


\section{Changes in version 3.1}

\begin{itemize}
\item updated author's address
\item cleaned up some code
\item trailing contents on \cmd{\begin}\marg{env} line is always discarded
    even if you've done \cmd{\includecomment}\marg{env}
\item comments no longer define grouping!! you can even
  % 
\begin{verbatim}
\includecomment{env}
\begin{env}
\begin{itemize}
\end{env}
\end{verbatim}
  % 
  Isn't that something\ldots
\item included comments are written to file and input again.
\end{itemize}


\section{Changes in 3.2}

\begin{itemize}
\item \cmd{\specialcomment} brought up to date (thanks to Ivo Welch).
\end{itemize}


\section{Changes in 3.3}

\begin{itemize}
\item updated author's address again
\item parametrised \cmd{\CommentCutFile}
\end{itemize}


\section{Changes in 3.4}

\begin{itemize}
\item added GNU public license
\item added \cmd{\processcomment}, because Ivo's fix (above) brought an
  inconsistency to light.
\end{itemize}

  
\section{Changes in 3.5}

\begin{itemize}
\item corrected typo in header.
\item changed author email
\item corrected \cmd{\specialcomment} yet again.
\item fixed excludecomment of an earlier defined environment.
\end{itemize}


\section{Changes in 3.6}

\begin{itemize}
\item The `cut' file is now written more verbatim, using \cmd{\meaning};
  some people reported having trouble with ISO~latin~1, or \texttt{umlaute.sty}.
\item removed some \cmd{\newif} statements.
  Has this suddenly become \cmd{\outer} again?
\end{itemize}


\section{Changes in 3.8}

T1 font encoding is now supported. See t1test.tex.

%% \section{Known bugs:}

%% \begin{itemize}
%% \item \texttt{excludecomment} leads to one superfluous space
%% \item \texttt{processcomment} leads to a superfluous line break
%% \end{itemize}

\end{document}
