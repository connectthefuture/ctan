\documentclass[a4paper,twoside]{book}

\usepackage{etoolbox}
\usepackage{etex}
\usepackage{mathptmx}
\usepackage{courier}
\usepackage[scaled=0.9]{helvet}
\usepackage{makeidx}
\usepackage[landscape,margin=1in,top=1in,bottom=1in]{geometry}
\usepackage{color}
\usepackage[colorlinks,
            plainpages=false,
            linkcolor=black,
            bookmarksopen,
            pdfauthor={Nicola Talbot},
            pdftitle={Creating Flow Frames for Posters, Brochures or Magazines using flowfram.sty},
            pdfkeywords={LaTeX;text frames;posters;brochures;magazines;newspapers}]{hyperref}
\usepackage[style=altlist,toc,nonumberlist]{glossaries}
\usepackage{pgf}
\usepackage[norotate,ttbnotitle,ttbnum]{flowfram}
\usepackage{shapepar}
\usepackage{html}

% Define some commands for consistency and indexing
\newcommand{\styni}[1]{\textsf{#1}}
\newcommand{\envni}[1]{\textsf{#1}}
\newcommand{\ctrni}[1]{\textsf{#1}}
\newcommand{\pkgoptni}[1]{\textsf{#1}}

\newcommand{\sty}[1]{\styni{#1}\index{#1@\styni{#1} package}}
\newcommand{\env}[1]{\envni{#1}\index{#1@\envni{#1} environment}}
\newcommand{\ctr}[1]{\ctrni{#1}\index{#1@\ctrni{#1} counter}}
\newcommand{\cmdname}[1]{\texttt{\symbol{92}#1}\index{#1@\texttt{\symbol{92}#1}}}
\newcommand{\meta}[1]{%
\textnormal{\ensuremath{\langle}\emph{#1}\ensuremath{\rangle}}}
\newcommand{\marg}[1]{\texttt{\char`\{#1\char`\}}}
\newcommand{\oarg}[1]{\texttt{[#1]}}
\newcommand{\appname}[1]{\texttt{#1}\index{#1@\texttt{#1}}}

\newcommand{\pkgopt}[2][]{%
  \ifstrempty{#1}%
  {%
    \pkgoptni{#2}\index{#2@\pkgoptni{#2} option}%
  }%
  {%
    \pkgoptni{#2}=\pkgoptni{#1}\index{#2@\pkgoptni{#2}
option!#1@\pkgoptni{#1}}%
  }%
}

\newcommand{\pkgoptval}[2]{\pkgoptni{#2}\index{#1@\pkgoptni{#1} option!#2@\pkgoptni{#2}}}

\newcommand{\key}[1]{#1\index{frame settings!#1}}
\newcommand{\dq}[1]{``#1''}
\newcommand*{\Index}[1]{#1\index{#1}}

\newsavebox\dfnsbox
\definecolor{defcol}{cmyk}{0,0,0.2,0}
\newlength\dfnlen

\newenvironment{definition}{%
\par\vskip10pt\noindent
\setlength{\dfnlen}{\linewidth}%
\addtolength{\dfnlen}{-2\fboxsep}%
\begin{lrbox}{\dfnsbox}%
\begin{minipage}{\dfnlen}\ignorespaces
}{%
\end{minipage}%
\end{lrbox}%
\colorbox{defcol}{\usebox{\dfnsbox}}%
\par
\vskip10pt\noindent\ignorespacesafterend}

\newcommand{\chapdesc}[1]{%
 \appenddynamiccontents*{chaphead}{\par
   \normalfont\emph{#1}}%
}

\newcommand*{\glsindex}[1]{\protect\index{#1}}

\newcommand*{\htmlnav}{}

\begin{htmlonly}
\renewcommand{\LaTeX}{LaTeX}
\renewcommand{\TeX}{TeX}
\renewcommand{\LaTeXe}{LaTeX2e}
\renewcommand{\latextohtml}{LaTeX2HTML}
\renewcommand{\styni}[1]{\texttt{#1}}
\renewcommand{\envni}[1]{\texttt{#1}}
\renewcommand{\ctrni}[1]{\texttt{#1}}
\renewcommand{\pkgoptni}[1]{\texttt{#1}}
\renewcommand{\meta}[1]{\emph{#1}}
\renewcommand{\dq}[1]{"#1"}

\renewenvironment{definition}{\par}{\par}

\renewcommand*{\chapdesc}[1]{\par\emph{#1}\par}

\newcommand*{\glsindex}[1]{}

\newcommand*{\htmlnav}{%
\par\htmlref{Top}{top}~\textbar~\htmlref{Index}{index}}

\usepackage{amsmath}
\keywords{LaTeX; text frames; posters; brochures; magazines; newspapers}
\end{htmlonly}

% Glossary Stuff

\newglossaryentry{typeblock}{name=typeblock\glsindex{typeblock},%
description={The area of the page where the main body of the text goes.
The width and height of this area are given by 
\protect\cmdname{textwidth} and \protect\cmdname{textheight}}}

\newglossaryentry{flow}{name=flow frame\glsindex{frame!flow},%
description={The frames in a document such that the contents of the 
\protect\env{document} environment flow from one frame to the next in 
the order that they were defined. There must be at least one flow frame 
on every page}}

\newglossaryentry{static}{name=static frame\glsindex{frame!static},%
description={Frames in which text is fixed in place.  The contents are fixed until
explicitly changed or cleared via the \protect\key{clear} key
in \protect\cmdname{setstaticcontents}}}

\newglossaryentry{dynamic}{name=dynamic frame\glsindex{frame!dynamic},%
description={Frames
in which text is fixed in place, but the contents are re-typeset
each time the frame is displayed}}

\newcommand*{\staticordynamic}{\glslink{static}{static} or 
\glslink{dynamic}{dynamic} frame}

\newglossaryentry{frame}{name=frame\glsindex{frame},%
description={A rectangular
area of the page in which text can be placed (not to be
confused with a frame making command). There are three types:
flow, static and dynamic}}

\newglossaryentry{fcmd}{name=frame making command\glsindex{frame
making command},description={A
\LaTeX\ command which places some kind of border around its
argument. For example: \protect\cmdname{fbox}}}

\newglossaryentry{pglist}{name=page list\glsindex{page list},%
description={A list of
pages. This can either be a single keyword: \texttt{all},
\texttt{odd}, \texttt{even} or \texttt{none}, or it can
be a comma-separated list of individual page numbers or
page ranges. For example: \texttt{\textless3,5,7-11,\textgreater15}
indicates pages 1,2,5,7,8,9,10,11 and all pages after page 15.
These numbers refer to the decimal value of the page counter by
default. To make them refer to the absolute physical page number use
the package option \pkgopt[absolute]{pages}}}

\newglossaryentry{pgrange}{name=page range\glsindex{page range},%
description={Page
ranges can be closed, e.g.\ \texttt{5-10}, or open, e.g.
\texttt{\textless7} or \texttt{\textgreater9}}}

\newglossaryentry{bbox}{name=bounding box\glsindex{bounding box},
plural=bounding boxes\glsindex{bounding box},
firstplural=bounding boxes\glsindex{bounding box},
description={The bounding box of a frame is the area allocated for the 
contents of that frame. However the text may not completely fill that 
area, and it is possible that the text may overflow that area}}

\newglossaryentry{idn}{name={identification number 
(IDN)\glsindex{IDN}\glsindex{identification number|see{IDN}}},
text={IDN\glsindex{IDN}},%
first={identification number (IDN)\glsindex{IDN}},
plural={IDNs\glsindex{IDN}},%
firstplural={identification numbers (IDNs)\glsindex{IDN}},
description={A unique
number assigned to each frame, which you can use to identify
the frame when modifying its appearance. Example: if you
have defined 3 flow frames, 2 static frames and 1 dynamic
frame, the flow frames will have IDNs 1, 2 and 3, the static
frames will have IDNs 1 and 2, and the dynamic frame will
have IDN 1}}

\newglossaryentry{idl}{name={identification label 
(IDL)\glsindex{IDL}\glsindex{identification label|see{IDL}}},
text={IDL\glsindex{IDL}},%
first={identification label (IDL)\glsindex{IDL}},
plural={IDLs\glsindex{IDL}},%
firstplural={identification labels (IDLs)\glsindex{IDL}},
description={A unique
label which can be assigned to a frame, enabling you to
refer to the frame by label instead of by its 
IDN}}

\makeglossaries
\makeindex

% Page layout stuff

% set up some background frames to liven up the title page
\newlength{\leftwidth}
\newlength{\rightwidth}

\computeleftedgeodd{\leftwidth}
\setlength{\leftwidth}{-\leftwidth}
\addtolength{\leftwidth}{0.4\textwidth}
\setlength{\rightwidth}{\paperwidth}
\addtolength{\rightwidth}{-\leftwidth}
% only defined on page 1 unfortunately the document
% has more than one page 1, so will need to change the settings after the title page
\vtwotone[1]{\leftwidth}{magenta}{backleft}{\rightwidth}{[cmyk]{0,0.48,0,0}}{backright}

% This is for the back cover.
\vtwotone[none]{\rightwidth}{[cmyk]{0,0.48,0,0}}{lastbackright}{\leftwidth}{magenta}{lastbackleft}

\vtwotonetop[odd]{1cm}{\leftwidth}{magenta}{oddtopleft}{\rightwidth}{[cmyk]{0,0.48,0,0}}{oddtopright}
\vtwotonetop[even]{1cm}{\rightwidth}{[cmyk]{0,0.48,0,0}}{eventopleft}{\leftwidth}{magenta}{eventopright}

% Set the margin width
\setlength{\marginparwidth}{2cm}

% now set up main document frames. Each page has a dynamic
% frame for the chapter heading, and a flow frame for the
% text.
\newflowframe{0.6\textwidth}{\textheight}{0pt}{0pt}[main]
\newdynamicframe{0.38\textwidth}{\textheight}{0.62\textwidth}{0pt}[chaphead]

% swap them round on even pages
\setflowframe*{main}{evenx=0.4\textwidth}
\setdynamicframe*{chaphead}{evenx=0pt,clear}

% make a frames to illustrate shaped frames
\newstaticframe[none]{0.38\textwidth}{0.55\textheight}{0.62\textwidth}{0.45\textheight}[shapedt]
\setstaticframe*{shapedt}{evenx=0pt}

\newstaticframe[none]{0.38\textwidth}{0.45\textheight}{0.62\textwidth}{0pt}[shapedb]
\setstaticframe*{shapedb}{evenx=0pt}

% set the margins to appear on the spine side of the page
\setflowframe*{main}{margin=inner}

% put chapter headings in dynamic frame with IDL chaphead
\dfchaphead*{chaphead}

% append chapter minitocs to same dynamic frame.
\appenddfminitoc*{chaphead}

% change the style of the chapter headings

% numbered chapters:
\renewcommand{\DFchapterstyle}[1]{%
{\raggedright\sffamily\bfseries\Huge\color{blue}\thechapter. #1\par
}}

% unnumbered chapters:
\renewcommand{\DFschapterstyle}[1]{{\raggedright\sffamily\bfseries\Huge\color{blue} #1\par
}}

% Make thumb tabs (specify each tab to be 0.75in high)

\makethumbtabs{0.75in}
\enableminitoc

% Thumbtabs are grey by default which looks a bit boring,
% so change the colours
\setthumbtab{1}{backcolor=[rgb]{0.15,0.15,1}}
\setthumbtab{2}{backcolor=[rgb]{0.2,0.2,1}}
\setthumbtab{3}{backcolor=[rgb]{0.25,0.25,1}}
\setthumbtab{4}{backcolor=[rgb]{0.3,0.3,1}}
\setthumbtab{5}{backcolor=[rgb]{0.35,0.35,1}}
\setthumbtab{6}{backcolor=[rgb]{0.4,0.4,1}}
\setthumbtab{7}{backcolor=[rgb]{0.45,0.45,1}}
\setthumbtab{8}{backcolor=[rgb]{0.5,0.5,1}}

% change the text style on the thumbtabs
\newcommand{\thumbtabstyle}[1]{{\hypersetup{linkcolor=white}%
\textbf{\large\sffamily #1}}}
\setthumbtab{all}{style=thumbtabstyle,textcolor=white}

% set default page style

\pagestyle{plain}

% Put headers and footers in dynamic frames
\makedfheaderfooter

\newlength{\xoffset}
\computerightedgeodd{\xoffset}
\addtolength{\xoffset}{-2cm}
\newlength{\yoffset}
\computebottomedge{\yoffset}

\newcommand{\footstyle}[1]{\bfseries\LARGE #1}

% pages is initially set to none, as I don't want the footer to
% appear on the title page.

\setdynamicframe*{footer}{oddx=\xoffset,y=\yoffset,width=2cm,height=2cm,
backcolor=blue,textcolor=white,style=footstyle,pages=none}

% now work out the x offset for the even pages

\computeleftedgeeven{\xoffset}
\setdynamicframe*{footer}{evenx=\xoffset}

% Now create some frames for the index, and modify
% theindex environment

% These are for the odd pages

\twocolumninarea[none]{0.6\textwidth}{\textheight}{0pt}{0pt}
% Give a label to the last 2 flow frames:
\newcounter{N}
\newcounter{I}
\setcounter{N}{\value{maxflow}}
\addtocounter{N}{-2}
\whiledo{\value{N}<\value{maxflow}}{%
\stepcounter{N}\stepcounter{I}
\setflowframe{\value{N}}{label=oddcol\theI}}

% These are for the even pages

\twocolumninarea[none]{0.6\textwidth}{\textheight}{0.4\textwidth}{0pt}
% Give a label to the last 2 flow frames:

\setcounter{I}{0}
\setcounter{N}{\value{maxflow}}
\addtocounter{N}{-2}
\whiledo{\value{N}<\value{maxflow}}{%
\stepcounter{N}\stepcounter{I}
\setflowframe{\value{N}}{label=evencol\theI}}

\makeatletter
\renewenvironment{theindex}{%
\setflowframe*{oddcol1,oddcol2}{pages=odd}%
\setflowframe*{evencol1,evencol2}{pages=even}%
\setdynamicframe*{chaphead}{clear=false}%
\clearpage
\mbox{}\framebreak\phantomsection
\setflowframe*{main}{pages=none}%
\setdynamiccontents*{chaphead}{\DFschapterstyle{\indexname}%
\idxnav}%
\addcontentsline{toc}{chapter}{\indexname}%
\setlength{\parindent}{0pt}%
\setlength{\parskip}{0pt plus .3pt}%
\let\item\@idxitem
}{%
\setdynamicframe*{chaphead}{clear}%
}

% Define commands to create an index navigation bar.
\newcommand*{\idxgroupheading}[1]{%
\textbf{\hypertarget{idx:#1}{#1}}%
\protected@write\@auxout{}{\string\addtoidxnav{#1}}\indexspace}

\newcommand*{\idxgrouplink}[1]{%
\par\vskip5pt\textbf{\hyperlink{idx:#1}{#1}}}

\newcommand{\idxnav}{}

\newcommand*{\addtoidxnav}[1]{%
\toks@\expandafter{\idxnav}%
\xdef\idxnav{\the\toks@ \noexpand\idxgrouplink{#1}}%
}
\makeatother

\begin{document}\label{top}
\title{Creating Flow Frames for Posters, Brochures or 
Magazines using flowfram.sty version 1.17}
\author{Nicola L. C. Talbot}
\date{2014-09-30}

% swap frames around for title page
\ffswapoddeven*{main}
\dfswapoddeven*{chaphead}

\pagenumbering{alph}
\maketitle

%suppress the background
\setstaticframe*{backleft}{pages=none}
\setstaticframe*{backright}{pages=none}

\noindent
Dr Nicola Talbot\\
Dickimaw Books\\
\url{http://www.dickimaw-books.com/}

\frontmatter
% swap frames back again
\ffswapoddeven*{main}
\dfswapoddeven*{chaphead}

\thumbtabindex
\tableofcontents

% make the footers appear from this point on
\setdynamicframe*{footer}{pages=all}

\clearpage

\mainmatter
\chapter{Introduction}
\enablethumbtabs
\pagenumbering{arabic}
\chapdesc{This chapter provides a brief overview of the package, 
the package options and the various frame types.}

This document is the user manual for the \styni{flowfram} package.
Advanced users wanting further details of the package should read
the documented code \texttt{flowfram.pdf}. Sample files are provided
in the directory \meta{TEXMF}\texttt{/doc/latex/flowfram/samples/}
where \meta{TEXMF} indicates the root \TeX\ installation directory
for this package. (This document is located in 
\meta{TEXMF}\texttt{/doc/latex/flowfram/}.) 

The \styni{flowfram} package is a \LaTeXe\ package designed to 
enable you to create text \glspl{frame} in a document such that 
the contents of the \env{document} environment flow from one 
\gls*{frame} to the next in the order that they were defined.  
This is useful for creating posters
or magazines or any other form of document that does not 
conform to the standard one or two column layout. There's an
optional helper application called
\texttt{flowframtk}\footnote{\url{http://www.dickimaw-books.com/apps/flowframtk/}}
if you prefer to use a graphical user interface to set up the
document layout.

\textbf{The \styni{flowfram} package tries to make \TeX\ do
something it wasn't originally designed to do. It modifies the
output routine and may not always perform as desired. Extra care
must be taken if a paragraph spans frames of unequal width due to
the asynchronous nature of \TeX's output routine.  (See
\latexhtml{\autoref{sec:unexpectedoutput}}{\htmlref{Unexpected 
Output}{sec:unexpectedoutput}}.})

The \styni{flowfram} package provides three types of \gls*{frame}:
\glspl{flow}, \glspl{static} and \glspl{dynamic} with dimensions and 
positions specified by the 
user\footnote{Can I have arbitrary shaped frames? See 
\latexhtml{\autoref{sec:parshape}}{\htmlref{Non-Rectangular Frames}{sec:parshape}}}.
The main contents of the document environment flow from
one \gls*{flow} to the next in the order of definition,
whereas the contents of the static and dynamic frames
are set explicitly using commands described in 
\latexhtml{\autoref{sec:modattr}}{\htmlref{Modifying Frame 
Attributes}{sec:modattr}}. Note that
unless otherwise stated, all co-ordinates are relative to the
bottom left hand corner of the \gls{typeblock}. If you have
a two-sided document, the absolute position of the \gls*{typeblock} 
may vary depending on the values of \cmdname{oddsidemargin} 
and \cmdname{evensidemargin}, and all the \glspl*{frame} will shift
accordingly unless otherwise indicated.

This package has only been tested with a limited number of
class files and packages. Since it modifies the output routine,
it is likely to conflict with any other package which also
does this (such as \sty{longtable}).  

You should load \styni{flowfram} \emph{after} \sty{hyperref} and any 
colour package (e.g.\ \sty{color}).\htmlnav

\section{Package Options}

\begin{description}

 \item[\pkgopt{pages}] Determines whether the \gls{pglist} refers to
  the page number as given by the page counter (\pkgopt[relative]{pages})
  or the absolute page number (\pkgopt[absolute]{pages}). The
  default is \pkgoptval{pages}{relative} to ensure backward compatibility,
  but if you have a document where the page counter is reset it's
  best to use \pkgopt[absolute]{pages}.

 \item[\pkgopt{draft}] Switch on draft mode (see 
  \latexhtml{\autoref{sec:draft}}{\htmlref{Draft Option}{sec:draft}}).

 \item[\pkgopt{final}] Switch off draft mode (default).

 \item[\pkgopt{thumbtabs}] Controls thumbtab contents. See
  \latexhtml{\autoref{sec:thumbtabs}}{\htmlref{Thumbtabs}{sec:thumbtabs}}
  for details.

 \item[\pkgopt{LR}] When using the column style layouts described in 
  \latexhtml{\autoref{sec:Ncolumn}}{\htmlref{Column
  Styles}{sec:Ncolumn}}, define the \glspl{flow} from left to right.
  (Default.)

 \item[\pkgopt{RL}] When using the column style layouts described in 
  \latexhtml{\autoref{sec:Ncolumn}}{\htmlref{Column
  Styles}{sec:Ncolumn}}, define the \glspl{flow} from right to left.

 \item[\pkgopt{rotate}] May have the value \pkgoptval{rotate}{true}
  (rotate text in thumbtabs) or \pkgoptval{rotate}{false}
  (stack text in thumbtabs). (Default is \pkgoptval{rotate}{true}.)

 \item[\pkgopt{color}] May have the value \pkgoptval{color}{true}
 (allow frames to have colour settings) or \pkgoptval{color}{false}
 (disable colour in frame settings). The default is \pkgoptval{color}{true}.

 \item[\pkgopt{verbose}] May have the value \pkgoptval{verbose}{true} or 
 \pkgoptval{verbose}{false}. (Default is \pkgoptval{verbose}{false}.) 
  Provided to assist debugging.

\end{description}

\section{Floats}
\label{sec:floats}

The standard \env{figure} and \env{table} commands will 
behave as usual in the \glspl*{flow}, but their starred versions,
\env{figure*} and \env{table*} behave no differently
from \env{figure} and \env{table}\footnote{This is because
of the arbitrary layout of the flow frames.}.

Floats (such as figures and tables) can only go in 
\glspl*{flow}. However, this package provides
the additional environments: \env{staticfigure} and 
\env{statictable} which can be used in \glspl*{static} 
and \glspl*{dynamic}. Unlike their \env{figure} and
\env{table} counterparts, they are fixed in place, and
so do not take an optional placement specifier. The 
\cmdname{caption} and \cmdname{label} commands can 
be used within \env{staticfigure} and \env{statictable} as
usual, but remember that if the frame is displayed on multiple
pages, you may end up with multiply defined labels.
\htmlnav

\section{Draft Option}
\label{sec:draft}

The \styni{flowfram} package has the package option \pkgopt{draft} 
which will draw the \glspl{bbox} for
each \gls{frame} that has been defined.  At the bottom right of each
\gls*{bbox} (except for the \gls*{bbox} denoting the 
\gls{typeblock}), a marker will be shown in the form:
[\meta{T}:\meta{idn};\meta{idl}], where \meta{T} is a single
letter denoting the \gls*{frame} type, \meta{idn} is the \gls{idn}\
for the \gls*{frame} and \meta{idl} is the \gls{idl}\ for that
\gls*{frame}. Values of \meta{T} are: \texttt{F} (\gls{flow}),
\texttt{S} (\gls{static}) or \texttt{D} (\gls{dynamic}).
Markers of the form: [M:\meta{idn}] indicate that the
\gls*{bbox} is the area taken up by the margin for \gls*{flow} 
with \gls*{idn}\ \meta{idn}. Note that even if a \gls*{frame} 
has been rotated, the \gls*{bbox} will not be rotated.

If you want to show or hide specific types of bounding
boxes, you can use one of the following commands:
\begin{itemize}
\item 
\cmdname{showtypeblocktrue} Display the \gls*{bbox} 
for the \gls*{typeblock}.

\item
\cmdname{showtypeblockfalse} Do not display the \gls*{bbox} 
for the \gls*{typeblock}.

\item
\cmdname{showmarginstrue} Display the \gls*{bbox} 
for the margins.

\item
\cmdname{showmarginsfalse} Do not display the \gls*{bbox} 
for the margins.

\item
\cmdname{showframebboxtrue} Display the \gls*{bbox} 
for the \glspl*{frame}.

\item
\cmdname{showframebboxfalse} Do not display the \gls*{bbox} 
for the \glspl*{frame}.

\end{itemize}

You can see the layout for the current page (irrespective of
whether or not the \pkgopt{draft} option has been set) using
the command:
\begin{definition}
\cmdname{flowframeshowlayout}
\end{definition}

The \styni{flowfram} package also has the options \pkgopt[false]{color} 
and \pkgopt[false]{rotate} for previewers that can not process
colour or rotating specials. (Otherwise you may end up with
large black rectangles obscuring your text, instead of
the pale background colour you were hoping for.)\htmlnav

\section{Chapters}

If the \cmdname{chapter} command has been defined, the \styni{flowfram}
package will modify its definition so that it sets the page style to
\cmdname{chapterfirstpagestyle} for the first page of each chapter. This
command defaults to \texttt{plain}, which is the usual page style
for the first page of a chapter. If you want to use a different
style, you will need to redefine \cmdname{chapterfirstpagestyle} to the
name of the relevant page style. A hook
\begin{definition}
\cmdname{ffprechapterhook}
\end{definition}
is used at the start of \cmdname{chapter} definition \emph{before
\cmdname{clearpage} or \cmdname{cleardoublepage} is called.}

Chapter titles can be placed in a \gls{dynamic} (as in \html{the PDF version of }this document). See 
\latexhtml{\autoref{sec:dfchaphead}}{\htmlref{Putting Chapter Titles
in a Dynamic Frame}{sec:dfchaphead}} for further details.
\htmlnav

\section{Frame Stacking Order}
\label{sec:stacking}

The material on each page is placed in the following order:
\begin{enumerate}
\item Each \gls{static} defined for that page in ascending
order of \gls{idn}.

\item Each \gls{flow} defined for that page in ascending
order of \gls{idn}.

\item Each \gls{dynamic} defined for that page in ascending
order of \gls{idn}.

\item \Glspl{bbox} if the \pkgopt{draft} 
package option has been used.
\end{enumerate}

This ordering can be used to determine if you want something
to overlay or underlay everything else on the page.
Note that the \glspl{frame} do not interact with each other. If
you have two or more overlapping \glspl*{frame}, the text in each 
\gls*{frame} will not attempt to wrap around the other 
\glspl*{frame}, but will simply overwrite 
them.\footnote{Can I have arbitrary
shaped frames? See 
\latexhtml{\autoref{sec:parshape}}{\htmlref{Non-Rectangular
Frames}{sec:parshape}}.}\htmlnav

\section{HTML}

The \styni{flowfram} package now comes with a \latextohtml\ style
file \texttt{flowfram.perl}. However this style file is not meant
to emulate the \styni{flowfram} package, but is provided to facilitate
creating a plain HTML document from the \LaTeX\ source. All 
\gls{frame}-related information is ignored. By default, the contents
of any \staticordynamic{}s are ignored, but this can be changed using
\begin{verbatim}
\HTMLset{showstaticcontents}{1}
\end{verbatim}
to show the contents of the \glspl*{static} or
\begin{verbatim}
\HTMLset{showdynamiccontents}{1}
\end{verbatim}
to show the contents of the \glspl*{dynamic} (where \verb|\HTMLset|
is defined in the \sty{html} package). Note that this places the
text at the point in the document where the contents are set.
This style file does not create HTML frames. It can therefore be
used to create an accessible version of the PDF document
\latexhtml{such as the HTML version of this document, 
\texttt{ffuserguide.html}}{such as this document}.\htmlnav

\chapter{Defining New Frames}
\chapdesc{This chapter describes how to define new frames, and how to 
identify and set frame contents. See also 
\latexhtml{\autoref{sec:layouts}}{\htmlref{Predefined 
Layouts}{sec:layouts}}.}\htmlnav

\section{Flow Frames}

The \gls{flow} is the principle type of \gls{frame}.
The text of the \env{document} environment will flow from 
one \gls*{frame} to the next in order of definition. Each 
\gls*{flow} has an associated width, height, 
position on the page and optionally a border. To define
a new \gls*{flow} use:
\begin{definition}
\cmdname{newflowframe}\oarg{\meta{page list}}%%
\marg{\meta{width}}%
\marg{\meta{height}}%
\marg{\meta{x}}%
\marg{\meta{y}}%
\oarg{\meta{label}}
\end{definition}
where \meta{width} is the width of the \gls*{frame}, \meta{height} is 
the height of the \gls*{frame}, (\meta{x},\meta{y}) is the
position of the bottom left hand corner of the \gls*{frame}
relative to the bottom left hand corner of the 
\gls{typeblock}\footnote{See \latexhtml{query~\ref{itm:absval} on 
page~\pageref{itm:absval} if you want to convert from absolute
page co-ordinates to co-ordinates relative to the 
typeblock}{\htmlref{converting from absolute to relative page 
co-ordinates}{itm:absval}}}.
The first optional argument, \meta{page list}, indicates the 
list of pages for which this \gls*{frame} is defined. 

A \gls{pglist} can either be specified by the keywords: 
\texttt{all}, \texttt{odd}, \texttt{even} or \texttt{none}, or 
by a comma-separated list of either individual page numbers or 
\glspl{pgrange}. If \meta{page list} is
omitted, \texttt{all} is assumed.  
A \gls*{pgrange} can be a closed
range (e.g.\ \verb+2-8+) or an open range (e.g.\ 
\verb+<10+ or \verb/>5/). For example: \verb'<3,5,7-11,>15'
indicates pages 1, 2, 5, 7, 8, 9, 10, 11 and all pages 
greater than page 15. These page numbers refer to the integer value of
the page counter\footnote{Why can't I use the page number format?
See query~\ref{itm:whynot}\latex{ on page~\pageref{itm:whynot}}.}\
by default, so if you have a page~i and a page~1, they
will both have the same layout (unless you change the
page list setting somewhere between the two pages).

As from version 1.4, if you use the package option 
\pkgopt[absolute]{pages} then the numbers in the page list refer to
the absolute page number. In which case page~1 refers to the first
page of the document only, regardless of whether there is another
page~1 or page~i later in the document.

Each \gls*{frame} has its own unique \gls{idn}, 
corresponding to the order in which it was defined. So the first 
\gls*{flow} to be defined has \gls*{idn}~1, 
the second has \gls*{idn}~2, and so on. This number can then
be used to identify the \gls*{frame} when you want to modify its
settings. Alternatively, you can assign a unique \gls{idl}\ to the 
\gls*{frame} using the final optional argument \meta{label}.

You can retrieve the \gls*{idl} for a given \gls*{flow} 
from its \gls*{idn} using:
\begin{definition}
\cmdname{getflowlabel}\marg{\meta{idn}}
\end{definition}
Conversely, you can retrieve the \gls*{idn} for a given \gls*{flow}
from its \gls*{idl} using:
\begin{definition}
\cmdname{getflowid}\marg{\meta{cmd}}\marg{\meta{idl}}
\end{definition}
where \meta{cmd} is a control sequence which will be used to
store the frame's \gls*{idn}.
For example:
\begin{verbatim}
The label for the first flow frame is ``\getflowlabel{1}''.
The flow frame labelled ``main'' has IDN \getflowid{\myid}{main}\myid.
\end{verbatim}
\latex{produces: The label for the first flow frame is
``\getflowlabel{1}''.
The flow frame labelled ``main'' has IDN
\getflowid{\myid}{main}\myid.} (See also 
\latexhtml{\autoref{sec:counters}}{\htmlref{Counters}{sec:counters}}.)

Note that \cmdname{getflowlabel} doesn't perform any check to
determine whether the supplied \gls*{idn} is valid, but
\cmdname{getflowid} will generate an error if the supplied
\gls*{idl} is undefined.

By default, the \gls*{flow} will not have a border, but the 
starred form 
\begin{definition}
\cmdname{newflowframe*}\oarg{\meta{page list}}%%
\marg{\meta{width}}%
\marg{\meta{height}}%
\marg{\meta{x}}%
\marg{\meta{y}}%
\oarg{\meta{label}}
\end{definition}
will place a plain border around the \gls*{flow}.
(See \latexhtml{\autoref{sec:modattr}}{\htmlref{Modifying frame
attributes}{sec:modattr}} if you want a different border.)

Note that if the document continues beyond the last
defined \gls*{flow} (for example, the \glspl*{flow} have only
been defined on pages~1 to~10, but the document contains 11 
pages) then a single \gls*{flow} will be defined, 
emulating one column mode for all subsequent pages.

In \latexhtml{this document, I have}{the PDF version of this document, 
I} used the command
\begin{verbatim}
\newflowframe{0.6\textwidth}{\textheight}{0pt}{0pt}[main]
\end{verbatim}
to define the main \gls*{flow}\footnote{the
position for the even pages is set using \cmdname{setflowframe} 
defined in \latexhtml{\autoref{sec:modattr}}{\htmlref{Modifying
Frame Attributes}{sec:modattr}}}\latex{ (i.e.\ this one)}.\htmlnav

\subsection{Prematurely Ending a Flow Frame}
\label{sec:framebreak}

You can force text to move immediately to the next defined
\gls{flow} using one of the commands: \cmdname{newpage},
\cmdname{pagebreak} or \cmdname{framebreak}.
The first two work in an analogous way to the way they
work in standard two column mode. The last one, 
\cmdname{framebreak}, is required
when a paragraph spans two \glspl*{flow} 
of different widths, as \TeX's output routine does not 
adjust to the new value of \cmdname{hsize} until the last 
paragraph of the previous \gls{frame} has ended. As a 
result, the end of the paragraph at the beginning of the new
\gls*{flow} retains the width of the previous \gls*{flow}.

If a paragraph does span two \glspl*{flow} of unequal width without
using \cmdname{framebreak} a warning will be issued. If a subtle
difference in frame widths is caused by rounding errors (for
example, if the frames were created using \texttt{flowframtk} or
\texttt{jpgfdraw}) you can adjust
the tolerance to suppress these warnings. The default tolerance is
2pt. To change this, set the length register \cmdname{fftolerance}
to the required tolerance. For example, to suppress warnings where
the difference in width is less than 3pt, do
\begin{verbatim}
\setlength{\fftolerance}{3pt}
\end{verbatim}

If you want to start a new page, rather than simply move to the 
next \gls*{frame}, use the command \cmdname{clearpage}, 
or for two-sided documents, to start on the next odd page
do \cmdname{cleardoublepage}.\htmlnav

\section{Static Frames}
\label{sec:static}

A \gls{static} is a rectangular area in which text neither
flows into nor flows out of.\footnote{By \dq{neither flows into nor 
flows out of} I mean you have to explicitly set the contents of 
this frame. Note that it may appear to contain text if another 
frame overlaps it, but this text belongs to the other frame.}  
The contents must be set explicitly, and once set, the contents 
of the \gls*{static} will remain the same on each page until it is 
explicitly changed.  Thus, a \gls*{static} can be used, for 
example, to make a company logo appear in the same place on every 
page.

As from version 1.03 it is now possible to have
\glspl*{static} with non-rectangular contents, see
\latexhtml{\autoref{sec:parshape}}{\htmlref{Non-Rectangular 
Frames}{sec:parshape}} for further details.

A new \gls*{static} is defined using the command:
\begin{definition}
\cmdname{newstaticframe}\oarg{\meta{page list}}%
\marg{\meta{width}}%
\marg{\meta{height}}%
\marg{\meta{x}}%
\marg{\meta{y}}%
\oarg{\meta{label}}
\end{definition}
where, as with \cmdname{newflowframe}, \meta{width} is the width of 
the \gls*{frame}, \meta{height} is the height of the \gls*{frame},
(\meta{x},\meta{y}) is the position of the bottom left hand 
corner of the \gls*{frame} relative to the bottom left hand 
corner of the \gls{typeblock}. The first optional argument,
\meta{page list}, indicates the \gls{pglist} for which this
\gls*{static} should appear, and the final optional argument,
\meta{label} is a unique textual \gls{idl}\ which you can use to
identify this \gls*{frame}.  If no label is specified, you
can refer to this \gls*{frame} by its unique \gls{idn}.
The first \gls*{static} to be defined has \gls*{idn}~1, the second
has \gls*{idn}~2, and so on.

You can retrieve the \gls*{idl} for a given \gls*{static} 
from its \gls*{idn} using:
\begin{definition}
\cmdname{getstaticlabel}\marg{\meta{idn}}
\end{definition}
Conversely, you can retrieve the \gls*{idn} for a given \gls*{static} 
from its \gls*{idl} using:
\begin{definition}
\cmdname{getstaticid}\marg{\meta{cmd}}\marg{\meta{idl}}
\end{definition}
where \meta{cmd} is a control sequence which will be used to
store the frame's \gls*{idn}.
For example:
\begin{verbatim}
The label for the first static frame is ``\getstaticlabel{1}''.
The static frame labelled ``backleft'' has IDN
\getstaticid{\myid}{backleft}\myid.
\end{verbatim}
\latex{produces: The label for the first static frame is
``\getstaticlabel{1}''.
The static frame labelled ``backleft'' has IDN
\getstaticid{\myid}{backleft}\myid.}

Note that \cmdname{getstaticlabel} doesn't perform any check to
determine whether the supplied \gls*{idn} is valid, but
\cmdname{getstaticid} will generate an error if the supplied
\gls*{idl} is undefined.

As with \cmdname{newflowframe}, there is a starred version 
\begin{definition}
\cmdname{newstaticframe*}\oarg{\meta{page list}}%
\marg{\meta{width}}%
\marg{\meta{height}}%
\marg{\meta{x}}%
\marg{\meta{y}}%
\oarg{\meta{label}}
\end{definition}
which will place a border around that \gls*{static}.

To set the contents of a particular \gls*{static}, you can
either use the \env{staticcontents} environment:
\begin{definition}
\verb/\begin{staticcontents}/\marg{\meta{IDN}}\\
\meta{contents}\\
\verb/\end{staticcontents}/
\end{definition}
where \meta{IDN} is the unique \gls*{idn}\ associated with
that \gls*{static} and \meta{contents} is the contents of the
\gls*{static}, or you can use the 
command:
\begin{definition}
\cmdname{setstaticcontents}\marg{\meta{IDN}}\marg{\meta{contents}}
\end{definition}
which will do the same thing.

There are starred versions available for both the environment
and the command to enable you to identify the \gls*{static} 
by its associated \gls*{idl}\ rather than its \gls*{idn}:
\begin{definition}
\verb/\begin{staticcontents*}/\marg{\meta{IDL}}\\
\meta{contents}\\
\verb/\end{staticcontents*}/
\end{definition}
or the equivalent:
\begin{definition}
\cmdname{setstaticcontents*}\marg{\meta{IDL}}\marg{\meta{contents}}
\end{definition}

In the body of \env{staticcontents} or \env{staticcontents*},
or in the second argument of \cmdname{setstaticcontents} 
or \cmdname{setstaticcontents*}, you can move onto another 
\gls*{static} using:
\begin{definition}
\cmdname{continueonframe}\oarg{\meta{continuation text}}\marg{\meta{id}}
\end{definition}
If \env{staticcontents*} or \cmdname{setstaticcontents*} are 
being used, \meta{id} refers to the \gls*{idl} of the next 
\gls*{static}, otherwise \meta{id} refers to the \gls*{idn} of the 
next \gls*{static}.
The optional argument specifies some continuation text to place
at the end of the first \gls*{static}. For example, suppose I have
defined two \glspl*{static} labelled \dq{frame1} and \dq{frame2}, then
\begin{verbatim}
\begin{staticcontents*}{frame1}
Some text in the first frame. (Let's
assume this frame is somewhere on the
left half of the page.)
\continueonframe[Continued on the right]{frame2}
This is some text in the second frame. (Let's
assume this frame is somewhere on the
right half of the same page.)
\end{staticcontents*}
\end{verbatim}
is equivalent to:
\begin{verbatim}
\begin{staticcontents*}{frame1}
Some text in the first frame. (Let's
assume this frame is somewhere on the
left half of the page.)
\ffcontinuedtextlayout{Continued on the right}
\end{staticcontents*}
\begin{staticcontents*}{frame2}\par\noindent
This is some text in the second frame. (Let's
assume this frame is somewhere on the
right half of the same page.)
\end{staticcontents*}
\end{verbatim}
where
\begin{definition}
\cmdname{ffcontinuedtextlayout}\marg{\meta{text}}
\end{definition}
governs how the continuation text should be displayed. The font used
to display the continuation text is given by 
\begin{definition}
\cmdname{ffcontinuedtextfont}\marg{\meta{text}}
\end{definition}

Note that this assumes that it should appear that no paragraph
break occurs in the transition between the two \glspl*{frame}. If you 
want a paragraph break you need to explicitly put one before
and after \cmdname{continueonframe}. For example:
\begin{verbatim}
\begin{staticcontents*}{frame1}
Some text in the first frame. (Let's
assume this frame is somewhere on the
left half of the page.)

\continueonframe[Continued on the right]{frame2}

This is some text in the second frame. (Let's
assume this frame is somewhere on the
right half of the same page.)
\end{staticcontents*}
\end{verbatim}
\htmlnav

\subsection{Important Notes}

\begin{itemize}
\item When you set the contents of a \gls{static}, the contents are 
immediately typeset and stored in a box until it is time to put
the contents on the page. This means that if you use any information 
that varies throughout the document (such as the page number) the
value that is current when you set the \gls*{static}['s] contents
will be the value used.

\item However, if \cmdname{label} is used inside a \gls*{static},
the label information will be written to the auxiliary file each 
time the \gls*{static} is displayed until the contents of that
frame have been changed. This means that you may
end up with multiply defined labels.
\end{itemize}
\htmlnav

\section{Dynamic Frames}
\label{sec:dynamic}

A \gls{dynamic} is similar to a \gls{static} except that its contents
are re-typeset on each page. (A \gls*{static} stores its 
contents in a savebox, whereas a \gls*{dynamic} stores its
contents in a macro.\footnote{which means that you can have
\Index{verbatim text} in the body of the \env{staticcontents} environment
but not in the body of the \env{dynamiccontents} environment\latex{
(see page~\pageref{pg:verb})}})

As from version 1.03 it is now possible to have
\glspl*{dynamic} with non-rectangular contents, see
\latexhtml{\autoref{sec:parshape}}{\htmlref{Non-Rectangular 
Frames}{sec:parshape}} for further details.

To create a new \gls*{dynamic}, use the command:
\begin{definition}
\cmdname{newdynamicframe}\oarg{\meta{page list}}%
\marg{\meta{width}}%
\marg{\meta{height}}%
\marg{\meta{x}}%
\marg{\meta{y}}%
\oarg{\meta{label}}
\end{definition}
The parameters are exactly the same as for \cmdname{newflowframe} 
and \cmdname{newstaticframe}.
Again, each \gls*{dynamic} has an associated unique \gls{idn},
starting from~1 for the first \gls*{dynamic} to be defined, and
a unique \gls{idl}\ can also be set using the final optional
argument \meta{label}.

You can retrieve the \gls*{idl} for a given \gls*{dynamic} 
from its \gls*{idn} using:
\begin{definition}
\cmdname{getdynamiclabel}\marg{\meta{idn}}
\end{definition}
Conversely, you can retrieve the \gls*{idn} for a given \gls*{dynamic} 
from its \gls*{idl} using:
\begin{definition}
\cmdname{getdynamicid}\marg{\meta{cmd}}\marg{\meta{idl}}
\end{definition}
where \meta{cmd} is a control sequence which will be used to
store the frame's \gls*{idn}.

For example:
\begin{verbatim}
The label for the first dynamic frame is ``\getdynamiclabel{1}''.
The dynamic frame labelled ``chaphead'' has IDN
\getdynamicid{\myid}{chaphead}\myid.
\end{verbatim}
\latex{produces: The label for the first dynamic frame is
``\getdynamiclabel{1}''.
The dynamic frame labelled ``chaphead'' has IDN
\getdynamicid{\myid}{chaphead}\myid.}

Note that \cmdname{getdynamiclabel} doesn't perform any check to
determine whether the supplied \gls*{idn} is valid, but
\cmdname{getdynamicid} will generate an error if the supplied
\gls*{idl} is undefined.

As with the other \gls{frame} types, there is also a starred 
version
\begin{definition}
\cmdname{newdynamicframe*}\oarg{\meta{page list}}%
\marg{\meta{width}}%
\marg{\meta{height}}%
\marg{\meta{x}}%
\marg{\meta{y}}%
\oarg{\meta{label}}
\end{definition}
which will place a plain border around the \gls*{dynamic}.
For example, in \latexhtml{this document I have}{the PDF version
of this document I } used the command
\begin{verbatim}
\newdynamicframe{0.38\textwidth}{\textheight}{0.62\textwidth}{0pt}[chaphead]
\end{verbatim}
which \latex{has }created \latexhtml{the}{a} \gls*{frame} on the 
right on odd pages and on the left on even pages. (The
position for the even pages is set using \cmdname{setdynamicframe} 
defined in \latexhtml{\autoref{sec:modattr}}{\htmlref{Modifying
Frame Attributes}{sec:modattr}}.)

The contents of a \gls*{dynamic} are set using the command:
\begin{definition}
\cmdname{setdynamiccontents}\marg{\meta{id}}\marg{\meta{contents}}
\end{definition}
where \meta{id} is the unique \gls*{idn}\ associated with that
\gls*{dynamic}, and \meta{contents} is the contents of the
\gls*{dynamic}. Alternatively, if you have assigned an \gls*{idl}, 
\meta{label}, to the \gls*{dynamic}, you can use the starred 
version:
\begin{definition}
\cmdname{setdynamiccontents*}\marg{\meta{label}}\marg{\meta{contents}}
\end{definition}
As with most \LaTeX\ commands, you can't use \Index{verbatim text} in
\meta{contents}.

As from version 1.09, the contents can also be set using the 
\env{dynamiccontents} environment:
\begin{definition}
\verb|\begin{dynamiccontents}|\marg{\meta{id}}\\
\meta{contents}\\
\verb|\end{dynamiccontents}|
\end{definition}
or the \env{dynamiccontents*} environment:
\begin{definition}
\verb|\begin{dynamiccontents*}|\marg{label}\\
\meta{contents}\\
\verb|\end{dynamiccontents*}|
\end{definition}
\label{pg:verb}Note that you can't use \Index{verbatim text} within the 
\env{dynamiccontents} or \env{dynamiccontents*} environments.

You can additionally append text to a \gls*{dynamic} using
either:
\begin{definition}
\cmdname{appenddynamiccontents}\marg{\meta{id}}\marg{\meta{contents}}
\end{definition}
or:
\begin{definition}
\cmdname{appenddynamiccontents*}\marg{\meta{label}}\marg{\meta{contents}}
\end{definition}

\subsection{Putting Chapter Titles in a Dynamic Frame}
\label{sec:dfchaphead}

If \cmdname{chapter} is defined, you can make the chapter titles 
appear in a dynamic frame using the command
\begin{definition}
\cmdname{dfchaphead}\marg{\meta{IDN}}
\end{definition}
where \meta{IDN} is the \gls{idn}\ of the dynamic frame. There is
also a starred version of this command if you want to
use the \gls{idl}\ instead of the \gls*{idn}. For example, in 
\html{the PDF version of }this document, I used the command:
\begin{verbatim}
\dfchaphead*{chaphead}
\end{verbatim}

If you use \cmdname{dfchaphead}, you can adjust the format of the
chapter headings by redefining
\begin{definition}
\cmdname{DFchapterstyle}\marg{\meta{title}}
\end{definition}
for the numbered chapters and 
\begin{definition}
\cmdname{DFschapterstyle}\marg{\meta{title}}
\end{definition}
for the unnumbered chapters. For example, this document redefined
those commands as follows:
\begin{verbatim}
\renewcommand{\DFchapterstyle}[1]{%
 {\raggedright\sffamily\bfseries\Huge
  \color{blue}\thechapter. #1\par
 }%
}

\renewcommand{\DFschapterstyle}[1]{%
 {\raggedright\sffamily\bfseries\Huge
  \color{blue} #1\par
 }%
}
\end{verbatim}
There is no facility for placing other sectional types in a
\gls{dynamic}.
\htmlnav

\subsection{Putting Headers and Footers in a Dynamic Frame}

The headers and footers can be turned into \glspl*{dynamic} 
using the command
\begin{definition}
\cmdname{makedfheaderfooter}
\end{definition}
This will create two \glspl*{dynamic} with \glspl*{idl} 
\texttt{header} and \texttt{footer}.  The page style will be used as 
usual, but you can then move or resize the header and footer using
\cmdname{setdynamicframe} (described \latexhtml{in 
\autoref{sec:modattr}}{\htmlref{later}{sec:modattr}}).
\htmlnav

\subsection{Continued Text}

In the body of \env{dynamiccontents} or \env{dynamiccontents*}, 
you can move onto another \gls*{dynamic} using:
\begin{definition}
\cmdname{continueonframe}\oarg{\meta{continuation text}}\marg{id}
\end{definition}
If this command occurs within \env{dynamiccontents*}, \meta{id} 
refers to the \gls*{idl} of the new frame, otherwise it refers
to the \gls*{idn} of the new frame.
The optional argument specifies some continuation text to place
at the end of the first \gls*{dynamic}. For example, suppose I have
defined two \glspl*{dynamic} labelled \dq{frame1} and \dq{frame2}, then
\begin{verbatim}
\begin{dynamiccontents*}{frame1}
Some text in the first frame. (Let's
assume this frame is somewhere on the
left half of the page.)
\continueonframe[Continued on the right]{frame2}
This is some text in the second frame. (Let's
assume this frame is somewhere on the
right half of the same page.)
\end{dynamiccontents*}
\end{verbatim}
is equivalent to:
\begin{verbatim}
\begin{dynamiccontents*}{frame1}
Some text in the first frame. (Let's
assume this frame is somewhere on the
left half of the page.)
\ffcontinuedtextlayout{Continued on the right}
\end{dynamiccontents*}
\begin{dynamiccontents*}{frame2}\par\noindent
This is some text in the second frame. (Let's
assume this frame is somewhere on the
right half of the same page.)
\end{dynamiccontents*}
\end{verbatim}
where
\begin{definition}
\cmdname{ffcontinuedtextlayout}\marg{\meta{text}}
\end{definition}
governs how the continuation text should be displayed. The font used
to display the continuation text is given by 
\begin{definition}
\cmdname{ffcontinuedtextfont}\marg{\meta{text}}
\end{definition}

Note that this assumes that it should appear that no paragraph
break occurs in the transition between the two frames. If you 
want a paragraph break you need to explicitly put one before
and after \cmdname{continueonframe}. For example:
\begin{verbatim}
\begin{dynamiccontents*}{frame1}
Some text in the first frame. (Let's
assume this frame is somewhere on the
left half of the page.)

\continueonframe[Continued on the right]{frame2}

This is some text in the second frame. (Let's
assume this frame is somewhere on the
right half of the same page.)
\end{dynamiccontents*}
\end{verbatim}
\htmlnav

\subsection{Important Notes}

\begin{itemize}
\item Verbatim text\index{verbatim text} can't be used in a 
\gls{dynamic}. This includes
the body of the \env{dynamiccontents} and \env{dynamiccontents*} 
environments.

\item \cmdname{continueonframe} can't be used in the
argument of any of the commands that set the contents of a 
\gls*{dynamic}, such as \cmdname{setdynamiccontents}.

\item \Glspl*{dynamic} are painted on the page after all the
static and flow \glspl{frame}. If the location of a \gls*{dynamic} 
overlaps the location of any static or flow frames, the contents
of the \gls*{dynamic} will obscure the contents of the overlapping
frames.
\end{itemize}
\htmlnav

\chapter{Modifying Frame Attributes}
\label{sec:modattr}
\chapdesc{This chapter describes how to modify frame attributes,
such as the size and location.}

Once you have defined the \glspl{flow}, \glspl{static} and
\glspl{dynamic}, their attributes can be changed. The three types of
\gls{frame} mostly have the same set of attributes, but some are
specific to a certain type.

\Gls*{flow} attributes are modified using 
either the command:
\begin{definition}
\cmdname{setflowframe}\marg{\meta{idn list}}\marg{\meta{key-val list}}
\end{definition}
or the starred version:
\begin{definition}
\cmdname{setflowframe*}\marg{\meta{label list}}\marg{\meta{key-val list}}
\end{definition}
or the attributes for all \glspl*{flow} can be set using:
\begin{definition}
\cmdname{setallflowframes}\marg{\meta{key-val list}}
\end{definition}
\Gls*{static} attributes are modified using either the command:
\begin{definition}
\cmdname{setstaticframe}\marg{\meta{idn list}}\marg{\meta{key-val list}}
\end{definition}
or the starred version:
\begin{definition}
\cmdname{setstaticframe*}\marg{\meta{label list}}\marg{\meta{key-val list}}
\end{definition}
or the attributes for all \glspl*{static} 
can be set using:
\begin{definition}
\cmdname{setallstaticframes}\marg{\meta{key-val list}}
\end{definition}
\Gls*{dynamic} attributes are modified using either the command:
\begin{definition}
\cmdname{setdynamicframe}\marg{\meta{idn list}}\marg{\meta{key-val list}}
\end{definition}
or the starred version:
\begin{definition}
\cmdname{setdynamicframe*}\marg{\meta{label list}}\marg{\meta{key-val list}}
\end{definition}
or the attributes for all \glspl*{dynamic} can be set using:
\begin{definition}
\cmdname{setalldynamicframes}\marg{\meta{key-val list}}
\end{definition}

In each of the above, \meta{idn list} can either be one of the
keywords: \texttt{all}, \texttt{odd} or \texttt{even} (indicating
all \glspl*{frame} of that type, \glspl*{frame} of that type whose 
\gls{idn} is odd or \glspl*{frame} of that type whose \gls*{idn} 
is even) or it can be a comma-separated list of ID numbers, or 
\gls*{idn} ranges.

For the starred versions, \meta{label list} should be
a comma-separated list of \glspl{idl}. Note that you can't use the 
above keywords or have ranges in \meta{label list}.

The \meta{key-val list} argument must be a comma-separated
list of \meta{key}=\meta{value} pairs, indicating which
attributes to modify. \textbf{Make sure you group \meta{value}
if it contains one or more commas or equal signs.} 
The available values are as follows:

\begin{description}
\item[\key{width}=\meta{length}]  The width of the \gls*{frame}.

\item[\key{height}=\meta{length}] The height of the \gls*{frame}.

\item[\key{x}=\meta{length}] The x-coordinate of the \gls*{frame} 
for all pages on which it is defined.

\item[\key{y}=\meta{length}] The y-coordinate of the \gls*{frame} 
for all pages on which it is defined.

\item[\key{evenx}=\meta{length}] The x-coordinate of the 
\gls*{frame} for all even pages on which it is defined, but only if
the document is a two-sided document.

For example, in \latexhtml{this document, I have}{the PDF version of
this document, I} used the commands
\begin{verbatim}
\setflowframe*{main}{evenx=0.4\textwidth}
\setdynamicframe*{chaphead}{evenx=0pt}
\end{verbatim}
to switch the positions of the \gls*{flow} and \gls*{dynamic} 
containing the document text and chapter headings, respectively,
on even pages.

You can swap the odd and even values using the commands:
\begin{definition}
\cmdname{ffswapoddeven}\marg{\meta{IDN}}
\end{definition}
(for \glspl*{flow})
\begin{definition}
\cmdname{sfswapoddeven}\marg{\meta{IDN}}
\end{definition}
(for \glspl*{static}) or 
\begin{definition}
\cmdname{dfswapoddeven}\marg{\meta{IDN}}
\end{definition}
(for \glspl*{dynamic}). These
commands all have starred versions which take the frame's
\gls*{idl}\ instead of its \gls*{idn}.

\item[\key{eveny}=\meta{length}] The y-coordinate of the 
\gls*{frame} for all even pages on which it is defined, but only if
the document is a two-sided document.

\item[\key{oddx}=\meta{length}] The x-coordinate of the 
\gls*{frame} for all odd pages on which it is defined, if the
document is two-sided.

\item[\key{oddy}=\meta{length}] The y-coordinate of the 
\gls*{frame} for all odd pages on which it is defined,
if the document is two-sided.

\item[\key{valign}=\meta{pos}] Change the vertical alignment
of material inside a static or dynamic frame. The value \meta{pos} 
may be one of: \texttt{c}, \texttt{t} or \texttt{b}.
The default for \glspl*{static} is \texttt{c}, the default
for \glspl*{dynamic} is \texttt{t}.
This key is not available for \glspl*{flow}.

\item[\key{label}=\meta{text}] Assign an \gls*{idl} to the \gls*{frame}.
(If you do not specify a label when you first define a \gls*{frame} 
it will be given a label identical to its \gls*{idn}.) This key is
provided to allow the user to label frames that have been 
generated by certain predefined layout commands described
in \latexhtml{\autoref{sec:layouts}}{\htmlref{Predefined 
Layouts}{sec:layouts}}.

\item[\key{border}=\meta{style}] The style of the border around the 
\gls*{frame}, this can take the values: \texttt{none} (no border),
\texttt{plain} (plain border) or the name of a \LaTeX\ 
\gls{fcmd} without the preceding backslash. (I admit
the notation is a little confusing, a \gls*{fcmd} 
is a command that places some kind of border around its 
argument, such as \cmdname{fbox}, or if you are using the
\sty{fancybox} package: \cmdname{doublebox}, \cmdname{ovalbox},
\cmdname{Ovalbox} and \cmdname{shadowbox}.) 
The value \texttt{fbox} is equivalent to \texttt{plain}.

For example, to make the first \gls*{static} have an oval border:
\begin{verbatim}
\setstaticframe{1}{border=ovalbox}
\end{verbatim}
Or you can define your own border:
\begin{verbatim}
\newcommand{\greenyellowbox}[1]{\fcolorbox{green}{yellow}{#1}}
\setstaticframe{1}{border=greenyellowbox}
\end{verbatim}

This next example uses the \sty{tikz} package to define a fancy
frame, so you need to use:
\begin{verbatim}
\usepackage{tikz}
\usetikzlibrary{snakes}
\end{verbatim}
The border command is defined as follows:
\begin{verbatim}
\newlength\fancywidth
\newlength\fancyheight
\newlength\fancydepth

\newcommand{\fancyborder}[1]{%
\settowidth{\fancywidth}{#1}%
\settoheight{\fancyheight}{#1}%
\settodepth{\fancydepth}{#1}%
\addtolength{\fancyheight}{\fancydepth}%
\hspace{-\flowframesep}%
\tikz[baseline=0pt]{%
\draw[snake=bumps,raise snake=\flowframesep,
      line width=\flowframerule]
  (0pt,0pt) rectangle (\fancywidth,\fancyheight);
}}
\end{verbatim}
This makes a bumpy border, but it uses \cmdname{flowframesep} to 
determine the gap between the border and the text and uses
\cmdname{flowframerule} to set the line width. This ensures that
the offset (see below) is correctly computed.

This new border can now be applied to a frame:
\begin{verbatim}
\setstaticframe{1}{border=fancyborder}
\end{verbatim}

\item[\key{offset}=\meta{offset}] The border offset, if it is a 
user-defined border.  This is the distance from the outer
edge of the left hand border to the left edge of the
\gls{bbox} of the text inside the border.  The \styni{flowfram} 
package is able to compute the border for the following
known \glspl*{fcmd}: \cmdname{fbox}, \cmdname{ovalbox},
\cmdname{Ovalbox}, \cmdname{doublebox} and \cmdname{shadowbox}. 
For all other borders, the offset is assumed to be
\latexhtml{$-$\cmdname{flowframesep}$-$\cmdname{flowframerule}%
}{\texttt{-\textbackslash flowframesep-\textbackslash flowframerule}}.
If you define your own \gls*{fcmd}, you may need to 
specify the offset explicitly, or the flow/static/dynamic frames 
may end up shifted to the right or left.

The above examples can compute their own offsets, however,
if you were to do, for example:
\begin{verbatim}
\newcommand{\thickgreenyellowbox}[1]{%
{\setlength{\fboxsep}{5pt}\setlength{\fboxrule}{6pt}%
\fcolorbox{green}{yellow}{#1}}}
\end{verbatim}
Then you would have to specify the offset.  In this example,
the offset is \latexhtml{$-5\mathrm{pt}-\mathrm{6pt}=-11\mathrm{pt}$%
}{-5pt-6pt=-11pt},
so you would need to do:
\begin{verbatim}
\setstaticframe{1}{border=thickgreenyellowbox,offset=-11pt}
\end{verbatim}

\item[\key{bordercolor}=\meta{colour}] The colour of the border
if you are using a standard \gls*{fcmd}.
The colour can either be specified as, e.g.\ \texttt{green},
or including the colour model, e.g. \verb/[rgb]{0,1,0}/.
For example:
\begin{verbatim}
\setallflowframes{border=doublebox,bordercolor=[rgb]{1,0,0.5}}
\end{verbatim}

\item[\key{textcolor}=\meta{colour}] The text colour for that 
\gls*{frame}. Again, the colour can either be specified as, 
e.g.\ \texttt{green}, or including the colour model, 
e.g. \verb/[rgb]{0,1,0}/.

\item[\key{backcolor}=\meta{colour}] The background colour for 
that \gls*{frame}. Again, the colour can either be specified as, 
e.g.\ \texttt{green}, or including the colour model, 
e.g. \verb/[rgb]{0,1,0}/. Note that the background colour
only extends as far as the \gls*{bbox}, not the border.
If you want it to extend as far as the border, you
will need to define your own border type (see above).

\item[\key{pages}=\meta{page list}] The \glslink{pglist}{list of 
pages} for which the \gls*{frame} should appear. This can either have 
the values: \texttt{all}, \texttt{even}, \texttt{odd} or \texttt{none} 
(the latter removes the \gls*{frame} from that point on---useful if you
have multiple pages with the same number), or it can be a 
comma-separated list of single pages, or 
\glspl{pgrange}.
For example:
\begin{verbatim}
\setdynamicframe{1}{pages={1,5,8-10}}
\end{verbatim}
Recall that the numbers in the list either refer to the integer
value of the page counter
(when used with the package option \pkgopt[relative]{pages})
or the absolute page number (when used with the package option
\pkgopt[absolute]{pages}).

As from version 1.14, there is also a quick way of setting the page
list that doesn't have the overhead of parsing the
\meta{key}=\meta{value} format of commands such as
\cmdname{setflowframe}:
\begin{definition}
\cmdname{flowsetpagelist}\marg{\meta{idn}}\marg{\meta{page
list}}\newline
\cmdname{dynamicsetpagelist}\marg{\meta{idn}}\marg{\meta{page
list}}\newline
\cmdname{staticsetpagelist}\marg{\meta{idn}}\marg{\meta{page list}}
\end{definition}
\textbf{Note that these commands don't have starred variants. The
first argument must be a single \gls{idn}.}

See also \latexhtml{\autoref{sec:switch}}{\htmlref{Switching Frames
On and Off On-The-Fly}{sec:switch}}.

\item[\key{excludepages}=\meta{list}] (New to version~1.14.) 
A comma-separated list of page
numbers where the frame should not appear. Note that this overrides
any page given by the \key{pages} key. For this key, \meta{list} may
only contain comma-separated numbers. Ranges are not permitted. For
example:
\begin{verbatim}
\setdynamicframe{1}{excludepages={7}}
\setdynamicframe{1}{pages={1-10}}
\end{verbatim}
This will make the dynamic frame appear on pages~1 to~6 and~8 to~10.

Again, there is also a quick way of setting the exclusion
list that doesn't have the overhead of parsing the
\meta{key}=\meta{value} format of commands such as
\cmdname{setflowframe}:
\begin{definition}
\cmdname{flowsetexclusion}\marg{\meta{idn}}\marg{\meta{list}}\newline
\cmdname{dynamicsetexclusion}\marg{\meta{idn}}\marg{\meta{list}}\newline
\cmdname{staticsetexclusion}\marg{\meta{idn}}\marg{\meta{list}}
\end{definition}
or you can add to an exclusion list using:
\begin{definition}
\cmdname{flowaddexclusion}\marg{\meta{idn}}\marg{\meta{list}}\newline
\cmdname{dynamicaddexclusion}\marg{\meta{idn}}\marg{\meta{list}}\newline
\cmdname{staticaddexclusion}\marg{\meta{idn}}\marg{\meta{list}}
\end{definition}
\textbf{Note that these commands don't have starred variants. The
first argument must be a single \gls{idn}.}

See also \latexhtml{\autoref{sec:switch}}{\htmlref{Switching Frames
On and Off On-The-Fly}{sec:switch}}.

\item[\key{hide}=\meta{boolean}] If this value is set, the static or
dynamic frame will be hidden regardless of the \key{pages} or
\key{excludedpages} settings.
(New to version 1.16.)

\item[\key{hidethis}=\meta{boolean}] Similar to \key{hide}, but is
always reset back to false by the output routine, so it only affects
the current page.
(New to version 1.16.)

\item[\key{margin}=\meta{side}] The side of
the \gls*{flow} that its corresponding margin should go on. This
can take the values \texttt{left}, \texttt{right}, 
\texttt{inner} or \texttt{outer}. This setting is only available
for \glspl*{flow}.

\item[\key{clear}=\meta{boolean}] 
If this value is set, the static or dynamic frame will be 
cleared at the start of the
next page, otherwise it will only be cleared on the next
occurrence of \cmdname{setstaticcontents} or the 
\env{staticcontents} environment, or the 
\cmdname{setdynamiccontents}, depending on the frame type.
This value is not set by default. This setting is not 
available for \glspl*{flow}.

For example, \latexhtml{to prevent the chapter heading reappearing on
every page, I have used}{in the PDF version of this document, I
prevented the chapter heading reappearing on every page using} 
the command:
\begin{verbatim}
\setdynamicframe*{chaphead}{clear}
\end{verbatim}

If you want to put \cmdname{label} in a static or dynamic frame, you
should use the \key{clear} key to prevent the label from being 
multiply defined.

\item[\key{style}=\meta{cmd}] This should be
the name of a command \emph{without} the preceding backslash, 
to be applied to the contents of the specified \gls*{dynamic}. 
The command may either be a declaration, for example:
\begin{verbatim}
\setalldynamicframes{style=large}
\end{verbatim}
which will set the contents of all the \glspl*{dynamic} in a
large font, or it can be a command that takes a single argument,
for example:
\begin{verbatim}
\setalldynamicframes{style=textbf}
\end{verbatim}
which will make the text for all the \glspl*{dynamic} come out in 
bold.  To unset a style, do \verb/style=none/.
This setting is only available for \glspl*{dynamic}.

\item[\key{angle}=\meta{n}]  Rotate the contents of the
\gls*{frame} by \meta{n} degrees (new to 
version 1.02). Note that the \glspl{bbox} will not 
appear rotated.

\item[\key{shape}=\meta{shape command}] Define a shape for
the contents of a \gls*{static} or \gls*{dynamic} (new
to version 1.03). If \meta{shape command} is \cmdname{relax}, no
paragraph shape will be applied. See 
\latexhtml{\autoref{sec:parshape}}{\htmlref{Non-Rectangular 
Frames}{sec:parshape}} for further details.

\end{description}
\htmlnav

\section{Non-Rectangular Frames}
\label{sec:parshape}

As from version 1.03, it is now possible to specify 
non-rectangular \staticordynamic{}s (but not \glspl{flow}).  Note 
that the
\gls{bbox} will still appear as a rectangle despite the
\gls{frame}['s] shape setting. You may use either \TeX's 
\cmdname{parshape} command, or the
\cmdname{shapepar}\slash\cmdname{Shapepar}
commands defined in Donald~Arseneau's \sty{shapepar} 
package (if using \cmdname{shapepar} or \cmdname{Shapepar}, remember to include
the \sty{shapepar} package.)

The \cmdname{shapepar} or \cmdname{Shapepar} commands provide
greater flexibility in the type of shape that can be used. However,
be aware of the advice given in the \sty{shapepar} documentation.
\begin{description}
\item[\cmdname{parshape}]
With \cmdname{parshape} you can not have cut-outs in the middle, 
top or bottom of a frame, however it is possible to have cut-outs 
in the left or right side of the \gls*{frame}. When used with the 
\key{shape} key for static or dynamic frames, the effects of 
\cmdname{par} and the sectioning commands are modified to allow 
the paragraph shape to extend beyond a single paragraph, and
to allow sectioning commands (but not \cmdname{chapter} 
or \cmdname{part}).

\item[\cmdname{shapepar}/\cmdname{Shapepar}] With \cmdname{shapepar}
or \cmdname{Shapepar} you may
have cut-outs, but you may not have any sectioning commands, 
paragraph breaks, vertical spacing or mathematics. You can
simulate a paragraph break using \cmdname{simpar}, but this
is not recommended. The size of the shape depends on the
amount of text, so the shape will expand or contract as you
add or delete text. In general, \cmdname{Shapepar} is better suited
for use as a frame shape than \cmdname{shapepar}. See the 
\sty{shapepar} documentation for more details of these commands.
\end{description}

To restore a \gls*{frame} to its default rectangular setting use 
\key{shape}=\cmdname{relax}.

For those unfamiliar with \TeX's \cmdname{parshape} command,
the syntax is as follows:\\[10pt]
\cmdname{parshape}=\latexhtml{$n$ $i_1$ $l_1$ $i_2$ $l_2$ \ldots\ $i_n$ $l_n$}{\meta{n} \meta{i\textsubscript{1}} \meta{l\textsubscript{1}} 
\meta{i\textsubscript{2}} \meta{l\textsubscript{2}} \ldots\ 
\meta{i\textsubscript{n}} \meta{l\textsubscript{n}}}\\[10pt]
where \latexhtml{$n$}{\meta{n}} is the number of (\latexhtml{$i_j$ 
$l_j$}{\meta{i\textsubscript{j}} \meta{l\textsubscript{j}}}) pairs and
\latexhtml{$i_j$}{\meta{i\textsubscript{j}}} specifies the left 
indentation for the \latexhtml{$j$th}{jth} line and 
\latexhtml{$l_j$}{\meta{l\textsubscript{j}}} specifies the length of 
the \latexhtml{$j$th}{jth} line.

\latexhtml{The \gls*{static} on the top 
\ifthenelse{\isodd{page}}{right}{left} 
was assigned a zigzag shape using:}{For example, to create a 
zigzag shaped static frame (whose \gls*{idn} is \texttt{shapedt}):}
\setstaticframe*{shapedb,shapedt}{pages=all}
\begin{verbatim}
\setstaticframe*{shapedt}{shape={\parshape=20 
0.6\linewidth 0.4\linewidth 0.5\linewidth 0.4\linewidth
0.4\linewidth 0.4\linewidth 0.3\linewidth 0.4\linewidth
0.2\linewidth 0.4\linewidth 0.1\linewidth 0.4\linewidth
0pt 0.4\linewidth 0.1\linewidth 0.4\linewidth
0.2\linewidth 0.4\linewidth 0.3\linewidth 0.4\linewidth
0.4\linewidth 0.4\linewidth 0.5\linewidth 0.4\linewidth
0.6\linewidth 0.4\linewidth 0.5\linewidth 0.4\linewidth
0.4\linewidth 0.4\linewidth 0.3\linewidth 0.4\linewidth
0.2\linewidth 0.4\linewidth 0.1\linewidth 0.4\linewidth
0pt 0.4\linewidth 0.1\linewidth 0.4\linewidth
}}
\end{verbatim}
\setstaticframe*{shapedt}{shape={\parshape=20 
0.6\linewidth 0.4\linewidth
0.5\linewidth 0.4\linewidth
0.4\linewidth 0.4\linewidth
0.3\linewidth 0.4\linewidth
0.2\linewidth 0.4\linewidth
0.1\linewidth 0.4\linewidth
0pt 0.4\linewidth
0.1\linewidth 0.4\linewidth
0.2\linewidth 0.4\linewidth
0.3\linewidth 0.4\linewidth
0.4\linewidth 0.4\linewidth
0.5\linewidth 0.4\linewidth
0.6\linewidth 0.4\linewidth
0.5\linewidth 0.4\linewidth
0.4\linewidth 0.4\linewidth
0.3\linewidth 0.4\linewidth
0.2\linewidth 0.4\linewidth
0.1\linewidth 0.4\linewidth
0pt 0.4\linewidth
0.1\linewidth 0.4\linewidth
}}

\begin{staticcontents*}{shapedt}
This is an example of a static frame with a non-rectangular
shape. This zigzag shape was specified using the \key{shape} 
key setting in \cmdname{setstaticframe}. The \cmdname{parshape} 
command was used to set the shape.

Using the \key{shape} key rather than explicitly using
\cmdname{parshape} within the \env{staticcontents} environment
means that I can have paragraph breaks, sectioning commands,
and even some mathematics
\begin{equation}
E=mc^2
\end{equation}
whilst retaining the shape.
\end{staticcontents*}

The syntax for \cmdname{shapepar} and \cmdname{Shapepar} is more complicated, see
the \sty{shapepar} documentation for more details. In general:\\[10pt]
\cmdname{shapepar}\{\meta{shape specs}\}\\[10pt]
The \sty{shapepar} package has four predefined shapes:
\cmdname{squareshape}, \cmdname{diamondshape}, 
\cmdname{heartshape} and \cmdname{nutshape}. 

\latexhtml{The \gls{static} on the bottom 
\ifthenelse{\isodd{page}}{right} {left}
was assigned a heart shape using the command:}{For example, to assign
a heart shape to the static frame whose \gls*{idl} is \texttt{shapedb}:}
\begin{verbatim}
\setstaticframe*{shapedb}{shape={\shapepar\heartshape}}
\end{verbatim}
To reset the frame back to its original rectangular shape
do:
\begin{verbatim}
\setstaticframe*{shapedb}{shape=\relax}
\end{verbatim}

\afterpage{\setstaticframe*{shapedt,shapedb}{pages=none}}
\setstaticframe*{shapedb}{shape={\shapepar\heartshape}}
\begin{staticcontents*}{shapedb}
This example has a more complicated shape that can not 
be generated using \TeX's \cmdname{parshape} command, so
\cmdname{shapepar} was used instead. Note that this document
must include the \sty{shapepar} package in this instance,
whereas no extra packages are required to use \cmdname{parshape}.
No mathematics or sectioning commands are allowed here.
The shape will expand as more text is added to it.
\end{staticcontents*}

The \styni{flowfram} package currently does not support any other
paragraph shape making commands. Any other commands would
have to be used explicitly within the contents of the frame.\htmlnav

\section{Switching Frames On and Off On-The-Fly}
\label{sec:switch}

Modifying the \gls{pglist} (or the page exclusion list) within the
\env{document} environment is a risky business. This list must
be up-to-date before the output routine looks for the next frame.
To make this a little easier, as from version~1.14 there are
commands that help you do this. \textbf{If you want to use these
commands, it's best to use the package option \pkgopt[absolute]{pages}.}

The commands described in this section update the \glspl{pglist}
(and possibly the exclusion list) \emph{when the output routine is
next used}. They are designed to switch frames on or off either on
the next page or on the next odd page. You therefore need to take
care where you place these commands. For example, if you have a
two-sided document and you do:
\begin{verbatim}
\dynamicswitchonnextodd{1}
\mainmatter
\chapter{Introduction}
\end{verbatim}
This will set the dynamic frame whose \gls{idn} is~1 to be visible for
the first page of chapter~1. However, if you do
\begin{verbatim}
\mainmatter
\dynamicswitchonnextodd{1}
\chapter{Introduction}
\end{verbatim}
This will have a different effect as \cmdname{mainmatter} issues a
\cmdname{cleardoublepage} so the command to switch on the dynamic
frame is on the same page as the start of chapter~1. This means that
the dynamic frame won't appear until the following odd page
(page~3).

These commands all have the same syntax with one argument that may
be a comma-separated list. The starred version uses
\glspl{idl} and the unstarred version uses \glspl{idn}.

\begin{definition}
\cmdname{flowswitchonnext}\marg{\meta{IDN list}}\\
\cmdname{flowswitchonnext*}\marg{\meta{IDL list}}
\end{definition}%
Switch on the listed \glspl{flow} from the following page onwards.
\begin{definition}
\cmdname{flowswitchoffnext}\marg{\meta{IDN list}}\\
\cmdname{flowswitchoffnext*}\marg{\meta{IDL list}}
\end{definition}%
Switch off the listed \glspl{flow} from the following page onwards.
\begin{definition}
\cmdname{flowswitchonnextodd}\marg{\meta{IDN list}}\\
\cmdname{flowswitchonnextodd*}\marg{\meta{IDL list}}
\end{definition}%
Switch on the listed \glspl{flow} from the next odd page onwards.
\begin{definition}
\cmdname{flowswitchoffnextodd}\marg{\meta{IDN list}}\\
\cmdname{flowswitchoffnextodd*}\marg{\meta{IDL list}}
\end{definition}%
Switch off the listed \glspl{flow} from the next odd page onwards.

\begin{definition}
\cmdname{flowswitchonnextonly}\marg{\meta{IDN list}}\\
\cmdname{flowswitchonnextonly*}\marg{\meta{IDL list}}
\end{definition}%
Switch on the listed \glspl{flow} just for the following page.
\begin{definition}
\cmdname{flowswitchoffnextonly}\marg{\meta{IDN list}}\\
\cmdname{flowswitchoffnextonly*}\marg{\meta{IDL list}}
\end{definition}%
Switch off the listed \glspl{flow} just for the following page.
\begin{definition}
\cmdname{flowswitchonnextoddonly}\marg{\meta{IDN list}}\\
\cmdname{flowswitchonnextoddonly*}\marg{\meta{IDL list}}
\end{definition}%
Switch on the listed \glspl{flow} just for the next odd page.
\begin{definition}
\cmdname{flowswitchoffnextoddonly}\marg{\meta{IDN list}}\\
\cmdname{flowswitchoffnextoddonly*}\marg{\meta{IDL list}}
\end{definition}%
Switch off the listed \glspl{flow} just for the next odd page.

\begin{definition}
\cmdname{dynamicswitchonnext}\marg{\meta{IDN list}}\\
\cmdname{dynamicswitchonnext*}\marg{\meta{IDL list}}
\end{definition}%
Switch on the listed \glspl{dynamic} from the following page onwards.
\begin{definition}
\cmdname{dynamicswitchoffnext}\marg{\meta{IDN list}}\\
\cmdname{dynamicswitchoffnext*}\marg{\meta{IDL list}}
\end{definition}%
Switch off the listed \glspl{dynamic} from the following page onwards.
\begin{definition}
\cmdname{dynamicswitchonnextodd}\marg{\meta{IDN list}}\\
\cmdname{dynamicswitchonnextodd*}\marg{\meta{IDL list}}
\end{definition}%
Switch on the listed \glspl{dynamic} from the next odd page onwards.
\begin{definition}
\cmdname{dynamicswitchoffnextodd}\marg{\meta{IDN list}}\\
\cmdname{dynamicswitchoffnextodd*}\marg{\meta{IDL list}}
\end{definition}%
Switch off the listed \glspl{dynamic} from the next odd page onwards.

\begin{definition}
\cmdname{dynamicswitchonnextonly}\marg{\meta{IDN list}}\\
\cmdname{dynamicswitchonnextonly*}\marg{\meta{IDL list}}
\end{definition}%
Switch on the listed \glspl{dynamic} just for the following page.
\begin{definition}
\cmdname{dynamicswitchoffnextonly}\marg{\meta{IDN list}}\\
\cmdname{dynamicswitchoffnextonly*}\marg{\meta{IDL list}}
\end{definition}%
Switch off the listed \glspl{dynamic} just for the following page.
\begin{definition}
\cmdname{dynamicswitchonnextoddonly}\marg{\meta{IDN list}}\\
\cmdname{dynamicswitchonnextoddonly*}\marg{\meta{IDL list}}
\end{definition}%
Switch on the listed \glspl{dynamic} just for the next odd page.
\begin{definition}
\cmdname{dynamicswitchoffnextoddonly}\marg{\meta{IDN list}}\\
\cmdname{dynamicswitchoffnextoddonly*}\marg{\meta{IDL list}}
\end{definition}%
Switch off the listed \glspl{dynamic} just for the next odd page.

\begin{definition}
\cmdname{staticswitchonnext}\marg{\meta{IDN list}}\\
\cmdname{staticswitchonnext*}\marg{\meta{IDL list}}
\end{definition}%
Switch on the listed \glspl{static} from the following page onwards.
\begin{definition}
\cmdname{staticswitchoffnext}\marg{\meta{IDN list}}\\
\cmdname{staticswitchoffnext*}\marg{\meta{IDL list}}
\end{definition}%
Switch off the listed \glspl{static} from the following page onwards.
\begin{definition}
\cmdname{staticswitchonnextodd}\marg{\meta{IDN list}}\\
\cmdname{staticswitchonnextodd*}\marg{\meta{IDL list}}
\end{definition}%
Switch on the listed \glspl{static} from the next odd page onwards.
\begin{definition}
\cmdname{staticswitchoffnextodd}\marg{\meta{IDN list}}\\
\cmdname{staticswitchoffnextodd*}\marg{\meta{IDL list}}
\end{definition}%
Switch off the listed \glspl{static} from the next odd page onwards.

\begin{definition}
\cmdname{staticswitchonnextonly}\marg{\meta{IDN list}}\\
\cmdname{staticswitchonnextonly*}\marg{\meta{IDL list}}
\end{definition}%
Switch on the listed \glspl{static} just for the following page.
\begin{definition}
\cmdname{staticswitchoffnextonly}\marg{\meta{IDN list}}\\
\cmdname{staticswitchoffnextonly*}\marg{\meta{IDL list}}
\end{definition}%
Switch off the listed \glspl{static} just for the following page.
\begin{definition}
\cmdname{staticswitchonnextoddonly}\marg{\meta{IDN list}}\\
\cmdname{staticswitchonnextoddonly*}\marg{\meta{IDL list}}
\end{definition}%
Switch on the listed \glspl{static} just for the next odd page.
\begin{definition}
\cmdname{staticswitchoffnextoddonly}\marg{\meta{IDN list}}\\
\cmdname{staticswitchoffnextoddonly*}\marg{\meta{IDL list}}
\end{definition}%
Switch off the listed \glspl{static} just for the next odd page.

The \styni{flowfram} package comes with a sample file
\texttt{sample-pages.tex} that uses some of these commands.

\chapter{Locations and Dimensions}
\chapdesc{This chapter describes some of the commands provided to
determine the locations and dimensions of frames.}

This chapter describes some of the commands available that can
be used to determine the locations and dimensions of \glspl{frame}.
See the accompanying document \texttt{flowfram.pdf} for more details of 
these commands or for other commands not listed here.\htmlnav

\section{Determining the Location of the Typeblock}
\label{sec:typeblockloc}

As mentioned earlier, when you create new \glspl{frame},
you must specify their location relative to the \gls{typeblock}, but
what if you want to position a \gls*{frame} a set distance from the
edge of the paper? The \styni{flowfram} package provides the following
commands that compute the distance from the \gls*{typeblock} to the
paper boundary:

\begin{definition}
\cmdname{computeleftedgeodd}\marg{\meta{length}}
\end{definition}
This computes the position of the left edge of the (odd) page, relative 
to the left side of the \gls*{typeblock}, and stores the result in 
\meta{length}.

\begin{definition}
\cmdname{computeleftedgeeven}\marg{\meta{length}}
\end{definition}
As above, but for even pages.

\begin{definition}
\cmdname{computetopedge}\marg{\meta{length}}
\end{definition}
This computes the top edge of the page, relative to the bottom of the
\gls*{typeblock}, and stores the result in \meta{length}.

\begin{definition}
\cmdname{computebottomedge}\marg{\meta{length}}
\end{definition}
This computes the bottom edge of the page, relative to the bottom of the
\gls*{typeblock}, and stores the result in \meta{length}.

\begin{definition}
\cmdname{computerightedgeodd}\marg{\meta{length}}
\end{definition}
This computes the position of the right edge of the (odd) page, 
relative to the left side of the \gls*{typeblock}, and store the result 
in \meta{length}.

\begin{definition}
\cmdname{computerightedgeeven}\marg{\meta{length}}
\end{definition}
As above, but for even pages.

Note that in all cases \meta{length} must be a \LaTeX\ length
command.

For example, if you want to create a frame whose bottom
left corner is one inch from the left edge of the page
and half an inch from the bottom edge of the page (this
assumes odd and even pages have the same margins):
\begin{verbatim}
% define two new lengths to represent the x and y coords
\newlength{\myX}
\newlength{\myY}
% compute the distance from the typeblock to the paper edge
\computeleftedgeodd{\myX}
\computebottomedge{\myY}
% Add the absolute co-ordinates to get co-ordinates
% relative to the typeblock
\addtolength{\myX}{1in}
\addtolength{\myY}{0.5in}
\end{verbatim}
\htmlnav

\section{Determining the Dimensions and Locations of Frames}

It is possible to determine the dimensions and locations
of a \gls{frame} using one of the following commands:

\begin{itemize}
\item \cmdname{getstaticbounds}\marg{\meta{IDN}}
\item \cmdname{getstaticbounds*}\marg{\meta{IDL}}
\item \cmdname{getflowbounds}\marg{\meta{IDN}}
\item \cmdname{getflowbounds*}\marg{\meta{IDL}}
\item \cmdname{getdynamicbounds}\marg{\meta{IDN}}
\item \cmdname{getdynamicbounds*}\marg{\meta{IDL}}
\end{itemize}

For each command, the starred version takes an \gls{idl}\ as the
argument, and the unstarred version takes an \gls{idn}\ as the
argument.  Each command stores the relevant information in the
lengths \cmdname{ffareawidth}, \cmdname{ffareaheight},
\cmdname{ffareax} and \cmdname{ffareay}.

For other related commands, see the section \dq{Determining Dimensions
and Locations} in the accompanying document 
\texttt{flowfram.pdf}.\htmlnav

\section{Relative Locations}

To print the relative location of one \gls{frame} from another do:
\begin{definition}
\cmdname{relativeframelocation}\marg{\meta{type1}}\marg{\meta{idn1}}\marg{\meta{type2}}\marg{\meta{idn2}}
\end{definition}
where \meta{type1} and \meta{idn1} indicate the type and \gls{idn} of
the first frame, and \meta{type2} and \meta{idn2} indicate the type
and \gls{idn} of the second frame. There is also a starred version:
\begin{definition}
\cmdname{relativeframelocation*}\marg{\meta{type1}}\marg{\meta{idl1}}\marg{\meta{type2}}\marg{\meta{idll}}
\end{definition}
where \meta{idl1} and \meta{idl2} indicate the \gls{idl} of the
first and second frames, respectively. Both the above commands will
print one of the following:

\begin{itemize}
\item \cmdname{FFaboveleft} if the first frame is above left of the
second frame.

\item \cmdname{FFaboveright} if the first frame is above right of the
second frame.

\item \cmdname{FFabove} if the first frame is above the second frame.

\item \cmdname{FFbelowleft} if the first frame is below left of the
second frame.

\item \cmdname{FFbelowright} if the first frame is below right of the
second frame.

\item \cmdname{FFbelow} if the first frame is below the second frame.

\item \cmdname{FFleft} if the first frame is to the left of the 
second frame.

\item \cmdname{FFright} if the first frame is to the right of the
second frame.

\item \cmdname{FFoverlap} if both frames overlap.
\end{itemize}

A frame is considered to be above another frame if the bottom edge
of the first frame is higher than the top edge of the second frame. 

A frame is considered to be below another frame if the top edge
of the first frame is lower than the bottom edge of the second frame.

A frame is considered to be to the left of another frame if the right 
edge of the first frame is to the left of the left edge of the second
frame.

A frame is considered to be to the right of another frame if the left
edge of the first frame is to the right of the right edge of the
second frame.

Note that the relative locations are taken for the current page,
regardless of whether either of the two frames are displayed on that
page. If the current page is odd, then the frame settings for odd pages
will be compared, otherwise the frame settings for even pages will
be compared. However remember that the first paragraph of each page
retains the settings in place at the start of the paragraph, so if the
frames have different locations for odd and even pages, then
\cmdname{relativeframelocation} may use the wrong page settings if it
is used at the start of the page.

For example, \html{the PDF version of }this document defined a 
\gls{flow} labelled \texttt{main} \latex{(this one)} and a 
\gls{dynamic} labelled \texttt{chaphead} which is
used to display the chapter headings. The following code
\begin{verbatim}
The dynamic frame is 
\relativeframelocation*{dynamic}{chaphead}{flow}{main}
of the flow frame.
\end{verbatim}
produces:
The dynamic frame is 
\latexhtml{\relativeframelocation*{dynamic}{chaphead}{flow}{main}}{on
the left} of the flow frame.

There are some short cut commands for \glspl*{frame} of the
same type:
\begin{definition}
\cmdname{reldynamicloc}\marg{\meta{idn1}}\marg{\meta{idn2}}
\end{definition}
This is equivalent to:\\[10pt]
\cmdname{relativeframelocation}\marg{dynamic}\marg{\meta{idn1}}\marg{dynamic}\marg{\meta{idn2}}
\begin{definition}
\cmdname{relstaticloc}\marg{\meta{idn1}}\marg{\meta{idn2}}
\end{definition}
This is equivalent to:\\[10pt]
\cmdname{relativeframelocation}\marg{static}\marg{\meta{idn1}}\marg{static}\marg{\meta{idn2}}
\begin{definition}
\cmdname{relflowloc}\marg{\meta{idn1}}\marg{\meta{idn2}}
\end{definition}
This is equivalent to:\\[10pt]
\cmdname{relativeframelocation}\marg{flow}\marg{\meta{idn1}}\marg{flow}\marg{\meta{idn2}}
\\[10pt]
Each of the above commands also has a starred version that uses the
\gls{idl} instead of the \gls{idn}.

These commands may be used in the optional argument of 
\cmdname{continueonframe} for frames that are on the same page.
For example:
\begin{verbatim}
\begin{dynamiccontents}{1}
Some text in the first dynamic frame that goes on for 
quite a bit longer than this example.
\continueonframe[continued \reldynamicloc{2}{1}]{2}
This text is in the second dynamic frame which is 
somewhere on the same page.
\end{dynamiccontents}
\end{verbatim}

For additional commands that determine the relative location of one
frame from another, see the section \dq{Determining the relative 
location of one frame from another} in the accompanying document 
\texttt{flowfram.pdf}.\htmlnav

\chapter{Predefined Layouts}
\label{sec:layouts}
\chapdesc{This chapter describes commands that create frames 
arranged in a predefined layout.}

The \styni{flowfram} package has a number of commands which 
create \glspl{frame} in a predefined layout. These commands
may only be used in the preamble.\htmlnav

\section{Column Styles}
\label{sec:Ncolumn}

The standard \LaTeX\ commands \cmdname{onecolumn} and
\cmdname{twocolumn} are redefined to create one or two
\glspl{flow} that fill the entire \gls{typeblock} separated
from each other (in the case of \cmdname{twocolumn}) by a
gap of width \cmdname{columnsep}. The height of these
\glspl*{flow} may not be exactly as high as the \gls*{typeblock},
as their height is adjusted to make them an integer
multiple of \cmdname{baselineskip}. You can switch off this
automatic adjustment using the command:
\begin{definition}
\cmdname{ffvadjustfalse}
\end{definition}

The \cmdname{onecolumn} and \cmdname{twocolumn} commands
also take an optional argument which is the \gls{pglist} 
for which those \glspl*{flow} are defined. In addition
to \cmdname{onecolumn} and \cmdname{twocolumn}, the 
following commands are also defined:

\begin{definition}
\cmdname{Ncolumn}\oarg{\meta{pages}}\marg{\meta{n}}
\end{definition}
This creates \meta{n} column \glspl*{flow} each separated by a 
distance of \cmdname{columnsep}.

\begin{definition}
\cmdname{onecolumninarea}\oarg{\meta{pages}}%
\marg{\meta{width}}\marg{\meta{height}}\marg{\meta{x}}\marg{\meta{y}}
\end{definition}
This creates a single \gls*{flow} to fill the given area,
adjusting the height so that it is an integer multiple of
\cmdname{baselineskip}.

\begin{definition}
\cmdname{twocolumninarea}\oarg{\meta{pages}}%
\marg{\meta{width}}\marg{\meta{height}}\marg{\meta{x}}\marg{\meta{y}}
\end{definition}
This creates two column \glspl*{flow} separated by a distance
of \cmdname{columnsep} filling the entire area specified, again
adjusting the height so that it is an integer multiple of
\cmdname{baselineskip}. The columns are separated by a gap of
\cmdname{columnsep}.

\begin{definition}
\cmdname{Ncolumninarea}\oarg{\meta{pages}}\marg{\meta{n}}%
\marg{\meta{width}}\marg{\meta{height}}\marg{\meta{x}}\marg{\meta{y}}
\end{definition}
This is a more general form of \cmdname{twocolumninarea} making 
\meta{n} \glspl*{flow} instead of two.\htmlnav

\section{Column Styles with an Additional Frame}

As well as the column-style \glspl{flow} defined 
above, it is also possible to define
\meta{n} columns with an additional \gls{frame} spanning either 
above or below them. There will be a vertical gap of 
approximately\footnote{It may not be exact, as the \glspl*{flow} 
are adjusted so that their height is an integer multiple
of \cmdname{baselineskip}, which may increase the gap.} 
\cmdname{vcolumnsep} between the columns
and the extra frame. In each of the following definitions,
the argument \meta{pages} is the \gls{pglist} for which
the \glspl*{frame} are defined, \meta{n} is the number of
columns required, \meta{type} is the type of frame to go
above or below the columns (this may be one of: \texttt{flow},
\texttt{static} or \texttt{dynamic}). The area in which the
new frames should fill is defined by \meta{width}, \meta{height} 
(the width and height of the area) and \meta{x}, \meta{y} 
(the position of the bottom left hand corner of the area
relative to the bottom left hand corner of the \gls{typeblock}.)

The height of the additional frame at the top or bottom of
the columns is given by \meta{H}.

\begin{definition}
\cmdname{onecolumntopinarea}\oarg{\meta{pages}}%
\marg{\meta{type}}\marg{\meta{H}}\marg{\meta{width}}%
\marg{\meta{height}}\marg{\meta{x}}\marg{\meta{y}}
\end{definition}
This creates one \gls*{flow} with a \meta{type} frame above it, filling
the area specified.

\begin{definition}
\cmdname{twocolumntopinarea}\oarg{\meta{pages}}%
\marg{\meta{type}}\marg{\meta{H}}\marg{\meta{width}}%
\marg{\meta{height}}\marg{\meta{x}}\marg{\meta{y}}
\end{definition}
This creates two column-style \glspl*{flow} with a \meta{type} frame 
above them, filling the area specified.

\begin{definition}
\cmdname{Ncolumntopinarea}\oarg{\meta{pages}}\marg{\meta{type}}%
\marg{\meta{n}}\marg{\meta{H}}\marg{\meta{width}}\marg{\meta{height}}%
\marg{\meta{x}}\marg{\meta{y}}
\end{definition}
This creates \meta{n} column-style \glspl*{flow} with a \meta{type}  
frame above them, filling the area specified.

\begin{definition}
\cmdname{onecolumnbottominarea}\oarg{\meta{pages}}%
\marg{\meta{type}}\marg{\meta{H}}\marg{\meta{width}}%
\marg{\meta{height}}\marg{\meta{x}}\marg{\meta{y}}
\end{definition}
This creates one \gls*{flow} with a \meta{type} frame underneath it, 
filling the area specified.

\begin{definition}
\cmdname{twocolumnbottominarea}\oarg{\meta{pages}}%
\marg{\meta{type}}\marg{\meta{H}}\marg{\meta{width}}%
\marg{\meta{height}}\marg{\meta{x}}\marg{\meta{y}}
\end{definition}
This creates two column-style \glspl*{flow} with a \meta{type} frame 
below them, filling the area specified.

\begin{definition}
\cmdname{Ncolumnbottominarea}\oarg{\meta{pages}}%
\marg{\meta{type}}\marg{\meta{n}}\marg{\meta{H}}%
\marg{\meta{width}}\marg{\meta{height}}\marg{\meta{x}}%
\marg{\meta{y}}
\end{definition}
This creates \meta{n} column-style \glspl*{flow} with a \meta{type} 
frame below them, filling the area specified.

The following commands are special cases of the above:

\begin{definition}
\cmdname{onecolumnStopinarea}\oarg{\meta{pages}}%
\marg{\meta{H}}\marg{\meta{width}}%
\marg{\meta{height}}\marg{\meta{x}}\marg{\meta{y}}
\end{definition}
This is equivalent to:\newline
\cmdname{onecolumntopinarea}\oarg{\meta{pages}}%
\marg{static}\marg{\meta{H}}\marg{\meta{width}}%
\marg{\meta{height}}\marg{\meta{x}}\marg{\meta{y}}

\begin{definition}
\cmdname{onecolumnDtopinarea}\oarg{\meta{pages}}%
\marg{\meta{H}}\marg{\meta{width}}%
\marg{\meta{height}}\marg{\meta{x}}\marg{\meta{y}}
\end{definition}
This is equivalent to:\newline
\cmdname{onecolumntopinarea}\oarg{\meta{pages}}%
\marg{dynamic}\marg{\meta{H}}\marg{\meta{width}}%
\marg{\meta{height}}\marg{\meta{x}}\marg{\meta{y}}

\begin{definition}
\cmdname{onecolumntop}\oarg{\meta{pages}}\marg{\meta{type}}%
\marg{\meta{H}}
\end{definition}
As \cmdname{onecolumntopinarea} where the area is the entire
\gls*{typeblock}.

\begin{definition}
\cmdname{onecolumnStop}\oarg{\meta{pages}}\marg{\meta{H}}
\end{definition}
This is equivalent to: 
\cmdname{onecolumntop}\oarg{\meta{pages}}\marg{static}\marg{\meta{H}}

\begin{definition}
\cmdname{onecolumnDtop}\oarg{\meta{pages}}\marg{\meta{H}}
\end{definition}
This is equivalent to: 
\cmdname{onecolumntop}\oarg{\meta{pages}}\marg{dynamic}\marg{\meta{H}}

\begin{definition}
\cmdname{twocolumnStopinarea}\oarg{\meta{pages}}%
\marg{\meta{H}}\marg{\meta{width}}%
\marg{\meta{height}}\marg{\meta{x}}\marg{\meta{y}}
\end{definition}
This is equivalent to:\newline
\cmdname{twocolumntopinarea}\oarg{\meta{pages}}%
\marg{static}\marg{\meta{H}}\marg{\meta{width}}%
\marg{\meta{height}}\marg{\meta{x}}\marg{\meta{y}}

\begin{definition}
\cmdname{twocolumnDtopinarea}\oarg{\meta{pages}}%
\marg{\meta{H}}\marg{\meta{width}}%
\marg{\meta{height}}\marg{\meta{x}}\marg{\meta{y}}
\end{definition}
This is equivalent to:\newline
\cmdname{twocolumntopinarea}\oarg{\meta{pages}}%
\marg{dynamic}\marg{\meta{H}}\marg{\meta{width}}%
\marg{\meta{height}}\marg{\meta{x}}\marg{\meta{y}}

\begin{definition}
\cmdname{twocolumntop}\oarg{\meta{pages}}%
\marg{\meta{type}}\marg{\meta{H}}
\end{definition}
As \cmdname{twocolumntopinarea} where the area is the entire
\gls*{typeblock}.

\begin{definition}
\cmdname{twocolumnStop}\oarg{\meta{pages}}\marg{\meta{H}}
\end{definition}
This is equivalent to:
\cmdname{twocolumntop}\oarg{\meta{pages}}\marg{static}\marg{\meta{H}}

\begin{definition}
\cmdname{twocolumnDtop}\oarg{\meta{pages}}\marg{\meta{H}}
\end{definition}
This is equivalent to:
\cmdname{twocolumntop}\oarg{\meta{pages}}\marg{dynamic}\marg{\meta{H}}

\begin{definition}
\cmdname{NcolumnStopinarea}\oarg{\meta{pages}}%
\marg{\meta{n}}\marg{\meta{H}}\marg{\meta{width}}\marg{\meta{height}}%
\marg{\meta{x}}\marg{\meta{y}}
\end{definition}
This is equivalent to:\newline
\cmdname{Ncolumntopinarea}\oarg{\meta{pages}}\marg{static}%
\marg{\meta{n}}\marg{\meta{H}}\marg{\meta{width}}\marg{\meta{height}}%
\marg{\meta{x}}\marg{\meta{y}}

\begin{definition}
\cmdname{NcolumnDtopinarea}\oarg{\meta{pages}}%
\marg{\meta{n}}\marg{\meta{H}}\marg{\meta{width}}\marg{\meta{height}}%
\marg{\meta{x}}\marg{\meta{y}}
\end{definition}
This is equivalent to:\newline
\cmdname{Ncolumntopinarea}\oarg{\meta{pages}}\marg{dynamic}%
\marg{\meta{n}}\marg{\meta{H}}\marg{\meta{width}}\marg{\meta{height}}%
\marg{\meta{x}}\marg{\meta{y}}

\begin{definition}
\cmdname{Ncolumntop}\oarg{\meta{pages}}%
\marg{\meta{type}}\marg{\meta{n}}\marg{\meta{H}}
\end{definition}
As \cmdname{Ncolumntopinarea} where the area is the entire
\gls*{typeblock}.

\begin{definition}
\cmdname{NcolumnStop}\oarg{\meta{pages}}\marg{\meta{n}}\marg{\meta{H}}
\end{definition}
This is equivalent to:
\cmdname{Ncolumntop}\oarg{\meta{pages}}\marg{static}\marg{\meta{n}}\marg{\meta{H}}

\begin{definition}
\cmdname{NcolumnDtop}\oarg{\meta{pages}}\marg{\meta{n}}\marg{\meta{H}}
\end{definition}
This is equivalent to:
\cmdname{Ncolumntop}\oarg{\meta{pages}}\marg{dynamic}\marg{\meta{n}}\marg{\meta{H}}

\begin{definition}
\cmdname{onecolumnSbottominarea}\oarg{\meta{pages}}%
\marg{\meta{H}}\marg{\meta{width}}%
\marg{\meta{height}}\marg{\meta{x}}\marg{\meta{y}}
\end{definition}
This is equivalent to:\newline
\cmdname{onecolumnbottominarea}\oarg{\meta{pages}}%
\marg{static}\marg{\meta{H}}\marg{\meta{width}}%
\marg{\meta{height}}\marg{\meta{x}}\marg{\meta{y}}

\begin{definition}
\cmdname{onecolumnDbottominarea}\oarg{\meta{pages}}%
\marg{\meta{H}}\marg{\meta{width}}%
\marg{\meta{height}}\marg{\meta{x}}\marg{\meta{y}}
\end{definition}
This is equivalent to:\newline
\cmdname{onecolumnbottominarea}\oarg{\meta{pages}}%
\marg{dynamic}\marg{\meta{H}}\marg{\meta{width}}%
\marg{\meta{height}}\marg{\meta{x}}\marg{\meta{y}}

\begin{definition}
\cmdname{onecolumnbottom}\oarg{\meta{pages}}%
\marg{\meta{type}}\marg{\meta{H}}
\end{definition}
As \cmdname{onecolumnbottominarea} where the area is the entire
\gls*{typeblock}.

\begin{definition}
\cmdname{onecolumnSbottom}\oarg{\meta{pages}}\marg{\meta{H}}
\end{definition}
This is equivalent to:
\cmdname{onecolumnbottom}\oarg{\meta{pages}}\marg{static}\marg{\meta{H}}

\begin{definition}
\cmdname{onecolumnDbottom}\oarg{\meta{pages}}\marg{\meta{H}}
\end{definition}
This is equivalent to:
\cmdname{onecolumnbottom}\oarg{\meta{pages}}\marg{dynamic}\marg{\meta{H}}

\begin{definition}
\cmdname{twocolumnSbottominarea}\oarg{\meta{pages}}%
\marg{\meta{H}}\marg{\meta{width}}%
\marg{\meta{height}}\marg{\meta{x}}\marg{\meta{y}}
\end{definition}
This is equivalent to:\newline
\cmdname{twocolumnbottominarea}\oarg{\meta{pages}}%
\marg{static}\marg{\meta{H}}\marg{\meta{width}}%
\marg{\meta{height}}\marg{\meta{x}}\marg{\meta{y}}

\begin{definition}
\cmdname{twocolumnDbottominarea}\oarg{\meta{pages}}%
\marg{\meta{H}}\marg{\meta{width}}%
\marg{\meta{height}}\marg{\meta{x}}\marg{\meta{y}}
\end{definition}
This is equivalent to:\newline
\cmdname{twocolumnbottominarea}\oarg{\meta{pages}}%
\marg{dynamic}\marg{\meta{H}}\marg{\meta{width}}%
\marg{\meta{height}}\marg{\meta{x}}\marg{\meta{y}}

\begin{definition}
\cmdname{twocolumnbottom}\oarg{\meta{pages}}%
\marg{\meta{type}}\marg{\meta{H}}
\end{definition}
As \cmdname{twocolumnbottominarea} where the area is the entire
\gls*{typeblock}.

\begin{definition}
\cmdname{twocolumnSbottom}\oarg{\meta{pages}}\marg{\meta{H}}
\end{definition}
This is equivalent to:
\cmdname{twocolumnbottom}\oarg{\meta{pages}}\marg{static}\marg{\meta{H}}

\begin{definition}
\cmdname{twocolumnDbottom}\oarg{\meta{pages}}\marg{\meta{H}}
\end{definition}
This is equivalent to:
\cmdname{twocolumnbottom}\oarg{\meta{pages}}\marg{dynamic}\marg{\meta{H}}

\begin{definition}
\cmdname{NcolumnSbottominarea}\oarg{\meta{pages}}%
\marg{\meta{n}}\marg{\meta{H}}\marg{\meta{width}}\marg{\meta{height}}%
\marg{\meta{x}}\marg{\meta{y}}
\end{definition}
This is equivalent to:\newline
\cmdname{Ncolumnbottominarea}\oarg{\meta{pages}}\marg{static}%
\marg{\meta{n}}\marg{\meta{H}}\marg{\meta{width}}\marg{\meta{height}}%
\marg{\meta{x}}\marg{\meta{y}}

\begin{definition}
\cmdname{NcolumnDbottominarea}\oarg{\meta{pages}}%
\marg{\meta{n}}\marg{\meta{H}}\marg{\meta{width}}\marg{\meta{height}}%
\marg{\meta{x}}\marg{\meta{y}}
\end{definition}
This is equivalent to:\newline
\cmdname{Ncolumnbottominarea}\oarg{\meta{pages}}\marg{dynamic}%
\marg{\meta{n}}\marg{\meta{H}}\marg{\meta{width}}\marg{\meta{height}}%
\marg{\meta{x}}\marg{\meta{y}}

\begin{definition}
\cmdname{Ncolumnbottom}\oarg{\meta{pages}}%
\marg{\meta{type}}\marg{\meta{n}}\marg{\meta{H}}
\end{definition}
As \cmdname{Ncolumnbottominarea} where the area is the entire
\gls*{typeblock}.

\begin{definition}
\cmdname{NcolumnSbottom}\oarg{\meta{pages}}\marg{\meta{n}}\marg{\meta{H}}
\end{definition}
This is equivalent to:
\cmdname{Ncolumnbottom}\oarg{\meta{pages}}\marg{static}\marg{\meta{n}}\marg{\meta{H}}

\begin{definition}
\cmdname{NcolumnDbottom}\oarg{\meta{pages}}\marg{\meta{n}}\marg{\meta{H}}
\end{definition}
This is equivalent to:
\cmdname{Ncolumnbottom}\oarg{\meta{pages}}\marg{dynamic}\marg{\meta{n}}\marg{\meta{H}}
\htmlnav

\section{Right to Left Columns}
\label{sec:RL}

The default behaviour for the commands defined above is
to create the \glspl{flow} from left to right. However if you are
typesetting from right to left, you will probably prefer to 
define the \glspl{flow} in the reverse order. This can be done
via the package option \pkgopt{RL}. Alternatively you can use the
command:
\begin{definition}
\cmdname{lefttorightcolumnsfalse}
\end{definition}
before using commands such as \cmdname{twocolumn} or 
\cmdname{Ncolumn}. Two switch back to left-to-right columns use:
\begin{definition}
\cmdname{lefttorightcolumnstrue}
\end{definition}

\section{Backdrop Effects}

\Glspl{static} can be used to produce a 
backdrop. There are a number of commands which create
\glspl*{static} that can be used as a backdrop. In the
following definitions, \meta{pages} is the \gls{pglist} for
which those \glspl*{static} are defined (\texttt{all} is the default). 
For the vertical strips:
\meta{xoffset} is the amount by which the frames should be
shifted horizontally (0pt by default), \meta{W1} is the width of the first frame,
with colour specified by \meta{C1} and \gls{idl}\ \meta{L1},
and so on up to \meta{Wn} the width of the \meta{n}th frame
with colour specified by \meta{Cn} and \gls*{idl}\ \meta{Ln}.
For the vertical strips:
\meta{yoffset} is the amount by which the frames should be
shifted vertically (0pt by default), \meta{H1} is the height of the first frame,
with colour specified by \meta{C1} and \gls*{idl}\ \meta{L1},
and so on up to \meta{Hn} the height of the \meta{n}th frame
with colour specified by \meta{Cn} and \gls*{idl}\ \meta{Ln}.

\textbf{NOTE:} unlike the earlier commands, these commands
are all relative to the actual page, not the \gls{typeblock}.
So an \latexhtml{$x$}{x} offset of 0pt indicates the first vertical 
frame is flush with the left hand edge of the page, and a 
\latexhtml{$y$}{y} offset of 0pt indicates the first horizontal frame 
is flush with the bottom edge of the page. This is because backdrops 
tend to span the entire page, not just the \gls*{typeblock}.

The colour specification must be completely enclosed in braces,
for example \verb"{[rgb]{1,0,1}}" not \verb"[rgb]{1,0,1}".\htmlnav

\subsection{Vertical stripe effects}

\begin{definition}
\cmdname{vtwotone}\oarg{\meta{pages}}\oarg{\meta{xoffset}}%
\marg{\meta{W1}}\marg{\meta{C1}}\marg{\meta{L1}}%
\marg{\meta{W2}}\marg{\meta{C2}}\marg{\meta{L2}}
\end{definition}
Defines two \glspl{static} to create a two tone vertical strip effect. 
(This command was
used to create the coloured background on the title page
of \html{the PDF version of }this document.)

\begin{definition}
\cmdname{vNtone}\oarg{\meta{pages}}\oarg{\meta{xoffset}}%
\marg{\meta{n}}\marg{\meta{W1}}\marg{\meta{C1}}\marg{\meta{L1}}%
\ldots\marg{\meta{Wn}}\marg{\meta{Cn}}\marg{\meta{Ln}}
\end{definition}
This is similar to \cmdname{vtwotone} but with \meta{n} \glspl*{static} 
instead of two.

\begin{definition}
\cmdname{vtwotonebottom}\oarg{\meta{pages}}\oarg{\meta{xoffset}}%
\marg{\meta{H}}\marg{\meta{W1}}\marg{\meta{C1}}\marg{\meta{L1}}%
\marg{\meta{W2}}\marg{\meta{C2}}\marg{\meta{L2}}
\end{definition}
This is similar to \cmdname{vtwotone} but the \glspl*{static} are only
\meta{H} high, instead of the entire height of the page.
The frames are aligned along the bottom edge of the page.

\begin{definition}
\cmdname{vtwotonetop}\oarg{\meta{pages}}\oarg{\meta{xoffset}}%
\marg{\meta{H}}\marg{\meta{W1}}\marg{\meta{C1}}\marg{\meta{L1}}%
\marg{\meta{W2}}\marg{\meta{C2}}\newline\marg{\meta{L2}}
\end{definition}
This is similar to \cmdname{vtwotone} but the \glspl*{static} are only
\meta{H} high, instead of the entire height of the page.
The frames are aligned along the top edge of the page.
(This command was used to create the border effect along
the top of every page in \html{the PDF version of }this document. 
Two \cmdname{vtwotonetop}
commands were used, one for the even pages, and the other
for the odd pages.)

\begin{definition}
\cmdname{vNtonebottom}\oarg{\meta{pages}}\oarg{\meta{xoffset}}%
\marg{\meta{H}}\marg{\meta{n}}\marg{\meta{W1}}\marg{\meta{C1}}%
\marg{\meta{L1}}%
\ldots
\marg{\meta{Wn}}\marg{\meta{Cn}}\marg{\meta{Ln}}
\end{definition}
This is a general version of \cmdname{vtwotonebottom} for
\meta{n} frames instead of two.

\begin{definition}
\cmdname{vNtonetop}\oarg{\meta{pages}}\oarg{\meta{xoffset}}%
\marg{\meta{H}}\marg{\meta{n}}\marg{\meta{W1}}\marg{\meta{C1}}%
\marg{\meta{L1}}%
\ldots
\marg{\meta{Wn}}\marg{\meta{Cn}}\marg{\meta{Ln}}
\end{definition}
This is a general version of \cmdname{vtwotonetop} for
\meta{n} frames instead of two.\htmlnav

\subsection{Horizontal stripe effect}

\begin{definition}
\cmdname{htwotone}\oarg{\meta{pages}}\oarg{\meta{yoffset}}%
\marg{\meta{H1}}\marg{\meta{C1}}\marg{\meta{L1}}%
\marg{\meta{H2}}\marg{\meta{C2}}\marg{\meta{L2}}
\end{definition}
This defines two \glspl{static} to create a two tone horizontal strip 
effect.

\begin{definition}
\cmdname{hNtone}\oarg{\meta{pages}}\oarg{\meta{yoffset}}%
\marg{\meta{n}}\marg{\meta{H1}}\marg{\meta{C1}}\marg{\meta{L1}}%
\ldots
\marg{\meta{Hn}}\marg{\meta{Cn}}\marg{\meta{Ln}}
\end{definition}
This is similar to \cmdname{htwotone} but with \meta{n} \glspl*{static} 
instead of two.

\begin{definition}
\cmdname{htwotoneleft}\oarg{\meta{pages}}\oarg{\meta{yoffset}}%
\marg{\meta{W}}\marg{\meta{H1}}\marg{\meta{C1}}\marg{\meta{L1}}%
\marg{\meta{H2}}\marg{\meta{C2}}\marg{\meta{L2}}
\end{definition}
This is similar to \cmdname{htwotone} but the \glspl*{static} are only
\meta{W} wide, instead of the entire width of the page.
The frames are aligned along the left edge of the page.

\begin{definition}
\cmdname{htwotoneright}\oarg{\meta{pages}}%
\oarg{\meta{yoffset}}\marg{\meta{W}}\marg{\meta{H1}}%
\marg{\meta{C1}}\marg{\meta{L1}}%
\marg{\meta{H2}}\marg{\meta{C2}}\marg{\meta{L2}}
\end{definition}
This is similar to \cmdname{htwotone} but the \glspl*{static} are only
\meta{W} wide, instead of the entire width of the page.
The frames are aligned along the right edge of the page.

\begin{definition}
\cmdname{hNtoneleft}\oarg{\meta{pages}}%
\oarg{\meta{yoffset}}\marg{\meta{W}}\marg{\meta{n}}%
\marg{\meta{H1}}\marg{\meta{C1}}\marg{\meta{L1}}%
\ldots
\marg{\meta{Hn}}\marg{\meta{Cn}}\marg{\meta{Ln}}
\end{definition}
This is a general version of \cmdname{htwotoneleft} for
\meta{n} frames instead of two.

\begin{definition}
\cmdname{hNtoneright}\oarg{\meta{pages}}%
\oarg{\meta{yoffset}}\marg{\meta{W}}\marg{\meta{n}}%
\marg{\meta{H1}}\marg{\meta{C1}}\marg{\meta{L1}}%
\ldots
\marg{\meta{Hn}}\marg{\meta{Cn}}\marg{\meta{Ln}}
\end{definition}
This is a general version of \cmdname{htwotoneright} for
\meta{n} frames instead of two.\htmlnav

\subsection{Background Frame}

To make a single \gls{static} covering the entire page, use:
\begin{definition}
\cmdname{makebackgroundframe}\oarg{\meta{pages}}\oarg{\meta{IDL}}
\end{definition}
Note that this \gls*{frame} should be created before any other
\gls*{static} as it will obscure all other \glspl*{static} created 
before it if it is given a background colour.\htmlnav

\subsection{Vertical and Horizontal Rules}
\label{sec:insertrule}

You can create vertical or horizontal rules between two
\glspl{frame} using the commands:

\begin{definition}
\cmdname{insertvrule}\oarg{\meta{y top}}%
\oarg{\meta{y bottom}}\marg{\meta{T1}}\marg{\meta{IDN1}}%
\marg{\meta{T2}}\marg{\meta{IDN2}}
\end{definition}
This creates a new \gls{static} which fits between \meta{T1} 
\gls*{frame} with \gls{idn}\ \meta{IDN1} and \meta{T2} \gls*{frame} 
with \gls*{idn}\ \meta{IDN2}, and places a vertical rule in it
extending from the highest point of the highest frame to
the lowest point of the lowest frame. The first optional
argument \meta{y top} (default 0pt) extends the rule by that 
much above the highest point, and the second optional argument
\meta{y bottom} (default 0pt) extends the rule by that much 
below the lowest point. If either of the optional arguments
are negative, the rule will be shortened instead of extended.
The width of the rule is given by
\begin{definition}
\cmdname{ffcolumnseprule}
\end{definition}
\textbf{Note} that this has changed as from version 1.09: versions
prior to 1.09 used \cmdname{columnseprule}.

The vertical rule drawn by \cmdname{insertvrule} is created using 
the command:
\begin{definition}
\cmdname{ffvrule}\marg{offset}\marg{width}\marg{height}
\end{definition}
This can be redefined if a more ornate rule is required
(see below).

\begin{definition}
\cmdname{inserthrule}\oarg{\meta{x left}}%
\oarg{\meta{x right}}\marg{\meta{T1}}\marg{\meta{IDN1}}%
\marg{\meta{T2}}\marg{\meta{IDN2}}
\end{definition}
This creates a new \gls{static} which fits between \meta{T1} 
\gls*{frame} with \gls{idn}\ \meta{IDN1} and \meta{T2} \gls*{frame} 
with \gls*{idn}\ \meta{IDN2}, and places a horizontal rule in it
extending from the leftmost point of the left frame to
the rightmost point of the right frame. The first optional
argument \meta{x left} (default 0pt) extends the rule by that 
much before the leftmost point, and the second optional argument
\meta{x right} (default 0pt) extends the rule by that much 
beyond the rightmost point. If either of the optional arguments
are negative, the rule will be shortened instead of extended.
The height of the rule is given by
\begin{definition}
\cmdname{ffcolumnseprule}
\end{definition}

The horizontal rule drawn by \cmdname{inserthrule} is created using 
the command:
\begin{definition}
\cmdname{ffhrule}\marg{offset}\marg{width}\marg{height}
\end{definition}
This can be redefined if a more ornate rule is required (see below).

The default value for \cmdname{ffcolumnseprule} is 2pt.  Both
\cmdname{insertvrule} and \cmdname{inserthrule} have starred versions
which allow you to identify the \gls*{frame} by \gls{idl}\ instead of
\gls*{idn}.  The \gls*{frame} types, \meta{T1} and \meta{T2} can be one
of the following keywords: \texttt{flow}, \texttt{static} or
\texttt{dynamic}.

The command
\begin{definition}
\cmdname{ffruledeclarations}
\end{definition}
can be redefined to set declarations that affect how the rule is drawn. 
The most likely use of this command is to set the rule colour. 
For example:
\begin{verbatim}
\twocolumnStop{2in}

\renewcommand{\ffruledeclarations}{\color{red}}
\insertvrule{flow}{1}{flow}{2}

\renewcommand{\ffruledeclarations}{\color{blue}}
\inserthrule{static}{1}{flow}{1}
\end{verbatim}
This will create a layout with two columns (\glspl{flow}~1 and~2) 
with a \gls{static} above. A red vertical rule is placed in a 
\gls*{static} between \glspl*{flow}~1 and~2, and a blue horizontal
rule is placed between the \gls*{static} and the first 
\gls*{flow}. (However the horizontal rule will span both 
\glspl*{flow} since that is the width of the \gls*{static}.)

In the following example, the rules have been redefined
to use a zigzag pattern (which is obtained using the \sty{tikz} 
package):
\begin{verbatim}
\usepackage{flowfram}
\usepackage{tikz}
\usetikzlibrary{snakes}

\twocolumnStop

\renewcommand{\ffvrule}[3]{%
\hfill
\tikz{\draw[snake=zigzag,line width=#2,yshift=-#1] (0,0) -- (0pt,#3);}%
\hfill\mbox{}}

\insertvrule{flow}{1}{flow}{2}

\renewcommand{\ffhrule}[3]{%
\tikz{\draw[snake=zigzag,line width=#3,xshift=-#1] (0,0) -- (#2,0pt);}}

\inserthrule{static}{1}{flow}{1}
\end{verbatim}
\htmlnav

\chapter{Thumbtabs and Minitocs}
\chapdesc{This chapter describes how to create thumbtabs and minitocs,
such as those used in \html{the PDF version of }this document.}

\section{Thumbtabs}
\label{sec:thumbtabs}

On the right hand side of \latexhtml{this page}{the odd pages 
and on the left hand side of the even pages in the
PDF version of this document}, there is a blue rectangle
with the chapter number in it. This is a thumbtab, and it gives
you a rough idea whereabouts in the document you are when you
quickly flick through the pages. Each thumbtab is in fact
a dynamic frame, and you can control whether to make the
number and/or title appear in the thumbtab by using
the package option \pkgopt{thumbtabs}. This is a key=value option,
where the value may be one of \pkgoptval{thumbtabs}{title} 
(show the title but not the number---default), 
\pkgoptval{thumbtabs}{number} (show the number but not the title), 
\pkgoptval{thumbtabs}{both} (show the number and the title) 
and \pkgoptval{thumbtabs}{none} (don't show the number or title).

If you want thumbtabs in your document, you need to use
the command
\begin{definition}
\cmdname{makethumbtabs}\oarg{\meta{y offset}}\marg{\meta{height}}%
\oarg{\meta{section type}}
\end{definition} 
in the document
preamble. By default, the topmost thumbtab is level with the
top of the \gls{typeblock}, but can be shifted vertically
using the first optional argument \meta{y offset}. Each
thumbtab will be \meta{height} high, and will correspond to the
sectioning type \meta{section type}. If \meta{section type} 
is omitted, chapters will be used if the \cmdname{chapter} 
command is defined, otherwise sections will be used.
The width of the thumbtabs is given by the length 
\begin{definition}
\cmdname{thumbtabwidth}
\end{definition}
which is 1cm by default.
The command
\begin{definition}
\cmdname{thumbtabindex}
\end{definition}
will display the thumbtab index (all thumbtabs) on the current page. 
You then need to use
\begin{definition}
\cmdname{enablethumbtabs}
\end{definition}
to start the individual thumbtabs and
\begin{definition}
\cmdname{disablethumbtabs}
\end{definition}
to make them go away.
You can align the table of contents with the thumbtabs\footnote{but
only do this if there is enough room on the page!} using
the command
\begin{definition}
\cmdname{tocandthumbtabindex}
\end{definition}
instead of
the commands \cmdname{tableofcontents} and 
\cmdname{thumbtabindex}. If you are using the \sty{hyperref} 
package, the text on the thumbtab index will be a hyperlink
to the corresponding part of the document. Note that when using
\cmdname{tocandthumbtabindex} you may
need to shift the thumbtabs vertically up or down to make
sure that they align correctly with the table of contents.

The format of the text on the thumbtabs is given by the command
\begin{definition}
\cmdname{thumbtabindexformat}
\end{definition}
for the thumbtab index entries, and
\begin{definition}
\cmdname{thumbtabformat}
\end{definition}
for the individual thumbtabs. By
default the text on the thumbtabs will be rotated, but as
rotating is not implemented by some previewers, the package option
\pkgopt{norotate} is provided, which will stack the letters 
vertically. This does not look as good as the rotated text.
Note also that some previewers do not put the hyperlink in the
correct place when the link has been rotated, so this may
also cause a problem.

The thumbtab attributes can be changed using 
\begin{definition}
\cmdname{setthumbtab}\marg{\meta{n}}\marg{\meta{key value list}}
\end{definition}
where \meta{n} is the thumbtab number starting
from 1 (for the top thumbtab) to the value given by the counter
\ctr{maxthumbtabs}
(for the bottom thumbtab). Note that these numbers are not
related to the associated \gls*{frame} \gls{idn}. You may also use
the keyword \texttt{all} instead of \meta{n} to indicate that the
new attributes should apply to all thumbtabs.

To just change the settings for the thumbtab index, use
\begin{definition}
\cmdname{setthumbtabindex}\marg{\meta{n}}\marg{\meta{key value list}}
\end{definition}
The \meta{key value list} for both these commands is the same as that
for \cmdname{setdynamicframe}. Again \meta{n} may either be the
thumbtab index or the keyword \texttt{all}.

By default, the thumbtabs have a grey background. In \html{the PDF
version of }this document, I \latex{have }used:
\begin{verbatim}
\setthumbtab{1}{backcolor=[rgb]{0.15,0.15,1}}
\setthumbtab{2}{backcolor=[rgb]{0.2,0.2,1}}
\setthumbtab{3}{backcolor=[rgb]{0.25,0.25,1}}
\setthumbtab{4}{backcolor=[rgb]{0.3,0.3,1}}
\setthumbtab{5}{backcolor=[rgb]{0.35,0.35,1}}
\setthumbtab{6}{backcolor=[rgb]{0.4,0.4,1}}
\setthumbtab{7}{backcolor=[rgb]{0.45,0.45,1}}
\setthumbtab{8}{backcolor=[rgb]{0.5,0.5,1}}
\end{verbatim}
to change the thumbtab background colour to shades of blue.

I \latex{have }also changed the style of the thumbtab text using:
\begin{verbatim}
\newcommand{\thumbtabstyle}[1]{{\hypersetup{linkcolor=white}%
\textbf{\large\sffamily #1}}}
\setthumbtab{all}{style=thumbtabstyle,textcolor=white}
\end{verbatim}
Note that the style uses \cmdname{hypersetup}\footnote{defined by the
\sty{hyperref} package} to change the colour of the hyperlink text,
since the hyperlink overrides the text colour.\htmlnav

\section{Minitocs}

In \html{the PDF version of }this document, after each chapter heading, 
there is a mini table of contents for that chapter. To enable minitocs,
use the command 
\begin{definition}
\cmdname{enableminitoc}\oarg{\meta{section type}}
\end{definition}
The default \meta{section type} is the same as that used by 
the thumbtabs.

If you want the minitocs to appear in a \gls{dynamic}, you
can use
\begin{definition}
\cmdname{appenddfminitoc}\marg{\meta{IDN}}
\end{definition}
where \meta{IDN} is the \gls{idn}\ of the appropriate \gls*{dynamic}.
There is also a starred version available if you want to 
use the \gls{idl}\ instead of the \gls*{idn}.

For example, in \html{the PDF version of }this document I 
\latex{have }used the command:
\begin{verbatim}
\appenddfminitoc*{chaphead}
\end{verbatim}
in the preamble, which \latex{has }appended the minitocs to the
\gls*{dynamic} with \gls*{idl}\ \texttt{chaphead}.

The style of the minitoc text is given by the command
\begin{definition}
\cmdname{minitocstyle}\marg{\meta{contents}}
\end{definition}
where the argument is the contents
of the minitoc. This command may be redefined if you want
to change the minitoc style. The gap before the minitoc
is given by the length
\begin{definition}
\cmdname{beforeminitocskip}
\end{definition}
and the gap after the minitoc is given by the length 
\begin{definition}
\cmdname{afterminitocskip}
\end{definition}
These lengths may be changed using \cmdname{setlength}.\htmlnav

\chapter{Global Settings}
\chapdesc{This section describes style macros, lengths and counters used
by the \styni{flowfram} package.}

\section{Macros}
The following macros can be changed using \cmdname{renewcommand}:

\begin{itemize}
\item \cmdname{setffdraftcolor} 

This sets the colour of the \gls{bbox}
when it is displayed in draft mode.  The default value is:
\verb/\color[gray]{0.8}/. For example, if you want a darker grey,
do:
\begin{verbatim}
\renewcommand{\setffdraftcolor}{\color[gray]{0.3}}
\end{verbatim}

\item \cmdname{setffdrafttypeblockcolor}

This sets the colour of
the \gls{bbox} of the \gls{typeblock} when it is displayed
in draft mode.  The default value is: \verb/\color[gray]{0.9}/.
For example, if you want a medium grey, do:
\begin{verbatim}
\renewcommand{\setffdrafttypeblockcolor}{\color[gray]{0.5}}
\end{verbatim}

\item \cmdname{fflabelfont}

This sets the font size for the \gls{bbox} markers in 
draft mode. The default value is: \verb/\small\sffamily/.
For example, if you want a larger font, do:
\begin{verbatim}
\renewcommand{\fflabelfont}{\large\sffamily}
\end{verbatim}

\item \cmdname{ffruledeclarations}

This sets the declarations that affect the rules created using
\cmdname{insertvrule} and \cmdname{inserthrule}. The default
definition does nothing.
See \latexhtml{\autoref{sec:insertrule}}{\htmlref{Vertical and
Horizontal Rules}{sec:insertrule}} for further details.

\item \cmdname{ffcontinuedtextfont}\marg{\meta{text}}

This sets \meta{text} in the continuation font. The default definition
does \verb|\emph{\small |\meta{text}\verb|}|. See
\latexhtml{\autoref{sec:static}}{\htmlref{Static Frames}{sec:static}} 
and \latexhtml{\autoref{sec:dynamic}}{\htmlref{Dynamic 
Frames}{sec:dynamic}} for further details.

\end{itemize}
\htmlnav

\section{Lengths}
The following are lengths, which can be changed using
\cmdname{setlength}:

\begin{itemize}
\item \cmdname{fflabelsep}

This is the distance from the right hand side of the
\gls{bbox} at which to place the \gls{bbox} marker. The
default value is: \texttt{1pt}

\item \cmdname{flowframesep}

This is the gap between the text of the \gls{frame} and
its border, for the standard border types. 

\item \cmdname{flowframerule}

This is the width of the \gls{frame}['s] border, if using
a border given by a \gls{fcmd} that uses \cmdname{fboxsep}
to set its border width (e.g.\ \cmdname{fbox}).

\item \cmdname{sdfparindent}

This is the paragraph indentation within \staticordynamic{}s.
The default value is 0pt.

\item \cmdname{vcolumnsep}

This is the approximate vertical distance between the top frame 
and the column frames when using \cmdname{Ncolumntop} etc.
(The height of the \gls{flow} may be adjusted to make it
an integer multiple of \cmdname{baselineskip}.)

\item \cmdname{columnsep}

This is the horizontal distance between the column frames
when using \cmdname{Ncolumn} or \cmdname{Ncolumntop} etc

\item \cmdname{ffcolumnseprule}

This is the width of vertical rules created using 
\cmdname{insertvrule} or the height of horizontal rules created
using \cmdname{inserthrule}.

\item \cmdname{beforeminitocskip}

This is the vertical distance before the minitoc.

\item \cmdname{afterminitocskip}

This is the vertical distance after the minitoc.

\item \cmdname{fftolerance}

The output routine will issue a warning when a paragraph spans two
\glspl{flow} of unequal width, unless the difference in width is
less than the value of \cmdname{fftolerance}.
\end{itemize}
\htmlnav

\section{Counters}
\label{sec:counters}
The following are counters that can be accessed via 
\cmdname{value}\marg{\meta{counter name}} or via
\verb|\the|\meta{counter name}. However the value of these counters 
\emph{should not be modified}.

\begin{description}
\item[\ctr{maxflow}] The total number of \glspl{flow} that have
been defined so far.

\item[\ctr{thisframe}] Stores the \gls{idn} of the current
\gls*{flow}. You can label and reference the \gls{idn} using
\begin{definition}
\cmdname{labelflowid}\marg{\meta{label}}
\end{definition}
This is analogous to the standard \cmdname{label} command, but will
always refer to the \gls*{idn} of the current \gls*{flow}. It can
then be referenced using \cmdname{ref}\marg{\meta{label}}. Note that
this will always refer to the current \gls*{flow} even when used
in the contents of a \staticordynamic.

Don't use more than one instance of \cmdname{labelflowid} in a given
\gls*{flow} for a given page or you will get a \dq{multiply defined 
references} warning.

\item[\ctr{displayedframe}] Stores the index of the currently 
displayed \gls{flow}. This will be the same as the \gls{idn} if all
\glspl*{flow} are displayed on the current page, but if some are
hidden, the values may be different. You can label this counter 
using
\begin{definition}
\cmdname{labelflow}\marg{\meta{label}}
\end{definition}
and reference it elsewhere in the document using 
\cmdname{ref}\marg{\meta{label}}. For example, if you are using
a column layout, you might want to do something like:
\begin{verbatim}
This text is about hippos\labelflow{hippos}.

% Somewhere else in the document
See column~\ref{hippos} on page~\pageref{hippos}
for information on hippos.
\end{verbatim}
Don't use more than one instance of \cmdname{labelflow} in a given
\gls*{flow} for a given page or you will get a \dq{multiply defined 
references} warning.  Note that
\cmdname{labelflow} will always refer to the current \gls*{flow} 
even when used in the contents of a static or dynamic \gls{frame}.

\item[\ctr{maxstatic}] The total number of \glspl{static} that have
been defined so far.

\item[\ctr{maxdynamic}] The total number of \glspl{dynamic} that
have been defined so far.

\item[\ctr{maxthumbtabs}] The total number of thumbtabs.

\item[\ctr{absolutepage}] The absolute page number.
\end{description}
\htmlnav

\chapter{Troubleshooting}
\chapdesc{This chapter should be consulted if you experience any
problems using the \styni{flowfram} package.}

For an up-to-date list of frequently asked questions, see
\latexhtml{%
\url{http://www.dickimaw-books.com/faqs/flowframfaq.html}}{the
\htmladdnormallink{flowfram
FAQ}{http://www.dickimaw-books.com/faqs/flowframfaq.html}}. If you
have a query that is not addressed here, please try there first. If
that doesn't answer your query, try posting a message to
\TeX\ on StackExchange (\url{http://tex.stackexchange.com/}),
the \LaTeX\ Community Forum (\url{http://latex-community.org/forum/})
or the \texttt{comp.text.tex} newsgroup. I generally answer questions in
those places much quicker than queries that are emailed to me, which
tend to get lost in my inbox.

Bugs can be reported using the \latexhtml{form at
\url{http://www.dickimaw-books.com/bug-report.html}}{\htmladdnormallink{bug
report form}{http://www.dickimaw-books.com/bug-report.html}}
\htmlnav

\section{General Queries}

\begin{enumerate}
\item If all my \glspl{flow} are only defined on, say, 
pages 1-10, what happens if I then add some extra text so that
the document exceeds 10 pages?

The output routine will create a new \gls{flow} the size 
of the \gls{typeblock} and use that.

\item Can I use the formatted page number in \glspl{pglist}?

No.

\item\label{itm:whynot} Why not?

When the output routine finishes with one \gls{flow} it looks for the
next \gls{flow} defined on that page. If there are none left, it then
searches through the \gls{pglist} of all the defined \glspl{flow} to
see if the next page lies in that range. If there are none defined on
that page, it ships out that page, and tries the next page.  This gives
rise to two problems:

\begin{enumerate}
\item \LaTeX\ is not clairvoyant. If it is currently
on page 14, and on the next page the page numbering changes
to A, it has no way of knowing this until it has reached
that point, which it hasn't yet. So it is looking for a 
\gls{flow} defined on page~15, not on page~A.

\item How does \LaTeX\ tell if page C lies between 
pages A and D\@? It would require an algorithm that can convert
from a formatted number back to an integer. Given that there
are many different ways of formatting the value of a counter
(besides the standard Roman and alphabetical formats) it
would be impossible to write an algorithm to do this
for some arbitrary format.
\end{enumerate}

\item Can I have an arbitrarily shaped \gls{frame}?
\label{itm:parshape}

You can assign certain irregular shapes to \staticordynamic{}s, 
using the \key{shape} key (see 
\latexhtml{\autoref{sec:parshape}}{\htmlref{Non-Rectangular 
Frames}{sec:parshape}}).
Note that the \gls{bbox} will still appear as a rectangle with
the dimensions of the given \gls{frame} which may not correspond
to the assigned shape.
This function is not available for \glspl{flow}.

\item Why has the text from my \gls{flow} appeared in a
\gls{static} or \gls{dynamic}?

Assuming you haven't inadvertently set that text as the contents
of the \staticordynamic, the frames are most likely 
overlapping (see \latexhtml{\autoref{sec:stacking}}{\htmlref{Frame
Stacking Order}{sec:stacking}}).
In an attempt to clarify what's going on, suppose you have 
defined a \gls{static}, a \gls{dynamic} and two \glspl{flow}. The 
following is an approximate\footnote{The pedantic may point out 
that \TeX\ may make several attempts to fill in the flow frames 
depending on penalties and so on.} analogy: \TeX\ has a sheet of 
paper on the table, and has pencilled\footnote{actually it hasn't
drawn anything really, but it has in its mind's eye.} in a 
rectangle denoting the \gls{typeblock}.  The paper is put to one 
side for now.  \TeX\ also has four rectangular sheets
of transparent paper. The first (which I shall call sheet~1)
represents the \gls{static}, the next two (which I shall call
sheets~2 and~3) represent the \glspl{flow}, and the last one
(which I shall call sheet~4) represents the \gls{dynamic}.
\TeX\ starts work on filling sheet~2 with the document text.
Once it has put as much text on that sheet as it considers 
possible (according to its views on aesthetics), it puts sheet~2
into the \dq{in tray}, and then continues on sheet~3. While it's
filling in sheets~2 and~3, if it encounters a command or 
environment that tells it what to put in the \gls{static}, 
it fills in sheet~1 and then puts sheet~1 into the \dq{in tray} and
resumes where it left off on sheet~2 or~3. Similarly, if
it encounters a command that tells it what to put in the
\gls{dynamic}, it stops what it's doing, fills in sheet~4, then
puts sheet~4 into the \dq{in tray}, and resumes where it left off.
Only when it has finished sheet~3 (the last \gls{flow} defined
on that page), will it gather together all
the transparent sheets, and fix them onto the page starting
with sheet~1 through to sheet~4, measuring the bottom left hand 
corner of each transparent sheet relative to the bottom left hand 
corner of the \gls{typeblock}.  \TeX\ will then put that page 
aside, and start work on the next page. If two or more of the
transparent sheets overlap, you will see through the top one into
the one below (unless of course the top one has been painted
over, either by setting a background colour, or by adding an
image that has a non-transparent background.)

Note that it's also possible that the overlap is caused by an 
overfull hbox that's causing the text to poke out the side of the 
\gls{flow} into a neighbouring \gls{frame}. Check the log file for
warnings or use the \pkgoptni{draft} option to the document class
which will place a filled rectangle on the end of overfull lines.

\item Why do I get lots of overfull hbox messages?

Probably because you have narrow \glspl{frame}. It's better to use
ragged right formatting when the frames are narrow.

\item Why do I keep getting multiply-defined warnings?

Probably because you have used \cmdname{label} in a \staticordynamic\
that is displayed on more than one page. Try using the \key{clear} 
key to ensure that the frame is always cleared at the end of each
page.

\item What happens if I use a command or environment
that switches to two-column mode (e.g. \env{theindex})?
                                                          
As from version 1.01, any \cmdname{onecolumn} or 
\cmdname{twocolumn} commands that occur outside of the
preamble will print the contents of the optional argument,
and issue a warning. It is recommended that you set up
your own frames for use in the index. See the source code of this
document, \texttt{ffuserguide.tex}, for an example.

\item How do I change the vertical alignment
of material inside a \staticordynamic?

Use the \key{valign} key in  \cmdname{setstaticframe} or 
\cmdname{setdynamicframe} (new to version 1.03).

\item\label{itm:absval} How do I compute the distance from the edge
of the page instead of the \gls{typeblock}?

See \latexhtml{\autoref{sec:typeblockloc}}{\htmlref{Determining the 
Location of the Typeblock}{sec:typeblockloc}}.

\item Is there a GUI I can use to make it easier to create the
\glspl{frame}?

Yes, \texttt{flowframtk} which can be downloaded from:
\url{http://www.dickimaw-books.com/apps/flowframtk/}
\end{enumerate}
\htmlnav

\section{Unexpected Output}
\label{sec:unexpectedoutput}

\begin{enumerate}
\item The lines at the beginning of my \glspl{flow} are the
wrong width.

This is a problem that will occur if you have \glspl{flow} 
with different widths, as the change in \cmdname{hsize} 
does not come into effect until a paragraph break. So if
you have a paragraph that spans two \glspl{flow}, the end
of the paragraph at the beginning of the second \gls{flow} 
will retain the width it had at the start of the 
paragraph at the bottom of the previous \gls{flow}. You can
fix the problem by inserting \cmdname{framebreak} at the
point where the \gls{frame} break occurs 
(see \latexhtml{\autoref{sec:framebreak}}{\htmlref{Prematurely Ending a 
Flow Frame}{sec:framebreak}}).

\item My \glspl{frame} shift to the right when I add a border.

This may occur if you use a border that is not recognised
by the \styni{flowfram} package. You will need to set the
offset using the \texttt{offset} key (see 
\latexhtml{\autoref{sec:modattr}}{\htmlref{Modifying Frame 
Attributes}{sec:modattr}}).

\item I have a vertical white strip along the right hand side
of every page.

This can happen if you have, say, an A4 document, and 
\appname{ghostscript} has letter as the default paper size. You can 
change the default paper size by editing the file \verb|gs_init.ps|. 
Change:
\begin{verbatim}
% Optionally choose a default paper size other than U.S. letter.
% (a4)
\end{verbatim}
to:
\begin{verbatim}
% Optionally choose a default paper size other than U.S. letter.
(a4)
\end{verbatim}

\item I don't have any output.

All your \glspl{flow} are empty. \TeX\ doesn't put the
frames onto the page until it has finished putting text
into the \glspl*{flow}. So if there is no text to go in the
\glspl*{flow} it won't output the page. If you only want the
\glspl{static} or \glspl{dynamic} filled in, and nothing 
outside of them, just do \verb|\mbox{}\clearpage|. 
This will put
an invisible something with zero area into your
\gls*{flow}, but it's enough to convince \TeX\ that the
document contains some text.

\item The last page hasn't appeared.

See the previous answer.

\item There is no paragraph indentation inside my 
\staticordynamic{}s.

The paragraph indentation in \staticordynamic{}s is governed
by the length \cmdname{sdfparindent} which is set to 0pt by
default. To make the indentation the same as that used by 
\glspl{flow} place the following in the preamble:
\begin{verbatim}
\setlength{\sdfparindent}{\parindent}
\end{verbatim}

\item My section numbering is in the wrong order.

Remember that the contents of the \glspl{dynamic} are not set
until the page is shipped out, and the contents will be set
in the order of \gls{idn}, so if you have any sectioning commands
occuring within \glspl*{dynamic}, they may not be set in the same
order as they are in your input file.

\item The contents of my \staticordynamic\ have shifted
to the left when I used \cmdname{parshape}.

This will happen if your \cmdname{parshape} specification
exceeds the linewidth. For example:
\begin{verbatim}
\parshape=1 0.4\linewidth 0.7\linewidth
\end{verbatim}
This specifies a line with overall length 
\verb"1.1\linewidth" which is too long.

\end{enumerate}
\htmlnav

\section{Error Messages}

\begin{enumerate}
\item \verb/Illegal unit of measure (pt inserted)/

All lengths must have units. Remember to include the
units when defining new \glspl{frame}. The following
keys require lengths: \texttt{width}, \texttt{height},
\texttt{x}, \texttt{y} and \texttt{offset}\footnote{%
\texttt{offset} can also have the value \texttt{compute}}.

\item \verb/Missing number, treated as zero/

\LaTeX\ is expecting a number. There are a number of 
possible causes:

\begin{enumerate}
\item You have used an \gls{idl}\ instead of an \gls{idn}. If you
want to refer to a frame by its label, you need to remember
to use the starred versions of the 
\verb/\set/\meta{type}\verb/frame/ commands, or when setting
the contents of \glspl{static} or \glspl{dynamic}.

\item When specifying page lists, you have mixed keywords
with page ranges. For example: \texttt{1,even} is invalid.
\end{enumerate}

\item \verb/Flow frame IDL '/\meta{label}\verb+' already defined+

All \glspl{idl} within each \gls{frame} type must be
unique. There are similar error messages for duplicate 
\glspl*{idl} for \glspl{static} and \glspl{dynamic}.

\item \verb/Can't find flow frame id/

You have specified a non-existent \gls*{flow} \gls{idl}. There are
similar error messages for \glspl{static} and \glspl{dynamic}.
Check to make sure you have spelt the label correctly, and
check you are using the correct \gls{frame} type command.
(For example, if a \gls*{static} has the \gls*{idl}\ \texttt{mylabel},
and you attempt to do 
\cmdname{setflowframe*}\marg{mylabel}\marg{\meta{options}}, 
then you will get this error, because \texttt{mylabel} refers
to a \gls*{static} not a \gls{flow}.)

\item \verb/Key 'clear' is boolean/

The \texttt{clear} key can only have the values \texttt{true} 
or \texttt{false}.

\item \verb/Key 'clear' not available/

The \texttt{clear} key is only available for \staticordynamic{}s.

\item \verb/Key 'style' not available/

The \texttt{style} key is only available for \glspl{dynamic}.

\item \verb/Key 'margin' not available/

The \texttt{margin} key is only available for \glspl{flow}.

\item \verb/Key 'shape' not available/

The \texttt{shape} key is only available for \staticordynamic{}s.

\item \verb/Dynamic frame style '/\meta{style}\verb+' not defined+

The specified style \meta{style} must be the name of a command
without the preceding backslash.  It is possible that you have
mis-spelt the name, or you have forgotten to define the command.

\item \verb/Argument of \fbox has an extra }/

This error will occur if you do, say, \verb/border=\fbox/
instead of \verb/border=fbox/. Remember not to include
the initial backslash.

\item \verb/Not in outer par mode/

You can not have floats (such as figures, tables or marginal
notes) in \staticordynamic{}s. If you want
a figure or table within a \staticordynamic\ 
use \env{staticfigure} or \env{statictable}.

\item \verb/Somethings wrong---maybe missing \item/

Assuming that all your list type of environments start
with \cmdname{item}, this may be caused by something going
wrong with the toc (table of contents), ttb (thumbtab)
or aux (auxiliary) files in the previous run. Try deleting
them, and try again.

\item \verb/No room for a new \skip/

You have exceeded \TeX's 256 register limit. Use the \sty{etex} 
package.

\item \verb|\verb illegal in command argument|

As a general rule, you can't use \Index{verbatim text} in a command 
argument. 
This rule applies to all the commands defined by the \styni{flowfram} 
package.  See also below.

\item I get \verb|\verb illegal in command argument| when using
\Index{verbatim text} inside the \env{dynamiccontents} environment.

You can not use \Index{verbatim text} inside either the starred or
unstarred version of the \env{dynamiccontents} environment.
(See \latexhtml{page~\pageref{pg:verb}}{\htmlref{earlier}{pg:verb}}.)
\htmlnav

\end{enumerate}
\disablethumbtabs

\printglossary
\htmlnav

\html{\label{index}}
\printindex

\htmlnav

\ifthenelse{\isodd{page}}{}{\clearpage\mbox{}}
% have a backcover:
\setdynamicframe*{footer}{pages=none}
\setstaticframe*{lastbackleft,lastbackright}{pages=even}
\clearpage\mbox{}
\end{document}
