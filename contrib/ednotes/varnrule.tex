%% varnrule.tex -- demo for varying footnote rules with 
%% ednotes.sty (manyfoot feature now supported by ednotes 
%% -- search ednotes.sty for `Customizing *footnote rules*'). 
%% Uwe Lueck http://contact-ednotes.sty.de.vu 2006/07/06 

%% Present task: Sergei Mariev wants to use ordinary (numbered) 
%  footnotes as well as critical apparatuses. An ordinal footnote 
%  rule should separate the main text from the footnotes in any 
%  case. The additional footnotes should be separated by usual 
%  footnote rules as well. However, there shouldn't be a rule 
%  between the ordinary numbered footnotes and the remaining 
%  ones. 
% 
\documentclass[12pt]{article}

%  Below \endinput here, there is a sample for experiencing the 
%  manyfoot feature without ednotes ...: 
% 
%%%% \iffalse %% To switch to the mere manyfoot sample -- needs 
%%%%          %% removing percent signs preceding corresponding 
%%%%          %% \fi as well! 

\newcommand{\SelectAnoteRule}{[1]{no}}
\newcommand{\SelectBnoteRule}{[0]{default}}
% <- Priorities 1 and 0 will be overridden by priority 2 
%    of the Standard LateX footnote rule, 
%    so there will be a footnote rule between main text and the 
%    footnotes in any case. Priority 1 above ensures that 
%    there will never be a visible rule between ordinary footnotes 
%    and the new A/B/C footnotes handled by manyfoot. 
%    `no' corresponds to \newcommand{\nofootnoterule} below; 
%    `default' refers to the standard footnote rule as stored 
%    by manyfoot. 
%    -- \newcommand{\Select... must come before loading ednotes. 

\usepackage[Bpara,Cpara]{ednotes}

\renewcommand{\footnoterulepriority}{2} 
% <- setting default priority, especially for the ordinary LaTeX 
%    footnotes. This priority is the maximal one here and thus 
%    overrides all the remaining ones. So there will be a rule 
%    between main text and the footnotes in any case. 
%    -- The previous line must come /after/ loading ednotes. 

\newcommand{\nofootnoterule}{} 
% <- or \let\nofootnoterule\empty; or call it \emptyfootnoterule. 
%    If so, use `empty' instead of `no' above. 
%    -- This line may come anywhere before the first page 
%    is output. 

\begin{document} %%%%%%%%%%%%%%%%%%%%%%%%%%%%%%%%%%%%%%%%%%%%%%%%%%%%% 

\linenumbers 

%% Remove or add some percent signs for playing: 
Some text.\footnote{Ordinary footnote.} 
% \Anote{A lemma.}{A note.} 
\Bnote{B lemma.}{B note.} 
\Cnote{C lemma.}{C note.} 

\end{document} %%%%%%%%%%%%%%%%%%%%%%%%%%%%%%%%%%%%%%%%%%%%%%%%%%%%%%% 


Code demonstrating manyfoot's feature of customized footnote rules: 

%%%% \fi 

\nofiles 

\usepackage[para,ruled]{manyfoot}
% <- I wasn't able to get the below -- much wanted -- results 
%    without the `ruled' option -- may someone do better!? 

\renewcommand{\footnoterulepriority}{2} 
% <- setting default priority, especially for the ordinary LaTeX 
%    footnotes. This priority is the maximal one here and thus 
%    overrides all the remaining ones. So there will be a rule 
%    between main text and the footnotes in any case. 

\newcommand{\nofootnoterule}{} 
% <- or \let\nofootnoterule\empty; or call it \emptyfootnoterule. 
%    If so, use `empty' instead of `no' below. 

\SelectFootnoteRule[1]{no}
\newfootnote[para]{A} 
% <- Priorities 1 here and 0 below will be overridden by priority 
%    2 of the Standard LateX footnote rule, 
%    so there will be a footnote rule between main text and the 
%    footnotes in any case. Priority 1 just above ensures that 
%    there will never be a visible rule between ordinary footnotes 
%    and the new ("level") footnotes handled by manyfoot. 

\SelectFootnoteRule[0]{default}
% <- `default' refers to the standard footnote rule as stored 
%    by manyfoot. 
\newfootnote[para]{B} 
\newfootnote[para]{C} 
% <- Notes in A, B, C will be separated by rules in the usual way. 

\begin{document} %%%%%%%%%%%%%%%%%%%%%%%%%%%%%%%%%%%%%%%%%%%%%%%%%%%%% 

%% Remove or add some percent signs for playing: 
Some text.\footnote{Ordinary footnote.}% 
% \FootnotetextA{}{A}% 
\FootnotetextB{}{B}%
\FootnotetextC{}{C}

\end{document} %%%%%%%%%%%%%%%%%%%%%%%%%%%%%%%%%%%%%%%%%%%%%%%%%%%%%%% 

End of varnrule.tex

