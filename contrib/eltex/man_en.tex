\documentstyle[a4,bezier]{article}

%%%%%%%%%%% accordingly the CSN IEC 617-1 to 11 (1993-1995) %%%%%%%%
%%%%%%%%%%% Libor Gajdosik 2009 eltex version 2.0 %%%%%%%%%%%%%

\unitlength=1mm
%%%%%%%%grid -generating help grid 10x10mm to simplify orientation%%%%%%
\newcommand{\grid}[2]{
\linethickness{0.1mm}
\unitlength=10mm
  \newcounter{sloupec}
  \setcounter{sloupec}{10}
\put(0,0){\line(0,1){#2}}
\multiput(1,0)(1,0){#1}{\line(0,1){#2}}
\multiput(0.7,-0.5)(1,0){#1}{\arabic{sloupec}\addtocounter{sloupec}{10}}
  \setcounter{sloupec}{10}
\multiput(0.6,#2)(1,0){#1}{\makebox(0.7,0.6)
       {\arabic{sloupec}\addtocounter{sloupec}{10}}}
   \newcounter{radek}
 \put(-0.3,-0.3){\arabic{radek}}
   \setcounter{radek}{10}
\put(0,0){\line(1,0){#1}}
\multiput(0,1)(0,1){#2}{\line(1,0){#1}}
\multiput(-0.8,0.7)(0,1){#2}{\makebox(0.7,0.5)[r]
      {\arabic{radek}\addtocounter{radek}{10}}}
   \setcounter{radek}{10}
\multiput(#1,0.6)(0,1){#2}{\makebox(0.7,0.6)[r]
    {\arabic{radek}\addtocounter{radek}{10}}}
\thicklines
\unitlength=1mm
}
%%%%basic pasive devices (source, resistor, capacitor, inductor)%%%%
%horizontal voltage source
\newcommand{\hsourcev}{
\put(0,0){\line(1,0){30}}
\put(15,0){\circle{10}}
}
%vertical voltage source
\newcommand{\vsourcev}{
\put(0,0){\line(0,1){30}}
``\put(0,15){\circle{10}}
}
%horizontal current source
\newcommand{\hsourcec}{
\put(0,0){\line(1,0){10}}
\put(15,-5){\line(0,1){10}}
\put(15,0){\circle{10}}
\put(20,0){\line(1,0){10}}
}
%vertical current source
\newcommand{\vsourcec}{
\put(0,0){\line(0,1){10}}
\put(5,15){\line(-1,0){10}}
\put(0,15){\circle{10}}
\put(0,20){\line(0,1){10}}
}
\newcommand{\hhsourcev}{
\put(0,0){\line(1,0){20}}
\put(10,0){\circle{10}}
}
%vertical voltage source
\newcommand{\vvsourcev}{
\put(0,0){\line(0,1){20}}
\put(0,10){\circle{10}}
}
%diagonal voltage source
\newcommand{\dsourcev}[1]{
\ifx#1U   %source up right oriented
  \put(0,0){\line(1,1){20}}
  \put(10,10){\circle{10}}
\else \ifx#1D  %source down right oriented
   \put(0,0){\line(1,-1){20}}
   \put(10,-10){\circle{10}}
\fi \fi
}
%horizontal current source
\newcommand{\hhsourcec}{
\put(0,0){\line(1,0){5}}
\put(10,-5){\line(0,1){10}}
\put(10,0){\circle{10}}
\put(15,0){\line(1,0){5}}
}
%vertical current source
\newcommand{\vvsourcec}{
\put(0,0){\line(0,1){5}}
\put(5,10){\line(-1,0){10}}
\put(0,10){\circle{10}}
\put(0,15){\line(0,1){5}}
}
%diagonal current source
\newcommand{\dsourcec}[1]{
\ifx#1U   %source up right oriented
  \put(0,0){\line(1,1){6.5}}
  \put(6.5,13.5){\line(1,-1){7}}
  \put(10,10){\circle{10}}
  \put(13.5,13.5){\line(1,1){6.5}}
\else \ifx#1D  %source down right oriented
  \put(0,0){\line(1,-1){6.5}}
  \put(13.5,-6.5){\line(-1,-1){7}}
  \put(10,-10){\circle{10}}
  \put(13.5,-13.5){\line(1,-1){6.5}}
\fi \fi
}
%horizontal rezistor
\newcommand{\hhrez}[1]{
\put(0,0){\line(1,0){5}}
\put(5,-2){\framebox(10,4){}\put(0,2){\line(1,0){5}}}
\ifnum#1=1           % rez. adjustable by tool (medium pin down oriented)
\put(8,-6){\line(1,3){4}}
\put(10,7){\line(3,-1){4}}
\else \ifnum#1=2     % rez. adjustable by tool (medium pin up oriented)
\put(8,6){\line(1,-3){4}}
\put(10,-7){\line(3,1){4}}
\else \ifnum#1=3     % rez. adjustable by tool (medium pin shortly
                                                        %conected)
\put(8,6){\line(1,-3){4}}
\put(10,-7){\line(3,1){4}}
\put(8,6){\line(-1,0){5}}
\put(3,6){\line(0,-1){6}}
\put(3,0){\node}
\else \ifnum#1=4       % rez. adjustable (medium pin down oriented)
\put(6,-4){\vector(1,1){9}}
\else \ifnum#1=5       % rez. adjustable (medium pin up oriented)
\put(6,4){\vector(1,-1){9}}
\else \ifnum#1=6       % rez. adjustable (medium pin shortly conected)
\put(8,6){\vector(1,-3){4}}
\put(8,6){\line(-1,0){5}}
\put(3,6){\line(0,-1){6}}
\put(3,0){\node}
\else \ifnum#1=7           %nonlinearly dependent
\put(6,-4){\line(1,1){8}}
\put(3,-4){\line(1,0){3}}


\fi \fi \fi \fi \fi \fi \fi
  }
%vertical rezistor
\newcommand{\vvrez}[1]{
\put(0,0){\line(0,1){5}}
\put(-2,5){\framebox(4,10){}\put(-2,10){\line(0,1){5}}}
\ifnum#1=1          % rez. adjustable by tool (medium pin left oriented)
\put(-6,8){\line(3,1){12}}
\put(5,14){\line(1,-2){2}}
\else \ifnum#1=2    % rez. adjustable by tool (medium pin right oriented)
\put(-6,12){\line(3,-1){12}}
\put(-7,10){\line(1,2){2}}
\else \ifnum#1=3   % rez. adjustable by tool (medium pin shortly
                                                        %conected)
\put(-6,7){\line(3,1){12}}
\put(5,13){\line(1,-2){2}}
\put(-6,7){\line(0,-1){4}}
\put(-6,3){\line(1,0){6}}
\put(0,3){\node}
\else \ifnum#1=4       % rez. adjustable (medium pin left oriented)
\put(-4,6){\vector(1,1){9}}
\else \ifnum#1=5       % rez. adjustable (medium pin right oriented)
\put(4,6){\vector(-1,1){9}}
\else \ifnum#1=6      % rez. adjustable (medium pin shortly conected)
\put(-6,8){\vector(3,1){12}}
\put(-6,8){\line(0,-1){5}}
\put(-6,3){\line(1,0){6}}
\put(0,3){\node}
\else \ifnum#1=7
\put(-4,6){\line(1,1){8}}
\put(4,14){\line(1,0){3}}

\fi \fi \fi \fi \fi \fi \fi
  }
%diagonal rezistor
\newcommand{\drez}[2]{
\ifx#1U      % up left oriented
\put(0,0){\line(1,1){7}\put(2.2,5.2){\line(-1,1){4}}
\put(-1.9,9){\line(1,1){7}}\put(2.2,5){\line(1,1){7}}
\put(9,12){\line(-1,1){4}}\put(7,14){\line(1,1){6}}}
      \ifnum#2=1        % rez. adjustable by tool (medium pin left oriented)
\put(2,10){\line(1,0){16}}
\put(18,12){\line(0,-1){4}}
      \fi \ifnum#2=2    % rez. adjustable by tool (medium pin right oriented)
\put(2,10){\line(1,0){16}}
\put(2,12){\line(0,-1){4}}
      \fi \ifnum#2=3   % rez. adjustable by tool (medium pin shortly
                                                        %conected)
\put(2,10){\line(1,0){16}}
\put(18,12){\line(0,-1){4}}
\put(2,10){\line(0,-1){4}}
\put(2,6){\line(1,0){4}}
\put(6,6){\node}
      \fi \ifnum#2=4         % rez. adjustable (medium pin left oriented)
\put(2,10){\vector(1,0){16}}
      \fi \ifnum#2=5         % rez. adjustable (medium pin right oriented)
\put(18,10){\vector(-1,0){16}}
      \fi \ifnum#2=6         % rez. adjustable (medium pin shortly conected)
\put(2,10){\vector(1,0){16}}
\put(2,10){\line(0,-1){4}}
\put(2,6){\line(1,0){4}}
\put(6,6){\node}
       \fi \ifnum#2=7
\put(2,10){\line(1,0){16}}
\put(18,10){\line(1,1){4}}
       \fi   \fi

\ifx#1D              %down right oriented
\put(0,0){\line(1,-1){7}\put(-1.9,-9){\line(1,1){4}}
\put(2,-4.9){\line(1,-1){7}}\put(-1.9,-9){\line(1,-1){7}}}
\put(12,-16){\line(1,1){4}\put(-2,2){\line(1,-1){6}}}
       \ifnum#2=1   % rez. adjustable by tool (medium pin left oriented)
\put(2,-10){\line(1,0){16}}
\put(18,-8){\line(0,-1){4}}
       \else \ifnum#2=2  % rez. adjustable by tool (medium pin right oriented)
\put(2,-10){\line(1,0){16}}
\put(2,-8){\line(0,-1){4}}
      \else \ifnum#2=3   % rez. adjustable by tool (medium pin shortly
                                                        %conected)
\put(2,-10){\line(1,0){16}}
\put(18,-8){\line(0,-1){4}}
\put(2,-10){\line(0,1){4}}
\put(2,-6){\line(1,0){4}}
\put(6,-6){\node}
     \else \ifnum#2=4    % rez. adjustable (medium pin left oriented)
\put(2,-10){\vector(1,0){16}}
     \else \ifnum#2=5    % rez. adjustable (medium pin right oriented)
\put(18,-10){\vector(-1,0){16}}
     \else \ifnum#2=6    % rez. adjustable (medium pin shortly conected)
\put(2,-10){\vector(1,0){16}}
\put(2,-10){\line(0,1){4}}
\put(2,-6){\line(1,0){4}}
\put(6,-6){\node}
      \else \ifnum#2=7
\put(2,-10){\line(1,0){16}}
\put(18,-10){\line(1,1){4}}
\fi  \fi  \fi \fi  \fi \fi \fi \fi
}

%diagonal rezistor (short)
\newcommand{\ddrez}[2]{
\ifx#1U      % up left oriented
\put(0,0){\line(1,1){4}\put(2,2){\line(-1,1){4}}
\put(2,2){\line(1,1){7}}\put(-2,6){\line(1,1){7}}
\put(9,9){\line(-1,1){4}}\put(7,11){\line(1,1){4}}}
      \ifnum#2=1        % rez. adjustable by tool (medium pin left oriented)
\put(-1,7){\line(1,0){16}}
\put(15,9){\line(0,-1){4}}
      \fi \ifnum#2=2    % rez. adjustable by tool (medium pin right oriented)
\put(0,7){\line(1,0){16}}
\put(0,9){\line(0,-1){4}}
      \fi \ifnum#2=3   % rez. adjustable by tool (medium pin shortly
                                                        %conected)
\put(-1,8){\line(1,0){16}}
\put(15,10){\line(0,-1){4}}
\put(-1,8){\line(0,-1){6}}
\put(-1,2){\line(1,0){3}}
\put(2,2){\node}
      \fi \ifnum#2=4         % rez. adjustable (medium pin left oriented)
\put(1,8){\vector(1,0){16}}
      \fi \ifnum#2=5         % rez. adjustable (medium pin right oriented)
\put(16,8){\vector(-1,0){16}}
      \fi \ifnum#2=6         % rez. adjustable (medium pin shortly conected)
\put(-1,8){\vector(1,0){18}}
\put(-1,8){\line(0,-1){6}}
\put(-1,2){\line(1,0){3}}
\put(2,2){\node}
       \fi \ifnum#2=7
\put(2,8){\line(1,0){16}}
\put(18,8){\line(1,1){4}}
       \fi   \fi

\ifx#1D              %down right oriented
\put(0,0){\line(1,-1){4}\put(-2,-6){\line(1,1){4}}
\put(-2,-6){\line(1,-1){7}}\put(2,-2){\line(1,-1){7}}}
\put(9,-13){\line(1,1){4}\put(-2,2){\line(1,-1){4}}}
       \ifnum#2=1   % rez. adjustable by tool (medium pin left oriented)
\put(0,-8){\line(1,0){16}}
\put(16,-6){\line(0,-1){4}}
       \else \ifnum#2=2  % rez. adjustable by tool (medium pin right oriented)
\put(0,-8){\line(1,0){16}}
\put(0,-6){\line(0,-1){4}}
      \else \ifnum#2=3   % rez. adjustable by tool (medium pin shortly
                                                        %conected)
\put(0,-8){\line(1,0){16}}
\put(16,-6){\line(0,-1){4}}
\put(0,-8){\line(0,1){6}}
\put(0,-2){\line(1,0){2}}
\put(2,-2){\node}
     \else \ifnum#2=4    % rez. adjustable (medium pin left oriented)
\put(0,-8){\vector(1,0){16}}
     \else \ifnum#2=5    % rez. adjustable (medium pin right oriented)
\put(16,-8){\vector(-1,0){16}}
     \else \ifnum#2=6    % rez. adjustable (medium pin shortly conected)
\put(0,-8){\vector(1,0){16}}
\put(0,-8){\line(0,1){6}}
\put(0,-2){\line(1,0){2}}
\put(2,-2){\node}
      \else \ifnum#2=7
\put(0,-8){\line(1,0){16}}
\put(16,-8){\line(1,1){4}}
\fi  \fi  \fi \fi  \fi \fi \fi \fi
}
%horizontal rezistor
\newcommand{\hrez}[1]{
\put(0,0){\line(1,0){10}}
\put(10,-2){\framebox(10,4){}\put(0,2){\line(1,0){10}}}
\ifnum#1=1           % rez. adjustable by tool (medium pin down oriented)
\put(12,-6){\line(1,3){4}}
\put(14,7){\line(3,-1){4}}
\else \ifnum#1=2     % rez. adjustable by tool (medium pin up oriented)
\put(12,6){\line(1,-3){4}}
\put(14,-7){\line(3,1){4}}
\else \ifnum#1=3     % rez. adjustable by tool (medium pin shortly
                                                        %conected)
\put(12,6){\line(1,-3){4}}
\put(14,-7){\line(3,1){4}}
\put(12,6){\line(-1,0){4}}
\put(8,6){\line(0,-1){6}}
\put(8,0){\node}
\else \ifnum#1=4       % rez. adjustable (medium pin down oriented)
\put(11,-4){\vector(1,1){9}}
\else \ifnum#1=5       % rez. adjustable (medium pin up oriented)
\put(11,4){\vector(1,-1){9}}
\else \ifnum#1=6       % rez. adjustable (medium pin shortly conected)
\put(12,6){\vector(1,-3){4}}
\put(12,6){\line(-1,0){4}}
\put(8,6){\line(0,-1){6}}
\put(8,0){\node}
\else \ifnum#1=7             %%nonlinearly  dependent
\put(11,-4){\line(1,1){8}}
\put(19,4){\line(1,0){3}}
\fi \fi \fi \fi \fi \fi  \fi
  }
%vertical rezistor
\newcommand{\vrez}[1]{
\put(0,0){\line(0,1){10}}
\put(-2,10){\framebox(4,10){}\put(-2,10){\line(0,1){10}}}
\ifnum#1=1          % rez. adjustable by tool (medium pin left oriented)
\put(-6,12){\line(3,1){12}}
\put(5,18){\line(1,-2){2}}
\else \ifnum#1=2    % rez. adjustable by tool (medium pin right oriented)
\put(-6,18){\line(3,-1){12}}
\put(-7,16){\line(1,2){2}}
\else \ifnum#1=3   % rez. adjustable by tool (medium pin shortly
                                                        %conected)
\put(-6,12){\line(3,1){12}}
\put(5,18){\line(1,-2){2}}
\put(-6,12){\line(0,-1){4}}
\put(-6,8){\line(1,0){6}}
\put(0,8){\node}
\else \ifnum#1=4       % rez. adjustable (medium pin left oriented)
\put(-4,11){\vector(1,1){9}}
\else \ifnum#1=5       % rez. adjustable (medium pin right oriented)
\put(4,11){\vector(-1,1){9}}
\else \ifnum#1=6      % rez. adjustable (medium pin shortly conected)
\put(-6,12){\vector(3,1){12}}
\put(-6,12){\line(0,-1){4}}
\put(-6,8){\line(1,0){6}}
\put(0,8){\node}
\else \ifnum#1=7
\put(-4,11){\line(1,1){8}}
\put(4,19){\line(1,0){3}}
\fi \fi \fi \fi \fi \fi \fi
 }
%horizontal capacitor
\newcommand{\hcap}[1]{
\put(0,0){\line(1,0){14.5}}
\multiput(14.5,3)(1,0){2}{\line(0,-1){6}}
\put(15.5,0){\line(1,0){14.5}}
\ifnum#1=1              % capacitor adjustable by tool
\put(18,-3){\line(-1,1){6}}
\put(9.5,0.5){\makebox(5,5){/}}
 \else \ifnum#1=4       %capacitor adjustable
\put(18,-3){\vector(-1,1){6}}
 \fi \fi
  }
%vertical capacitor
\newcommand{\vcap}[1]{
\put(0,0){\line(0,1){14.5}}
\multiput(-3,14.5)(0,1){2}{\line(1,0){6}}
\put(0,15.5){\line(0,1){14.5}}
 \ifnum#1=1                %cap. adjustable by tool
\put(2.5,12.5){\line(-1,1){6}}
\put(-6,16){\makebox(5,5){/}}
 \else \ifnum#1=4             %cap. adjustable
\put(-3,12){\vector(1,1){7}}
 \fi \fi
  }
%diagonal capacitor
\newcommand{\dcap}[2]{
\ifx#1U          % up right oriented
\put(0,0){\line(1,1){9.25}}
\multiput(7.25,11.75)(1,1){2}{\line(1,-1){4.5}}
\put(10.75,10.75){\line(1,1){9.25}}
  \ifnum#2=1                    %cap. adjustable with tool
\put(5,10){\line(1,0){10}}
\put(15,8){\line(0,1){4}}
  \else \ifnum#2=4             %cap. adjustable
\put(5,10){\vector(1,0){12}}
   \fi \fi
\fi
\ifx#1D                      % down right oriented
\put(0,0){\line(1,-1){9.3}}
\multiput(7.25,-11.75)(1,-1){2}{\line(1,1){4.5}}
\put(10.75,-10.75){\line(1,-1){9.25}}
  \ifnum#2=1                        %cap. adjustable with tool
\put(10,-15){\line(0,1){10}}
\put(8,-5){\line(1,0){4}}
  \else \ifnum#2=4            %cap. adjustable
\put(10,-15){\vector(0,1){11}}
  \fi \fi
\fi
}
%horizontal capacitor
\newcommand{\hhcap}[1]{
\put(0,0){\line(1,0){4.5}}
\multiput(4.5,3)(1,0){2}{\line(0,-1){6}}
\put(5.5,0){\line(1,0){4.5}}
\ifnum#1=1              % capacitor adjustable by tool
\put(8,-3){\line(-1,1){6}}
\put(-0.5,0.5){\makebox(5,5){/}}
 \else \ifnum#1=4       %capacitor adjustable
\put(8,-3){\vector(-1,1){7}}
 \fi \fi
  }
%vertical capacitor
\newcommand{\vvcap}[1]{
\put(0,0){\line(0,1){4.5}}
\multiput(-3,4.5)(0,1){2}{\line(1,0){6}}
\put(0,5.5){\line(0,1){4.5}}
 \ifnum#1=1                %cap. adjustable by tool
\put(2.5,2.5){\line(-1,1){6}}
\put(-6,6){\makebox(5,5){/}}
 \else \ifnum#1=4             %cap. adjustable
\put(-3,2){\vector(1,1){7}}
 \fi \fi
  }
%\horizontal 2 turns
\newcommand{\hturn}[2]{
\ifx#1U               %up arces
\multiput(1.5,0)(3,0){2}{\oval(3,3)[t]}
  \ifnum#2=1            %variable inducance
\put(1.5,4){\line(1,-2){3}}
\put(-0.5,3.25){\line(2,1){4}}
  \fi
\else \ifx#1D         %down arces
\multiput(1.5,0)(3,0){2}{\oval(3,3)[b]}
 \ifnum#2=1            %variable inductance
\put(1.5,-4){\line(1,2){3}}
\put(-0.5,-3.5){\line(2,-1){4}}
  \fi
\fi \fi
}
%vertical 2 turns
\newcommand{\vturn}[2]{
\ifx#1L               %left arces
\multiput(0,1.5)(0,3){2}{\oval(3,3)[l]}
   \ifnum#2=1           %variable inductance
 \put(1,4.5){\line(-2,-1){4}}
 \put(-4.5,4.5){\line(1,-2){2}}
    \fi
\else \ifx#1R        %right arces
\multiput(0,1.5)(0,3){2}{\oval(3,3)[r]}
    \ifnum#2=1           %variable inductance
 \put(-1,1.5){\line(2,1){4}}
 \put(2.4,5.5){\line(1,-2){2}}
    \fi
\fi \fi
}


%\horizontal inductor
\newcommand{\hind}[2]{
\put(0,0){\line(1,0){9}}
\ifx#1U               %up arces
\multiput(10.5,0)(3,0){4}{\oval(3,3)[t]}
  \ifnum#2=1            %variable inducance
\put(12,7){\line(1,-2){5}}
\put(15.25,-4.25){\line(2,1){4}}
  \fi
\else \ifx#1D         %down arces
\multiput(10.5,0)(3,0){4}{\oval(3,3)[b]}
 \ifnum#2=1            %variable inductance
\put(12,-7){\line(1,2){5}}
\put(15.25,4.25){\line(2,-1){4}}
  \fi
\fi \fi
\put(21,0){\line(1,0){9}}
}
%vertical inductor
\newcommand{\vind}[2]{
\put(0,0){\line(0,1){9}}
\ifx#1L               %left arces
\multiput(0,10.5)(0,3){4}{\oval(3,3)[l]}
   \ifnum#2=1           %variable inductance
 \put(4,18){\line(-2,-1){10}}
 \put(3,20){\line(1,-2){2}}
    \fi
\else \ifx#1R        %right arces
\multiput(0,10.5)(0,3){4}{\oval(3,3)[r]}
    \ifnum#2=1           %variable inductance
 \put(-4,13){\line(2,1){10}}
 \put(-5,15){\line(1,-2){2}}
    \fi
\fi \fi
\put(0,21){\line(0,1){9}}
}
%diagonal inductor
\newcommand{\dind}[3]{
\ifx#1D                 %down oriented
\put(0,0){\line(1,-1){5}}
\put(15,-15){\line(1,-1){5}}
       \ifx#2R          %right arces
\multiput(5,-5)(2.5,-2.5){4}{\bezier{40}(0,0)(3.5,1)(2.5,-2.5)}
        \ifnum#3=1          %variable inductance
    \put(5,-10.5){\line(1,0){12}}
    \put(5,-8.5){\line(0,-1){4}}
      \fi
    \fi
       \ifx#2L          %left arces
\multiput(5,-5)(2.5,-2.5){4}{\bezier{40}(0,0)(-1,-3.5)(2.5,-2.5)}
       \ifnum#3=1          %variable inductance
    \put(15,-11){\line(-1,0){12}}
    \put(15,-9){\line(0,-1){4}}
      \fi
  \fi
\fi
\ifx#1U                  %up oriented
\put(0,0){\line(1,1){5}}
\put(15,15){\line(1,1){5}}
   \ifx#2R               %right arces
\multiput(5,5)(2.5,2.5){4}{\bezier{40}(0,0)(3.5,-1)(2.5,2.5)}
    \ifnum#3=1              %variable inductance
  \put(6,11){\line(1,0){12}}
  \put(6,13){\line(0,-1){4}}
    \fi
    \fi
  \ifx#2L                %left arces
\multiput(5,5)(2.5,2.5){4}{\bezier{40}(0,0)(-1,3.5)(2.5,2.5)}
        \ifnum#3=1          %variable inductance
     \put(15,11){\line(-1,0){12}}
     \put(15,13){\line(0,-1){4}}
        \fi
   \fi
\fi
}
%\horizontal inductor
\newcommand{\hhind}[2]{
\put(0,0){\line(1,0){4}}
\ifx#1U               %up arces
\multiput(5.5,0)(3,0){4}{\oval(3,3)[t]}
  \ifnum#2=1            %variable inducance
\put(7,7){\line(1,-2){5}}
\put(10.25,-4.25){\line(2,1){4}}
  \fi
\else \ifx#1D         %down arces
\multiput(5.5,0)(3,0){4}{\oval(3,3)[b]}
 \ifnum#2=1            %variable inductance
\put(7,-7){\line(1,2){5}}
\put(10.25,4.25){\line(2,-1){4}}
  \fi
\fi \fi
\put(16,0){\line(1,0){4}}
}
%vertical inductor
\newcommand{\vvind}[2]{
\put(0,0){\line(0,1){4}}
\ifx#1L               %left arces
\multiput(0,5.5)(0,3){4}{\oval(3,3)[l]}
   \ifnum#2=1           %variable inductance
 \put(4,13){\line(-2,-1){10}}
 \put(3,15){\line(1,-2){2}}
    \fi
\else \ifx#1R        %right arces
\multiput(0,5.5)(0,3){4}{\oval(3,3)[r]}
    \ifnum#2=1           %variable inductance
 \put(-4,8){\line(2,1){10}}
 \put(-5,10){\line(1,-2){2}}
    \fi
\fi \fi
\put(0,16){\line(0,1){4}}
}
%diagonal coil core
\newcommand{\dcore}[3]{
\ifx#1D                    %down oriented
  \ifx#3I          %iron core
\put(0,0){\line(1,-1){#2}}
  \else \ifx#3F    %ferrite core
\multiput(0,0)(5,-5){#2}{\line(1,-1){4}}
   \fi \fi
 \else \ifx#1U             %up oriented
  \ifx#3I         %iron core
\put(0,0){\line(1,1){#2}}
  \else \ifx#3F   %ferrit core
\multiput(0,0)(5,5){#2}{\line(1,1){4}}
  \fi \fi
\fi \fi
}
%vertical coil core
\newcommand{\vcore}[2]{
 \ifx#2I            %iron core
\put(0,0){\line(0,1){#1}}
 \else \ifx#2F      %ferrite core
\multiput(0,0)(0,4){#1}{\line(0,1){3}}
 \fi \fi
}
%horizontal coil core
\newcommand{\hcore}[2]{
 \ifx#2I             %iron core
\put(0,0){\line(1,0){#1}}
 \else \ifx#2F       %ferrite core
\multiput(0,0)(4,0){#1}{\line(1,0){3}}
 \fi \fi
}
%%%switch%%%%%%%%%%%%
%horizontal switch
\newcommand{\hswitch}[1]{
\put(0,0){\line(1,0){2.5}}
  \ifnum#1=1                   %switch on
\put(2.5,0){\line(3,1){6}}
  \fi \ifnum#1=0               %switch off
\put(2.5,0){\line(3,1){6}}
\put(7.5,0){\line(0,1){1.9}}
       \fi
\put(7.5,0){\line(1,0){2.5}}
}
%vertical switch
\newcommand{\vswitch}[1]{
\put(0,0){\line(0,-1){2.5}}
  \ifnum#1=1               %switch on
\put(0,-2.5){\line(1,-3){2}}
  \fi \ifnum#1=0           %switch off
\put(0,-2.5){\line(1,-3){2}}
\put(0,-7.5){\line(1,0){1.9}}
       \fi
\put(0,-7.5){\line(0,-1){2.5}}
}
%horizontal overswitch
\newcommand{\hoswitch}[2]{
\ifx#1R                    %right oriented
  \ifnum#2=0               %switch off
\put(0,0){\line(1,0){2.5}}
\put(2.5,0){\line(2,1){6}}
\put(7.5,2.3){\line(0,1){2.8}}
\put(7.5,5){\line(1,0){2.5}}
\put(7.5,0){\line(1,0){2.5}}
       \fi
  \ifnum#2=1                   %switch on
\put(0,0){\line(1,0){2.5}}
\put(2.5,0){\line(2,-1){6}}
\put(7.5,-2.3){\line(0,-1){2.8}}
\put(7.5,-5){\line(1,0){2.5}}
\put(7.5,0){\line(1,0){2.5}}
  \fi
\fi
\ifx#1L                    %left oriented
  \ifnum#2=0               %switch off
\put(0,0){\line(1,0){2.5}}
\put(7.5,0){\line(-2,1){6}}
\put(2.5,2.3){\line(0,1){2.8}}
\put(0,5){\line(1,0){2.5}}
\put(7.5,0){\line(1,0){2.5}}
       \fi
  \ifnum#2=1                   %switch on
\put(0,0){\line(1,0){2.5}}
\put(7.5,0){\line(-2,-1){6}}
\put(2.5,-2.3){\line(0,-1){2.8}}
\put(0,-5){\line(1,0){2.5}}
\put(7.5,0){\line(1,0){2.5}}
  \fi
\fi
}
%vertical overswitch
\newcommand{\voswitch}[2]{
\ifx#1D                     %down oriented
    \ifnum#2=0           %switch off
\put(0,0){\line(0,-1){2.5}}
\put(0,-2.5){\line(-1,-2){3}}
\put(-5,-7.5){\line(1,0){2.5}}
\put(0,-7.5){\line(0,-1){2.5}}
\put(-5,-7.5){\line(0,-1){2.5}}
       \fi
  \ifnum#2=1               %switch on
\put(0,0){\line(0,-1){2.5}}
\put(0,-2.5){\line(1,-2){3}}
\put(2.5,-7.5){\line(1,0){2.5}}
\put(0,-7.5){\line(0,-1){2.5}}
\put(5,-7.5){\line(0,-1){2.5}}
  \fi
\fi
\ifx#1U                      %up oriented
    \ifnum#2=0           %switch off
\put(0,0){\line(0,1){2.5}}
\put(0,2.5){\line(-1,2){3}}
\put(-5,7.5){\line(1,0){2.5}}
\put(0,7.5){\line(0,1){2.5}}
\put(-5,7.5){\line(0,1){2.5}}
       \fi
  \ifnum#2=1               %switch on
\put(0,0){\line(0,1){2.5}}
\put(0,2.5){\line(1,2){3}}
\put(2.5,7.5){\line(1,0){2.5}}
\put(0,7.5){\line(0,1){2.5}}
\put(5,7.5){\line(0,1){2.5}}
  \fi
\fi
}
%%%%node, loop, pin, wire, earth, chassis%%%%%%%%%
\newcommand{\node}{       %conection of devices
\put(0,0){\circle*{1}}
}
\newcommand{\pin}{
\put(0,0){\circle{1.5}}
}
\newcommand{\hwire}[1]{
\put(0,0){\line(1,0){#1}}  % length mm
}
\newcommand{\vwire}[1]{
\put(0,0){\line(0,1){#1}}  % length mm
}
\newcommand{\dwire}[2]{
\ifx#1U                    %up right oriented
\put(0,0){\line(1,1){#2}}  %length mm
\fi
\ifx#1D                    %down right oriented
\put(0,0){\line(1,-1){#2}} %length mm
\fi
}
%simbol
\newcommand{\simb}[1]{
\ifnum#1=1
\put(0,0){\circle*{1.5}}
 \else \ifnum#1=2
\put(0,0){$\Box$}
 \else \ifnum#1=3
\put(0,0){$\triangle$}
 \fi \fi \fi
}
%current loop oriented
\newcommand{\cloop}[2]{
\ifx#1L             %anti-clockwise oriented
\put(0,0){\oval(12,12)[t]}
\put(0,0){\oval(12,12)[l]\put(0,-6){\vector(1,0){1}}}
\put(-5,-5){\makebox(10,10){#2}}   %label
   \else \ifx#1R       %clockwise oriented
\put(0,0){\oval(12,12)[t]}
\put(0,0){\oval(12,12)[l]\put(6,0){\vector(0,-1){1}}}
\put(-5,-5){\makebox(10,10){#2}}    %label
\fi \fi
}
%earth
\newcommand{\earth}[1]{
  \ifx#1D
    \put(0,0){\line(0,-1){5}}
    \put(-4,-5){\line(1,0){8}}
    \put(-3,-6){\line(1,0){6}}
    \put(-2,-7){\line(1,0){4}}
   \fi
   \ifx#1U
    \put(0,0){\line(0,1){5}}
    \put(-4,5){\line(1,0){8}}
    \put(-3,6){\line(1,0){6}}
    \put(-2,7){\line(1,0){4}}
   \fi
   \ifx#1L
    \put(0,0){\line(-1,0){5}}
    \put(-5,-4){\line(0,1){8}}
    \put(-6,-3){\line(0,1){6}}
    \put(-7,-2){\line(0,1){4}}
   \fi
   \ifx#1R
    \put(0,0){\line(1,0){5}}
    \put(5,-4){\line(0,1){8}}
    \put(6,-3){\line(0,1){6}}
    \put(7,-2){\line(0,1){4}}
   \fi

}
%chassis
\newcommand{\chassis}[1]{
\ifx#1D
 \put(0,0){\line(0,-1){5}}
 {\thicklines \put(-2.5,-5){\line(1,0){5}} }
\fi
\ifx#1U
 \put(0,0){\line(0,1){5}}
 {\thicklines \put(-2.5,5){\line(1,0){5}}}
\fi
\ifx#1L
 \put(0,0){\line(-1,0){5}}
{\thicklines \put(-5,-2.5){\line(0,1){5}}}
\fi
\ifx#1R
 \put(0,0){\line(1,0){5}}
{\thicklines \put(5,-2.5){\line(0,1){5}} }
\fi
}
%horizontal measuring instrument
\newcommand{\hmeasure}[1]{
\put(0,0){\line(1,0){10}}
\put(15,0){\circle{10}}
\put(20,0){\line(1,0){10}}
\put(10,-4.75){\makebox(10,10){#1}}    %label
}
\newcommand{\hhmeasure}[1]{
\put(0,0){\line(1,0){5}}
\put(10,0){\circle{10}}
\put(15,0){\line(1,0){5}}
\put(5,-4.75){\makebox(10,10){#1}}    %label
}
%vertical measuring instrument
\newcommand{\vmeasure}[1]{
\put(0,0){\line(0,1){10}}
\put(0,15){\circle{10}}
\put(0,20){\line(0,1){10}}
\put(-4.75,10){\makebox(10,10){#1}}  %label
}
\newcommand{\vvmeasure}[1]{
\put(0,0){\line(0,1){5}}
\put(0,10){\circle{10}}
\put(0,15){\line(0,1){5}}
\put(-4.75,5){\makebox(10,10){#1}}  %label
}
\newcommand{\osc}[1]{     %oscilloscope
\put(0,0){\line(0,1){5}}
   \ifx#1I
\put(0,10){\circle{10}}     %indicating
   \fi  \ifx#1R
\put(-5,5){\framebox(10,10){}}   %recording
   \fi
\put(0,15){\line(0,1){5}}
\multiput(-4,9)(4,0){2}{\line(2,1){4}}
\put(0,9){\line(0,1){2}}
}

\endinput

%%%%%%semiconductors%%%%%%%%%%%%%%%%%%%%%%%%%%%%%%%%%%%%%%%%
%%%%%%diode, thyristor, triac, diac, optomember, opamplifier
%%%%%%OTA, BOTA, 
%%%%%%tranzistor bjt, jfet, mosfet, ujt%%%%%%%%%%%%%%%%%
%%%%%%%graetz, Hall's generator, magnistor, thermistor, varistor,
%%%%%%%photoresistor,
\newcommand{\graetz}[1]{
   \ifx#1R
\put(0,0){\ddiode{U}{U}}
\put(20,20){\ddiode{D}{D}}
\put(0,0){\ddiode{D}{D}}
\put(20,-20){\ddiode{U}{U}}
   \fi
   \ifx#1L
\put(0,0){\ddiode{U}{D}}
\put(20,20){\ddiode{D}{U}}
\put(0,0){\ddiode{D}{U}}
\put(20,-20){\ddiode{U}{D}}
   \fi
}
%diagonal diode
\newcommand{\ddiode}[2]{
\ifx#1D                %down right oriented
    \put(0,0){\line(1,-1){20}}
    \put(10.5,-5.5){\line(-1,-1){5}}
      \ifx#2D                   %cathode down
   \put(10.5,-5.5){\line(1,-3){2.5}}
   \put(5.5,-10.5){\line(3,-1){7}}
      \fi \ifx#2U              %cathode up
   \put(10.5,-15.5){\line(-1,3){2.5}}
   \put(15.5,-10.5){\line(-3,1){7}}
           \fi
   \put(15.5,-10.5){\line(-1,-1){5}}
\fi

\ifx#1U                       %up right oriented
   \put(0,0){\line(1,1){20}}
   \put(10.5,5.5){\line(-1,1){5}}
      \ifx#2U                    %cathode up
   \put(10.5,5.5){\line(1,3){2.5}}
   \put(5.5,10.5){\line(3,1){7}}
      \fi \ifx#2D               %cathode down
   \put(10.5,15.5){\line(-1,-3){2.5}}
   \put(15.5,10.5){\line(-3,-1){7}}
           \fi
   \put(15.5,10.5){\line(-1,1){5}}
   \fi
}
%horizontal diode
\newcommand{\hdiode}[2]{
\ifx#1R                  %cathode right
 \put(0,0){\line(1,0){20}}
 \put(7,3){\line(0,-1){6}}
 \put(7,3){\line(2,-1){6}}
 \put(7,-3){\line(2,1){6}}
 \put(13,3){\line(0,-1){6}}
     \ifx#2C                %varicap
 \multiput(9,4)(3,0){2}{\line(1,0){2}}
 \multiput(11,2.5)(1,0){2}{\line(0,1){3}}
    \else \ifx#2Z                %zener diode
  \put(13,-3){\line(-1,0){2}}
    \else \ifx#2E                %esaki diode
  \put(13,-3){\line(-1,0){2}}
  \put(13,3){\line(-1,0){2}}
    \else \ifx#2S                %shottky diode
  \put(13,-3){\line(-1,0){1}}
  \put(12,-3){\line(0,1){1}}
  \put(13,3){\line(1,0){1}}
  \put(14,3){\line(0,-1){1}}
    \else \ifx#2D                %led diode
  \multiput(10,3)(2,2){2}{\vector(-1,1){4}}
    \else \ifx#2L                %laser diode
  \multiput(10,4)(2,2){2}{\vector(-1,1){4}}
  \put(9,3){\line(1,1){4}}
    \else \ifx#2P                       %photo diode
  \multiput(5,8)(2,2){2}{\vector(1,-1){4}}
    \fi \fi  \fi  \fi  \fi \fi \fi \fi

\ifx#1L                      %cathode left
 \put(7,3){\line(0,-1){6}}
 \put(13,3){\line(-2,-1){6}}
 \put(13,-3){\line(-2,1){6}}
 \put(13,3){\line(0,-1){6}}
 \put(0,0){\line(1,0){20}}
    \ifx#2C                %varicap
 \multiput(6,4)(3,0){2}{\line(1,0){2}}
 \multiput(8,2.5)(1,0){2}{\line(0,1){3}}
    \else \ifx#2Z                %zener diode
 \put(7,-3){\line(1,0){2}}
    \else \ifx#2E                %esaki diode
 \put(7,-3){\line(1,0){2}}
 \put(7,3){\line(1,0){2}}
    \else \ifx#2S                %shottky diode
 \put(7,-3){\line(-1,0){1}}
 \put(6,-3){\line(0,1){1}}
 \put(7,3){\line(1,0){1}}
 \put(8,3){\line(0,-1){1}}
    \else \ifx#2D                %led diode
 \multiput(10,3)(-2,2){2}{\vector(1,1){4}}
    \else \ifx#2L                %laser diode
 \multiput(10,4)(-2,2){2}{\vector(1,1){4}}
 \put(11,3){\line(-1,1){4}}
    \else \ifx#2P                  %photo diode
 \multiput(13,10)(2,-2){2}{\vector(-1,-1){4}}
   \fi \fi  \fi  \fi  \fi \fi \fi \fi
}
%vertical diode
\newcommand{\vdiode}[2]{
\ifx#1U                  %cathode up
 \put(0,0){\line(0,1){20}}
 \put(3,7){\line(-1,0){6}}
 \put(3,7){\line(-1,2){3}}
 \put(-3,7){\line(1,2){3}}
 \put(3,13){\line(-1,0){6}}
    \ifx#2C                 %varicap
  \multiput(6,8)(0,3){2}{\line(0,1){2}}
  \multiput(4.5,10)(0,1){2}{\line(1,0){3}}
    \else \ifx#2Z                %zener diode
  \put(3,13){\line(0,-1){2}}
     \else \ifx#2E                %esaki diode
  \put(-3,13){\line(0,-1){2}}
  \put(3,13){\line(0,-1){2}}
    \else \ifx#2S                %shottky diode
  \put(-3,13){\line(0,1){1}}
  \put(-3,14){\line(1,0){1}}
  \put(3,13){\line(0,-1){1}}
  \put(3,12){\line(-1,0){1}}
    \else \ifx#2D                %led diode
  \multiput(3,10)(2,2){2}{\vector(1,-1){4}}
    \else \ifx#2L                %laser diode
  \multiput(4,10)(2,2){2}{\vector(1,-1){4}}
  \put(3,9){\line(1,1){4}}
    \else \ifx#2P                       %photo diode
  \multiput(10,7)(-2,-2){2}{\vector(-1,1){4}}
   \fi \fi  \fi  \fi  \fi \fi \fi \fi

\ifx#1D                  %cathode down
 \put(0,0){\line(0,1){20}}
 \put(3,7){\line(-1,0){6}}
 \put(3,13){\line(-1,-2){3}}
 \put(-3,13){\line(1,-2){3}}
 \put(3,13){\line(-1,0){6}}
    \ifx#2C                 %varicap
  \multiput(6,8)(0,3){2}{\line(0,1){2}}
  \multiput(4.5,10)(0,1){2}{\line(1,0){3}}
    \else \ifx#2Z                %zener diode
  \put(3,7){\line(0,1){2}}
     \else \ifx#2E                %esaki diode
  \put(-3,7){\line(0,1){2}}
  \put(3,7){\line(0,1){2}}
    \else \ifx#2S                %shottky diode
  \put(-3,7){\line(0,1){1}}
  \put(-3,8){\line(1,0){1}}
  \put(3,7){\line(0,-1){1}}
  \put(3,6){\line(-1,0){1}}
    \else \ifx#2D                %led diode
  \multiput(3,10)(2,-2){2}{\vector(1,1){4}}
    \else \ifx#2L                %laser diode
  \multiput(4,10)(2,-2){2}{\vector(1,1){4}}
  \put(3,11){\line(1,-1){4}}
    \else \ifx#2P                       %photo diode
  \multiput(7,14)(2,-2){2}{\vector(-1,-1){4}}
   \fi \fi  \fi  \fi  \fi \fi \fi \fi

}
%vertical triac
\newcommand{\vtriac}[1]{
 \put(0,0){\line(0,1){7}}
 \put(6,7){\line(-1,0){6}}
 \put(6,7){\line(-1,2){3}}
 \put(0,7){\line(1,2){3}}
 \put(6,13){\line(-1,0){6}}
 \put(0,13){\line(0,1){7}}
 \put(0,7){\line(-1,0){6}}
 \put(0,13){\line(-1,-2){3}}
 \put(-6,13){\line(1,-2){3}}
 \put(-6,13){\line(1,0){7}}
     \ifx#1A                         %gate down left
   \put(0,7){\line(-1,-1){5}}
   \put(-5,2){\line(-1,0){5}}
     \fi  \ifx#1B                    %gate up left
  \put(0,13){\line(-1,1){4}}
  \put(-4,17){\line(-1,0){6}}
    \fi   \ifx#1C                      %gate down right
  \put(0,7){\line(1,-1){5}}
  \put(5,2){\line(1,0){5}}
    \fi   \ifx#1D                       %gate up right
  \put(0,13){\line(1,1){4}}
  \put(4,17){\line(1,0){6}}
    \fi
}
%horizontal triac
\newcommand{\htriac}[1]{
 \put(0,0){\line(1,0){7}}
 \put(7,0){\line(0,-1){6}}
 \put(7,0){\line(2,-1){6}}
 \put(7,-6){\line(2,1){6}}
 \put(13,0){\line(0,-1){6}}
 \put(13,0){\line(1,0){7}}
 \put(7,0){\line(0,1){6}}
 \put(13,6){\line(-2,-1){6}}
 \put(13,0){\line(-2,1){6}}
 \put(13,0){\line(0,1){6}}
    \ifx#1A                    %gate down left
  \put(7,0){\line(-1,-1){5}}
  \put(2,-5){\line(0,-1){5}}
    \fi \ifx#1C                %gate down right
  \put(13,0){\line(1,-1){4}}
  \put(17,-4){\line(0,-1){6}}
    \fi \ifx#1D                        %gate up right
  \put(13,0){\line(1,1){4}}
  \put(17,4){\line(0,1){6}}
    \fi \ifx#1B                        %gate up left
  \put(7,0){\line(-1,1){4}}
  \put(3,4){\line(0,1){6}}
    \fi
}
%diagonal thyristor
\newcommand{\dthyristor}[3]{
 \ifx#1D                           %down oriented
   \put(0,0){\line(1,-1){8}}
   \put(10.5,-5.5){\line(-1,-1){5}}
      \ifx#2D                      %cathode down
   \put(10.5,-5.5){\line(1,-3){2.5}}
   \put(5.5,-10.5){\line(3,-1){7}}
      \fi \ifx#2U                  %cathode up
   \put(10.5,-15.5){\line(-1,3){2.5}}
   \put(15.5,-10.5){\line(-3,1){7}}
           \fi
   \put(15.5,-10.5){\line(-1,-1){5}}
   \put(13,-13){\line(1,-1){7}}

     \ifx#3A                           %gate down left
       \put(13,-13){\line(0,-1){4}}
       \put(13,-17){\line(-1,-1){4}}
          \fi   \ifx#3B                     %gate up left
       \put(8,-8){\line(-1,0){6}}
       \put(2,-8){\line(-1,-1){4}}
          \fi
     \ifx#3C                           %gate down right
       \put(13,-13){\line(1,0){4}}
       \put(17,-13){\line(1,1){4}}
         \fi   \ifx#3D                     %gate up right
       \put(8,-8){\line(0,1){6}}
       \put(8,-2){\line(1,1){4}}
\fi  \fi


\ifx#1U                            %up oriented
\put(0,0){\line(1,1){8}}
\put(10.5,5.5){\line(-1,1){5}}
      \ifx#2U                      %cathode up
\put(10.5,5.5){\line(1,3){2.5}}
\put(5.5,10.5){\line(3,1){7}}
      \fi \ifx#2D                  %cathode down
\put(10.5,15.5){\line(-1,-3){2.5}}
\put(15.5,10.5){\line(-3,-1){7}}
           \fi
\put(15.5,10.5){\line(-1,1){5}}
\put(13,13){\line(1,1){7}}

  \ifx#3A                          %gate down left
\put(8,8){\line(-1,0){6}}
\put(2,8){\line(-1,1){4}}
  \fi  \ifx#3B                      %gate up left
\put(13,13){\line(0,1){4}}
\put(13,17){\line(-1,1){4}}
 \fi  \ifx#3C                      %gate  down right
\put(8,8){\line(0,-1){6}}
\put(8,2){\line(1,-1){4}}
 \fi  \ifx#3D                       %gate up right
\put(13,13){\line(1,0){4}}
\put(17,13){\line(1,-1){4}}
 \fi \fi
}

%vertical thyristor
\newcommand{\vthyristor}[2]{
 \ifx#2A                          %gate down left
\put(0,7){\line(-1,-1){5}}
\put(-5,2){\line(-1,0){5}}
 \fi \ifx#2B                      %gate up left
\put(0,13){\line(-1,1){4}}
\put(-4,17){\line(-1,0){6}}
 \fi  \ifx#2C                   %gate down right
\put(0,7){\line(1,-1){5}}
\put(5,2){\line(1,0){5}}
 \fi \ifx#2D                   %gate up right
\put(0,13){\line(1,1){4}}
\put(4,17){\line(1,0){6}}
 \fi
   \ifx#1U                 %cathode up
\put(3,7){\line(-1,0){6}}
\put(3,7){\line(-1,2){3}}
\put(-3,7){\line(1,2){3}}
\put(3,13){\line(-1,0){6}}
   \fi \ifx#1D              %cathode down
\put(-3,7){\line(1,0){6}}
\put(3,13){\line(-1,-2){3}}
\put(-3,13){\line(1,-2){3}}
\put(-3,13){\line(1,0){6}}
  \fi
\put(0,0){\line(0,1){20}}
}
%horizontal thyristor
\newcommand{\hthyristor}[2]{
 \ifx#2A                           %gate down left
\put(7,0){\line(-1,-1){5}}
\put(2,-5){\line(0,-1){5}}
 \fi \ifx#2C                      %gate down right
\put(13,0){\line(1,-1){4}}
\put(17,-4){\line(0,-1){6}}
 \fi  \ifx#2B                      %gate up left
\put(7,0){\line(-1,1){5}}
\put(2,5){\line(0,1){5}}
 \fi \ifx#2D                        %gate up right
\put(13,0){\line(1,1){4}}
\put(17,4){\line(0,1){6}}
 \fi
   \ifx#1R                     %cathode right
\put(7,3){\line(0,-1){6}}
\put(7,3){\line(2,-1){6}}
\put(7,-3){\line(2,1){6}}
\put(13,3){\line(0,-1){6}}
   \fi \ifx#1L                 %cathode left
\put(7,-3){\line(0,1){6}}
\put(13,3){\line(-2,-1){6}}
\put(13,-3){\line(-2,1){6}}
\put(13,-3){\line(0,1){6}}
  \fi
\put(0,0){\line(1,0){20}}
}
%horizontal diac
\newcommand{\hdiac}{
\put(0,0){\line(1,0){7}}
\put(7,0){\line(0,-1){6}}
\put(7,0){\line(2,-1){6}}
\put(7,-6){\line(2,1){6}}
\put(13,0){\line(0,-1){6}}
\put(13,0){\line(1,0){7}}
\put(7,0){\line(0,1){6}}
\put(13,6){\line(-2,-1){6}}
\put(13,0){\line(-2,1){6}}
\put(13,0){\line(0,1){6}}
}
%vertical diac
\newcommand{\vdiac}{
\put(0,0){\line(0,1){7}}
\put(6,7){\line(-1,0){6}}
\put(6,7){\line(-1,2){3}}
\put(0,7){\line(1,2){3}}
\put(6,13){\line(-1,0){6}}
\put(0,13){\line(0,1){7}}
\put(0,7){\line(-1,0){6}}
\put(0,13){\line(-1,-2){3}}
\put(-6,13){\line(1,-2){3}}
\put(-6,13){\line(1,0){6}}
}
%optocoupler
\newcommand{\opto}[1]{
 \ifx#1R
\put(0,0){\line(0,1){20}}
\put(-2,8){\line(1,0){4}}
\put(0,8){\line(-1,2){2}}
\put(0,8){\line(1,2){2}}
\put(-2,12){\line(1,0){4}}
\multiput(3,9)(0,2){2}{\vector(1,0){4}}
\put(11,8){\line(0,1){4}}
\put(15,0){\line(0,1){7.5}}
\put(15,12.5){\line(0,1){7.5}}
\put(11,10.5){\line(2,1){4}}
\put(11,9.5){\vector(2,-1){4}}
\multiput(-5,5)(25,0){2}{\line(0,1){10}}
\multiput(-5,5)(0,10){2}{\line(1,0){25}}
 \fi  \ifx#1L
\put(0,0){\line(0,1){20}}
\put(2,8){\line(-1,0){4}}
\put(0,8){\line(1,2){2}}
\put(0,8){\line(-1,2){2}}
\put(2,12){\line(-1,0){4}}
\put(0,12){\line(0,1){8}}
\multiput(-3,9)(0,2){2}{\vector(-1,0){4}}
\put(-11,8){\line(0,1){4}}
\put(-15,0){\line(0,1){7.5}}
\put(-15,12.5){\line(0,1){7.5}}
\put(-11,10.5){\line(-2,1){4}}
\put(-11,9.5){\vector(-2,-1){4}}
\multiput(5,5)(-25,0){2}{\line(0,1){10}}
\multiput(5,5)(0,10){2}{\line(-1,0){25}}
 \fi
}

%operational amplifier
\newcommand{\opamp}[2]{
  \ifx#1R                      %right oriented
\multiput(0,0)(0,10){2}{\line(1,0){5}}
\put(25,5){\line(1,0){5}}
\put(5,-5){\line(0,1){20}}
\put(5,-5){\line(2,1){20}}
\put(5,15){\line(2,-1){20}}
  \ifx#2D                       %-input down
\put(7,0){\line(1,0){3}}
\put(7,7){\makebox(5,5)[l]{+}}
  \fi  \ifx#2U                  %-input up
\put(7,10){\line(1,0){3}}
\put(7,-2){\makebox(5,5)[l]{+}}
  \fi
   \fi  \ifx#1L                  %left oriented
\multiput(0,0)(0,10){2}{\line(-1,0){5}}
\put(-25,5){\line(-1,0){5}}
\put(-5,-5){\line(0,1){20}}
\put(-5,-5){\line(-2,1){20}}
\put(-5,15){\line(-2,-1){20}}
  \ifx#2D                        %-input down
\put(-7,0){\line(-1,0){3}}
\put(-12,7){\makebox(5,5)[r]{+}}
  \fi  \ifx#2U                   %-input up
\put(-7,10){\line(-1,0){3}}
\put(-12,-2){\makebox(5,5)[r]{+}}
  \fi  \fi  \ifx#1U                  %up oriented
\multiput(0,0)(10,0){2}{\line(0,1){5}}
\put(5,25){\line(0,1){5}}
\put(-5,5){\line(1,0){20}}
\put(-5,5){\line(1,2){10}}
\put(15,5){\line(-1,2){10}}
  \ifx#2L                        %-input left
\put(2,8){\line(-1,0){3}}
\put(6,6){\makebox(5,5)[r]{+}}
  \fi  \ifx#2R                   %-input right
\put(11,8){\line(-1,0){3}}
\put(-3,6){\makebox(5,5)[r]{+}}
  \fi   \fi \ifx#1D                  %down oriented
\multiput(0,0)(10,0){2}{\line(0,-1){5}}
\put(5,-25){\line(0,-1){5}}
\put(-5,-5){\line(1,0){20}}
\put(-5,-5){\line(1,-2){10}}
\put(15,-5){\line(-1,-2){10}}
  \ifx#2L                        %-input left
\put(2,-8){\line(-1,0){3}}
\put(6,-11){\makebox(5,5)[r]{+}}
  \fi  \ifx#2R                   %-input right
\put(11,-8){\line(-1,0){3}}
\put(-3,-11){\makebox(5,5)[r]{+}}
  \fi
   \fi
}

%operational transconductance amplifier
\newcommand{\ota}[3]{
  \ifx#1R                      %right oriented
\multiput(0,0)(0,10){2}{\line(1,0){5}}
\put(15,5){\line(1,0){5}}
\put(5,-5){\line(0,1){20}}
\put(15,0){\line(0,1){10}}
\put(5,-5){\line(2,1){10}}
\put(5,15){\line(2,-1){10}}
     \ifx#3Y              %with linearization diodes
  \put(3.5,6.5){$\bigtriangleup$ }
  \put(3.5,2){$\bigtriangledown$ }
  \multiput(3.5,1.5)(0,7.25){2}{\line(1,0){3}}
  \put(0,5){\line(1,0){5}}
     \fi
  \ifx#2D                       %-input down
\put(7,0){\line(1,0){3}}
\put(7,7){\makebox(5,5)[l]{+}}
  \fi  \ifx#2U                  %-input up
\put(7,10){\line(1,0){3}}
\put(7,-2){\makebox(5,5)[l]{+}}
  \fi
   \fi  \ifx#1L                  %left oriented
\multiput(0,0)(0,10){2}{\line(-1,0){5}}
\put(-15,5){\line(-1,0){5}}
\put(-5,-5){\line(0,1){20}}
\put(-15,0){\line(0,1){10}}
\put(-5,-5){\line(-2,1){10}}
\put(-5,15){\line(-2,-1){10}}
     \ifx#3Y              %with linearization diodes
  \put(-6.5,6.5){$\bigtriangleup$ }
  \put(-6.5,2){$\bigtriangledown$ }
  \multiput(-6.5,1.5)(0,7.25){2}{\line(1,0){3}}
  \put(0,5){\line(-1,0){5}}
     \fi
  \ifx#2D                        %-input down
\put(-7,0){\line(-1,0){3}}
\put(-12,7){\makebox(5,5)[r]{+}}
  \fi  \ifx#2U                   %-input up
\put(-7,10){\line(-1,0){3}}
\put(-12,-2){\makebox(5,5)[r]{+}}
  \fi  \fi  \ifx#1U                  %up oriented
\multiput(0,0)(10,0){2}{\line(0,1){5}}
\put(5,15){\line(0,1){5}}
\put(0,15){\line(1,0){10}}
\put(-5,5){\line(1,0){20}}
\put(-5,5){\line(1,2){5}}
\put(15,5){\line(-1,2){5}}
     \ifx#3Y              %with linearization diodes
  \put(7.2,4.2){$\triangleright$ }
  \put(1.4,4.2){$\triangleleft$ }
  \multiput(1.5,3.5)(7.25,0){2}{\line(0,1){3}}
  \put(5,0){\line(0,1){5}}
     \fi
  \ifx#2L                        %-input left
\put(2,8){\line(-1,0){3}}
\put(6,6){\makebox(5,5)[r]{+}}
  \fi  \ifx#2R                   %-input right
\put(11,8){\line(-1,0){3}}
\put(-3,6){\makebox(5,5)[r]{+}}
  \fi   \fi \ifx#1D                  %down oriented
\multiput(0,0)(10,0){2}{\line(0,-1){5}}
\put(5,-15){\line(0,-1){5}}
\put(0,-15){\line(1,0){10}}
\put(-5,-5){\line(1,0){20}}
\put(-5,-5){\line(1,-2){5}}
\put(15,-5){\line(-1,-2){5}}
     \ifx#3Y              %with linearization diodes
  \put(7.2,-5.8){$\triangleright$ }
  \put(1.4,-5.8){$\triangleleft$ }
  \multiput(1.5,-3.5)(7.25,0){2}{\line(0,-1){3}}
  \put(5,0){\line(0,-1){5}}
     \fi
  \ifx#2L                        %-input left
\put(2,-8){\line(-1,0){3}}
\put(6,-11){\makebox(5,5)[r]{+}}
  \fi  \ifx#2R                   %-input right
\put(11,-8){\line(-1,0){3}}
\put(-3,-11){\makebox(5,5)[r]{+}}
  \fi
   \fi
}

% balanced operational transconductance amplifier
\newcommand{\bota}[3]{
  \ifx#1R                      %right oriented
\multiput(0,0)(0,10){2}{\line(1,0){5}}
\multiput(15,2.5)(0,5){2}{\line(1,0){5}}
\put(5,-5){\line(0,1){20}}
\put(15,0){\line(0,1){10}}
\put(5,-5){\line(2,1){10}}
\put(5,15){\line(2,-1){10}}
     \ifx#3Y              %with linearization diodes
  \put(3.5,6.5){$\bigtriangleup$ }
  \put(3.5,2){$\bigtriangledown$ }
  \multiput(3.5,1.5)(0,7.25){2}{\line(1,0){3}}
  \put(0,5){\line(1,0){5}}
     \fi
  \ifx#2D                       %-input down
\put(7,0){\line(1,0){3}}
\put(7,7){\makebox(5,5)[l]{+}}
  \fi  \ifx#2U                  %-input up
\put(7,10){\line(1,0){3}}
\put(7,-2){\makebox(5,5)[l]{+}}
  \fi
   \fi  \ifx#1L                  %left oriented
\multiput(0,0)(0,10){2}{\line(-1,0){5}}
\multiput(-15,2.5)(0,5){2}{\line(-1,0){5}}
\put(-5,-5){\line(0,1){20}}
\put(-15,0){\line(0,1){10}}
\put(-5,-5){\line(-2,1){10}}
\put(-5,15){\line(-2,-1){10}}
     \ifx#3Y              %with linearization diodes
  \put(-6.5,6.5){$\bigtriangleup$ }
  \put(-6.5,2){$\bigtriangledown$ }
  \multiput(-6.5,1.5)(0,7.25){2}{\line(1,0){3}}
  \put(0,5){\line(-1,0){5}}
     \fi
  \ifx#2D                        %-input down
\put(-7,0){\line(-1,0){3}}
\put(-12,7){\makebox(5,5)[r]{+}}
  \fi  \ifx#2U                   %-input up
\put(-7,10){\line(-1,0){3}}
\put(-12,-2){\makebox(5,5)[r]{+}}
  \fi  \fi  \ifx#1U                  %up oriented
\multiput(0,0)(10,0){2}{\line(0,1){5}}
\multiput(2.5,15)(5,0){2}{\line(0,1){5}}
\put(0,15){\line(1,0){10}}
\put(-5,5){\line(1,0){20}}
\put(-5,5){\line(1,2){5}}
\put(15,5){\line(-1,2){5}}
     \ifx#3Y              %with linearization diodes
  \put(7.2,4.2){$\triangleright$ }
  \put(1.4,4.2){$\triangleleft$ }
  \multiput(1.5,3.5)(7.25,0){2}{\line(0,1){3}}
  \put(5,0){\line(0,1){5}}
     \fi
  \ifx#2L                        %-input left
\put(2,8){\line(-1,0){3}}
\put(6,6){\makebox(5,5)[r]{+}}
  \fi  \ifx#2R                   %-input right
\put(11,8){\line(-1,0){3}}
\put(-3,6){\makebox(5,5)[r]{+}}
  \fi   \fi \ifx#1D                  %down oriented
\multiput(0,0)(10,0){2}{\line(0,-1){5}}
\multiput(2.5,-15)(5,0){2}{\line(0,-1){5}}
\put(0,-15){\line(1,0){10}}
\put(-5,-5){\line(1,0){20}}
\put(-5,-5){\line(1,-2){5}}
\put(15,-5){\line(-1,-2){5}}
     \ifx#3Y              %with linearization diodes
  \put(7.2,-5.8){$\triangleright$ }
  \put(1.4,-5.8){$\triangleleft$ }
  \multiput(1.5,-3.5)(7.25,0){2}{\line(0,-1){3}}
  \put(5,0){\line(0,-1){5}}
     \fi
  \ifx#2L                        %-input left
\put(2,-8){\line(-1,0){3}}
\put(6,-11){\makebox(5,5)[r]{+}}
  \fi  \ifx#2R                   %-input right
\put(11,-8){\line(-1,0){3}}
\put(-3,-11){\makebox(5,5)[r]{+}}
  \fi
   \fi
}
%%%%%%tranzistors bjt, jfet, mosfet,%%%%%%%%%%%%
%bipolar transistor
\newcommand{\bjt}[4]{
 \ifx#2R                       %right oriented
   \put(0,0){\line(1,0){10}}
   \put(10,3){\line(0,-1){6}}
   \put(10,1){\line(1,1){9}}
   \put(10,-1){\line(1,-1){9}}
          \ifx#4N
            \else
               \put(11,0){\circle{10}}
          \fi
      \ifx#3D                     %emitor down
           \ifx#1N                  %npn type
         \put(10,-1){\vector(1,-1){3}}
           \else \ifx#1P            %pnp type
         \put(13,-4){\vector(-1,1){3}}
           \fi \fi
     \else \ifx#3U              %emitor up
           \ifx#1N              %npn type
         \put(10,1){\vector(1,1){3}}
           \else \ifx#1P        %pnp type
         \put(13,4){\vector(-1,-1){3}}
           \fi \fi
     \fi \fi
\fi
\ifx#2L                         %left oriented
   \put(0,0){\line(-1,0){10}}
   \put(-10,1){\line(-1,1){9}}
   \put(-10,-1){\line(-1,-1){9}}
   \put(-10,3){\line(0,-1){6}}
             \ifx#4N
                \else
                    \put(-11,0){\circle{10}}
             \fi
       \ifx#3D                         %emitor down
           \ifx#1N                      %npn type
         \put(-10,-1){\vector(-1,-1){3}}
           \else \ifx#1P                %pnp type
         \put(-13,-4){\vector(1,1){3}}
           \fi \fi
      \else \ifx#3U                   %emitor up
           \ifx#1N                      %npn type
         \put(-10,1){\vector(-1,1){3}}
           \else \ifx#1P                %pnp type
         \put(-13,4){\vector(1,-1){3}}
           \fi \fi
      \fi \fi
\fi
\ifx#2U                            %up oriented
   \put(0,0){\line(0,1){10}}
   \put(-3,10){\line(1,0){6}}
   \put(1,10){\line(1,1){9}}
   \put(-1,10){\line(-1,1){9}}
             \ifx#4N
                \else
                   \put(0,11){\circle{10}}
             \fi
     \ifx#3R                        %emitor right
          \ifx#1N                       %npn type
       \put(1,10){\vector(1,1){3}}
          \else \ifx#1P                %pnp type
       \put(4,13){\vector(-1,-1){3}}
          \fi \fi
     \else \ifx#3L                 %emitor left
          \ifx#1N                      %npn type
        \put(-1,10){\vector(-1,1){3}}
          \else \ifx#1P                %pnp type
        \put(-4,13){\vector(1,-1){3}}
          \fi \fi
     \fi \fi
\fi
\ifx#2D                         %down oriented
   \put(0,0){\line(0,-1){10}}
   \put(-3,-10){\line(1,0){6}}
   \put(1,-10){\line(1,-1){9}}
   \put(-1,-10){\line(-1,-1){9}}
              \ifx#4N
                 \else
                     \put(0,-11){\circle{10}}
              \fi
       \ifx#3R                       %emitor right
            \ifx#1N                     %npn type
         \put(1,-10){\vector(1,-1){3}}
            \else \ifx#1P               %pnp type
         \put(4,-13){\vector(-1,1){3}}
            \fi \fi
       \else \ifx#3L                 %emitor left
            \ifx#1N                     %npn type
         \put(-1,-10){\vector(-1,-1){3}}
            \else \ifx#1P               %pnp type
         \put(-4,-13){\vector(1,1){3}}
            \fi \fi
       \fi \fi
\fi
}
%jfet transistor
\newcommand{\jfet}[4]{
\ifx#2R                    %right oriented
 \ifx#3D                   %source down
   \ifx#1N                  %n channel
      \put(12,0){\vector(1,0){0}}
    \fi  \ifx#1P             %p channel
      \put(8,0){\vector(-1,0){0}}
         \fi
              \ifx#4N
                \else
                  \put(12,2.5){\circle{10}}
              \fi
   \put(12,5){\line(1,0){8}}
   \put(0,0){\line(1,0){20}}
   \put(12,6){\line(0,-1){7}}
 \fi \ifx#3U                  %source up
         \ifx#1N                  %n channel
           \put(12,0){\vector(1,0){0}}
         \fi \ifx#1P             %p channel
           \put(8,0){\vector(-1,0){0}}
             \fi
                \ifx#4N
                   \else
                     \put(12,-2.5){\circle{10}}
                \fi
      \put(12,-5){\line(1,0){8}}
      \put(0,0){\line(1,0){20}}
      \put(12,-6){\line(0,1){7}}
    \fi
\fi \ifx#2L                  %left oriented
      \ifx#3U                     %source up
            \ifx#1N                  %n channel
          \put(-12,0){\vector(-1,0){0}}
             \fi \ifx#1P             %p channel
          \put(-8,0){\vector(1,0){0}}
                 \fi
                     \ifx#4N
                       \else
                          \put(-12,-2.5){\circle{10}}
                     \fi
        \put(-12,-5){\line(-1,0){8}}
        \put(0,0){\line(-1,0){20}}
        \put(-12,1){\line(0,-1){7}}
      \fi \ifx#3D                     %source down
                \ifx#1N                  %n channel
              \put(-12,0){\vector(-1,0){0}}
                \fi \ifx#1P             %p channel
              \put(-8,0){\vector(1,0){0}}
                \fi
                  \ifx#4N
                     \else
                         \put(-12,2.5){\circle{10}}
                  \fi
            \put(-12,5){\line(-1,0){8}}
            \put(0,0){\line(-1,0){20}}
            \put(-12,6){\line(0,-1){7}}
          \fi \fi

\ifx#2U                    %up oriented
          \ifx#3L                   %source left
                 \ifx#1N                  %n channel
               \put(0,12){\vector(0,1){0}}
                 \fi \ifx#1P             %p channel
               \put(0,8){\vector(0,-1){0}}
                     \fi
                        \ifx#4N
                           \else
                              \put(2.5,12){\circle{10}}
                        \fi
             \put(5,12){\line(0,1){8}}
             \put(0,0){\line(0,1){20}}
             \put(6,12){\line(-1,0){7}}
          \fi \ifx#3R                  %source right
                   \ifx#1N                  %n channel
                 \put(0,12){\vector(0,1){0}}
                   \fi \ifx#1P             %p channel
                 \put(0,8){\vector(0,-1){0}}
                       \fi
                          \ifx#4N
                             \else
                                \put(-2.5,12){\circle{10}}
                          \fi
                \put(-5,12){\line(0,1){8}}
                \put(0,0){\line(0,1){20}}
                \put(1,12){\line(-1,0){7}}
              \fi

\fi \ifx#2D                  %down oriented
           \ifx#3R                     %source right
                  \ifx#1N                  %n channel
                \put(0,-12){\vector(0,-1){0}}
                  \fi \ifx#1P             %p channel
                \put(0,-8){\vector(0,1){0}}
                  \fi
                      \ifx#4N
                         \else
                            \put(-2.5,-12){\circle{10}}
                      \fi
             \put(-5,-12){\line(0,-1){8}}
             \put(0,0){\line(0,-1){20}}
             \put(1,-12){\line(-1,0){7}}
           \fi \ifx#3L                     %source left
                    \ifx#1N                  %n channel
                  \put(0,0){\vector(0,-1){11}}
                    \fi \ifx#1P             %p channel
                  \put(0,-11.5){\vector(0,1){4}}
                         \fi
                            \ifx#4N
                               \else
                                   \put(2.5,-12){\circle{10}}
                            \fi
                \put(5,-12){\line(0,-1){8}}
                \put(0,0){\line(0,-1){20}}
                \put(6,-12){\line(-1,0){7}}
               \fi \fi
}
%mos transistor
\newcommand{\mos}[6]{
 \ifx#2R          %right oriented
   \ifx#3D      %source down
      \ifnum#5=1  % one gate
       \put(0,0){\line(1,0){10}}
       \put(10,5){\line(0,-1){5}}
      \fi
      \ifnum#5=2  % two gates
       \multiput(0,0)(0,3.5){2}{\line(1,0){10}}
       \multiput(10,1.5)(0,3.5){2}{\line(0,-1){1.5}}
      \fi
        \ifx#6N
          \else
             \put(12,2.5){\circle{10}}
        \fi
    \put(11,5){\line(1,0){9}}
    \put(11,0){\line(1,0){9}}
          \ifx#4E                     %enhancement mode
        \multiput(11,0.75)(0,2.5){3}{\put(0,0){\line(0,-1){1.5}}}
          \else \ifx#4D               %depletion mode
        \put(11,6){\line(0,-1){7}}
          \fi \fi
       \ifx#1P                    %p type
     \put(11,2.5){\vector(1,0){4}}
       \else \ifx#1N              %n type
     \put(15,2.5){\vector(-1,0){4}}
       \fi \fi

   \else \ifx#3U   %source up
       \ifnum#5=1  % one gate
        \put(0,0){\line(1,0){10}}
        \put(10,0){\line(0,-1){5}}
       \fi
       \ifnum#5=2  % two gates
       \multiput(0,0)(0,-3.5){2}{\line(1,0){10}}
       \multiput(10,0)(0,-3.5){2}{\line(0,-1){1.5}}
      \fi
        \ifx#6N
          \else
            \put(12,-2.5){\circle{10}}
        \fi
     \put(11,-5){\line(1,0){9}}
     \put(11,0){\line(1,0){9}}
          \ifx#4E                     %enhancement mode
        \multiput(11,0.75)(0,-2.5){3}{\put(0,0){\line(0,-1){1.5}}}
          \else \ifx#4D               %depletion mode
        \put(11,-6){\line(0,1){7}}
          \fi \fi
      \ifx#1P                    %p type
    \put(11,-2.5){\vector(1,0){4}}
      \else \ifx#1N              %n type
    \put(15,-2.5){\vector(-1,0){4}}
       \fi \fi
   \fi \fi
 \fi

  \ifx#2L          %left oriented
       \ifx#3D      %source down
            \ifnum#5=1 % one gate
               \put(0,0){\line(-1,0){10}}
               \put(-10,5){\line(0,-1){5}}
            \fi
            \ifnum#5=2  % two gates
               \multiput(0,0)(0,3.5){2}{\line(-1,0){10}}
               \multiput(-10,5)(0,-3.5){2}{\line(0,-1){1.5}}
            \fi
               \ifx#6N
                 \else
                   \put(-12,2.5){\circle{10}}
               \fi
          \put(-11,5){\line(-1,0){9}}
          \put(-11,0){\line(-1,0){9}}
               \ifx#4E                     %enhancement mode
           \multiput(-11,0.75)(0,2.5){3}{\put(0,0){\line(0,-1){1.5}}}
               \else \ifx#4D               %depletion mode
           \put(-11,6){\line(0,-1){7}}
               \fi \fi
             \ifx#1P                    %p type
            \put(-11,2.5){\vector(-1,0){4}}
             \else \ifx#1N              %n type
            \put(-15,2.5){\vector(1,0){4}}
             \fi \fi

      \else \ifx#3U   %source up
          \ifnum#5=1
            \put(0,0){\line(-1,0){10}}
            \put(-10,0){\line(0,-1){5}}
          \fi
          \ifnum#5=2  % two gates
            \multiput(0,0)(0,-3.5){2}{\line(-1,0){10}}
            \multiput(-10,0)(0,-3.5){2}{\line(0,-1){1.5}}
          \fi
             \ifx#6N
               \else
                  \put(-12,-2.5){\circle{10}}
             \fi
    \put(-11,-5){\line(-1,0){9}}
    \put(-11,0){\line(-1,0){9}}
          \ifx#4E                     %enhancement mode
        \multiput(-11,0.75)(0,-2.5){3}{\put(0,0){\line(0,-1){1.5}}}
          \else \ifx#4D               %depletion mode
        \put(-11,-6){\line(0,1){7}}
          \fi \fi
       \ifx#1P                    %p type
      \put(-11,-2.5){\vector(-1,0){4}}
       \else \ifx#1N              %n type
      \put(-15,-2.5){\vector(1,0){4}}
       \fi \fi
      \fi \fi
 \fi

  \ifx#2U          %up oriented
       \ifx#3L      %source left
          \ifnum#5=1 % one gate
            \put(0,0){\line(0,1){10}}
            \put(5,10){\line(-1,0){5}}
          \fi
          \ifnum#5=2  % two gates
            \multiput(0,0)(3.5,0){2}{\line(0,1){10}}
            \multiput(5,10)(-3.5,0){2}{\line(-1,0){1.5}}
          \fi
            \ifx#6N
              \else
                 \put(2.5,12){\circle{10}}
            \fi
         \put(5,11){\line(0,1){9}}
         \put(0,11){\line(0,1){9}}
               \ifx#4E                     %enhancement mode
            \multiput(0.75,11)(2.5,0){3}{\put(0,0){\line(-1,0){1.5}}}
               \else \ifx#4D               %depletion mode
            \put(6,11){\line(-1,0){7}}
               \fi \fi
           \ifx#1P                    %p type
          \put(2.5,11){\vector(0,1){4}}
           \else \ifx#1N              %n type
          \put(2.5,15){\vector(0,-1){4}}
           \fi \fi

       \else \ifx#3R   %source right
                \ifnum#5=1  % one gate
                  \put(0,0){\line(0,1){10}}
                  \put(0,10){\line(-1,0){5}}
                \fi
                \ifnum#5=2  % two gates
                  \multiput(0,0)(-3.5,0){2}{\line(0,1){10}}
                  \multiput(0,10)(-3.5,0){2}{\line(-1,0){1.5}}
                \fi
                    \ifx#6N
                      \else
                         \put(-2.5,12){\circle{10}}
                    \fi
              \put(-5,11){\line(0,1){9}}
              \put(0,11){\line(0,1){9}}
                   \ifx#4E                     %enhancement mode
            \multiput(0.75,11)(-2.5,0){3}{\put(0,0){\line(-1,0){1.5}}}
                   \else \ifx#4D               %depletion mode
            \put(-6,11){\line(1,0){7}}
                   \fi \fi
          \ifx#1P                    %p type
         \put(-2.5,11){\vector(0,1){4}}
          \else \ifx#1N              %n type
         \put(-2.5,15){\vector(0,-1){4}}
          \fi \fi
       \fi \fi
 \fi

  \ifx#2D          %down oriented
       \ifx#3L      %source left
           \ifnum#5=1  % one gate
            \put(0,0){\line(0,-1){10}}
            \put(5,-10){\line(-1,0){5}}
           \fi
           \ifnum#5=2  % two gates
            \multiput(0,0)(3.5,0){2}{\line(0,-1){10}}
            \multiput(5,-10)(-3.5,0){2}{\line(-1,0){1.5}}
           \fi
               \ifx#6N
                 \else
                    \put(2.5,-12){\circle{10}}
               \fi
         \put(5,-11){\line(0,-1){9}}
         \put(0,-11){\line(0,-1){9}}
             \ifx#4E                     %enhancement mode
            \multiput(0.75,-11)(2.5,0){3}{\put(0,0){\line(-1,0){1.5}}}
             \else \ifx#4D               %depletion mode
            \put(6,-11){\line(-1,0){7}}
             \fi \fi
          \ifx#1P                    %p type
            \put(2.5,-11){\vector(0,-1){4}}
          \else \ifx#1N              %n type
            \put(2.5,-15){\vector(0,1){4}}
          \fi \fi

       \else \ifx#3R   %source right
              \ifnum#5=1 % one gate
                \put(0,0){\line(0,-1){10}}
                \put(0,-10){\line(-1,0){5}}
              \fi
              \ifnum#5=2  % two gates
                \multiput(0,0)(-3.5,0){2}{\line(0,-1){10}}
                \multiput(0,-10)(-3.5,0){2}{\line(-1,0){1.5}}
              \fi
                  \ifx#6N
                    \else
                       \put(-2.5,-12){\circle{10}}
                  \fi
            \put(-5,-11){\line(0,-1){9}}
            \put(0,-11){\line(0,-1){9}}
          \ifx#4E                     %enhancement mode
            \multiput(0.75,-11)(-2.5,0){3}{\put(0,0){\line(-1,0){1.5}}}
          \else \ifx#4D               %depletion mode
            \put(1,-11){\line(-1,0){7}}
          \fi \fi
             \ifx#1P                    %p type
            \put(-2.5,-11){\vector(0,-1){4}}
             \else \ifx#1N              %n type
            \put(-2.5,-15){\vector(0,1){4}}
             \fi \fi
       \fi \fi
 \fi
}
%ujt transistor
\newcommand{\ujt}[4]{
\ifx#2R                    %right oriented
 \ifx#3U                   %B1 up
   \ifx#1P                  %p type emiter
      \put(9,2.5){\vector(4,1){0}}
    \fi  \ifx#1N             %n type emiter
      \put(5.5,1.25){\vector(-4,-1){0}}
         \fi
            \ifx#4N
               \else
                  \put(9,2.5){\circle{10}}
            \fi
   \multiput(9,5)(0,-5){2}{\line(1,0){8}}
   \put(0,0){\line(4,1){9}}
   \put(9,6){\line(0,-1){7}}
 \fi \ifx#3D                  %B1 down
         \ifx#1P                  %p type emiter
           \put(9.4,-2.4){\vector(4,-1){0}}
         \fi \ifx#1N             %n type emitor
           \put(5.5,-1.3){\vector(-4,1){0}}
             \fi
                \ifx#4N
                   \else
                       \put(9,-2.5){\circle{10}}
                \fi
      \multiput(9,-5)(0,5){2}{\line(1,0){8}}
      \put(0,0){\line(4,-1){9}}
      \put(9,-6){\line(0,1){7}}
    \fi
\fi \ifx#2L                  %left oriented
      \ifx#3D                     %B1 down
            \ifx#1P                  %p type emiter
          \put(-9.4,-2.4){\vector(-4,-1){0}}
             \fi \ifx#1N             %n type emiter
          \put(-5.5,-1.3){\vector(4,1){0}}
                 \fi
                     \ifx#4N
                        \else
                           \put(-9,-2.5){\circle{10}}
                     \fi
        \multiput(-9,-5)(0,5){2}{\line(-1,0){8}}
        \put(0,0){\line(-4,-1){9}}
        \put(-9,-6){\line(0,1){7}}
      \fi \ifx#3U                     %B1 up
                \ifx#1P                  %p type emiter
              \put(-9.4,2.4){\vector(-4,1){0}}
                \fi \ifx#1N             %n type emiter
              \put(-5.5,1.3){\vector(4,-1){0}}
                \fi
                   \ifx#4N
                      \else
                         \put(-9,2.5){\circle{10}}
                   \fi
            \multiput(-9,5)(0,-5){2}{\line(-1,0){8}}
            \put(0,0){\line(-4,1){9}}
            \put(-9,6){\line(0,-1){7}}
          \fi \fi

\ifx#2U                    %up oriented
          \ifx#3R                   %B1 right
                 \ifx#1P                  %p type emiter
               \put(2.5,9.4){\vector(1,4){0}}
                 \fi \ifx#1N             %n type emiter
               \put(1.3,5.5){\vector(-1,-4){0}}
                     \fi
                         \ifx#4N
                            \else
                               \put(2.5,9){\circle{10}}
                          \fi
             \multiput(0,9)(5,0){2}{\line(0,1){8}}
             \put(0,0){\line(1,4){2.2}}
             \put(6,9){\line(-1,0){7}}
          \fi \ifx#3L                  %B1 left
                   \ifx#1P                  %p type emiter
                 \put(-2.5,9.4){\vector(-1,4){0}}
                   \fi \ifx#1N             %n type emiter
                 \put(-1.3,5.5){\vector(1,-4){0}}
                       \fi
                          \ifx#4N
                             \else
                               \put(-2.5,9){\circle{10}}
                             \fi
                \multiput(-5.,9)(5,0){2}{\line(0,1){8}}
                \put(0,0){\line(-1,4){2.2}}
                \put(-6,9){\line(1,0){7}}
              \fi

\fi \ifx#2D                  %down oriented
           \ifx#3L                     %B1 left
                  \ifx#1P                  %p type emiter
                \put(-2.5,-9.4){\vector(-1,-4){0}}
                  \fi \ifx#1N             %n type emiter
                \put(-1.3,-5.5){\vector(1,4){0}}
                  \fi
                       \ifx#4N
                          \else
                             \put(-2.5,-9){\circle{10}}
                       \fi
             \multiput(-5,-9)(5,0){2}{\line(0,-1){8}}
             \put(0,0){\line(-1,-4){2.2}}
             \put(-6,-9){\line(1,0){7}}
           \fi \ifx#3R                     %B1 right
                    \ifx#1P                  %p type emiter
                  \put(2.5,-9.4){\vector(1,-4){0}}
                    \fi \ifx#1N             %n type emiter
                  \put(1.3,-5.5){\vector(-1,4){0}}
                         \fi
                            \ifx#4N
                               \else
                                  \put(2.5,-9){\circle{10}}
                            \fi
                \multiput(0,-9)(5,0){2}{\line(0,-1){8}}
                \put(0,0){\line(1,-4){2.2}}
                \put(6,-9){\line(-1,0){7}}
               \fi \fi
}
%%%%% Hall's generator
\newcommand{\hall}{
\put(0,0){\line(1,0){10}}
\put(10,-2){\framebox(10,4){}\put(0,2){\line(1,0){10}}}
\multiput(15,-5)(0,7){2}{\line(0,1){3}}
\put(13.7,-0.8){$\times $}
}
%%%%% horizontal magnistor
\newcommand{\hmag}{
\put(0,0){\line(1,0){10}}
\put(10,-2){\framebox(10,4){}\put(0,2){\line(1,0){10}}}
\put(11,-4){\line(1,1){8}}
\put(8,-4){\line(1,0){3}}
\put(15,-5){$\times $}
}
%vertical magnistor
\newcommand{\vmag}{
\put(0,0){\line(0,1){10}}
\put(-2,10){\framebox(4,10){}\put(-2,10){\line(0,1){10}}}
\put(-6,17){\line(3,-1){12}}
\put(6,10){\line(0,1){3}}
\put(3,17){$\times $}
}
%horizontal thermistor PTC
\newcommand{\hptc}{
\put(0,0){\line(1,0){5}}
\put(5,-2){\framebox(10,4){}\put(0,2){\line(1,0){5}}}
\put(6,-4){\line(1,1){8}}
\put(3,-4){\line(1,0){3}}
\multiput(17,2)(2,0){2}{\vector(0,1){4}}
\put(16,-4){$\vartheta $}
}
%horizontal thermistor NTC
\newcommand{\hntc}{
\put(0,0){\line(1,0){5}}
\put(5,-2){\framebox(10,4){}\put(0,2){\line(1,0){5}}}
\put(6,-4){\line(1,1){8}}
\put(3,-4){\line(1,0){3}}
\put(17,2){\vector(0,1){4}}
\put(19,6){\vector(0,-1){4}}
\put(16,-4){$\vartheta $}
}
%horizontal varistor
\newcommand{\hvar}{
\put(0,0){\line(1,0){5}}
\put(5,-2){\framebox(10,4){}\put(0,2){\line(1,0){5}}}
\put(6,-4){\line(1,1){8}}
\put(3,-4){\line(1,0){3}}
\put(16,-4){$\rm U $}
}
%horizontal photorezistor
\newcommand{\hprez}{
\put(0,0){\line(1,0){5}}
\put(5,-2){\framebox(10,4){}\put(0,2){\line(1,0){5}}}
\multiput(5,8)(2,0){2}{\vector(1,-1){4}}
}
%vertical thermistor PTC
\newcommand{\vptc}{
\put(0,0){\line(0,1){5}}
\put(-2,5){\framebox(4,10){}\put(-2,10){\line(0,1){5}}}
\put(4,6){\line(-1,1){8}}
\put(4,3){\line(0,1){3}}
\multiput(-2,17)(0,2){2}{\vector(-1,0){4}}
\put(4,16){$\vartheta $}
}
%vertical thermistor NTC
\newcommand{\vntc}{
\put(0,0){\line(0,1){5}}
\put(-2,5){\framebox(4,10){}\put(-2,10){\line(0,1){5}}}
\put(4,6){\line(-1,1){8}}
\put(4,3){\line(0,1){3}}
\put(-2,17){\vector(-1,0){4}}
\put(-6,19){\vector(1,0){4}}
\put(4,16){$\vartheta $}
}
%vertical varistor
\newcommand{\vvar}{
\put(0,0){\line(0,1){5}}
\put(-2,5){\framebox(4,10){}\put(-2,10){\line(0,1){5}}}
\put(4,6){\line(-1,1){8}}
\put(4,3){\line(0,1){3}}
\put(4,16){$\rm U $}
}
%vertical photorezistor
\newcommand{\vprez}{
\put(0,0){\line(0,1){5}}
\put(-2,5){\framebox(4,10){}\put(-2,10){\line(0,1){5}}}
\multiput(-7,5)(0,2){2}{\vector(1,1){4}}
}
%horizontal peltier's cell
\newcommand{\hpelt}{
\put(0,0){\line(1,0){15}}
\put(5,-2){\framebox(10,4){}\put(0,2){\line(1,0){5}}}
\multiput(5.5,-2)(2,0){5}{$\prime $}
}
\endinput



\begin{document}
\section{The instructions for the using the macro eltex 2.0}
This is the version 2.0 of the macro \verb?eltex? for the drawing the
circuit diagrams in \LaTeX . Here is the several differences on the
comparison with the previous version.
In the version 2.0 are the circuit symbols accordingly IEC 617-1 to 11.
Here is the helpful grid for the easier placing of the circuit symbols
and also the commands for the circuit symbols are different
then in the previous version. 
Old and new version are not compatible.
The reason for it is that in the praxis the commands had many
parameters and the labels with the fixed coordinates mostly had to be
placed in the different place in the drawing. The new macro
contains some frequently used symbols "two times"
with the shorter and longer leads. This is time saving when you draw
the circuit diagram.
When you have only small area there is also the symbol with shorter
lead. The circuit symbols were created on the actual use
for the making the education materials in the communication technics,
circuit theory and electronics area. The size of macro is 100kB.
Not to occupy many RAM during the compilation, the
circuit symbols are distributed into several files eltex1 ... eltex7
accordingly the area of interest.
So you can use only several files and not all files.

This principle of the drawing the circuit diagrams is not comfortable,
it would be better to create some interactive graphical editor with the
saving the (eltex) commands to the file. 




\section{The description of the commands}
\subsection{Basic}
This circuit symbols are contained in macro \verb?\eltex1? \\

\noindent
\verb?\grid{x}{y}?
The command creates the grid with the step 10mm.
x - width of the picture in cm, y - height of the picture in cm,
only natural number.

\subsubsection{Sources}

\verb?\hsourcev?
Ideal voltage source horizontally oriented.

\noindent
\verb?\hhsourcev?
Ideal voltage source with reduced leads horizontally oriented.

\noindent
\verb?\vsourcev?
Ideal voltage source vertically oriented.

\noindent
\verb?\vvsourcev?
Ideal voltage source with reduced leads vertically oriented.

\noindent
\verb?\dsourcev{x}?
Ideal voltage source diagonally oriented, x - letter U -
up oriented, D - down oriented.

\noindent
\verb?\hsourcec?
Ideal current source horizontally oriented.

\noindent
\verb?\hhsourcec?
Ideal current source with reduced leads horizontally oriented.

\noindent
\verb?\vsourcec?
Ideal current source vertically oriented.

\noindent
\verb?\vvsourcec?
Ideal current source with reduced leads vertically oriented.

\noindent
\verb?\dsourcec{x}?
Ideal current source diagonally oriented, x - letter U -
up oriented, D - down oriented.

\subsubsection{Resistors}
\verb?\hrez{n}?
Resistor horizontally oriented. Number n determines the type of
the resistor.
0 - basic symbol
1 to 6 variable value resistor,
7 - nonlinear dependent resistor,

\noindent
\verb?\hhrez{n}?
Resistor with reduced leads horizontally oriented.

\noindent
\verb?\vrez{n}?
Resistor vertically oriented.

\noindent
\verb?\vvrez{n}?
Resistor with reduced leads vertically oriented.

\noindent
\verb?\drez{x}{n}?
Resistor diagonally oriented x - letter U -
up oriented, D - down oriented, n - number - type of the symbol,

\noindent
\verb?\ddrez{x}{n}?
Resistor with reduced leads diagonally oriented.


\subsubsection{Capacitors}
\verb?\hcap{n}?
Capacitor horizontally oriented. Number n type of the symbol.
0 - basic symbol
1, 4 variable capacitor


\noindent
\verb?\hhcap{n}?
Capacitor with reduced leads horizontally oriented.

\noindent
\verb?\vcap{n}?
Capacitor horizontally oriented.

\noindent
\verb?\vvcap{n}?
Capacitor with reduced leads vertically oriented.

\noindent
\verb?\dcap{x}{n}?
Capacitor diagonally oriented. x letter -  U up
oriented, D down oriented, n number - type of the symbol.



\subsubsection{Inductors}
\verb?\hturn{x}{n}?
Two turns horizontally oriented, x - letter - U position up oriented,
D down, n = 0 basic symbol, n = 1 variable inductor.


\noindent
\verb?\vturn{x}{n}?
Two turns vertically oriented, x - letter - L position left oriented,
R right, n = 0 basic symbol, n = 1 variable inductor.


\noindent
\verb?\hind{x}{n}?
Inductor horizontally oriented, x - letter - U position up,
D down, n = 0 basic symbol, n = 1 variable inductor.

\noindent
\verb?\hhind{x}{n}?
Inductor with reduced leads horizontally oriented.

\noindent
\verb?\vind{x}{n}?
Inductor vertically oriented, x - letter - L left position,
R right, n = 0 basic symbol, n = 1 variable inductor.


\noindent
\verb?\vvind{x}{n}?
Inductor with reduced leads vertically oriented.


\noindent
\verb?\dind{x}{y}{n}?
Inductor diagonally oriented, Bezier macro must be selected,
x - letter - U symbol up oriented,
D down, y - letter - L turns left oriented, R right,
n = 0 basic symbol, n = 1 variable inductor.



\subsubsection{Cores}
\verb?\hcore{n}{x}?
Core horizontally oriented, x - letter - core type I iron F ferrite, n
- natural number - length of the symbol in mm for (I) case,
or member of the dashed lines for (F) case.

\noindent
\verb?\vcore{n}{x}?
Core vertically oriented.

\noindent
\verb?\dcore{x}{n}{y}?
Core diagonally oriented, x - letter - U symbol up,
D symbol down, n - natural number - length of the symbol for (I) case,
or number of the dashed lines for (F) case,
y - core type I iron F ferrite.


\noindent
\verb?\simb{n}?
Symbols of the mutual orientation of winding,
n number 1 $\bullet$, 2 = $\Box$, 3 =
$\triangle $.


\subsubsection{Switches}
\verb?\hswitch{n}?
Switch horizontally oriented, n=0 switched on, 1 off.

\noindent
\verb?\vswitch{n}?
Switch vertically oriented, n=0 switched on, 1 off.


\noindent
\verb?\hoswitch{x}{n}?
Overswitch horizontally oriented x - letter - R right,
L left, n=0 switched on, 1 off.

\noindent
\verb?\voswitch{x}{n}?
Overswitch vertically oriented , x - letter - symbol U up,
D down, n=0 switched on, 1 off.

\subsubsection{Basic symbols}
\verb?\node?
connection of the leads.

\noindent
\verb?\pin?
terminal

\noindent
\verb?\hwire{n}?
wire in horizontal position, n length of the symbol in mm.


\noindent
\verb?\vwire{n}?
wire in vertical position, n length of the symbol in mm.

\noindent
\verb?\dwire{x}{n}?
wire in the diagonal position, x letter - position U up , D
down, n  length of the symbol.

\noindent
\verb?\cloop{x}{y}?
Symbol for the simple loop (mesh), x letter- R clockwise orientation,
L anticlockwise, y - text- label of the loop.

\noindent
\verb?\earth{x}?
Earth, x - letter - position D down, U up, L left, R right.


\noindent
\verb?\chassis{x}?
Chassis, x - letter - position D down, U up, L left, R right.


\noindent
\verb?\hmeasure{x}?
Measuring instrument indicating horizontally oriented,
x - text - symbol of the measured quantity.

\noindent
\verb?\hhmeasure{x}?
Measuring instrument indicating horizontally oriented, reduced leads.

\noindent
\verb?\vmeasure{x}?
Measuring instrument indicating vertically oriented,

\noindent
\verb?\vvmeasure{x}?
Measuring instrument indicating vertically oriented, reduced leads.

\noindent
\verb?\osc{x}?
Oscilloscope, x - letter - I  indicating, R recording.

\subsection{Semiconductors}

This circuit symbols are contained in macro \verb?\eltex2? \\

\noindent
\verb?\graetz{x}?
Diode bridge rectifier, x - letter - R plus pole right,
L plus pole left.

\noindent
\verb?\hdiode{x}{y}?
Diode horizontally oriented, x - letter - R cathode right, L
cathode left, y - letter - C varicap, Z Zener diode, E Esaki diode, S
Shottky diode, D LED, P photodiode, L laser diode,
blanc letter - basic symbol,

\noindent
\verb?\vdiode{x}{y}?
Diode vertically oriented, x - letter - U cathode up, D
cathode down, y - letter - type of the diode.

\noindent
\verb?\ddiode{x}{y}?
Diode diagonally oriented, x - letter - U symbol up,
D down, y - letter - U cathode up, D down.


\noindent
\verb?\htriac{x}?
Triac horizontally oriented, x - letter - gate
A left down, B left up, C right down, D right up.


\noindent
\verb?\vtriac{x}?
Triac vertically oriented, x - letter - gate
A left down, B left up, C right down, D right up.


\noindent
\verb?\hthyristor{x}{y}?
Thyristor horizontally oriented, x - letter - R cathode right, L
left, y - gate
A left down, B left up, C right down, D right up.


\noindent
\verb?\vthyristor{x}{y}?
Thyristor vertically oriented, x - letter - U cathode up, D
down, y - gate
A left down, B left up, C right down, D right up.


\noindent
\verb?\dthyristor{x}{y}{z}?
Thyristor diagonally oriented, x - letter - symbol U up,
D down, y - U cathode up D down,
z -gate
A left down, B left up, C right down, D right up.


\noindent
\verb?\hdiac?
Diac horizontally oriented.

\noindent
\verb?\vdiac?
Diac vertically oriented.

\noindent
\verb?\opto{x}?
Optocoupler horizontally oriented, x - letter - R
transistor right, L left.

\noindent
\verb?\opamp{x}{y}?
Operational amplifier, x - letter - output R right, L left,
U up, D down, y - inverting input U up, D down, R
right, L left.

\noindent
\verb?\ota{x}{y}{z}?
Operational transconductance amplifier,
x - letter - output R right, L left,
U up, D down, y - inverting input U up, D down, R
right, L left, z - letter Y OTA with linearising diodes, blanc -
without diodes.

\noindent
\verb?\bota{x}{y}{z}?
Balanced output transconductance amplifier, (x z y as in OTA)

\noindent
\verb?\bjt{x}{y}{z}{v}?
Bipolar junction transistor, x - type of the conductivity
N npn, P pnp, y -
collector emitter R right, L left, U up, D down, z - emitter position
R right, L left, U up, D down, v - symbol in the circle N no, Y yes.


\noindent
\verb?\jfet{x}{y}{z}{v}?
JFET, x - letter - channel N or P, y - letter
- position drain source R right, L left , U up, D down, z
- letter - position source R right, L left, U up, D down, v -
letter - circle Y yes, N (or blanc) no.


\noindent
\verb?\mos{x}{y}{z}{v}{w}{r}?
MOS transistor, x - letter - channel N or P, y - letter
- position drain source R right, L left , U up, D down, z
- letter - position source R right, L left, U up, D down, v -
letter - mode E enhancement, D depletion, w - number of the gates
1 or 2, r - letter - circle Y yes, N (or blanc) no.

\noindent
\verb?\ujt{x}{y}{z}{v}?
unijunction transistor (twobase diode),
x - letter - emitter N or P, y - letter
- circuit symbol position B1 B2, R right, L left , U up, D down, z
- letter - position B1 - R right, L left, U up, D down, v -
letter - circle Y yes, N (or blanc) no.

\noindent
\verb?\hall?
Hall's generator

\noindent
\verb?\hmag?
Magnistor horizontally.

\noindent
\verb?\vmag?
Magnistor vertically.

\noindent
\verb?\hptc?
Thermistor PTC horizontally.

\noindent
\verb?\vptc?
Thermistor PTC vertically.

\noindent
\verb?\hntc?
Thermistor NTC horizontally.

\noindent
\verb?\vntc?
Thermistor NTC vertically.

\noindent
\verb?\hvar?
Varistor horizontally.

\noindent
\verb?\var?
Varistor vertically.

\noindent
\verb?\hprez?
Photoresistor horizontally.

\noindent
\verb?\vprez?
Photoresistor vertically.

\noindent
\verb?\hpelt?
Peltier's cell horizontally.



\subsection{Special circuits}

This circuit symbols are contained in macro \verb?\eltex3? \\

\noindent
\verb?\deltaload? Load in the delta configuration.

\noindent
\verb?\starload? Load in the star configuration.

\noindent
\verb?\starsource? Sources in the star configuration.

\noindent
\verb?\neta \netb \netc \netd \nete? Elementary
twoports configurations.

\noindent
\verb?\inet?
I network.

\noindent
\verb?\gneta \gnetb?
Gamma network.

\noindent
\verb?\lneta \lnetb?
L network.

\noindent
\verb?\pineta \pinetb?
$\pi$ network.

\noindent
\verb?\tneta \tnetb?
T network.

\noindent
\verb?\xnet{n}?
X network, n - number - 1 general impedances, 2
symmetrical X network.

\noindent
\verb?\bhnet?
Bridget H network.

\noindent
\verb?\hnet?
H network.

\noindent
\verb?\btnet?
Bridget T network.

\noindent
\verb?\ttnet?
Double T network.

\noindent
\verb?\gyrator?
Gyrator.

\noindent
\verb?\unistor{x}?
Unistor, x - letter - orientation R right, L left, U up, D
down.

\noindent
\verb?\flow{x}?
Closed loop in the flowgraphs, x - letter - loop orientation U
up, D down.

\subsection{Electron tubes}

This circuit symbols are contained in macro \verb?\eltex4? \\


\noindent
\verb?\diode{x}{y}?
Diode, x - letter - filament Y yes, N (or blanc) no,
y - letter - equipotential cathode Y yes, N (or blanc) no.

\noindent
\verb?\triode{x}{y}?
Triode, x - letter - filament Y yes, N (or blanc) no,
y - letter - equipotential cathode Y yes, N (or blanc) no.

\noindent
\verb?\tetrode{x}{y}?
Tetrode, x - letter - filament Y yes, N (or blanc) no,
y - letter - equipotential cathode Y yes, N (or blanc) no.

\noindent
\verb?\pentode{x}{y}{z}?
Pentode, x - letter - filament Y yes, N (or blanc) no,
y - letter - equipotential cathode Y yes, N (or blanc) no,
z - letter - suppressor grid and cathode connection
Y yes, N (or blanc) no.


\noindent
\verb?\heptode{x}{y}{z}?
Heptode, x - letter - filament Y yes, N (or blanc) no,
y - letter - equipotential cathode Y yes, N (or blanc) no,
z - letter - cathode and
suppressor grid and screen grids g2 and g4 connection Y yes,
N  (or blanc) no.


\subsection{Other basic circuit symbols}

This circuit symbols are contained in macro \verb?\eltex5? \\

\noindent
\verb?\hfuse?
Fuse horizontally.

\noindent
\verb?\vfuse?
Fuse vertically.


\noindent
\verb?\hrelay?
Relay horizontally.


\noindent
\verb?\vrelay?
Relay vertically.

\noindent
\verb?\hlight?
Light source general (bulb) horizontally.

\noindent
\verb?\vlight?
Light source general (bulb) vertically.


\noindent
\verb?\bell?
Bell horizontally.

\noindent
\verb?\buzzer?
Buzzer horizontally.


\noindent
\verb?\siren?
Siren horizontally.


\noindent
\verb?\microphone{x}?
Microphone, x - letter - R right oriented,
L left.


\noindent
\verb?\earphone{x}?
Eatrphone, x - letter - R right oriented,
L left.

\noindent
\verb?\loudspeaker{x}?
Loudspeaker, x - letter - R right oriented,
L left.

\noindent
\verb?\hcrystal?
Quartz crystall horizontally.

\noindent
\verb?\vcrystal?
Crystall vertically.


\noindent
\verb?\hgap?
Spark gap horizontally.

\noindent
\verb?\vgap?
Spark gap vertically.


\noindent
\verb?\antenna{n}?
Antenna, n - number - 0 transmitting, 1 receiving, 2 transmission
and reception alternatively, 3 transmission and reception
instantaneously.

\noindent
\verb?\dipole{n}?
Dipole, n - number - 0 single dipole, 1 folded dipole.

\noindent
\verb?\loopant?
Loop antenna.

\noindent
\verb?\cell{x}{y}?
Galvanic cell horizontally or vertically,
x - letter - plus pole R right, L left,
U up, D down, y - letter - Y photocell, N (or blanc) galvanic cell.


\noindent
\verb?\hglow?
Glow lamp horizontally.


\noindent
\verb?\vglow?
Glow lamp vertically.

\noindent
\verb?\hdlamp?
Discharge lamp horizontally.

\noindent
\verb?\vdlamp?
Discharge lamp vertically.

\noindent
\verb?\hflamp?
Fluorescent lamp horizontally.

\noindent
\verb?\vflamp?
Fluorescent lamp vertically.


\noindent
\verb?\sensor{x}?
Sensor of the quantity, output in the left. x - text - symbol of the
quantity e. g. $\vartheta$ temperature.

\noindent
\verb?\head{n}?
converter head, output on the left. n - number -1 mechanical stereo
receiving,
2 magnetic mono receiving, 3 magnetic mono recording,
4 magnetic mono cleaning, 5 magnetic mono combined,
6 optical receiving.


\subsection{Block symbols}

This circuit symbols are contained in macro \verb?\eltex6? \\

\noindent
\verb?\ptran?
Transformer

\noindent
\verb?\fgen{n}?
Wave generator, n - number - 1 sine-wave, 2 square-wave, 3 sine-wave
with variable frequency.

\noindent
\verb?\delay?
Delay line

\noindent
\verb?\amplifier{x}?
Amplifier, x - letter - orientation R right, L left.

\noindent
\verb?\filter{n}?
Frequency filter, n - number - 1 low pass, 2 high pass, 3
band pass, 4 band stop.

\noindent
\verb?\compressor?
Compressor.


\noindent
\verb?\expander?
Expander.

\noindent
\verb?\deemphase?
Filter deemphase.

\noindent
\verb?\preemphase?
Filter preemphase.


\noindent
\verb?\artline?
Artificial line.

\noindent
\verb?\converter?
Converter.

\noindent
\verb?\corrector{n}?
Corrector, n - number - 1 amplitude distortion corrector,
2 phase corrector, 3 group delay corrector.

\noindent
\verb?\limiter?
Limiter.

\noindent
\verb?\balance?
Balance.

\noindent
\verb?\termin{n}?
Termination set, n - number - 1 with balancing network,
other number - without balancing network.

\noindent
\verb?\hybrid?
Hybrid transformer.

\noindent
\verb?\modulator?
Modulator.


\noindent
\verb?\atenuator?
Attenuator.


\noindent
\verb?\carrier{n}?
Carrier frequency, n - number - 1 carrier,
2 partially supressed carrier, 3 supressed carrier.


\noindent
\verb?\freq{n}?
Frequency, n - number - 1 pilot frequency,
2 signalling frequency. 3 measuring frequency.

\noindent
\verb?\pilot{n}?
Pilot frequency, n - 1 basic group, 2 super group,
3 master group, 4 super master group.

\noindent
\verb?\band{n}?
Frequency band, n - 1 frequency noninverted, 2 band ad 1
phase inverted, 3 frequency inverted, 4 band ad 3 phase
inverted.


\noindent
\verb?\ltran{x}{n}?
Light transmitter, x - letter - light transmission
R right, L left, n - number - light - 1 coherent,
other number - uncoherent.


\noindent
\verb?\lrec{x}{n}?
Light receiver, x - letter - light reception
R right, L left, n - number - light 1 coherent,
other number - uncoherent.


\noindent
\verb?\fibre{x}?
Optical fibre, x - letter - S single mode step refraction index,
M multi mode step refraction index, blank letter -
general optical fibre.

\noindent
\verb?\threephase{x}?
Three phase source, x - letter - D delta, S star.

\noindent
\verb?\rgraetz?
Bridge rectifier.


\noindent
\verb?\trafo{x}?
Single phase transformer, x - letter - position
H horizontally, V vertically.


\noindent
\verb?\engine{n}?
Engine, n - number - 0  direct current, 1 single phase,
3 three phase ,
4 linear, 5 stepping.

\noindent
\verb?\gener{n}?
Generator, n - number - 0 direct current, 1 single phase,
3 three phase.

\subsection{Logical circuit}

This circuit symbols are contained in macro \verb?\eltex7? \\

\noindent
\noindent
\verb?\andnand{n}{m}{x}?
Logical gate AND, n - number - number of inputs 2 two inputs, 3 three
inputs,
m - number - type of the gate 0 basic, 1 power,  2 basic with open
collector, 3 power with open collector, x - letter - I
inverted output, N noninverted output.


\noindent
\verb?\ornor{n}{m}{x}?
Logical gate OR, n - number - number of inputs 2 two inputs, 3 three
inputs,
m - number - type of the gate 0 basic, 1 power,  2 basic with open
collector, 3 power with open collector, x - letter - I
inverted output, N noninverted output.

\noindent
\verb?\invert{m}{x}?
Inverter,
m - number - type of the gate 0 basic, 1 power,  2 basic with open
collector, 3 power with open collector, x - letter - I
inverted output, N noninverted output.



\section{Using of the commands}
It is necessary to  select the
macro \verb?eltex? a \verb?bezier?   In the case of "old"
\LaTeX  2.09 there are the commands:

 \begin{verbatim}
\documentstyle[a4,bezier]{article}
%%%%%%%%%%% accordingly the CSN IEC 617-1 to 11 (1993-1995) %%%%%%%%
%%%%%%%%%%% Libor Gajdosik 2009 eltex version 2.0 %%%%%%%%%%%%%

\unitlength=1mm
%%%%%%%%grid -generating help grid 10x10mm to simplify orientation%%%%%%
\newcommand{\grid}[2]{
\linethickness{0.1mm}
\unitlength=10mm
  \newcounter{sloupec}
  \setcounter{sloupec}{10}
\put(0,0){\line(0,1){#2}}
\multiput(1,0)(1,0){#1}{\line(0,1){#2}}
\multiput(0.7,-0.5)(1,0){#1}{\arabic{sloupec}\addtocounter{sloupec}{10}}
  \setcounter{sloupec}{10}
\multiput(0.6,#2)(1,0){#1}{\makebox(0.7,0.6)
       {\arabic{sloupec}\addtocounter{sloupec}{10}}}
   \newcounter{radek}
 \put(-0.3,-0.3){\arabic{radek}}
   \setcounter{radek}{10}
\put(0,0){\line(1,0){#1}}
\multiput(0,1)(0,1){#2}{\line(1,0){#1}}
\multiput(-0.8,0.7)(0,1){#2}{\makebox(0.7,0.5)[r]
      {\arabic{radek}\addtocounter{radek}{10}}}
   \setcounter{radek}{10}
\multiput(#1,0.6)(0,1){#2}{\makebox(0.7,0.6)[r]
    {\arabic{radek}\addtocounter{radek}{10}}}
\thicklines
\unitlength=1mm
}
%%%%basic pasive devices (source, resistor, capacitor, inductor)%%%%
%horizontal voltage source
\newcommand{\hsourcev}{
\put(0,0){\line(1,0){30}}
\put(15,0){\circle{10}}
}
%vertical voltage source
\newcommand{\vsourcev}{
\put(0,0){\line(0,1){30}}
``\put(0,15){\circle{10}}
}
%horizontal current source
\newcommand{\hsourcec}{
\put(0,0){\line(1,0){10}}
\put(15,-5){\line(0,1){10}}
\put(15,0){\circle{10}}
\put(20,0){\line(1,0){10}}
}
%vertical current source
\newcommand{\vsourcec}{
\put(0,0){\line(0,1){10}}
\put(5,15){\line(-1,0){10}}
\put(0,15){\circle{10}}
\put(0,20){\line(0,1){10}}
}
\newcommand{\hhsourcev}{
\put(0,0){\line(1,0){20}}
\put(10,0){\circle{10}}
}
%vertical voltage source
\newcommand{\vvsourcev}{
\put(0,0){\line(0,1){20}}
\put(0,10){\circle{10}}
}
%diagonal voltage source
\newcommand{\dsourcev}[1]{
\ifx#1U   %source up right oriented
  \put(0,0){\line(1,1){20}}
  \put(10,10){\circle{10}}
\else \ifx#1D  %source down right oriented
   \put(0,0){\line(1,-1){20}}
   \put(10,-10){\circle{10}}
\fi \fi
}
%horizontal current source
\newcommand{\hhsourcec}{
\put(0,0){\line(1,0){5}}
\put(10,-5){\line(0,1){10}}
\put(10,0){\circle{10}}
\put(15,0){\line(1,0){5}}
}
%vertical current source
\newcommand{\vvsourcec}{
\put(0,0){\line(0,1){5}}
\put(5,10){\line(-1,0){10}}
\put(0,10){\circle{10}}
\put(0,15){\line(0,1){5}}
}
%diagonal current source
\newcommand{\dsourcec}[1]{
\ifx#1U   %source up right oriented
  \put(0,0){\line(1,1){6.5}}
  \put(6.5,13.5){\line(1,-1){7}}
  \put(10,10){\circle{10}}
  \put(13.5,13.5){\line(1,1){6.5}}
\else \ifx#1D  %source down right oriented
  \put(0,0){\line(1,-1){6.5}}
  \put(13.5,-6.5){\line(-1,-1){7}}
  \put(10,-10){\circle{10}}
  \put(13.5,-13.5){\line(1,-1){6.5}}
\fi \fi
}
%horizontal rezistor
\newcommand{\hhrez}[1]{
\put(0,0){\line(1,0){5}}
\put(5,-2){\framebox(10,4){}\put(0,2){\line(1,0){5}}}
\ifnum#1=1           % rez. adjustable by tool (medium pin down oriented)
\put(8,-6){\line(1,3){4}}
\put(10,7){\line(3,-1){4}}
\else \ifnum#1=2     % rez. adjustable by tool (medium pin up oriented)
\put(8,6){\line(1,-3){4}}
\put(10,-7){\line(3,1){4}}
\else \ifnum#1=3     % rez. adjustable by tool (medium pin shortly
                                                        %conected)
\put(8,6){\line(1,-3){4}}
\put(10,-7){\line(3,1){4}}
\put(8,6){\line(-1,0){5}}
\put(3,6){\line(0,-1){6}}
\put(3,0){\node}
\else \ifnum#1=4       % rez. adjustable (medium pin down oriented)
\put(6,-4){\vector(1,1){9}}
\else \ifnum#1=5       % rez. adjustable (medium pin up oriented)
\put(6,4){\vector(1,-1){9}}
\else \ifnum#1=6       % rez. adjustable (medium pin shortly conected)
\put(8,6){\vector(1,-3){4}}
\put(8,6){\line(-1,0){5}}
\put(3,6){\line(0,-1){6}}
\put(3,0){\node}
\else \ifnum#1=7           %nonlinearly dependent
\put(6,-4){\line(1,1){8}}
\put(3,-4){\line(1,0){3}}


\fi \fi \fi \fi \fi \fi \fi
  }
%vertical rezistor
\newcommand{\vvrez}[1]{
\put(0,0){\line(0,1){5}}
\put(-2,5){\framebox(4,10){}\put(-2,10){\line(0,1){5}}}
\ifnum#1=1          % rez. adjustable by tool (medium pin left oriented)
\put(-6,8){\line(3,1){12}}
\put(5,14){\line(1,-2){2}}
\else \ifnum#1=2    % rez. adjustable by tool (medium pin right oriented)
\put(-6,12){\line(3,-1){12}}
\put(-7,10){\line(1,2){2}}
\else \ifnum#1=3   % rez. adjustable by tool (medium pin shortly
                                                        %conected)
\put(-6,7){\line(3,1){12}}
\put(5,13){\line(1,-2){2}}
\put(-6,7){\line(0,-1){4}}
\put(-6,3){\line(1,0){6}}
\put(0,3){\node}
\else \ifnum#1=4       % rez. adjustable (medium pin left oriented)
\put(-4,6){\vector(1,1){9}}
\else \ifnum#1=5       % rez. adjustable (medium pin right oriented)
\put(4,6){\vector(-1,1){9}}
\else \ifnum#1=6      % rez. adjustable (medium pin shortly conected)
\put(-6,8){\vector(3,1){12}}
\put(-6,8){\line(0,-1){5}}
\put(-6,3){\line(1,0){6}}
\put(0,3){\node}
\else \ifnum#1=7
\put(-4,6){\line(1,1){8}}
\put(4,14){\line(1,0){3}}

\fi \fi \fi \fi \fi \fi \fi
  }
%diagonal rezistor
\newcommand{\drez}[2]{
\ifx#1U      % up left oriented
\put(0,0){\line(1,1){7}\put(2.2,5.2){\line(-1,1){4}}
\put(-1.9,9){\line(1,1){7}}\put(2.2,5){\line(1,1){7}}
\put(9,12){\line(-1,1){4}}\put(7,14){\line(1,1){6}}}
      \ifnum#2=1        % rez. adjustable by tool (medium pin left oriented)
\put(2,10){\line(1,0){16}}
\put(18,12){\line(0,-1){4}}
      \fi \ifnum#2=2    % rez. adjustable by tool (medium pin right oriented)
\put(2,10){\line(1,0){16}}
\put(2,12){\line(0,-1){4}}
      \fi \ifnum#2=3   % rez. adjustable by tool (medium pin shortly
                                                        %conected)
\put(2,10){\line(1,0){16}}
\put(18,12){\line(0,-1){4}}
\put(2,10){\line(0,-1){4}}
\put(2,6){\line(1,0){4}}
\put(6,6){\node}
      \fi \ifnum#2=4         % rez. adjustable (medium pin left oriented)
\put(2,10){\vector(1,0){16}}
      \fi \ifnum#2=5         % rez. adjustable (medium pin right oriented)
\put(18,10){\vector(-1,0){16}}
      \fi \ifnum#2=6         % rez. adjustable (medium pin shortly conected)
\put(2,10){\vector(1,0){16}}
\put(2,10){\line(0,-1){4}}
\put(2,6){\line(1,0){4}}
\put(6,6){\node}
       \fi \ifnum#2=7
\put(2,10){\line(1,0){16}}
\put(18,10){\line(1,1){4}}
       \fi   \fi

\ifx#1D              %down right oriented
\put(0,0){\line(1,-1){7}\put(-1.9,-9){\line(1,1){4}}
\put(2,-4.9){\line(1,-1){7}}\put(-1.9,-9){\line(1,-1){7}}}
\put(12,-16){\line(1,1){4}\put(-2,2){\line(1,-1){6}}}
       \ifnum#2=1   % rez. adjustable by tool (medium pin left oriented)
\put(2,-10){\line(1,0){16}}
\put(18,-8){\line(0,-1){4}}
       \else \ifnum#2=2  % rez. adjustable by tool (medium pin right oriented)
\put(2,-10){\line(1,0){16}}
\put(2,-8){\line(0,-1){4}}
      \else \ifnum#2=3   % rez. adjustable by tool (medium pin shortly
                                                        %conected)
\put(2,-10){\line(1,0){16}}
\put(18,-8){\line(0,-1){4}}
\put(2,-10){\line(0,1){4}}
\put(2,-6){\line(1,0){4}}
\put(6,-6){\node}
     \else \ifnum#2=4    % rez. adjustable (medium pin left oriented)
\put(2,-10){\vector(1,0){16}}
     \else \ifnum#2=5    % rez. adjustable (medium pin right oriented)
\put(18,-10){\vector(-1,0){16}}
     \else \ifnum#2=6    % rez. adjustable (medium pin shortly conected)
\put(2,-10){\vector(1,0){16}}
\put(2,-10){\line(0,1){4}}
\put(2,-6){\line(1,0){4}}
\put(6,-6){\node}
      \else \ifnum#2=7
\put(2,-10){\line(1,0){16}}
\put(18,-10){\line(1,1){4}}
\fi  \fi  \fi \fi  \fi \fi \fi \fi
}

%diagonal rezistor (short)
\newcommand{\ddrez}[2]{
\ifx#1U      % up left oriented
\put(0,0){\line(1,1){4}\put(2,2){\line(-1,1){4}}
\put(2,2){\line(1,1){7}}\put(-2,6){\line(1,1){7}}
\put(9,9){\line(-1,1){4}}\put(7,11){\line(1,1){4}}}
      \ifnum#2=1        % rez. adjustable by tool (medium pin left oriented)
\put(-1,7){\line(1,0){16}}
\put(15,9){\line(0,-1){4}}
      \fi \ifnum#2=2    % rez. adjustable by tool (medium pin right oriented)
\put(0,7){\line(1,0){16}}
\put(0,9){\line(0,-1){4}}
      \fi \ifnum#2=3   % rez. adjustable by tool (medium pin shortly
                                                        %conected)
\put(-1,8){\line(1,0){16}}
\put(15,10){\line(0,-1){4}}
\put(-1,8){\line(0,-1){6}}
\put(-1,2){\line(1,0){3}}
\put(2,2){\node}
      \fi \ifnum#2=4         % rez. adjustable (medium pin left oriented)
\put(1,8){\vector(1,0){16}}
      \fi \ifnum#2=5         % rez. adjustable (medium pin right oriented)
\put(16,8){\vector(-1,0){16}}
      \fi \ifnum#2=6         % rez. adjustable (medium pin shortly conected)
\put(-1,8){\vector(1,0){18}}
\put(-1,8){\line(0,-1){6}}
\put(-1,2){\line(1,0){3}}
\put(2,2){\node}
       \fi \ifnum#2=7
\put(2,8){\line(1,0){16}}
\put(18,8){\line(1,1){4}}
       \fi   \fi

\ifx#1D              %down right oriented
\put(0,0){\line(1,-1){4}\put(-2,-6){\line(1,1){4}}
\put(-2,-6){\line(1,-1){7}}\put(2,-2){\line(1,-1){7}}}
\put(9,-13){\line(1,1){4}\put(-2,2){\line(1,-1){4}}}
       \ifnum#2=1   % rez. adjustable by tool (medium pin left oriented)
\put(0,-8){\line(1,0){16}}
\put(16,-6){\line(0,-1){4}}
       \else \ifnum#2=2  % rez. adjustable by tool (medium pin right oriented)
\put(0,-8){\line(1,0){16}}
\put(0,-6){\line(0,-1){4}}
      \else \ifnum#2=3   % rez. adjustable by tool (medium pin shortly
                                                        %conected)
\put(0,-8){\line(1,0){16}}
\put(16,-6){\line(0,-1){4}}
\put(0,-8){\line(0,1){6}}
\put(0,-2){\line(1,0){2}}
\put(2,-2){\node}
     \else \ifnum#2=4    % rez. adjustable (medium pin left oriented)
\put(0,-8){\vector(1,0){16}}
     \else \ifnum#2=5    % rez. adjustable (medium pin right oriented)
\put(16,-8){\vector(-1,0){16}}
     \else \ifnum#2=6    % rez. adjustable (medium pin shortly conected)
\put(0,-8){\vector(1,0){16}}
\put(0,-8){\line(0,1){6}}
\put(0,-2){\line(1,0){2}}
\put(2,-2){\node}
      \else \ifnum#2=7
\put(0,-8){\line(1,0){16}}
\put(16,-8){\line(1,1){4}}
\fi  \fi  \fi \fi  \fi \fi \fi \fi
}
%horizontal rezistor
\newcommand{\hrez}[1]{
\put(0,0){\line(1,0){10}}
\put(10,-2){\framebox(10,4){}\put(0,2){\line(1,0){10}}}
\ifnum#1=1           % rez. adjustable by tool (medium pin down oriented)
\put(12,-6){\line(1,3){4}}
\put(14,7){\line(3,-1){4}}
\else \ifnum#1=2     % rez. adjustable by tool (medium pin up oriented)
\put(12,6){\line(1,-3){4}}
\put(14,-7){\line(3,1){4}}
\else \ifnum#1=3     % rez. adjustable by tool (medium pin shortly
                                                        %conected)
\put(12,6){\line(1,-3){4}}
\put(14,-7){\line(3,1){4}}
\put(12,6){\line(-1,0){4}}
\put(8,6){\line(0,-1){6}}
\put(8,0){\node}
\else \ifnum#1=4       % rez. adjustable (medium pin down oriented)
\put(11,-4){\vector(1,1){9}}
\else \ifnum#1=5       % rez. adjustable (medium pin up oriented)
\put(11,4){\vector(1,-1){9}}
\else \ifnum#1=6       % rez. adjustable (medium pin shortly conected)
\put(12,6){\vector(1,-3){4}}
\put(12,6){\line(-1,0){4}}
\put(8,6){\line(0,-1){6}}
\put(8,0){\node}
\else \ifnum#1=7             %%nonlinearly  dependent
\put(11,-4){\line(1,1){8}}
\put(19,4){\line(1,0){3}}
\fi \fi \fi \fi \fi \fi  \fi
  }
%vertical rezistor
\newcommand{\vrez}[1]{
\put(0,0){\line(0,1){10}}
\put(-2,10){\framebox(4,10){}\put(-2,10){\line(0,1){10}}}
\ifnum#1=1          % rez. adjustable by tool (medium pin left oriented)
\put(-6,12){\line(3,1){12}}
\put(5,18){\line(1,-2){2}}
\else \ifnum#1=2    % rez. adjustable by tool (medium pin right oriented)
\put(-6,18){\line(3,-1){12}}
\put(-7,16){\line(1,2){2}}
\else \ifnum#1=3   % rez. adjustable by tool (medium pin shortly
                                                        %conected)
\put(-6,12){\line(3,1){12}}
\put(5,18){\line(1,-2){2}}
\put(-6,12){\line(0,-1){4}}
\put(-6,8){\line(1,0){6}}
\put(0,8){\node}
\else \ifnum#1=4       % rez. adjustable (medium pin left oriented)
\put(-4,11){\vector(1,1){9}}
\else \ifnum#1=5       % rez. adjustable (medium pin right oriented)
\put(4,11){\vector(-1,1){9}}
\else \ifnum#1=6      % rez. adjustable (medium pin shortly conected)
\put(-6,12){\vector(3,1){12}}
\put(-6,12){\line(0,-1){4}}
\put(-6,8){\line(1,0){6}}
\put(0,8){\node}
\else \ifnum#1=7
\put(-4,11){\line(1,1){8}}
\put(4,19){\line(1,0){3}}
\fi \fi \fi \fi \fi \fi \fi
 }
%horizontal capacitor
\newcommand{\hcap}[1]{
\put(0,0){\line(1,0){14.5}}
\multiput(14.5,3)(1,0){2}{\line(0,-1){6}}
\put(15.5,0){\line(1,0){14.5}}
\ifnum#1=1              % capacitor adjustable by tool
\put(18,-3){\line(-1,1){6}}
\put(9.5,0.5){\makebox(5,5){/}}
 \else \ifnum#1=4       %capacitor adjustable
\put(18,-3){\vector(-1,1){6}}
 \fi \fi
  }
%vertical capacitor
\newcommand{\vcap}[1]{
\put(0,0){\line(0,1){14.5}}
\multiput(-3,14.5)(0,1){2}{\line(1,0){6}}
\put(0,15.5){\line(0,1){14.5}}
 \ifnum#1=1                %cap. adjustable by tool
\put(2.5,12.5){\line(-1,1){6}}
\put(-6,16){\makebox(5,5){/}}
 \else \ifnum#1=4             %cap. adjustable
\put(-3,12){\vector(1,1){7}}
 \fi \fi
  }
%diagonal capacitor
\newcommand{\dcap}[2]{
\ifx#1U          % up right oriented
\put(0,0){\line(1,1){9.25}}
\multiput(7.25,11.75)(1,1){2}{\line(1,-1){4.5}}
\put(10.75,10.75){\line(1,1){9.25}}
  \ifnum#2=1                    %cap. adjustable with tool
\put(5,10){\line(1,0){10}}
\put(15,8){\line(0,1){4}}
  \else \ifnum#2=4             %cap. adjustable
\put(5,10){\vector(1,0){12}}
   \fi \fi
\fi
\ifx#1D                      % down right oriented
\put(0,0){\line(1,-1){9.3}}
\multiput(7.25,-11.75)(1,-1){2}{\line(1,1){4.5}}
\put(10.75,-10.75){\line(1,-1){9.25}}
  \ifnum#2=1                        %cap. adjustable with tool
\put(10,-15){\line(0,1){10}}
\put(8,-5){\line(1,0){4}}
  \else \ifnum#2=4            %cap. adjustable
\put(10,-15){\vector(0,1){11}}
  \fi \fi
\fi
}
%horizontal capacitor
\newcommand{\hhcap}[1]{
\put(0,0){\line(1,0){4.5}}
\multiput(4.5,3)(1,0){2}{\line(0,-1){6}}
\put(5.5,0){\line(1,0){4.5}}
\ifnum#1=1              % capacitor adjustable by tool
\put(8,-3){\line(-1,1){6}}
\put(-0.5,0.5){\makebox(5,5){/}}
 \else \ifnum#1=4       %capacitor adjustable
\put(8,-3){\vector(-1,1){7}}
 \fi \fi
  }
%vertical capacitor
\newcommand{\vvcap}[1]{
\put(0,0){\line(0,1){4.5}}
\multiput(-3,4.5)(0,1){2}{\line(1,0){6}}
\put(0,5.5){\line(0,1){4.5}}
 \ifnum#1=1                %cap. adjustable by tool
\put(2.5,2.5){\line(-1,1){6}}
\put(-6,6){\makebox(5,5){/}}
 \else \ifnum#1=4             %cap. adjustable
\put(-3,2){\vector(1,1){7}}
 \fi \fi
  }
%\horizontal 2 turns
\newcommand{\hturn}[2]{
\ifx#1U               %up arces
\multiput(1.5,0)(3,0){2}{\oval(3,3)[t]}
  \ifnum#2=1            %variable inducance
\put(1.5,4){\line(1,-2){3}}
\put(-0.5,3.25){\line(2,1){4}}
  \fi
\else \ifx#1D         %down arces
\multiput(1.5,0)(3,0){2}{\oval(3,3)[b]}
 \ifnum#2=1            %variable inductance
\put(1.5,-4){\line(1,2){3}}
\put(-0.5,-3.5){\line(2,-1){4}}
  \fi
\fi \fi
}
%vertical 2 turns
\newcommand{\vturn}[2]{
\ifx#1L               %left arces
\multiput(0,1.5)(0,3){2}{\oval(3,3)[l]}
   \ifnum#2=1           %variable inductance
 \put(1,4.5){\line(-2,-1){4}}
 \put(-4.5,4.5){\line(1,-2){2}}
    \fi
\else \ifx#1R        %right arces
\multiput(0,1.5)(0,3){2}{\oval(3,3)[r]}
    \ifnum#2=1           %variable inductance
 \put(-1,1.5){\line(2,1){4}}
 \put(2.4,5.5){\line(1,-2){2}}
    \fi
\fi \fi
}


%\horizontal inductor
\newcommand{\hind}[2]{
\put(0,0){\line(1,0){9}}
\ifx#1U               %up arces
\multiput(10.5,0)(3,0){4}{\oval(3,3)[t]}
  \ifnum#2=1            %variable inducance
\put(12,7){\line(1,-2){5}}
\put(15.25,-4.25){\line(2,1){4}}
  \fi
\else \ifx#1D         %down arces
\multiput(10.5,0)(3,0){4}{\oval(3,3)[b]}
 \ifnum#2=1            %variable inductance
\put(12,-7){\line(1,2){5}}
\put(15.25,4.25){\line(2,-1){4}}
  \fi
\fi \fi
\put(21,0){\line(1,0){9}}
}
%vertical inductor
\newcommand{\vind}[2]{
\put(0,0){\line(0,1){9}}
\ifx#1L               %left arces
\multiput(0,10.5)(0,3){4}{\oval(3,3)[l]}
   \ifnum#2=1           %variable inductance
 \put(4,18){\line(-2,-1){10}}
 \put(3,20){\line(1,-2){2}}
    \fi
\else \ifx#1R        %right arces
\multiput(0,10.5)(0,3){4}{\oval(3,3)[r]}
    \ifnum#2=1           %variable inductance
 \put(-4,13){\line(2,1){10}}
 \put(-5,15){\line(1,-2){2}}
    \fi
\fi \fi
\put(0,21){\line(0,1){9}}
}
%diagonal inductor
\newcommand{\dind}[3]{
\ifx#1D                 %down oriented
\put(0,0){\line(1,-1){5}}
\put(15,-15){\line(1,-1){5}}
       \ifx#2R          %right arces
\multiput(5,-5)(2.5,-2.5){4}{\bezier{40}(0,0)(3.5,1)(2.5,-2.5)}
        \ifnum#3=1          %variable inductance
    \put(5,-10.5){\line(1,0){12}}
    \put(5,-8.5){\line(0,-1){4}}
      \fi
    \fi
       \ifx#2L          %left arces
\multiput(5,-5)(2.5,-2.5){4}{\bezier{40}(0,0)(-1,-3.5)(2.5,-2.5)}
       \ifnum#3=1          %variable inductance
    \put(15,-11){\line(-1,0){12}}
    \put(15,-9){\line(0,-1){4}}
      \fi
  \fi
\fi
\ifx#1U                  %up oriented
\put(0,0){\line(1,1){5}}
\put(15,15){\line(1,1){5}}
   \ifx#2R               %right arces
\multiput(5,5)(2.5,2.5){4}{\bezier{40}(0,0)(3.5,-1)(2.5,2.5)}
    \ifnum#3=1              %variable inductance
  \put(6,11){\line(1,0){12}}
  \put(6,13){\line(0,-1){4}}
    \fi
    \fi
  \ifx#2L                %left arces
\multiput(5,5)(2.5,2.5){4}{\bezier{40}(0,0)(-1,3.5)(2.5,2.5)}
        \ifnum#3=1          %variable inductance
     \put(15,11){\line(-1,0){12}}
     \put(15,13){\line(0,-1){4}}
        \fi
   \fi
\fi
}
%\horizontal inductor
\newcommand{\hhind}[2]{
\put(0,0){\line(1,0){4}}
\ifx#1U               %up arces
\multiput(5.5,0)(3,0){4}{\oval(3,3)[t]}
  \ifnum#2=1            %variable inducance
\put(7,7){\line(1,-2){5}}
\put(10.25,-4.25){\line(2,1){4}}
  \fi
\else \ifx#1D         %down arces
\multiput(5.5,0)(3,0){4}{\oval(3,3)[b]}
 \ifnum#2=1            %variable inductance
\put(7,-7){\line(1,2){5}}
\put(10.25,4.25){\line(2,-1){4}}
  \fi
\fi \fi
\put(16,0){\line(1,0){4}}
}
%vertical inductor
\newcommand{\vvind}[2]{
\put(0,0){\line(0,1){4}}
\ifx#1L               %left arces
\multiput(0,5.5)(0,3){4}{\oval(3,3)[l]}
   \ifnum#2=1           %variable inductance
 \put(4,13){\line(-2,-1){10}}
 \put(3,15){\line(1,-2){2}}
    \fi
\else \ifx#1R        %right arces
\multiput(0,5.5)(0,3){4}{\oval(3,3)[r]}
    \ifnum#2=1           %variable inductance
 \put(-4,8){\line(2,1){10}}
 \put(-5,10){\line(1,-2){2}}
    \fi
\fi \fi
\put(0,16){\line(0,1){4}}
}
%diagonal coil core
\newcommand{\dcore}[3]{
\ifx#1D                    %down oriented
  \ifx#3I          %iron core
\put(0,0){\line(1,-1){#2}}
  \else \ifx#3F    %ferrite core
\multiput(0,0)(5,-5){#2}{\line(1,-1){4}}
   \fi \fi
 \else \ifx#1U             %up oriented
  \ifx#3I         %iron core
\put(0,0){\line(1,1){#2}}
  \else \ifx#3F   %ferrit core
\multiput(0,0)(5,5){#2}{\line(1,1){4}}
  \fi \fi
\fi \fi
}
%vertical coil core
\newcommand{\vcore}[2]{
 \ifx#2I            %iron core
\put(0,0){\line(0,1){#1}}
 \else \ifx#2F      %ferrite core
\multiput(0,0)(0,4){#1}{\line(0,1){3}}
 \fi \fi
}
%horizontal coil core
\newcommand{\hcore}[2]{
 \ifx#2I             %iron core
\put(0,0){\line(1,0){#1}}
 \else \ifx#2F       %ferrite core
\multiput(0,0)(4,0){#1}{\line(1,0){3}}
 \fi \fi
}
%%%switch%%%%%%%%%%%%
%horizontal switch
\newcommand{\hswitch}[1]{
\put(0,0){\line(1,0){2.5}}
  \ifnum#1=1                   %switch on
\put(2.5,0){\line(3,1){6}}
  \fi \ifnum#1=0               %switch off
\put(2.5,0){\line(3,1){6}}
\put(7.5,0){\line(0,1){1.9}}
       \fi
\put(7.5,0){\line(1,0){2.5}}
}
%vertical switch
\newcommand{\vswitch}[1]{
\put(0,0){\line(0,-1){2.5}}
  \ifnum#1=1               %switch on
\put(0,-2.5){\line(1,-3){2}}
  \fi \ifnum#1=0           %switch off
\put(0,-2.5){\line(1,-3){2}}
\put(0,-7.5){\line(1,0){1.9}}
       \fi
\put(0,-7.5){\line(0,-1){2.5}}
}
%horizontal overswitch
\newcommand{\hoswitch}[2]{
\ifx#1R                    %right oriented
  \ifnum#2=0               %switch off
\put(0,0){\line(1,0){2.5}}
\put(2.5,0){\line(2,1){6}}
\put(7.5,2.3){\line(0,1){2.8}}
\put(7.5,5){\line(1,0){2.5}}
\put(7.5,0){\line(1,0){2.5}}
       \fi
  \ifnum#2=1                   %switch on
\put(0,0){\line(1,0){2.5}}
\put(2.5,0){\line(2,-1){6}}
\put(7.5,-2.3){\line(0,-1){2.8}}
\put(7.5,-5){\line(1,0){2.5}}
\put(7.5,0){\line(1,0){2.5}}
  \fi
\fi
\ifx#1L                    %left oriented
  \ifnum#2=0               %switch off
\put(0,0){\line(1,0){2.5}}
\put(7.5,0){\line(-2,1){6}}
\put(2.5,2.3){\line(0,1){2.8}}
\put(0,5){\line(1,0){2.5}}
\put(7.5,0){\line(1,0){2.5}}
       \fi
  \ifnum#2=1                   %switch on
\put(0,0){\line(1,0){2.5}}
\put(7.5,0){\line(-2,-1){6}}
\put(2.5,-2.3){\line(0,-1){2.8}}
\put(0,-5){\line(1,0){2.5}}
\put(7.5,0){\line(1,0){2.5}}
  \fi
\fi
}
%vertical overswitch
\newcommand{\voswitch}[2]{
\ifx#1D                     %down oriented
    \ifnum#2=0           %switch off
\put(0,0){\line(0,-1){2.5}}
\put(0,-2.5){\line(-1,-2){3}}
\put(-5,-7.5){\line(1,0){2.5}}
\put(0,-7.5){\line(0,-1){2.5}}
\put(-5,-7.5){\line(0,-1){2.5}}
       \fi
  \ifnum#2=1               %switch on
\put(0,0){\line(0,-1){2.5}}
\put(0,-2.5){\line(1,-2){3}}
\put(2.5,-7.5){\line(1,0){2.5}}
\put(0,-7.5){\line(0,-1){2.5}}
\put(5,-7.5){\line(0,-1){2.5}}
  \fi
\fi
\ifx#1U                      %up oriented
    \ifnum#2=0           %switch off
\put(0,0){\line(0,1){2.5}}
\put(0,2.5){\line(-1,2){3}}
\put(-5,7.5){\line(1,0){2.5}}
\put(0,7.5){\line(0,1){2.5}}
\put(-5,7.5){\line(0,1){2.5}}
       \fi
  \ifnum#2=1               %switch on
\put(0,0){\line(0,1){2.5}}
\put(0,2.5){\line(1,2){3}}
\put(2.5,7.5){\line(1,0){2.5}}
\put(0,7.5){\line(0,1){2.5}}
\put(5,7.5){\line(0,1){2.5}}
  \fi
\fi
}
%%%%node, loop, pin, wire, earth, chassis%%%%%%%%%
\newcommand{\node}{       %conection of devices
\put(0,0){\circle*{1}}
}
\newcommand{\pin}{
\put(0,0){\circle{1.5}}
}
\newcommand{\hwire}[1]{
\put(0,0){\line(1,0){#1}}  % length mm
}
\newcommand{\vwire}[1]{
\put(0,0){\line(0,1){#1}}  % length mm
}
\newcommand{\dwire}[2]{
\ifx#1U                    %up right oriented
\put(0,0){\line(1,1){#2}}  %length mm
\fi
\ifx#1D                    %down right oriented
\put(0,0){\line(1,-1){#2}} %length mm
\fi
}
%simbol
\newcommand{\simb}[1]{
\ifnum#1=1
\put(0,0){\circle*{1.5}}
 \else \ifnum#1=2
\put(0,0){$\Box$}
 \else \ifnum#1=3
\put(0,0){$\triangle$}
 \fi \fi \fi
}
%current loop oriented
\newcommand{\cloop}[2]{
\ifx#1L             %anti-clockwise oriented
\put(0,0){\oval(12,12)[t]}
\put(0,0){\oval(12,12)[l]\put(0,-6){\vector(1,0){1}}}
\put(-5,-5){\makebox(10,10){#2}}   %label
   \else \ifx#1R       %clockwise oriented
\put(0,0){\oval(12,12)[t]}
\put(0,0){\oval(12,12)[l]\put(6,0){\vector(0,-1){1}}}
\put(-5,-5){\makebox(10,10){#2}}    %label
\fi \fi
}
%earth
\newcommand{\earth}[1]{
  \ifx#1D
    \put(0,0){\line(0,-1){5}}
    \put(-4,-5){\line(1,0){8}}
    \put(-3,-6){\line(1,0){6}}
    \put(-2,-7){\line(1,0){4}}
   \fi
   \ifx#1U
    \put(0,0){\line(0,1){5}}
    \put(-4,5){\line(1,0){8}}
    \put(-3,6){\line(1,0){6}}
    \put(-2,7){\line(1,0){4}}
   \fi
   \ifx#1L
    \put(0,0){\line(-1,0){5}}
    \put(-5,-4){\line(0,1){8}}
    \put(-6,-3){\line(0,1){6}}
    \put(-7,-2){\line(0,1){4}}
   \fi
   \ifx#1R
    \put(0,0){\line(1,0){5}}
    \put(5,-4){\line(0,1){8}}
    \put(6,-3){\line(0,1){6}}
    \put(7,-2){\line(0,1){4}}
   \fi

}
%chassis
\newcommand{\chassis}[1]{
\ifx#1D
 \put(0,0){\line(0,-1){5}}
 {\thicklines \put(-2.5,-5){\line(1,0){5}} }
\fi
\ifx#1U
 \put(0,0){\line(0,1){5}}
 {\thicklines \put(-2.5,5){\line(1,0){5}}}
\fi
\ifx#1L
 \put(0,0){\line(-1,0){5}}
{\thicklines \put(-5,-2.5){\line(0,1){5}}}
\fi
\ifx#1R
 \put(0,0){\line(1,0){5}}
{\thicklines \put(5,-2.5){\line(0,1){5}} }
\fi
}
%horizontal measuring instrument
\newcommand{\hmeasure}[1]{
\put(0,0){\line(1,0){10}}
\put(15,0){\circle{10}}
\put(20,0){\line(1,0){10}}
\put(10,-4.75){\makebox(10,10){#1}}    %label
}
\newcommand{\hhmeasure}[1]{
\put(0,0){\line(1,0){5}}
\put(10,0){\circle{10}}
\put(15,0){\line(1,0){5}}
\put(5,-4.75){\makebox(10,10){#1}}    %label
}
%vertical measuring instrument
\newcommand{\vmeasure}[1]{
\put(0,0){\line(0,1){10}}
\put(0,15){\circle{10}}
\put(0,20){\line(0,1){10}}
\put(-4.75,10){\makebox(10,10){#1}}  %label
}
\newcommand{\vvmeasure}[1]{
\put(0,0){\line(0,1){5}}
\put(0,10){\circle{10}}
\put(0,15){\line(0,1){5}}
\put(-4.75,5){\makebox(10,10){#1}}  %label
}
\newcommand{\osc}[1]{     %oscilloscope
\put(0,0){\line(0,1){5}}
   \ifx#1I
\put(0,10){\circle{10}}     %indicating
   \fi  \ifx#1R
\put(-5,5){\framebox(10,10){}}   %recording
   \fi
\put(0,15){\line(0,1){5}}
\multiput(-4,9)(4,0){2}{\line(2,1){4}}
\put(0,9){\line(0,1){2}}
}

\endinput

%  if you need also %%%%%%semiconductors%%%%%%%%%%%%%%%%%%%%%%%%%%%%%%%%%%%%%%%%
%%%%%%diode, thyristor, triac, diac, optomember, opamplifier
%%%%%%OTA, BOTA, 
%%%%%%tranzistor bjt, jfet, mosfet, ujt%%%%%%%%%%%%%%%%%
%%%%%%%graetz, Hall's generator, magnistor, thermistor, varistor,
%%%%%%%photoresistor,
\newcommand{\graetz}[1]{
   \ifx#1R
\put(0,0){\ddiode{U}{U}}
\put(20,20){\ddiode{D}{D}}
\put(0,0){\ddiode{D}{D}}
\put(20,-20){\ddiode{U}{U}}
   \fi
   \ifx#1L
\put(0,0){\ddiode{U}{D}}
\put(20,20){\ddiode{D}{U}}
\put(0,0){\ddiode{D}{U}}
\put(20,-20){\ddiode{U}{D}}
   \fi
}
%diagonal diode
\newcommand{\ddiode}[2]{
\ifx#1D                %down right oriented
    \put(0,0){\line(1,-1){20}}
    \put(10.5,-5.5){\line(-1,-1){5}}
      \ifx#2D                   %cathode down
   \put(10.5,-5.5){\line(1,-3){2.5}}
   \put(5.5,-10.5){\line(3,-1){7}}
      \fi \ifx#2U              %cathode up
   \put(10.5,-15.5){\line(-1,3){2.5}}
   \put(15.5,-10.5){\line(-3,1){7}}
           \fi
   \put(15.5,-10.5){\line(-1,-1){5}}
\fi

\ifx#1U                       %up right oriented
   \put(0,0){\line(1,1){20}}
   \put(10.5,5.5){\line(-1,1){5}}
      \ifx#2U                    %cathode up
   \put(10.5,5.5){\line(1,3){2.5}}
   \put(5.5,10.5){\line(3,1){7}}
      \fi \ifx#2D               %cathode down
   \put(10.5,15.5){\line(-1,-3){2.5}}
   \put(15.5,10.5){\line(-3,-1){7}}
           \fi
   \put(15.5,10.5){\line(-1,1){5}}
   \fi
}
%horizontal diode
\newcommand{\hdiode}[2]{
\ifx#1R                  %cathode right
 \put(0,0){\line(1,0){20}}
 \put(7,3){\line(0,-1){6}}
 \put(7,3){\line(2,-1){6}}
 \put(7,-3){\line(2,1){6}}
 \put(13,3){\line(0,-1){6}}
     \ifx#2C                %varicap
 \multiput(9,4)(3,0){2}{\line(1,0){2}}
 \multiput(11,2.5)(1,0){2}{\line(0,1){3}}
    \else \ifx#2Z                %zener diode
  \put(13,-3){\line(-1,0){2}}
    \else \ifx#2E                %esaki diode
  \put(13,-3){\line(-1,0){2}}
  \put(13,3){\line(-1,0){2}}
    \else \ifx#2S                %shottky diode
  \put(13,-3){\line(-1,0){1}}
  \put(12,-3){\line(0,1){1}}
  \put(13,3){\line(1,0){1}}
  \put(14,3){\line(0,-1){1}}
    \else \ifx#2D                %led diode
  \multiput(10,3)(2,2){2}{\vector(-1,1){4}}
    \else \ifx#2L                %laser diode
  \multiput(10,4)(2,2){2}{\vector(-1,1){4}}
  \put(9,3){\line(1,1){4}}
    \else \ifx#2P                       %photo diode
  \multiput(5,8)(2,2){2}{\vector(1,-1){4}}
    \fi \fi  \fi  \fi  \fi \fi \fi \fi

\ifx#1L                      %cathode left
 \put(7,3){\line(0,-1){6}}
 \put(13,3){\line(-2,-1){6}}
 \put(13,-3){\line(-2,1){6}}
 \put(13,3){\line(0,-1){6}}
 \put(0,0){\line(1,0){20}}
    \ifx#2C                %varicap
 \multiput(6,4)(3,0){2}{\line(1,0){2}}
 \multiput(8,2.5)(1,0){2}{\line(0,1){3}}
    \else \ifx#2Z                %zener diode
 \put(7,-3){\line(1,0){2}}
    \else \ifx#2E                %esaki diode
 \put(7,-3){\line(1,0){2}}
 \put(7,3){\line(1,0){2}}
    \else \ifx#2S                %shottky diode
 \put(7,-3){\line(-1,0){1}}
 \put(6,-3){\line(0,1){1}}
 \put(7,3){\line(1,0){1}}
 \put(8,3){\line(0,-1){1}}
    \else \ifx#2D                %led diode
 \multiput(10,3)(-2,2){2}{\vector(1,1){4}}
    \else \ifx#2L                %laser diode
 \multiput(10,4)(-2,2){2}{\vector(1,1){4}}
 \put(11,3){\line(-1,1){4}}
    \else \ifx#2P                  %photo diode
 \multiput(13,10)(2,-2){2}{\vector(-1,-1){4}}
   \fi \fi  \fi  \fi  \fi \fi \fi \fi
}
%vertical diode
\newcommand{\vdiode}[2]{
\ifx#1U                  %cathode up
 \put(0,0){\line(0,1){20}}
 \put(3,7){\line(-1,0){6}}
 \put(3,7){\line(-1,2){3}}
 \put(-3,7){\line(1,2){3}}
 \put(3,13){\line(-1,0){6}}
    \ifx#2C                 %varicap
  \multiput(6,8)(0,3){2}{\line(0,1){2}}
  \multiput(4.5,10)(0,1){2}{\line(1,0){3}}
    \else \ifx#2Z                %zener diode
  \put(3,13){\line(0,-1){2}}
     \else \ifx#2E                %esaki diode
  \put(-3,13){\line(0,-1){2}}
  \put(3,13){\line(0,-1){2}}
    \else \ifx#2S                %shottky diode
  \put(-3,13){\line(0,1){1}}
  \put(-3,14){\line(1,0){1}}
  \put(3,13){\line(0,-1){1}}
  \put(3,12){\line(-1,0){1}}
    \else \ifx#2D                %led diode
  \multiput(3,10)(2,2){2}{\vector(1,-1){4}}
    \else \ifx#2L                %laser diode
  \multiput(4,10)(2,2){2}{\vector(1,-1){4}}
  \put(3,9){\line(1,1){4}}
    \else \ifx#2P                       %photo diode
  \multiput(10,7)(-2,-2){2}{\vector(-1,1){4}}
   \fi \fi  \fi  \fi  \fi \fi \fi \fi

\ifx#1D                  %cathode down
 \put(0,0){\line(0,1){20}}
 \put(3,7){\line(-1,0){6}}
 \put(3,13){\line(-1,-2){3}}
 \put(-3,13){\line(1,-2){3}}
 \put(3,13){\line(-1,0){6}}
    \ifx#2C                 %varicap
  \multiput(6,8)(0,3){2}{\line(0,1){2}}
  \multiput(4.5,10)(0,1){2}{\line(1,0){3}}
    \else \ifx#2Z                %zener diode
  \put(3,7){\line(0,1){2}}
     \else \ifx#2E                %esaki diode
  \put(-3,7){\line(0,1){2}}
  \put(3,7){\line(0,1){2}}
    \else \ifx#2S                %shottky diode
  \put(-3,7){\line(0,1){1}}
  \put(-3,8){\line(1,0){1}}
  \put(3,7){\line(0,-1){1}}
  \put(3,6){\line(-1,0){1}}
    \else \ifx#2D                %led diode
  \multiput(3,10)(2,-2){2}{\vector(1,1){4}}
    \else \ifx#2L                %laser diode
  \multiput(4,10)(2,-2){2}{\vector(1,1){4}}
  \put(3,11){\line(1,-1){4}}
    \else \ifx#2P                       %photo diode
  \multiput(7,14)(2,-2){2}{\vector(-1,-1){4}}
   \fi \fi  \fi  \fi  \fi \fi \fi \fi

}
%vertical triac
\newcommand{\vtriac}[1]{
 \put(0,0){\line(0,1){7}}
 \put(6,7){\line(-1,0){6}}
 \put(6,7){\line(-1,2){3}}
 \put(0,7){\line(1,2){3}}
 \put(6,13){\line(-1,0){6}}
 \put(0,13){\line(0,1){7}}
 \put(0,7){\line(-1,0){6}}
 \put(0,13){\line(-1,-2){3}}
 \put(-6,13){\line(1,-2){3}}
 \put(-6,13){\line(1,0){7}}
     \ifx#1A                         %gate down left
   \put(0,7){\line(-1,-1){5}}
   \put(-5,2){\line(-1,0){5}}
     \fi  \ifx#1B                    %gate up left
  \put(0,13){\line(-1,1){4}}
  \put(-4,17){\line(-1,0){6}}
    \fi   \ifx#1C                      %gate down right
  \put(0,7){\line(1,-1){5}}
  \put(5,2){\line(1,0){5}}
    \fi   \ifx#1D                       %gate up right
  \put(0,13){\line(1,1){4}}
  \put(4,17){\line(1,0){6}}
    \fi
}
%horizontal triac
\newcommand{\htriac}[1]{
 \put(0,0){\line(1,0){7}}
 \put(7,0){\line(0,-1){6}}
 \put(7,0){\line(2,-1){6}}
 \put(7,-6){\line(2,1){6}}
 \put(13,0){\line(0,-1){6}}
 \put(13,0){\line(1,0){7}}
 \put(7,0){\line(0,1){6}}
 \put(13,6){\line(-2,-1){6}}
 \put(13,0){\line(-2,1){6}}
 \put(13,0){\line(0,1){6}}
    \ifx#1A                    %gate down left
  \put(7,0){\line(-1,-1){5}}
  \put(2,-5){\line(0,-1){5}}
    \fi \ifx#1C                %gate down right
  \put(13,0){\line(1,-1){4}}
  \put(17,-4){\line(0,-1){6}}
    \fi \ifx#1D                        %gate up right
  \put(13,0){\line(1,1){4}}
  \put(17,4){\line(0,1){6}}
    \fi \ifx#1B                        %gate up left
  \put(7,0){\line(-1,1){4}}
  \put(3,4){\line(0,1){6}}
    \fi
}
%diagonal thyristor
\newcommand{\dthyristor}[3]{
 \ifx#1D                           %down oriented
   \put(0,0){\line(1,-1){8}}
   \put(10.5,-5.5){\line(-1,-1){5}}
      \ifx#2D                      %cathode down
   \put(10.5,-5.5){\line(1,-3){2.5}}
   \put(5.5,-10.5){\line(3,-1){7}}
      \fi \ifx#2U                  %cathode up
   \put(10.5,-15.5){\line(-1,3){2.5}}
   \put(15.5,-10.5){\line(-3,1){7}}
           \fi
   \put(15.5,-10.5){\line(-1,-1){5}}
   \put(13,-13){\line(1,-1){7}}

     \ifx#3A                           %gate down left
       \put(13,-13){\line(0,-1){4}}
       \put(13,-17){\line(-1,-1){4}}
          \fi   \ifx#3B                     %gate up left
       \put(8,-8){\line(-1,0){6}}
       \put(2,-8){\line(-1,-1){4}}
          \fi
     \ifx#3C                           %gate down right
       \put(13,-13){\line(1,0){4}}
       \put(17,-13){\line(1,1){4}}
         \fi   \ifx#3D                     %gate up right
       \put(8,-8){\line(0,1){6}}
       \put(8,-2){\line(1,1){4}}
\fi  \fi


\ifx#1U                            %up oriented
\put(0,0){\line(1,1){8}}
\put(10.5,5.5){\line(-1,1){5}}
      \ifx#2U                      %cathode up
\put(10.5,5.5){\line(1,3){2.5}}
\put(5.5,10.5){\line(3,1){7}}
      \fi \ifx#2D                  %cathode down
\put(10.5,15.5){\line(-1,-3){2.5}}
\put(15.5,10.5){\line(-3,-1){7}}
           \fi
\put(15.5,10.5){\line(-1,1){5}}
\put(13,13){\line(1,1){7}}

  \ifx#3A                          %gate down left
\put(8,8){\line(-1,0){6}}
\put(2,8){\line(-1,1){4}}
  \fi  \ifx#3B                      %gate up left
\put(13,13){\line(0,1){4}}
\put(13,17){\line(-1,1){4}}
 \fi  \ifx#3C                      %gate  down right
\put(8,8){\line(0,-1){6}}
\put(8,2){\line(1,-1){4}}
 \fi  \ifx#3D                       %gate up right
\put(13,13){\line(1,0){4}}
\put(17,13){\line(1,-1){4}}
 \fi \fi
}

%vertical thyristor
\newcommand{\vthyristor}[2]{
 \ifx#2A                          %gate down left
\put(0,7){\line(-1,-1){5}}
\put(-5,2){\line(-1,0){5}}
 \fi \ifx#2B                      %gate up left
\put(0,13){\line(-1,1){4}}
\put(-4,17){\line(-1,0){6}}
 \fi  \ifx#2C                   %gate down right
\put(0,7){\line(1,-1){5}}
\put(5,2){\line(1,0){5}}
 \fi \ifx#2D                   %gate up right
\put(0,13){\line(1,1){4}}
\put(4,17){\line(1,0){6}}
 \fi
   \ifx#1U                 %cathode up
\put(3,7){\line(-1,0){6}}
\put(3,7){\line(-1,2){3}}
\put(-3,7){\line(1,2){3}}
\put(3,13){\line(-1,0){6}}
   \fi \ifx#1D              %cathode down
\put(-3,7){\line(1,0){6}}
\put(3,13){\line(-1,-2){3}}
\put(-3,13){\line(1,-2){3}}
\put(-3,13){\line(1,0){6}}
  \fi
\put(0,0){\line(0,1){20}}
}
%horizontal thyristor
\newcommand{\hthyristor}[2]{
 \ifx#2A                           %gate down left
\put(7,0){\line(-1,-1){5}}
\put(2,-5){\line(0,-1){5}}
 \fi \ifx#2C                      %gate down right
\put(13,0){\line(1,-1){4}}
\put(17,-4){\line(0,-1){6}}
 \fi  \ifx#2B                      %gate up left
\put(7,0){\line(-1,1){5}}
\put(2,5){\line(0,1){5}}
 \fi \ifx#2D                        %gate up right
\put(13,0){\line(1,1){4}}
\put(17,4){\line(0,1){6}}
 \fi
   \ifx#1R                     %cathode right
\put(7,3){\line(0,-1){6}}
\put(7,3){\line(2,-1){6}}
\put(7,-3){\line(2,1){6}}
\put(13,3){\line(0,-1){6}}
   \fi \ifx#1L                 %cathode left
\put(7,-3){\line(0,1){6}}
\put(13,3){\line(-2,-1){6}}
\put(13,-3){\line(-2,1){6}}
\put(13,-3){\line(0,1){6}}
  \fi
\put(0,0){\line(1,0){20}}
}
%horizontal diac
\newcommand{\hdiac}{
\put(0,0){\line(1,0){7}}
\put(7,0){\line(0,-1){6}}
\put(7,0){\line(2,-1){6}}
\put(7,-6){\line(2,1){6}}
\put(13,0){\line(0,-1){6}}
\put(13,0){\line(1,0){7}}
\put(7,0){\line(0,1){6}}
\put(13,6){\line(-2,-1){6}}
\put(13,0){\line(-2,1){6}}
\put(13,0){\line(0,1){6}}
}
%vertical diac
\newcommand{\vdiac}{
\put(0,0){\line(0,1){7}}
\put(6,7){\line(-1,0){6}}
\put(6,7){\line(-1,2){3}}
\put(0,7){\line(1,2){3}}
\put(6,13){\line(-1,0){6}}
\put(0,13){\line(0,1){7}}
\put(0,7){\line(-1,0){6}}
\put(0,13){\line(-1,-2){3}}
\put(-6,13){\line(1,-2){3}}
\put(-6,13){\line(1,0){6}}
}
%optocoupler
\newcommand{\opto}[1]{
 \ifx#1R
\put(0,0){\line(0,1){20}}
\put(-2,8){\line(1,0){4}}
\put(0,8){\line(-1,2){2}}
\put(0,8){\line(1,2){2}}
\put(-2,12){\line(1,0){4}}
\multiput(3,9)(0,2){2}{\vector(1,0){4}}
\put(11,8){\line(0,1){4}}
\put(15,0){\line(0,1){7.5}}
\put(15,12.5){\line(0,1){7.5}}
\put(11,10.5){\line(2,1){4}}
\put(11,9.5){\vector(2,-1){4}}
\multiput(-5,5)(25,0){2}{\line(0,1){10}}
\multiput(-5,5)(0,10){2}{\line(1,0){25}}
 \fi  \ifx#1L
\put(0,0){\line(0,1){20}}
\put(2,8){\line(-1,0){4}}
\put(0,8){\line(1,2){2}}
\put(0,8){\line(-1,2){2}}
\put(2,12){\line(-1,0){4}}
\put(0,12){\line(0,1){8}}
\multiput(-3,9)(0,2){2}{\vector(-1,0){4}}
\put(-11,8){\line(0,1){4}}
\put(-15,0){\line(0,1){7.5}}
\put(-15,12.5){\line(0,1){7.5}}
\put(-11,10.5){\line(-2,1){4}}
\put(-11,9.5){\vector(-2,-1){4}}
\multiput(5,5)(-25,0){2}{\line(0,1){10}}
\multiput(5,5)(0,10){2}{\line(-1,0){25}}
 \fi
}

%operational amplifier
\newcommand{\opamp}[2]{
  \ifx#1R                      %right oriented
\multiput(0,0)(0,10){2}{\line(1,0){5}}
\put(25,5){\line(1,0){5}}
\put(5,-5){\line(0,1){20}}
\put(5,-5){\line(2,1){20}}
\put(5,15){\line(2,-1){20}}
  \ifx#2D                       %-input down
\put(7,0){\line(1,0){3}}
\put(7,7){\makebox(5,5)[l]{+}}
  \fi  \ifx#2U                  %-input up
\put(7,10){\line(1,0){3}}
\put(7,-2){\makebox(5,5)[l]{+}}
  \fi
   \fi  \ifx#1L                  %left oriented
\multiput(0,0)(0,10){2}{\line(-1,0){5}}
\put(-25,5){\line(-1,0){5}}
\put(-5,-5){\line(0,1){20}}
\put(-5,-5){\line(-2,1){20}}
\put(-5,15){\line(-2,-1){20}}
  \ifx#2D                        %-input down
\put(-7,0){\line(-1,0){3}}
\put(-12,7){\makebox(5,5)[r]{+}}
  \fi  \ifx#2U                   %-input up
\put(-7,10){\line(-1,0){3}}
\put(-12,-2){\makebox(5,5)[r]{+}}
  \fi  \fi  \ifx#1U                  %up oriented
\multiput(0,0)(10,0){2}{\line(0,1){5}}
\put(5,25){\line(0,1){5}}
\put(-5,5){\line(1,0){20}}
\put(-5,5){\line(1,2){10}}
\put(15,5){\line(-1,2){10}}
  \ifx#2L                        %-input left
\put(2,8){\line(-1,0){3}}
\put(6,6){\makebox(5,5)[r]{+}}
  \fi  \ifx#2R                   %-input right
\put(11,8){\line(-1,0){3}}
\put(-3,6){\makebox(5,5)[r]{+}}
  \fi   \fi \ifx#1D                  %down oriented
\multiput(0,0)(10,0){2}{\line(0,-1){5}}
\put(5,-25){\line(0,-1){5}}
\put(-5,-5){\line(1,0){20}}
\put(-5,-5){\line(1,-2){10}}
\put(15,-5){\line(-1,-2){10}}
  \ifx#2L                        %-input left
\put(2,-8){\line(-1,0){3}}
\put(6,-11){\makebox(5,5)[r]{+}}
  \fi  \ifx#2R                   %-input right
\put(11,-8){\line(-1,0){3}}
\put(-3,-11){\makebox(5,5)[r]{+}}
  \fi
   \fi
}

%operational transconductance amplifier
\newcommand{\ota}[3]{
  \ifx#1R                      %right oriented
\multiput(0,0)(0,10){2}{\line(1,0){5}}
\put(15,5){\line(1,0){5}}
\put(5,-5){\line(0,1){20}}
\put(15,0){\line(0,1){10}}
\put(5,-5){\line(2,1){10}}
\put(5,15){\line(2,-1){10}}
     \ifx#3Y              %with linearization diodes
  \put(3.5,6.5){$\bigtriangleup$ }
  \put(3.5,2){$\bigtriangledown$ }
  \multiput(3.5,1.5)(0,7.25){2}{\line(1,0){3}}
  \put(0,5){\line(1,0){5}}
     \fi
  \ifx#2D                       %-input down
\put(7,0){\line(1,0){3}}
\put(7,7){\makebox(5,5)[l]{+}}
  \fi  \ifx#2U                  %-input up
\put(7,10){\line(1,0){3}}
\put(7,-2){\makebox(5,5)[l]{+}}
  \fi
   \fi  \ifx#1L                  %left oriented
\multiput(0,0)(0,10){2}{\line(-1,0){5}}
\put(-15,5){\line(-1,0){5}}
\put(-5,-5){\line(0,1){20}}
\put(-15,0){\line(0,1){10}}
\put(-5,-5){\line(-2,1){10}}
\put(-5,15){\line(-2,-1){10}}
     \ifx#3Y              %with linearization diodes
  \put(-6.5,6.5){$\bigtriangleup$ }
  \put(-6.5,2){$\bigtriangledown$ }
  \multiput(-6.5,1.5)(0,7.25){2}{\line(1,0){3}}
  \put(0,5){\line(-1,0){5}}
     \fi
  \ifx#2D                        %-input down
\put(-7,0){\line(-1,0){3}}
\put(-12,7){\makebox(5,5)[r]{+}}
  \fi  \ifx#2U                   %-input up
\put(-7,10){\line(-1,0){3}}
\put(-12,-2){\makebox(5,5)[r]{+}}
  \fi  \fi  \ifx#1U                  %up oriented
\multiput(0,0)(10,0){2}{\line(0,1){5}}
\put(5,15){\line(0,1){5}}
\put(0,15){\line(1,0){10}}
\put(-5,5){\line(1,0){20}}
\put(-5,5){\line(1,2){5}}
\put(15,5){\line(-1,2){5}}
     \ifx#3Y              %with linearization diodes
  \put(7.2,4.2){$\triangleright$ }
  \put(1.4,4.2){$\triangleleft$ }
  \multiput(1.5,3.5)(7.25,0){2}{\line(0,1){3}}
  \put(5,0){\line(0,1){5}}
     \fi
  \ifx#2L                        %-input left
\put(2,8){\line(-1,0){3}}
\put(6,6){\makebox(5,5)[r]{+}}
  \fi  \ifx#2R                   %-input right
\put(11,8){\line(-1,0){3}}
\put(-3,6){\makebox(5,5)[r]{+}}
  \fi   \fi \ifx#1D                  %down oriented
\multiput(0,0)(10,0){2}{\line(0,-1){5}}
\put(5,-15){\line(0,-1){5}}
\put(0,-15){\line(1,0){10}}
\put(-5,-5){\line(1,0){20}}
\put(-5,-5){\line(1,-2){5}}
\put(15,-5){\line(-1,-2){5}}
     \ifx#3Y              %with linearization diodes
  \put(7.2,-5.8){$\triangleright$ }
  \put(1.4,-5.8){$\triangleleft$ }
  \multiput(1.5,-3.5)(7.25,0){2}{\line(0,-1){3}}
  \put(5,0){\line(0,-1){5}}
     \fi
  \ifx#2L                        %-input left
\put(2,-8){\line(-1,0){3}}
\put(6,-11){\makebox(5,5)[r]{+}}
  \fi  \ifx#2R                   %-input right
\put(11,-8){\line(-1,0){3}}
\put(-3,-11){\makebox(5,5)[r]{+}}
  \fi
   \fi
}

% balanced operational transconductance amplifier
\newcommand{\bota}[3]{
  \ifx#1R                      %right oriented
\multiput(0,0)(0,10){2}{\line(1,0){5}}
\multiput(15,2.5)(0,5){2}{\line(1,0){5}}
\put(5,-5){\line(0,1){20}}
\put(15,0){\line(0,1){10}}
\put(5,-5){\line(2,1){10}}
\put(5,15){\line(2,-1){10}}
     \ifx#3Y              %with linearization diodes
  \put(3.5,6.5){$\bigtriangleup$ }
  \put(3.5,2){$\bigtriangledown$ }
  \multiput(3.5,1.5)(0,7.25){2}{\line(1,0){3}}
  \put(0,5){\line(1,0){5}}
     \fi
  \ifx#2D                       %-input down
\put(7,0){\line(1,0){3}}
\put(7,7){\makebox(5,5)[l]{+}}
  \fi  \ifx#2U                  %-input up
\put(7,10){\line(1,0){3}}
\put(7,-2){\makebox(5,5)[l]{+}}
  \fi
   \fi  \ifx#1L                  %left oriented
\multiput(0,0)(0,10){2}{\line(-1,0){5}}
\multiput(-15,2.5)(0,5){2}{\line(-1,0){5}}
\put(-5,-5){\line(0,1){20}}
\put(-15,0){\line(0,1){10}}
\put(-5,-5){\line(-2,1){10}}
\put(-5,15){\line(-2,-1){10}}
     \ifx#3Y              %with linearization diodes
  \put(-6.5,6.5){$\bigtriangleup$ }
  \put(-6.5,2){$\bigtriangledown$ }
  \multiput(-6.5,1.5)(0,7.25){2}{\line(1,0){3}}
  \put(0,5){\line(-1,0){5}}
     \fi
  \ifx#2D                        %-input down
\put(-7,0){\line(-1,0){3}}
\put(-12,7){\makebox(5,5)[r]{+}}
  \fi  \ifx#2U                   %-input up
\put(-7,10){\line(-1,0){3}}
\put(-12,-2){\makebox(5,5)[r]{+}}
  \fi  \fi  \ifx#1U                  %up oriented
\multiput(0,0)(10,0){2}{\line(0,1){5}}
\multiput(2.5,15)(5,0){2}{\line(0,1){5}}
\put(0,15){\line(1,0){10}}
\put(-5,5){\line(1,0){20}}
\put(-5,5){\line(1,2){5}}
\put(15,5){\line(-1,2){5}}
     \ifx#3Y              %with linearization diodes
  \put(7.2,4.2){$\triangleright$ }
  \put(1.4,4.2){$\triangleleft$ }
  \multiput(1.5,3.5)(7.25,0){2}{\line(0,1){3}}
  \put(5,0){\line(0,1){5}}
     \fi
  \ifx#2L                        %-input left
\put(2,8){\line(-1,0){3}}
\put(6,6){\makebox(5,5)[r]{+}}
  \fi  \ifx#2R                   %-input right
\put(11,8){\line(-1,0){3}}
\put(-3,6){\makebox(5,5)[r]{+}}
  \fi   \fi \ifx#1D                  %down oriented
\multiput(0,0)(10,0){2}{\line(0,-1){5}}
\multiput(2.5,-15)(5,0){2}{\line(0,-1){5}}
\put(0,-15){\line(1,0){10}}
\put(-5,-5){\line(1,0){20}}
\put(-5,-5){\line(1,-2){5}}
\put(15,-5){\line(-1,-2){5}}
     \ifx#3Y              %with linearization diodes
  \put(7.2,-5.8){$\triangleright$ }
  \put(1.4,-5.8){$\triangleleft$ }
  \multiput(1.5,-3.5)(7.25,0){2}{\line(0,-1){3}}
  \put(5,0){\line(0,-1){5}}
     \fi
  \ifx#2L                        %-input left
\put(2,-8){\line(-1,0){3}}
\put(6,-11){\makebox(5,5)[r]{+}}
  \fi  \ifx#2R                   %-input right
\put(11,-8){\line(-1,0){3}}
\put(-3,-11){\makebox(5,5)[r]{+}}
  \fi
   \fi
}
%%%%%%tranzistors bjt, jfet, mosfet,%%%%%%%%%%%%
%bipolar transistor
\newcommand{\bjt}[4]{
 \ifx#2R                       %right oriented
   \put(0,0){\line(1,0){10}}
   \put(10,3){\line(0,-1){6}}
   \put(10,1){\line(1,1){9}}
   \put(10,-1){\line(1,-1){9}}
          \ifx#4N
            \else
               \put(11,0){\circle{10}}
          \fi
      \ifx#3D                     %emitor down
           \ifx#1N                  %npn type
         \put(10,-1){\vector(1,-1){3}}
           \else \ifx#1P            %pnp type
         \put(13,-4){\vector(-1,1){3}}
           \fi \fi
     \else \ifx#3U              %emitor up
           \ifx#1N              %npn type
         \put(10,1){\vector(1,1){3}}
           \else \ifx#1P        %pnp type
         \put(13,4){\vector(-1,-1){3}}
           \fi \fi
     \fi \fi
\fi
\ifx#2L                         %left oriented
   \put(0,0){\line(-1,0){10}}
   \put(-10,1){\line(-1,1){9}}
   \put(-10,-1){\line(-1,-1){9}}
   \put(-10,3){\line(0,-1){6}}
             \ifx#4N
                \else
                    \put(-11,0){\circle{10}}
             \fi
       \ifx#3D                         %emitor down
           \ifx#1N                      %npn type
         \put(-10,-1){\vector(-1,-1){3}}
           \else \ifx#1P                %pnp type
         \put(-13,-4){\vector(1,1){3}}
           \fi \fi
      \else \ifx#3U                   %emitor up
           \ifx#1N                      %npn type
         \put(-10,1){\vector(-1,1){3}}
           \else \ifx#1P                %pnp type
         \put(-13,4){\vector(1,-1){3}}
           \fi \fi
      \fi \fi
\fi
\ifx#2U                            %up oriented
   \put(0,0){\line(0,1){10}}
   \put(-3,10){\line(1,0){6}}
   \put(1,10){\line(1,1){9}}
   \put(-1,10){\line(-1,1){9}}
             \ifx#4N
                \else
                   \put(0,11){\circle{10}}
             \fi
     \ifx#3R                        %emitor right
          \ifx#1N                       %npn type
       \put(1,10){\vector(1,1){3}}
          \else \ifx#1P                %pnp type
       \put(4,13){\vector(-1,-1){3}}
          \fi \fi
     \else \ifx#3L                 %emitor left
          \ifx#1N                      %npn type
        \put(-1,10){\vector(-1,1){3}}
          \else \ifx#1P                %pnp type
        \put(-4,13){\vector(1,-1){3}}
          \fi \fi
     \fi \fi
\fi
\ifx#2D                         %down oriented
   \put(0,0){\line(0,-1){10}}
   \put(-3,-10){\line(1,0){6}}
   \put(1,-10){\line(1,-1){9}}
   \put(-1,-10){\line(-1,-1){9}}
              \ifx#4N
                 \else
                     \put(0,-11){\circle{10}}
              \fi
       \ifx#3R                       %emitor right
            \ifx#1N                     %npn type
         \put(1,-10){\vector(1,-1){3}}
            \else \ifx#1P               %pnp type
         \put(4,-13){\vector(-1,1){3}}
            \fi \fi
       \else \ifx#3L                 %emitor left
            \ifx#1N                     %npn type
         \put(-1,-10){\vector(-1,-1){3}}
            \else \ifx#1P               %pnp type
         \put(-4,-13){\vector(1,1){3}}
            \fi \fi
       \fi \fi
\fi
}
%jfet transistor
\newcommand{\jfet}[4]{
\ifx#2R                    %right oriented
 \ifx#3D                   %source down
   \ifx#1N                  %n channel
      \put(12,0){\vector(1,0){0}}
    \fi  \ifx#1P             %p channel
      \put(8,0){\vector(-1,0){0}}
         \fi
              \ifx#4N
                \else
                  \put(12,2.5){\circle{10}}
              \fi
   \put(12,5){\line(1,0){8}}
   \put(0,0){\line(1,0){20}}
   \put(12,6){\line(0,-1){7}}
 \fi \ifx#3U                  %source up
         \ifx#1N                  %n channel
           \put(12,0){\vector(1,0){0}}
         \fi \ifx#1P             %p channel
           \put(8,0){\vector(-1,0){0}}
             \fi
                \ifx#4N
                   \else
                     \put(12,-2.5){\circle{10}}
                \fi
      \put(12,-5){\line(1,0){8}}
      \put(0,0){\line(1,0){20}}
      \put(12,-6){\line(0,1){7}}
    \fi
\fi \ifx#2L                  %left oriented
      \ifx#3U                     %source up
            \ifx#1N                  %n channel
          \put(-12,0){\vector(-1,0){0}}
             \fi \ifx#1P             %p channel
          \put(-8,0){\vector(1,0){0}}
                 \fi
                     \ifx#4N
                       \else
                          \put(-12,-2.5){\circle{10}}
                     \fi
        \put(-12,-5){\line(-1,0){8}}
        \put(0,0){\line(-1,0){20}}
        \put(-12,1){\line(0,-1){7}}
      \fi \ifx#3D                     %source down
                \ifx#1N                  %n channel
              \put(-12,0){\vector(-1,0){0}}
                \fi \ifx#1P             %p channel
              \put(-8,0){\vector(1,0){0}}
                \fi
                  \ifx#4N
                     \else
                         \put(-12,2.5){\circle{10}}
                  \fi
            \put(-12,5){\line(-1,0){8}}
            \put(0,0){\line(-1,0){20}}
            \put(-12,6){\line(0,-1){7}}
          \fi \fi

\ifx#2U                    %up oriented
          \ifx#3L                   %source left
                 \ifx#1N                  %n channel
               \put(0,12){\vector(0,1){0}}
                 \fi \ifx#1P             %p channel
               \put(0,8){\vector(0,-1){0}}
                     \fi
                        \ifx#4N
                           \else
                              \put(2.5,12){\circle{10}}
                        \fi
             \put(5,12){\line(0,1){8}}
             \put(0,0){\line(0,1){20}}
             \put(6,12){\line(-1,0){7}}
          \fi \ifx#3R                  %source right
                   \ifx#1N                  %n channel
                 \put(0,12){\vector(0,1){0}}
                   \fi \ifx#1P             %p channel
                 \put(0,8){\vector(0,-1){0}}
                       \fi
                          \ifx#4N
                             \else
                                \put(-2.5,12){\circle{10}}
                          \fi
                \put(-5,12){\line(0,1){8}}
                \put(0,0){\line(0,1){20}}
                \put(1,12){\line(-1,0){7}}
              \fi

\fi \ifx#2D                  %down oriented
           \ifx#3R                     %source right
                  \ifx#1N                  %n channel
                \put(0,-12){\vector(0,-1){0}}
                  \fi \ifx#1P             %p channel
                \put(0,-8){\vector(0,1){0}}
                  \fi
                      \ifx#4N
                         \else
                            \put(-2.5,-12){\circle{10}}
                      \fi
             \put(-5,-12){\line(0,-1){8}}
             \put(0,0){\line(0,-1){20}}
             \put(1,-12){\line(-1,0){7}}
           \fi \ifx#3L                     %source left
                    \ifx#1N                  %n channel
                  \put(0,0){\vector(0,-1){11}}
                    \fi \ifx#1P             %p channel
                  \put(0,-11.5){\vector(0,1){4}}
                         \fi
                            \ifx#4N
                               \else
                                   \put(2.5,-12){\circle{10}}
                            \fi
                \put(5,-12){\line(0,-1){8}}
                \put(0,0){\line(0,-1){20}}
                \put(6,-12){\line(-1,0){7}}
               \fi \fi
}
%mos transistor
\newcommand{\mos}[6]{
 \ifx#2R          %right oriented
   \ifx#3D      %source down
      \ifnum#5=1  % one gate
       \put(0,0){\line(1,0){10}}
       \put(10,5){\line(0,-1){5}}
      \fi
      \ifnum#5=2  % two gates
       \multiput(0,0)(0,3.5){2}{\line(1,0){10}}
       \multiput(10,1.5)(0,3.5){2}{\line(0,-1){1.5}}
      \fi
        \ifx#6N
          \else
             \put(12,2.5){\circle{10}}
        \fi
    \put(11,5){\line(1,0){9}}
    \put(11,0){\line(1,0){9}}
          \ifx#4E                     %enhancement mode
        \multiput(11,0.75)(0,2.5){3}{\put(0,0){\line(0,-1){1.5}}}
          \else \ifx#4D               %depletion mode
        \put(11,6){\line(0,-1){7}}
          \fi \fi
       \ifx#1P                    %p type
     \put(11,2.5){\vector(1,0){4}}
       \else \ifx#1N              %n type
     \put(15,2.5){\vector(-1,0){4}}
       \fi \fi

   \else \ifx#3U   %source up
       \ifnum#5=1  % one gate
        \put(0,0){\line(1,0){10}}
        \put(10,0){\line(0,-1){5}}
       \fi
       \ifnum#5=2  % two gates
       \multiput(0,0)(0,-3.5){2}{\line(1,0){10}}
       \multiput(10,0)(0,-3.5){2}{\line(0,-1){1.5}}
      \fi
        \ifx#6N
          \else
            \put(12,-2.5){\circle{10}}
        \fi
     \put(11,-5){\line(1,0){9}}
     \put(11,0){\line(1,0){9}}
          \ifx#4E                     %enhancement mode
        \multiput(11,0.75)(0,-2.5){3}{\put(0,0){\line(0,-1){1.5}}}
          \else \ifx#4D               %depletion mode
        \put(11,-6){\line(0,1){7}}
          \fi \fi
      \ifx#1P                    %p type
    \put(11,-2.5){\vector(1,0){4}}
      \else \ifx#1N              %n type
    \put(15,-2.5){\vector(-1,0){4}}
       \fi \fi
   \fi \fi
 \fi

  \ifx#2L          %left oriented
       \ifx#3D      %source down
            \ifnum#5=1 % one gate
               \put(0,0){\line(-1,0){10}}
               \put(-10,5){\line(0,-1){5}}
            \fi
            \ifnum#5=2  % two gates
               \multiput(0,0)(0,3.5){2}{\line(-1,0){10}}
               \multiput(-10,5)(0,-3.5){2}{\line(0,-1){1.5}}
            \fi
               \ifx#6N
                 \else
                   \put(-12,2.5){\circle{10}}
               \fi
          \put(-11,5){\line(-1,0){9}}
          \put(-11,0){\line(-1,0){9}}
               \ifx#4E                     %enhancement mode
           \multiput(-11,0.75)(0,2.5){3}{\put(0,0){\line(0,-1){1.5}}}
               \else \ifx#4D               %depletion mode
           \put(-11,6){\line(0,-1){7}}
               \fi \fi
             \ifx#1P                    %p type
            \put(-11,2.5){\vector(-1,0){4}}
             \else \ifx#1N              %n type
            \put(-15,2.5){\vector(1,0){4}}
             \fi \fi

      \else \ifx#3U   %source up
          \ifnum#5=1
            \put(0,0){\line(-1,0){10}}
            \put(-10,0){\line(0,-1){5}}
          \fi
          \ifnum#5=2  % two gates
            \multiput(0,0)(0,-3.5){2}{\line(-1,0){10}}
            \multiput(-10,0)(0,-3.5){2}{\line(0,-1){1.5}}
          \fi
             \ifx#6N
               \else
                  \put(-12,-2.5){\circle{10}}
             \fi
    \put(-11,-5){\line(-1,0){9}}
    \put(-11,0){\line(-1,0){9}}
          \ifx#4E                     %enhancement mode
        \multiput(-11,0.75)(0,-2.5){3}{\put(0,0){\line(0,-1){1.5}}}
          \else \ifx#4D               %depletion mode
        \put(-11,-6){\line(0,1){7}}
          \fi \fi
       \ifx#1P                    %p type
      \put(-11,-2.5){\vector(-1,0){4}}
       \else \ifx#1N              %n type
      \put(-15,-2.5){\vector(1,0){4}}
       \fi \fi
      \fi \fi
 \fi

  \ifx#2U          %up oriented
       \ifx#3L      %source left
          \ifnum#5=1 % one gate
            \put(0,0){\line(0,1){10}}
            \put(5,10){\line(-1,0){5}}
          \fi
          \ifnum#5=2  % two gates
            \multiput(0,0)(3.5,0){2}{\line(0,1){10}}
            \multiput(5,10)(-3.5,0){2}{\line(-1,0){1.5}}
          \fi
            \ifx#6N
              \else
                 \put(2.5,12){\circle{10}}
            \fi
         \put(5,11){\line(0,1){9}}
         \put(0,11){\line(0,1){9}}
               \ifx#4E                     %enhancement mode
            \multiput(0.75,11)(2.5,0){3}{\put(0,0){\line(-1,0){1.5}}}
               \else \ifx#4D               %depletion mode
            \put(6,11){\line(-1,0){7}}
               \fi \fi
           \ifx#1P                    %p type
          \put(2.5,11){\vector(0,1){4}}
           \else \ifx#1N              %n type
          \put(2.5,15){\vector(0,-1){4}}
           \fi \fi

       \else \ifx#3R   %source right
                \ifnum#5=1  % one gate
                  \put(0,0){\line(0,1){10}}
                  \put(0,10){\line(-1,0){5}}
                \fi
                \ifnum#5=2  % two gates
                  \multiput(0,0)(-3.5,0){2}{\line(0,1){10}}
                  \multiput(0,10)(-3.5,0){2}{\line(-1,0){1.5}}
                \fi
                    \ifx#6N
                      \else
                         \put(-2.5,12){\circle{10}}
                    \fi
              \put(-5,11){\line(0,1){9}}
              \put(0,11){\line(0,1){9}}
                   \ifx#4E                     %enhancement mode
            \multiput(0.75,11)(-2.5,0){3}{\put(0,0){\line(-1,0){1.5}}}
                   \else \ifx#4D               %depletion mode
            \put(-6,11){\line(1,0){7}}
                   \fi \fi
          \ifx#1P                    %p type
         \put(-2.5,11){\vector(0,1){4}}
          \else \ifx#1N              %n type
         \put(-2.5,15){\vector(0,-1){4}}
          \fi \fi
       \fi \fi
 \fi

  \ifx#2D          %down oriented
       \ifx#3L      %source left
           \ifnum#5=1  % one gate
            \put(0,0){\line(0,-1){10}}
            \put(5,-10){\line(-1,0){5}}
           \fi
           \ifnum#5=2  % two gates
            \multiput(0,0)(3.5,0){2}{\line(0,-1){10}}
            \multiput(5,-10)(-3.5,0){2}{\line(-1,0){1.5}}
           \fi
               \ifx#6N
                 \else
                    \put(2.5,-12){\circle{10}}
               \fi
         \put(5,-11){\line(0,-1){9}}
         \put(0,-11){\line(0,-1){9}}
             \ifx#4E                     %enhancement mode
            \multiput(0.75,-11)(2.5,0){3}{\put(0,0){\line(-1,0){1.5}}}
             \else \ifx#4D               %depletion mode
            \put(6,-11){\line(-1,0){7}}
             \fi \fi
          \ifx#1P                    %p type
            \put(2.5,-11){\vector(0,-1){4}}
          \else \ifx#1N              %n type
            \put(2.5,-15){\vector(0,1){4}}
          \fi \fi

       \else \ifx#3R   %source right
              \ifnum#5=1 % one gate
                \put(0,0){\line(0,-1){10}}
                \put(0,-10){\line(-1,0){5}}
              \fi
              \ifnum#5=2  % two gates
                \multiput(0,0)(-3.5,0){2}{\line(0,-1){10}}
                \multiput(0,-10)(-3.5,0){2}{\line(-1,0){1.5}}
              \fi
                  \ifx#6N
                    \else
                       \put(-2.5,-12){\circle{10}}
                  \fi
            \put(-5,-11){\line(0,-1){9}}
            \put(0,-11){\line(0,-1){9}}
          \ifx#4E                     %enhancement mode
            \multiput(0.75,-11)(-2.5,0){3}{\put(0,0){\line(-1,0){1.5}}}
          \else \ifx#4D               %depletion mode
            \put(1,-11){\line(-1,0){7}}
          \fi \fi
             \ifx#1P                    %p type
            \put(-2.5,-11){\vector(0,-1){4}}
             \else \ifx#1N              %n type
            \put(-2.5,-15){\vector(0,1){4}}
             \fi \fi
       \fi \fi
 \fi
}
%ujt transistor
\newcommand{\ujt}[4]{
\ifx#2R                    %right oriented
 \ifx#3U                   %B1 up
   \ifx#1P                  %p type emiter
      \put(9,2.5){\vector(4,1){0}}
    \fi  \ifx#1N             %n type emiter
      \put(5.5,1.25){\vector(-4,-1){0}}
         \fi
            \ifx#4N
               \else
                  \put(9,2.5){\circle{10}}
            \fi
   \multiput(9,5)(0,-5){2}{\line(1,0){8}}
   \put(0,0){\line(4,1){9}}
   \put(9,6){\line(0,-1){7}}
 \fi \ifx#3D                  %B1 down
         \ifx#1P                  %p type emiter
           \put(9.4,-2.4){\vector(4,-1){0}}
         \fi \ifx#1N             %n type emitor
           \put(5.5,-1.3){\vector(-4,1){0}}
             \fi
                \ifx#4N
                   \else
                       \put(9,-2.5){\circle{10}}
                \fi
      \multiput(9,-5)(0,5){2}{\line(1,0){8}}
      \put(0,0){\line(4,-1){9}}
      \put(9,-6){\line(0,1){7}}
    \fi
\fi \ifx#2L                  %left oriented
      \ifx#3D                     %B1 down
            \ifx#1P                  %p type emiter
          \put(-9.4,-2.4){\vector(-4,-1){0}}
             \fi \ifx#1N             %n type emiter
          \put(-5.5,-1.3){\vector(4,1){0}}
                 \fi
                     \ifx#4N
                        \else
                           \put(-9,-2.5){\circle{10}}
                     \fi
        \multiput(-9,-5)(0,5){2}{\line(-1,0){8}}
        \put(0,0){\line(-4,-1){9}}
        \put(-9,-6){\line(0,1){7}}
      \fi \ifx#3U                     %B1 up
                \ifx#1P                  %p type emiter
              \put(-9.4,2.4){\vector(-4,1){0}}
                \fi \ifx#1N             %n type emiter
              \put(-5.5,1.3){\vector(4,-1){0}}
                \fi
                   \ifx#4N
                      \else
                         \put(-9,2.5){\circle{10}}
                   \fi
            \multiput(-9,5)(0,-5){2}{\line(-1,0){8}}
            \put(0,0){\line(-4,1){9}}
            \put(-9,6){\line(0,-1){7}}
          \fi \fi

\ifx#2U                    %up oriented
          \ifx#3R                   %B1 right
                 \ifx#1P                  %p type emiter
               \put(2.5,9.4){\vector(1,4){0}}
                 \fi \ifx#1N             %n type emiter
               \put(1.3,5.5){\vector(-1,-4){0}}
                     \fi
                         \ifx#4N
                            \else
                               \put(2.5,9){\circle{10}}
                          \fi
             \multiput(0,9)(5,0){2}{\line(0,1){8}}
             \put(0,0){\line(1,4){2.2}}
             \put(6,9){\line(-1,0){7}}
          \fi \ifx#3L                  %B1 left
                   \ifx#1P                  %p type emiter
                 \put(-2.5,9.4){\vector(-1,4){0}}
                   \fi \ifx#1N             %n type emiter
                 \put(-1.3,5.5){\vector(1,-4){0}}
                       \fi
                          \ifx#4N
                             \else
                               \put(-2.5,9){\circle{10}}
                             \fi
                \multiput(-5.,9)(5,0){2}{\line(0,1){8}}
                \put(0,0){\line(-1,4){2.2}}
                \put(-6,9){\line(1,0){7}}
              \fi

\fi \ifx#2D                  %down oriented
           \ifx#3L                     %B1 left
                  \ifx#1P                  %p type emiter
                \put(-2.5,-9.4){\vector(-1,-4){0}}
                  \fi \ifx#1N             %n type emiter
                \put(-1.3,-5.5){\vector(1,4){0}}
                  \fi
                       \ifx#4N
                          \else
                             \put(-2.5,-9){\circle{10}}
                       \fi
             \multiput(-5,-9)(5,0){2}{\line(0,-1){8}}
             \put(0,0){\line(-1,-4){2.2}}
             \put(-6,-9){\line(1,0){7}}
           \fi \ifx#3R                     %B1 right
                    \ifx#1P                  %p type emiter
                  \put(2.5,-9.4){\vector(1,-4){0}}
                    \fi \ifx#1N             %n type emiter
                  \put(1.3,-5.5){\vector(-1,4){0}}
                         \fi
                            \ifx#4N
                               \else
                                  \put(2.5,-9){\circle{10}}
                            \fi
                \multiput(0,-9)(5,0){2}{\line(0,-1){8}}
                \put(0,0){\line(1,-4){2.2}}
                \put(6,-9){\line(-1,0){7}}
               \fi \fi
}
%%%%% Hall's generator
\newcommand{\hall}{
\put(0,0){\line(1,0){10}}
\put(10,-2){\framebox(10,4){}\put(0,2){\line(1,0){10}}}
\multiput(15,-5)(0,7){2}{\line(0,1){3}}
\put(13.7,-0.8){$\times $}
}
%%%%% horizontal magnistor
\newcommand{\hmag}{
\put(0,0){\line(1,0){10}}
\put(10,-2){\framebox(10,4){}\put(0,2){\line(1,0){10}}}
\put(11,-4){\line(1,1){8}}
\put(8,-4){\line(1,0){3}}
\put(15,-5){$\times $}
}
%vertical magnistor
\newcommand{\vmag}{
\put(0,0){\line(0,1){10}}
\put(-2,10){\framebox(4,10){}\put(-2,10){\line(0,1){10}}}
\put(-6,17){\line(3,-1){12}}
\put(6,10){\line(0,1){3}}
\put(3,17){$\times $}
}
%horizontal thermistor PTC
\newcommand{\hptc}{
\put(0,0){\line(1,0){5}}
\put(5,-2){\framebox(10,4){}\put(0,2){\line(1,0){5}}}
\put(6,-4){\line(1,1){8}}
\put(3,-4){\line(1,0){3}}
\multiput(17,2)(2,0){2}{\vector(0,1){4}}
\put(16,-4){$\vartheta $}
}
%horizontal thermistor NTC
\newcommand{\hntc}{
\put(0,0){\line(1,0){5}}
\put(5,-2){\framebox(10,4){}\put(0,2){\line(1,0){5}}}
\put(6,-4){\line(1,1){8}}
\put(3,-4){\line(1,0){3}}
\put(17,2){\vector(0,1){4}}
\put(19,6){\vector(0,-1){4}}
\put(16,-4){$\vartheta $}
}
%horizontal varistor
\newcommand{\hvar}{
\put(0,0){\line(1,0){5}}
\put(5,-2){\framebox(10,4){}\put(0,2){\line(1,0){5}}}
\put(6,-4){\line(1,1){8}}
\put(3,-4){\line(1,0){3}}
\put(16,-4){$\rm U $}
}
%horizontal photorezistor
\newcommand{\hprez}{
\put(0,0){\line(1,0){5}}
\put(5,-2){\framebox(10,4){}\put(0,2){\line(1,0){5}}}
\multiput(5,8)(2,0){2}{\vector(1,-1){4}}
}
%vertical thermistor PTC
\newcommand{\vptc}{
\put(0,0){\line(0,1){5}}
\put(-2,5){\framebox(4,10){}\put(-2,10){\line(0,1){5}}}
\put(4,6){\line(-1,1){8}}
\put(4,3){\line(0,1){3}}
\multiput(-2,17)(0,2){2}{\vector(-1,0){4}}
\put(4,16){$\vartheta $}
}
%vertical thermistor NTC
\newcommand{\vntc}{
\put(0,0){\line(0,1){5}}
\put(-2,5){\framebox(4,10){}\put(-2,10){\line(0,1){5}}}
\put(4,6){\line(-1,1){8}}
\put(4,3){\line(0,1){3}}
\put(-2,17){\vector(-1,0){4}}
\put(-6,19){\vector(1,0){4}}
\put(4,16){$\vartheta $}
}
%vertical varistor
\newcommand{\vvar}{
\put(0,0){\line(0,1){5}}
\put(-2,5){\framebox(4,10){}\put(-2,10){\line(0,1){5}}}
\put(4,6){\line(-1,1){8}}
\put(4,3){\line(0,1){3}}
\put(4,16){$\rm U $}
}
%vertical photorezistor
\newcommand{\vprez}{
\put(0,0){\line(0,1){5}}
\put(-2,5){\framebox(4,10){}\put(-2,10){\line(0,1){5}}}
\multiput(-7,5)(0,2){2}{\vector(1,1){4}}
}
%horizontal peltier's cell
\newcommand{\hpelt}{
\put(0,0){\line(1,0){15}}
\put(5,-2){\framebox(10,4){}\put(0,2){\line(1,0){5}}}
\multiput(5.5,-2)(2,0){5}{$\prime $}
}
\endinput

 .....  %%%%%%%%%%%%basic logical circuits%%%%%%%%%%
%%%AND (NAND)
\newcommand{\andnand}[3]{
\put(0,0){\framebox(10,15){}}
   \ifnum#1=2
\multiput(0,5)(0,5){2}{\line(-1,0){5}}  % 2 inputs
   \else \ifnum#1=3
\multiput(0,2.5)(0,5){3}{\line(-1,0){5}} % 3 inputs
   \fi  \fi
    \ifnum#2=0
\put(4,10){\&}                % normal
\ifx#3I
\put(10.75,7.5){\circle{1.5}} %inverted output
\put(11.5,7.5){\line(1,0){3}}
      \else  \ifx#3N
\put(10,7.5){\line(1,0){5}}    %noninverted output
      \fi \fi
     \else  \ifnum#2=1
\put(3,10){\&$\triangleright$} % power
\ifx#3I
\put(10.75,7.5){\circle{1.5}} %inverted output
\put(11.5,7.5){\line(1,0){3}}
      \else  \ifx#3N
\put(10,7.5){\line(1,0){5}}    %noninverted output
      \fi \fi
     \else \ifnum#2=2
\put(4,10){\&}               %normal open collector
\put(5,5){$\diamondsuit$}
\put(5.5,4.5){\line(1,0){2}}
\ifx#3I
     \put(10.75,4.5){\circle{1.5}} %inverted output
     \put(11.5,4.5){\line(1,0){3}}
           \else \ifx#3N
      \put(10,4.5){\line(1,0){5}}    %noninverted output
           \fi \fi
     \else \ifnum#2=3
\put(3,10){\&$\triangleright$}   %power open collector
\put(5,5){$\diamondsuit$}
\put(5.5,4.5){\line(1,0){2}}
\ifx#3I
     \put(10.75,4.5){\circle{1.5}} %inverted output
     \put(11.5,4.5){\line(1,0){3}}
           \else \ifx#3N
      \put(10,4.5){\line(1,0){5}}    %noninverted output
           \fi \fi
     \fi \fi \fi \fi
}
%%%%%%OR (NOR)
\newcommand{\ornor}[3]{
\put(0,0){\framebox(10,15){}}
   \ifnum#1=2
\multiput(0,5)(0,5){2}{\line(-1,0){5}}  % 2 inputs
   \else \ifnum#1=3
\multiput(0,2.5)(0,5){3}{\line(-1,0){5}} % 3 inputs
   \fi  \fi
    \ifnum#2=0
\put(4,10){$1$}                % normal
\ifx#3I
\put(10.75,7.5){\circle{1.5}} %inverted output
\put(11.5,7.5){\line(1,0){3}}
      \else  \ifx#3N
\put(10,7.5){\line(1,0){5}}    %noninverted output
      \fi \fi
     \else  \ifnum#2=1
\put(3,10){$1\triangleright$} % power
\ifx#3I
\put(10.75,7.5){\circle{1.5}} %inverted output
\put(11.5,7.5){\line(1,0){3}}
      \else  \ifx#3N
\put(10,7.5){\line(1,0){5}}    %noninverted output
      \fi \fi
     \else \ifnum#2=2
\put(4,10){$1$}               %normal open collector
\put(5,5){$\diamondsuit$}
\put(5.5,4.5){\line(1,0){2}}
\ifx#3I
     \put(10.75,4.5){\circle{1.5}} %inverted output
     \put(11.5,4.5){\line(1,0){3}}
           \else \ifx#3N
      \put(10,4.5){\line(1,0){5}}    %noninverted output
           \fi \fi
     \else \ifnum#2=3
\put(3,10){$1\triangleright$}   %power open collector
\put(5,5){$\diamondsuit$}
\put(5.5,4.5){\line(1,0){2}}
\ifx#3I
     \put(10.75,4.5){\circle{1.5}} %inverted output
     \put(11.5,4.5){\line(1,0){3}}
           \else \ifx#3N
      \put(10,4.5){\line(1,0){5}}    %noninverted output
           \fi \fi
     \fi \fi \fi \fi

}
%%%inverter
\newcommand{\invert}[2]{
\put(0,0){\framebox(10,15){}}
\put(0,7.5){\line(-1,0){5}}  % 1 input
    \ifnum#1=0
\put(4,10){$1$}                % normal
\ifx#2I
\put(10.75,7.5){\circle{1.5}} %inverted output
\put(11.5,7.5){\line(1,0){3}}
      \else  \ifx#2N
\put(10,7.5){\line(1,0){5}}    %noninverted output
      \fi \fi
     \else  \ifnum#1=1
\put(3,10){$1\triangleright$} % power
\ifx#2I
\put(10.75,7.5){\circle{1.5}} %inverted output
\put(11.5,7.5){\line(1,0){3}}
      \else  \ifx#2N
\put(10,7.5){\line(1,0){5}}    %noninverted output
      \fi \fi
     \else \ifnum#1=2
\put(4,10){$1$}               %normal open collector
\put(5,5){$\diamondsuit$}
\put(5.5,4.5){\line(1,0){2}}
\ifx#2I
     \put(10.75,4.5){\circle{1.5}} %inverted output
     \put(11.5,4.5){\line(1,0){3}}
           \else \ifx#2N
      \put(10,4.5){\line(1,0){5}}    %noninverted output
           \fi \fi
     \else \ifnum#1=3
\put(3,10){$1\triangleright$}   %power open collector
\put(5,5){$\diamondsuit$}
\put(5.5,4.5){\line(1,0){2}}
\ifx#2I
     \put(10.75,4.5){\circle{1.5}} %inverted output
     \put(11.5,4.5){\line(1,0){3}}
           \else \ifx#2N
      \put(10,4.5){\line(1,0){5}}    %noninverted output
           \fi \fi
     \fi \fi \fi \fi
}


\endinput

\begin{document}

\begin{figure}
\begin{center}
\begin{picture}(100,80)(0,0)  % picture size 100 x 80 mm

\grid{10}{8}                  % grid with the step 10 mm
                              % mesh is numbered in  mm
                              % to make easy the orientation
                              % After finishing the picture
                              % you can cancel this command
\put(30,30){\dind{U}{R}{1}}

\end{picture}
\end{center}
\caption{\it Circuit diagram.}
\label{fig:agic1}
\end{figure}

here is any text


\end{document}

\end{verbatim}

In the case of \LaTeX 2e  is used only different heading
of the document.

\section{Example}

\begin{verbatim}
\begin{figure}
\begin{center}
\begin{picture}(100,80)(0,0)
\grid{10}{8}
\put(40,40){\mos{N}{R}{D}{D}{2}{Y}}
\put(10,40){\hcap{0}}
\put(40,10){\vrez{0}}
\put(40,40){\vrez{1}}
\put(30,10){\vrez{0}}
\put(30,40){\vrez{0}}
\put(60,10){\vrez{0}}
\put(70,10){\vcap{0}}
\put(60,45){\vvind{R}{0}}
\put(70,50){\vturn{L}{0}}
\put(65,50){\vcore{3}{F}}
\put(60,10){\chassis{D}}
\put(60,65){\vwire{5}}
\put(60,70){\hwire{30}}
\put(60,40){\hwire{10}}
\put(10,10){\hwire{80}}
\put(70,50){\hwire{10}}
\put(70,56){\hwire{10}}
\put(10,70){\hwire{30}}
\put(30,10){\node}
\put(40,10){\node}
\put(60,10){\node}
\put(70,10){\node}
\put(30,40){\node}
\put(40,43.5){\node}
\put(60,40){\node}
\put(30,70){\node}
\put(90.5,10){\pin}
\put(80.5,50){\pin}
\put(80.5,56){\pin}
\put(90.5,70){\pin}
\put(9,10){\pin}
\put(9,70){\pin}
\put(9,40){\pin}
\put(20,45){$C_{1}$}
\put(20,55){$R_{1}$}
\put(20,25){$R_{2}$}
\put(45,62){$R_{3}$}
\put(45,25){$R_{4}$}
\put(53,25){$R_{5}$}
\put(75,25){$C_{2}$}
\put(90,75){$+U_{CC}$}
\put(9,75){$\pm U_{r}$}
\end{picture}
\end{center}
\caption{\it Circuit diagram.}
\label{fig:agic1}
\end{figure}


\end{verbatim}

\begin{figure}
\begin{center}
\begin{picture}(100,80)(0,0)
\grid{10}{8}
\put(40,40){\mos{N}{R}{D}{D}{2}{Y}}
\put(10,40){\hcap{0}}
\put(40,10){\vrez{0}}
\put(40,40){\vrez{1}}
\put(30,10){\vrez{0}}
\put(30,40){\vrez{0}}
\put(60,10){\vrez{0}}
\put(70,10){\vcap{0}}
\put(60,45){\vvind{R}{0}}
\put(70,50){\vturn{L}{0}}
\put(65,50){\vcore{3}{F}}
\put(60,10){\chassis{D}}
\put(60,65){\vwire{5}}
\put(60,70){\hwire{30}}
\put(60,40){\hwire{10}}
\put(10,10){\hwire{80}}
\put(70,50){\hwire{10}}
\put(70,56){\hwire{10}}
\put(10,70){\hwire{30}}
\put(30,10){\node}
\put(40,10){\node}
\put(60,10){\node}
\put(70,10){\node}
\put(30,40){\node}
\put(40,43.5){\node}
\put(60,40){\node}
\put(30,70){\node}
\put(90.5,10){\pin}
\put(80.5,50){\pin}
\put(80.5,56){\pin}
\put(90.5,70){\pin}
\put(9,10){\pin}
\put(9,70){\pin}
\put(9,40){\pin}
\put(20,45){$C_{1}$}
\put(20,55){$R_{1}$}
\put(20,25){$R_{2}$}
\put(45,62){$R_{3}$}
\put(45,25){$R_{4}$}
\put(53,25){$R_{5}$}
\put(75,25){$C_{2}$}
\put(90,75){$+U_{CC}$}
\put(9,75){$\pm U_{r}$}
\end{picture}
\end{center}
\caption{\it Circuit diagram.}
\label{fig:agic1}
\end{figure}






\end{document}
